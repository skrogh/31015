\section{Implementation}\label{sec:implemenation}
Our implementation consists of a realtime Linux desktop machine, handling the
localisation and control logic, and a microcontroller positioned on the arm. The microcontroller handles measurements from the MEMs accelerometer and rate gyroscope as well as the ultrasound
distance sensor. The IR distance sensor is connected directly to the desktop machine through a National Instruments ADC/DAC, as is the 
potentiometer at the base of the arm which is used for reference measurements.

We have made several changes to the original setup by Hermansen. The most important of these are

\begin{itemize}
		\item A complete overhaul of the communications subsystem in order to support sending accelerometer, rate gyroscope and 
			distance measurements at 1kHz and avoid intermittent transmission errors. 
		\item The original complementary filter has been replaced by a static kalman filter for state estimation and a linear 
			quadratic regulator (LQR)\cite{Hendricks2008} for motor command control.
			\item The addition of an ultrasound proximity sensor for distance measurements.
\end{itemize}

The microcontroller on the arm is an ARM\textregistered Cortex\(^{\text{TM}}\)
M4F mounted on a Tiva\(^{\text{TM}}\) Launchpad development board from Texas Instruments.
It is connected to an InvenSense MPU9150 on which the MEMs accelerometer and
rate gyroscope resides. It is also connected to a Parallax Ping))) ultrasonic distance sensor. The
test setup can be seen in figure~\ref{fig:labsetup}
\begin{figure}
	\centering
	\includegraphics[width=\columnwidth]{pictures/arbejds_tegning}
	\caption{View of the test setup showing earth and sensor reference frames, placement of the IMU containing the 
	accelerometer and rate gyroscope as well as the placement of the Tiva Launchpad. The arm length is 1.04m.}
	\label{fig:labsetup}
\end{figure}
\begin{figure}
	\centering
	\includegraphics[width=\columnwidth]{pictures/process}
	\caption{View of the program flow in the main loop.} 
	\label{fig:controlflow}
\end{figure}

\subsection{Communications subsystem}\label{sec:comms}
The Tiva Launcpad recieves measurements from the accelerometer, rate gyroscope and ultrasound sensor. The accelerometer and 
rate gyroscope is sampled at 1kHz, while the ultrasound sensor has a varying sample rate that depends on the distance to the nearest
object.
It is advantageous to sample at 1kHz because the rotor movement creates
vibration mainly at \(\sim\)70Hz, but also at the second and third harmonics thereof.
Hermansen should have been able to see this as he sampled at 200Hz, resulting in folding of the second and third
harmonic.

The measurement data is transformed into a packet where the leading bit of every byte contains a synchronization bit. When
the bit is set it signifies the start of a packet. These packets are then sent to the desktop machine at 1kHz. When the ultrasound 
sensor is not ready with a new measurement, zeroes are transmitted as its data
instead.
The packets are received on the desktop machine, which is running a control loop at 200Hz. Here the packets are unpacked to their
original form and corrupted packets are attempted detected and thrown away.
\subsection{Localisation}\label{sec:localisation}
Our localisation algorithm is a three step process. A flow diagram can be seen figure~\ref{fig:controlflow}.
First the rate gyroscope measurements are used to determine the orientation of
the arm. This is done using the Madgwick AHRS algorithm\cite{Madgwick2011}. It
is necessary to know the orientation of the arm before
the acceleration can be calculated because the acceleration measured by the tri-axis MEMs accelerometer is relative to the 
accelerometer frame. Additionally, gravity results in an offset in the accelerometer measurements that has constant magnitude, but
orientation varying with the orientation of the accelerometer. The acceleration
is then rotated into earth frame, the gravity vector is subtracted and the
vertical acceleration is isolated since we only care about the height of the arm.
This is then used as the control input to a static multi rate kalman filter\cite{Welch2006}.

Because the filter is static we
only predict the new state 
\begin{equation*}
	\hat{\mathbf{x}}_{k\vert k-1} = \mathbf{F}\hat{\mathbf{x}}_{k-1\vert k-1} + \mathbf{B}a
\end{equation*}
Compute the measurement error 
\begin{equation*}
	y = z_k - \mathbf{H}\hat{\mathbf{x}}_{k\vert k-1} 
\end{equation*}
And update the state
\begin{equation*}
	\hat{\mathbf{x}}_{k\vert k} = \hat{\mathbf{x}}_{k\vert k-1} + \mathbf{K} y
\end{equation*}

The kalman filter executes a predict step for every measurement coming from the accelerometer at 1kHz and an update step as soon 
as a measurement is received from the distance sensor, be it ultrasound or infrared. When using the IR distance 
sensor a measurement is always ready, however it is only updated at
\(\sim\)20Hz. With the ultrasound sensor, a measurement is only valid in the same sample it is recorded. The ultrasound sensor is also afflicted by intermittent noise that results in false readings,
typically far away from previous values. We employ a gating filter that culls out measurements too far from a moving average.
\begin{figure*}
	\centering
	\subfloat[][]{\newlength\figureheight
		\newlength\figurewidth
		\setlength\figureheight{4cm}
		\setlength\figurewidth{6cm}
		% This file was created by matlab2tikz v0.4.7 running on MATLAB 8.0.
% Copyright (c) 2008--2014, Nico Schlömer <nico.schloemer@gmail.com>
% All rights reserved.
% Minimal pgfplots version: 1.3
% 
% The latest updates can be retrieved from
%   http://www.mathworks.com/matlabcentral/fileexchange/22022-matlab2tikz
% where you can also make suggestions and rate matlab2tikz.
% 
\begin{tikzpicture}

\begin{axis}[%
width=\figurewidth,
height=\figureheight,
scale only axis,
xmin=0,
xmax=92.605,
xlabel={time [s]},
ymin=-0.1,
ymax=0.25,
ylabel={distance [m] / angle [rad]},
axis x line*=bottom,
axis y line*=left,
legend style={at={(0.03,0.97)},anchor=north west,draw=black,fill=white,legend cell align=left},
%legend columns = 2
]
\addplot [color=black!50!green,solid,line width=0.2pt]
  table[row sep=crcr]{0	0\\
0.0500026997840173	0\\
0.100005399568035	0\\
0.150008099352052	0\\
0.200010799136069	0\\
0.250013498920086	0\\
0.300016198704104	0\\
0.350018898488121	0\\
0.400021598272138	0\\
0.450024298056156	0\\
0.500026997840173	0\\
0.55002969762419	0\\
0.600032397408207	0\\
0.650035097192225	0\\
0.700037796976242	0\\
0.750040496760259	0\\
0.800043196544276	0\\
0.850045896328294	0\\
0.900048596112311	0\\
0.950051295896328	0\\
1.00005399568035	0\\
1.05005669546436	0\\
1.10005939524838	0\\
1.1500620950324	0\\
1.20006479481641	0\\
1.25006749460043	0\\
1.30007019438445	0\\
1.35007289416847	0\\
1.40007559395248	0\\
1.4500782937365	0\\
1.50008099352052	0\\
1.55008369330454	0\\
1.60008639308855	0\\
1.65008909287257	0\\
1.70009179265659	0\\
1.7500944924406	0\\
1.80009719222462	0\\
1.85009989200864	0\\
1.90010259179266	0\\
1.95010529157667	0\\
2.00010799136069	0\\
2.05011069114471	0\\
2.10011339092873	0\\
2.15011609071274	0\\
2.20011879049676	0\\
2.25012149028078	0\\
2.30012419006479	0\\
2.35012688984881	0\\
2.40012958963283	0\\
2.45013228941685	0\\
2.50013498920086	0\\
2.55013768898488	0\\
2.6001403887689	0\\
2.65014308855292	0\\
2.70014578833693	0\\
2.75014848812095	0\\
2.80015118790497	0\\
2.85015388768899	0\\
2.900156587473	0\\
2.95015928725702	0\\
3.00016198704104	0\\
3.05016468682505	0\\
3.10016738660907	0\\
3.15017008639309	0\\
3.20017278617711	0\\
3.25017548596112	0\\
3.30017818574514	0\\
3.35018088552916	0\\
3.40018358531318	0\\
3.45018628509719	0\\
3.50018898488121	0\\
3.55019168466523	0\\
3.60019438444924	0\\
3.65019708423326	0\\
3.70019978401728	0\\
3.7502024838013	0\\
3.80020518358531	0\\
3.85020788336933	0\\
3.90021058315335	0\\
3.95021328293736	0\\
4.00021598272138	0\\
4.0502186825054	0\\
4.10022138228942	0\\
4.15022408207343	0\\
4.20022678185745	0\\
4.25022948164147	0\\
4.30023218142549	0\\
4.3502348812095	0\\
4.40023758099352	0\\
4.45024028077754	0\\
4.50024298056155	0\\
4.55024568034557	0\\
4.60024838012959	0\\
4.65025107991361	0\\
4.70025377969762	0\\
4.75025647948164	0\\
4.80025917926566	0\\
4.85026187904968	0\\
4.90026457883369	0\\
4.95026727861771	0\\
5.00026997840173	0\\
5.05027267818575	0\\
5.10027537796976	0\\
5.15027807775378	0\\
5.2002807775378	0\\
5.25028347732181	0\\
5.30028617710583	0\\
5.35028887688985	0\\
5.40029157667387	0\\
5.45029427645788	0\\
5.5002969762419	0\\
5.55029967602592	0\\
5.60030237580994	0\\
5.65030507559395	0\\
5.70030777537797	0\\
5.75031047516199	0\\
5.800313174946	0\\
5.85031587473002	0\\
5.90031857451404	0\\
5.95032127429806	0\\
6.00032397408207	0\\
6.05032667386609	0\\
6.10032937365011	0\\
6.15033207343413	0\\
6.20033477321814	0\\
6.25033747300216	0\\
6.30034017278618	0\\
6.3503428725702	0\\
6.40034557235421	0\\
6.45034827213823	0\\
6.50035097192225	0\\
6.55035367170626	0\\
6.60035637149028	0\\
6.6503590712743	0\\
6.70036177105832	0\\
6.75036447084233	0\\
6.80036717062635	0\\
6.85036987041037	0\\
6.90037257019439	0\\
6.9503752699784	0\\
7.00037796976242	0\\
7.05038066954644	0\\
7.10038336933045	0\\
7.15038606911447	0\\
7.20038876889849	0\\
7.25039146868251	0\\
7.30039416846652	0\\
7.35039686825054	0\\
7.40039956803456	0\\
7.45040226781857	0\\
7.50040496760259	0\\
7.55040766738661	0\\
7.60041036717063	0\\
7.65041306695464	0\\
7.70041576673866	0\\
7.75041846652268	0\\
7.8004211663067	0\\
7.85042386609071	0\\
7.90042656587473	0\\
7.95042926565875	0\\
8.00043196544276	0\\
8.05043466522678	0\\
8.1004373650108	0\\
8.15044006479482	0\\
8.20044276457883	0\\
8.25044546436285	0\\
8.30044816414687	0\\
8.35045086393089	0\\
8.4004535637149	0\\
8.45045626349892	0\\
8.50045896328294	0\\
8.55046166306696	0\\
8.60046436285097	0\\
8.65046706263499	0\\
8.70046976241901	0\\
8.75047246220302	0\\
8.80047516198704	0\\
8.85047786177106	0\\
8.90048056155508	0\\
8.95048326133909	0\\
9.00048596112311	0\\
9.05048866090713	0\\
9.10049136069114	0\\
9.15049406047516	0\\
9.20049676025918	0\\
9.2504994600432	0\\
9.30050215982721	0\\
9.35050485961123	0\\
9.40050755939525	0\\
9.45051025917927	0\\
9.50051295896328	0\\
9.5505156587473	0\\
9.60051835853132	0\\
9.65052105831533	0\\
9.70052375809935	0\\
9.75052645788337	0\\
9.80052915766739	0\\
9.8505318574514	0\\
9.90053455723542	0\\
9.95053725701944	0\\
10.0005399568035	0\\
10.0505426565875	0\\
10.1005453563715	0\\
10.1505480561555	0\\
10.2005507559395	0\\
10.2505534557235	0\\
10.3005561555076	0\\
10.3505588552916	0\\
10.4005615550756	0\\
10.4505642548596	0\\
10.5005669546436	0\\
10.5505696544276	0\\
10.6005723542117	0\\
10.6505750539957	0\\
10.7005777537797	0\\
10.7505804535637	0\\
10.8005831533477	0\\
10.8505858531317	0\\
10.9005885529158	0\\
10.9505912526998	0\\
11.0005939524838	0\\
11.0505966522678	0\\
11.1005993520518	0\\
11.1506020518359	0\\
11.2006047516199	0\\
11.2506074514039	0\\
11.3006101511879	0\\
11.3506128509719	0\\
11.4006155507559	0\\
11.45061825054	0\\
11.500620950324	0\\
11.550623650108	0\\
11.600626349892	0\\
11.650629049676	0\\
11.70063174946	0\\
11.7506344492441	0\\
11.8006371490281	0\\
11.8506398488121	0\\
11.9006425485961	0\\
11.9506452483801	0\\
12.0006479481641	0\\
12.0506506479482	0\\
12.1006533477322	0\\
12.1506560475162	0\\
12.2006587473002	0\\
12.2506614470842	0\\
12.3006641468683	0\\
12.3506668466523	0\\
12.4006695464363	0\\
12.4506722462203	0\\
12.5006749460043	0\\
12.5506776457883	0\\
12.6006803455724	0\\
12.6506830453564	0\\
12.7006857451404	0\\
12.7506884449244	0\\
12.8006911447084	0\\
12.8506938444924	0\\
12.9006965442765	0\\
12.9506992440605	0\\
13.0007019438445	0\\
13.0507046436285	0\\
13.1007073434125	0\\
13.1507100431965	0\\
13.2007127429806	0\\
13.2507154427646	0\\
13.3007181425486	0\\
13.3507208423326	0\\
13.4007235421166	0\\
13.4507262419006	0\\
13.5007289416847	0\\
13.5507316414687	0\\
13.6007343412527	0\\
13.6507370410367	0\\
13.7007397408207	0\\
13.7507424406048	0\\
13.8007451403888	0\\
13.8507478401728	0\\
13.9007505399568	0\\
13.9507532397408	0\\
14.0007559395248	0\\
14.0507586393089	0\\
14.1007613390929	0\\
14.1507640388769	0\\
14.2007667386609	0\\
14.2507694384449	0\\
14.3007721382289	0\\
14.350774838013	0\\
14.400777537797	0\\
14.450780237581	0\\
14.500782937365	0\\
14.550785637149	0\\
14.600788336933	0\\
14.6507910367171	0\\
14.7007937365011	0\\
14.7507964362851	0\\
14.8007991360691	0\\
14.8508018358531	0\\
14.9008045356372	0\\
14.9508072354212	0\\
15.0008099352052	0\\
15.0508126349892	0\\
15.1008153347732	0\\
15.1508180345572	0\\
15.2008207343413	0\\
15.2508234341253	0\\
15.3008261339093	0\\
15.3508288336933	0\\
15.4008315334773	0\\
15.4508342332613	0\\
15.5008369330454	0\\
15.5508396328294	0\\
15.6008423326134	0\\
15.6508450323974	0\\
15.7008477321814	0\\
15.7508504319654	0\\
15.8008531317495	0\\
15.8508558315335	0\\
15.9008585313175	0\\
15.9508612311015	0\\
16.0008639308855	0\\
16.0508666306695	0\\
16.1008693304536	0\\
16.1508720302376	0\\
16.2008747300216	0\\
16.2508774298056	0\\
16.3008801295896	0\\
16.3508828293737	0\\
16.4008855291577	0\\
16.4508882289417	0\\
16.5008909287257	0\\
16.5508936285097	0\\
16.6008963282937	0\\
16.6508990280778	0\\
16.7009017278618	0\\
16.7509044276458	0\\
16.8009071274298	0\\
16.8509098272138	0\\
16.9009125269978	0\\
16.9509152267819	0\\
17.0009179265659	0\\
17.0509206263499	0\\
17.1009233261339	0\\
17.1509260259179	0\\
17.2009287257019	0\\
17.250931425486	0\\
17.30093412527	0\\
17.350936825054	0\\
17.400939524838	0\\
17.450942224622	0\\
17.500944924406	0\\
17.5509476241901	0\\
17.6009503239741	0\\
17.6509530237581	0\\
17.7009557235421	0\\
17.7509584233261	0\\
17.8009611231102	0\\
17.8509638228942	0\\
17.9009665226782	0\\
17.9509692224622	0\\
18.0009719222462	0\\
18.0509746220302	0\\
18.1009773218143	0\\
18.1509800215983	0\\
18.2009827213823	0\\
18.2509854211663	0\\
18.3009881209503	0\\
18.3509908207343	0\\
18.4009935205184	0\\
18.4509962203024	0\\
18.5009989200864	0\\
18.5510016198704	0\\
18.6010043196544	0\\
18.6510070194384	0\\
18.7010097192225	0\\
18.7510124190065	0\\
18.8010151187905	0\\
18.8510178185745	0\\
18.9010205183585	0\\
18.9510232181425	0\\
19.0010259179266	0\\
19.0510286177106	0\\
19.1010313174946	0\\
19.1510340172786	0\\
19.2010367170626	0\\
19.2510394168467	0\\
19.3010421166307	0\\
19.3510448164147	0\\
19.4010475161987	0\\
19.4510502159827	0\\
19.5010529157667	0\\
19.5510556155508	0\\
19.6010583153348	0\\
19.6510610151188	0\\
19.7010637149028	0\\
19.7510664146868	0\\
19.8010691144708	0\\
19.8510718142549	0\\
19.9010745140389	0\\
19.9510772138229	0\\
20.0010799136069	0\\
20.0510826133909	0\\
20.1010853131749	0\\
20.151088012959	0\\
20.201090712743	0\\
20.251093412527	0\\
20.301096112311	0\\
20.351098812095	0\\
20.4011015118791	0\\
20.4511042116631	0\\
20.5011069114471	0\\
20.5511096112311	0\\
20.6011123110151	0\\
20.6511150107991	0\\
20.7011177105832	0\\
20.7511204103672	0\\
20.8011231101512	0\\
20.8511258099352	0\\
20.9011285097192	0\\
20.9511312095032	0\\
21.0011339092873	0\\
21.0511366090713	0\\
21.1011393088553	0\\
21.1511420086393	0\\
21.2011447084233	0\\
21.2511474082073	0\\
21.3011501079914	0\\
21.3511528077754	0\\
21.4011555075594	0\\
21.4511582073434	0\\
21.5011609071274	0\\
21.5511636069114	0\\
21.6011663066955	0\\
21.6511690064795	0\\
21.7011717062635	0\\
21.7511744060475	0\\
21.8011771058315	0\\
21.8511798056156	0\\
21.9011825053996	0\\
21.9511852051836	0\\
22.0011879049676	0\\
22.0511906047516	0\\
22.1011933045356	0\\
22.1511960043197	0\\
22.2011987041037	0\\
22.2512014038877	0\\
22.3012041036717	0\\
22.3512068034557	0\\
22.4012095032397	0\\
22.4512122030238	0\\
22.5012149028078	0\\
22.5512176025918	0\\
22.6012203023758	0\\
22.6512230021598	0\\
22.7012257019438	0\\
22.7512284017279	0\\
22.8012311015119	0\\
22.8512338012959	0\\
22.9012365010799	0\\
22.9512392008639	0\\
23.0012419006479	0\\
23.051244600432	0\\
23.101247300216	0\\
23.15125	0\\
23.201252699784	0\\
23.251255399568	0\\
23.3012580993521	0\\
23.3512607991361	0\\
23.4012634989201	0\\
23.4512661987041	0\\
23.5012688984881	0\\
23.5512715982721	0\\
23.6012742980562	0\\
23.6512769978402	0\\
23.7012796976242	0\\
23.7512823974082	0\\
23.8012850971922	0\\
23.8512877969762	0\\
23.9012904967603	0\\
23.9512931965443	0\\
24.0012958963283	0\\
24.0512985961123	0\\
24.1013012958963	0\\
24.1513039956803	0\\
24.2013066954644	0\\
24.2513093952484	0\\
24.3013120950324	0\\
24.3513147948164	0\\
24.4013174946004	0\\
24.4513201943845	0\\
24.5013228941685	0\\
24.5513255939525	0\\
24.6013282937365	0\\
24.6513309935205	0\\
24.7013336933045	0\\
24.7513363930886	0\\
24.8013390928726	0\\
24.8513417926566	0\\
24.9013444924406	0\\
24.9513471922246	0\\
25.0013498920086	0\\
25.0513525917927	0\\
25.1013552915767	0\\
25.1513579913607	0\\
25.2013606911447	0\\
25.2513633909287	0\\
25.3013660907127	0\\
25.3513687904968	0\\
25.4013714902808	0\\
25.4513741900648	0\\
25.5013768898488	0\\
25.5513795896328	0\\
25.6013822894168	0\\
25.6513849892009	0\\
25.7013876889849	0\\
25.7513903887689	0\\
25.8013930885529	0\\
25.8513957883369	0\\
25.901398488121	0\\
25.951401187905	0\\
26.001403887689	0\\
26.051406587473	0\\
26.101409287257	0\\
26.151411987041	0\\
26.2014146868251	0\\
26.2514173866091	0\\
26.3014200863931	0\\
26.3514227861771	0\\
26.4014254859611	0\\
26.4514281857451	0\\
26.5014308855292	0\\
26.5514335853132	0\\
26.6014362850972	0\\
26.6514389848812	0\\
26.7014416846652	0\\
26.7514443844492	0\\
26.8014470842333	0\\
26.8514497840173	0\\
26.9014524838013	0\\
26.9514551835853	0\\
27.0014578833693	0\\
27.0514605831534	0\\
27.1014632829374	0\\
27.1514659827214	0\\
27.2014686825054	0\\
27.2514713822894	0\\
27.3014740820734	0\\
27.3514767818575	0\\
27.4014794816415	0\\
27.4514821814255	0\\
27.5014848812095	0\\
27.5514875809935	0\\
27.6014902807775	0\\
27.6514929805616	0\\
27.7014956803456	0\\
27.7514983801296	0\\
27.8015010799136	0\\
27.8515037796976	0\\
27.9015064794816	0\\
27.9515091792657	0\\
28.0015118790497	0\\
28.0515145788337	0\\
28.1015172786177	0\\
28.1515199784017	0\\
28.2015226781857	0\\
28.2515253779698	0\\
28.3015280777538	0\\
28.3515307775378	0\\
28.4015334773218	0\\
28.4515361771058	0\\
28.5015388768899	0\\
28.5515415766739	0\\
28.6015442764579	0\\
28.6515469762419	0\\
28.7015496760259	0\\
28.7515523758099	0\\
28.801555075594	0\\
28.851557775378	0\\
28.901560475162	0\\
28.951563174946	0\\
29.00156587473	0\\
29.051568574514	0\\
29.1015712742981	0\\
29.1515739740821	0\\
29.2015766738661	0\\
29.2515793736501	0\\
29.3015820734341	0\\
29.3515847732181	0\\
29.4015874730022	0\\
29.4515901727862	0\\
29.5015928725702	0\\
29.5515955723542	0\\
29.6015982721382	0\\
29.6516009719222	0\\
29.7016036717063	0\\
29.7516063714903	0\\
29.8016090712743	0\\
29.8516117710583	0\\
29.9016144708423	0\\
29.9516171706264	0\\
30.0016198704104	0\\
30.0516225701944	0\\
30.1016252699784	0\\
30.1516279697624	0\\
30.2016306695464	0\\
30.2516333693305	0\\
30.3016360691145	0\\
30.3516387688985	0\\
30.4016414686825	0\\
30.4516441684665	0\\
30.5016468682505	0\\
30.5516495680346	0\\
30.6016522678186	0\\
30.6516549676026	0\\
30.7016576673866	0\\
30.7516603671706	0\\
30.8016630669546	0\\
30.8516657667387	0\\
30.9016684665227	0\\
30.9516711663067	0\\
31.0016738660907	0\\
31.0516765658747	0\\
31.1016792656587	0\\
31.1516819654428	0\\
31.2016846652268	0\\
31.2516873650108	0\\
31.3016900647948	0\\
31.3516927645788	0\\
31.4016954643629	0\\
31.4516981641469	0\\
31.5017008639309	0\\
31.5517035637149	0\\
31.6017062634989	0\\
31.6517089632829	0\\
31.701711663067	0\\
31.751714362851	0\\
31.801717062635	0\\
31.851719762419	0\\
31.901722462203	0\\
31.951725161987	0\\
32.0017278617711	0\\
32.0517305615551	0\\
32.1017332613391	0\\
32.1517359611231	0\\
32.2017386609071	0\\
32.2517413606911	0\\
32.3017440604752	0\\
32.3517467602592	0\\
32.4017494600432	0\\
32.4517521598272	0\\
32.5017548596112	0\\
32.5517575593952	0\\
32.6017602591793	0\\
32.6517629589633	0\\
32.7017656587473	0\\
32.7517683585313	0\\
32.8017710583153	0\\
32.8517737580993	0\\
32.9017764578834	0\\
32.9517791576674	0\\
33.0017818574514	0\\
33.0517845572354	0\\
33.1017872570194	0\\
33.1517899568035	0\\
33.2017926565875	0\\
33.2517953563715	0\\
33.3017980561555	0\\
33.3518007559395	0\\
33.4018034557235	0\\
33.4518061555076	0\\
33.5018088552916	0\\
33.5518115550756	0\\
33.6018142548596	0\\
33.6518169546436	0\\
33.7018196544277	0\\
33.7518223542117	0\\
33.8018250539957	0\\
33.8518277537797	0\\
33.9018304535637	0\\
33.9518331533477	0\\
34.0018358531318	0\\
34.0518385529158	0\\
34.1018412526998	0\\
34.1518439524838	0\\
34.2018466522678	0\\
34.2518493520518	0\\
34.3018520518359	0\\
34.3518547516199	0\\
34.4018574514039	0\\
34.4518601511879	0\\
34.5018628509719	0\\
34.5518655507559	0\\
34.60186825054	0\\
34.651870950324	0\\
34.701873650108	0\\
34.751876349892	0\\
34.801879049676	0\\
34.85188174946	0\\
34.9018844492441	0\\
34.9518871490281	0\\
35.0018898488121	0\\
35.0518925485961	0\\
35.1018952483801	0\\
35.1518979481641	0\\
35.2019006479482	0\\
35.2519033477322	0\\
35.3019060475162	0\\
35.3519087473002	0\\
35.4019114470842	0\\
35.4519141468683	0\\
35.5019168466523	0\\
35.5519195464363	0\\
35.6019222462203	0\\
35.6519249460043	0\\
35.7019276457883	0\\
35.7519303455724	0\\
35.8019330453564	0\\
35.8519357451404	0\\
35.9019384449244	0\\
35.9519411447084	0\\
36.0019438444924	0\\
36.0519465442765	0\\
36.1019492440605	0\\
36.1519519438445	0\\
36.2019546436285	0\\
36.2519573434125	0\\
36.3019600431965	0\\
36.3519627429806	0\\
36.4019654427646	0\\
36.4519681425486	0\\
36.5019708423326	0\\
36.5519735421166	0\\
36.6019762419006	0\\
36.6519789416847	0\\
36.7019816414687	0\\
36.7519843412527	0\\
36.8019870410367	0\\
36.8519897408207	0\\
36.9019924406048	0\\
36.9519951403888	0\\
37.0019978401728	0\\
37.0520005399568	0\\
37.1020032397408	0\\
37.1520059395248	0\\
37.2020086393089	0\\
37.2520113390929	0\\
37.3020140388769	0\\
37.3520167386609	0\\
37.4020194384449	0\\
37.4520221382289	0\\
37.502024838013	0\\
37.552027537797	0\\
37.602030237581	0\\
37.652032937365	0\\
37.702035637149	0\\
37.752038336933	0\\
37.8020410367171	0\\
37.8520437365011	0\\
37.9020464362851	0\\
37.9520491360691	0\\
38.0020518358531	0\\
38.0520545356372	0\\
38.1020572354212	0\\
38.1520599352052	0\\
38.2020626349892	0\\
38.2520653347732	0\\
38.3020680345572	0\\
38.3520707343413	0\\
38.4020734341253	0\\
38.4520761339093	0\\
38.5020788336933	0\\
38.5520815334773	0\\
38.6020842332613	0\\
38.6520869330454	0\\
38.7020896328294	0\\
38.7520923326134	0\\
38.8020950323974	0\\
38.8520977321814	0\\
38.9021004319654	0\\
38.9521031317495	0\\
39.0021058315335	0\\
39.0521085313175	0\\
39.1021112311015	0\\
39.1521139308855	0\\
39.2021166306696	0\\
39.2521193304536	0\\
39.3021220302376	0\\
39.3521247300216	0\\
39.4021274298056	0\\
39.4521301295896	0\\
39.5021328293737	0\\
39.5521355291577	0\\
39.6021382289417	0\\
39.6521409287257	0\\
39.7021436285097	0\\
39.7521463282937	0\\
39.8021490280778	0\\
39.8521517278618	0\\
39.9021544276458	0\\
39.9521571274298	0\\
40.0021598272138	0\\
40.0521625269978	0\\
40.1021652267819	0\\
40.1521679265659	0\\
40.2021706263499	0\\
40.2521733261339	0\\
40.3021760259179	0\\
40.3521787257019	0\\
40.402181425486	0\\
40.45218412527	0\\
40.502186825054	0\\
40.552189524838	0\\
40.602192224622	0\\
40.652194924406	0\\
40.7021976241901	0\\
40.7522003239741	0\\
40.8022030237581	0\\
40.8522057235421	0\\
40.9022084233261	0\\
40.9522111231102	0\\
41.0022138228942	0\\
41.0522165226782	0\\
41.1022192224622	0\\
41.1522219222462	0\\
41.2022246220302	0\\
41.2522273218143	0\\
41.3022300215983	0\\
41.3522327213823	0\\
41.4022354211663	0\\
41.4522381209503	0\\
41.5022408207343	0\\
41.5522435205184	0\\
41.6022462203024	0\\
41.6522489200864	0\\
41.7022516198704	0\\
41.7522543196544	0\\
41.8022570194385	0\\
41.8522597192225	0\\
41.9022624190065	0\\
41.9522651187905	0\\
42.0022678185745	0\\
42.0522705183585	0\\
42.1022732181426	0\\
42.1522759179266	0\\
42.2022786177106	0\\
42.2522813174946	0\\
42.3022840172786	0\\
42.3522867170626	0\\
42.4022894168467	0\\
42.4522921166307	0\\
42.5022948164147	0\\
42.5522975161987	0\\
42.6023002159827	0\\
42.6523029157667	0\\
42.7023056155508	0\\
42.7523083153348	0\\
42.8023110151188	0\\
42.8523137149028	0\\
42.9023164146868	0\\
42.9523191144708	0\\
43.0023218142549	0\\
43.0523245140389	0\\
43.1023272138229	0\\
43.1523299136069	0\\
43.2023326133909	0\\
43.2523353131749	0\\
43.302338012959	0\\
43.352340712743	0\\
43.402343412527	0\\
43.452346112311	0\\
43.502348812095	0\\
43.5523515118791	0\\
43.6023542116631	0\\
43.6523569114471	0\\
43.7023596112311	0\\
43.7523623110151	0\\
43.8023650107991	0\\
43.8523677105832	0\\
43.9023704103672	0\\
43.9523731101512	0\\
44.0023758099352	0\\
44.0523785097192	0\\
44.1023812095032	0\\
44.1523839092873	0\\
44.2023866090713	0\\
44.2523893088553	0\\
44.3023920086393	0\\
44.3523947084233	0\\
44.4023974082073	0\\
44.4524001079914	0\\
44.5024028077754	0\\
44.5524055075594	0\\
44.6024082073434	0\\
44.6524109071274	0\\
44.7024136069114	0\\
44.7524163066955	0\\
44.8024190064795	0\\
44.8524217062635	0\\
44.9024244060475	0\\
44.9524271058315	0\\
45.0024298056155	0\\
45.0524325053996	0\\
45.1024352051836	0\\
45.1524379049676	0\\
45.2024406047516	0\\
45.2524433045356	0\\
45.3024460043197	0\\
45.3524487041037	0\\
45.4024514038877	0\\
45.4524541036717	0\\
45.5024568034557	0\\
45.5524595032397	0\\
45.6024622030238	0\\
45.6524649028078	0\\
45.7024676025918	0\\
45.7524703023758	0\\
45.8024730021598	0\\
45.8524757019439	0\\
45.9024784017279	0\\
45.9524811015119	0\\
46.0024838012959	0\\
46.0524865010799	0\\
46.1024892008639	0\\
46.152491900648	0\\
46.202494600432	0\\
46.252497300216	0\\
46.3025	0\\
46.352502699784	0\\
46.402505399568	0\\
46.4525080993521	0\\
46.5025107991361	0\\
46.5525134989201	0\\
46.6025161987041	0\\
46.6525188984881	0\\
46.7025215982721	0\\
46.7525242980562	0\\
46.8025269978402	0\\
46.8525296976242	0\\
46.9025323974082	0\\
46.9525350971922	0\\
47.0025377969762	0\\
47.0525404967603	0\\
47.1025431965443	0\\
47.1525458963283	0\\
47.2025485961123	0\\
47.2525512958963	0\\
47.3025539956803	0\\
47.3525566954644	0\\
47.4025593952484	0\\
47.4525620950324	0\\
47.5025647948164	0\\
47.5525674946004	0\\
47.6025701943845	0\\
47.6525728941685	0\\
47.7025755939525	0\\
47.7525782937365	0\\
47.8025809935205	0\\
47.8525836933045	0\\
47.9025863930886	0\\
47.9525890928726	0\\
48.0025917926566	0\\
48.0525944924406	0\\
48.1025971922246	0\\
48.1525998920086	0\\
48.2026025917927	0\\
48.2526052915767	0\\
48.3026079913607	0\\
48.3526106911447	0\\
48.4026133909287	0\\
48.4526160907127	0\\
48.5026187904968	0\\
48.5526214902808	0\\
48.6026241900648	0\\
48.6526268898488	0\\
48.7026295896328	0\\
48.7526322894168	0\\
48.8026349892009	0\\
48.8526376889849	0\\
48.9026403887689	0\\
48.9526430885529	0\\
49.0026457883369	0\\
49.052648488121	0\\
49.102651187905	0\\
49.152653887689	0\\
49.202656587473	0\\
49.252659287257	0\\
49.302661987041	0\\
49.3526646868251	0\\
49.4026673866091	0\\
49.4526700863931	0\\
49.5026727861771	0\\
49.5526754859611	0\\
49.6026781857451	0\\
49.6526808855292	0\\
49.7026835853132	0\\
49.7526862850972	0\\
49.8026889848812	0\\
49.8526916846652	0\\
49.9026943844492	0\\
49.9526970842333	0\\
50.0026997840173	0\\
50.0527024838013	0\\
50.1027051835853	0\\
50.1527078833693	0\\
50.2027105831534	0\\
50.2527132829374	0\\
50.3027159827214	0\\
50.3527186825054	0\\
50.4027213822894	0\\
50.4527240820734	0\\
50.5027267818575	0\\
50.5527294816415	0\\
50.6027321814255	0\\
50.6527348812095	0\\
50.7027375809935	0\\
50.7527402807775	0\\
50.8027429805616	0\\
50.8527456803456	0\\
50.9027483801296	0\\
50.9527510799136	0\\
51.0027537796976	0\\
51.0527564794816	0\\
51.1027591792657	0\\
51.1527618790497	0\\
51.2027645788337	0\\
51.2527672786177	0\\
51.3027699784017	0\\
51.3527726781858	0\\
51.4027753779698	0\\
51.4527780777538	0\\
51.5027807775378	0\\
51.5527834773218	0\\
51.6027861771058	0\\
51.6527888768899	0\\
51.7027915766739	0\\
51.7527942764579	0\\
51.8027969762419	0\\
51.8527996760259	0\\
51.9028023758099	0\\
51.952805075594	0\\
52.002807775378	0\\
52.052810475162	0\\
52.102813174946	0\\
52.15281587473	0\\
52.202818574514	0\\
52.2528212742981	0\\
52.3028239740821	0\\
52.3528266738661	0\\
52.4028293736501	0\\
52.4528320734341	0\\
52.5028347732181	0\\
52.5528374730022	0\\
52.6028401727862	0\\
52.6528428725702	0\\
52.7028455723542	0\\
52.7528482721382	0\\
52.8028509719222	0\\
52.8528536717063	0\\
52.9028563714903	0\\
52.9528590712743	0\\
53.0028617710583	0\\
53.0528644708423	0\\
53.1028671706264	0\\
53.1528698704104	0\\
53.2028725701944	0\\
53.2528752699784	0\\
53.3028779697624	0\\
53.3528806695464	0\\
53.4028833693305	0\\
53.4528860691145	0\\
53.5028887688985	0\\
53.5528914686825	0\\
53.6028941684665	0\\
53.6528968682505	0\\
53.7028995680346	0\\
53.7529022678186	0\\
53.8029049676026	0\\
53.8529076673866	0\\
53.9029103671706	0\\
53.9529130669547	0\\
54.0029157667387	0\\
54.0529184665227	0\\
54.1029211663067	0\\
54.1529238660907	0\\
54.2029265658747	0\\
54.2529292656588	0\\
54.3029319654428	0\\
54.3529346652268	0\\
54.4029373650108	0\\
54.4529400647948	0\\
54.5029427645788	0\\
54.5529454643629	0\\
54.6029481641469	0\\
54.6529508639309	0\\
54.7029535637149	0\\
54.7529562634989	0\\
54.8029589632829	0\\
54.852961663067	0\\
54.902964362851	0\\
54.952967062635	0\\
55.002969762419	0\\
55.052972462203	0\\
55.102975161987	0\\
55.1529778617711	0\\
55.2029805615551	0\\
55.2529832613391	0\\
55.3029859611231	0\\
55.3529886609071	0\\
55.4029913606911	0\\
55.4529940604752	0\\
55.5029967602592	0\\
55.5529994600432	0\\
55.6030021598272	0\\
55.6530048596112	0\\
55.7030075593953	0\\
55.7530102591793	0\\
55.8030129589633	0\\
55.8530156587473	0\\
55.9030183585313	0\\
55.9530210583153	0\\
56.0030237580994	0\\
56.0530264578834	0\\
56.1030291576674	0\\
56.1530318574514	0\\
56.2030345572354	0\\
56.2530372570194	0\\
56.3030399568035	0\\
56.3530426565875	0\\
56.4030453563715	0\\
56.4530480561555	0\\
56.5030507559395	0\\
56.5530534557235	0\\
56.6030561555076	0\\
56.6530588552916	0\\
56.7030615550756	0\\
56.7530642548596	0\\
56.8030669546436	0\\
56.8530696544276	0\\
56.9030723542117	0\\
56.9530750539957	0\\
57.0030777537797	0\\
57.0530804535637	0\\
57.1030831533477	0\\
57.1530858531318	0\\
57.2030885529158	0\\
57.2530912526998	0\\
57.3030939524838	0\\
57.3530966522678	0\\
57.4030993520518	0\\
57.4531020518358	0\\
57.5031047516199	0\\
57.5531074514039	0\\
57.6031101511879	0\\
57.6531128509719	0\\
57.7031155507559	0\\
57.75311825054	0\\
57.803120950324	0\\
57.853123650108	0\\
57.903126349892	0\\
57.953129049676	0\\
58.00313174946	0\\
58.0531344492441	0\\
58.1031371490281	0\\
58.1531398488121	0\\
58.2031425485961	0\\
58.2531452483801	0\\
58.3031479481642	0\\
58.3531506479482	0\\
58.4031533477322	0\\
58.4531560475162	0\\
58.5031587473002	0\\
58.5531614470842	0\\
58.6031641468683	0\\
58.6531668466523	0\\
58.7031695464363	0\\
58.7531722462203	0\\
58.8031749460043	0\\
58.8531776457883	0\\
58.9031803455724	0\\
58.9531830453564	0\\
59.0031857451404	0\\
59.0531884449244	0\\
59.1031911447084	0\\
59.1531938444924	0\\
59.2031965442765	0\\
59.2531992440605	0\\
59.3032019438445	0\\
59.3532046436285	0\\
59.4032073434125	0\\
59.4532100431965	0\\
59.5032127429806	0\\
59.5532154427646	0\\
59.6032181425486	0\\
59.6532208423326	0\\
59.7032235421166	0\\
59.7532262419006	0\\
59.8032289416847	0\\
59.8532316414687	0\\
59.9032343412527	0\\
59.9532370410367	0\\
60.0032397408207	0\\
60.0532424406048	0\\
60.1032451403888	0\\
60.1532478401728	0\\
60.2032505399568	0\\
60.2532532397408	0\\
60.3032559395248	0\\
60.3532586393089	0\\
60.4032613390929	0\\
60.4532640388769	0\\
60.5032667386609	0\\
60.5532694384449	0\\
60.6032721382289	0\\
60.653274838013	0\\
60.703277537797	0\\
60.753280237581	0\\
60.803282937365	0\\
60.853285637149	0\\
60.9032883369331	0\\
60.9532910367171	0\\
61.0032937365011	0\\
61.0532964362851	0\\
61.1032991360691	0\\
61.1533018358531	0\\
61.2033045356371	0\\
61.2533072354212	0\\
61.3033099352052	0\\
61.3533126349892	0\\
61.4033153347732	0\\
61.4533180345572	0\\
61.5033207343413	0\\
61.5533234341253	0\\
61.6033261339093	0\\
61.6533288336933	0\\
61.7033315334773	0\\
61.7533342332613	0\\
61.8033369330454	0\\
61.8533396328294	0\\
61.9033423326134	0\\
61.9533450323974	0\\
62.0033477321814	0\\
62.0533504319654	0\\
62.1033531317495	0\\
62.1533558315335	0\\
62.2033585313175	0\\
62.2533612311015	0\\
62.3033639308855	0\\
62.3533666306695	0\\
62.4033693304536	0\\
62.4533720302376	0\\
62.5033747300216	0\\
62.5483771598272	0.2\\
62.5983798596112	0.2\\
62.6483825593953	0.2\\
62.6983852591793	0.2\\
62.7483879589633	0.2\\
62.7983906587473	0.2\\
62.8483933585313	0.2\\
62.8983960583153	0.2\\
62.9483987580994	0.2\\
62.9984014578834	0.2\\
63.0484041576674	0.2\\
63.0984068574514	0.2\\
63.1484095572354	0.2\\
63.1984122570194	0.2\\
63.2484149568035	0.2\\
63.2984176565875	0.2\\
63.3484203563715	0.2\\
63.3984230561555	0.2\\
63.4484257559395	0.2\\
63.4984284557235	0.2\\
63.5484311555076	0.2\\
63.5984338552916	0.2\\
63.6484365550756	0.2\\
63.6984392548596	0.2\\
63.7484419546436	0.2\\
63.7984446544276	0.2\\
63.8484473542117	0.2\\
63.8984500539957	0.2\\
63.9484527537797	0.2\\
63.9984554535637	0.2\\
64.0484581533477	0.2\\
64.0984608531318	0.2\\
64.1484635529158	0.2\\
64.1984662526998	0.2\\
64.2484689524838	0.2\\
64.2984716522678	0.2\\
64.3484743520518	0.2\\
64.3984770518359	0.2\\
64.4484797516199	0.2\\
64.4984824514039	0.2\\
64.5484851511879	0.2\\
64.5984878509719	0.2\\
64.648490550756	0.2\\
64.69849325054	0.2\\
64.748495950324	0.2\\
64.798498650108	0.2\\
64.848501349892	0.2\\
64.898504049676	0.2\\
64.94850674946	0.2\\
64.9985094492441	0.2\\
65.0485121490281	0.2\\
65.0985148488121	0.2\\
65.1485175485961	0.2\\
65.1985202483801	0.2\\
65.2485229481642	0.2\\
65.2985256479482	0.2\\
65.3485283477322	0.2\\
65.3985310475162	0.2\\
65.4485337473002	0.2\\
65.4985364470842	0.2\\
65.5485391468683	0.2\\
65.5985418466523	0.2\\
65.6485445464363	0.2\\
65.6985472462203	0.2\\
65.7485499460043	0.2\\
65.7985526457883	0.2\\
65.8485553455724	0.2\\
65.8985580453564	0.2\\
65.9485607451404	0.2\\
65.9985634449244	0.2\\
66.0485661447084	0.2\\
66.0985688444924	0.2\\
66.1485715442765	0.2\\
66.1985742440605	0.2\\
66.2485769438445	0.2\\
66.2985796436285	0.2\\
66.3485823434125	0.2\\
66.3985850431965	0.2\\
66.4485877429806	0.2\\
66.4985904427646	0.2\\
66.5485931425486	0.2\\
66.5985958423326	0.2\\
66.6485985421166	0.2\\
66.6986012419006	0.2\\
66.7486039416847	0.2\\
66.7986066414687	0.2\\
66.8486093412527	0.2\\
66.8986120410367	0.2\\
66.9486147408207	0.2\\
66.9986174406047	0.2\\
67.0486201403888	0.2\\
67.0986228401728	0.2\\
67.1486255399568	0.2\\
67.1986282397408	0.2\\
67.2486309395248	0.2\\
67.2986336393089	0.2\\
67.3486363390929	0.2\\
67.3986390388769	0.2\\
67.4486417386609	0.2\\
67.4986444384449	0.2\\
67.5486471382289	0.2\\
67.598649838013	0.2\\
67.648652537797	0.2\\
67.698655237581	0.2\\
67.748657937365	0.2\\
67.798660637149	0.2\\
67.8486633369331	0.2\\
67.8986660367171	0.2\\
67.9486687365011	0.2\\
67.9986714362851	0.2\\
68.0486741360691	0.2\\
68.0986768358531	0.2\\
68.1486795356372	0.2\\
68.1986822354212	0.2\\
68.2486849352052	0.2\\
68.2986876349892	0.2\\
68.3486903347732	0.2\\
68.3986930345572	0.2\\
68.4486957343413	0.2\\
68.4986984341253	0.2\\
68.5487011339093	0.2\\
68.5987038336933	0.2\\
68.6487065334773	0.2\\
68.6987092332613	0.2\\
68.7487119330454	0.2\\
68.7987146328294	0.2\\
68.8487173326134	0.2\\
68.8987200323974	0.2\\
68.9487227321814	0.2\\
68.9987254319654	0.2\\
69.0487281317495	0.2\\
69.0987308315335	0.2\\
69.1487335313175	0.2\\
69.1987362311015	0.2\\
69.2487389308855	0.2\\
69.2987416306696	0.2\\
69.3487443304536	0.2\\
69.3987470302376	0.2\\
69.4487497300216	0.2\\
69.4987524298056	0.2\\
69.5487551295896	0.2\\
69.5987578293737	0.2\\
69.6487605291577	0.2\\
69.6987632289417	0.2\\
69.7487659287257	0.2\\
69.7987686285097	0.2\\
69.8487713282937	0.2\\
69.8987740280778	0.2\\
69.9487767278618	0.2\\
69.9987794276458	0.2\\
70.0487821274298	0.2\\
70.0987848272138	0.2\\
70.1487875269979	0.2\\
70.1987902267819	0.2\\
70.2487929265659	0.2\\
70.2987956263499	0.2\\
70.3487983261339	0.2\\
70.3988010259179	0.2\\
70.4488037257019	0.2\\
70.498806425486	0.2\\
70.54880912527	0.2\\
70.598811825054	0.2\\
70.648814524838	0.2\\
70.698817224622	0.2\\
70.7488199244061	0.2\\
70.7988226241901	0.2\\
70.8488253239741	0.2\\
70.8988280237581	0.2\\
70.9488307235421	0.2\\
70.9988334233261	0.2\\
71.0488361231101	0.2\\
71.0988388228942	0.2\\
71.1488415226782	0.2\\
71.1988442224622	0.2\\
71.2488469222462	0.2\\
71.2988496220302	0.2\\
71.3488523218143	0.2\\
71.3988550215983	0.2\\
71.4488577213823	0.2\\
71.4988604211663	0.2\\
71.5488631209503	0.2\\
71.5988658207343	0.2\\
71.6488685205184	0.2\\
71.6988712203024	0.2\\
71.7488739200864	0.2\\
71.7988766198704	0.2\\
71.8488793196544	0.2\\
71.8988820194385	0.2\\
71.9488847192225	0.2\\
71.9988874190065	0.2\\
72.0488901187905	0.2\\
72.0988928185745	0.2\\
72.1488955183585	0.2\\
72.1988982181425	0.2\\
72.2489009179266	0.2\\
72.2989036177106	0.2\\
72.3489063174946	0.2\\
72.3989090172786	0.2\\
72.4489117170626	0.2\\
72.4989144168467	0.2\\
72.5489171166307	0.2\\
72.5989198164147	0.2\\
72.6489225161987	0.2\\
72.6989252159827	0.2\\
72.7489279157667	0.2\\
72.7989306155508	0.2\\
72.8489333153348	0.2\\
72.8989360151188	0.2\\
72.9489387149028	0.2\\
72.9989414146868	0.2\\
73.0489441144708	0.2\\
73.0989468142549	0.2\\
73.1489495140389	0.2\\
73.1989522138229	0.2\\
73.2489549136069	0.2\\
73.2989576133909	0.2\\
73.348960313175	0.2\\
73.398963012959	0.2\\
73.448965712743	0.2\\
73.498968412527	0.2\\
73.548971112311	0.2\\
73.598973812095	0.2\\
73.6489765118791	0.2\\
73.6989792116631	0.2\\
73.7489819114471	0.2\\
73.7989846112311	0.2\\
73.8489873110151	0.2\\
73.8989900107991	0.2\\
73.9489927105832	0.2\\
73.9989954103672	0.2\\
74.0489981101512	0.2\\
74.0990008099352	0.2\\
74.1490035097192	0.2\\
74.1990062095032	0.2\\
74.2490089092873	0.2\\
74.2990116090713	0.2\\
74.3490143088553	0.2\\
74.3990170086393	0.2\\
74.4490197084233	0.2\\
74.4990224082073	0.2\\
74.5490251079914	0.2\\
74.5990278077754	0.2\\
74.6490305075594	0.2\\
74.6990332073434	0.2\\
74.7490359071274	0.2\\
74.7990386069115	0.2\\
74.8490413066955	0.2\\
74.8990440064795	0.2\\
74.9490467062635	0.2\\
74.9990494060475	0.2\\
75.0490521058315	0.2\\
75.0990548056156	0.2\\
75.1490575053996	0.2\\
75.1990602051836	0.2\\
75.2490629049676	0.2\\
75.2990656047516	0.2\\
75.3490683045356	0.2\\
75.3990710043197	0.2\\
75.4490737041037	0.2\\
75.4990764038877	0.2\\
75.5490791036717	0.2\\
75.5990818034557	0.2\\
75.6490845032397	0.2\\
75.6990872030238	0.2\\
75.7490899028078	0.2\\
75.7990926025918	0.2\\
75.8490953023758	0.2\\
75.8990980021598	0.2\\
75.9491007019438	0.2\\
75.9991034017279	0.2\\
76.0491061015119	0.2\\
76.0991088012959	0.2\\
76.1491115010799	0.2\\
76.1991142008639	0.2\\
76.2491169006479	0.2\\
76.299119600432	0.2\\
76.349122300216	0.2\\
76.399125	0.2\\
76.449127699784	0.2\\
76.499130399568	0.2\\
76.5491330993521	0.2\\
76.5991357991361	0.2\\
76.6491384989201	0.2\\
76.6991411987041	0.2\\
76.7491438984881	0.2\\
76.7991465982721	0.2\\
76.8491492980562	0.2\\
76.8991519978402	0.2\\
76.9491546976242	0.2\\
76.9991573974082	0.2\\
77.0491600971922	0.2\\
77.0991627969762	0.2\\
77.1491654967603	0.2\\
77.1991681965443	0.2\\
77.2491708963283	0.2\\
77.2991735961123	0.2\\
77.3491762958963	0.2\\
77.3991789956804	0.2\\
77.4491816954644	0.2\\
77.4991843952484	0.2\\
77.5491870950324	0.2\\
77.5991897948164	0.2\\
77.6491924946004	0.2\\
77.6991951943844	0.2\\
77.7491978941685	0.2\\
77.7992005939525	0.2\\
77.8492032937365	0.2\\
77.8992059935205	0.2\\
77.9492086933045	0.2\\
77.9992113930886	0.2\\
78.0492140928726	0.2\\
78.0992167926566	0.2\\
78.1492194924406	0.2\\
78.1992221922246	0.2\\
78.2492248920086	0.2\\
78.2992275917927	0.2\\
78.3492302915767	0.2\\
78.3992329913607	0.2\\
78.4492356911447	0.2\\
78.4992383909287	0.2\\
78.5492410907127	0.2\\
78.5992437904968	0.2\\
78.6492464902808	0.2\\
78.6992491900648	0.2\\
78.7492518898488	0.2\\
78.7992545896328	0.2\\
78.8492572894169	0.2\\
78.8992599892009	0.2\\
78.9492626889849	0.2\\
78.9992653887689	0.2\\
79.0492680885529	0.2\\
79.0992707883369	0.2\\
79.149273488121	0.2\\
79.199276187905	0.2\\
79.249278887689	0.2\\
79.299281587473	0.2\\
79.349284287257	0.2\\
79.3992869870411	0.2\\
79.4492896868251	0.2\\
79.4992923866091	0.2\\
79.5492950863931	0.2\\
79.5992977861771	0.2\\
79.6493004859611	0.2\\
79.6993031857451	0.2\\
79.7493058855292	0.2\\
79.7993085853132	0.2\\
79.8493112850972	0.2\\
79.8993139848812	0.2\\
79.9493166846652	0.2\\
79.9993193844493	0.2\\
80.0493220842333	0.2\\
80.0993247840173	0.2\\
80.1493274838013	0.2\\
80.1993301835853	0.2\\
80.2493328833693	0.2\\
80.2993355831534	0.2\\
80.3493382829374	0.2\\
80.3993409827214	0.2\\
80.4493436825054	0.2\\
80.4993463822894	0.2\\
80.5493490820734	0.2\\
80.5993517818575	0.2\\
80.6493544816415	0.2\\
80.6993571814255	0.2\\
80.7493598812095	0.2\\
80.7993625809935	0.2\\
80.8493652807775	0.2\\
80.8993679805616	0.2\\
80.9493706803456	0.2\\
80.9993733801296	0.2\\
81.0493760799136	0.2\\
81.0993787796976	0.2\\
81.1493814794816	0.2\\
81.1993841792657	0.2\\
81.2493868790497	0.2\\
81.2993895788337	0.2\\
81.3493922786177	0.2\\
81.3993949784017	0.2\\
81.4493976781857	0.2\\
81.4994003779698	0.2\\
81.5494030777538	0.2\\
81.5994057775378	0.2\\
81.6494084773218	0.2\\
81.6994111771058	0.2\\
81.7494138768898	0.2\\
81.7994165766739	0.2\\
81.8494192764579	0.2\\
81.8994219762419	0.2\\
81.9494246760259	0.2\\
81.9994273758099	0.2\\
82.049430075594	0.2\\
82.099432775378	0.2\\
82.149435475162	0.2\\
82.199438174946	0.2\\
82.24944087473	0.2\\
82.299443574514	0.2\\
82.3494462742981	0.2\\
82.3994489740821	0.2\\
82.4494516738661	0.2\\
82.4994543736501	0.2\\
82.5494570734341	0.2\\
82.5994597732181	0.2\\
82.6294613930885	0\\
82.6794640928726	0\\
82.7294667926566	0\\
82.7794694924406	0\\
82.8294721922246	0\\
82.8794748920086	0\\
82.9294775917927	0\\
82.9794802915767	0\\
83.0294829913607	0\\
83.0794856911447	0\\
83.1294883909287	0\\
83.1794910907127	0\\
83.2294937904968	0\\
83.2794964902808	0\\
83.3294991900648	0\\
83.3795018898488	0\\
83.4295045896328	0\\
83.4795072894169	0\\
83.5295099892009	0\\
83.5795126889849	0\\
83.6295153887689	0\\
83.6795180885529	0\\
83.7295207883369	0\\
83.779523488121	0\\
83.829526187905	0\\
83.879528887689	0\\
83.929531587473	0\\
83.979534287257	0\\
84.029536987041	0\\
84.0795396868251	0\\
84.1295423866091	0\\
84.1795450863931	0\\
84.2295477861771	0\\
84.2795504859611	0\\
84.3295531857451	0\\
84.3795558855292	0\\
84.4295585853132	0\\
84.4795612850972	0\\
84.5295639848812	0\\
84.5795666846652	0\\
84.6295693844492	0\\
84.6795720842333	0\\
84.7295747840173	0\\
84.7795774838013	0\\
84.8295801835853	0\\
84.8795828833693	0\\
84.9295855831534	0\\
84.9795882829374	0\\
85.0295909827214	0\\
85.0795936825054	0\\
85.1295963822894	0\\
85.1795990820734	0\\
85.2296017818575	0\\
85.2796044816415	0\\
85.3296071814255	0\\
85.3796098812095	0\\
85.4296125809935	0\\
85.4796152807775	0\\
85.5296179805616	0\\
85.5796206803456	0\\
85.6296233801296	0\\
85.6796260799136	0\\
85.7296287796976	0\\
85.7796314794817	0\\
85.8296341792657	0\\
85.8796368790497	0\\
85.9296395788337	0\\
85.9796422786177	0\\
86.0296449784017	0\\
86.0796476781857	0\\
86.1296503779698	0\\
86.1796530777538	0\\
86.2296557775378	0\\
86.2796584773218	0\\
86.3296611771058	0\\
86.3796638768899	0\\
86.4296665766739	0\\
86.4796692764579	0\\
86.5296719762419	0\\
86.5796746760259	0\\
86.6296773758099	0\\
86.679680075594	0\\
86.729682775378	0\\
86.779685475162	0\\
86.829688174946	0\\
86.87969087473	0\\
86.929693574514	0\\
86.9796962742981	0\\
87.0296989740821	0\\
87.0797016738661	0\\
87.1297043736501	0\\
87.1797070734341	0\\
87.2297097732181	0\\
87.2797124730022	0\\
87.3297151727862	0\\
87.3797178725702	0\\
87.4297205723542	0\\
87.4797232721382	0\\
87.5297259719222	0\\
87.5797286717063	0\\
87.6297313714903	0\\
87.6797340712743	0\\
87.7297367710583	0\\
87.7797394708423	0\\
87.8297421706264	0\\
87.8797448704104	0\\
87.9297475701944	0\\
87.9797502699784	0\\
88.0297529697624	0\\
88.0797556695464	0\\
88.1297583693305	0\\
88.1797610691145	0\\
88.2297637688985	0\\
88.2797664686825	0\\
88.3297691684665	0\\
88.3797718682505	0\\
88.4297745680346	0\\
88.4797772678186	0\\
88.5297799676026	0\\
88.5797826673866	0\\
88.6297853671706	0\\
88.6797880669546	0\\
88.7297907667387	0\\
88.7797934665227	0\\
88.8297961663067	0\\
88.8797988660907	0\\
88.9298015658747	0\\
88.9798042656588	0\\
89.0298069654428	0\\
89.0798096652268	0\\
89.1298123650108	0\\
89.1798150647948	0\\
89.2298177645788	0\\
89.2798204643629	0\\
89.3298231641469	0\\
89.3798258639309	0\\
89.4298285637149	0\\
89.4798312634989	0\\
89.5298339632829	0\\
89.579836663067	0\\
89.629839362851	0\\
89.679842062635	0\\
89.729844762419	0\\
89.779847462203	0\\
89.8298501619871	0\\
89.8798528617711	0\\
89.9298555615551	0\\
89.9798582613391	0\\
90.0298609611231	0\\
90.0798636609071	0\\
90.1298663606911	0\\
90.1798690604752	0\\
90.2298717602592	0\\
90.2798744600432	0\\
90.3298771598272	0\\
90.3798798596112	0\\
90.4298825593953	0\\
90.4798852591793	0\\
90.5298879589633	0\\
90.5798906587473	0\\
90.6298933585313	0\\
90.6798960583153	0\\
90.7298987580994	0\\
90.7799014578834	0\\
90.8299041576674	0\\
90.8799068574514	0\\
90.9299095572354	0\\
90.9799122570195	0\\
91.0299149568035	0\\
91.0799176565875	0\\
91.1299203563715	0\\
91.1799230561555	0\\
91.2299257559395	0\\
91.2799284557236	0\\
91.3299311555076	0\\
91.3799338552916	0\\
91.4299365550756	0\\
91.4799392548596	0\\
91.5299419546436	0\\
91.5799446544276	0\\
91.6299473542117	0\\
91.6799500539957	0\\
91.7299527537797	0\\
91.7799554535637	0\\
91.8299581533477	0\\
91.8799608531317	0\\
91.9299635529158	0\\
91.9799662526998	0\\
92.0299689524838	0\\
92.0799716522678	0\\
92.1299743520518	0\\
92.1799770518358	0\\
92.2299797516199	0\\
92.2799824514039	0\\
92.3299851511879	0\\
92.3799878509719	0\\
92.4299905507559	0\\
92.47999325054	0\\
92.529995950324	0\\
92.579998650108	0\\
92.605	0\\
};
\addlegendentry{set point};

\addplot [color=gray,solid,line width=0.2pt]
  table[row sep=crcr]{0	0.01\\
0.0500026997840173	0.01\\
0.100005399568035	0.01\\
0.150008099352052	0.01\\
0.200010799136069	0.01\\
0.250013498920086	0.01\\
0.300016198704104	0.01\\
0.350018898488121	0.01\\
0.400021598272138	0.01\\
0.450024298056156	0.01\\
0.500026997840173	0.01\\
0.55002969762419	0.01\\
0.600032397408207	0.01\\
0.650035097192225	0.01\\
0.700037796976242	0.01\\
0.750040496760259	0.01\\
0.800043196544276	0.01\\
0.850045896328294	0.01\\
0.900048596112311	0.01\\
0.950051295896328	0.01\\
1.00005399568035	0.01\\
1.05005669546436	0.01\\
1.10005939524838	0.01\\
1.1500620950324	0.01\\
1.20006479481641	0.01\\
1.25006749460043	0.01\\
1.30007019438445	0.01\\
1.35007289416847	0.01\\
1.40007559395248	0.01\\
1.4500782937365	0.01\\
1.50008099352052	0.01\\
1.55008369330454	0.01\\
1.60008639308855	0.01\\
1.65008909287257	0.01\\
1.70009179265659	0.01\\
1.7500944924406	0.01\\
1.80009719222462	0.01\\
1.85009989200864	0.01\\
1.90010259179266	0.01\\
1.95010529157667	0.01\\
2.00010799136069	0.01\\
2.05011069114471	0.01\\
2.10011339092873	0.01\\
2.15011609071274	0.01\\
2.20011879049676	0.01\\
2.25012149028078	0.01\\
2.30012419006479	0.01\\
2.35012688984881	0.01\\
2.40012958963283	0.01\\
2.45013228941685	0.01\\
2.50013498920086	0.01\\
2.55013768898488	0.01\\
2.6001403887689	0.01\\
2.65014308855292	0.01\\
2.70014578833693	0.01\\
2.75014848812095	0.01\\
2.80015118790497	0.01\\
2.85015388768899	0.01\\
2.900156587473	0.01\\
2.95015928725702	0.01\\
3.00016198704104	0.01\\
3.05016468682505	0.01\\
3.10016738660907	0.01\\
3.15017008639309	0.01\\
3.20017278617711	0.01\\
3.25017548596112	0.01\\
3.30017818574514	0.01\\
3.35018088552916	0.01\\
3.40018358531318	0.01\\
3.45018628509719	0.01\\
3.50018898488121	0.01\\
3.55019168466523	0.01\\
3.60019438444924	0.01\\
3.65019708423326	0.01\\
3.70019978401728	0.01\\
3.7502024838013	0.01\\
3.80020518358531	0.01\\
3.85020788336933	0.01\\
3.90021058315335	0.01\\
3.95021328293736	0.01\\
4.00021598272138	0.01\\
4.0502186825054	0.01\\
4.10022138228942	0.01\\
4.15022408207343	0.01\\
4.20022678185745	0.01\\
4.25022948164147	0.01\\
4.30023218142549	0.01\\
4.3502348812095	0.01\\
4.40023758099352	0.01\\
4.45024028077754	0.01\\
4.50024298056155	0.01\\
4.55024568034557	0.01\\
4.60024838012959	0.01\\
4.65025107991361	0.01\\
4.70025377969762	0.01\\
4.75025647948164	0.01\\
4.80025917926566	0.01\\
4.85026187904968	0.01\\
4.90026457883369	0.01\\
4.95026727861771	0.01\\
5.00026997840173	0.01\\
5.05027267818575	0.01\\
5.10027537796976	0.01\\
5.15027807775378	0.01\\
5.2002807775378	0.01\\
5.25028347732181	0.01\\
5.30028617710583	0.01\\
5.35028887688985	0.01\\
5.40029157667387	0.01\\
5.45029427645788	0.01\\
5.5002969762419	0.01\\
5.55029967602592	0.01\\
5.60030237580994	0.01\\
5.65030507559395	0.01\\
5.70030777537797	0.01\\
5.75031047516199	0.01\\
5.800313174946	0.01\\
5.85031587473002	0.01\\
5.90031857451404	0.01\\
5.95032127429806	0.01\\
6.00032397408207	0.01\\
6.05032667386609	0.01\\
6.10032937365011	0.01\\
6.15033207343413	0.01\\
6.20033477321814	0.01\\
6.25033747300216	0.01\\
6.30034017278618	0.01\\
6.3503428725702	0.01\\
6.40034557235421	0.01\\
6.45034827213823	0.01\\
6.50035097192225	0.01\\
6.55035367170626	0.01\\
6.60035637149028	0.01\\
6.6503590712743	0.01\\
6.70036177105832	0.01\\
6.75036447084233	0.01\\
6.80036717062635	0.01\\
6.85036987041037	0.01\\
6.90037257019439	0.01\\
6.9503752699784	0.01\\
7.00037796976242	0.01\\
7.05038066954644	0.01\\
7.10038336933045	0.01\\
7.15038606911447	0.01\\
7.20038876889849	0.01\\
7.25039146868251	0.01\\
7.30039416846652	0.01\\
7.35039686825054	0.01\\
7.40039956803456	0.01\\
7.45040226781857	0.01\\
7.50040496760259	0.01\\
7.55040766738661	0.01\\
7.60041036717063	0.01\\
7.65041306695464	0.01\\
7.70041576673866	0.01\\
7.75041846652268	0.01\\
7.8004211663067	0.01\\
7.85042386609071	0.01\\
7.90042656587473	0.01\\
7.95042926565875	0.01\\
8.00043196544276	0.01\\
8.05043466522678	0.01\\
8.1004373650108	0.01\\
8.15044006479482	0.01\\
8.20044276457883	0.01\\
8.25044546436285	0.01\\
8.30044816414687	0.01\\
8.35045086393089	0.01\\
8.4004535637149	0.01\\
8.45045626349892	0.01\\
8.50045896328294	0.01\\
8.55046166306696	0.01\\
8.60046436285097	0.01\\
8.65046706263499	0.01\\
8.70046976241901	0.01\\
8.75047246220302	0.01\\
8.80047516198704	0.01\\
8.85047786177106	0.01\\
8.90048056155508	0.01\\
8.95048326133909	0.01\\
9.00048596112311	0.01\\
9.05048866090713	0.01\\
9.10049136069114	0.01\\
9.15049406047516	0.01\\
9.20049676025918	0.01\\
9.2504994600432	0.01\\
9.30050215982721	0.01\\
9.35050485961123	0.01\\
9.40050755939525	0.01\\
9.45051025917927	0.01\\
9.50051295896328	0.01\\
9.5505156587473	0.01\\
9.60051835853132	0.01\\
9.65052105831533	0.01\\
9.70052375809935	0.01\\
9.75052645788337	0.01\\
9.80052915766739	0.01\\
9.8505318574514	0.01\\
9.90053455723542	0.01\\
9.95053725701944	0.01\\
10.0005399568035	0.01\\
10.0505426565875	0.01\\
10.1005453563715	0.01\\
10.1505480561555	0.01\\
10.2005507559395	0.01\\
10.2505534557235	0.01\\
10.3005561555076	0.01\\
10.3505588552916	0.01\\
10.4005615550756	0.01\\
10.4505642548596	0.01\\
10.5005669546436	0.01\\
10.5505696544276	0.01\\
10.6005723542117	0.01\\
10.6505750539957	0.01\\
10.7005777537797	0.01\\
10.7505804535637	0.01\\
10.8005831533477	0.01\\
10.8505858531317	0.01\\
10.9005885529158	0.01\\
10.9505912526998	0.01\\
11.0005939524838	0.01\\
11.0505966522678	0.01\\
11.1005993520518	0.01\\
11.1506020518359	0.01\\
11.2006047516199	0.01\\
11.2506074514039	0.01\\
11.3006101511879	0.01\\
11.3506128509719	0.01\\
11.4006155507559	0.01\\
11.45061825054	0.01\\
11.500620950324	0.01\\
11.550623650108	0.01\\
11.600626349892	0.01\\
11.650629049676	0.01\\
11.70063174946	0.01\\
11.7506344492441	0.01\\
11.8006371490281	0.01\\
11.8506398488121	0.01\\
11.9006425485961	0.01\\
11.9506452483801	0.01\\
12.0006479481641	0.01\\
12.0506506479482	0.01\\
12.1006533477322	0.01\\
12.1506560475162	0.01\\
12.2006587473002	0.01\\
12.2506614470842	0.01\\
12.3006641468683	0.01\\
12.3506668466523	0.01\\
12.4006695464363	0.01\\
12.4506722462203	0.01\\
12.5006749460043	0.01\\
12.5506776457883	0.01\\
12.6006803455724	0.01\\
12.6506830453564	0.01\\
12.7006857451404	0.01\\
12.7506884449244	0.01\\
12.8006911447084	0.01\\
12.8506938444924	0.01\\
12.9006965442765	0.01\\
12.9506992440605	0.01\\
13.0007019438445	0.01\\
13.0507046436285	0.01\\
13.1007073434125	0.01\\
13.1507100431965	0.01\\
13.2007127429806	0.01\\
13.2507154427646	0.01\\
13.3007181425486	0.01\\
13.3507208423326	0.01\\
13.4007235421166	0.01\\
13.4507262419006	0.01\\
13.5007289416847	0.01\\
13.5507316414687	0.01\\
13.6007343412527	0.01\\
13.6507370410367	0.01\\
13.7007397408207	0.01\\
13.7507424406048	0.01\\
13.8007451403888	0.01\\
13.8507478401728	0.01\\
13.9007505399568	0.01\\
13.9507532397408	0.01\\
14.0007559395248	0.01\\
14.0507586393089	0.01\\
14.1007613390929	0.01\\
14.1507640388769	0.01\\
14.2007667386609	0.01\\
14.2507694384449	0.01\\
14.3007721382289	0.01\\
14.350774838013	0.01\\
14.400777537797	0.01\\
14.450780237581	0.01\\
14.500782937365	0.01\\
14.550785637149	0.01\\
14.600788336933	0.01\\
14.6507910367171	0.01\\
14.7007937365011	0.01\\
14.7507964362851	0.01\\
14.8007991360691	0.01\\
14.8508018358531	0.01\\
14.9008045356372	0.01\\
14.9508072354212	0.01\\
15.0008099352052	0.01\\
15.0508126349892	0.01\\
15.1008153347732	0.01\\
15.1508180345572	0.01\\
15.2008207343413	0.01\\
15.2508234341253	0.01\\
15.3008261339093	0.01\\
15.3508288336933	0.01\\
15.4008315334773	0.01\\
15.4508342332613	0.01\\
15.5008369330454	0.01\\
15.5508396328294	0.01\\
15.6008423326134	0.01\\
15.6508450323974	0.01\\
15.7008477321814	0.01\\
15.7508504319654	0.01\\
15.8008531317495	0.01\\
15.8508558315335	0.01\\
15.9008585313175	0.01\\
15.9508612311015	0.01\\
16.0008639308855	0.01\\
16.0508666306695	0.01\\
16.1008693304536	0.01\\
16.1508720302376	0.01\\
16.2008747300216	0.01\\
16.2508774298056	0.01\\
16.3008801295896	0.01\\
16.3508828293737	0.01\\
16.4008855291577	0.01\\
16.4508882289417	0.01\\
16.5008909287257	0.01\\
16.5508936285097	0.01\\
16.6008963282937	0.01\\
16.6508990280778	0.01\\
16.7009017278618	0.01\\
16.7509044276458	0.01\\
16.8009071274298	0.01\\
16.8509098272138	0.01\\
16.9009125269978	0.01\\
16.9509152267819	0.01\\
17.0009179265659	0.01\\
17.0509206263499	0.01\\
17.1009233261339	0.01\\
17.1509260259179	0.01\\
17.2009287257019	0.01\\
17.250931425486	0.01\\
17.30093412527	0.01\\
17.350936825054	0.01\\
17.400939524838	0.01\\
17.450942224622	0.01\\
17.500944924406	0.01\\
17.5509476241901	0.01\\
17.6009503239741	0.01\\
17.6509530237581	0.01\\
17.7009557235421	0.01\\
17.7509584233261	0.01\\
17.8009611231102	0.01\\
17.8509638228942	0.01\\
17.9009665226782	0.01\\
17.9509692224622	0.01\\
18.0009719222462	0.01\\
18.0509746220302	0.01\\
18.1009773218143	0.01\\
18.1509800215983	0.01\\
18.2009827213823	0.01\\
18.2509854211663	0.01\\
18.3009881209503	0.01\\
18.3509908207343	0.01\\
18.4009935205184	0.01\\
18.4509962203024	0.01\\
18.5009989200864	0.01\\
18.5510016198704	0.01\\
18.6010043196544	0.01\\
18.6510070194384	0.01\\
18.7010097192225	0.01\\
18.7510124190065	0.01\\
18.8010151187905	0.01\\
18.8510178185745	0.01\\
18.9010205183585	0.01\\
18.9510232181425	0.01\\
19.0010259179266	0.01\\
19.0510286177106	0.01\\
19.1010313174946	0.01\\
19.1510340172786	0.01\\
19.2010367170626	0.01\\
19.2510394168467	0.01\\
19.3010421166307	0.01\\
19.3510448164147	0.01\\
19.4010475161987	0.01\\
19.4510502159827	0.01\\
19.5010529157667	0.01\\
19.5510556155508	0.01\\
19.6010583153348	0.01\\
19.6510610151188	0.01\\
19.7010637149028	0.01\\
19.7510664146868	0.01\\
19.8010691144708	0.01\\
19.8510718142549	0.01\\
19.9010745140389	0.01\\
19.9510772138229	0.01\\
20.0010799136069	0.01\\
20.0510826133909	0.01\\
20.1010853131749	0.01\\
20.151088012959	0.01\\
20.201090712743	0.01\\
20.251093412527	0.01\\
20.301096112311	0.01\\
20.351098812095	0.01\\
20.4011015118791	0.01\\
20.4511042116631	0.01\\
20.5011069114471	0.01\\
20.5511096112311	0.01\\
20.6011123110151	0.01\\
20.6511150107991	0.01\\
20.7011177105832	0.01\\
20.7511204103672	0.01\\
20.8011231101512	0.01\\
20.8511258099352	0.01\\
20.9011285097192	0.01\\
20.9511312095032	0.01\\
21.0011339092873	0.01\\
21.0511366090713	0.01\\
21.1011393088553	0.01\\
21.1511420086393	0.01\\
21.2011447084233	0.01\\
21.2511474082073	0.01\\
21.3011501079914	0.01\\
21.3511528077754	0.01\\
21.4011555075594	0.01\\
21.4511582073434	0.01\\
21.5011609071274	0.01\\
21.5511636069114	0.01\\
21.6011663066955	0.01\\
21.6511690064795	0.01\\
21.7011717062635	0.01\\
21.7511744060475	0.01\\
21.8011771058315	0.01\\
21.8511798056156	0.01\\
21.9011825053996	0.01\\
21.9511852051836	0.01\\
22.0011879049676	0.01\\
22.0511906047516	0.01\\
22.1011933045356	0.01\\
22.1511960043197	0.01\\
22.2011987041037	0.01\\
22.2512014038877	0.01\\
22.3012041036717	0.01\\
22.3512068034557	0.01\\
22.4012095032397	0.01\\
22.4512122030238	0.01\\
22.5012149028078	0.01\\
22.5512176025918	0.01\\
22.6012203023758	0.01\\
22.6512230021598	0.01\\
22.7012257019438	0.01\\
22.7512284017279	0.01\\
22.8012311015119	0.01\\
22.8512338012959	0.01\\
22.9012365010799	0.01\\
22.9512392008639	0.01\\
23.0012419006479	0.01\\
23.051244600432	0.01\\
23.101247300216	0.01\\
23.15125	0.01\\
23.201252699784	0.01\\
23.251255399568	0.01\\
23.3012580993521	0.01\\
23.3512607991361	0.01\\
23.4012634989201	0.01\\
23.4512661987041	0.01\\
23.5012688984881	0.01\\
23.5512715982721	0.01\\
23.6012742980562	0.01\\
23.6512769978402	0.01\\
23.7012796976242	0.01\\
23.7512823974082	0.01\\
23.8012850971922	0.01\\
23.8512877969762	0.01\\
23.9012904967603	0.01\\
23.9512931965443	0.01\\
24.0012958963283	0.01\\
24.0512985961123	0.01\\
24.1013012958963	0.01\\
24.1513039956803	0.01\\
24.2013066954644	0.01\\
24.2513093952484	0.01\\
24.3013120950324	0.01\\
24.3513147948164	0.01\\
24.4013174946004	0.01\\
24.4513201943845	0.01\\
24.5013228941685	0.01\\
24.5513255939525	0.01\\
24.6013282937365	0.01\\
24.6513309935205	0.01\\
24.7013336933045	0.01\\
24.7513363930886	0.01\\
24.8013390928726	0.01\\
24.8513417926566	0.01\\
24.9013444924406	0.01\\
24.9513471922246	0.01\\
25.0013498920086	0.01\\
25.0513525917927	0.01\\
25.1013552915767	0.01\\
25.1513579913607	0.01\\
25.2013606911447	0.01\\
25.2513633909287	0.01\\
25.3013660907127	0.01\\
25.3513687904968	0.01\\
25.4013714902808	0.01\\
25.4513741900648	0.01\\
25.5013768898488	0.01\\
25.5513795896328	0.01\\
25.6013822894168	0.01\\
25.6513849892009	0.01\\
25.7013876889849	0.01\\
25.7513903887689	0.01\\
25.8013930885529	0.01\\
25.8513957883369	0.01\\
25.901398488121	0.01\\
25.951401187905	0.01\\
26.001403887689	0.01\\
26.051406587473	0.01\\
26.101409287257	0.01\\
26.151411987041	0.01\\
26.2014146868251	0.01\\
26.2514173866091	0.01\\
26.3014200863931	0.01\\
26.3514227861771	0.01\\
26.4014254859611	0.01\\
26.4514281857451	0.01\\
26.5014308855292	0.01\\
26.5514335853132	0.01\\
26.6014362850972	0.01\\
26.6514389848812	0.01\\
26.7014416846652	0.01\\
26.7514443844492	0.01\\
26.8014470842333	0.01\\
26.8514497840173	0.01\\
26.9014524838013	0.01\\
26.9514551835853	0.01\\
27.0014578833693	0.01\\
27.0514605831534	0.01\\
27.1014632829374	0.01\\
27.1514659827214	0.01\\
27.2014686825054	0.01\\
27.2514713822894	0.01\\
27.3014740820734	0.01\\
27.3514767818575	0.01\\
27.4014794816415	0.01\\
27.4514821814255	0.01\\
27.5014848812095	0.01\\
27.5514875809935	0.01\\
27.6014902807775	0.01\\
27.6514929805616	0.01\\
27.7014956803456	0.01\\
27.7514983801296	0.01\\
27.8015010799136	0.01\\
27.8515037796976	0.01\\
27.9015064794816	0.01\\
27.9515091792657	0.01\\
28.0015118790497	0.01\\
28.0515145788337	0.01\\
28.1015172786177	0.01\\
28.1515199784017	0.01\\
28.2015226781857	0.01\\
28.2515253779698	0.01\\
28.3015280777538	0.01\\
28.3515307775378	0.01\\
28.4015334773218	0.01\\
28.4515361771058	0.01\\
28.5015388768899	0.01\\
28.5515415766739	0.01\\
28.6015442764579	0.01\\
28.6515469762419	0.01\\
28.7015496760259	0.01\\
28.7515523758099	0.01\\
28.801555075594	0.01\\
28.851557775378	0.01\\
28.901560475162	0.01\\
28.951563174946	0.01\\
29.00156587473	0.01\\
29.051568574514	0.01\\
29.1015712742981	0.01\\
29.1515739740821	0.01\\
29.2015766738661	0.01\\
29.2515793736501	0.01\\
29.3015820734341	0.01\\
29.3515847732181	0.01\\
29.4015874730022	0.01\\
29.4515901727862	0.01\\
29.5015928725702	0.01\\
29.5515955723542	0.01\\
29.6015982721382	0.01\\
29.6516009719222	0.01\\
29.7016036717063	0.01\\
29.7516063714903	0.01\\
29.8016090712743	0.01\\
29.8516117710583	0.01\\
29.9016144708423	0.01\\
29.9516171706264	0.01\\
30.0016198704104	0.01\\
30.0516225701944	0.01\\
30.1016252699784	0.01\\
30.1516279697624	0.01\\
30.2016306695464	0.01\\
30.2516333693305	0.01\\
30.3016360691145	0.01\\
30.3516387688985	0.01\\
30.4016414686825	0.01\\
30.4516441684665	0.01\\
30.5016468682505	0.01\\
30.5516495680346	0.01\\
30.6016522678186	0.01\\
30.6516549676026	0.01\\
30.7016576673866	0.01\\
30.7516603671706	0.01\\
30.8016630669546	0.01\\
30.8516657667387	0.01\\
30.9016684665227	0.01\\
30.9516711663067	0.01\\
31.0016738660907	0.01\\
31.0516765658747	0.01\\
31.1016792656587	0.01\\
31.1516819654428	0.01\\
31.2016846652268	0.01\\
31.2516873650108	0.01\\
31.3016900647948	0.01\\
31.3516927645788	0.01\\
31.4016954643629	0.01\\
31.4516981641469	0.01\\
31.5017008639309	0.01\\
31.5517035637149	0.01\\
31.6017062634989	0.01\\
31.6517089632829	0.01\\
31.701711663067	0.01\\
31.751714362851	0.01\\
31.801717062635	0.01\\
31.851719762419	0.01\\
31.901722462203	0.01\\
31.951725161987	0.01\\
32.0017278617711	0.01\\
32.0517305615551	0.01\\
32.1017332613391	0.01\\
32.1517359611231	0.01\\
32.2017386609071	0.01\\
32.2517413606911	0.01\\
32.3017440604752	0.01\\
32.3517467602592	0.01\\
32.4017494600432	0.01\\
32.4517521598272	0.01\\
32.5017548596112	0.01\\
32.5517575593952	0.01\\
32.6017602591793	0.01\\
32.6517629589633	0.01\\
32.7017656587473	0.01\\
32.7517683585313	0.01\\
32.8017710583153	0.01\\
32.8517737580993	0.01\\
32.9017764578834	0.01\\
32.9517791576674	0.01\\
33.0017818574514	0.01\\
33.0517845572354	0.01\\
33.1017872570194	0.01\\
33.1517899568035	0.01\\
33.2017926565875	0.01\\
33.2517953563715	0.01\\
33.3017980561555	0.01\\
33.3518007559395	0.01\\
33.4018034557235	0.01\\
33.4518061555076	0.01\\
33.5018088552916	0.01\\
33.5518115550756	0.01\\
33.6018142548596	0.01\\
33.6518169546436	0.01\\
33.7018196544277	0.01\\
33.7518223542117	0.01\\
33.8018250539957	0.01\\
33.8518277537797	0.01\\
33.9018304535637	0.01\\
33.9518331533477	0.01\\
34.0018358531318	0.01\\
34.0518385529158	0.01\\
34.1018412526998	0.01\\
34.1518439524838	0.01\\
34.2018466522678	0.01\\
34.2518493520518	0.01\\
34.3018520518359	0.01\\
34.3518547516199	0.01\\
34.4018574514039	0.01\\
34.4518601511879	0.01\\
34.5018628509719	0.01\\
34.5518655507559	0.01\\
34.60186825054	0.01\\
34.651870950324	0.01\\
34.701873650108	0.01\\
34.751876349892	0.01\\
34.801879049676	0.01\\
34.85188174946	0.01\\
34.9018844492441	0.01\\
34.9518871490281	0.01\\
35.0018898488121	0.01\\
35.0518925485961	0.01\\
35.1018952483801	0.01\\
35.1518979481641	0.01\\
35.2019006479482	0.01\\
35.2519033477322	0.01\\
35.3019060475162	0.01\\
35.3519087473002	0.01\\
35.4019114470842	0.01\\
35.4519141468683	0.01\\
35.5019168466523	0.01\\
35.5519195464363	0.01\\
35.6019222462203	0.01\\
35.6519249460043	0.01\\
35.7019276457883	0.01\\
35.7519303455724	0.01\\
35.8019330453564	0.01\\
35.8519357451404	0.01\\
35.9019384449244	0.01\\
35.9519411447084	0.01\\
36.0019438444924	0.01\\
36.0519465442765	0.01\\
36.1019492440605	0.01\\
36.1519519438445	0.01\\
36.2019546436285	0.01\\
36.2519573434125	0.01\\
36.3019600431965	0.01\\
36.3519627429806	0.01\\
36.4019654427646	0.01\\
36.4519681425486	0.01\\
36.5019708423326	0.01\\
36.5519735421166	0.01\\
36.6019762419006	0.01\\
36.6519789416847	0.01\\
36.7019816414687	0.01\\
36.7519843412527	0.01\\
36.8019870410367	0.01\\
36.8519897408207	0.01\\
36.9019924406048	0.01\\
36.9519951403888	0.01\\
37.0019978401728	0.01\\
37.0520005399568	0.01\\
37.1020032397408	0.01\\
37.1520059395248	0.01\\
37.2020086393089	0.01\\
37.2520113390929	0.01\\
37.3020140388769	0.01\\
37.3520167386609	0.01\\
37.4020194384449	0.01\\
37.4520221382289	0.01\\
37.502024838013	0.01\\
37.552027537797	0.01\\
37.602030237581	0.01\\
37.652032937365	0.01\\
37.702035637149	0.01\\
37.752038336933	0.01\\
37.8020410367171	0.01\\
37.8520437365011	0.01\\
37.9020464362851	0.01\\
37.9520491360691	0.01\\
38.0020518358531	0.01\\
38.0520545356372	0.01\\
38.1020572354212	0.01\\
38.1520599352052	0.01\\
38.2020626349892	0.01\\
38.2520653347732	0.01\\
38.3020680345572	0.01\\
38.3520707343413	0.01\\
38.4020734341253	0.01\\
38.4520761339093	0.01\\
38.5020788336933	0.01\\
38.5520815334773	0.01\\
38.6020842332613	0.01\\
38.6520869330454	0.01\\
38.7020896328294	0.01\\
38.7520923326134	0.01\\
38.8020950323974	0.01\\
38.8520977321814	0.01\\
38.9021004319654	0.01\\
38.9521031317495	0.01\\
39.0021058315335	0.01\\
39.0521085313175	0.01\\
39.1021112311015	0.01\\
39.1521139308855	0.01\\
39.2021166306696	0.01\\
39.2521193304536	0.01\\
39.3021220302376	0.01\\
39.3521247300216	0.01\\
39.4021274298056	0.01\\
39.4521301295896	0.01\\
39.5021328293737	0.01\\
39.5521355291577	0.01\\
39.6021382289417	0.01\\
39.6521409287257	0.01\\
39.7021436285097	0.01\\
39.7521463282937	0.01\\
39.8021490280778	0.01\\
39.8521517278618	0.01\\
39.9021544276458	0.01\\
39.9521571274298	0.01\\
40.0021598272138	0.01\\
40.0521625269978	0.01\\
40.1021652267819	0.01\\
40.1521679265659	0.01\\
40.2021706263499	0.01\\
40.2521733261339	0.01\\
40.3021760259179	0.01\\
40.3521787257019	0.01\\
40.402181425486	0.01\\
40.45218412527	0.01\\
40.502186825054	0.01\\
40.552189524838	0.01\\
40.602192224622	0.01\\
40.652194924406	0.01\\
40.7021976241901	0.01\\
40.7522003239741	0.01\\
40.8022030237581	0.01\\
40.8522057235421	0.01\\
40.9022084233261	0.01\\
40.9522111231102	0.01\\
41.0022138228942	0.01\\
41.0522165226782	0.01\\
41.1022192224622	0.01\\
41.1522219222462	0.01\\
41.2022246220302	0.01\\
41.2522273218143	0.01\\
41.3022300215983	0.01\\
41.3522327213823	0.01\\
41.4022354211663	0.01\\
41.4522381209503	0.01\\
41.5022408207343	0.01\\
41.5522435205184	0.01\\
41.6022462203024	0.01\\
41.6522489200864	0.01\\
41.7022516198704	0.01\\
41.7522543196544	0.01\\
41.8022570194385	0.01\\
41.8522597192225	0.01\\
41.9022624190065	0.01\\
41.9522651187905	0.01\\
42.0022678185745	0.01\\
42.0522705183585	0.01\\
42.1022732181426	0.01\\
42.1522759179266	0.01\\
42.2022786177106	0.01\\
42.2522813174946	0.01\\
42.3022840172786	0.01\\
42.3522867170626	0.01\\
42.4022894168467	0.01\\
42.4522921166307	0.01\\
42.5022948164147	0.01\\
42.5522975161987	0.01\\
42.6023002159827	0.01\\
42.6523029157667	0.01\\
42.7023056155508	0.01\\
42.7523083153348	0.01\\
42.8023110151188	0.01\\
42.8523137149028	0.01\\
42.9023164146868	0.01\\
42.9523191144708	0.01\\
43.0023218142549	0.01\\
43.0523245140389	0.01\\
43.1023272138229	0.01\\
43.1523299136069	0.01\\
43.2023326133909	0.01\\
43.2523353131749	0.01\\
43.302338012959	0.01\\
43.352340712743	0.01\\
43.402343412527	0.01\\
43.452346112311	0.01\\
43.502348812095	0.01\\
43.5523515118791	0.01\\
43.6023542116631	0.01\\
43.6523569114471	0.01\\
43.7023596112311	0.01\\
43.7523623110151	0.01\\
43.8023650107991	0.01\\
43.8523677105832	0.01\\
43.9023704103672	0.01\\
43.9523731101512	0.01\\
44.0023758099352	0.01\\
44.0523785097192	0.01\\
44.1023812095032	0.01\\
44.1523839092873	0.01\\
44.2023866090713	0.01\\
44.2523893088553	0.01\\
44.3023920086393	0.01\\
44.3523947084233	0.01\\
44.4023974082073	0.01\\
44.4524001079914	0.01\\
44.5024028077754	0.01\\
44.5524055075594	0.01\\
44.6024082073434	0.01\\
44.6524109071274	0.01\\
44.7024136069114	0.01\\
44.7524163066955	0.01\\
44.8024190064795	0.01\\
44.8524217062635	0.01\\
44.9024244060475	0.01\\
44.9524271058315	0.01\\
45.0024298056155	0.01\\
45.0524325053996	0.01\\
45.1024352051836	0.01\\
45.1524379049676	0.01\\
45.2024406047516	0.01\\
45.2524433045356	0.01\\
45.3024460043197	0.01\\
45.3524487041037	0.01\\
45.4024514038877	0.01\\
45.4524541036717	0.01\\
45.5024568034557	0.01\\
45.5524595032397	0.01\\
45.6024622030238	0.01\\
45.6524649028078	0.01\\
45.7024676025918	0.01\\
45.7524703023758	0.01\\
45.8024730021598	0.01\\
45.8524757019439	0.01\\
45.9024784017279	0.01\\
45.9524811015119	0.01\\
46.0024838012959	0.01\\
46.0524865010799	0.01\\
46.1024892008639	0.01\\
46.152491900648	0.01\\
46.202494600432	0.01\\
46.252497300216	0.01\\
46.3025	0.01\\
46.352502699784	0.01\\
46.402505399568	0.01\\
46.4525080993521	0.01\\
46.5025107991361	0.01\\
46.5525134989201	0.01\\
46.6025161987041	0.01\\
46.6525188984881	0.01\\
46.7025215982721	0.01\\
46.7525242980562	0.01\\
46.8025269978402	0.01\\
46.8525296976242	0.01\\
46.9025323974082	0.01\\
46.9525350971922	0.01\\
47.0025377969762	0.01\\
47.0525404967603	0.01\\
47.1025431965443	0.01\\
47.1525458963283	0.01\\
47.2025485961123	0.01\\
47.2525512958963	0.01\\
47.3025539956803	0.01\\
47.3525566954644	0.01\\
47.4025593952484	0.01\\
47.4525620950324	0.01\\
47.5025647948164	0.01\\
47.5525674946004	0.01\\
47.6025701943845	0.01\\
47.6525728941685	0.01\\
47.7025755939525	0.01\\
47.7525782937365	0.01\\
47.8025809935205	0.01\\
47.8525836933045	0.01\\
47.9025863930886	0.01\\
47.9525890928726	0.01\\
48.0025917926566	0.01\\
48.0525944924406	0.01\\
48.1025971922246	0.01\\
48.1525998920086	0.01\\
48.2026025917927	0.01\\
48.2526052915767	0.01\\
48.3026079913607	0.01\\
48.3526106911447	0.01\\
48.4026133909287	0.01\\
48.4526160907127	0.01\\
48.5026187904968	0.01\\
48.5526214902808	0.01\\
48.6026241900648	0.01\\
48.6526268898488	0.01\\
48.7026295896328	0.01\\
48.7526322894168	0.01\\
48.8026349892009	0.01\\
48.8526376889849	0.01\\
48.9026403887689	0.01\\
48.9526430885529	0.01\\
49.0026457883369	0.01\\
49.052648488121	0.01\\
49.102651187905	0.01\\
49.152653887689	0.01\\
49.202656587473	0.01\\
49.252659287257	0.01\\
49.302661987041	0.01\\
49.3526646868251	0.01\\
49.4026673866091	0.01\\
49.4526700863931	0.01\\
49.5026727861771	0.01\\
49.5526754859611	0.01\\
49.6026781857451	0.01\\
49.6526808855292	0.01\\
49.7026835853132	0.01\\
49.7526862850972	0.01\\
49.8026889848812	0.01\\
49.8526916846652	0.01\\
49.9026943844492	0.01\\
49.9526970842333	0.01\\
50.0026997840173	0.01\\
50.0527024838013	0.01\\
50.1027051835853	0.01\\
50.1527078833693	0.01\\
50.2027105831534	0.01\\
50.2527132829374	0.01\\
50.3027159827214	0.01\\
50.3527186825054	0.01\\
50.4027213822894	0.01\\
50.4527240820734	0.01\\
50.5027267818575	0.01\\
50.5527294816415	0.01\\
50.6027321814255	0.01\\
50.6527348812095	0.01\\
50.7027375809935	0.01\\
50.7527402807775	0.01\\
50.8027429805616	0.01\\
50.8527456803456	0.01\\
50.9027483801296	0.01\\
50.9527510799136	0.01\\
51.0027537796976	0.01\\
51.0527564794816	0.01\\
51.1027591792657	0.01\\
51.1527618790497	0.01\\
51.2027645788337	0.01\\
51.2527672786177	0.01\\
51.3027699784017	0.01\\
51.3527726781858	0.01\\
51.4027753779698	0.01\\
51.4527780777538	0.01\\
51.5027807775378	0.01\\
51.5527834773218	0.01\\
51.6027861771058	0.01\\
51.6527888768899	0.01\\
51.7027915766739	0.01\\
51.7527942764579	0.01\\
51.8027969762419	0.01\\
51.8527996760259	0.01\\
51.9028023758099	0.01\\
51.952805075594	0.01\\
52.002807775378	0.01\\
52.052810475162	0.01\\
52.102813174946	0.01\\
52.15281587473	0.01\\
52.202818574514	0.01\\
52.2528212742981	0.01\\
52.3028239740821	0.01\\
52.3528266738661	0.01\\
52.4028293736501	0.01\\
52.4528320734341	0.01\\
52.5028347732181	0.01\\
52.5528374730022	0.01\\
52.6028401727862	0.01\\
52.6528428725702	0.01\\
52.7028455723542	0.01\\
52.7528482721382	0.01\\
52.8028509719222	0.01\\
52.8528536717063	0.01\\
52.9028563714903	0.01\\
52.9528590712743	0.01\\
53.0028617710583	0.01\\
53.0528644708423	0.01\\
53.1028671706264	0.01\\
53.1528698704104	0.01\\
53.2028725701944	0.01\\
53.2528752699784	0.01\\
53.3028779697624	0.01\\
53.3528806695464	0.01\\
53.4028833693305	0.01\\
53.4528860691145	0.01\\
53.5028887688985	0.01\\
53.5528914686825	0.01\\
53.6028941684665	0.01\\
53.6528968682505	0.01\\
53.7028995680346	0.01\\
53.7529022678186	0.01\\
53.8029049676026	0.01\\
53.8529076673866	0.01\\
53.9029103671706	0.01\\
53.9529130669547	0.01\\
54.0029157667387	0.01\\
54.0529184665227	0.01\\
54.1029211663067	0.01\\
54.1529238660907	0.01\\
54.2029265658747	0.01\\
54.2529292656588	0.01\\
54.3029319654428	0.01\\
54.3529346652268	0.01\\
54.4029373650108	0.01\\
54.4529400647948	0.01\\
54.5029427645788	0.01\\
54.5529454643629	0.01\\
54.6029481641469	0.01\\
54.6529508639309	0.01\\
54.7029535637149	0.01\\
54.7529562634989	0.01\\
54.8029589632829	0.01\\
54.852961663067	0.01\\
54.902964362851	0.01\\
54.952967062635	0.01\\
55.002969762419	0.01\\
55.052972462203	0.01\\
55.102975161987	0.01\\
55.1529778617711	0.01\\
55.2029805615551	0.01\\
55.2529832613391	0.01\\
55.3029859611231	0.01\\
55.3529886609071	0.01\\
55.4029913606911	0.01\\
55.4529940604752	0.01\\
55.5029967602592	0.01\\
55.5529994600432	0.01\\
55.6030021598272	0.01\\
55.6530048596112	0.01\\
55.7030075593953	0.01\\
55.7530102591793	0.01\\
55.8030129589633	0.01\\
55.8530156587473	0.01\\
55.9030183585313	0.01\\
55.9530210583153	0.01\\
56.0030237580994	0.01\\
56.0530264578834	0.01\\
56.1030291576674	0.01\\
56.1530318574514	0.01\\
56.2030345572354	0.01\\
56.2530372570194	0.01\\
56.3030399568035	0.01\\
56.3530426565875	0.01\\
56.4030453563715	0.01\\
56.4530480561555	0.01\\
56.5030507559395	0.01\\
56.5530534557235	0.01\\
56.6030561555076	0.01\\
56.6530588552916	0.01\\
56.7030615550756	0.01\\
56.7530642548596	0.01\\
56.8030669546436	0.01\\
56.8530696544276	0.01\\
56.9030723542117	0.01\\
56.9530750539957	0.01\\
57.0030777537797	0.01\\
57.0530804535637	0.01\\
57.1030831533477	0.01\\
57.1530858531318	0.01\\
57.2030885529158	0.01\\
57.2530912526998	0.01\\
57.3030939524838	0.01\\
57.3530966522678	0.01\\
57.4030993520518	0.01\\
57.4531020518358	0.01\\
57.5031047516199	0.01\\
57.5531074514039	0.01\\
57.6031101511879	0.01\\
57.6531128509719	0.01\\
57.7031155507559	0.01\\
57.75311825054	0.01\\
57.803120950324	0.01\\
57.853123650108	0.01\\
57.903126349892	0.01\\
57.953129049676	0.01\\
58.00313174946	0.01\\
58.0531344492441	0.01\\
58.1031371490281	0.01\\
58.1531398488121	0.01\\
58.2031425485961	0.01\\
58.2531452483801	0.01\\
58.3031479481642	0.01\\
58.3531506479482	0.01\\
58.4031533477322	0.01\\
58.4531560475162	0.01\\
58.5031587473002	0.01\\
58.5531614470842	0.01\\
58.6031641468683	0.01\\
58.6531668466523	0.01\\
58.7031695464363	0.01\\
58.7531722462203	0.01\\
58.8031749460043	0.01\\
58.8531776457883	0.01\\
58.9031803455724	0.01\\
58.9531830453564	0.01\\
59.0031857451404	0.01\\
59.0531884449244	0.01\\
59.1031911447084	0.01\\
59.1531938444924	0.01\\
59.2031965442765	0.01\\
59.2531992440605	0.01\\
59.3032019438445	0.01\\
59.3532046436285	0.01\\
59.4032073434125	0.01\\
59.4532100431965	0.01\\
59.5032127429806	0.01\\
59.5532154427646	0.01\\
59.6032181425486	0.01\\
59.6532208423326	0.01\\
59.7032235421166	0.01\\
59.7532262419006	0.01\\
59.8032289416847	0.01\\
59.8532316414687	0.01\\
59.9032343412527	0.01\\
59.9532370410367	0.01\\
60.0032397408207	0.01\\
60.0532424406048	0.01\\
60.1032451403888	0.01\\
60.1532478401728	0.01\\
60.2032505399568	0.01\\
60.2532532397408	0.01\\
60.3032559395248	0.01\\
60.3532586393089	0.01\\
60.4032613390929	0.01\\
60.4532640388769	0.01\\
60.5032667386609	0.01\\
60.5532694384449	0.01\\
60.6032721382289	0.01\\
60.653274838013	0.01\\
60.703277537797	0.01\\
60.753280237581	0.01\\
60.803282937365	0.01\\
60.853285637149	0.01\\
60.9032883369331	0.01\\
60.9532910367171	0.01\\
61.0032937365011	0.01\\
61.0532964362851	0.01\\
61.1032991360691	0.01\\
61.1533018358531	0.01\\
61.2033045356371	0.01\\
61.2533072354212	0.01\\
61.3033099352052	0.01\\
61.3533126349892	0.01\\
61.4033153347732	0.01\\
61.4533180345572	0.01\\
61.5033207343413	0.01\\
61.5533234341253	0.01\\
61.6033261339093	0.01\\
61.6533288336933	0.01\\
61.7033315334773	0.01\\
61.7533342332613	0.01\\
61.8033369330454	0.01\\
61.8533396328294	0.01\\
61.9033423326134	0.01\\
61.9533450323974	0.01\\
62.0033477321814	0.01\\
62.0533504319654	0.01\\
62.1033531317495	0.01\\
62.1533558315335	0.01\\
62.2033585313175	0.01\\
62.2533612311015	0.01\\
62.3033639308855	0.01\\
62.3533666306695	0.01\\
62.4033693304536	0.01\\
62.4533720302376	0.01\\
62.5033747300216	0.01\\
62.5483771598272	0.21\\
62.5983798596112	0.21\\
62.6483825593953	0.21\\
62.6983852591793	0.21\\
62.7483879589633	0.21\\
62.7983906587473	0.21\\
62.8483933585313	0.21\\
62.8983960583153	0.21\\
62.9483987580994	0.21\\
62.9984014578834	0.21\\
63.0484041576674	0.21\\
63.0984068574514	0.21\\
63.1484095572354	0.21\\
63.1984122570194	0.21\\
63.2484149568035	0.21\\
63.2984176565875	0.21\\
63.3484203563715	0.21\\
63.3984230561555	0.21\\
63.4484257559395	0.21\\
63.4984284557235	0.21\\
63.5484311555076	0.21\\
63.5984338552916	0.21\\
63.6484365550756	0.21\\
63.6984392548596	0.21\\
63.7484419546436	0.21\\
63.7984446544276	0.21\\
63.8484473542117	0.21\\
63.8984500539957	0.21\\
63.9484527537797	0.21\\
63.9984554535637	0.21\\
64.0484581533477	0.21\\
64.0984608531318	0.21\\
64.1484635529158	0.21\\
64.1984662526998	0.21\\
64.2484689524838	0.21\\
64.2984716522678	0.21\\
64.3484743520518	0.21\\
64.3984770518359	0.21\\
64.4484797516199	0.21\\
64.4984824514039	0.21\\
64.5484851511879	0.21\\
64.5984878509719	0.21\\
64.648490550756	0.21\\
64.69849325054	0.21\\
64.748495950324	0.21\\
64.798498650108	0.21\\
64.848501349892	0.21\\
64.898504049676	0.21\\
64.94850674946	0.21\\
64.9985094492441	0.21\\
65.0485121490281	0.21\\
65.0985148488121	0.21\\
65.1485175485961	0.21\\
65.1985202483801	0.21\\
65.2485229481642	0.21\\
65.2985256479482	0.21\\
65.3485283477322	0.21\\
65.3985310475162	0.21\\
65.4485337473002	0.21\\
65.4985364470842	0.21\\
65.5485391468683	0.21\\
65.5985418466523	0.21\\
65.6485445464363	0.21\\
65.6985472462203	0.21\\
65.7485499460043	0.21\\
65.7985526457883	0.21\\
65.8485553455724	0.21\\
65.8985580453564	0.21\\
65.9485607451404	0.21\\
65.9985634449244	0.21\\
66.0485661447084	0.21\\
66.0985688444924	0.21\\
66.1485715442765	0.21\\
66.1985742440605	0.21\\
66.2485769438445	0.21\\
66.2985796436285	0.21\\
66.3485823434125	0.21\\
66.3985850431965	0.21\\
66.4485877429806	0.21\\
66.4985904427646	0.21\\
66.5485931425486	0.21\\
66.5985958423326	0.21\\
66.6485985421166	0.21\\
66.6986012419006	0.21\\
66.7486039416847	0.21\\
66.7986066414687	0.21\\
66.8486093412527	0.21\\
66.8986120410367	0.21\\
66.9486147408207	0.21\\
66.9986174406047	0.21\\
67.0486201403888	0.21\\
67.0986228401728	0.21\\
67.1486255399568	0.21\\
67.1986282397408	0.21\\
67.2486309395248	0.21\\
67.2986336393089	0.21\\
67.3486363390929	0.21\\
67.3986390388769	0.21\\
67.4486417386609	0.21\\
67.4986444384449	0.21\\
67.5486471382289	0.21\\
67.598649838013	0.21\\
67.648652537797	0.21\\
67.698655237581	0.21\\
67.748657937365	0.21\\
67.798660637149	0.21\\
67.8486633369331	0.21\\
67.8986660367171	0.21\\
67.9486687365011	0.21\\
67.9986714362851	0.21\\
68.0486741360691	0.21\\
68.0986768358531	0.21\\
68.1486795356372	0.21\\
68.1986822354212	0.21\\
68.2486849352052	0.21\\
68.2986876349892	0.21\\
68.3486903347732	0.21\\
68.3986930345572	0.21\\
68.4486957343413	0.21\\
68.4986984341253	0.21\\
68.5487011339093	0.21\\
68.5987038336933	0.21\\
68.6487065334773	0.21\\
68.6987092332613	0.21\\
68.7487119330454	0.21\\
68.7987146328294	0.21\\
68.8487173326134	0.21\\
68.8987200323974	0.21\\
68.9487227321814	0.21\\
68.9987254319654	0.21\\
69.0487281317495	0.21\\
69.0987308315335	0.21\\
69.1487335313175	0.21\\
69.1987362311015	0.21\\
69.2487389308855	0.21\\
69.2987416306696	0.21\\
69.3487443304536	0.21\\
69.3987470302376	0.21\\
69.4487497300216	0.21\\
69.4987524298056	0.21\\
69.5487551295896	0.21\\
69.5987578293737	0.21\\
69.6487605291577	0.21\\
69.6987632289417	0.21\\
69.7487659287257	0.21\\
69.7987686285097	0.21\\
69.8487713282937	0.21\\
69.8987740280778	0.21\\
69.9487767278618	0.21\\
69.9987794276458	0.21\\
70.0487821274298	0.21\\
70.0987848272138	0.21\\
70.1487875269979	0.21\\
70.1987902267819	0.21\\
70.2487929265659	0.21\\
70.2987956263499	0.21\\
70.3487983261339	0.21\\
70.3988010259179	0.21\\
70.4488037257019	0.21\\
70.498806425486	0.21\\
70.54880912527	0.21\\
70.598811825054	0.21\\
70.648814524838	0.21\\
70.698817224622	0.21\\
70.7488199244061	0.21\\
70.7988226241901	0.21\\
70.8488253239741	0.21\\
70.8988280237581	0.21\\
70.9488307235421	0.21\\
70.9988334233261	0.21\\
71.0488361231101	0.21\\
71.0988388228942	0.21\\
71.1488415226782	0.21\\
71.1988442224622	0.21\\
71.2488469222462	0.21\\
71.2988496220302	0.21\\
71.3488523218143	0.21\\
71.3988550215983	0.21\\
71.4488577213823	0.21\\
71.4988604211663	0.21\\
71.5488631209503	0.21\\
71.5988658207343	0.21\\
71.6488685205184	0.21\\
71.6988712203024	0.21\\
71.7488739200864	0.21\\
71.7988766198704	0.21\\
71.8488793196544	0.21\\
71.8988820194385	0.21\\
71.9488847192225	0.21\\
71.9988874190065	0.21\\
72.0488901187905	0.21\\
72.0988928185745	0.21\\
72.1488955183585	0.21\\
72.1988982181425	0.21\\
72.2489009179266	0.21\\
72.2989036177106	0.21\\
72.3489063174946	0.21\\
72.3989090172786	0.21\\
72.4489117170626	0.21\\
72.4989144168467	0.21\\
72.5489171166307	0.21\\
72.5989198164147	0.21\\
72.6489225161987	0.21\\
72.6989252159827	0.21\\
72.7489279157667	0.21\\
72.7989306155508	0.21\\
72.8489333153348	0.21\\
72.8989360151188	0.21\\
72.9489387149028	0.21\\
72.9989414146868	0.21\\
73.0489441144708	0.21\\
73.0989468142549	0.21\\
73.1489495140389	0.21\\
73.1989522138229	0.21\\
73.2489549136069	0.21\\
73.2989576133909	0.21\\
73.348960313175	0.21\\
73.398963012959	0.21\\
73.448965712743	0.21\\
73.498968412527	0.21\\
73.548971112311	0.21\\
73.598973812095	0.21\\
73.6489765118791	0.21\\
73.6989792116631	0.21\\
73.7489819114471	0.21\\
73.7989846112311	0.21\\
73.8489873110151	0.21\\
73.8989900107991	0.21\\
73.9489927105832	0.21\\
73.9989954103672	0.21\\
74.0489981101512	0.21\\
74.0990008099352	0.21\\
74.1490035097192	0.21\\
74.1990062095032	0.21\\
74.2490089092873	0.21\\
74.2990116090713	0.21\\
74.3490143088553	0.21\\
74.3990170086393	0.21\\
74.4490197084233	0.21\\
74.4990224082073	0.21\\
74.5490251079914	0.21\\
74.5990278077754	0.21\\
74.6490305075594	0.21\\
74.6990332073434	0.21\\
74.7490359071274	0.21\\
74.7990386069115	0.21\\
74.8490413066955	0.21\\
74.8990440064795	0.21\\
74.9490467062635	0.21\\
74.9990494060475	0.21\\
75.0490521058315	0.21\\
75.0990548056156	0.21\\
75.1490575053996	0.21\\
75.1990602051836	0.21\\
75.2490629049676	0.21\\
75.2990656047516	0.21\\
75.3490683045356	0.21\\
75.3990710043197	0.21\\
75.4490737041037	0.21\\
75.4990764038877	0.21\\
75.5490791036717	0.21\\
75.5990818034557	0.21\\
75.6490845032397	0.21\\
75.6990872030238	0.21\\
75.7490899028078	0.21\\
75.7990926025918	0.21\\
75.8490953023758	0.21\\
75.8990980021598	0.21\\
75.9491007019438	0.21\\
75.9991034017279	0.21\\
76.0491061015119	0.21\\
76.0991088012959	0.21\\
76.1491115010799	0.21\\
76.1991142008639	0.21\\
76.2491169006479	0.21\\
76.299119600432	0.21\\
76.349122300216	0.21\\
76.399125	0.21\\
76.449127699784	0.21\\
76.499130399568	0.21\\
76.5491330993521	0.21\\
76.5991357991361	0.21\\
76.6491384989201	0.21\\
76.6991411987041	0.21\\
76.7491438984881	0.21\\
76.7991465982721	0.21\\
76.8491492980562	0.21\\
76.8991519978402	0.21\\
76.9491546976242	0.21\\
76.9991573974082	0.21\\
77.0491600971922	0.21\\
77.0991627969762	0.21\\
77.1491654967603	0.21\\
77.1991681965443	0.21\\
77.2491708963283	0.21\\
77.2991735961123	0.21\\
77.3491762958963	0.21\\
77.3991789956804	0.21\\
77.4491816954644	0.21\\
77.4991843952484	0.21\\
77.5491870950324	0.21\\
77.5991897948164	0.21\\
77.6491924946004	0.21\\
77.6991951943844	0.21\\
77.7491978941685	0.21\\
77.7992005939525	0.21\\
77.8492032937365	0.21\\
77.8992059935205	0.21\\
77.9492086933045	0.21\\
77.9992113930886	0.21\\
78.0492140928726	0.21\\
78.0992167926566	0.21\\
78.1492194924406	0.21\\
78.1992221922246	0.21\\
78.2492248920086	0.21\\
78.2992275917927	0.21\\
78.3492302915767	0.21\\
78.3992329913607	0.21\\
78.4492356911447	0.21\\
78.4992383909287	0.21\\
78.5492410907127	0.21\\
78.5992437904968	0.21\\
78.6492464902808	0.21\\
78.6992491900648	0.21\\
78.7492518898488	0.21\\
78.7992545896328	0.21\\
78.8492572894169	0.21\\
78.8992599892009	0.21\\
78.9492626889849	0.21\\
78.9992653887689	0.21\\
79.0492680885529	0.21\\
79.0992707883369	0.21\\
79.149273488121	0.21\\
79.199276187905	0.21\\
79.249278887689	0.21\\
79.299281587473	0.21\\
79.349284287257	0.21\\
79.3992869870411	0.21\\
79.4492896868251	0.21\\
79.4992923866091	0.21\\
79.5492950863931	0.21\\
79.5992977861771	0.21\\
79.6493004859611	0.21\\
79.6993031857451	0.21\\
79.7493058855292	0.21\\
79.7993085853132	0.21\\
79.8493112850972	0.21\\
79.8993139848812	0.21\\
79.9493166846652	0.21\\
79.9993193844493	0.21\\
80.0493220842333	0.21\\
80.0993247840173	0.21\\
80.1493274838013	0.21\\
80.1993301835853	0.21\\
80.2493328833693	0.21\\
80.2993355831534	0.21\\
80.3493382829374	0.21\\
80.3993409827214	0.21\\
80.4493436825054	0.21\\
80.4993463822894	0.21\\
80.5493490820734	0.21\\
80.5993517818575	0.21\\
80.6493544816415	0.21\\
80.6993571814255	0.21\\
80.7493598812095	0.21\\
80.7993625809935	0.21\\
80.8493652807775	0.21\\
80.8993679805616	0.21\\
80.9493706803456	0.21\\
80.9993733801296	0.21\\
81.0493760799136	0.21\\
81.0993787796976	0.21\\
81.1493814794816	0.21\\
81.1993841792657	0.21\\
81.2493868790497	0.21\\
81.2993895788337	0.21\\
81.3493922786177	0.21\\
81.3993949784017	0.21\\
81.4493976781857	0.21\\
81.4994003779698	0.21\\
81.5494030777538	0.21\\
81.5994057775378	0.21\\
81.6494084773218	0.21\\
81.6994111771058	0.21\\
81.7494138768898	0.21\\
81.7994165766739	0.21\\
81.8494192764579	0.21\\
81.8994219762419	0.21\\
81.9494246760259	0.21\\
81.9994273758099	0.21\\
82.049430075594	0.21\\
82.099432775378	0.21\\
82.149435475162	0.21\\
82.199438174946	0.21\\
82.24944087473	0.21\\
82.299443574514	0.21\\
82.3494462742981	0.21\\
82.3994489740821	0.21\\
82.4494516738661	0.21\\
82.4994543736501	0.21\\
82.5494570734341	0.21\\
82.5994597732181	0.21\\
82.6294613930885	0.01\\
82.6794640928726	0.01\\
82.7294667926566	0.01\\
82.7794694924406	0.01\\
82.8294721922246	0.01\\
82.8794748920086	0.01\\
82.9294775917927	0.01\\
82.9794802915767	0.01\\
83.0294829913607	0.01\\
83.0794856911447	0.01\\
83.1294883909287	0.01\\
83.1794910907127	0.01\\
83.2294937904968	0.01\\
83.2794964902808	0.01\\
83.3294991900648	0.01\\
83.3795018898488	0.01\\
83.4295045896328	0.01\\
83.4795072894169	0.01\\
83.5295099892009	0.01\\
83.5795126889849	0.01\\
83.6295153887689	0.01\\
83.6795180885529	0.01\\
83.7295207883369	0.01\\
83.779523488121	0.01\\
83.829526187905	0.01\\
83.879528887689	0.01\\
83.929531587473	0.01\\
83.979534287257	0.01\\
84.029536987041	0.01\\
84.0795396868251	0.01\\
84.1295423866091	0.01\\
84.1795450863931	0.01\\
84.2295477861771	0.01\\
84.2795504859611	0.01\\
84.3295531857451	0.01\\
84.3795558855292	0.01\\
84.4295585853132	0.01\\
84.4795612850972	0.01\\
84.5295639848812	0.01\\
84.5795666846652	0.01\\
84.6295693844492	0.01\\
84.6795720842333	0.01\\
84.7295747840173	0.01\\
84.7795774838013	0.01\\
84.8295801835853	0.01\\
84.8795828833693	0.01\\
84.9295855831534	0.01\\
84.9795882829374	0.01\\
85.0295909827214	0.01\\
85.0795936825054	0.01\\
85.1295963822894	0.01\\
85.1795990820734	0.01\\
85.2296017818575	0.01\\
85.2796044816415	0.01\\
85.3296071814255	0.01\\
85.3796098812095	0.01\\
85.4296125809935	0.01\\
85.4796152807775	0.01\\
85.5296179805616	0.01\\
85.5796206803456	0.01\\
85.6296233801296	0.01\\
85.6796260799136	0.01\\
85.7296287796976	0.01\\
85.7796314794817	0.01\\
85.8296341792657	0.01\\
85.8796368790497	0.01\\
85.9296395788337	0.01\\
85.9796422786177	0.01\\
86.0296449784017	0.01\\
86.0796476781857	0.01\\
86.1296503779698	0.01\\
86.1796530777538	0.01\\
86.2296557775378	0.01\\
86.2796584773218	0.01\\
86.3296611771058	0.01\\
86.3796638768899	0.01\\
86.4296665766739	0.01\\
86.4796692764579	0.01\\
86.5296719762419	0.01\\
86.5796746760259	0.01\\
86.6296773758099	0.01\\
86.679680075594	0.01\\
86.729682775378	0.01\\
86.779685475162	0.01\\
86.829688174946	0.01\\
86.87969087473	0.01\\
86.929693574514	0.01\\
86.9796962742981	0.01\\
87.0296989740821	0.01\\
87.0797016738661	0.01\\
87.1297043736501	0.01\\
87.1797070734341	0.01\\
87.2297097732181	0.01\\
87.2797124730022	0.01\\
87.3297151727862	0.01\\
87.3797178725702	0.01\\
87.4297205723542	0.01\\
87.4797232721382	0.01\\
87.5297259719222	0.01\\
87.5797286717063	0.01\\
87.6297313714903	0.01\\
87.6797340712743	0.01\\
87.7297367710583	0.01\\
87.7797394708423	0.01\\
87.8297421706264	0.01\\
87.8797448704104	0.01\\
87.9297475701944	0.01\\
87.9797502699784	0.01\\
88.0297529697624	0.01\\
88.0797556695464	0.01\\
88.1297583693305	0.01\\
88.1797610691145	0.01\\
88.2297637688985	0.01\\
88.2797664686825	0.01\\
88.3297691684665	0.01\\
88.3797718682505	0.01\\
88.4297745680346	0.01\\
88.4797772678186	0.01\\
88.5297799676026	0.01\\
88.5797826673866	0.01\\
88.6297853671706	0.01\\
88.6797880669546	0.01\\
88.7297907667387	0.01\\
88.7797934665227	0.01\\
88.8297961663067	0.01\\
88.8797988660907	0.01\\
88.9298015658747	0.01\\
88.9798042656588	0.01\\
89.0298069654428	0.01\\
89.0798096652268	0.01\\
89.1298123650108	0.01\\
89.1798150647948	0.01\\
89.2298177645788	0.01\\
89.2798204643629	0.01\\
89.3298231641469	0.01\\
89.3798258639309	0.01\\
89.4298285637149	0.01\\
89.4798312634989	0.01\\
89.5298339632829	0.01\\
89.579836663067	0.01\\
89.629839362851	0.01\\
89.679842062635	0.01\\
89.729844762419	0.01\\
89.779847462203	0.01\\
89.8298501619871	0.01\\
89.8798528617711	0.01\\
89.9298555615551	0.01\\
89.9798582613391	0.01\\
90.0298609611231	0.01\\
90.0798636609071	0.01\\
90.1298663606911	0.01\\
90.1798690604752	0.01\\
90.2298717602592	0.01\\
90.2798744600432	0.01\\
90.3298771598272	0.01\\
90.3798798596112	0.01\\
90.4298825593953	0.01\\
90.4798852591793	0.01\\
90.5298879589633	0.01\\
90.5798906587473	0.01\\
90.6298933585313	0.01\\
90.6798960583153	0.01\\
90.7298987580994	0.01\\
90.7799014578834	0.01\\
90.8299041576674	0.01\\
90.8799068574514	0.01\\
90.9299095572354	0.01\\
90.9799122570195	0.01\\
91.0299149568035	0.01\\
91.0799176565875	0.01\\
91.1299203563715	0.01\\
91.1799230561555	0.01\\
91.2299257559395	0.01\\
91.2799284557236	0.01\\
91.3299311555076	0.01\\
91.3799338552916	0.01\\
91.4299365550756	0.01\\
91.4799392548596	0.01\\
91.5299419546436	0.01\\
91.5799446544276	0.01\\
91.6299473542117	0.01\\
91.6799500539957	0.01\\
91.7299527537797	0.01\\
91.7799554535637	0.01\\
91.8299581533477	0.01\\
91.8799608531317	0.01\\
91.9299635529158	0.01\\
91.9799662526998	0.01\\
92.0299689524838	0.01\\
92.0799716522678	0.01\\
92.1299743520518	0.01\\
92.1799770518358	0.01\\
92.2299797516199	0.01\\
92.2799824514039	0.01\\
92.3299851511879	0.01\\
92.3799878509719	0.01\\
92.4299905507559	0.01\\
92.47999325054	0.01\\
92.529995950324	0.01\\
92.579998650108	0.01\\
92.605	0.01\\
};
\addlegendentry{+ 1cm};

\addplot [color=gray,solid,line width=0.2pt]
  table[row sep=crcr]{0	-0.01\\
0.0500026997840173	-0.01\\
0.100005399568035	-0.01\\
0.150008099352052	-0.01\\
0.200010799136069	-0.01\\
0.250013498920086	-0.01\\
0.300016198704104	-0.01\\
0.350018898488121	-0.01\\
0.400021598272138	-0.01\\
0.450024298056156	-0.01\\
0.500026997840173	-0.01\\
0.55002969762419	-0.01\\
0.600032397408207	-0.01\\
0.650035097192225	-0.01\\
0.700037796976242	-0.01\\
0.750040496760259	-0.01\\
0.800043196544276	-0.01\\
0.850045896328294	-0.01\\
0.900048596112311	-0.01\\
0.950051295896328	-0.01\\
1.00005399568035	-0.01\\
1.05005669546436	-0.01\\
1.10005939524838	-0.01\\
1.1500620950324	-0.01\\
1.20006479481641	-0.01\\
1.25006749460043	-0.01\\
1.30007019438445	-0.01\\
1.35007289416847	-0.01\\
1.40007559395248	-0.01\\
1.4500782937365	-0.01\\
1.50008099352052	-0.01\\
1.55008369330454	-0.01\\
1.60008639308855	-0.01\\
1.65008909287257	-0.01\\
1.70009179265659	-0.01\\
1.7500944924406	-0.01\\
1.80009719222462	-0.01\\
1.85009989200864	-0.01\\
1.90010259179266	-0.01\\
1.95010529157667	-0.01\\
2.00010799136069	-0.01\\
2.05011069114471	-0.01\\
2.10011339092873	-0.01\\
2.15011609071274	-0.01\\
2.20011879049676	-0.01\\
2.25012149028078	-0.01\\
2.30012419006479	-0.01\\
2.35012688984881	-0.01\\
2.40012958963283	-0.01\\
2.45013228941685	-0.01\\
2.50013498920086	-0.01\\
2.55013768898488	-0.01\\
2.6001403887689	-0.01\\
2.65014308855292	-0.01\\
2.70014578833693	-0.01\\
2.75014848812095	-0.01\\
2.80015118790497	-0.01\\
2.85015388768899	-0.01\\
2.900156587473	-0.01\\
2.95015928725702	-0.01\\
3.00016198704104	-0.01\\
3.05016468682505	-0.01\\
3.10016738660907	-0.01\\
3.15017008639309	-0.01\\
3.20017278617711	-0.01\\
3.25017548596112	-0.01\\
3.30017818574514	-0.01\\
3.35018088552916	-0.01\\
3.40018358531318	-0.01\\
3.45018628509719	-0.01\\
3.50018898488121	-0.01\\
3.55019168466523	-0.01\\
3.60019438444924	-0.01\\
3.65019708423326	-0.01\\
3.70019978401728	-0.01\\
3.7502024838013	-0.01\\
3.80020518358531	-0.01\\
3.85020788336933	-0.01\\
3.90021058315335	-0.01\\
3.95021328293736	-0.01\\
4.00021598272138	-0.01\\
4.0502186825054	-0.01\\
4.10022138228942	-0.01\\
4.15022408207343	-0.01\\
4.20022678185745	-0.01\\
4.25022948164147	-0.01\\
4.30023218142549	-0.01\\
4.3502348812095	-0.01\\
4.40023758099352	-0.01\\
4.45024028077754	-0.01\\
4.50024298056155	-0.01\\
4.55024568034557	-0.01\\
4.60024838012959	-0.01\\
4.65025107991361	-0.01\\
4.70025377969762	-0.01\\
4.75025647948164	-0.01\\
4.80025917926566	-0.01\\
4.85026187904968	-0.01\\
4.90026457883369	-0.01\\
4.95026727861771	-0.01\\
5.00026997840173	-0.01\\
5.05027267818575	-0.01\\
5.10027537796976	-0.01\\
5.15027807775378	-0.01\\
5.2002807775378	-0.01\\
5.25028347732181	-0.01\\
5.30028617710583	-0.01\\
5.35028887688985	-0.01\\
5.40029157667387	-0.01\\
5.45029427645788	-0.01\\
5.5002969762419	-0.01\\
5.55029967602592	-0.01\\
5.60030237580994	-0.01\\
5.65030507559395	-0.01\\
5.70030777537797	-0.01\\
5.75031047516199	-0.01\\
5.800313174946	-0.01\\
5.85031587473002	-0.01\\
5.90031857451404	-0.01\\
5.95032127429806	-0.01\\
6.00032397408207	-0.01\\
6.05032667386609	-0.01\\
6.10032937365011	-0.01\\
6.15033207343413	-0.01\\
6.20033477321814	-0.01\\
6.25033747300216	-0.01\\
6.30034017278618	-0.01\\
6.3503428725702	-0.01\\
6.40034557235421	-0.01\\
6.45034827213823	-0.01\\
6.50035097192225	-0.01\\
6.55035367170626	-0.01\\
6.60035637149028	-0.01\\
6.6503590712743	-0.01\\
6.70036177105832	-0.01\\
6.75036447084233	-0.01\\
6.80036717062635	-0.01\\
6.85036987041037	-0.01\\
6.90037257019439	-0.01\\
6.9503752699784	-0.01\\
7.00037796976242	-0.01\\
7.05038066954644	-0.01\\
7.10038336933045	-0.01\\
7.15038606911447	-0.01\\
7.20038876889849	-0.01\\
7.25039146868251	-0.01\\
7.30039416846652	-0.01\\
7.35039686825054	-0.01\\
7.40039956803456	-0.01\\
7.45040226781857	-0.01\\
7.50040496760259	-0.01\\
7.55040766738661	-0.01\\
7.60041036717063	-0.01\\
7.65041306695464	-0.01\\
7.70041576673866	-0.01\\
7.75041846652268	-0.01\\
7.8004211663067	-0.01\\
7.85042386609071	-0.01\\
7.90042656587473	-0.01\\
7.95042926565875	-0.01\\
8.00043196544276	-0.01\\
8.05043466522678	-0.01\\
8.1004373650108	-0.01\\
8.15044006479482	-0.01\\
8.20044276457883	-0.01\\
8.25044546436285	-0.01\\
8.30044816414687	-0.01\\
8.35045086393089	-0.01\\
8.4004535637149	-0.01\\
8.45045626349892	-0.01\\
8.50045896328294	-0.01\\
8.55046166306696	-0.01\\
8.60046436285097	-0.01\\
8.65046706263499	-0.01\\
8.70046976241901	-0.01\\
8.75047246220302	-0.01\\
8.80047516198704	-0.01\\
8.85047786177106	-0.01\\
8.90048056155508	-0.01\\
8.95048326133909	-0.01\\
9.00048596112311	-0.01\\
9.05048866090713	-0.01\\
9.10049136069114	-0.01\\
9.15049406047516	-0.01\\
9.20049676025918	-0.01\\
9.2504994600432	-0.01\\
9.30050215982721	-0.01\\
9.35050485961123	-0.01\\
9.40050755939525	-0.01\\
9.45051025917927	-0.01\\
9.50051295896328	-0.01\\
9.5505156587473	-0.01\\
9.60051835853132	-0.01\\
9.65052105831533	-0.01\\
9.70052375809935	-0.01\\
9.75052645788337	-0.01\\
9.80052915766739	-0.01\\
9.8505318574514	-0.01\\
9.90053455723542	-0.01\\
9.95053725701944	-0.01\\
10.0005399568035	-0.01\\
10.0505426565875	-0.01\\
10.1005453563715	-0.01\\
10.1505480561555	-0.01\\
10.2005507559395	-0.01\\
10.2505534557235	-0.01\\
10.3005561555076	-0.01\\
10.3505588552916	-0.01\\
10.4005615550756	-0.01\\
10.4505642548596	-0.01\\
10.5005669546436	-0.01\\
10.5505696544276	-0.01\\
10.6005723542117	-0.01\\
10.6505750539957	-0.01\\
10.7005777537797	-0.01\\
10.7505804535637	-0.01\\
10.8005831533477	-0.01\\
10.8505858531317	-0.01\\
10.9005885529158	-0.01\\
10.9505912526998	-0.01\\
11.0005939524838	-0.01\\
11.0505966522678	-0.01\\
11.1005993520518	-0.01\\
11.1506020518359	-0.01\\
11.2006047516199	-0.01\\
11.2506074514039	-0.01\\
11.3006101511879	-0.01\\
11.3506128509719	-0.01\\
11.4006155507559	-0.01\\
11.45061825054	-0.01\\
11.500620950324	-0.01\\
11.550623650108	-0.01\\
11.600626349892	-0.01\\
11.650629049676	-0.01\\
11.70063174946	-0.01\\
11.7506344492441	-0.01\\
11.8006371490281	-0.01\\
11.8506398488121	-0.01\\
11.9006425485961	-0.01\\
11.9506452483801	-0.01\\
12.0006479481641	-0.01\\
12.0506506479482	-0.01\\
12.1006533477322	-0.01\\
12.1506560475162	-0.01\\
12.2006587473002	-0.01\\
12.2506614470842	-0.01\\
12.3006641468683	-0.01\\
12.3506668466523	-0.01\\
12.4006695464363	-0.01\\
12.4506722462203	-0.01\\
12.5006749460043	-0.01\\
12.5506776457883	-0.01\\
12.6006803455724	-0.01\\
12.6506830453564	-0.01\\
12.7006857451404	-0.01\\
12.7506884449244	-0.01\\
12.8006911447084	-0.01\\
12.8506938444924	-0.01\\
12.9006965442765	-0.01\\
12.9506992440605	-0.01\\
13.0007019438445	-0.01\\
13.0507046436285	-0.01\\
13.1007073434125	-0.01\\
13.1507100431965	-0.01\\
13.2007127429806	-0.01\\
13.2507154427646	-0.01\\
13.3007181425486	-0.01\\
13.3507208423326	-0.01\\
13.4007235421166	-0.01\\
13.4507262419006	-0.01\\
13.5007289416847	-0.01\\
13.5507316414687	-0.01\\
13.6007343412527	-0.01\\
13.6507370410367	-0.01\\
13.7007397408207	-0.01\\
13.7507424406048	-0.01\\
13.8007451403888	-0.01\\
13.8507478401728	-0.01\\
13.9007505399568	-0.01\\
13.9507532397408	-0.01\\
14.0007559395248	-0.01\\
14.0507586393089	-0.01\\
14.1007613390929	-0.01\\
14.1507640388769	-0.01\\
14.2007667386609	-0.01\\
14.2507694384449	-0.01\\
14.3007721382289	-0.01\\
14.350774838013	-0.01\\
14.400777537797	-0.01\\
14.450780237581	-0.01\\
14.500782937365	-0.01\\
14.550785637149	-0.01\\
14.600788336933	-0.01\\
14.6507910367171	-0.01\\
14.7007937365011	-0.01\\
14.7507964362851	-0.01\\
14.8007991360691	-0.01\\
14.8508018358531	-0.01\\
14.9008045356372	-0.01\\
14.9508072354212	-0.01\\
15.0008099352052	-0.01\\
15.0508126349892	-0.01\\
15.1008153347732	-0.01\\
15.1508180345572	-0.01\\
15.2008207343413	-0.01\\
15.2508234341253	-0.01\\
15.3008261339093	-0.01\\
15.3508288336933	-0.01\\
15.4008315334773	-0.01\\
15.4508342332613	-0.01\\
15.5008369330454	-0.01\\
15.5508396328294	-0.01\\
15.6008423326134	-0.01\\
15.6508450323974	-0.01\\
15.7008477321814	-0.01\\
15.7508504319654	-0.01\\
15.8008531317495	-0.01\\
15.8508558315335	-0.01\\
15.9008585313175	-0.01\\
15.9508612311015	-0.01\\
16.0008639308855	-0.01\\
16.0508666306695	-0.01\\
16.1008693304536	-0.01\\
16.1508720302376	-0.01\\
16.2008747300216	-0.01\\
16.2508774298056	-0.01\\
16.3008801295896	-0.01\\
16.3508828293737	-0.01\\
16.4008855291577	-0.01\\
16.4508882289417	-0.01\\
16.5008909287257	-0.01\\
16.5508936285097	-0.01\\
16.6008963282937	-0.01\\
16.6508990280778	-0.01\\
16.7009017278618	-0.01\\
16.7509044276458	-0.01\\
16.8009071274298	-0.01\\
16.8509098272138	-0.01\\
16.9009125269978	-0.01\\
16.9509152267819	-0.01\\
17.0009179265659	-0.01\\
17.0509206263499	-0.01\\
17.1009233261339	-0.01\\
17.1509260259179	-0.01\\
17.2009287257019	-0.01\\
17.250931425486	-0.01\\
17.30093412527	-0.01\\
17.350936825054	-0.01\\
17.400939524838	-0.01\\
17.450942224622	-0.01\\
17.500944924406	-0.01\\
17.5509476241901	-0.01\\
17.6009503239741	-0.01\\
17.6509530237581	-0.01\\
17.7009557235421	-0.01\\
17.7509584233261	-0.01\\
17.8009611231102	-0.01\\
17.8509638228942	-0.01\\
17.9009665226782	-0.01\\
17.9509692224622	-0.01\\
18.0009719222462	-0.01\\
18.0509746220302	-0.01\\
18.1009773218143	-0.01\\
18.1509800215983	-0.01\\
18.2009827213823	-0.01\\
18.2509854211663	-0.01\\
18.3009881209503	-0.01\\
18.3509908207343	-0.01\\
18.4009935205184	-0.01\\
18.4509962203024	-0.01\\
18.5009989200864	-0.01\\
18.5510016198704	-0.01\\
18.6010043196544	-0.01\\
18.6510070194384	-0.01\\
18.7010097192225	-0.01\\
18.7510124190065	-0.01\\
18.8010151187905	-0.01\\
18.8510178185745	-0.01\\
18.9010205183585	-0.01\\
18.9510232181425	-0.01\\
19.0010259179266	-0.01\\
19.0510286177106	-0.01\\
19.1010313174946	-0.01\\
19.1510340172786	-0.01\\
19.2010367170626	-0.01\\
19.2510394168467	-0.01\\
19.3010421166307	-0.01\\
19.3510448164147	-0.01\\
19.4010475161987	-0.01\\
19.4510502159827	-0.01\\
19.5010529157667	-0.01\\
19.5510556155508	-0.01\\
19.6010583153348	-0.01\\
19.6510610151188	-0.01\\
19.7010637149028	-0.01\\
19.7510664146868	-0.01\\
19.8010691144708	-0.01\\
19.8510718142549	-0.01\\
19.9010745140389	-0.01\\
19.9510772138229	-0.01\\
20.0010799136069	-0.01\\
20.0510826133909	-0.01\\
20.1010853131749	-0.01\\
20.151088012959	-0.01\\
20.201090712743	-0.01\\
20.251093412527	-0.01\\
20.301096112311	-0.01\\
20.351098812095	-0.01\\
20.4011015118791	-0.01\\
20.4511042116631	-0.01\\
20.5011069114471	-0.01\\
20.5511096112311	-0.01\\
20.6011123110151	-0.01\\
20.6511150107991	-0.01\\
20.7011177105832	-0.01\\
20.7511204103672	-0.01\\
20.8011231101512	-0.01\\
20.8511258099352	-0.01\\
20.9011285097192	-0.01\\
20.9511312095032	-0.01\\
21.0011339092873	-0.01\\
21.0511366090713	-0.01\\
21.1011393088553	-0.01\\
21.1511420086393	-0.01\\
21.2011447084233	-0.01\\
21.2511474082073	-0.01\\
21.3011501079914	-0.01\\
21.3511528077754	-0.01\\
21.4011555075594	-0.01\\
21.4511582073434	-0.01\\
21.5011609071274	-0.01\\
21.5511636069114	-0.01\\
21.6011663066955	-0.01\\
21.6511690064795	-0.01\\
21.7011717062635	-0.01\\
21.7511744060475	-0.01\\
21.8011771058315	-0.01\\
21.8511798056156	-0.01\\
21.9011825053996	-0.01\\
21.9511852051836	-0.01\\
22.0011879049676	-0.01\\
22.0511906047516	-0.01\\
22.1011933045356	-0.01\\
22.1511960043197	-0.01\\
22.2011987041037	-0.01\\
22.2512014038877	-0.01\\
22.3012041036717	-0.01\\
22.3512068034557	-0.01\\
22.4012095032397	-0.01\\
22.4512122030238	-0.01\\
22.5012149028078	-0.01\\
22.5512176025918	-0.01\\
22.6012203023758	-0.01\\
22.6512230021598	-0.01\\
22.7012257019438	-0.01\\
22.7512284017279	-0.01\\
22.8012311015119	-0.01\\
22.8512338012959	-0.01\\
22.9012365010799	-0.01\\
22.9512392008639	-0.01\\
23.0012419006479	-0.01\\
23.051244600432	-0.01\\
23.101247300216	-0.01\\
23.15125	-0.01\\
23.201252699784	-0.01\\
23.251255399568	-0.01\\
23.3012580993521	-0.01\\
23.3512607991361	-0.01\\
23.4012634989201	-0.01\\
23.4512661987041	-0.01\\
23.5012688984881	-0.01\\
23.5512715982721	-0.01\\
23.6012742980562	-0.01\\
23.6512769978402	-0.01\\
23.7012796976242	-0.01\\
23.7512823974082	-0.01\\
23.8012850971922	-0.01\\
23.8512877969762	-0.01\\
23.9012904967603	-0.01\\
23.9512931965443	-0.01\\
24.0012958963283	-0.01\\
24.0512985961123	-0.01\\
24.1013012958963	-0.01\\
24.1513039956803	-0.01\\
24.2013066954644	-0.01\\
24.2513093952484	-0.01\\
24.3013120950324	-0.01\\
24.3513147948164	-0.01\\
24.4013174946004	-0.01\\
24.4513201943845	-0.01\\
24.5013228941685	-0.01\\
24.5513255939525	-0.01\\
24.6013282937365	-0.01\\
24.6513309935205	-0.01\\
24.7013336933045	-0.01\\
24.7513363930886	-0.01\\
24.8013390928726	-0.01\\
24.8513417926566	-0.01\\
24.9013444924406	-0.01\\
24.9513471922246	-0.01\\
25.0013498920086	-0.01\\
25.0513525917927	-0.01\\
25.1013552915767	-0.01\\
25.1513579913607	-0.01\\
25.2013606911447	-0.01\\
25.2513633909287	-0.01\\
25.3013660907127	-0.01\\
25.3513687904968	-0.01\\
25.4013714902808	-0.01\\
25.4513741900648	-0.01\\
25.5013768898488	-0.01\\
25.5513795896328	-0.01\\
25.6013822894168	-0.01\\
25.6513849892009	-0.01\\
25.7013876889849	-0.01\\
25.7513903887689	-0.01\\
25.8013930885529	-0.01\\
25.8513957883369	-0.01\\
25.901398488121	-0.01\\
25.951401187905	-0.01\\
26.001403887689	-0.01\\
26.051406587473	-0.01\\
26.101409287257	-0.01\\
26.151411987041	-0.01\\
26.2014146868251	-0.01\\
26.2514173866091	-0.01\\
26.3014200863931	-0.01\\
26.3514227861771	-0.01\\
26.4014254859611	-0.01\\
26.4514281857451	-0.01\\
26.5014308855292	-0.01\\
26.5514335853132	-0.01\\
26.6014362850972	-0.01\\
26.6514389848812	-0.01\\
26.7014416846652	-0.01\\
26.7514443844492	-0.01\\
26.8014470842333	-0.01\\
26.8514497840173	-0.01\\
26.9014524838013	-0.01\\
26.9514551835853	-0.01\\
27.0014578833693	-0.01\\
27.0514605831534	-0.01\\
27.1014632829374	-0.01\\
27.1514659827214	-0.01\\
27.2014686825054	-0.01\\
27.2514713822894	-0.01\\
27.3014740820734	-0.01\\
27.3514767818575	-0.01\\
27.4014794816415	-0.01\\
27.4514821814255	-0.01\\
27.5014848812095	-0.01\\
27.5514875809935	-0.01\\
27.6014902807775	-0.01\\
27.6514929805616	-0.01\\
27.7014956803456	-0.01\\
27.7514983801296	-0.01\\
27.8015010799136	-0.01\\
27.8515037796976	-0.01\\
27.9015064794816	-0.01\\
27.9515091792657	-0.01\\
28.0015118790497	-0.01\\
28.0515145788337	-0.01\\
28.1015172786177	-0.01\\
28.1515199784017	-0.01\\
28.2015226781857	-0.01\\
28.2515253779698	-0.01\\
28.3015280777538	-0.01\\
28.3515307775378	-0.01\\
28.4015334773218	-0.01\\
28.4515361771058	-0.01\\
28.5015388768899	-0.01\\
28.5515415766739	-0.01\\
28.6015442764579	-0.01\\
28.6515469762419	-0.01\\
28.7015496760259	-0.01\\
28.7515523758099	-0.01\\
28.801555075594	-0.01\\
28.851557775378	-0.01\\
28.901560475162	-0.01\\
28.951563174946	-0.01\\
29.00156587473	-0.01\\
29.051568574514	-0.01\\
29.1015712742981	-0.01\\
29.1515739740821	-0.01\\
29.2015766738661	-0.01\\
29.2515793736501	-0.01\\
29.3015820734341	-0.01\\
29.3515847732181	-0.01\\
29.4015874730022	-0.01\\
29.4515901727862	-0.01\\
29.5015928725702	-0.01\\
29.5515955723542	-0.01\\
29.6015982721382	-0.01\\
29.6516009719222	-0.01\\
29.7016036717063	-0.01\\
29.7516063714903	-0.01\\
29.8016090712743	-0.01\\
29.8516117710583	-0.01\\
29.9016144708423	-0.01\\
29.9516171706264	-0.01\\
30.0016198704104	-0.01\\
30.0516225701944	-0.01\\
30.1016252699784	-0.01\\
30.1516279697624	-0.01\\
30.2016306695464	-0.01\\
30.2516333693305	-0.01\\
30.3016360691145	-0.01\\
30.3516387688985	-0.01\\
30.4016414686825	-0.01\\
30.4516441684665	-0.01\\
30.5016468682505	-0.01\\
30.5516495680346	-0.01\\
30.6016522678186	-0.01\\
30.6516549676026	-0.01\\
30.7016576673866	-0.01\\
30.7516603671706	-0.01\\
30.8016630669546	-0.01\\
30.8516657667387	-0.01\\
30.9016684665227	-0.01\\
30.9516711663067	-0.01\\
31.0016738660907	-0.01\\
31.0516765658747	-0.01\\
31.1016792656587	-0.01\\
31.1516819654428	-0.01\\
31.2016846652268	-0.01\\
31.2516873650108	-0.01\\
31.3016900647948	-0.01\\
31.3516927645788	-0.01\\
31.4016954643629	-0.01\\
31.4516981641469	-0.01\\
31.5017008639309	-0.01\\
31.5517035637149	-0.01\\
31.6017062634989	-0.01\\
31.6517089632829	-0.01\\
31.701711663067	-0.01\\
31.751714362851	-0.01\\
31.801717062635	-0.01\\
31.851719762419	-0.01\\
31.901722462203	-0.01\\
31.951725161987	-0.01\\
32.0017278617711	-0.01\\
32.0517305615551	-0.01\\
32.1017332613391	-0.01\\
32.1517359611231	-0.01\\
32.2017386609071	-0.01\\
32.2517413606911	-0.01\\
32.3017440604752	-0.01\\
32.3517467602592	-0.01\\
32.4017494600432	-0.01\\
32.4517521598272	-0.01\\
32.5017548596112	-0.01\\
32.5517575593952	-0.01\\
32.6017602591793	-0.01\\
32.6517629589633	-0.01\\
32.7017656587473	-0.01\\
32.7517683585313	-0.01\\
32.8017710583153	-0.01\\
32.8517737580993	-0.01\\
32.9017764578834	-0.01\\
32.9517791576674	-0.01\\
33.0017818574514	-0.01\\
33.0517845572354	-0.01\\
33.1017872570194	-0.01\\
33.1517899568035	-0.01\\
33.2017926565875	-0.01\\
33.2517953563715	-0.01\\
33.3017980561555	-0.01\\
33.3518007559395	-0.01\\
33.4018034557235	-0.01\\
33.4518061555076	-0.01\\
33.5018088552916	-0.01\\
33.5518115550756	-0.01\\
33.6018142548596	-0.01\\
33.6518169546436	-0.01\\
33.7018196544277	-0.01\\
33.7518223542117	-0.01\\
33.8018250539957	-0.01\\
33.8518277537797	-0.01\\
33.9018304535637	-0.01\\
33.9518331533477	-0.01\\
34.0018358531318	-0.01\\
34.0518385529158	-0.01\\
34.1018412526998	-0.01\\
34.1518439524838	-0.01\\
34.2018466522678	-0.01\\
34.2518493520518	-0.01\\
34.3018520518359	-0.01\\
34.3518547516199	-0.01\\
34.4018574514039	-0.01\\
34.4518601511879	-0.01\\
34.5018628509719	-0.01\\
34.5518655507559	-0.01\\
34.60186825054	-0.01\\
34.651870950324	-0.01\\
34.701873650108	-0.01\\
34.751876349892	-0.01\\
34.801879049676	-0.01\\
34.85188174946	-0.01\\
34.9018844492441	-0.01\\
34.9518871490281	-0.01\\
35.0018898488121	-0.01\\
35.0518925485961	-0.01\\
35.1018952483801	-0.01\\
35.1518979481641	-0.01\\
35.2019006479482	-0.01\\
35.2519033477322	-0.01\\
35.3019060475162	-0.01\\
35.3519087473002	-0.01\\
35.4019114470842	-0.01\\
35.4519141468683	-0.01\\
35.5019168466523	-0.01\\
35.5519195464363	-0.01\\
35.6019222462203	-0.01\\
35.6519249460043	-0.01\\
35.7019276457883	-0.01\\
35.7519303455724	-0.01\\
35.8019330453564	-0.01\\
35.8519357451404	-0.01\\
35.9019384449244	-0.01\\
35.9519411447084	-0.01\\
36.0019438444924	-0.01\\
36.0519465442765	-0.01\\
36.1019492440605	-0.01\\
36.1519519438445	-0.01\\
36.2019546436285	-0.01\\
36.2519573434125	-0.01\\
36.3019600431965	-0.01\\
36.3519627429806	-0.01\\
36.4019654427646	-0.01\\
36.4519681425486	-0.01\\
36.5019708423326	-0.01\\
36.5519735421166	-0.01\\
36.6019762419006	-0.01\\
36.6519789416847	-0.01\\
36.7019816414687	-0.01\\
36.7519843412527	-0.01\\
36.8019870410367	-0.01\\
36.8519897408207	-0.01\\
36.9019924406048	-0.01\\
36.9519951403888	-0.01\\
37.0019978401728	-0.01\\
37.0520005399568	-0.01\\
37.1020032397408	-0.01\\
37.1520059395248	-0.01\\
37.2020086393089	-0.01\\
37.2520113390929	-0.01\\
37.3020140388769	-0.01\\
37.3520167386609	-0.01\\
37.4020194384449	-0.01\\
37.4520221382289	-0.01\\
37.502024838013	-0.01\\
37.552027537797	-0.01\\
37.602030237581	-0.01\\
37.652032937365	-0.01\\
37.702035637149	-0.01\\
37.752038336933	-0.01\\
37.8020410367171	-0.01\\
37.8520437365011	-0.01\\
37.9020464362851	-0.01\\
37.9520491360691	-0.01\\
38.0020518358531	-0.01\\
38.0520545356372	-0.01\\
38.1020572354212	-0.01\\
38.1520599352052	-0.01\\
38.2020626349892	-0.01\\
38.2520653347732	-0.01\\
38.3020680345572	-0.01\\
38.3520707343413	-0.01\\
38.4020734341253	-0.01\\
38.4520761339093	-0.01\\
38.5020788336933	-0.01\\
38.5520815334773	-0.01\\
38.6020842332613	-0.01\\
38.6520869330454	-0.01\\
38.7020896328294	-0.01\\
38.7520923326134	-0.01\\
38.8020950323974	-0.01\\
38.8520977321814	-0.01\\
38.9021004319654	-0.01\\
38.9521031317495	-0.01\\
39.0021058315335	-0.01\\
39.0521085313175	-0.01\\
39.1021112311015	-0.01\\
39.1521139308855	-0.01\\
39.2021166306696	-0.01\\
39.2521193304536	-0.01\\
39.3021220302376	-0.01\\
39.3521247300216	-0.01\\
39.4021274298056	-0.01\\
39.4521301295896	-0.01\\
39.5021328293737	-0.01\\
39.5521355291577	-0.01\\
39.6021382289417	-0.01\\
39.6521409287257	-0.01\\
39.7021436285097	-0.01\\
39.7521463282937	-0.01\\
39.8021490280778	-0.01\\
39.8521517278618	-0.01\\
39.9021544276458	-0.01\\
39.9521571274298	-0.01\\
40.0021598272138	-0.01\\
40.0521625269978	-0.01\\
40.1021652267819	-0.01\\
40.1521679265659	-0.01\\
40.2021706263499	-0.01\\
40.2521733261339	-0.01\\
40.3021760259179	-0.01\\
40.3521787257019	-0.01\\
40.402181425486	-0.01\\
40.45218412527	-0.01\\
40.502186825054	-0.01\\
40.552189524838	-0.01\\
40.602192224622	-0.01\\
40.652194924406	-0.01\\
40.7021976241901	-0.01\\
40.7522003239741	-0.01\\
40.8022030237581	-0.01\\
40.8522057235421	-0.01\\
40.9022084233261	-0.01\\
40.9522111231102	-0.01\\
41.0022138228942	-0.01\\
41.0522165226782	-0.01\\
41.1022192224622	-0.01\\
41.1522219222462	-0.01\\
41.2022246220302	-0.01\\
41.2522273218143	-0.01\\
41.3022300215983	-0.01\\
41.3522327213823	-0.01\\
41.4022354211663	-0.01\\
41.4522381209503	-0.01\\
41.5022408207343	-0.01\\
41.5522435205184	-0.01\\
41.6022462203024	-0.01\\
41.6522489200864	-0.01\\
41.7022516198704	-0.01\\
41.7522543196544	-0.01\\
41.8022570194385	-0.01\\
41.8522597192225	-0.01\\
41.9022624190065	-0.01\\
41.9522651187905	-0.01\\
42.0022678185745	-0.01\\
42.0522705183585	-0.01\\
42.1022732181426	-0.01\\
42.1522759179266	-0.01\\
42.2022786177106	-0.01\\
42.2522813174946	-0.01\\
42.3022840172786	-0.01\\
42.3522867170626	-0.01\\
42.4022894168467	-0.01\\
42.4522921166307	-0.01\\
42.5022948164147	-0.01\\
42.5522975161987	-0.01\\
42.6023002159827	-0.01\\
42.6523029157667	-0.01\\
42.7023056155508	-0.01\\
42.7523083153348	-0.01\\
42.8023110151188	-0.01\\
42.8523137149028	-0.01\\
42.9023164146868	-0.01\\
42.9523191144708	-0.01\\
43.0023218142549	-0.01\\
43.0523245140389	-0.01\\
43.1023272138229	-0.01\\
43.1523299136069	-0.01\\
43.2023326133909	-0.01\\
43.2523353131749	-0.01\\
43.302338012959	-0.01\\
43.352340712743	-0.01\\
43.402343412527	-0.01\\
43.452346112311	-0.01\\
43.502348812095	-0.01\\
43.5523515118791	-0.01\\
43.6023542116631	-0.01\\
43.6523569114471	-0.01\\
43.7023596112311	-0.01\\
43.7523623110151	-0.01\\
43.8023650107991	-0.01\\
43.8523677105832	-0.01\\
43.9023704103672	-0.01\\
43.9523731101512	-0.01\\
44.0023758099352	-0.01\\
44.0523785097192	-0.01\\
44.1023812095032	-0.01\\
44.1523839092873	-0.01\\
44.2023866090713	-0.01\\
44.2523893088553	-0.01\\
44.3023920086393	-0.01\\
44.3523947084233	-0.01\\
44.4023974082073	-0.01\\
44.4524001079914	-0.01\\
44.5024028077754	-0.01\\
44.5524055075594	-0.01\\
44.6024082073434	-0.01\\
44.6524109071274	-0.01\\
44.7024136069114	-0.01\\
44.7524163066955	-0.01\\
44.8024190064795	-0.01\\
44.8524217062635	-0.01\\
44.9024244060475	-0.01\\
44.9524271058315	-0.01\\
45.0024298056155	-0.01\\
45.0524325053996	-0.01\\
45.1024352051836	-0.01\\
45.1524379049676	-0.01\\
45.2024406047516	-0.01\\
45.2524433045356	-0.01\\
45.3024460043197	-0.01\\
45.3524487041037	-0.01\\
45.4024514038877	-0.01\\
45.4524541036717	-0.01\\
45.5024568034557	-0.01\\
45.5524595032397	-0.01\\
45.6024622030238	-0.01\\
45.6524649028078	-0.01\\
45.7024676025918	-0.01\\
45.7524703023758	-0.01\\
45.8024730021598	-0.01\\
45.8524757019439	-0.01\\
45.9024784017279	-0.01\\
45.9524811015119	-0.01\\
46.0024838012959	-0.01\\
46.0524865010799	-0.01\\
46.1024892008639	-0.01\\
46.152491900648	-0.01\\
46.202494600432	-0.01\\
46.252497300216	-0.01\\
46.3025	-0.01\\
46.352502699784	-0.01\\
46.402505399568	-0.01\\
46.4525080993521	-0.01\\
46.5025107991361	-0.01\\
46.5525134989201	-0.01\\
46.6025161987041	-0.01\\
46.6525188984881	-0.01\\
46.7025215982721	-0.01\\
46.7525242980562	-0.01\\
46.8025269978402	-0.01\\
46.8525296976242	-0.01\\
46.9025323974082	-0.01\\
46.9525350971922	-0.01\\
47.0025377969762	-0.01\\
47.0525404967603	-0.01\\
47.1025431965443	-0.01\\
47.1525458963283	-0.01\\
47.2025485961123	-0.01\\
47.2525512958963	-0.01\\
47.3025539956803	-0.01\\
47.3525566954644	-0.01\\
47.4025593952484	-0.01\\
47.4525620950324	-0.01\\
47.5025647948164	-0.01\\
47.5525674946004	-0.01\\
47.6025701943845	-0.01\\
47.6525728941685	-0.01\\
47.7025755939525	-0.01\\
47.7525782937365	-0.01\\
47.8025809935205	-0.01\\
47.8525836933045	-0.01\\
47.9025863930886	-0.01\\
47.9525890928726	-0.01\\
48.0025917926566	-0.01\\
48.0525944924406	-0.01\\
48.1025971922246	-0.01\\
48.1525998920086	-0.01\\
48.2026025917927	-0.01\\
48.2526052915767	-0.01\\
48.3026079913607	-0.01\\
48.3526106911447	-0.01\\
48.4026133909287	-0.01\\
48.4526160907127	-0.01\\
48.5026187904968	-0.01\\
48.5526214902808	-0.01\\
48.6026241900648	-0.01\\
48.6526268898488	-0.01\\
48.7026295896328	-0.01\\
48.7526322894168	-0.01\\
48.8026349892009	-0.01\\
48.8526376889849	-0.01\\
48.9026403887689	-0.01\\
48.9526430885529	-0.01\\
49.0026457883369	-0.01\\
49.052648488121	-0.01\\
49.102651187905	-0.01\\
49.152653887689	-0.01\\
49.202656587473	-0.01\\
49.252659287257	-0.01\\
49.302661987041	-0.01\\
49.3526646868251	-0.01\\
49.4026673866091	-0.01\\
49.4526700863931	-0.01\\
49.5026727861771	-0.01\\
49.5526754859611	-0.01\\
49.6026781857451	-0.01\\
49.6526808855292	-0.01\\
49.7026835853132	-0.01\\
49.7526862850972	-0.01\\
49.8026889848812	-0.01\\
49.8526916846652	-0.01\\
49.9026943844492	-0.01\\
49.9526970842333	-0.01\\
50.0026997840173	-0.01\\
50.0527024838013	-0.01\\
50.1027051835853	-0.01\\
50.1527078833693	-0.01\\
50.2027105831534	-0.01\\
50.2527132829374	-0.01\\
50.3027159827214	-0.01\\
50.3527186825054	-0.01\\
50.4027213822894	-0.01\\
50.4527240820734	-0.01\\
50.5027267818575	-0.01\\
50.5527294816415	-0.01\\
50.6027321814255	-0.01\\
50.6527348812095	-0.01\\
50.7027375809935	-0.01\\
50.7527402807775	-0.01\\
50.8027429805616	-0.01\\
50.8527456803456	-0.01\\
50.9027483801296	-0.01\\
50.9527510799136	-0.01\\
51.0027537796976	-0.01\\
51.0527564794816	-0.01\\
51.1027591792657	-0.01\\
51.1527618790497	-0.01\\
51.2027645788337	-0.01\\
51.2527672786177	-0.01\\
51.3027699784017	-0.01\\
51.3527726781858	-0.01\\
51.4027753779698	-0.01\\
51.4527780777538	-0.01\\
51.5027807775378	-0.01\\
51.5527834773218	-0.01\\
51.6027861771058	-0.01\\
51.6527888768899	-0.01\\
51.7027915766739	-0.01\\
51.7527942764579	-0.01\\
51.8027969762419	-0.01\\
51.8527996760259	-0.01\\
51.9028023758099	-0.01\\
51.952805075594	-0.01\\
52.002807775378	-0.01\\
52.052810475162	-0.01\\
52.102813174946	-0.01\\
52.15281587473	-0.01\\
52.202818574514	-0.01\\
52.2528212742981	-0.01\\
52.3028239740821	-0.01\\
52.3528266738661	-0.01\\
52.4028293736501	-0.01\\
52.4528320734341	-0.01\\
52.5028347732181	-0.01\\
52.5528374730022	-0.01\\
52.6028401727862	-0.01\\
52.6528428725702	-0.01\\
52.7028455723542	-0.01\\
52.7528482721382	-0.01\\
52.8028509719222	-0.01\\
52.8528536717063	-0.01\\
52.9028563714903	-0.01\\
52.9528590712743	-0.01\\
53.0028617710583	-0.01\\
53.0528644708423	-0.01\\
53.1028671706264	-0.01\\
53.1528698704104	-0.01\\
53.2028725701944	-0.01\\
53.2528752699784	-0.01\\
53.3028779697624	-0.01\\
53.3528806695464	-0.01\\
53.4028833693305	-0.01\\
53.4528860691145	-0.01\\
53.5028887688985	-0.01\\
53.5528914686825	-0.01\\
53.6028941684665	-0.01\\
53.6528968682505	-0.01\\
53.7028995680346	-0.01\\
53.7529022678186	-0.01\\
53.8029049676026	-0.01\\
53.8529076673866	-0.01\\
53.9029103671706	-0.01\\
53.9529130669547	-0.01\\
54.0029157667387	-0.01\\
54.0529184665227	-0.01\\
54.1029211663067	-0.01\\
54.1529238660907	-0.01\\
54.2029265658747	-0.01\\
54.2529292656588	-0.01\\
54.3029319654428	-0.01\\
54.3529346652268	-0.01\\
54.4029373650108	-0.01\\
54.4529400647948	-0.01\\
54.5029427645788	-0.01\\
54.5529454643629	-0.01\\
54.6029481641469	-0.01\\
54.6529508639309	-0.01\\
54.7029535637149	-0.01\\
54.7529562634989	-0.01\\
54.8029589632829	-0.01\\
54.852961663067	-0.01\\
54.902964362851	-0.01\\
54.952967062635	-0.01\\
55.002969762419	-0.01\\
55.052972462203	-0.01\\
55.102975161987	-0.01\\
55.1529778617711	-0.01\\
55.2029805615551	-0.01\\
55.2529832613391	-0.01\\
55.3029859611231	-0.01\\
55.3529886609071	-0.01\\
55.4029913606911	-0.01\\
55.4529940604752	-0.01\\
55.5029967602592	-0.01\\
55.5529994600432	-0.01\\
55.6030021598272	-0.01\\
55.6530048596112	-0.01\\
55.7030075593953	-0.01\\
55.7530102591793	-0.01\\
55.8030129589633	-0.01\\
55.8530156587473	-0.01\\
55.9030183585313	-0.01\\
55.9530210583153	-0.01\\
56.0030237580994	-0.01\\
56.0530264578834	-0.01\\
56.1030291576674	-0.01\\
56.1530318574514	-0.01\\
56.2030345572354	-0.01\\
56.2530372570194	-0.01\\
56.3030399568035	-0.01\\
56.3530426565875	-0.01\\
56.4030453563715	-0.01\\
56.4530480561555	-0.01\\
56.5030507559395	-0.01\\
56.5530534557235	-0.01\\
56.6030561555076	-0.01\\
56.6530588552916	-0.01\\
56.7030615550756	-0.01\\
56.7530642548596	-0.01\\
56.8030669546436	-0.01\\
56.8530696544276	-0.01\\
56.9030723542117	-0.01\\
56.9530750539957	-0.01\\
57.0030777537797	-0.01\\
57.0530804535637	-0.01\\
57.1030831533477	-0.01\\
57.1530858531318	-0.01\\
57.2030885529158	-0.01\\
57.2530912526998	-0.01\\
57.3030939524838	-0.01\\
57.3530966522678	-0.01\\
57.4030993520518	-0.01\\
57.4531020518358	-0.01\\
57.5031047516199	-0.01\\
57.5531074514039	-0.01\\
57.6031101511879	-0.01\\
57.6531128509719	-0.01\\
57.7031155507559	-0.01\\
57.75311825054	-0.01\\
57.803120950324	-0.01\\
57.853123650108	-0.01\\
57.903126349892	-0.01\\
57.953129049676	-0.01\\
58.00313174946	-0.01\\
58.0531344492441	-0.01\\
58.1031371490281	-0.01\\
58.1531398488121	-0.01\\
58.2031425485961	-0.01\\
58.2531452483801	-0.01\\
58.3031479481642	-0.01\\
58.3531506479482	-0.01\\
58.4031533477322	-0.01\\
58.4531560475162	-0.01\\
58.5031587473002	-0.01\\
58.5531614470842	-0.01\\
58.6031641468683	-0.01\\
58.6531668466523	-0.01\\
58.7031695464363	-0.01\\
58.7531722462203	-0.01\\
58.8031749460043	-0.01\\
58.8531776457883	-0.01\\
58.9031803455724	-0.01\\
58.9531830453564	-0.01\\
59.0031857451404	-0.01\\
59.0531884449244	-0.01\\
59.1031911447084	-0.01\\
59.1531938444924	-0.01\\
59.2031965442765	-0.01\\
59.2531992440605	-0.01\\
59.3032019438445	-0.01\\
59.3532046436285	-0.01\\
59.4032073434125	-0.01\\
59.4532100431965	-0.01\\
59.5032127429806	-0.01\\
59.5532154427646	-0.01\\
59.6032181425486	-0.01\\
59.6532208423326	-0.01\\
59.7032235421166	-0.01\\
59.7532262419006	-0.01\\
59.8032289416847	-0.01\\
59.8532316414687	-0.01\\
59.9032343412527	-0.01\\
59.9532370410367	-0.01\\
60.0032397408207	-0.01\\
60.0532424406048	-0.01\\
60.1032451403888	-0.01\\
60.1532478401728	-0.01\\
60.2032505399568	-0.01\\
60.2532532397408	-0.01\\
60.3032559395248	-0.01\\
60.3532586393089	-0.01\\
60.4032613390929	-0.01\\
60.4532640388769	-0.01\\
60.5032667386609	-0.01\\
60.5532694384449	-0.01\\
60.6032721382289	-0.01\\
60.653274838013	-0.01\\
60.703277537797	-0.01\\
60.753280237581	-0.01\\
60.803282937365	-0.01\\
60.853285637149	-0.01\\
60.9032883369331	-0.01\\
60.9532910367171	-0.01\\
61.0032937365011	-0.01\\
61.0532964362851	-0.01\\
61.1032991360691	-0.01\\
61.1533018358531	-0.01\\
61.2033045356371	-0.01\\
61.2533072354212	-0.01\\
61.3033099352052	-0.01\\
61.3533126349892	-0.01\\
61.4033153347732	-0.01\\
61.4533180345572	-0.01\\
61.5033207343413	-0.01\\
61.5533234341253	-0.01\\
61.6033261339093	-0.01\\
61.6533288336933	-0.01\\
61.7033315334773	-0.01\\
61.7533342332613	-0.01\\
61.8033369330454	-0.01\\
61.8533396328294	-0.01\\
61.9033423326134	-0.01\\
61.9533450323974	-0.01\\
62.0033477321814	-0.01\\
62.0533504319654	-0.01\\
62.1033531317495	-0.01\\
62.1533558315335	-0.01\\
62.2033585313175	-0.01\\
62.2533612311015	-0.01\\
62.3033639308855	-0.01\\
62.3533666306695	-0.01\\
62.4033693304536	-0.01\\
62.4533720302376	-0.01\\
62.5033747300216	-0.01\\
62.5483771598272	0.19\\
62.5983798596112	0.19\\
62.6483825593953	0.19\\
62.6983852591793	0.19\\
62.7483879589633	0.19\\
62.7983906587473	0.19\\
62.8483933585313	0.19\\
62.8983960583153	0.19\\
62.9483987580994	0.19\\
62.9984014578834	0.19\\
63.0484041576674	0.19\\
63.0984068574514	0.19\\
63.1484095572354	0.19\\
63.1984122570194	0.19\\
63.2484149568035	0.19\\
63.2984176565875	0.19\\
63.3484203563715	0.19\\
63.3984230561555	0.19\\
63.4484257559395	0.19\\
63.4984284557235	0.19\\
63.5484311555076	0.19\\
63.5984338552916	0.19\\
63.6484365550756	0.19\\
63.6984392548596	0.19\\
63.7484419546436	0.19\\
63.7984446544276	0.19\\
63.8484473542117	0.19\\
63.8984500539957	0.19\\
63.9484527537797	0.19\\
63.9984554535637	0.19\\
64.0484581533477	0.19\\
64.0984608531318	0.19\\
64.1484635529158	0.19\\
64.1984662526998	0.19\\
64.2484689524838	0.19\\
64.2984716522678	0.19\\
64.3484743520518	0.19\\
64.3984770518359	0.19\\
64.4484797516199	0.19\\
64.4984824514039	0.19\\
64.5484851511879	0.19\\
64.5984878509719	0.19\\
64.648490550756	0.19\\
64.69849325054	0.19\\
64.748495950324	0.19\\
64.798498650108	0.19\\
64.848501349892	0.19\\
64.898504049676	0.19\\
64.94850674946	0.19\\
64.9985094492441	0.19\\
65.0485121490281	0.19\\
65.0985148488121	0.19\\
65.1485175485961	0.19\\
65.1985202483801	0.19\\
65.2485229481642	0.19\\
65.2985256479482	0.19\\
65.3485283477322	0.19\\
65.3985310475162	0.19\\
65.4485337473002	0.19\\
65.4985364470842	0.19\\
65.5485391468683	0.19\\
65.5985418466523	0.19\\
65.6485445464363	0.19\\
65.6985472462203	0.19\\
65.7485499460043	0.19\\
65.7985526457883	0.19\\
65.8485553455724	0.19\\
65.8985580453564	0.19\\
65.9485607451404	0.19\\
65.9985634449244	0.19\\
66.0485661447084	0.19\\
66.0985688444924	0.19\\
66.1485715442765	0.19\\
66.1985742440605	0.19\\
66.2485769438445	0.19\\
66.2985796436285	0.19\\
66.3485823434125	0.19\\
66.3985850431965	0.19\\
66.4485877429806	0.19\\
66.4985904427646	0.19\\
66.5485931425486	0.19\\
66.5985958423326	0.19\\
66.6485985421166	0.19\\
66.6986012419006	0.19\\
66.7486039416847	0.19\\
66.7986066414687	0.19\\
66.8486093412527	0.19\\
66.8986120410367	0.19\\
66.9486147408207	0.19\\
66.9986174406047	0.19\\
67.0486201403888	0.19\\
67.0986228401728	0.19\\
67.1486255399568	0.19\\
67.1986282397408	0.19\\
67.2486309395248	0.19\\
67.2986336393089	0.19\\
67.3486363390929	0.19\\
67.3986390388769	0.19\\
67.4486417386609	0.19\\
67.4986444384449	0.19\\
67.5486471382289	0.19\\
67.598649838013	0.19\\
67.648652537797	0.19\\
67.698655237581	0.19\\
67.748657937365	0.19\\
67.798660637149	0.19\\
67.8486633369331	0.19\\
67.8986660367171	0.19\\
67.9486687365011	0.19\\
67.9986714362851	0.19\\
68.0486741360691	0.19\\
68.0986768358531	0.19\\
68.1486795356372	0.19\\
68.1986822354212	0.19\\
68.2486849352052	0.19\\
68.2986876349892	0.19\\
68.3486903347732	0.19\\
68.3986930345572	0.19\\
68.4486957343413	0.19\\
68.4986984341253	0.19\\
68.5487011339093	0.19\\
68.5987038336933	0.19\\
68.6487065334773	0.19\\
68.6987092332613	0.19\\
68.7487119330454	0.19\\
68.7987146328294	0.19\\
68.8487173326134	0.19\\
68.8987200323974	0.19\\
68.9487227321814	0.19\\
68.9987254319654	0.19\\
69.0487281317495	0.19\\
69.0987308315335	0.19\\
69.1487335313175	0.19\\
69.1987362311015	0.19\\
69.2487389308855	0.19\\
69.2987416306696	0.19\\
69.3487443304536	0.19\\
69.3987470302376	0.19\\
69.4487497300216	0.19\\
69.4987524298056	0.19\\
69.5487551295896	0.19\\
69.5987578293737	0.19\\
69.6487605291577	0.19\\
69.6987632289417	0.19\\
69.7487659287257	0.19\\
69.7987686285097	0.19\\
69.8487713282937	0.19\\
69.8987740280778	0.19\\
69.9487767278618	0.19\\
69.9987794276458	0.19\\
70.0487821274298	0.19\\
70.0987848272138	0.19\\
70.1487875269979	0.19\\
70.1987902267819	0.19\\
70.2487929265659	0.19\\
70.2987956263499	0.19\\
70.3487983261339	0.19\\
70.3988010259179	0.19\\
70.4488037257019	0.19\\
70.498806425486	0.19\\
70.54880912527	0.19\\
70.598811825054	0.19\\
70.648814524838	0.19\\
70.698817224622	0.19\\
70.7488199244061	0.19\\
70.7988226241901	0.19\\
70.8488253239741	0.19\\
70.8988280237581	0.19\\
70.9488307235421	0.19\\
70.9988334233261	0.19\\
71.0488361231101	0.19\\
71.0988388228942	0.19\\
71.1488415226782	0.19\\
71.1988442224622	0.19\\
71.2488469222462	0.19\\
71.2988496220302	0.19\\
71.3488523218143	0.19\\
71.3988550215983	0.19\\
71.4488577213823	0.19\\
71.4988604211663	0.19\\
71.5488631209503	0.19\\
71.5988658207343	0.19\\
71.6488685205184	0.19\\
71.6988712203024	0.19\\
71.7488739200864	0.19\\
71.7988766198704	0.19\\
71.8488793196544	0.19\\
71.8988820194385	0.19\\
71.9488847192225	0.19\\
71.9988874190065	0.19\\
72.0488901187905	0.19\\
72.0988928185745	0.19\\
72.1488955183585	0.19\\
72.1988982181425	0.19\\
72.2489009179266	0.19\\
72.2989036177106	0.19\\
72.3489063174946	0.19\\
72.3989090172786	0.19\\
72.4489117170626	0.19\\
72.4989144168467	0.19\\
72.5489171166307	0.19\\
72.5989198164147	0.19\\
72.6489225161987	0.19\\
72.6989252159827	0.19\\
72.7489279157667	0.19\\
72.7989306155508	0.19\\
72.8489333153348	0.19\\
72.8989360151188	0.19\\
72.9489387149028	0.19\\
72.9989414146868	0.19\\
73.0489441144708	0.19\\
73.0989468142549	0.19\\
73.1489495140389	0.19\\
73.1989522138229	0.19\\
73.2489549136069	0.19\\
73.2989576133909	0.19\\
73.348960313175	0.19\\
73.398963012959	0.19\\
73.448965712743	0.19\\
73.498968412527	0.19\\
73.548971112311	0.19\\
73.598973812095	0.19\\
73.6489765118791	0.19\\
73.6989792116631	0.19\\
73.7489819114471	0.19\\
73.7989846112311	0.19\\
73.8489873110151	0.19\\
73.8989900107991	0.19\\
73.9489927105832	0.19\\
73.9989954103672	0.19\\
74.0489981101512	0.19\\
74.0990008099352	0.19\\
74.1490035097192	0.19\\
74.1990062095032	0.19\\
74.2490089092873	0.19\\
74.2990116090713	0.19\\
74.3490143088553	0.19\\
74.3990170086393	0.19\\
74.4490197084233	0.19\\
74.4990224082073	0.19\\
74.5490251079914	0.19\\
74.5990278077754	0.19\\
74.6490305075594	0.19\\
74.6990332073434	0.19\\
74.7490359071274	0.19\\
74.7990386069115	0.19\\
74.8490413066955	0.19\\
74.8990440064795	0.19\\
74.9490467062635	0.19\\
74.9990494060475	0.19\\
75.0490521058315	0.19\\
75.0990548056156	0.19\\
75.1490575053996	0.19\\
75.1990602051836	0.19\\
75.2490629049676	0.19\\
75.2990656047516	0.19\\
75.3490683045356	0.19\\
75.3990710043197	0.19\\
75.4490737041037	0.19\\
75.4990764038877	0.19\\
75.5490791036717	0.19\\
75.5990818034557	0.19\\
75.6490845032397	0.19\\
75.6990872030238	0.19\\
75.7490899028078	0.19\\
75.7990926025918	0.19\\
75.8490953023758	0.19\\
75.8990980021598	0.19\\
75.9491007019438	0.19\\
75.9991034017279	0.19\\
76.0491061015119	0.19\\
76.0991088012959	0.19\\
76.1491115010799	0.19\\
76.1991142008639	0.19\\
76.2491169006479	0.19\\
76.299119600432	0.19\\
76.349122300216	0.19\\
76.399125	0.19\\
76.449127699784	0.19\\
76.499130399568	0.19\\
76.5491330993521	0.19\\
76.5991357991361	0.19\\
76.6491384989201	0.19\\
76.6991411987041	0.19\\
76.7491438984881	0.19\\
76.7991465982721	0.19\\
76.8491492980562	0.19\\
76.8991519978402	0.19\\
76.9491546976242	0.19\\
76.9991573974082	0.19\\
77.0491600971922	0.19\\
77.0991627969762	0.19\\
77.1491654967603	0.19\\
77.1991681965443	0.19\\
77.2491708963283	0.19\\
77.2991735961123	0.19\\
77.3491762958963	0.19\\
77.3991789956804	0.19\\
77.4491816954644	0.19\\
77.4991843952484	0.19\\
77.5491870950324	0.19\\
77.5991897948164	0.19\\
77.6491924946004	0.19\\
77.6991951943844	0.19\\
77.7491978941685	0.19\\
77.7992005939525	0.19\\
77.8492032937365	0.19\\
77.8992059935205	0.19\\
77.9492086933045	0.19\\
77.9992113930886	0.19\\
78.0492140928726	0.19\\
78.0992167926566	0.19\\
78.1492194924406	0.19\\
78.1992221922246	0.19\\
78.2492248920086	0.19\\
78.2992275917927	0.19\\
78.3492302915767	0.19\\
78.3992329913607	0.19\\
78.4492356911447	0.19\\
78.4992383909287	0.19\\
78.5492410907127	0.19\\
78.5992437904968	0.19\\
78.6492464902808	0.19\\
78.6992491900648	0.19\\
78.7492518898488	0.19\\
78.7992545896328	0.19\\
78.8492572894169	0.19\\
78.8992599892009	0.19\\
78.9492626889849	0.19\\
78.9992653887689	0.19\\
79.0492680885529	0.19\\
79.0992707883369	0.19\\
79.149273488121	0.19\\
79.199276187905	0.19\\
79.249278887689	0.19\\
79.299281587473	0.19\\
79.349284287257	0.19\\
79.3992869870411	0.19\\
79.4492896868251	0.19\\
79.4992923866091	0.19\\
79.5492950863931	0.19\\
79.5992977861771	0.19\\
79.6493004859611	0.19\\
79.6993031857451	0.19\\
79.7493058855292	0.19\\
79.7993085853132	0.19\\
79.8493112850972	0.19\\
79.8993139848812	0.19\\
79.9493166846652	0.19\\
79.9993193844493	0.19\\
80.0493220842333	0.19\\
80.0993247840173	0.19\\
80.1493274838013	0.19\\
80.1993301835853	0.19\\
80.2493328833693	0.19\\
80.2993355831534	0.19\\
80.3493382829374	0.19\\
80.3993409827214	0.19\\
80.4493436825054	0.19\\
80.4993463822894	0.19\\
80.5493490820734	0.19\\
80.5993517818575	0.19\\
80.6493544816415	0.19\\
80.6993571814255	0.19\\
80.7493598812095	0.19\\
80.7993625809935	0.19\\
80.8493652807775	0.19\\
80.8993679805616	0.19\\
80.9493706803456	0.19\\
80.9993733801296	0.19\\
81.0493760799136	0.19\\
81.0993787796976	0.19\\
81.1493814794816	0.19\\
81.1993841792657	0.19\\
81.2493868790497	0.19\\
81.2993895788337	0.19\\
81.3493922786177	0.19\\
81.3993949784017	0.19\\
81.4493976781857	0.19\\
81.4994003779698	0.19\\
81.5494030777538	0.19\\
81.5994057775378	0.19\\
81.6494084773218	0.19\\
81.6994111771058	0.19\\
81.7494138768898	0.19\\
81.7994165766739	0.19\\
81.8494192764579	0.19\\
81.8994219762419	0.19\\
81.9494246760259	0.19\\
81.9994273758099	0.19\\
82.049430075594	0.19\\
82.099432775378	0.19\\
82.149435475162	0.19\\
82.199438174946	0.19\\
82.24944087473	0.19\\
82.299443574514	0.19\\
82.3494462742981	0.19\\
82.3994489740821	0.19\\
82.4494516738661	0.19\\
82.4994543736501	0.19\\
82.5494570734341	0.19\\
82.5994597732181	0.19\\
82.6294613930885	-0.01\\
82.6794640928726	-0.01\\
82.7294667926566	-0.01\\
82.7794694924406	-0.01\\
82.8294721922246	-0.01\\
82.8794748920086	-0.01\\
82.9294775917927	-0.01\\
82.9794802915767	-0.01\\
83.0294829913607	-0.01\\
83.0794856911447	-0.01\\
83.1294883909287	-0.01\\
83.1794910907127	-0.01\\
83.2294937904968	-0.01\\
83.2794964902808	-0.01\\
83.3294991900648	-0.01\\
83.3795018898488	-0.01\\
83.4295045896328	-0.01\\
83.4795072894169	-0.01\\
83.5295099892009	-0.01\\
83.5795126889849	-0.01\\
83.6295153887689	-0.01\\
83.6795180885529	-0.01\\
83.7295207883369	-0.01\\
83.779523488121	-0.01\\
83.829526187905	-0.01\\
83.879528887689	-0.01\\
83.929531587473	-0.01\\
83.979534287257	-0.01\\
84.029536987041	-0.01\\
84.0795396868251	-0.01\\
84.1295423866091	-0.01\\
84.1795450863931	-0.01\\
84.2295477861771	-0.01\\
84.2795504859611	-0.01\\
84.3295531857451	-0.01\\
84.3795558855292	-0.01\\
84.4295585853132	-0.01\\
84.4795612850972	-0.01\\
84.5295639848812	-0.01\\
84.5795666846652	-0.01\\
84.6295693844492	-0.01\\
84.6795720842333	-0.01\\
84.7295747840173	-0.01\\
84.7795774838013	-0.01\\
84.8295801835853	-0.01\\
84.8795828833693	-0.01\\
84.9295855831534	-0.01\\
84.9795882829374	-0.01\\
85.0295909827214	-0.01\\
85.0795936825054	-0.01\\
85.1295963822894	-0.01\\
85.1795990820734	-0.01\\
85.2296017818575	-0.01\\
85.2796044816415	-0.01\\
85.3296071814255	-0.01\\
85.3796098812095	-0.01\\
85.4296125809935	-0.01\\
85.4796152807775	-0.01\\
85.5296179805616	-0.01\\
85.5796206803456	-0.01\\
85.6296233801296	-0.01\\
85.6796260799136	-0.01\\
85.7296287796976	-0.01\\
85.7796314794817	-0.01\\
85.8296341792657	-0.01\\
85.8796368790497	-0.01\\
85.9296395788337	-0.01\\
85.9796422786177	-0.01\\
86.0296449784017	-0.01\\
86.0796476781857	-0.01\\
86.1296503779698	-0.01\\
86.1796530777538	-0.01\\
86.2296557775378	-0.01\\
86.2796584773218	-0.01\\
86.3296611771058	-0.01\\
86.3796638768899	-0.01\\
86.4296665766739	-0.01\\
86.4796692764579	-0.01\\
86.5296719762419	-0.01\\
86.5796746760259	-0.01\\
86.6296773758099	-0.01\\
86.679680075594	-0.01\\
86.729682775378	-0.01\\
86.779685475162	-0.01\\
86.829688174946	-0.01\\
86.87969087473	-0.01\\
86.929693574514	-0.01\\
86.9796962742981	-0.01\\
87.0296989740821	-0.01\\
87.0797016738661	-0.01\\
87.1297043736501	-0.01\\
87.1797070734341	-0.01\\
87.2297097732181	-0.01\\
87.2797124730022	-0.01\\
87.3297151727862	-0.01\\
87.3797178725702	-0.01\\
87.4297205723542	-0.01\\
87.4797232721382	-0.01\\
87.5297259719222	-0.01\\
87.5797286717063	-0.01\\
87.6297313714903	-0.01\\
87.6797340712743	-0.01\\
87.7297367710583	-0.01\\
87.7797394708423	-0.01\\
87.8297421706264	-0.01\\
87.8797448704104	-0.01\\
87.9297475701944	-0.01\\
87.9797502699784	-0.01\\
88.0297529697624	-0.01\\
88.0797556695464	-0.01\\
88.1297583693305	-0.01\\
88.1797610691145	-0.01\\
88.2297637688985	-0.01\\
88.2797664686825	-0.01\\
88.3297691684665	-0.01\\
88.3797718682505	-0.01\\
88.4297745680346	-0.01\\
88.4797772678186	-0.01\\
88.5297799676026	-0.01\\
88.5797826673866	-0.01\\
88.6297853671706	-0.01\\
88.6797880669546	-0.01\\
88.7297907667387	-0.01\\
88.7797934665227	-0.01\\
88.8297961663067	-0.01\\
88.8797988660907	-0.01\\
88.9298015658747	-0.01\\
88.9798042656588	-0.01\\
89.0298069654428	-0.01\\
89.0798096652268	-0.01\\
89.1298123650108	-0.01\\
89.1798150647948	-0.01\\
89.2298177645788	-0.01\\
89.2798204643629	-0.01\\
89.3298231641469	-0.01\\
89.3798258639309	-0.01\\
89.4298285637149	-0.01\\
89.4798312634989	-0.01\\
89.5298339632829	-0.01\\
89.579836663067	-0.01\\
89.629839362851	-0.01\\
89.679842062635	-0.01\\
89.729844762419	-0.01\\
89.779847462203	-0.01\\
89.8298501619871	-0.01\\
89.8798528617711	-0.01\\
89.9298555615551	-0.01\\
89.9798582613391	-0.01\\
90.0298609611231	-0.01\\
90.0798636609071	-0.01\\
90.1298663606911	-0.01\\
90.1798690604752	-0.01\\
90.2298717602592	-0.01\\
90.2798744600432	-0.01\\
90.3298771598272	-0.01\\
90.3798798596112	-0.01\\
90.4298825593953	-0.01\\
90.4798852591793	-0.01\\
90.5298879589633	-0.01\\
90.5798906587473	-0.01\\
90.6298933585313	-0.01\\
90.6798960583153	-0.01\\
90.7298987580994	-0.01\\
90.7799014578834	-0.01\\
90.8299041576674	-0.01\\
90.8799068574514	-0.01\\
90.9299095572354	-0.01\\
90.9799122570195	-0.01\\
91.0299149568035	-0.01\\
91.0799176565875	-0.01\\
91.1299203563715	-0.01\\
91.1799230561555	-0.01\\
91.2299257559395	-0.01\\
91.2799284557236	-0.01\\
91.3299311555076	-0.01\\
91.3799338552916	-0.01\\
91.4299365550756	-0.01\\
91.4799392548596	-0.01\\
91.5299419546436	-0.01\\
91.5799446544276	-0.01\\
91.6299473542117	-0.01\\
91.6799500539957	-0.01\\
91.7299527537797	-0.01\\
91.7799554535637	-0.01\\
91.8299581533477	-0.01\\
91.8799608531317	-0.01\\
91.9299635529158	-0.01\\
91.9799662526998	-0.01\\
92.0299689524838	-0.01\\
92.0799716522678	-0.01\\
92.1299743520518	-0.01\\
92.1799770518358	-0.01\\
92.2299797516199	-0.01\\
92.2799824514039	-0.01\\
92.3299851511879	-0.01\\
92.3799878509719	-0.01\\
92.4299905507559	-0.01\\
92.47999325054	-0.01\\
92.529995950324	-0.01\\
92.579998650108	-0.01\\
92.605	-0.01\\
};
\addlegendentry{- 1cm};

\addplot [color=red,solid,line width=0.2pt]
  table[row sep=crcr]{0	-0.075003\\
0.0500026997840173	-0.074983\\
0.100005399568035	-0.074955\\
0.150008099352052	-0.073954\\
0.200010799136069	-0.075066\\
0.250013498920086	-0.074146\\
0.300016198704104	-0.073995\\
0.350018898488121	-0.073456\\
0.400021598272138	-0.073304\\
0.450024298056156	-0.073288\\
0.500026997840173	-0.073859\\
0.55002969762419	-0.073271\\
0.600032397408207	-0.075337\\
0.650035097192225	-0.073829\\
0.700037796976242	-0.073349\\
0.750040496760259	-0.073073\\
0.800043196544276	-0.073011\\
0.850045896328294	-0.074833\\
0.900048596112311	-0.073879\\
0.950051295896328	-0.074291\\
1.00005399568035	-0.073886\\
1.05005669546436	-0.073226\\
1.10005939524838	-0.07297\\
1.1500620950324	-0.07345\\
1.20006479481641	-0.073253\\
1.25006749460043	-0.073608\\
1.30007019438445	-0.073357\\
1.35007289416847	-0.072791\\
1.40007559395248	-0.072473\\
1.4500782937365	-0.072299\\
1.50008099352052	-0.071944\\
1.55008369330454	-0.071813\\
1.60008639308855	-0.072657\\
1.65008909287257	-0.072566\\
1.70009179265659	-0.072121\\
1.7500944924406	-0.071908\\
1.80009719222462	-0.071798\\
1.85009989200864	-0.072185\\
1.90010259179266	-0.072439\\
1.95010529157667	-0.072051\\
2.00010799136069	-0.072449\\
2.05011069114471	-0.070773\\
2.10011339092873	-0.066297\\
2.15011609071274	-0.060216\\
2.20011879049676	-0.049576\\
2.25012149028078	-0.038458\\
2.30012419006479	-0.030138\\
2.35012688984881	-0.02659\\
2.40012958963283	-0.024174\\
2.45013228941685	-0.021831\\
2.50013498920086	-0.019046\\
2.55013768898488	-0.016436\\
2.6001403887689	-0.013247\\
2.65014308855292	-0.012173\\
2.70014578833693	-0.011279\\
2.75014848812095	-0.010226\\
2.80015118790497	-0.009076\\
2.85015388768899	-0.007366\\
2.900156587473	-0.007382\\
2.95015928725702	-0.006207\\
3.00016198704104	-0.006207\\
3.05016468682505	-0.00545\\
3.10016738660907	-0.003491\\
3.15017008639309	-0.003284\\
3.20017278617711	-0.003722\\
3.25017548596112	-0.004076\\
3.30017818574514	-0.004393\\
3.35018088552916	-0.003226\\
3.40018358531318	-0.002523\\
3.45018628509719	-0.003838\\
3.50018898488121	-0.004748\\
3.55019168466523	-0.005132\\
3.60019438444924	-0.004096\\
3.65019708423326	-0.004085\\
3.70019978401728	-0.00422\\
3.7502024838013	-0.00456\\
3.80020518358531	-0.004908\\
3.85020788336933	-0.0044\\
3.90021058315335	-0.003887\\
3.95021328293736	-0.002921\\
4.00021598272138	-0.001625\\
4.0502186825054	-0.001229\\
4.10022138228942	-0.002481\\
4.15022408207343	-0.003454\\
4.20022678185745	-0.003244\\
4.25022948164147	-0.003109\\
4.30023218142549	-0.00209\\
4.3502348812095	-0.002396\\
4.40023758099352	-0.002751\\
4.45024028077754	-0.001355\\
4.50024298056155	-0.001316\\
4.55024568034557	-0.001413\\
4.60024838012959	-0.001619\\
4.65025107991361	-0.001125\\
4.70025377969762	-0.001186\\
4.75025647948164	-0.000338\\
4.80025917926566	-0.000139\\
4.85026187904968	-0.000115\\
4.90026457883369	-0.00022\\
4.95026727861771	7.6e-05\\
5.00026997840173	-0.000148\\
5.05027267818575	-0.000546\\
5.10027537796976	-0.000892\\
5.15027807775378	-0.001127\\
5.2002807775378	-0.000283\\
5.25028347732181	0.000874\\
5.30028617710583	0.000295\\
5.35028887688985	8.6e-05\\
5.40029157667387	0.000259\\
5.45029427645788	-0.000391\\
5.5002969762419	-0.001348\\
5.55029967602592	-0.000577\\
5.60030237580994	-0.000573\\
5.65030507559395	-0.000616\\
5.70030777537797	-0.000678\\
5.75031047516199	-0.000872\\
5.800313174946	-0.00113\\
5.85031587473002	-0.000484\\
5.90031857451404	-0.000644\\
5.95032127429806	-0.000821\\
6.00032397408207	-0.00062\\
6.05032667386609	-0.00029\\
6.10032937365011	0.000829\\
6.15033207343413	0.000662\\
6.20033477321814	0.000289\\
6.25033747300216	0.001395\\
6.30034017278618	0.00207\\
6.3503428725702	0.001229\\
6.40034557235421	0.000862\\
6.45034827213823	0.00131\\
6.50035097192225	0.000227\\
6.55035367170626	0.001733\\
6.60035637149028	0.001414\\
6.6503590712743	-0.000218\\
6.70036177105832	-0.001463\\
6.75036447084233	-0.001581\\
6.80036717062635	-0.000933\\
6.85036987041037	0.001031\\
6.90037257019439	-0.000786\\
6.9503752699784	-0.002603\\
7.00037796976242	-0.003766\\
7.05038066954644	-0.004201\\
7.10038336933045	-0.003364\\
7.15038606911447	-0.003124\\
7.20038876889849	-0.003674\\
7.25039146868251	-0.002977\\
7.30039416846652	-0.003123\\
7.35039686825054	-0.003868\\
7.40039956803456	-0.003522\\
7.45040226781857	-0.003455\\
7.50040496760259	-0.002184\\
7.55040766738661	-0.00125\\
7.60041036717063	0.000722\\
7.65041306695464	-0.000264\\
7.70041576673866	-0.000339\\
7.75041846652268	-0.001041\\
7.8004211663067	-0.000131\\
7.85042386609071	0.00348\\
7.90042656587473	0.002925\\
7.95042926565875	0.001898\\
8.00043196544276	0.001288\\
8.05043466522678	0.000726\\
8.1004373650108	0.001815\\
8.15044006479482	0.001858\\
8.20044276457883	0.001495\\
8.25044546436285	0.000999\\
8.30044816414687	-0.000346\\
8.35045086393089	-0.003011\\
8.4004535637149	-0.004537\\
8.45045626349892	-0.004557\\
8.50045896328294	-0.002099\\
8.55046166306696	-0.002409\\
8.60046436285097	-0.003178\\
8.65046706263499	-0.004498\\
8.70046976241901	-0.005089\\
8.75047246220302	-0.004953\\
8.80047516198704	-0.004918\\
8.85047786177106	-0.0049\\
8.90048056155508	-0.00396\\
8.95048326133909	-0.00345\\
9.00048596112311	-0.001789\\
9.05048866090713	-0.001724\\
9.10049136069114	-0.001457\\
9.15049406047516	-0.000356\\
9.20049676025918	-0.000451\\
9.2504994600432	-0.000413\\
9.30050215982721	-0.000287\\
9.35050485961123	0.000213\\
9.40050755939525	0.000973\\
9.45051025917927	0.000947\\
9.50051295896328	0.002084\\
9.5505156587473	0.001466\\
9.60051835853132	0.001353\\
9.65052105831533	0.000178\\
9.70052375809935	-0.000551\\
9.75052645788337	-0.001248\\
9.80052915766739	-0.00129\\
9.8505318574514	-0.001355\\
9.90053455723542	-0.00132\\
9.95053725701944	-0.000797\\
10.0005399568035	-0.000423\\
10.0505426565875	-0.000156\\
10.1005453563715	0.000418\\
10.1505480561555	0.00023\\
10.2005507559395	0.000394\\
10.2505534557235	0.001086\\
10.3005561555076	0.000637\\
10.3505588552916	0.000321\\
10.4005615550756	-4.1e-05\\
10.4505642548596	-0.000723\\
10.5005669546436	-0.000904\\
10.5505696544276	-0.000589\\
10.6005723542117	-0.00056\\
10.6505750539957	-0.000241\\
10.7005777537797	-0.000559\\
10.7505804535637	0.00028\\
10.8005831533477	0.001932\\
10.8505858531317	0.001772\\
10.9005885529158	0.000876\\
10.9505912526998	0.001343\\
11.0005939524838	0.001237\\
11.0505966522678	0.000811\\
11.1005993520518	0.001141\\
11.1506020518359	0.000993\\
11.2006047516199	0.001048\\
11.2506074514039	0.001857\\
11.3006101511879	0.001288\\
11.3506128509719	0.000592\\
11.4006155507559	0.00039\\
11.45061825054	0.000264\\
11.500620950324	-4e-05\\
11.550623650108	0.000601\\
11.600626349892	0.000572\\
11.650629049676	0.000258\\
11.70063174946	0.000704\\
11.7506344492441	0.000452\\
11.8006371490281	-0.000178\\
11.8506398488121	-0.000898\\
11.9006425485961	-0.001077\\
11.9506452483801	-0.000951\\
12.0006479481641	-0.000318\\
12.0506506479482	-5.8e-05\\
12.1006533477322	-9.6e-05\\
12.1506560475162	-3e-06\\
12.2006587473002	0.000441\\
12.2506614470842	0.001098\\
12.3006641468683	-2.6e-05\\
12.3506668466523	-0.000746\\
12.4006695464363	-0.000981\\
12.4506722462203	-0.000914\\
12.5006749460043	-0.000464\\
12.5506776457883	-0.000378\\
12.6006803455724	-0.000711\\
12.6506830453564	-0.000839\\
12.7006857451404	0.00066\\
12.7506884449244	0.000303\\
12.8006911447084	0.000347\\
12.8506938444924	0.000356\\
12.9006965442765	-0.001082\\
12.9506992440605	-0.001425\\
13.0007019438445	-0.001715\\
13.0507046436285	-0.001476\\
13.1007073434125	-0.001141\\
13.1507100431965	-0.001183\\
13.2007127429806	-0.001379\\
13.2507154427646	-0.000768\\
13.3007181425486	-0.001726\\
13.3507208423326	-0.002396\\
13.4007235421166	-0.00239\\
13.4507262419006	-0.001559\\
13.5007289416847	-0.002046\\
13.5507316414687	-0.001881\\
13.6007343412527	-0.001983\\
13.6507370410367	-0.001245\\
13.7007397408207	-0.001056\\
13.7507424406048	-0.001496\\
13.8007451403888	-0.001744\\
13.8507478401728	-0.002279\\
13.9007505399568	-0.00252\\
13.9507532397408	-0.002517\\
14.0007559395248	-0.001835\\
14.0507586393089	-0.000818\\
14.1007613390929	-0.000685\\
14.1507640388769	-0.000948\\
14.2007667386609	-0.001333\\
14.2507694384449	-0.001336\\
14.3007721382289	-0.000471\\
14.350774838013	0.000806\\
14.400777537797	0.001366\\
14.450780237581	0.001269\\
14.500782937365	0.000187\\
14.550785637149	0.000404\\
14.600788336933	-0.000773\\
14.6507910367171	-0.001171\\
14.7007937365011	0.00059\\
14.7507964362851	-2.8e-05\\
14.8007991360691	-0.000833\\
14.8508018358531	-0.001261\\
14.9008045356372	-0.002331\\
14.9508072354212	-0.003037\\
15.0008099352052	-0.003226\\
15.0508126349892	-0.002441\\
15.1008153347732	-0.001735\\
15.1508180345572	-0.002867\\
15.2008207343413	-0.00366\\
15.2508234341253	-0.003202\\
15.3008261339093	-0.002875\\
15.3508288336933	-0.002258\\
15.4008315334773	-0.000562\\
15.4508342332613	-0.000544\\
15.5008369330454	-0.000525\\
15.5508396328294	-0.0013\\
15.6008423326134	-0.002581\\
15.6508450323974	-0.002729\\
15.7008477321814	-0.002326\\
15.7508504319654	-0.002416\\
15.8008531317495	-0.000636\\
15.8508558315335	0.000348\\
15.9008585313175	-0.000927\\
15.9508612311015	-0.000598\\
16.0008639308855	-0.000968\\
16.0508666306695	-0.000954\\
16.1008693304536	-0.001695\\
16.1508720302376	-0.00107\\
16.2008747300216	-0.00211\\
16.2508774298056	-0.00295\\
16.3008801295896	-0.003853\\
16.3508828293737	-0.003772\\
16.4008855291577	-0.003106\\
16.4508882289417	-0.002199\\
16.5008909287257	-0.000958\\
16.5508936285097	-1.2e-05\\
16.6008963282937	0.000271\\
16.6508990280778	-0.000899\\
16.7009017278618	-0.001266\\
16.7509044276458	-7.7e-05\\
16.8009071274298	-0.000147\\
16.8509098272138	-0.001036\\
16.9009125269978	-0.000567\\
16.9509152267819	-0.001218\\
17.0009179265659	-0.001096\\
17.0509206263499	-0.000867\\
17.1009233261339	-0.001656\\
17.1509260259179	-0.002278\\
17.2009287257019	-0.002688\\
17.250931425486	-0.002569\\
17.30093412527	-0.002739\\
17.350936825054	-0.001736\\
17.400939524838	-0.001091\\
17.450942224622	-0.000953\\
17.500944924406	-0.000666\\
17.5509476241901	-0.000664\\
17.6009503239741	-0.0001\\
17.6509530237581	0.000654\\
17.7009557235421	0.00044\\
17.7509584233261	0.000312\\
17.8009611231102	0.000842\\
17.8509638228942	0.001009\\
17.9009665226782	0.000912\\
17.9509692224622	0.001315\\
18.0009719222462	0.001191\\
18.0509746220302	0.000482\\
18.1009773218143	-0.000669\\
18.1509800215983	-0.000821\\
18.2009827213823	0.00016\\
18.2509854211663	0.00052\\
18.3009881209503	0.000571\\
18.3509908207343	0.000527\\
18.4009935205184	0.000782\\
18.4509962203024	0.00124\\
18.5009989200864	0.001153\\
18.5510016198704	0.002918\\
18.6010043196544	0.003492\\
18.6510070194384	0.001586\\
18.7010097192225	0.001251\\
18.7510124190065	0.000436\\
18.8010151187905	-4.6e-05\\
18.8510178185745	-0.000634\\
18.9010205183585	-0.000461\\
18.9510232181425	-0.001073\\
19.0010259179266	-0.000593\\
19.0510286177106	-0.000474\\
19.1010313174946	-0.00037\\
19.1510340172786	-0.000341\\
19.2010367170626	-0.000302\\
19.2510394168467	-0.000211\\
19.3010421166307	-0.000801\\
19.3510448164147	-0.001197\\
19.4010475161987	-0.001381\\
19.4510502159827	-0.001101\\
19.5010529157667	-0.001062\\
19.5510556155508	-0.000857\\
19.6010583153348	-0.001064\\
19.6510610151188	-0.001314\\
19.7010637149028	-0.000624\\
19.7510664146868	-0.000732\\
19.8010691144708	-0.000762\\
19.8510718142549	-0.000849\\
19.9010745140389	-0.000421\\
19.9510772138229	-0.000378\\
20.0010799136069	-0.000334\\
20.0510826133909	-0.000315\\
20.1010853131749	6.7e-05\\
20.151088012959	-9e-05\\
20.201090712743	-0.000298\\
20.251093412527	-0.000804\\
20.301096112311	-0.00085\\
20.351098812095	-0.000548\\
20.4011015118791	-0.000628\\
20.4511042116631	-0.000562\\
20.5011069114471	-0.000286\\
20.5511096112311	-0.000442\\
20.6011123110151	-0.00102\\
20.6511150107991	-0.000222\\
20.7011177105832	0.000443\\
20.7511204103672	0.000409\\
20.8011231101512	0.001064\\
20.8511258099352	0.001397\\
20.9011285097192	0.003367\\
20.9511312095032	0.004354\\
21.0011339092873	0.002344\\
21.0511366090713	0.000658\\
21.1011393088553	0.000298\\
21.1511420086393	0.000475\\
21.2011447084233	0.000384\\
21.2511474082073	0.000978\\
21.3011501079914	0.00066\\
21.3511528077754	0.000426\\
21.4011555075594	0.000426\\
21.4511582073434	0.00043\\
21.5011609071274	0.000806\\
21.5511636069114	0.00172\\
21.6011663066955	0.001087\\
21.6511690064795	0.00035\\
21.7011717062635	0.003488\\
21.7511744060475	0.003507\\
21.8011771058315	0.00285\\
21.8511798056156	0.002122\\
21.9011825053996	0.001713\\
21.9511852051836	0.001712\\
22.0011879049676	0.001462\\
22.0511906047516	0.001087\\
22.1011933045356	0.000533\\
22.1511960043197	8.3e-05\\
22.2011987041037	-0.000132\\
22.2512014038877	-0.001331\\
22.3012041036717	-0.004184\\
22.3512068034557	-0.005679\\
22.4012095032397	-0.005819\\
22.4512122030238	-0.004488\\
22.5012149028078	-0.003657\\
22.5512176025918	-0.003425\\
22.6012203023758	-0.002426\\
22.6512230021598	-0.002608\\
22.7012257019438	-0.003989\\
22.7512284017279	-0.00293\\
22.8012311015119	-0.001825\\
22.8512338012959	-0.001131\\
22.9012365010799	-0.000975\\
22.9512392008639	-0.000849\\
23.0012419006479	-0.000703\\
23.051244600432	-0.001702\\
23.101247300216	-0.002215\\
23.15125	-0.001022\\
23.201252699784	0.001431\\
23.251255399568	-0.000626\\
23.3012580993521	-0.001549\\
23.3512607991361	-0.000209\\
23.4012634989201	-0.000746\\
23.4512661987041	0.000256\\
23.5012688984881	0.001305\\
23.5512715982721	0.000309\\
23.6012742980562	9.7e-05\\
23.6512769978402	0.001325\\
23.7012796976242	-0.000449\\
23.7512823974082	-0.000801\\
23.8012850971922	-0.001959\\
23.8512877969762	-0.002036\\
23.9012904967603	-0.002507\\
23.9512931965443	-0.00281\\
24.0012958963283	-0.0019\\
24.0512985961123	0.000461\\
24.1013012958963	0.000797\\
24.1513039956803	-9.7e-05\\
24.2013066954644	-0.000129\\
24.2513093952484	0.000708\\
24.3013120950324	0.000809\\
24.3513147948164	0.000495\\
24.4013174946004	0.000436\\
24.4513201943845	0.000767\\
24.5013228941685	0.000331\\
24.5513255939525	0.000577\\
24.6013282937365	-0.00065\\
24.6513309935205	-0.001352\\
24.7013336933045	-0.00186\\
24.7513363930886	-0.000869\\
24.8013390928726	-0.000372\\
24.8513417926566	-0.000186\\
24.9013444924406	0.00017\\
24.9513471922246	-0.000184\\
25.0013498920086	-0.000705\\
25.0513525917927	-0.000624\\
25.1013552915767	-0.000371\\
25.1513579913607	0.000798\\
25.2013606911447	0.000267\\
25.2513633909287	0.002109\\
25.3013660907127	0.001984\\
25.3513687904968	0.001681\\
25.4013714902808	0.001423\\
25.4513741900648	0.000817\\
25.5013768898488	0.000909\\
25.5513795896328	0.000249\\
25.6013822894168	-0.0002\\
25.6513849892009	0.001483\\
25.7013876889849	0.001361\\
25.7513903887689	0.000378\\
25.8013930885529	0.000234\\
25.8513957883369	0.001121\\
25.901398488121	0.002784\\
25.951401187905	0.00306\\
26.001403887689	0.00249\\
26.051406587473	0.00134\\
26.101409287257	0.00015\\
26.151411987041	-0.00106\\
26.2014146868251	0.00017\\
26.2514173866091	0.000122\\
26.3014200863931	0.000275\\
26.3514227861771	0.001988\\
26.4014254859611	0.002515\\
26.4514281857451	0.000666\\
26.5014308855292	-0.000814\\
26.5514335853132	-0.000797\\
26.6014362850972	-0.000399\\
26.6514389848812	0.000624\\
26.7014416846652	-0.000467\\
26.7514443844492	-0.000968\\
26.8014470842333	-0.001945\\
26.8514497840173	-0.001372\\
26.9014524838013	-0.00144\\
26.9514551835853	-0.00101\\
27.0014578833693	0.000302\\
27.0514605831534	-0.00084\\
27.1014632829374	-0.001684\\
27.1514659827214	-0.001495\\
27.2014686825054	-0.000821\\
27.2514713822894	2.1e-05\\
27.3014740820734	9.8e-05\\
27.3514767818575	-5.5e-05\\
27.4014794816415	-0.000255\\
27.4514821814255	0.000702\\
27.5014848812095	0.002969\\
27.5514875809935	0.001437\\
27.6014902807775	-0.000664\\
27.6514929805616	-0.000659\\
27.7014956803456	0.00022\\
27.7514983801296	0.000681\\
27.8015010799136	0.001145\\
27.8515037796976	0.001482\\
27.9015064794816	0.002174\\
27.9515091792657	0.002821\\
28.0015118790497	0.002352\\
28.0515145788337	0.003329\\
28.1015172786177	0.003065\\
28.1515199784017	0.001469\\
28.2015226781857	0.001404\\
28.2515253779698	0.002168\\
28.3015280777538	0.002557\\
28.3515307775378	0.002174\\
28.4015334773218	0.002307\\
28.4515361771058	0.001834\\
28.5015388768899	0.001462\\
28.5515415766739	0.001309\\
28.6015442764579	0.000767\\
28.6515469762419	0.000287\\
28.7015496760259	-7.1e-05\\
28.7515523758099	0.000532\\
28.801555075594	0.000476\\
28.851557775378	0.000184\\
28.901560475162	0.00087\\
28.951563174946	0.000492\\
29.00156587473	0.000195\\
29.051568574514	4.2e-05\\
29.1015712742981	0.000132\\
29.1515739740821	0.000536\\
29.2015766738661	-0.001627\\
29.2515793736501	-0.002352\\
29.3015820734341	6.8e-05\\
29.3515847732181	0.002701\\
29.4015874730022	0.001543\\
29.4515901727862	-0.000644\\
29.5015928725702	-0.001642\\
29.5515955723542	-0.002692\\
29.6015982721382	-0.003291\\
29.6516009719222	-0.004052\\
29.7016036717063	-0.003789\\
29.7516063714903	-0.004365\\
29.8016090712743	-0.003849\\
29.8516117710583	-0.002884\\
29.9016144708423	-0.000968\\
29.9516171706264	-0.000265\\
30.0016198704104	-0.00035\\
30.0516225701944	-7.9e-05\\
30.1016252699784	5.9e-05\\
30.1516279697624	8e-05\\
30.2016306695464	-0.000124\\
30.2516333693305	-0.001495\\
30.3016360691145	-0.001724\\
30.3516387688985	-0.001759\\
30.4016414686825	-0.002406\\
30.4516441684665	-0.000678\\
30.5016468682505	0.000659\\
30.5516495680346	0.000427\\
30.6016522678186	0.000386\\
30.6516549676026	-0.000313\\
30.7016576673866	-6.7e-05\\
30.7516603671706	6.2e-05\\
30.8016630669546	0.000122\\
30.8516657667387	8e-06\\
30.9016684665227	-0.002042\\
30.9516711663067	-0.000988\\
31.0016738660907	0.000239\\
31.0516765658747	0.001477\\
31.1016792656587	0.002072\\
31.1516819654428	0.002352\\
31.2016846652268	0.001227\\
31.2516873650108	0.000156\\
31.3016900647948	-0.001407\\
31.3516927645788	-0.002247\\
31.4016954643629	-0.001753\\
31.4516981641469	-0.001927\\
31.5017008639309	-0.000482\\
31.5517035637149	-0.001014\\
31.6017062634989	-0.002057\\
31.6517089632829	-0.002118\\
31.701711663067	-0.001594\\
31.751714362851	-0.001417\\
31.801717062635	-0.0008\\
31.851719762419	-7.4e-05\\
31.901722462203	0.00142\\
31.951725161987	0.001301\\
32.0017278617711	0.001576\\
32.0517305615551	0.003084\\
32.1017332613391	0.003937\\
32.1517359611231	0.003751\\
32.2017386609071	0.004106\\
32.2517413606911	0.003356\\
32.3017440604752	0.003239\\
32.3517467602592	0.003579\\
32.4017494600432	0.004336\\
32.4517521598272	0.00576\\
32.5017548596112	0.004938\\
32.5517575593952	0.003938\\
32.6017602591793	0.003459\\
32.6517629589633	0.004388\\
32.7017656587473	0.002456\\
32.7517683585313	0.001991\\
32.8017710583153	0.000901\\
32.8517737580993	-0.000228\\
32.9017764578834	-0.000426\\
32.9517791576674	-0.000698\\
33.0017818574514	-0.000773\\
33.0517845572354	-1e-06\\
33.1017872570194	0.001055\\
33.1517899568035	0.001259\\
33.2017926565875	0.001874\\
33.2517953563715	0.001415\\
33.3017980561555	0.000956\\
33.3518007559395	0.001491\\
33.4018034557235	0.000896\\
33.4518061555076	0.000446\\
33.5018088552916	-0.000154\\
33.5518115550756	-0.00028\\
33.6018142548596	0.000636\\
33.6518169546436	0.000594\\
33.7018196544277	0.001963\\
33.7518223542117	0.001605\\
33.8018250539957	-0.000556\\
33.8518277537797	-0.000975\\
33.9018304535637	-0.001765\\
33.9518331533477	-0.001546\\
34.0018358531318	-0.000955\\
34.0518385529158	-0.000721\\
34.1018412526998	-0.000503\\
34.1518439524838	0.000312\\
34.2018466522678	0.000513\\
34.2518493520518	-0.000383\\
34.3018520518359	-0.001435\\
34.3518547516199	-0.001256\\
34.4018574514039	-0.000319\\
34.4518601511879	0.000336\\
34.5018628509719	0.000396\\
34.5518655507559	0.00021\\
34.60186825054	0.000896\\
34.651870950324	0.000974\\
34.701873650108	0.000541\\
34.751876349892	0.000916\\
34.801879049676	0.001037\\
34.85188174946	0.000144\\
34.9018844492441	-0.000927\\
34.9518871490281	-0.000795\\
35.0018898488121	-0.000744\\
35.0518925485961	-0.000466\\
35.1018952483801	-0.000585\\
35.1518979481641	-0.000563\\
35.2019006479482	-0.000468\\
35.2519033477322	-7.3e-05\\
35.3019060475162	0.000127\\
35.3519087473002	0.000304\\
35.4019114470842	0.000302\\
35.4519141468683	0.000382\\
35.5019168466523	0.0006\\
35.5519195464363	0.000268\\
35.6019222462203	0.001058\\
35.6519249460043	-0.000153\\
35.7019276457883	-0.000187\\
35.7519303455724	-9.9e-05\\
35.8019330453564	-0.000302\\
35.8519357451404	-0.000503\\
35.9019384449244	-0.000419\\
35.9519411447084	-0.00054\\
36.0019438444924	-0.000704\\
36.0519465442765	-0.000754\\
36.1019492440605	-0.000678\\
36.1519519438445	-0.000524\\
36.2019546436285	-0.00027\\
36.2519573434125	0.00022\\
36.3019600431965	0.000374\\
36.3519627429806	0.00045\\
36.4019654427646	0.000601\\
36.4519681425486	0.001176\\
36.5019708423326	0.001753\\
36.5519735421166	0.001199\\
36.6019762419006	0.000332\\
36.6519789416847	-0.000237\\
36.7019816414687	-0.000368\\
36.7519843412527	-0.00011\\
36.8019870410367	-0.000386\\
36.8519897408207	-0.000655\\
36.9019924406048	4.7e-05\\
36.9519951403888	0.003273\\
37.0019978401728	0.002291\\
37.0520005399568	0.000155\\
37.1020032397408	-0.000589\\
37.1520059395248	-0.000857\\
37.2020086393089	-0.001545\\
37.2520113390929	-0.002093\\
37.3020140388769	-0.002181\\
37.3520167386609	-0.002056\\
37.4020194384449	-0.001086\\
37.4520221382289	0.000163\\
37.502024838013	0.000101\\
37.552027537797	-0.001578\\
37.602030237581	-0.002819\\
37.652032937365	-0.002939\\
37.702035637149	-0.002286\\
37.752038336933	-0.003363\\
37.8020410367171	-0.002636\\
37.8520437365011	-0.002885\\
37.9020464362851	-0.002836\\
37.9520491360691	-0.00196\\
38.0020518358531	-0.001381\\
38.0520545356372	-0.001641\\
38.1020572354212	-0.002199\\
38.1520599352052	-0.002522\\
38.2020626349892	-0.001957\\
38.2520653347732	-0.000627\\
38.3020680345572	-0.000366\\
38.3520707343413	-0.000876\\
38.4020734341253	-0.000837\\
38.4520761339093	-0.001529\\
38.5020788336933	-0.002187\\
38.5520815334773	-0.001354\\
38.6020842332613	-0.001315\\
38.6520869330454	-0.001128\\
38.7020896328294	-0.001426\\
38.7520923326134	-0.001604\\
38.8020950323974	-0.000835\\
38.8520977321814	-0.001468\\
38.9021004319654	-0.000856\\
38.9521031317495	-0.001313\\
39.0021058315335	-0.001383\\
39.0521085313175	-0.001014\\
39.1021112311015	-0.000163\\
39.1521139308855	-2.4e-05\\
39.2021166306696	-0.00022\\
39.2521193304536	0.001562\\
39.3021220302376	0.001112\\
39.3521247300216	0.000787\\
39.4021274298056	0.00132\\
39.4521301295896	0.003684\\
39.5021328293737	0.002306\\
39.5521355291577	0.001448\\
39.6021382289417	0.000143\\
39.6521409287257	-7e-05\\
39.7021436285097	-0.000516\\
39.7521463282937	-0.001232\\
39.8021490280778	-0.002899\\
39.8521517278618	-0.002467\\
39.9021544276458	-0.001196\\
39.9521571274298	-0.000397\\
40.0021598272138	0.000115\\
40.0521625269978	-0.000184\\
40.1021652267819	-0.000202\\
40.1521679265659	-0.000804\\
40.2021706263499	-0.000118\\
40.2521733261339	-0.001098\\
40.3021760259179	-0.001513\\
40.3521787257019	-0.001057\\
40.402181425486	-0.000858\\
40.45218412527	-0.000206\\
40.502186825054	0.000655\\
40.552189524838	0.001291\\
40.602192224622	0.000446\\
40.652194924406	0.000979\\
40.7021976241901	0.000357\\
40.7522003239741	0.001848\\
40.8022030237581	0.004324\\
40.8522057235421	0.00229\\
40.9022084233261	9.5e-05\\
40.9522111231102	-0.001619\\
41.0022138228942	-0.000867\\
41.0522165226782	3.4e-05\\
41.1022192224622	0.00111\\
41.1522219222462	0.000494\\
41.2022246220302	0.001196\\
41.2522273218143	0.000845\\
41.3022300215983	0.000868\\
41.3522327213823	0.000314\\
41.4022354211663	-0.000309\\
41.4522381209503	-0.000806\\
41.5022408207343	-0.001845\\
41.5522435205184	-0.002723\\
41.6022462203024	-0.00334\\
41.6522489200864	-0.002688\\
41.7022516198704	-0.002991\\
41.7522543196544	-0.003819\\
41.8022570194385	-0.002887\\
41.8522597192225	-0.001264\\
41.9022624190065	-0.000734\\
41.9522651187905	-0.001823\\
42.0022678185745	-0.001766\\
42.0522705183585	-0.002119\\
42.1022732181426	-0.002127\\
42.1522759179266	-0.001823\\
42.2022786177106	-0.000605\\
42.2522813174946	-0.001301\\
42.3022840172786	-0.002154\\
42.3522867170626	-0.001791\\
42.4022894168467	-0.001799\\
42.4522921166307	-0.00134\\
42.5022948164147	-0.000558\\
42.5522975161987	-0.001274\\
42.6023002159827	-0.000149\\
42.6523029157667	0.000437\\
42.7023056155508	9.9e-05\\
42.7523083153348	-0.000723\\
42.8023110151188	-0.001561\\
42.8523137149028	-0.001574\\
42.9023164146868	-0.00024\\
42.9523191144708	-0.000911\\
43.0023218142549	0.000252\\
43.0523245140389	0.001542\\
43.1023272138229	0.003832\\
43.1523299136069	0.002894\\
43.2023326133909	4.8e-05\\
43.2523353131749	-0.000508\\
43.302338012959	-0.000639\\
43.352340712743	-0.000619\\
43.402343412527	-8e-06\\
43.452346112311	-0.001281\\
43.502348812095	-0.00331\\
43.5523515118791	-0.003527\\
43.6023542116631	-0.003863\\
43.6523569114471	-0.003416\\
43.7023596112311	-0.003178\\
43.7523623110151	-0.003327\\
43.8023650107991	-0.002505\\
43.8523677105832	-0.002666\\
43.9023704103672	-0.003542\\
43.9523731101512	-0.004918\\
44.0023758099352	-0.002447\\
44.0523785097192	-0.002436\\
44.1023812095032	-0.001803\\
44.1523839092873	-0.0018\\
44.2023866090713	-0.001768\\
44.2523893088553	-0.001645\\
44.3023920086393	-0.000576\\
44.3523947084233	-0.001367\\
44.4023974082073	-0.002905\\
44.4524001079914	-0.004984\\
44.5024028077754	-0.003465\\
44.5524055075594	-0.004206\\
44.6024082073434	-0.002896\\
44.6524109071274	-0.003545\\
44.7024136069114	-0.003577\\
44.7524163066955	-0.003469\\
44.8024190064795	-0.002618\\
44.8524217062635	-0.002806\\
44.9024244060475	-0.003069\\
44.9524271058315	-0.002938\\
45.0024298056155	-0.002644\\
45.0524325053996	-0.002632\\
45.1024352051836	-0.00244\\
45.1524379049676	-0.002071\\
45.2024406047516	-0.002047\\
45.2524433045356	-0.002471\\
45.3024460043197	-0.00216\\
45.3524487041037	-0.00199\\
45.4024514038877	-0.001866\\
45.4524541036717	-0.001641\\
45.5024568034557	-0.002111\\
45.5524595032397	-0.002515\\
45.6024622030238	-0.001963\\
45.6524649028078	-0.002063\\
45.7024676025918	-0.002239\\
45.7524703023758	-0.002363\\
45.8024730021598	-0.001057\\
45.8524757019439	-0.000384\\
45.9024784017279	-0.000584\\
45.9524811015119	-0.000986\\
46.0024838012959	-0.001498\\
46.0524865010799	-0.002185\\
46.1024892008639	-0.002259\\
46.152491900648	-0.002252\\
46.202494600432	-0.001481\\
46.252497300216	-0.002611\\
46.3025	-0.003405\\
46.352502699784	-0.001891\\
46.402505399568	-0.001959\\
46.4525080993521	-0.002243\\
46.5025107991361	-0.00304\\
46.5525134989201	-0.003185\\
46.6025161987041	-0.003029\\
46.6525188984881	-0.002869\\
46.7025215982721	-0.002152\\
46.7525242980562	-0.000599\\
46.8025269978402	0.000168\\
46.8525296976242	0.000693\\
46.9025323974082	-0.000146\\
46.9525350971922	-0.00033\\
47.0025377969762	-0.000447\\
47.0525404967603	-0.000408\\
47.1025431965443	-0.000201\\
47.1525458963283	-0.00068\\
47.2025485961123	-0.000731\\
47.2525512958963	-0.000779\\
47.3025539956803	-7.9e-05\\
47.3525566954644	-0.000812\\
47.4025593952484	-0.000704\\
47.4525620950324	0.000393\\
47.5025647948164	-0.001894\\
47.5525674946004	-0.003063\\
47.6025701943845	-0.004501\\
47.6525728941685	-0.004738\\
47.7025755939525	-0.005063\\
47.7525782937365	-0.005582\\
47.8025809935205	-0.005517\\
47.8525836933045	-0.004928\\
47.9025863930886	-0.004895\\
47.9525890928726	-0.003785\\
48.0025917926566	-0.00371\\
48.0525944924406	-0.004048\\
48.1025971922246	-0.004175\\
48.1525998920086	-0.003263\\
48.2026025917927	-0.001863\\
48.2526052915767	-0.001425\\
48.3026079913607	-0.001444\\
48.3526106911447	-0.00173\\
48.4026133909287	-0.00149\\
48.4526160907127	-0.000102\\
48.5026187904968	0.000626\\
48.5526214902808	-0.000229\\
48.6026241900648	-0.000453\\
48.6526268898488	-0.00125\\
48.7026295896328	-0.001913\\
48.7526322894168	-0.001937\\
48.8026349892009	-0.002132\\
48.8526376889849	-0.001852\\
48.9026403887689	-0.000987\\
48.9526430885529	0.000231\\
49.0026457883369	0.000141\\
49.052648488121	-4.1e-05\\
49.102651187905	-0.000822\\
49.152653887689	0.000427\\
49.202656587473	0.000446\\
49.252659287257	-0.000201\\
49.302661987041	-0.000481\\
49.3526646868251	-0.001367\\
49.4026673866091	-0.001303\\
49.4526700863931	-0.000603\\
49.5026727861771	-0.000909\\
49.5526754859611	-0.001499\\
49.6026781857451	-0.00151\\
49.6526808855292	-0.001485\\
49.7026835853132	-0.000807\\
49.7526862850972	-0.001407\\
49.8026889848812	-0.001112\\
49.8526916846652	-0.001677\\
49.9026943844492	-0.001922\\
49.9526970842333	-0.001382\\
50.0026997840173	8.8e-05\\
50.0527024838013	0.001873\\
50.1027051835853	0.001002\\
50.1527078833693	0.001078\\
50.2027105831534	0.000939\\
50.2527132829374	0.000328\\
50.3027159827214	-0.000237\\
50.3527186825054	-0.000787\\
50.4027213822894	-0.00029\\
50.4527240820734	-0.000632\\
50.5027267818575	-0.000151\\
50.5527294816415	0.000624\\
50.6027321814255	0.000355\\
50.6527348812095	5.1e-05\\
50.7027375809935	-4.1e-05\\
50.7527402807775	-0.000506\\
50.8027429805616	-0.001314\\
50.8527456803456	0.000359\\
50.9027483801296	-0.00096\\
50.9527510799136	-0.001901\\
51.0027537796976	-0.002947\\
51.0527564794816	-0.002002\\
51.1027591792657	-0.00142\\
51.1527618790497	-0.001424\\
51.2027645788337	-0.001347\\
51.2527672786177	0.000382\\
51.3027699784017	-0.000836\\
51.3527726781858	-0.002007\\
51.4027753779698	-0.003193\\
51.4527780777538	-0.002022\\
51.5027807775378	-0.001807\\
51.5527834773218	-0.001439\\
51.6027861771058	-0.002094\\
51.6527888768899	-0.000734\\
51.7027915766739	-0.001545\\
51.7527942764579	-0.001464\\
51.8027969762419	-0.001235\\
51.8527996760259	-0.001228\\
51.9028023758099	-0.001339\\
51.952805075594	-0.001694\\
52.002807775378	-0.002643\\
52.052810475162	-0.002978\\
52.102813174946	-0.003149\\
52.15281587473	-0.003454\\
52.202818574514	-0.003025\\
52.2528212742981	-0.00172\\
52.3028239740821	-0.001405\\
52.3528266738661	-0.001853\\
52.4028293736501	-0.002155\\
52.4528320734341	-0.003213\\
52.5028347732181	-0.002219\\
52.5528374730022	-0.00234\\
52.6028401727862	-0.002304\\
52.6528428725702	-0.001948\\
52.7028455723542	-0.001393\\
52.7528482721382	-0.000995\\
52.8028509719222	-0.000534\\
52.8528536717063	0.000495\\
52.9028563714903	0.000636\\
52.9528590712743	-0.000811\\
53.0028617710583	-0.000674\\
53.0528644708423	-0.000553\\
53.1028671706264	-0.001695\\
53.1528698704104	-0.002949\\
53.2028725701944	-0.002658\\
53.2528752699784	-0.000867\\
53.3028779697624	-0.001431\\
53.3528806695464	-0.002689\\
53.4028833693305	-0.003605\\
53.4528860691145	-0.003788\\
53.5028887688985	-0.003629\\
53.5528914686825	-0.003284\\
53.6028941684665	-0.002468\\
53.6528968682505	-0.002477\\
53.7028995680346	-0.001863\\
53.7529022678186	0.00067\\
53.8029049676026	0.001978\\
53.8529076673866	0.001269\\
53.9029103671706	-0.000812\\
53.9529130669547	-0.001214\\
54.0029157667387	-5.1e-05\\
54.0529184665227	0.000514\\
54.1029211663067	0.001192\\
54.1529238660907	0.002882\\
54.2029265658747	0.003517\\
54.2529292656588	0.001256\\
54.3029319654428	-0.001867\\
54.3529346652268	-0.001594\\
54.4029373650108	0.000661\\
54.4529400647948	-8e-05\\
54.5029427645788	-0.002743\\
54.5529454643629	-0.003544\\
54.6029481641469	-0.004488\\
54.6529508639309	-0.003031\\
54.7029535637149	-0.000565\\
54.7529562634989	-0.000271\\
54.8029589632829	-0.002065\\
54.852961663067	-0.003111\\
54.902964362851	-0.002274\\
54.952967062635	-0.001285\\
55.002969762419	-0.000808\\
55.052972462203	-0.000772\\
55.102975161987	-0.001004\\
55.1529778617711	-0.003166\\
55.2029805615551	-0.004529\\
55.2529832613391	-0.003937\\
55.3029859611231	-0.002646\\
55.3529886609071	-0.003674\\
55.4029913606911	-0.002948\\
55.4529940604752	-0.001488\\
55.5029967602592	-0.000413\\
55.5529994600432	-0.000574\\
55.6030021598272	-0.001775\\
55.6530048596112	-0.002\\
55.7030075593953	-0.002148\\
55.7530102591793	-0.001529\\
55.8030129589633	-0.00099\\
55.8530156587473	-0.001782\\
55.9030183585313	-0.002284\\
55.9530210583153	-0.002648\\
56.0030237580994	-0.002479\\
56.0530264578834	-0.001724\\
56.1030291576674	0.000197\\
56.1530318574514	0.000194\\
56.2030345572354	0.000278\\
56.2530372570194	-0.000147\\
56.3030399568035	-0.000184\\
56.3530426565875	0.00076\\
56.4030453563715	0.000574\\
56.4530480561555	0.000281\\
56.5030507559395	-0.000198\\
56.5530534557235	-0.000185\\
56.6030561555076	0.001157\\
56.6530588552916	0.001644\\
56.7030615550756	0.000788\\
56.7530642548596	0.000483\\
56.8030669546436	0.000289\\
56.8530696544276	0.000377\\
56.9030723542117	0.000883\\
56.9530750539957	0.000512\\
57.0030777537797	-0.000376\\
57.0530804535637	-0.001051\\
57.1030831533477	-0.001333\\
57.1530858531318	-0.000678\\
57.2030885529158	0.000921\\
57.2530912526998	0.00054\\
57.3030939524838	-0.00012\\
57.3530966522678	0.000372\\
57.4030993520518	7.6e-05\\
57.4531020518358	-7.2e-05\\
57.5031047516199	0.000286\\
57.5531074514039	-0.000418\\
57.6031101511879	-0.000726\\
57.6531128509719	-0.000283\\
57.7031155507559	-0.001326\\
57.75311825054	-0.002458\\
57.803120950324	-0.002663\\
57.853123650108	-0.002176\\
57.903126349892	-0.001027\\
57.953129049676	-0.001172\\
58.00313174946	0.000723\\
58.0531344492441	-0.00106\\
58.1031371490281	-0.002301\\
58.1531398488121	-0.002903\\
58.2031425485961	-0.002291\\
58.2531452483801	-0.000757\\
58.3031479481642	-0.000762\\
58.3531506479482	0.000244\\
58.4031533477322	-0.000239\\
58.4531560475162	0.001046\\
58.5031587473002	0.000343\\
58.5531614470842	-0.0001\\
58.6031641468683	-0.00076\\
58.6531668466523	-0.001297\\
58.7031695464363	-2.3e-05\\
58.7531722462203	0.000405\\
58.8031749460043	0.000761\\
58.8531776457883	-0.000399\\
58.9031803455724	-0.000771\\
58.9531830453564	-0.002974\\
59.0031857451404	-0.004057\\
59.0531884449244	-0.004759\\
59.1031911447084	-0.005312\\
59.1531938444924	-0.005328\\
59.2031965442765	-0.005358\\
59.2531992440605	-0.004697\\
59.3032019438445	-0.004715\\
59.3532046436285	-0.0044\\
59.4032073434125	-0.002977\\
59.4532100431965	-0.00188\\
59.5032127429806	-0.001484\\
59.5532154427646	-0.001\\
59.6032181425486	-0.001507\\
59.6532208423326	-0.001765\\
59.7032235421166	-0.000578\\
59.7532262419006	-0.000879\\
59.8032289416847	-1.7e-05\\
59.8532316414687	0.000717\\
59.9032343412527	0.000141\\
59.9532370410367	-0.000806\\
60.0032397408207	-0.002428\\
60.0532424406048	-0.002233\\
60.1032451403888	-0.002036\\
60.1532478401728	-0.001703\\
60.2032505399568	-0.001437\\
60.2532532397408	-0.001439\\
60.3032559395248	-0.000312\\
60.3532586393089	0.000129\\
60.4032613390929	-0.000573\\
60.4532640388769	0.000139\\
60.5032667386609	-0.001185\\
60.5532694384449	-0.002569\\
60.6032721382289	-0.003136\\
60.653274838013	-0.002828\\
60.703277537797	-0.003\\
60.753280237581	-0.003516\\
60.803282937365	-0.002424\\
60.853285637149	-0.001302\\
60.9032883369331	-0.001825\\
60.9532910367171	-0.002181\\
61.0032937365011	-0.001629\\
61.0532964362851	-0.001031\\
61.1032991360691	0.000179\\
61.1533018358531	0.000829\\
61.2033045356371	0.000852\\
61.2533072354212	3.6e-05\\
61.3033099352052	-0.000191\\
61.3533126349892	-0.000515\\
61.4033153347732	-0.000245\\
61.4533180345572	0.000949\\
61.5033207343413	0.000657\\
61.5533234341253	0.000554\\
61.6033261339093	0.000466\\
61.6533288336933	0.000853\\
61.7033315334773	0.000969\\
61.7533342332613	0.001278\\
61.8033369330454	0.001363\\
61.8533396328294	0.000617\\
61.9033423326134	0.000165\\
61.9533450323974	-0.000305\\
62.0033477321814	-0.000307\\
62.0533504319654	0.000556\\
62.1033531317495	0.000563\\
62.1533558315335	0.000515\\
62.2033585313175	0.000779\\
62.2533612311015	-1.7e-05\\
62.3033639308855	0.002252\\
62.3533666306695	0.002984\\
62.4033693304536	0.001301\\
62.4533720302376	-0.000137\\
62.5033747300216	6.5e-05\\
62.5533774298056	0.00034\\
62.6033801295896	0.002642\\
62.6533828293737	0.014417\\
62.6983852591793	0.036296\\
62.7433876889849	0.066141\\
62.7883901187905	0.09605\\
62.8333925485961	0.121222\\
62.8833952483801	0.141309\\
62.9333979481641	0.151022\\
62.9834006479482	0.155142\\
63.0334033477322	0.159414\\
63.0834060475162	0.166637\\
63.1334087473002	0.173664\\
63.1834114470842	0.178592\\
63.2334141468683	0.18183\\
63.2834168466523	0.184875\\
63.3334195464363	0.18768\\
63.3834222462203	0.190456\\
63.4334249460043	0.193105\\
63.4834276457883	0.195156\\
63.5334303455724	0.196708\\
63.5834330453564	0.197873\\
63.6334357451404	0.199034\\
63.6834384449244	0.19983\\
63.7334411447084	0.200343\\
63.7834438444924	0.200503\\
63.8334465442765	0.201718\\
63.8834492440605	0.202574\\
63.9334519438445	0.202726\\
63.9834546436285	0.202862\\
64.0334573434125	0.20272\\
64.0834600431965	0.203007\\
64.1334627429806	0.203288\\
64.1834654427646	0.203454\\
64.2334681425486	0.203445\\
64.2834708423326	0.203459\\
64.3334735421166	0.203516\\
64.3834762419007	0.202874\\
64.4334789416847	0.202032\\
64.4834816414687	0.201554\\
64.5334843412527	0.201426\\
64.5834870410367	0.201694\\
64.6334897408207	0.201893\\
64.6834924406048	0.202098\\
64.7334951403888	0.202161\\
64.7834978401728	0.201787\\
64.8335005399568	0.201817\\
64.8835032397408	0.201642\\
64.9335059395248	0.201283\\
64.9835086393089	0.200854\\
65.0335113390929	0.200952\\
65.0835140388769	0.201085\\
65.1335167386609	0.201288\\
65.1835194384449	0.201408\\
65.2335221382289	0.201285\\
65.283524838013	0.201574\\
65.333527537797	0.201591\\
65.383530237581	0.201806\\
65.433532937365	0.201915\\
65.483535637149	0.202041\\
65.533538336933	0.202123\\
65.5835410367171	0.20231\\
65.6335437365011	0.202297\\
65.6835464362851	0.202037\\
65.7335491360691	0.201688\\
65.7835518358531	0.201264\\
65.8335545356372	0.200999\\
65.8835572354212	0.200845\\
65.9335599352052	0.200777\\
65.9835626349892	0.200565\\
66.0335653347732	0.200237\\
66.0835680345572	0.200148\\
66.1335707343412	0.200118\\
66.1835734341253	0.200189\\
66.2335761339093	0.200502\\
66.2835788336933	0.200789\\
66.3335815334773	0.200927\\
66.3835842332613	0.200981\\
66.4335869330454	0.201087\\
66.4835896328294	0.201103\\
66.5335923326134	0.201112\\
66.5835950323974	0.200848\\
66.6335977321814	0.200899\\
66.6836004319654	0.201008\\
66.7336031317495	0.201146\\
66.7836058315335	0.201239\\
66.8336085313175	0.201274\\
66.8836112311015	0.201318\\
66.9336139308855	0.201404\\
66.9836166306695	0.201522\\
67.0336193304536	0.201643\\
67.0836220302376	0.201592\\
67.1336247300216	0.20131\\
67.1836274298056	0.201433\\
67.2336301295896	0.201725\\
67.2836328293737	0.201901\\
67.3336355291577	0.202029\\
67.3836382289417	0.201844\\
67.4336409287257	0.201894\\
67.4836436285097	0.201917\\
67.5336463282937	0.201727\\
67.5836490280778	0.201856\\
67.6336517278618	0.201946\\
67.6836544276458	0.202175\\
67.7336571274298	0.202373\\
67.7836598272138	0.202331\\
67.8336625269979	0.202191\\
67.8836652267819	0.202154\\
67.9336679265659	0.202235\\
67.9836706263499	0.202324\\
68.0336733261339	0.20307\\
68.0836760259179	0.203558\\
68.1336787257019	0.2039\\
68.183681425486	0.204087\\
68.23368412527	0.204074\\
68.283686825054	0.203999\\
68.333689524838	0.203588\\
68.383692224622	0.202999\\
68.4336949244061	0.202441\\
68.4836976241901	0.202203\\
68.5337003239741	0.202151\\
68.5837030237581	0.201865\\
68.6337057235421	0.201796\\
68.6837084233261	0.201512\\
68.7337111231102	0.201313\\
68.7837138228942	0.201374\\
68.8337165226782	0.201687\\
68.8837192224622	0.201784\\
68.9337219222462	0.201976\\
68.9837246220302	0.202052\\
69.0337273218143	0.2021\\
69.0837300215983	0.202204\\
69.1337327213823	0.202518\\
69.1837354211663	0.202701\\
69.2337381209503	0.202252\\
69.2837408207344	0.202488\\
69.3337435205184	0.203084\\
69.3837462203024	0.203379\\
69.4337489200864	0.203494\\
69.4837516198704	0.203539\\
69.5337543196544	0.203714\\
69.5837570194385	0.2032\\
69.6337597192225	0.202855\\
69.6837624190065	0.202469\\
69.7337651187905	0.202526\\
69.7837678185745	0.202685\\
69.8337705183585	0.20253\\
69.8837732181426	0.202437\\
69.9337759179266	0.202234\\
69.9837786177106	0.202711\\
70.0337813174946	0.203225\\
70.0837840172786	0.203595\\
70.1337867170626	0.203766\\
70.1837894168467	0.20321\\
70.2337921166307	0.202877\\
70.2837948164147	0.202761\\
70.3337975161987	0.202732\\
70.3838002159827	0.202549\\
70.4338029157667	0.202445\\
70.4838056155508	0.202154\\
70.5338083153348	0.202153\\
70.5838110151188	0.202065\\
70.6338137149028	0.202323\\
70.6838164146868	0.2023\\
70.7338191144708	0.202156\\
70.7838218142549	0.202008\\
70.8338245140389	0.201922\\
70.8838272138229	0.201973\\
70.9338299136069	0.202038\\
70.9838326133909	0.20207\\
71.0338353131749	0.202123\\
71.083838012959	0.201951\\
71.133840712743	0.202631\\
71.183843412527	0.203185\\
71.233846112311	0.202786\\
71.283848812095	0.203137\\
71.3338515118791	0.203348\\
71.3838542116631	0.203153\\
71.4338569114471	0.202946\\
71.4838596112311	0.202923\\
71.5338623110151	0.203298\\
71.5838650107991	0.203595\\
71.6338677105831	0.203618\\
71.6838704103672	0.203693\\
71.7338731101512	0.203516\\
71.7838758099352	0.20359\\
71.8338785097192	0.20356\\
71.8838812095032	0.203032\\
71.9338839092873	0.202657\\
71.9838866090713	0.202482\\
72.0338893088553	0.203069\\
72.0838920086393	0.203427\\
72.1338947084233	0.203588\\
72.1838974082073	0.20369\\
72.2339001079914	0.203769\\
72.2839028077754	0.203861\\
72.3339055075594	0.203207\\
72.3839082073434	0.202745\\
72.4339109071274	0.202324\\
72.4839136069115	0.20239\\
72.5339163066955	0.203073\\
72.5839190064795	0.203404\\
72.6339217062635	0.20338\\
72.6839244060475	0.203528\\
72.7339271058315	0.203598\\
72.7839298056156	0.203513\\
72.8339325053996	0.203075\\
72.8839352051836	0.202633\\
72.9339379049676	0.20241\\
72.9839406047516	0.202839\\
73.0339433045356	0.203345\\
73.0839460043197	0.203567\\
73.1339487041037	0.203682\\
73.1839514038877	0.203158\\
73.2339541036717	0.202369\\
73.2839568034557	0.202201\\
73.3339595032398	0.202239\\
73.3839622030238	0.202887\\
73.4339649028078	0.203255\\
73.4839676025918	0.203435\\
73.5339703023758	0.203239\\
73.5839730021598	0.20263\\
73.6339757019438	0.202221\\
73.6839784017279	0.202298\\
73.7339811015119	0.201874\\
73.7839838012959	0.202039\\
73.8339865010799	0.201898\\
73.8839892008639	0.201797\\
73.933991900648	0.201809\\
73.983994600432	0.201514\\
74.033997300216	0.201534\\
74.084	0.201587\\
74.134002699784	0.201616\\
74.184005399568	0.201402\\
74.2340080993521	0.201831\\
74.2840107991361	0.202178\\
74.3340134989201	0.202252\\
74.3840161987041	0.202331\\
74.4340188984881	0.202158\\
74.4840215982721	0.20227\\
74.5340242980562	0.20234\\
74.5840269978402	0.20268\\
74.6340296976242	0.203319\\
74.6840323974082	0.203421\\
74.7340350971922	0.20373\\
74.7840377969763	0.203927\\
74.8340404967603	0.203381\\
74.8840431965443	0.202858\\
74.9340458963283	0.202607\\
74.9840485961123	0.202408\\
75.0340512958963	0.201879\\
75.0840539956804	0.202421\\
75.1340566954644	0.202154\\
75.1840593952484	0.202187\\
75.2340620950324	0.201902\\
75.2840647948164	0.201807\\
75.3340674946004	0.201822\\
75.3840701943844	0.201893\\
75.4340728941685	0.201953\\
75.4840755939525	0.201979\\
75.5340782937365	0.20215\\
75.5840809935205	0.201999\\
75.6340836933045	0.201869\\
75.6840863930885	0.201785\\
75.7340890928726	0.201688\\
75.7840917926566	0.20164\\
75.8340944924406	0.201842\\
75.8840971922246	0.202001\\
75.9340998920086	0.20186\\
75.9841025917927	0.201815\\
76.0341052915767	0.201663\\
76.0841079913607	0.202289\\
76.1341106911447	0.202301\\
76.1841133909287	0.20228\\
76.2341160907127	0.201849\\
76.2841187904968	0.20173\\
76.3341214902808	0.20174\\
76.3841241900648	0.201855\\
76.4341268898488	0.201885\\
76.4841295896328	0.202387\\
76.5341322894169	0.202987\\
76.5841349892009	0.203171\\
76.6341376889849	0.202951\\
76.6841403887689	0.202687\\
76.7341430885529	0.202423\\
76.7841457883369	0.201811\\
76.834148488121	0.201889\\
76.884151187905	0.202296\\
76.934153887689	0.203021\\
76.984156587473	0.203057\\
77.034159287257	0.202906\\
77.084161987041	0.202708\\
77.1341646868251	0.202439\\
77.1841673866091	0.202025\\
77.2341700863931	0.201916\\
77.2841727861771	0.20203\\
77.3341754859611	0.202516\\
77.3841781857451	0.203135\\
77.4341808855292	0.20343\\
77.4841835853132	0.203513\\
77.5341862850972	0.202987\\
77.5841889848812	0.202095\\
77.6341916846652	0.201628\\
77.6841943844492	0.201458\\
77.7341970842333	0.200586\\
77.7841997840173	0.201127\\
77.8342024838013	0.202021\\
77.8842051835853	0.202771\\
77.9342078833693	0.203168\\
77.9842105831534	0.203315\\
78.0342132829374	0.202659\\
78.0842159827214	0.202179\\
78.1342186825054	0.201603\\
78.1842213822894	0.201541\\
78.2342240820734	0.201471\\
78.2842267818575	0.201387\\
78.3342294816415	0.201095\\
78.3842321814255	0.201105\\
78.4342348812095	0.200949\\
78.4842375809935	0.201073\\
78.5342402807775	0.201024\\
78.5842429805616	0.200975\\
78.6342456803456	0.201445\\
78.6842483801296	0.201945\\
78.7342510799136	0.201516\\
78.7842537796976	0.201501\\
78.8342564794817	0.201463\\
78.8842591792657	0.201274\\
78.9342618790497	0.201107\\
78.9842645788337	0.20094\\
79.0342672786177	0.2008\\
79.0842699784017	0.200691\\
79.1342726781857	0.200386\\
79.1842753779698	0.200392\\
79.2342780777538	0.200434\\
79.2842807775378	0.200567\\
79.3342834773218	0.20061\\
79.3842861771058	0.200759\\
79.4342888768899	0.200873\\
79.4842915766739	0.200876\\
79.5342942764579	0.200423\\
79.5842969762419	0.200559\\
79.6342996760259	0.200762\\
79.6843023758099	0.200911\\
79.734305075594	0.200966\\
79.784307775378	0.200924\\
79.834310475162	0.200608\\
79.884313174946	0.20082\\
79.93431587473	0.200875\\
79.984318574514	0.200685\\
80.0343212742981	0.200811\\
80.0843239740821	0.200996\\
80.1343266738661	0.201148\\
80.1843293736501	0.200803\\
80.2343320734341	0.201082\\
80.2843347732181	0.20139\\
80.3343374730022	0.20161\\
80.3843401727862	0.201675\\
80.4343428725702	0.201667\\
80.4843455723542	0.201527\\
80.5343482721382	0.201224\\
80.5843509719223	0.201313\\
80.6343536717063	0.20135\\
80.6843563714903	0.201304\\
80.7343590712743	0.201125\\
80.7843617710583	0.20136\\
80.8343644708423	0.20147\\
80.8843671706263	0.201202\\
80.9343698704104	0.201256\\
80.9843725701944	0.201153\\
81.0343752699784	0.201345\\
81.0843779697624	0.201435\\
81.1343806695464	0.20121\\
81.1843833693305	0.201289\\
81.2343860691145	0.201418\\
81.2843887688985	0.201394\\
81.3343914686825	0.201603\\
81.3843941684665	0.20174\\
81.4343968682505	0.201848\\
81.4843995680346	0.201917\\
81.5344022678186	0.201957\\
81.5844049676026	0.201923\\
81.6344076673866	0.201776\\
81.6844103671706	0.201694\\
81.7344130669546	0.20143\\
81.7844157667387	0.201274\\
81.8344184665227	0.201379\\
81.8844211663067	0.201832\\
81.9344238660907	0.202128\\
81.9844265658747	0.201958\\
82.0344292656588	0.201554\\
82.0844319654428	0.201199\\
82.1344346652268	0.200967\\
82.1844373650108	0.201212\\
82.2344400647948	0.20136\\
82.2844427645788	0.201626\\
82.3344454643629	0.201692\\
82.3844481641469	0.20138\\
82.4344508639309	0.201146\\
82.4844535637149	0.201043\\
82.5344562634989	0.200973\\
82.5844589632829	0.20116\\
82.634461663067	0.200959\\
82.684464362851	0.199022\\
82.734467062635	0.189471\\
82.784469762419	0.168776\\
82.8294721922246	0.139521\\
82.8744746220302	0.108195\\
82.9194770518359	0.083092\\
82.9694797516199	0.06655\\
83.0194824514039	0.057887\\
83.0694851511879	0.050696\\
83.1194878509719	0.041472\\
83.1694905507559	0.033\\
83.21949325054	0.026332\\
83.269495950324	0.022057\\
83.319498650108	0.018593\\
83.369501349892	0.014812\\
83.419504049676	0.01088\\
83.4695067494601	0.008768\\
83.5195094492441	0.008386\\
83.5695121490281	0.007319\\
83.6195148488121	0.005662\\
83.6695175485961	0.003801\\
83.7195202483801	0.003023\\
83.7695229481642	0.002741\\
83.8195256479482	0.002743\\
83.8695283477322	0.001931\\
83.9195310475162	0.001214\\
83.9695337473002	0.000596\\
84.0195364470842	0.00024\\
84.0695391468683	0.001009\\
84.1195418466523	0.001001\\
84.1695445464363	-0.000389\\
84.2195472462203	-0.002576\\
84.2695499460043	-0.002941\\
84.3195526457883	-0.001291\\
84.3695553455724	-0.000597\\
84.4195580453564	-0.000677\\
84.4695607451404	-0.001232\\
84.5195634449244	-0.001129\\
84.5695661447084	-0.000848\\
84.6195688444924	-0.000569\\
84.6695715442765	-0.00054\\
84.7195742440605	0.000295\\
84.7695769438445	0.000536\\
84.8195796436285	0.000497\\
84.8695823434125	0.000329\\
84.9195850431965	0.000358\\
84.9695877429806	0.001734\\
85.0195904427646	0.001105\\
85.0695931425486	0.000672\\
85.1195958423326	-0.000331\\
85.1695985421166	0.000362\\
85.2196012419006	0.000912\\
85.2696039416847	0.002235\\
85.3196066414687	0.002055\\
85.3696093412527	0.001953\\
85.4196120410367	0.002127\\
85.4696147408207	0.001322\\
85.5196174406048	0.000747\\
85.5696201403888	0.001082\\
85.6196228401728	0.000829\\
85.6696255399568	0.001906\\
85.7196282397408	0.001978\\
85.7696309395248	0.001035\\
85.8196336393089	0.000934\\
85.8696363390929	0.00222\\
85.9196390388769	0.002693\\
85.9696417386609	0.001568\\
86.0196444384449	0.000773\\
86.0696471382289	-0.000723\\
86.119649838013	-0.001303\\
86.169652537797	-0.000326\\
86.219655237581	-0.000527\\
86.269657937365	-0.000521\\
86.319660637149	-0.000839\\
86.369663336933	-0.001169\\
86.4196660367171	-0.001302\\
86.4696687365011	-0.000596\\
86.5196714362851	-0.000136\\
86.5696741360691	0.001274\\
86.6196768358531	0.001077\\
86.6696795356372	0.000311\\
86.7196822354212	0.000236\\
86.7696849352052	0.000744\\
86.8196876349892	0.000964\\
86.8696903347732	0.001639\\
86.9196930345572	0.001317\\
86.9696957343413	9.5e-05\\
87.0196984341253	-0.000386\\
87.0697011339093	0.000913\\
87.1197038336933	0.002098\\
87.1697065334773	0.001729\\
87.2197092332613	0.000379\\
87.2697119330454	-0.000594\\
87.3197146328294	-0.000562\\
87.3697173326134	-0.000397\\
87.4197200323974	-0.000693\\
87.4697227321814	-6.9e-05\\
87.5197254319655	-0.000607\\
87.5697281317495	-0.000568\\
87.6197308315335	0.001747\\
87.6697335313175	0.001617\\
87.7197362311015	0.000508\\
87.7697389308855	0.000241\\
87.8197416306695	0.00023\\
87.8697443304536	0.0005\\
87.9197470302376	-0.000248\\
87.9697497300216	-0.000481\\
88.0197524298056	-0.000898\\
88.0697551295896	-0.00091\\
88.1197578293737	-0.001023\\
88.1697605291577	-0.001069\\
88.2197632289417	-0.000959\\
88.2697659287257	-0.000366\\
88.3197686285097	0.000267\\
88.3697713282937	0.001655\\
88.4197740280778	0.001731\\
88.4697767278618	0.003801\\
88.5197794276458	0.00382\\
88.5697821274298	0.002682\\
88.6197848272138	0.000945\\
88.6697875269979	0.000375\\
88.7197902267819	4.5e-05\\
88.7697929265659	0.001715\\
88.8197956263499	0.00197\\
88.8697983261339	0.001677\\
88.9198010259179	0.001338\\
88.969803725702	0.000488\\
89.019806425486	0.0009\\
89.06980912527	0.000766\\
89.119811825054	0.000675\\
89.169814524838	0.001351\\
89.219817224622	0.002006\\
89.2698199244061	0.002146\\
89.3198226241901	0.002362\\
89.3698253239741	0.002307\\
89.4198280237581	0.001465\\
89.4698307235421	0.00156\\
89.5198334233261	0.001772\\
89.5698361231102	0.001424\\
89.6198388228942	0.001852\\
89.6698415226782	0.001046\\
89.7198442224622	0.00183\\
89.7698469222462	0.001939\\
89.8198496220302	0.002001\\
89.8698523218143	0.00098\\
89.9198550215983	-5.9e-05\\
89.9698577213823	-0.000762\\
90.0198604211663	8.2e-05\\
90.0698631209503	0.00051\\
90.1198658207343	-0.000239\\
90.1698685205184	-0.00082\\
90.2198712203024	-0.00081\\
90.2698739200864	0.000115\\
90.3198766198704	0.003246\\
90.3698793196544	0.00488\\
90.4198820194384	0.005064\\
90.4698847192225	0.002435\\
90.5198874190065	0.00157\\
90.5698901187905	0.000199\\
90.6198928185745	-7e-06\\
90.6698955183585	-0.000211\\
90.7198982181426	0.000249\\
90.7699009179266	-0.000424\\
90.8199036177106	-0.000941\\
90.8699063174946	-0.001261\\
90.9199090172786	-0.000779\\
90.9699117170626	-0.002059\\
91.0199144168467	-0.002635\\
91.0699171166307	-0.001968\\
91.1199198164147	-0.000395\\
91.1699225161987	0.000681\\
91.2199252159827	0.000896\\
91.2699279157667	0.001861\\
91.3199306155508	0.001365\\
91.3699333153348	0.00051\\
91.4199360151188	0.000113\\
91.4699387149028	0.000488\\
91.5199414146868	0.000654\\
91.5699441144708	0.001186\\
91.6199468142549	0.000824\\
91.6699495140389	0.001013\\
91.7199522138229	0.00069\\
91.7699549136069	0.002648\\
91.8199576133909	0.002392\\
91.8699603131749	0.002036\\
91.919963012959	0.001627\\
91.969965712743	0.001747\\
92.019968412527	0.002095\\
92.069971112311	0.001551\\
92.119973812095	0.002309\\
92.1699765118791	0.002372\\
92.2199792116631	0.001649\\
92.2699819114471	0.002331\\
92.3199846112311	0.001723\\
92.3699873110151	0.001579\\
92.4199900107991	0.002541\\
92.4699927105832	0.002829\\
92.5199954103672	0.001885\\
92.5699981101512	0.003252\\
92.605	0.003252\\
};
\addlegendentry{Estimated position [m]};

\addplot [color=blue,solid,line width=0.2pt]
  table[row sep=crcr]{0	-0.0749067719101089\\
0.0500026997840173	-0.0756575527496417\\
0.100005399568035	-0.0755463917366384\\
0.150008099352052	-0.0754819939799291\\
0.200010799136069	-0.0754424823058127\\
0.250013498920086	-0.0755106348605658\\
0.300016198704104	-0.0755070659702472\\
0.350018898488121	-0.0754657842409125\\
0.400021598272138	-0.0753940467724746\\
0.450024298056156	-0.0751573862684106\\
0.500026997840173	-0.0753222664025992\\
0.55002969762419	-0.075684518712868\\
0.600032397408207	-0.0756881669400238\\
0.650035097192225	-0.0755985775994107\\
0.700037796976242	-0.07560929083947\\
0.750040496760259	-0.0754711685200185\\
0.800043196544276	-0.0755896288163063\\
0.850045896328294	-0.0755178348716048\\
0.900048596112311	-0.0754425418342413\\
0.950051295896328	-0.0753779265865785\\
1.00005399568035	-0.0755356821504329\\
1.05005669546436	-0.0752613055781661\\
1.10005939524838	-0.075397633537239\\
1.1500620950324	-0.0757400724448426\\
1.20006479481641	-0.0756577405276306\\
1.25006749460043	-0.0755986035211804\\
1.30007019438445	-0.0754801426142594\\
1.35007289416847	-0.075196859228696\\
1.40007559395248	-0.0751913834723345\\
1.4500782937365	-0.075324119572121\\
1.50008099352052	-0.07524167792677\\
1.55008369330454	-0.0756630542406475\\
1.60008639308855	-0.0758100507059259\\
1.65008909287257	-0.0757096105790356\\
1.70009179265659	-0.0755895975825141\\
1.7500944924406	-0.0755322125550733\\
1.80009719222462	-0.0755231490336972\\
1.85009989200864	-0.0753617467508887\\
1.90010259179266	-0.0752578490561568\\
1.95010529157667	-0.0757687928378589\\
2.00010799136069	-0.0755965601809048\\
2.05011069114471	-0.0762455577957876\\
2.10011339092873	-0.0726628665586936\\
2.15011609071274	-0.0653693271160434\\
2.20011879049676	-0.0544194462585877\\
2.25012149028078	-0.0428974159908262\\
2.30012419006479	-0.0345433641546035\\
2.35012688984881	-0.0307157467317714\\
2.40012958963283	-0.0281870673800009\\
2.45013228941685	-0.0249247063455462\\
2.50013498920086	-0.0209469889515255\\
2.55013768898488	-0.0177849845374986\\
2.6001403887689	-0.0150774642480426\\
2.65014308855292	-0.0137183512275043\\
2.70014578833693	-0.0131682526312357\\
2.75014848812095	-0.0117226205947694\\
2.80015118790497	-0.0103904380092325\\
2.85015388768899	-0.00922714362317916\\
2.900156587473	-0.00824569123136266\\
2.95015928725702	-0.00735734185180289\\
3.00016198704104	-0.00691141401176108\\
3.05016468682505	-0.0064314624708795\\
3.10016738660907	-0.00584883465341678\\
3.15017008639309	-0.00575186768904735\\
3.20017278617711	-0.00563135917754207\\
3.25017548596112	-0.00571034804003099\\
3.30017818574514	-0.00511706579560307\\
3.35018088552916	-0.00470346959643639\\
3.40018358531318	-0.00483495896974648\\
3.45018628509719	-0.00509385439619457\\
3.50018898488121	-0.00537774069671827\\
3.55019168466523	-0.0048240401651585\\
3.60019438444924	-0.00405979815827595\\
3.65019708423326	-0.00367144993652071\\
3.70019978401728	-0.00355107045091932\\
3.7502024838013	-0.00313041447951463\\
3.80020518358531	-0.00247219680868018\\
3.85020788336933	-0.00202998694297108\\
3.90021058315335	-0.00139888902555756\\
3.95021328293736	-0.000708429778284784\\
4.00021598272138	-0.000312869911554552\\
4.0502186825054	-0.0005790396553982\\
4.10022138228942	-0.000836049662513396\\
4.15022408207343	-0.000981789379811676\\
4.20022678185745	-0.000246379813992724\\
4.25022948164147	-6.30099922993897e-05\\
4.30023218142549	0.000445929894498322\\
4.3502348812095	0.000174450119761246\\
4.40023758099352	0.000629349867364495\\
4.45024028077754	0.00117220930130371\\
4.50024298056155	0.00143476853460119\\
4.55024568034557	0.00150141914709917\\
4.60024838012959	0.00134848852764166\\
4.65025107991361	0.00163623835515118\\
4.70025377969762	0.00206417786312853\\
4.75025647948164	0.00248130669648662\\
4.80025917926566	0.00256577538035016\\
4.85026187904968	0.00252798535382201\\
4.90026457883369	0.00221510736971197\\
4.95026727861771	0.00188237788011579\\
5.00026997840173	0.00183204847945183\\
5.05027267818575	0.00191484830919665\\
5.10027537796976	0.00229254728696066\\
5.15027807775378	0.00278156379283156\\
5.2002807775378	0.00344854110048071\\
5.25028347732181	0.00381360960967787\\
5.30028617710583	0.00357973081268452\\
5.35028887688985	0.00345752188828196\\
5.40029157667387	0.0034863097553107\\
5.45029427645788	0.00338022158927959\\
5.5002969762419	0.00356714089445493\\
5.55029967602592	0.00337307059732386\\
5.60030237580994	0.0037505896846448\\
5.65030507559395	0.00366425036547586\\
5.70030777537797	0.00363008083642352\\
5.75031047516199	0.00352586078444241\\
5.800313174946	0.00376495747464059\\
5.85031587473002	0.00409410721593653\\
5.90031857451404	0.00397720776327857\\
5.95032127429806	0.00414442508379436\\
6.00032397408207	0.00434752436859698\\
6.05032667386609	0.00469445103656204\\
6.10032937365011	0.00470349107706353\\
6.15033207343413	0.00436377464668437\\
6.20033477321814	0.00379735889525196\\
6.25033747300216	0.00334426208278469\\
6.30034017278618	0.00301890337604044\\
6.3503428725702	0.00233376605724472\\
6.40034557235421	0.00173148828123556\\
6.45034827213823	0.00121363937391702\\
6.50035097192225	0.00113083954294815\\
6.55035367170626	0.00067608976409079\\
6.60035637149028	0.000359649799037169\\
6.6503590712743	-0.000469319777677215\\
6.70036177105832	-0.000604169807233929\\
6.75036447084233	-0.00067971958597613\\
6.80036717062635	-0.000582659069966439\\
6.85036987041037	-0.000665349433361255\\
6.90037257019439	-0.00103382926105933\\
6.9503752699784	-0.00170098892628634\\
7.00037796976242	-0.00155522873620708\\
7.05038066954644	-0.00108236901358709\\
7.10038336933045	-0.000677849513434459\\
7.15038606911447	-0.000179829947778432\\
7.20038876889849	-0.000267909928984278\\
7.25039146868251	-0.000127650031092487\\
7.30039416846652	0.000348829778529024\\
7.35039686825054	0.000639989738128178\\
7.40039956803456	0.00116678927838821\\
7.45040226781857	0.00183380797875145\\
7.50040496760259	0.00240929586277716\\
7.55040766738661	0.0026700456071208\\
7.60041036717063	0.00252429669239999\\
7.65041306695464	0.00199229810402074\\
7.70041576673866	0.00181060854596582\\
7.75041846652268	0.00188062790981115\\
7.8004211663067	0.00212343738400876\\
7.85042386609071	0.00192380731266607\\
7.90042656587473	0.00128923884119827\\
7.95042926565875	7.54799206101253e-05\\
8.00043196544276	-0.000650799634913074\\
8.05043466522678	-0.00117408888928762\\
8.1004373650108	-0.00150304833092228\\
8.15044006479482	-0.0021522171344814\\
8.20044276457883	-0.00317885347681124\\
8.25044546436285	-0.00416582620039677\\
8.30044816414687	-0.00477706030430347\\
8.35045086393089	-0.00531836128852041\\
8.4004535637149	-0.00496236401241241\\
8.45045626349892	-0.00413174634732054\\
8.50045896328294	-0.00310860412400277\\
8.55046166306696	-0.00258543608967034\\
8.60046436285097	-0.00263760475627263\\
8.65046706263499	-0.00247044629153745\\
8.70046976241901	-0.00192370799300143\\
8.75047246220302	-0.000911629516475116\\
8.80047516198704	-0.000181829699903031\\
8.85047786177106	0.000900789042198345\\
8.90048056155508	0.0019095668440927\\
8.95048326133909	0.00297379445880644\\
9.00048596112311	0.00333358285258419\\
9.05048866090713	0.00390355873692718\\
9.10049136069114	0.00434209461348836\\
9.15049406047516	0.0044105032495615\\
9.20049676025918	0.00468561960531963\\
9.2504994600432	0.00498053741970767\\
9.30050215982721	0.0053994223718181\\
9.35050485961123	0.00584706400516743\\
9.40050755939525	0.00575896506734892\\
9.45051025917927	0.00546941015793189\\
9.50051295896328	0.00544436001910002\\
9.5505156587473	0.00486555954972894\\
9.60051835853132	0.00413175614455598\\
9.65052105831533	0.00342335070781257\\
9.70052375809935	0.00345754965190532\\
9.75052645788337	0.00369661016272354\\
9.80052915766739	0.00429183598888206\\
9.8505318574514	0.0045003438742042\\
9.90053455723542	0.00469107012490006\\
9.95053725701944	0.00522869424075406\\
10.0005399568035	0.00566909859974647\\
10.0505426565875	0.00589197378142339\\
10.1005453563715	0.00578413634413553\\
10.1505480561555	0.0058291750820762\\
10.2005507559395	0.00595859153565075\\
10.2505534557235	0.00591901257978793\\
10.3005561555076	0.0056367267987096\\
10.3505588552916	0.00568357658257157\\
10.4005615550756	0.00562056899694706\\
10.4505642548596	0.00578943496028602\\
10.5005669546436	0.00605552949430342\\
10.5505696544276	0.00661300648173535\\
10.6005723542117	0.0068268715316075\\
10.6505750539957	0.00700321051556172\\
10.7005777537797	0.00739513071690921\\
10.7505804535637	0.00798303342142096\\
10.8005831533477	0.00830837154789158\\
10.8505858531317	0.00813405526012492\\
10.9005885529158	0.00801889732467205\\
10.9505912526998	0.00797936297252162\\
11.0005939524838	0.00805489127720868\\
11.0505966522678	0.00793811355732946\\
11.1005993520518	0.00787696459743851\\
11.1506020518359	0.0078374366830522\\
11.2006047516199	0.0081628350039113\\
11.2506074514039	0.00772597972670861\\
11.3006101511879	0.00742744927726425\\
11.3506128509719	0.00732139194216939\\
11.4006155507559	0.00708591878622384\\
11.45061825054	0.00698338943130212\\
11.500620950324	0.00712355846792163\\
11.550623650108	0.00656806874034679\\
11.600626349892	0.00668498784198731\\
11.650629049676	0.00605758015371286\\
11.70063174946	0.0057444448252748\\
11.7506344492441	0.00553598803698523\\
11.8006371490281	0.005395769644015\\
11.8506398488121	0.00518899538249185\\
11.9006425485961	0.0053940317341587\\
11.9506452483801	0.00557559854024851\\
12.0006479481641	0.00576629435182912\\
12.0506506479482	0.00589573367379907\\
12.1006533477322	0.00606812103310179\\
12.1506560475162	0.00593505363028023\\
12.2006587473002	0.00600351233950317\\
12.2506614470842	0.00616705837377553\\
12.3006641468683	0.00579851341459548\\
12.3506668466523	0.00545147024646954\\
12.4006695464363	0.005625879084251\\
12.4506722462203	0.00584712573548209\\
12.5006749460043	0.00574634603964122\\
12.5506776457883	0.00552335895131969\\
12.6006803455724	0.00538859074619017\\
12.6506830453564	0.00503785539425297\\
12.7006857451404	0.00503243801140823\\
12.7506884449244	0.00468189101033363\\
12.8006911447084	0.00430082299806487\\
12.8506938444924	0.00350421032530727\\
12.9006965442765	0.0024614864107951\\
12.9506992440605	0.00234460741733558\\
13.0007019438445	0.00285901489311952\\
13.0507046436285	0.00261620546990234\\
13.1007073434125	0.00268442514492828\\
13.1507100431965	0.00264306516826795\\
13.2007127429806	0.00230872720458827\\
13.2507154427646	0.0021827977121206\\
13.3007181425486	0.00243816698895168\\
13.3507208423326	0.00209289777212395\\
13.4007235421166	0.00233740635218109\\
13.4507262419006	0.00280305366232335\\
13.5007289416847	0.00277067529286448\\
13.5507316414687	0.00275458573435396\\
13.6007343412527	0.00283182547837866\\
13.6507370410367	0.0030638337666442\\
13.7007397408207	0.00323283219434706\\
13.7507424406048	0.00343057140178159\\
13.8007451403888	0.0031969534056526\\
13.8507478401728	0.00319861301619651\\
13.9007505399568	0.00300266421901345\\
13.9507532397408	0.00349168123155314\\
14.0007559395248	0.00376333873079369\\
14.0507586393089	0.00423608383689138\\
14.1007613390929	0.00472153760741455\\
14.1507640388769	0.00462442174792232\\
14.2007667386609	0.00448792325658788\\
14.2507694384449	0.0045004142193252\\
14.3007721382289	0.00475033028788265\\
14.350774838013	0.00511162397542966\\
14.400777537797	0.00503249685014477\\
14.450780237581	0.00418400617367782\\
14.500782937365	0.00336227124020106\\
14.550785637149	0.00272585462142152\\
14.600788336933	0.00223324679498001\\
14.6507910367171	0.00165782884638781\\
14.7007937365011	0.00165960857889543\\
14.7507964362851	0.000891809273292958\\
14.8007991360691	0.000503449821803863\\
14.8508018358531	-7.72598263329883e-05\\
14.9008045356372	-0.000524989729380052\\
14.9508072354212	-0.000350539489990333\\
15.0008099352052	-0.000242639592597299\\
15.0508126349892	0.000131270023620464\\
15.1008153347732	9.00999778047551e-06\\
15.1508180345572	-0.000213869846729365\\
15.2008207343413	-0.000381029714717765\\
15.2508234341253	0.000187029856140875\\
15.3008261339093	0.000985239164585737\\
15.3508288336933	0.00140957898752667\\
15.4008315334773	0.00147436905167427\\
15.4508342332613	0.00127646918384729\\
15.5008369330454	0.000679649669343693\\
15.5508396328294	-0.000183399887470766\\
15.6008423326134	-0.000359479889720392\\
15.6508450323974	-0.00032535959877671\\
15.7008477321814	-0.000523129613549287\\
15.7508504319654	-0.000783859553213259\\
15.8008531317495	-0.000789299707794974\\
15.8508558315335	-0.0010770294225889\\
15.9008585313175	-0.00144031864304545\\
15.9508612311015	-0.00185015790468893\\
16.0008639308855	-0.0020892474356563\\
16.0508666306695	-0.00203706797581211\\
16.1008693304536	-0.00175489868893561\\
16.1508720302376	-0.00142232869665297\\
16.2008747300216	-0.00133941854281566\\
16.2508774298056	-0.00148692834626524\\
16.3008801295896	-0.000890019669062207\\
16.3508828293737	-6.8370009657087e-05\\
16.4008855291577	0.00105725910850375\\
16.4508882289417	0.00168650850110484\\
16.5008909287257	0.00248124652279106\\
16.5508936285097	0.00268616607250207\\
16.6008963282937	0.00232291672702059\\
16.6508990280778	0.00225649710095432\\
16.7009017278618	0.00197054781448886\\
16.7509044276458	0.00184827845682872\\
16.8009071274298	0.00221687740795353\\
16.8509098272138	0.00213964702713628\\
16.9009125269978	0.00194718748019252\\
16.9509152267819	0.00184115847517189\\
17.0009179265659	0.00220450786234546\\
17.0509206263499	0.00254062624565287\\
17.1009233261339	0.00237705640968658\\
17.1509260259179	0.00266475514939011\\
17.2009287257019	0.00317537289540647\\
17.250931425486	0.00359416046776427\\
17.30093412527	0.00425771425306453\\
17.350936825054	0.00480415935936595\\
17.400939524838	0.00509197552660602\\
17.450942224622	0.00579136319276883\\
17.500944924406	0.0060322496742723\\
17.5509476241901	0.00591002400454603\\
17.6009503239741	0.00608798036753664\\
17.6509530237581	0.00641518345200116\\
17.7009557235421	0.00635413528889422\\
17.7509584233261	0.00613656914244406\\
17.8009611231102	0.006064571266275\\
17.8509638228942	0.00641335298189671\\
17.9009665226782	0.00658421083202115\\
17.9509692224622	0.00639913420711237\\
18.0009719222462	0.00590279374376315\\
18.0509746220302	0.00550003134815767\\
18.1009773218143	0.00530760269749266\\
18.1509800215983	0.00540107205864101\\
18.2009827213823	0.00559540596778667\\
18.2509854211663	0.00585971528192784\\
18.3009881209503	0.00592805169032435\\
18.3509908207343	0.00612030957310163\\
18.4009935205184	0.00620305797447331\\
18.4509962203024	0.00616359983378497\\
18.5009989200864	0.00570686372800143\\
18.5510016198704	0.00566191800160119\\
18.6010043196544	0.0052393436129763\\
18.6510070194384	0.00389245802818534\\
18.7010097192225	0.00317890120858389\\
18.7510124190065	0.0031464731119914\\
18.8010151187905	0.00283908459484358\\
18.8510178185745	0.00238783695361686\\
18.9010205183585	0.0026700062130619\\
18.9510232181425	0.00294156337315796\\
19.0010259179266	0.00334059319065943\\
19.0510286177106	0.00361032996086967\\
19.1010313174946	0.00394298696075862\\
19.1510340172786	0.00443748170465587\\
19.2010367170626	0.00424872462478818\\
19.2510394168467	0.00418217586376731\\
19.3010421166307	0.00448230419781984\\
19.3510448164147	0.00432765350822959\\
19.4010475161987	0.00418394609104758\\
19.4510502159827	0.0045057941798455\\
19.5010529157667	0.00486891723413525\\
19.5510556155508	0.00500374545666346\\
19.6010583153348	0.00508652731719281\\
19.6510610151188	0.00515840628204952\\
19.7010637149028	0.00532025234140701\\
19.7510664146868	0.00539558075737698\\
19.8010691144708	0.00571216671417141\\
19.8510718142549	0.0057804651641632\\
19.9010745140389	0.00538855912060207\\
19.9510772138229	0.00591895272869695\\
20.0010799136069	0.00594415387329483\\
20.0510826133909	0.00593514434938282\\
20.1010853131749	0.00569951505860091\\
20.151088012959	0.00604298666088811\\
20.201090712743	0.00586682504438395\\
20.251093412527	0.00545688235042049\\
20.301096112311	0.00517989625426309\\
20.351098812095	0.00530579272019762\\
20.4011015118791	0.00556477176752429\\
20.4511042116631	0.00552887989637731\\
20.5011069114471	0.00564928846398874\\
20.5511096112311	0.00527721051564733\\
20.6011123110151	0.00492290827400756\\
20.6511150107991	0.00480236039841602\\
20.7011177105832	0.00484381911603742\\
20.7511204103672	0.00517285921232203\\
20.8011231101512	0.00485642666793257\\
20.8511258099352	0.00454349227841524\\
20.9011285097192	0.00469630905831151\\
20.9511312095032	0.00431341548690792\\
21.0011339092873	0.00313386332241982\\
21.0511366090713	0.00230503700012473\\
21.1011393088553	0.00199932797195455\\
21.1511420086393	0.00252246684826434\\
21.2011447084233	0.00280296510736208\\
21.2511474082073	0.0020533269712745\\
21.3011501079914	0.00190418829130572\\
21.3511528077754	0.00165949875233573\\
21.4011555075594	0.00147428921674325\\
21.4511582073434	0.00148868884403736\\
21.5011609071274	0.00106613957271884\\
21.5511636069114	0.000719289688580044\\
21.6011663066955	0.000605909772445469\\
21.6511690064795	0.000185270072122915\\
21.7011717062635	-4.49099890778777e-05\\
21.7511744060475	-0.000634779446769759\\
21.8011771058315	-0.00176023783081997\\
21.8511798056156	-0.00293267437337592\\
21.9011825053996	-0.00377221723904167\\
21.9511852051836	-0.00392322776549649\\
22.0011879049676	-0.00446439887302389\\
22.0511906047516	-0.00562047359753456\\
22.1011933045356	-0.00668668864518672\\
22.1511960043197	-0.00756234226152629\\
22.2011987041037	-0.00842347693920958\\
22.2512014038877	-0.00948788368421323\\
22.3012041036717	-0.00959768103822648\\
22.3512068034557	-0.00950410401824841\\
22.4012095032397	-0.00864279957339023\\
22.4512122030238	-0.00777082862433696\\
22.5012149028078	-0.00694188627691208\\
22.5512176025918	-0.00667058764974037\\
22.6012203023758	-0.00594237331347719\\
22.6512230021598	-0.00569424725881398\\
22.7012257019438	-0.0053166225840675\\
22.7512284017279	-0.00435273282540516\\
22.8012311015119	-0.00330650142214388\\
22.8512338012959	-0.00249736673089154\\
22.9012365010799	-0.00231208694931201\\
22.9512392008639	-0.00219708765909093\\
23.0012419006479	-0.0017601987147139\\
23.051244600432	-0.00180878676022762\\
23.101247300216	-0.00138446879408728\\
23.15125	-0.000683279811616574\\
23.201252699784	-0.000426209596171138\\
23.251255399568	-0.000922379206049305\\
23.3012580993521	-0.000904539643661433\\
23.3512607991361	-0.000703059733203253\\
23.4012634989201	0.000291279830132503\\
23.4512661987041	0.000293009994653618\\
23.5012688984881	0.000183499837065832\\
23.5512715982721	0.000199669869154975\\
23.6012742980562	0.000330829907350269\\
23.6512769978402	0.000877299291191873\\
23.7012796976242	0.000882659655705361\\
23.7512823974082	0.000339829715217881\\
23.8012850971922	0.000323579872628062\\
23.8512877969762	0.000560989592868841\\
23.9012904967603	0.00133424928494087\\
23.9512931965443	0.00199208768806045\\
24.0012958963283	0.00304746347917761\\
24.0512985961123	0.00343963177213336\\
24.1013012958963	0.00338388182503185\\
24.1513039956803	0.00302422433184002\\
24.2013066954644	0.00279405551608377\\
24.2513093952484	0.00304029454027128\\
24.3013120950324	0.00327766276802533\\
24.3513147948164	0.00339289216229332\\
24.4013174946004	0.00350793142998796\\
24.4513201943845	0.0029900336611875\\
24.5013228941685	0.00274912550788514\\
24.5513255939525	0.00234646628786219\\
24.6013282937365	0.00222762717366508\\
24.6513309935205	0.00245069656379573\\
24.7013336933045	0.00272392556471188\\
24.7513363930886	0.00287312434467878\\
24.8013390928726	0.00364467993278136\\
24.8513417926566	0.00426846482277529\\
24.9013444924406	0.00425949406098344\\
24.9513471922246	0.00431334501480859\\
25.0013498920086	0.00439605413842229\\
25.0513525917927	0.00491204734217163\\
25.1013552915767	0.00524463370145238\\
25.1513579913607	0.00582552436931003\\
25.2013606911447	0.00567985717112449\\
25.2513633909287	0.00601967972365235\\
25.3013660907127	0.00603233176534936\\
25.3513687904968	0.00574471711316518\\
25.4013714902808	0.00558992622889527\\
25.4513741900648	0.00538133065885374\\
25.5013768898488	0.00502899628228773\\
25.5513795896328	0.00533467153965397\\
25.6013822894168	0.00542623111119736\\
25.6513849892009	0.00538315287928594\\
25.7013876889849	0.0054929397081371\\
25.7513903887689	0.00533828104981426\\
25.8013930885529	0.00513690557329559\\
25.8513957883369	0.00527888260450313\\
25.901398488121	0.00542457057739379\\
25.951401187905	0.00446086147071748\\
26.001403887689	0.00294862249540413\\
26.051406587473	0.00242372585260081\\
26.101409287257	0.00212707633642342\\
26.151411987041	0.00194183728102899\\
26.2014146868251	0.00255850590450707\\
26.2514173866091	0.00309269313183119\\
26.3014200863931	0.00300089453768856\\
26.3514227861771	0.00286612542595932\\
26.4014254859611	0.00237698688890595\\
26.4514281857451	0.00149958874320624\\
26.5014308855292	0.000458579703979448\\
26.5514335853132	0.000214039696794486\\
26.6014362850972	0.000550079369976169\\
26.6514389848812	0.000562609641191814\\
26.7014416846652	0.00037754985172562\\
26.7514443844492	0.000235649862124914\\
26.8014470842333	0.000427969665200734\\
26.8514497840173	0.000845139576806593\\
26.9014524838013	0.00123508888908566\\
26.9514551835853	0.00133940848276814\\
27.0014578833693	0.00112195872085287\\
27.0514605831534	0.000854009321760929\\
27.1014632829374	0.00110230915982468\\
27.1514659827214	0.00161652876177592\\
27.2014686825054	0.00227620723155335\\
27.2514713822894	0.00249744600693547\\
27.3014740820734	0.00243101613316888\\
27.3514767818575	0.00252445588599725\\
27.4014794816415	0.00274185542076253\\
27.4514821814255	0.00320938252485714\\
27.5014848812095	0.00343243152494549\\
27.5514875809935	0.0033281820415451\\
27.6014902807775	0.0033676021510743\\
27.6514929805616	0.00383869920534139\\
27.7014956803456	0.00519795415252214\\
27.7514983801296	0.00604481093774999\\
27.8015010799136	0.00632357652892712\\
27.8515037796976	0.00637205482835713\\
27.9015064794816	0.00639906393372067\\
27.9515091792657	0.00618496788575335\\
28.0015118790497	0.00605011039369309\\
28.0515145788337	0.00590102358395891\\
28.1015172786177	0.00539577070385783\\
28.1515199784017	0.00494093748665625\\
28.2015226781857	0.00485271951563599\\
28.2515253779698	0.00494633824803935\\
28.3015280777538	0.00509017649197831\\
28.3515307775378	0.0048707888261675\\
28.4015334773218	0.00417305663539301\\
28.4515361771058	0.00346645126854507\\
28.5015388768899	0.00277430430457159\\
28.5515415766739	0.00219903654294797\\
28.6015442764579	0.0013628586944874\\
28.6515469762419	0.000609399738529948\\
28.7015496760259	0.000424409897677328\\
28.7515523758099	0.000257199926784578\\
28.801555075594	-0.000300269713257299\\
28.851557775378	-0.000807279357423429\\
28.901560475162	-0.00147437882891631\\
28.951563174946	-0.00206219756514822\\
29.00156587473	-0.00242730643745416\\
29.051568574514	-0.00304584352863491\\
29.1015712742981	-0.00336942223407156\\
29.1515739740821	-0.00356720042703173\\
29.2015766738661	-0.00369660917123532\\
29.2515793736501	-0.00409044549999944\\
29.3015820734341	-0.00389077887153352\\
29.3515847732181	-0.00391068833624874\\
29.4015874730022	-0.00457968112662776\\
29.4515901727862	-0.00555944763101826\\
29.5015928725702	-0.0055683483910907\\
29.5515955723542	-0.00528610144531621\\
29.6015982721382	-0.00494642733178722\\
29.6516009719222	-0.00434203406009637\\
29.7016036717063	-0.00401487669103709\\
29.7516063714903	-0.00369494954444185\\
29.8016090712743	-0.00344680167686239\\
29.8516117710583	-0.00285152434063921\\
29.9016144708423	-0.00192738709972887\\
29.9516171706264	-0.0015462190395103\\
30.0016198704104	-0.00208903667733709\\
30.0516225701944	-0.00188240829751046\\
30.1016252699784	-0.00164520795980134\\
30.1516279697624	-0.00166853848167276\\
30.2016306695464	-0.00180690836465789\\
30.2516333693305	-0.00209106811928193\\
30.3016360691145	-0.00164347866771455\\
30.3516387688985	-0.00107523931664185\\
30.4016414686825	-0.000661679621226868\\
30.4516441684665	0.000224689915381493\\
30.5016468682505	0.000598779899376188\\
30.5516495680346	0.000724669758866384\\
30.6016522678186	0.000738919666674803\\
30.6516549676026	0.000496279637794361\\
30.7016576673866	0.000773259692716715\\
30.7516603671706	0.00108968933440783\\
30.8016630669546	0.00146538886216005\\
30.8516657667387	0.00137005901465024\\
30.9016684665227	0.00127660939645029\\
30.9516711663067	0.00172248855634045\\
31.0016738660907	0.00229426705440814\\
31.0516765658747	0.00248131679196432\\
31.1016792656587	0.00233740666053548\\
31.1516819654428	0.00226543748952018\\
31.2016846652268	0.0015731583037256\\
31.2516873650108	0.00119392929768067\\
31.3016900647948	0.000787459672460606\\
31.3516927645788	0.00099778936737817\\
31.4016954643629	0.00200842757799059\\
31.4516981641469	0.00224397723492307\\
31.5017008639309	0.00273642559510133\\
31.5517035637149	0.00327953303227104\\
31.6017062634989	0.0033982419296411\\
31.6517089632829	0.00432236363166546\\
31.701711663067	0.00492645716808901\\
31.751714362851	0.00577515532821822\\
31.801717062635	0.00645297169425284\\
31.851719762419	0.0074238991021229\\
31.901722462203	0.00821492380230794\\
31.951725161987	0.00870037637713264\\
32.0017278617711	0.0089466457051069\\
32.0517305615551	0.00927034303882632\\
32.1017332613391	0.00942674730853918\\
32.1517359611231	0.00950397299057843\\
32.2017386609071	0.00901486351539632\\
32.2517413606911	0.0085601513426075\\
32.3017440604752	0.0082276013610388\\
32.3517467602592	0.00794348029564186\\
32.4017494600432	0.00788971330447455\\
32.4517521598272	0.00734668058175393\\
32.5017548596112	0.00667606701884247\\
32.5517575593952	0.00589210217084455\\
32.6017602591793	0.00547140024737316\\
32.6517629589633	0.00522669404171772\\
32.7017656587473	0.00492651539021049\\
32.7517683585313	0.0042180651981794\\
32.8017710583153	0.00388183884366414\\
32.8517737580993	0.00379196929803827\\
32.9017764578834	0.00387828916932585\\
32.9517791576674	0.00433307405089926\\
33.0017818574514	0.0048526288558031\\
33.0517845572354	0.00520522355454912\\
33.1017872570194	0.00517638545550738\\
33.1517899568035	0.00520337364677479\\
33.2017926565875	0.00446444246368493\\
33.2517953563715	0.00404539708232453\\
33.3017980561555	0.0032417523116352\\
33.3518007559395	0.00330108317583371\\
33.4018034557235	0.00292892451958916\\
33.4518061555076	0.00238063661839234\\
33.5018088552916	0.00222242752488762\\
33.5518115550756	0.00201735793342532\\
33.6018142548596	0.00242012549989068\\
33.6518169546436	0.00208747721211707\\
33.7018196544277	0.00153909863093976\\
33.7518223542117	0.0012351589310057\\
33.8018250539957	0.000526670060462438\\
33.8518277537797	0.000640149741576534\\
33.9018304535637	0.00090620950017878\\
33.9518331533477	0.00139886917636172\\
34.0018358531318	0.00180696827719675\\
34.0518385529158	0.00235884698452354\\
34.1018412526998	0.00238757680239016\\
34.1518439524838	0.00281214485910528\\
34.2018466522678	0.00271138573153159\\
34.2518493520518	0.00290553470997876\\
34.3018520518359	0.00302783421751983\\
34.3518547516199	0.00357968144832602\\
34.4018574514039	0.00433315272128077\\
34.4518601511879	0.00491932620099826\\
34.5018628509719	0.00490491936017063\\
34.5518655507559	0.00471797025052832\\
34.60186825054	0.00496784784030365\\
34.651870950324	0.00512426575760255\\
34.701873650108	0.00540302101685182\\
34.751876349892	0.00511719487777197\\
34.801879049676	0.00530969294837072\\
34.85188174946	0.00512424570014062\\
34.9018844492441	0.00481493047084053\\
34.9518871490281	0.00505949704861202\\
35.0018898488121	0.00543526173109744\\
35.0518925485961	0.00566374789012745\\
35.1018952483801	0.00591905410253261\\
35.1518979481641	0.00580752530239062\\
35.2019006479482	0.00595328317091551\\
35.2519033477322	0.0062642769004534\\
35.3019060475162	0.00648724067393887\\
35.3519087473002	0.00639002294669871\\
35.4019114470842	0.00619579773228755\\
35.4519141468683	0.00621746646880665\\
35.5019168466523	0.00631642569470169\\
35.5519195464363	0.00613473801544346\\
35.6019222462203	0.00598012124262992\\
35.6519249460043	0.00601790050461425\\
35.7019276457883	0.00604495165217911\\
35.7519303455724	0.00597649304155158\\
35.8019330453564	0.00583632559660245\\
35.8519357451404	0.00573919489269142\\
35.9019384449244	0.00558110801468417\\
35.9519411447084	0.00534008234113301\\
36.0019438444924	0.00549458078840487\\
36.0519465442765	0.0056888674134802\\
36.1019492440605	0.00590288198363911\\
36.1519519438445	0.00575708511573964\\
36.2019546436285	0.00587777304969314\\
36.2519573434125	0.00590830377927248\\
36.3019600431965	0.00595141198642355\\
36.3519627429806	0.00590285152312133\\
36.4019654427646	0.00555046935881876\\
36.4519681425486	0.00521422526520973\\
36.5019708423326	0.00496791807295861\\
36.5519735421166	0.0045326223762739\\
36.6019762419006	0.00424491525572717\\
36.6519789416847	0.00334608173831331\\
36.7019816414687	0.00309965385426933\\
36.7519843412527	0.00261598495283963\\
36.8019870410367	0.00237695678537874\\
36.8519897408207	0.00262503582656922\\
36.9019924406048	0.00253864615488404\\
36.9519951403888	0.00228880717450668\\
37.0019978401728	0.00142751835636407\\
37.0520005399568	0.0003757201082467\\
37.1020032397408	5.75202021288718e-05\\
37.1520059395248	0.000384709809858219\\
37.2020086393089	0.000693959637150849\\
37.2520113390929	0.00122252920905154\\
37.3020140388769	0.00169184790198186\\
37.3520167386609	0.00179806827677754\\
37.4020194384449	0.00192208834049123\\
37.4520221382289	0.0018986682249785\\
37.502024838013	0.00126213895347824\\
37.552027537797	0.000356029868759906\\
37.602030237581	0.000438529513249345\\
37.652032937365	0.000449649749579412\\
37.702035637149	0.000886429507537473\\
37.752038336933	0.00130895867259276\\
37.8020410367171	0.00198507780632461\\
37.8520437365011	0.00234274643857579\\
37.9020464362851	0.00278506437823052\\
37.9520491360691	0.00326860236335206\\
38.0020518358531	0.00379371949994972\\
38.0520545356372	0.00369478929009691\\
38.1020572354212	0.00352582990986555\\
38.1520599352052	0.00375593894846816\\
38.2020626349892	0.00419286568241482\\
38.2520653347732	0.00461532150735514\\
38.3020680345572	0.00433489392248629\\
38.3520707343413	0.00456151308378561\\
38.4020734341253	0.00469088053628568\\
38.4520761339093	0.00444277250122623\\
38.5020788336933	0.00443743371814253\\
38.5520815334773	0.00459398125886192\\
38.6020842332613	0.00514222587576821\\
38.6520869330454	0.00477718030834715\\
38.7020896328294	0.00442306361084213\\
38.7520923326134	0.00467309062272553\\
38.8020950323974	0.00490486770907312\\
38.8520977321814	0.00480419918584166\\
38.9021004319654	0.00456693245711129\\
38.9521031317495	0.00471970034402126\\
39.0021058315335	0.00443915103149122\\
39.0521085313175	0.00459197128609839\\
39.1021112311015	0.00471787996161244\\
39.1521139308855	0.0046747610728386\\
39.2021166306696	0.00471435000436613\\
39.2521193304536	0.00401149688417061\\
39.3021220302376	0.00425594534659075\\
39.3521247300216	0.00389089701142924\\
39.4021274298056	0.00322913327345712\\
39.4521301295896	0.00265206543182994\\
39.5021328293737	0.00147975836109475\\
39.5521355291577	0.000251629811654192\\
39.6021382289417	-0.000195959896192169\\
39.6521409287257	-0.000508939605322318\\
39.7021436285097	-0.000794769295509856\\
39.7521463282937	-0.00109690892743403\\
39.8021490280778	-0.000618519511049038\\
39.8521517278618	-0.000373959801902219\\
39.9021544276458	0.000341629886748533\\
39.9521571274298	0.0015084487992411\\
40.0021598272138	0.00201369789093214\\
40.0521625269978	0.00221148762521484\\
40.1021652267819	0.00236620695748794\\
40.1521679265659	0.00216486756244311\\
40.2021706263499	0.00175126853189745\\
40.2521733261339	0.00172249782426752\\
40.3021760259179	0.00251180601809696\\
40.3521787257019	0.003137403298325\\
40.402181425486	0.00351323094244012\\
40.45218412527	0.00310880381685368\\
40.502186825054	0.00300264396792672\\
40.552189524838	0.00280841529006896\\
40.602192224622	0.00265751564958432\\
40.652194924406	0.00280850529174357\\
40.7021976241901	0.00285356538104834\\
40.7522003239741	0.00305294340439655\\
40.8022030237581	0.00285875509664806\\
40.8522057235421	0.00214331697248389\\
40.9022084233261	0.00129272876301639\\
40.9522111231102	0.00143475891559517\\
41.0022138228942	0.00200474739997376\\
41.0522165226782	0.00270767428946571\\
41.1022192224622	0.00294687302766485\\
41.1522219222462	0.00246319656078214\\
41.2022246220302	0.00187699806489323\\
41.2522273218143	0.0017888983648984\\
41.3022300215983	0.00143126902524554\\
41.3522327213823	0.000769619582045679\\
41.4022354211663	0.000237249876907835\\
41.4522381209503	-0.000271599753943099\\
41.5022408207343	-0.000649229505605303\\
41.5522435205184	-0.000127719768687107\\
41.6022462203024	0.000683219626375215\\
41.6522489200864	0.00128566892514395\\
41.7022516198704	0.00148513855018477\\
41.7522543196544	0.00176395790543268\\
41.8022570194385	0.00302055329641319\\
41.8522597192225	0.0038064287574006\\
41.9022624190065	0.00413356671983891\\
41.9522651187905	0.00368573894300746\\
42.0022678185745	0.00357445035956827\\
42.0522705183585	0.00386024946583087\\
42.1022732181426	0.00422526609184498\\
42.1522759179266	0.00457776281849368\\
42.2022786177106	0.0043638137093383\\
42.2522813174946	0.00454349276497787\\
42.3022840172786	0.00451663390717768\\
42.3522867170626	0.00444112130882181\\
42.4022894168467	0.00416953536806037\\
42.4522921166307	0.00466582111658377\\
42.5022948164147	0.00450770192757692\\
42.5522975161987	0.00433500458469667\\
42.6023002159827	0.00428276462297503\\
42.6523029157667	0.00388724781381283\\
42.7023056155508	0.00337306249996738\\
42.7523083153348	0.00276715545650488\\
42.8023110151188	0.00262315652112114\\
42.8523137149028	0.00293435382815401\\
42.9023164146868	0.00254779535671025\\
42.9523191144708	0.00235898696223635\\
43.0023218142549	0.00186078830750413\\
43.0523245140389	0.00128192916081836\\
43.1023272138229	0.0006508696609716\\
43.1523299136069	-0.000582509590815581\\
43.2023326133909	-0.0018248774012112\\
43.2523353131749	-0.00226916667362878\\
43.302338012959	-0.00278865373355642\\
43.352340712743	-0.00279585459994314\\
43.402343412527	-0.0032885824522709\\
43.452346112311	-0.00396999868065576\\
43.502348812095	-0.00437817383077025\\
43.5523515118791	-0.00423973618199759\\
43.6023542116631	-0.00408136703572331\\
43.6523569114471	-0.00336220223904603\\
43.7023596112311	-0.00351497065544172\\
43.7523623110151	-0.00337489255934782\\
43.8023650107991	-0.00363746065500374\\
43.8523677105832	-0.00337667161880986\\
43.9023704103672	-0.00326692266930452\\
43.9523731101512	-0.00357073113031836\\
44.0023758099352	-0.00311583306423877\\
44.0523785097192	-0.00268971583403272\\
44.1023812095032	-0.00257650581368028\\
44.1523839092873	-0.00292723440316876\\
44.2023866090713	-0.00281032574431531\\
44.2523893088553	-0.0027112950860048\\
44.3023920086393	-0.00293085451976324\\
44.3523947084233	-0.00337111189129592\\
44.4023974082073	-0.00352588037371459\\
44.4524001079914	-0.00404735663034087\\
44.5024028077754	-0.00330119203658418\\
44.5524055075594	-0.00239498619890092\\
44.6024082073434	-0.00182318822435234\\
44.6524109071274	-0.0016327186619111\\
44.7024136069114	-0.00154450897065468\\
44.7524163066955	-0.00100499952045453\\
44.8024190064795	-0.000591479774582077\\
44.8524217062635	-0.000424429839627637\\
44.9024244060475	-0.000363209716596425\\
44.9524271058315	-0.000264309720565549\\
45.0024298056155	-3.60698305887373e-05\\
45.0524325053996	0.000188700094289764\\
45.1024352051836	0.000176160282847857\\
45.1524379049676	0.000507079858725142\\
45.2024406047516	0.00025539989767019\\
45.2524433045356	0.000361489825815174\\
45.3024460043197	0.000476629871251091\\
45.3524487041037	0.000620299735768219\\
45.4024514038877	0.000632889640619591\\
45.4524541036717	0.000433339968929961\\
45.5024568034557	0.000190569984372465\\
45.5524595032397	-0.000133069933037499\\
45.6024622030238	5.57802689794284e-05\\
45.6524649028078	0.000744339485824126\\
45.7024676025918	0.00100868948637548\\
45.7524703023758	0.00126219910546192\\
45.8024730021598	0.0016721476023496\\
45.8524757019439	0.00148343854339895\\
45.9024784017279	0.000924289520510235\\
45.9524811015119	0.000226419996922166\\
46.0024838012959	6.11400047985896e-05\\
46.0524865010799	-7.55499949421792e-05\\
46.1024892008639	0.00013316998789784\\
46.152491900648	-0.000352339658275781\\
46.202494600432	-0.000474599764417297\\
46.252497300216	-0.00074621931414939\\
46.3025	-0.000879259477395778\\
46.352502699784	-0.000658109858804642\\
46.402505399568	-0.000593219186858262\\
46.4525080993521	-0.000920599231488993\\
46.5025107991361	-0.00113992946063732\\
46.5525134989201	-0.0011236692959578\\
46.6025161987041	-0.00068867959171149\\
46.6525188984881	-0.000535759779240732\\
46.7025215982721	-6.30801608565412e-05\\
46.7525242980562	-7.91399981448291e-05\\
46.8025269978402	-0.00033810964329721\\
46.8525296976242	-0.000600489652576696\\
46.9025323974082	-0.00124602908456639\\
46.9525350971922	-0.00189149793775259\\
47.0025377969762	-0.00201906783278563\\
47.0525404967603	-0.00210016748770153\\
47.1025431965443	-0.00227435698662607\\
47.1525458963283	-0.00259089534395604\\
47.2025485961123	-0.00307990301636289\\
47.2525512958963	-0.00293450481358719\\
47.3025539956803	-0.00323996250365947\\
47.3525566954644	-0.00352579036346967\\
47.4025593952484	-0.00439960356811501\\
47.4525620950324	-0.0045058513960571\\
47.5025647948164	-0.00433492415988929\\
47.5525674946004	-0.00444105340959154\\
47.6025701943845	-0.00409048697860514\\
47.6525728941685	-0.00334429131692292\\
47.7025755939525	-0.00309435393979714\\
47.7525782937365	-0.00230316641359661\\
47.8025809935205	-0.00127289906527233\\
47.8525836933045	-0.000674209783795028\\
47.9025863930886	-1.43500385224513e-05\\
47.9525890928726	0.000548359842153551\\
48.0025917926566	0.0010967493762624\\
48.0525944924406	0.00171708765235766\\
48.1025971922246	0.00230311651108682\\
48.1525998920086	0.00292723418306\\
48.2026025917927	0.00389445851159452\\
48.2526052915767	0.00453987222823123\\
48.3026079913607	0.00459211215953575\\
48.3526106911447	0.00442126353425929\\
48.4026133909287	0.0048653394910996\\
48.4526160907127	0.00494633766804388\\
48.5026187904968	0.00502708705030539\\
48.5526214902808	0.0046747206174938\\
48.6026241900648	0.00442130486781359\\
48.6526268898488	0.00413008564644757\\
48.7026295896328	0.00421450427938129\\
48.7526322894168	0.00449678219488376\\
48.8026349892009	0.00471438962464723\\
48.8526376889849	0.00428818556102667\\
48.9026403887689	0.00416593589708221\\
48.9526430885529	0.00383509717116442\\
49.0026457883369	0.00307453374203581\\
49.052648488121	0.00218451707970831\\
49.102651187905	0.00214313691955187\\
49.152653887689	0.00211085727990739\\
49.202656587473	0.00201914748782639\\
49.252659287257	0.00187535706382901\\
49.302661987041	0.000960089537344542\\
49.3526646868251	0.00115785921459375\\
49.4026673866091	0.0017782683235262\\
49.4526700863931	0.00162898853573733\\
49.5026727861771	0.00169179854438567\\
49.5526754859611	0.00168833794459515\\
49.6026781857451	0.0017889571952132\\
49.6526808855292	0.00165040867697787\\
49.7026835853132	0.00194722806751218\\
49.7526862850972	0.00211628692888515\\
49.8026889848812	0.00171522838953027\\
49.8526916846652	0.00194180735960732\\
49.9026943844492	0.00174578832698388\\
49.9526970842333	0.00221501750009263\\
50.0026997840173	0.00219347745321166\\
50.0527024838013	0.00222415708389657\\
50.1027051835853	0.00175305834118241\\
50.1527078833693	0.00114002951328118\\
50.2027105831534	0.000843169408893043\\
50.2527132829374	0.000647179529573017\\
50.3027159827214	0.00134115851637777\\
50.3527186825054	0.00198855791893899\\
50.4027213822894	0.00216669643758935\\
50.4527240820734	0.00252440498362477\\
50.5027267818575	0.00259627604106286\\
50.5527294816415	0.00224392730385893\\
50.6027321814255	0.001884448080172\\
50.6527348812095	0.00110927936035129\\
50.7027375809935	0.0014690290298229\\
50.7527402807775	0.00125861900930967\\
50.8027429805616	0.000897399358710261\\
50.8527456803456	0.000490919775444955\\
50.9027483801296	7.00799181265671e-05\\
50.9527510799136	-0.000510569831979365\\
51.0027537796976	-0.000631129729216842\\
51.0527564794816	-0.000600579690294316\\
51.1027591792657	-0.000348719656193978\\
51.1527618790497	-0.000327179476639057\\
51.2027645788337	-0.000221189907494259\\
51.2527672786177	-0.000151029998258198\\
51.3027699784017	-0.000129409950710511\\
51.3527726781858	-0.000264249938517082\\
51.4027753779698	-0.000233829978992817\\
51.4527780777538	0.000154569961317225\\
51.5027807775378	0.000332540004261411\\
51.5527834773218	6.47201584153381e-05\\
51.6027861771058	0.000177979952722506\\
51.6527888768899	0.000231919916805988\\
51.7027915766739	0.000280469925258247\\
51.7527942764579	0.000239099971425318\\
51.8027969762419	0.000602249595044983\\
51.8527996760259	0.00070300952998984\\
51.9028023758099	0.000348879852118325\\
51.952805075594	3.41200151886253e-05\\
52.002807775378	-0.000170919989100204\\
52.052810475162	-5.2190005358168e-05\\
52.102813174946	0.000474719858712922\\
52.15281587473	0.00111285935688178\\
52.202818574514	0.00163623845600037\\
52.2528212742981	0.00236976668816891\\
52.3028239740821	0.00227454682930199\\
52.3528266738661	0.00218441689787237\\
52.4028293736501	0.00205505740265017\\
52.4528320734341	0.00232676685205987\\
52.5028347732181	0.00283374471294499\\
52.5528374730022	0.00338213090457357\\
52.6028401727862	0.00347010096198404\\
52.6528428725702	0.00308363367616927\\
52.7028455723542	0.00330110178975981\\
52.7528482721382	0.00345581129041286\\
52.8028509719222	0.00290540309004659\\
52.8528536717063	0.00266814499704332\\
52.9028563714903	0.00281565500927903\\
52.9528590712743	0.00208384690586102\\
53.0028617710583	0.00133410872375079\\
53.0528644708423	0.00116699886135287\\
53.1028671706264	0.000284220353132462\\
53.1528698704104	0.000181679927191552\\
53.2028725701944	-0.000203129701990933\\
53.2528752699784	0.000320109924441037\\
53.3028779697624	-8.99198129256683e-05\\
53.3528806695464	-0.00049986963067135\\
53.4028833693305	-0.000920459536657\\
53.4528860691145	-0.000640039823458378\\
53.5028887688985	-0.000237349937412331\\
53.5528914686825	0.000255409876868627\\
53.6028941684665	0.000706639600517943\\
53.6528968682505	0.00133583905157533\\
53.7028995680346	0.0014815977191148\\
53.7529022678186	0.00178550788763731\\
53.8029049676026	0.00153726826925826\\
53.8529076673866	0.000285819692207322\\
53.9029103671706	-0.000426129737918173\\
53.9529130669547	-0.000881039637203463\\
54.0029157667387	-0.000476469864959895\\
54.0529184665227	-0.000751669435273273\\
54.1029211663067	-0.000767869542033553\\
54.1529238660907	-0.00129278914574383\\
54.2029265658747	-0.00202628779789695\\
54.2529292656588	-0.00297752400697056\\
54.3029319654428	-0.00382801884888528\\
54.3529346652268	-0.00415162556875782\\
54.4029373650108	-0.00368052933594275\\
54.4529400647948	-0.00365698995329947\\
54.5029427645788	-0.0040563154700856\\
54.5529454643629	-0.00472141964214133\\
54.6029481641469	-0.00404721708616086\\
54.6529508639309	-0.00301516405810238\\
54.7029535637149	-0.00285332511483205\\
54.7529562634989	-0.00294329469427896\\
54.8029589632829	-0.00335507120949101\\
54.852961663067	-0.00390703721405686\\
54.902964362851	-0.00331005317330301\\
54.952967062635	-0.00235719604092604\\
55.002969762419	-0.00212173744220744\\
55.052972462203	-0.00242370617805109\\
55.102975161987	-0.00293063403253336\\
55.1529778617711	-0.00344495182112723\\
55.2029805615551	-0.00274186402208192\\
55.2529832613391	-0.00175833802991206\\
55.3029859611231	-0.000906179602444737\\
55.3529886609071	-0.000458479794210678\\
55.4029913606911	-0.000215609801766176\\
55.4529940604752	0.000165430070748296\\
55.5029967602592	0.000587839904501372\\
55.5529994600432	0.000790949731627559\\
55.6030021598272	0.000872039416211696\\
55.6530048596112	0.00105366936145565\\
55.7030075593953	0.00144913885979256\\
55.7530102591793	0.00197593800268725\\
55.8030129589633	0.00262327636080257\\
55.8530156587473	0.00260170585973311\\
55.9030183585313	0.00273293472677179\\
55.9530210583153	0.00307640399613353\\
56.0030237580994	0.00415328535223158\\
56.0530264578834	0.00505965552493703\\
56.1030291576674	0.00570499553413222\\
56.1530318574514	0.00589560277591765\\
56.2030345572354	0.00537074122625559\\
56.2530372570194	0.00525934398197636\\
56.3030399568035	0.00522145339648332\\
56.3530426565875	0.00563142687374216\\
56.4030453563715	0.00548936738991504\\
56.4530480561555	0.00533651283469107\\
56.5030507559395	0.00500932686318114\\
56.5530534557235	0.00546411086599676\\
56.6030561555076	0.00576801596854608\\
56.6530588552916	0.00547673115483838\\
56.7030615550756	0.0049319077870376\\
56.7530642548596	0.00437278341398136\\
56.8030669546436	0.00471426076751116\\
56.8530696544276	0.00467650120627834\\
56.9030723542117	0.00489054565877607\\
56.9530750539957	0.00463889162594013\\
57.0030777537797	0.00403651646423361\\
57.0530804535637	0.00375594958531539\\
57.1030831533477	0.00435108558318214\\
57.1530858531318	0.00438002487969307\\
57.2030885529158	0.0045040127206874\\
57.2530912526998	0.00404909804593291\\
57.3030939524838	0.00281913479748756\\
57.3530966522678	0.00220237636332837\\
57.4030993520518	0.00210723768347943\\
57.4531020518358	0.00185902755558906\\
57.5031047516199	0.00191483753913722\\
57.5531074514039	0.0013863389442798\\
57.6031101511879	0.00138267898508889\\
57.6531128509719	0.0013412389205864\\
57.7031155507559	0.000981709447922823\\
57.75311825054	0.00118835933671663\\
57.803120950324	0.00133763876707803\\
57.853123650108	0.00165602810473862\\
57.903126349892	0.00183218751423801\\
57.953129049676	0.00221506753286109\\
58.00313174946	0.00194900816273889\\
58.0531344492441	0.00147251872789674\\
58.1031371490281	0.00107687948746148\\
58.1531398488121	0.00164340888797416\\
58.2031425485961	0.00245072697592513\\
58.2531452483801	0.00297391371142186\\
58.3031479481642	0.00314095278610468\\
58.3531506479482	0.00270412551897773\\
58.4031533477322	0.00186990680233634\\
58.4531560475162	0.00137737923772818\\
58.5031587473002	0.00093324972465021\\
58.5531614470842	0.000584339862454044\\
58.6031641468683	0.000127670019611553\\
58.6531668466523	-0.00039733978316605\\
58.7031695464363	-0.000809009666513718\\
58.7531722462203	-0.00133241829414712\\
58.8031749460043	-0.00204433752864902\\
58.8531776457883	-0.00247040625326514\\
58.9031803455724	-0.00342701154484009\\
58.9531830453564	-0.00386749702959152\\
59.0031857451404	-0.00373987950651004\\
59.0531884449244	-0.00343410228485632\\
59.1031911447084	-0.00274552400491497\\
59.1531938444924	-0.0020983272769289\\
59.2031965442765	-0.00173871832367263\\
59.2531992440605	-0.0013862688991717\\
59.3032019438445	-0.000987209512049864\\
59.3532046436285	-0.00041356979463572\\
59.4032073434125	0.000384719935650402\\
59.4532100431965	0.000909779325083631\\
59.5032127429806	0.000733559680435983\\
59.5532154427646	0.00101956934750335\\
59.6032181425486	0.00108060929596085\\
59.6532208423326	0.00129271918405674\\
59.7032235421166	0.00158398868364583\\
59.7532262419006	0.00181954828606523\\
59.8032289416847	0.00159659872175605\\
59.8532316414687	0.00103551927073854\\
59.9032343412527	0.000773049442948285\\
59.9532370410367	0.000571769714363663\\
60.0032397408207	0.000521469847180079\\
60.0532424406048	0.000780319318524992\\
60.1032451403888	0.00151751876226227\\
60.1532478401728	0.00153545856836991\\
60.2032505399568	0.001341248983216\\
60.2532532397408	0.00123160900508831\\
60.3032559395248	0.00131412870617014\\
60.3532586393089	0.00125492900510075\\
60.4032613390929	0.00059331967283991\\
60.4532640388769	-0.000282289746829835\\
60.5032667386609	-0.000958349166287875\\
60.5532694384449	-0.00125128937347248\\
60.6032721382289	-0.00114527931822214\\
60.653274838013	-0.000695939463907662\\
60.703277537797	-5.56900749343967e-05\\
60.753280237581	0.000436939442653485\\
60.803282937365	0.000906199531019596\\
60.853285637149	0.00148501889647518\\
60.9032883369331	0.00202448753273262\\
60.9532910367171	0.00252433641711433\\
61.0032937365011	0.00288764507782959\\
61.0532964362851	0.00357087079043846\\
61.1032991360691	0.00439795442000325\\
61.1533018358531	0.00448411290875431\\
61.2033045356371	0.00456507284757473\\
61.2533072354212	0.00451116279979938\\
61.3033099352052	0.00498580631086683\\
61.3533126349892	0.00516917418032289\\
61.4033153347732	0.00522297393019774\\
61.4533180345572	0.00566893766234594\\
61.5033207343413	0.00568702732932883\\
61.5533234341253	0.00577339605712297\\
61.6033261339093	0.00581823398624873\\
61.6533288336933	0.00583261364872622\\
61.7033315334773	0.00642425375363245\\
61.7533342332613	0.00638108299943514\\
61.8033369330454	0.00587392427279314\\
61.8533396328294	0.00558984756623299\\
61.9033423326134	0.00532385241643266\\
61.9533450323974	0.00534015075260241\\
62.0033477321814	0.00525371340870952\\
62.0533504319654	0.00533467286513017\\
62.1033531317495	0.00570509489316111\\
62.1533558315335	0.00513508431931235\\
62.2033585313175	0.00458495184243939\\
62.2533612311015	0.00428279313849757\\
62.3033639308855	0.00410131657483443\\
62.3533666306695	0.0038693381282825\\
62.4033693304536	0.00290382430229006\\
62.4533720302376	0.00248838663485624\\
62.5033747300216	0.00256201674308039\\
62.5533774298056	0.00293969373360033\\
62.6033801295896	0.00730158037406515\\
62.6533828293737	0.0209804584179161\\
62.6983852591793	0.0443165373128708\\
62.7433876889849	0.0773065720009305\\
62.7883901187905	0.107975776811472\\
62.8333925485961	0.132401873201575\\
62.8833952483801	0.151120863574408\\
62.9333979481641	0.161097951541105\\
62.9834006479482	0.167220659514066\\
63.0334033477322	0.174147197072698\\
63.0834060475162	0.183101929436818\\
63.1334087473002	0.191018891849376\\
63.1834114470842	0.196952668310739\\
63.2334141468683	0.20137008400624\\
63.2834168466523	0.2055281673067\\
63.3334195464363	0.209942728578313\\
63.3834222462203	0.213768035539077\\
63.4334249460043	0.21765871193351\\
63.4834276457883	0.220498912423392\\
63.5334303455724	0.222248807200975\\
63.5834330453564	0.223994402523092\\
63.6334357451404	0.225933825208333\\
63.6834384449244	0.226659074170141\\
63.7334411447084	0.227287712377391\\
63.7834438444924	0.228044111394442\\
63.8334465442765	0.228850803410512\\
63.8834492440605	0.228639097044047\\
63.9334519438445	0.22836246570446\\
63.9834546436285	0.228726745914892\\
64.0334573434125	0.228702082984959\\
64.0834600431965	0.228714222501722\\
64.1334627429806	0.228775612763585\\
64.1834654427646	0.22849902076264\\
64.2334681425486	0.227783169831619\\
64.2834708423326	0.227508275400174\\
64.3334735421166	0.227084606585917\\
64.3834762419007	0.226946348299605\\
64.4334789416847	0.226387543666201\\
64.4834816414687	0.226709732458104\\
64.5334843412527	0.227109004501926\\
64.5834870410367	0.227324459452586\\
64.6334897408207	0.227436576353541\\
64.6834924406048	0.227529219394336\\
64.7334951403888	0.226932039512644\\
64.7834978401728	0.226926746519794\\
64.8335005399568	0.226450733094179\\
64.8835032397408	0.227110785782134\\
64.9335059395248	0.22693746069593\\
64.9835086393089	0.226872687185046\\
65.0335113390929	0.226907733881642\\
65.0835140388769	0.227067041086207\\
65.1335167386609	0.227273712781603\\
65.1835194384449	0.227434715332753\\
65.2335221382289	0.227522273562301\\
65.283524838013	0.227471527756065\\
65.333527537797	0.227376846635155\\
65.383530237581	0.226872572709928\\
65.433532937365	0.227201924381569\\
65.483535637149	0.227235125397334\\
65.533538336933	0.227123026696113\\
65.5835410367171	0.227219347096309\\
65.6335437365011	0.226926821155741\\
65.6835464362851	0.226165088597998\\
65.7335491360691	0.225879490673085\\
65.7835518358531	0.225332937617479\\
65.8335545356372	0.225282246640105\\
65.8835572354212	0.224537774169903\\
65.9335599352052	0.224637606495278\\
65.9835626349892	0.224721799161723\\
66.0335653347732	0.22449215129886\\
66.0835680345572	0.224555079092082\\
66.1335707343412	0.225028024457346\\
66.1835734341253	0.225566080452077\\
66.2335761339093	0.225658819646868\\
66.2835788336933	0.225650094448491\\
66.3335815334773	0.225729092744312\\
66.3835842332613	0.225558787218479\\
66.4335869330454	0.225280391134651\\
66.4835896328294	0.225583456119156\\
66.5335923326134	0.225373412038953\\
66.5835950323974	0.225396041968847\\
66.6335977321814	0.225467957918357\\
66.6836004319654	0.225781515901777\\
66.7336031317495	0.225864009053866\\
66.7836058315335	0.225877951538374\\
66.8336085313175	0.225930399908062\\
66.8836112311015	0.225872458582879\\
66.9336139308855	0.226042467536164\\
66.9836166306695	0.226017913809417\\
67.0336193304536	0.226476886042622\\
67.0836220302376	0.226872527843495\\
67.1336247300216	0.226732422861491\\
67.1836274298056	0.22703012353998\\
67.2336301295896	0.227443163443539\\
67.2836328293737	0.227122988095744\\
67.3336355291577	0.227364690212217\\
67.3836382289417	0.227038910309663\\
67.4336409287257	0.227480376230828\\
67.4836436285097	0.227207120462992\\
67.5336463282937	0.22724375202725\\
67.5836490280778	0.227257646508258\\
67.6336517278618	0.227499491978991\\
67.6836544276458	0.227250789714039\\
67.7336571274298	0.227385699959371\\
67.7836598272138	0.227154650139796\\
67.8336625269979	0.227406620744997\\
67.8836652267819	0.22714934087652\\
67.9336679265659	0.227618602463614\\
67.9836706263499	0.227690235351587\\
68.0336733261339	0.227837173379935\\
68.0836760259179	0.228408125072835\\
68.1336787257019	0.228448291047751\\
68.183681425486	0.228079093063157\\
68.23368412527	0.227327815626012\\
68.283686825054	0.227052969375949\\
68.333689524838	0.226746614721429\\
68.383692224622	0.226434829804418\\
68.4336949244061	0.226610061979202\\
68.4836976241901	0.226340232830462\\
68.5337003239741	0.226646742762452\\
68.5837030237581	0.226580297187269\\
68.6337057235421	0.226695794527674\\
68.6837084233261	0.226257919852844\\
68.7337111231102	0.226205363321338\\
68.7837138228942	0.226515440713694\\
68.8337165226782	0.227003758094325\\
68.8837192224622	0.22711941622174\\
68.9337219222462	0.226741358759102\\
68.9837246220302	0.226956661451517\\
69.0337273218143	0.227177204943962\\
69.0837300215983	0.227655213972991\\
69.1337327213823	0.227326048888493\\
69.1837354211663	0.227376923985825\\
69.2337381209503	0.227319130717815\\
69.2837408207344	0.227578138183414\\
69.3337435205184	0.227347142970338\\
69.3837462203024	0.227541600837216\\
69.4337489200864	0.227370026449022\\
69.4837516198704	0.227312096753621\\
69.5337543196544	0.227441571157644\\
69.5837570194385	0.226905672147285\\
69.6337597192225	0.226669433326853\\
69.6837624190065	0.227047662009919\\
69.7337651187905	0.227348916321184\\
69.7837678185745	0.227159729101177\\
69.8337705183585	0.227154504937698\\
69.8837732181426	0.227024971077747\\
69.9337759179266	0.227494328659743\\
69.9837786177106	0.228135054219434\\
70.0337813174946	0.228261066731367\\
70.0837840172786	0.228077234370484\\
70.1337867170626	0.227301634220992\\
70.1837894168467	0.227028491548167\\
70.2337921166307	0.227103847987583\\
70.2837948164147	0.226867610939524\\
70.3337975161987	0.226723826073888\\
70.3838002159827	0.226595946763771\\
70.4338029157667	0.226816707711521\\
70.4838056155508	0.226930546123243\\
70.5338083153348	0.226906095921723\\
70.5838110151188	0.226709811050348\\
70.6338137149028	0.226830568400871\\
70.6838164146868	0.226718556109747\\
70.7338191144708	0.226506590256718\\
70.7838218142549	0.22648717111665\\
70.8338245140389	0.226750052453913\\
70.8838272138229	0.22698999273731\\
70.9338299136069	0.226867371123008\\
70.9838326133909	0.227039039599266\\
71.0338353131749	0.227427616418656\\
71.083838012959	0.227355957791133\\
71.133840712743	0.227515182888692\\
71.183843412527	0.227588768293458\\
71.233846112311	0.22743666743276\\
71.283848812095	0.227453878608007\\
71.3338515118791	0.227035431167785\\
71.3838542116631	0.227163287213078\\
71.4338569114471	0.227007489486918\\
71.4838596112311	0.227107179704546\\
71.5338623110151	0.227576491447949\\
71.5838650107991	0.227963450897876\\
71.6338677105831	0.22733665864003\\
71.6838704103672	0.227156352042792\\
71.7338731101512	0.22706339700041\\
71.7838758099352	0.22723145512094\\
71.8338785097192	0.226988198986808\\
71.8838812095032	0.2270477962494\\
71.9338839092873	0.227231715816556\\
71.9838866090713	0.227184321815122\\
72.0338893088553	0.227410199135695\\
72.0838920086393	0.22780775648971\\
72.1338947084233	0.227720129519331\\
72.1838974082073	0.227544977516418\\
72.2339001079914	0.227557131988372\\
72.2839028077754	0.227385664799045\\
72.3339055075594	0.227179060240081\\
72.3839082073434	0.226925134230673\\
72.4339109071274	0.227417117025614\\
72.4839136069115	0.227709582878879\\
72.5339163066955	0.227914323622725\\
72.5839190064795	0.227886356675259\\
72.6339217062635	0.227930154822992\\
72.6839244060475	0.227623901936983\\
72.7339271058315	0.227476676062433\\
72.7839298056156	0.227126553568641\\
72.8339325053996	0.227140658575291\\
72.8839352051836	0.227156236381577\\
72.9339379049676	0.22750823894483\\
72.9839406047516	0.227685203287089\\
73.0339433045356	0.227965323619949\\
73.0839460043197	0.227651878075114\\
73.1339487041037	0.227261425086144\\
73.1839514038877	0.226797461698874\\
73.2339541036717	0.226683531397192\\
73.2839568034557	0.227082832842588\\
73.3339595032398	0.227469901691643\\
73.3839622030238	0.227735821254563\\
73.4339649028078	0.22799158539563\\
73.4839676025918	0.227291128335084\\
73.5339703023758	0.226709768563354\\
73.5839730021598	0.226578654098223\\
73.6339757019438	0.226060103194358\\
73.6839784017279	0.226079312896566\\
73.7339811015119	0.226103898691942\\
73.7839838012959	0.226361325420596\\
73.8339865010799	0.226210754210587\\
73.8839892008639	0.22654875189556\\
73.933991900648	0.226375154311748\\
73.983994600432	0.226068778024761\\
74.033997300216	0.226187827759521\\
74.084	0.226275517337194\\
74.134002699784	0.226534628114258\\
74.184005399568	0.226536490918794\\
74.2340080993521	0.226983050977593\\
74.2840107991361	0.227200203238592\\
74.3340134989201	0.227219376662465\\
74.3840161987041	0.227357807776448\\
74.4340188984881	0.226969115053647\\
74.4840215982721	0.226541760186137\\
74.5340242980562	0.226678328429215\\
74.5840269978402	0.227186166227889\\
74.6340296976242	0.227532753582388\\
74.6840323974082	0.22711949294522\\
74.7340350971922	0.227131880975317\\
74.7840377969763	0.226886778470079\\
74.8340404967603	0.226618809860777\\
74.8840431965443	0.226103943890954\\
74.9340458963283	0.225951650651653\\
74.9840485961123	0.22597612800623\\
75.0340512958963	0.226012857081478\\
75.0840539956804	0.226372025233828\\
75.1340566954644	0.226319454436018\\
75.1840593952484	0.22624228638923\\
75.2340620950324	0.226079456372338\\
75.2840647948164	0.226047846078021\\
75.3340674946004	0.226051481237267\\
75.3840701943844	0.226291184471901\\
75.4340728941685	0.226352576639098\\
75.4840755939525	0.226352466419687\\
75.5340782937365	0.22633845015395\\
75.5840809935205	0.225923416112247\\
75.6340836933045	0.225732483505411\\
75.6840863930885	0.225742954972336\\
75.7340890928726	0.225862195875967\\
75.7840917926566	0.225758868919044\\
75.8340944924406	0.225594184963269\\
75.8840971922246	0.226042576618594\\
75.9340998920086	0.226207176563544\\
75.9841025917927	0.226674869987872\\
76.0341052915767	0.226569705227704\\
76.0841079913607	0.227065315844649\\
76.1341106911447	0.227263319470776\\
76.1841133909287	0.226709958379363\\
76.2341160907127	0.226834205941133\\
76.2841187904968	0.226527585686737\\
76.3341214902808	0.226814751098542\\
76.3841241900648	0.227170267270074\\
76.4341268898488	0.227326137955795\\
76.4841295896328	0.227769222171422\\
76.5341322894169	0.227658930105112\\
76.5841349892009	0.226972365566557\\
76.6341376889849	0.227063461430736\\
76.6841403887689	0.227070619105458\\
76.7341430885529	0.226737788411929\\
76.7841457883369	0.226785103425983\\
76.834148488121	0.227254352535118\\
76.884151187905	0.227622100696805\\
76.934153887689	0.227966967802951\\
76.984156587473	0.227459260151877\\
77.034159287257	0.227060126490483\\
77.084161987041	0.227142345400423\\
77.1341646868251	0.227147709389291\\
77.1841673866091	0.227287597684309\\
77.2341700863931	0.227809399484808\\
77.2841727861771	0.228040580442081\\
77.3341754859611	0.228240115045913\\
77.3841781857451	0.228287217031012\\
77.4341808855292	0.227583506489992\\
77.4841835853132	0.226907613790637\\
77.5341862850972	0.226198471084379\\
77.5841889848812	0.225800863252245\\
77.6341916846652	0.226292992588555\\
77.6841943844492	0.22652577233064\\
77.7341970842333	0.227168435203466\\
77.7841997840173	0.227560737529728\\
77.8342024838013	0.228019517279264\\
77.8842051835853	0.228166585510011\\
77.9342078833693	0.227392672471546\\
77.9842105831534	0.226646741080907\\
78.0342132829374	0.226371912043483\\
78.0842159827214	0.226004089530256\\
78.1342186825054	0.225835870025411\\
78.1842213822894	0.225839347922292\\
78.2342240820734	0.226100361072936\\
78.2842267818575	0.225748386030993\\
78.3342294816415	0.22581482153241\\
78.3842321814255	0.225532996866684\\
78.4342348812095	0.225489135494228\\
78.4842375809935	0.225728958328691\\
78.5342402807775	0.225883030352626\\
78.5842429805616	0.225883007659046\\
78.6342456803456	0.225860322397814\\
78.6842483801296	0.225615203108192\\
78.7342510799136	0.225350777965655\\
78.7842537796976	0.224830278935497\\
78.8342564794817	0.224536129369535\\
78.8842591792657	0.224483340143306\\
78.9342618790497	0.224557023322254\\
78.9842645788337	0.224467732484615\\
79.0342672786177	0.224190881247234\\
79.0842699784017	0.223959488736878\\
79.1342726781857	0.224111956333133\\
79.1842753779698	0.224388786617838\\
79.2342780777538	0.224527287664327\\
79.2842807775378	0.225043970277815\\
79.3342834773218	0.225284168587219\\
79.3842861771058	0.225355935248789\\
79.4342888768899	0.225492677255886\\
79.4842915766739	0.225170297187574\\
79.5342942764579	0.22496006401053\\
79.5842969762419	0.225044109560655\\
79.6342996760259	0.225352387514199\\
79.6843023758099	0.225422355400427\\
79.734305075594	0.225089612561671\\
79.784307775378	0.225105532850016\\
79.834310475162	0.224742923307069\\
79.884313174946	0.224753243234981\\
79.93431587473	0.224961699124244\\
79.984318574514	0.225319050886594\\
80.0343212742981	0.225415481210633\\
80.0843239740821	0.225853359755869\\
80.1343266738661	0.225639547277676\\
80.1843293736501	0.225744644360478\\
80.2343320734341	0.225858624266574\\
80.2843347732181	0.226373645830824\\
80.3343374730022	0.22670633837153\\
80.3843401727862	0.226366543418611\\
80.4343428725702	0.22560990033447\\
80.4843455723542	0.225361187951978\\
80.5343482721382	0.225508190431281\\
80.5843509719223	0.225769249165602\\
80.6343536717063	0.225797450016728\\
80.6843563714903	0.225779892118661\\
80.7343590712743	0.226098417023271\\
80.7843617710583	0.226525720871782\\
80.8343644708423	0.22626657100547\\
80.8843671706263	0.22615298150602\\
80.9343698704104	0.226177463583559\\
80.9843725701944	0.226345501555486\\
81.0343752699784	0.22653997323741\\
81.0843779697624	0.226518858993858\\
81.1343806695464	0.226487330802165\\
81.1843833693305	0.227072265561315\\
81.2343860691145	0.227279066251018\\
81.2843887688985	0.227531120229253\\
81.3343914686825	0.227711366428718\\
81.3843941684665	0.227644809087872\\
81.4343968682505	0.227389222295694\\
81.4843995680346	0.226760493708642\\
81.5344022678186	0.226371815538025\\
81.5844049676026	0.226065362503017\\
81.6344076673866	0.226237042191789\\
81.6844103671706	0.226089931916124\\
81.7344130669546	0.225720230860688\\
81.7844157667387	0.22549262045075\\
81.8344184665227	0.225643221108665\\
81.8844211663067	0.226023165353749\\
81.9344238660907	0.225974189084588\\
81.9844265658747	0.22573951252602\\
82.0344292656588	0.225089765593038\\
82.0844319654428	0.225072055829493\\
82.1344346652268	0.225434593042995\\
82.1844373650108	0.225707920790958\\
82.2344400647948	0.226284179412601\\
82.2844427645788	0.226189574606463\\
82.3344454643629	0.225515362037069\\
82.3844481641469	0.224933759103168\\
82.4344508639309	0.224837505188081\\
82.4844535637149	0.224968693661684\\
82.5344562634989	0.224891617880986\\
82.5844589632829	0.224318602284448\\
82.634461663067	0.223935175873868\\
82.684464362851	0.22035270169412\\
82.734467062635	0.207916524503569\\
82.784469762419	0.183328983417589\\
82.8294721922246	0.153412554190566\\
82.8744746220302	0.12306992036011\\
82.9194770518359	0.0996001673622788\\
82.9694797516199	0.0835282550061769\\
83.0194824514039	0.0740236799496861\\
83.0694851511879	0.0640471313885887\\
83.1194878509719	0.0531646232911571\\
83.1694905507559	0.0428059383998715\\
83.21949325054	0.0352712113741453\\
83.269495950324	0.0305683245920346\\
83.319498650108	0.0265279956712241\\
83.369501349892	0.0224460666228536\\
83.419504049676	0.0182972973913731\\
83.4695067494601	0.0153471979930889\\
83.5195094492441	0.0136158592903449\\
83.5695121490281	0.0119780296127673\\
83.6195148488121	0.0104244701914272\\
83.6695175485961	0.00904560210827476\\
83.7195202483801	0.00828147419313903\\
83.7695229481642	0.00735016887896697\\
83.8195256479482	0.007080625498911\\
83.8695283477322	0.00631999370314961\\
83.9195310475162	0.00531497205592703\\
83.9695337473002	0.00489239725917009\\
84.0195364470842	0.00448783180672328\\
84.0695391468683	0.00415345575496839\\
84.1195418466523	0.00358529011788749\\
84.1695445464363	0.00279418464566026\\
84.2195472462203	0.00212527764728735\\
84.2695499460043	0.00263225599875939\\
84.3195526457883	0.00377037962636581\\
84.3695553455724	0.00448961307073124\\
84.4195580453564	0.00433671471979191\\
84.4695607451404	0.00421999477444365\\
84.5195634449244	0.00443564368657821\\
84.5695661447084	0.00436913291121546\\
84.6195688444924	0.00520353342184274\\
84.6695715442765	0.00544432980749576\\
84.7195742440605	0.0056420470379737\\
84.7695769438445	0.00569967692191078\\
84.8195796436285	0.00574993661657542\\
84.8695823434125	0.00574649726373412\\
84.9195850431965	0.00573732698007875\\
84.9695877429806	0.00569234759153559\\
85.0195904427646	0.00564924806138927\\
85.0695931425486	0.00554143978801434\\
85.1195958423326	0.00566186775627214\\
85.1695985421166	0.00587409438145352\\
85.2196012419006	0.00582723545126923\\
85.2696039416847	0.00558444789600639\\
85.3196066414687	0.00548398075420764\\
85.3696093412527	0.00536334303105489\\
85.4196120410367	0.00559358981062458\\
85.4696147408207	0.00530966112351205\\
85.5196174406048	0.00533113191316668\\
85.5696201403888	0.0051675845428977\\
85.6196228401728	0.00550733916916022\\
85.6696255399568	0.00558094940450933\\
85.7196282397408	0.00563136891314282\\
85.7696309395248	0.0055774186954233\\
85.8196336393089	0.00566194855139695\\
85.8696363390929	0.00567428720235431\\
85.9196390388769	0.0053757925867606\\
85.9696417386609	0.00508650600980657\\
86.0196444384449	0.00411563611157582\\
86.0696471382289	0.00420916470850511\\
86.119649838013	0.00469822052844111\\
86.169652537797	0.00515118574319424\\
86.219655237581	0.00558447830372627\\
86.269657937365	0.00565826705706104\\
86.319660637149	0.00565469723049222\\
86.369663336933	0.00594239166309741\\
86.4196660367171	0.00647463328721985\\
86.4696687365011	0.00692219203157614\\
86.5196714362851	0.00714340651498583\\
86.5696741360691	0.00701212803431326\\
86.6196768358531	0.00684671257649742\\
86.6696795356372	0.0063954935961842\\
86.7196822354212	0.00634147460325669\\
86.7696849352052	0.00645842249476133\\
86.8196876349892	0.00644944239785992\\
86.8696903347732	0.00632904538683023\\
86.9196930345572	0.00592630228214516\\
86.9696957343413	0.00569604666040618\\
87.0196984341253	0.00574103695938795\\
87.0697011339093	0.00593697255520646\\
87.1197038336933	0.00630010605433688\\
87.1697065334773	0.00576972556807831\\
87.2197092332613	0.0052591332004222\\
87.2697119330454	0.00507735581727484\\
87.3197146328294	0.00518545437518709\\
87.3697173326134	0.00569443521090498\\
87.4197200323974	0.00558627965336371\\
87.4697227321814	0.00554142821307241\\
87.5197254319655	0.00556125918506151\\
87.5697281317495	0.00574109659625791\\
87.6197308315335	0.00598550313166618\\
87.6697335313175	0.00605017032841072\\
87.7197362311015	0.00582539429039633\\
87.7697389308855	0.00547856050937998\\
87.8197416306695	0.00597107952395287\\
87.8697443304536	0.00575900556352167\\
87.9197470302376	0.00560434771982412\\
87.9697497300216	0.00531131212919511\\
88.0197524298056	0.00559360949780868\\
88.0697551295896	0.00571385787175204\\
88.1197578293737	0.00573193635581792\\
88.1697605291577	0.00591189395675378\\
88.2197632289417	0.00610056030840477\\
88.2697659287257	0.00631454623234289\\
88.3197686285097	0.00639546412215314\\
88.3697713282937	0.00615614753873913\\
88.4197740280778	0.00597483097353473\\
88.4697767278618	0.005645726602368\\
88.5197794276458	0.00506311604119508\\
88.5697821274298	0.00440143372962299\\
88.6197848272138	0.00386395829387582\\
88.6697875269979	0.00376686943519728\\
88.7197902267819	0.00443562269655846\\
88.7697929265659	0.00509179384860369\\
88.8197956263499	0.00489408742832628\\
88.8697983261339	0.0045704517534367\\
88.9198010259179	0.00470699967205324\\
88.969803725702	0.00452548325542936\\
89.019806425486	0.0046945306376808\\
89.06980912527	0.00500741712408223\\
89.119811825054	0.0052375347252177\\
89.169814524838	0.0053255827683078\\
89.219817224622	0.00536343111505793\\
89.2698199244061	0.00518547549966283\\
89.3198226241901	0.00481851024787414\\
89.3698253239741	0.00480767921814178\\
89.4198280237581	0.00476294109881334\\
89.4698307235421	0.0047289004136007\\
89.5198334233261	0.00484561571415186\\
89.5698361231102	0.00483838946767312\\
89.6198388228942	0.00469281112172522\\
89.6698415226782	0.00442127429495032\\
89.7198442224622	0.00458299225642242\\
89.7698469222462	0.00455964087958895\\
89.8198496220302	0.0043961344652495\\
89.8698523218143	0.0038891471054401\\
89.9198550215983	0.00355650045240237\\
89.9698577213823	0.00360313053698644\\
90.0198604211663	0.00388185816533716\\
90.0698631209503	0.00404547601205708\\
90.1198658207343	0.00398791813281693\\
90.1698685205184	0.00413541695858906\\
90.2198712203024	0.00427017546977268\\
90.2698739200864	0.00488698903273049\\
90.3198766198704	0.00526268239567088\\
90.3698793196544	0.00478797758899237\\
90.4198820194384	0.00294157309306403\\
90.4698847192225	0.00150843833897814\\
90.5198874190065	0.000742709661971891\\
90.5698901187905	0.00087552959325023\\
90.6198928185745	0.00152655883007551\\
90.6698955183585	0.00217370736264713\\
90.7198982181426	0.00288207520349131\\
90.7699009179266	0.00305294427411397\\
90.8199036177106	0.00334256250167673\\
90.8699063174946	0.00425227544946609\\
90.9199090172786	0.00477191882215444\\
90.9699117170626	0.00515125500376009\\
91.0199144168467	0.0056366180481516\\
91.0699171166307	0.00667959577381927\\
91.1199198164147	0.0073322011123369\\
91.1699225161987	0.00780138910052387\\
91.2199252159827	0.00789672627052344\\
91.2699279157667	0.00738978010240725\\
91.3199306155508	0.00739852858948746\\
91.3699333153348	0.00751375742176313\\
91.4199360151188	0.00749391645678491\\
91.4699387149028	0.00777444750897286\\
91.5199414146868	0.00818815406393797\\
91.5699441144708	0.00828143290751964\\
91.6199468142549	0.00830476184911301\\
91.6699495140389	0.00843438619274514\\
91.7199522138229	0.00866986886449887\\
91.7699549136069	0.00859065090246094\\
91.8199576133909	0.00868423634309281\\
91.8699603131749	0.00799196061403987\\
91.919963012959	0.00767013220219464\\
91.969965712743	0.0074958261159595\\
92.019968412527	0.00745080583998396\\
92.069971112311	0.00739513043255994\\
92.119973812095	0.00740409904751166\\
92.1699765118791	0.00702485917642045\\
92.2199792116631	0.00649975256353147\\
92.2699819114471	0.00651950206378776\\
92.3199846112311	0.00627858656317666\\
92.3699873110151	0.00617968769436338\\
92.4199900107991	0.00621938835006525\\
92.4699927105832	0.00579686477326045\\
92.5199954103672	0.00524848408128662\\
92.5699981101512	0.00525540557448442\\
92.605	0.00521397637557539\\
};
\addlegendentry{Measured angle [rad]};

\end{axis}
\end{tikzpicture}%
	}
	\subfloat[][]{\setlength\figureheight{4cm}
		\setlength\figurewidth{6cm}
		% This file was created by matlab2tikz v0.4.7 running on MATLAB 8.0.
% Copyright (c) 2008--2014, Nico Schlömer <nico.schloemer@gmail.com>
% All rights reserved.
% Minimal pgfplots version: 1.3
% 
% The latest updates can be retrieved from
%   http://www.mathworks.com/matlabcentral/fileexchange/22022-matlab2tikz
% where you can also make suggestions and rate matlab2tikz.
% 
\begin{tikzpicture}

\begin{axis}[%
width=\figurewidth,
height=\figureheight,
scale only axis,
xmin=0,
xmax=92.605,
xlabel={time [s]},
ymin=-0.1,
ymax=0.25,
ylabel={distance [m] / angle [rad]},
axis x line*=bottom,
axis y line*=left,
legend style={at={(0.03,0.97)},anchor=north west,draw=black,fill=white,legend cell align=left},
%legend columns = 2
]
\addplot [color=black!50!green,solid,line width=0.2pt]
  table[row sep=crcr]{0	0\\
0.0500026997840173	0\\
0.100005399568035	0\\
0.150008099352052	0\\
0.200010799136069	0\\
0.250013498920086	0\\
0.300016198704104	0\\
0.350018898488121	0\\
0.400021598272138	0\\
0.450024298056156	0\\
0.500026997840173	0\\
0.55002969762419	0\\
0.600032397408207	0\\
0.650035097192225	0\\
0.700037796976242	0\\
0.750040496760259	0\\
0.800043196544276	0\\
0.850045896328294	0\\
0.900048596112311	0\\
0.950051295896328	0\\
1.00005399568035	0\\
1.05005669546436	0\\
1.10005939524838	0\\
1.1500620950324	0\\
1.20006479481641	0\\
1.25006749460043	0\\
1.30007019438445	0\\
1.35007289416847	0\\
1.40007559395248	0\\
1.4500782937365	0\\
1.50008099352052	0\\
1.55008369330454	0\\
1.60008639308855	0\\
1.65008909287257	0\\
1.70009179265659	0\\
1.7500944924406	0\\
1.80009719222462	0\\
1.85009989200864	0\\
1.90010259179266	0\\
1.95010529157667	0\\
2.00010799136069	0\\
2.05011069114471	0\\
2.10011339092873	0\\
2.15011609071274	0\\
2.20011879049676	0\\
2.25012149028078	0\\
2.30012419006479	0\\
2.35012688984881	0\\
2.40012958963283	0\\
2.45013228941685	0\\
2.50013498920086	0\\
2.55013768898488	0\\
2.6001403887689	0\\
2.65014308855292	0\\
2.70014578833693	0\\
2.75014848812095	0\\
2.80015118790497	0\\
2.85015388768899	0\\
2.900156587473	0\\
2.95015928725702	0\\
3.00016198704104	0\\
3.05016468682505	0\\
3.10016738660907	0\\
3.15017008639309	0\\
3.20017278617711	0\\
3.25017548596112	0\\
3.30017818574514	0\\
3.35018088552916	0\\
3.40018358531318	0\\
3.45018628509719	0\\
3.50018898488121	0\\
3.55019168466523	0\\
3.60019438444924	0\\
3.65019708423326	0\\
3.70019978401728	0\\
3.7502024838013	0\\
3.80020518358531	0\\
3.85020788336933	0\\
3.90021058315335	0\\
3.95021328293736	0\\
4.00021598272138	0\\
4.0502186825054	0\\
4.10022138228942	0\\
4.15022408207343	0\\
4.20022678185745	0\\
4.25022948164147	0\\
4.30023218142549	0\\
4.3502348812095	0\\
4.40023758099352	0\\
4.45024028077754	0\\
4.50024298056155	0\\
4.55024568034557	0\\
4.60024838012959	0\\
4.65025107991361	0\\
4.70025377969762	0\\
4.75025647948164	0\\
4.80025917926566	0\\
4.85026187904968	0\\
4.90026457883369	0\\
4.95026727861771	0\\
5.00026997840173	0\\
5.05027267818575	0\\
5.10027537796976	0\\
5.15027807775378	0\\
5.2002807775378	0\\
5.25028347732181	0\\
5.30028617710583	0\\
5.35028887688985	0\\
5.40029157667387	0\\
5.45029427645788	0\\
5.5002969762419	0\\
5.55029967602592	0\\
5.60030237580994	0\\
5.65030507559395	0\\
5.70030777537797	0\\
5.75031047516199	0\\
5.800313174946	0\\
5.85031587473002	0\\
5.90031857451404	0\\
5.95032127429806	0\\
6.00032397408207	0\\
6.05032667386609	0\\
6.10032937365011	0\\
6.15033207343413	0\\
6.20033477321814	0\\
6.25033747300216	0\\
6.30034017278618	0\\
6.3503428725702	0\\
6.40034557235421	0\\
6.45034827213823	0\\
6.50035097192225	0\\
6.55035367170626	0\\
6.60035637149028	0\\
6.6503590712743	0\\
6.70036177105832	0\\
6.75036447084233	0\\
6.80036717062635	0\\
6.85036987041037	0\\
6.90037257019439	0\\
6.9503752699784	0\\
7.00037796976242	0\\
7.05038066954644	0\\
7.10038336933045	0\\
7.15038606911447	0\\
7.20038876889849	0\\
7.25039146868251	0\\
7.30039416846652	0\\
7.35039686825054	0\\
7.40039956803456	0\\
7.45040226781857	0\\
7.50040496760259	0\\
7.55040766738661	0\\
7.60041036717063	0\\
7.65041306695464	0\\
7.70041576673866	0\\
7.75041846652268	0\\
7.8004211663067	0\\
7.85042386609071	0\\
7.90042656587473	0\\
7.95042926565875	0\\
8.00043196544276	0\\
8.05043466522678	0\\
8.1004373650108	0\\
8.15044006479482	0\\
8.20044276457883	0\\
8.25044546436285	0\\
8.30044816414687	0\\
8.35045086393089	0\\
8.4004535637149	0\\
8.45045626349892	0\\
8.50045896328294	0\\
8.55046166306696	0\\
8.60046436285097	0\\
8.65046706263499	0\\
8.70046976241901	0\\
8.75047246220302	0\\
8.80047516198704	0\\
8.85047786177106	0\\
8.90048056155508	0\\
8.95048326133909	0\\
9.00048596112311	0\\
9.05048866090713	0\\
9.10049136069114	0\\
9.15049406047516	0\\
9.20049676025918	0\\
9.2504994600432	0\\
9.30050215982721	0\\
9.35050485961123	0\\
9.40050755939525	0\\
9.45051025917927	0\\
9.50051295896328	0\\
9.5505156587473	0\\
9.60051835853132	0\\
9.65052105831533	0\\
9.70052375809935	0\\
9.75052645788337	0\\
9.80052915766739	0\\
9.8505318574514	0\\
9.90053455723542	0\\
9.95053725701944	0\\
10.0005399568035	0\\
10.0505426565875	0\\
10.1005453563715	0\\
10.1505480561555	0\\
10.2005507559395	0\\
10.2505534557235	0\\
10.3005561555076	0\\
10.3505588552916	0\\
10.4005615550756	0\\
10.4505642548596	0\\
10.5005669546436	0\\
10.5505696544276	0\\
10.6005723542117	0\\
10.6505750539957	0\\
10.7005777537797	0\\
10.7505804535637	0\\
10.8005831533477	0\\
10.8505858531317	0\\
10.9005885529158	0\\
10.9505912526998	0\\
11.0005939524838	0\\
11.0505966522678	0\\
11.1005993520518	0\\
11.1506020518359	0\\
11.2006047516199	0\\
11.2506074514039	0\\
11.3006101511879	0\\
11.3506128509719	0\\
11.4006155507559	0\\
11.45061825054	0\\
11.500620950324	0\\
11.550623650108	0\\
11.600626349892	0\\
11.650629049676	0\\
11.70063174946	0\\
11.7506344492441	0\\
11.8006371490281	0\\
11.8506398488121	0\\
11.9006425485961	0\\
11.9506452483801	0\\
12.0006479481641	0\\
12.0506506479482	0\\
12.1006533477322	0\\
12.1506560475162	0\\
12.2006587473002	0\\
12.2506614470842	0\\
12.3006641468683	0\\
12.3506668466523	0\\
12.4006695464363	0\\
12.4506722462203	0\\
12.5006749460043	0\\
12.5506776457883	0\\
12.6006803455724	0\\
12.6506830453564	0\\
12.7006857451404	0\\
12.7506884449244	0\\
12.8006911447084	0\\
12.8506938444924	0\\
12.9006965442765	0\\
12.9506992440605	0\\
13.0007019438445	0\\
13.0507046436285	0\\
13.1007073434125	0\\
13.1507100431965	0\\
13.2007127429806	0\\
13.2507154427646	0\\
13.3007181425486	0\\
13.3507208423326	0\\
13.4007235421166	0\\
13.4507262419006	0\\
13.5007289416847	0\\
13.5507316414687	0\\
13.6007343412527	0\\
13.6507370410367	0\\
13.7007397408207	0\\
13.7507424406048	0\\
13.8007451403888	0\\
13.8507478401728	0\\
13.9007505399568	0\\
13.9507532397408	0\\
14.0007559395248	0\\
14.0507586393089	0\\
14.1007613390929	0\\
14.1507640388769	0\\
14.2007667386609	0\\
14.2507694384449	0\\
14.3007721382289	0\\
14.350774838013	0\\
14.400777537797	0\\
14.450780237581	0\\
14.500782937365	0\\
14.550785637149	0\\
14.600788336933	0\\
14.6507910367171	0\\
14.7007937365011	0\\
14.7507964362851	0\\
14.8007991360691	0\\
14.8508018358531	0\\
14.9008045356372	0\\
14.9508072354212	0\\
15.0008099352052	0\\
15.0508126349892	0\\
15.1008153347732	0\\
15.1508180345572	0\\
15.2008207343413	0\\
15.2508234341253	0\\
15.3008261339093	0\\
15.3508288336933	0\\
15.4008315334773	0\\
15.4508342332613	0\\
15.5008369330454	0\\
15.5508396328294	0\\
15.6008423326134	0\\
15.6508450323974	0\\
15.7008477321814	0\\
15.7508504319654	0\\
15.8008531317495	0\\
15.8508558315335	0\\
15.9008585313175	0\\
15.9508612311015	0\\
16.0008639308855	0\\
16.0508666306695	0\\
16.1008693304536	0\\
16.1508720302376	0\\
16.2008747300216	0\\
16.2508774298056	0\\
16.3008801295896	0\\
16.3508828293737	0\\
16.4008855291577	0\\
16.4508882289417	0\\
16.5008909287257	0\\
16.5508936285097	0\\
16.6008963282937	0\\
16.6508990280778	0\\
16.7009017278618	0\\
16.7509044276458	0\\
16.8009071274298	0\\
16.8509098272138	0\\
16.9009125269978	0\\
16.9509152267819	0\\
17.0009179265659	0\\
17.0509206263499	0\\
17.1009233261339	0\\
17.1509260259179	0\\
17.2009287257019	0\\
17.250931425486	0\\
17.30093412527	0\\
17.350936825054	0\\
17.400939524838	0\\
17.450942224622	0\\
17.500944924406	0\\
17.5509476241901	0\\
17.6009503239741	0\\
17.6509530237581	0\\
17.7009557235421	0\\
17.7509584233261	0\\
17.8009611231102	0\\
17.8509638228942	0\\
17.9009665226782	0\\
17.9509692224622	0\\
18.0009719222462	0\\
18.0509746220302	0\\
18.1009773218143	0\\
18.1509800215983	0\\
18.2009827213823	0\\
18.2509854211663	0\\
18.3009881209503	0\\
18.3509908207343	0\\
18.4009935205184	0\\
18.4509962203024	0\\
18.5009989200864	0\\
18.5510016198704	0\\
18.6010043196544	0\\
18.6510070194384	0\\
18.7010097192225	0\\
18.7510124190065	0\\
18.8010151187905	0\\
18.8510178185745	0\\
18.9010205183585	0\\
18.9510232181425	0\\
19.0010259179266	0\\
19.0510286177106	0\\
19.1010313174946	0\\
19.1510340172786	0\\
19.2010367170626	0\\
19.2510394168467	0\\
19.3010421166307	0\\
19.3510448164147	0\\
19.4010475161987	0\\
19.4510502159827	0\\
19.5010529157667	0\\
19.5510556155508	0\\
19.6010583153348	0\\
19.6510610151188	0\\
19.7010637149028	0\\
19.7510664146868	0\\
19.8010691144708	0\\
19.8510718142549	0\\
19.9010745140389	0\\
19.9510772138229	0\\
20.0010799136069	0\\
20.0510826133909	0\\
20.1010853131749	0\\
20.151088012959	0\\
20.201090712743	0\\
20.251093412527	0\\
20.301096112311	0\\
20.351098812095	0\\
20.4011015118791	0\\
20.4511042116631	0\\
20.5011069114471	0\\
20.5511096112311	0\\
20.6011123110151	0\\
20.6511150107991	0\\
20.7011177105832	0\\
20.7511204103672	0\\
20.8011231101512	0\\
20.8511258099352	0\\
20.9011285097192	0\\
20.9511312095032	0\\
21.0011339092873	0\\
21.0511366090713	0\\
21.1011393088553	0\\
21.1511420086393	0\\
21.2011447084233	0\\
21.2511474082073	0\\
21.3011501079914	0\\
21.3511528077754	0\\
21.4011555075594	0\\
21.4511582073434	0\\
21.5011609071274	0\\
21.5511636069114	0\\
21.6011663066955	0\\
21.6511690064795	0\\
21.7011717062635	0\\
21.7511744060475	0\\
21.8011771058315	0\\
21.8511798056156	0\\
21.9011825053996	0\\
21.9511852051836	0\\
22.0011879049676	0\\
22.0511906047516	0\\
22.1011933045356	0\\
22.1511960043197	0\\
22.2011987041037	0\\
22.2512014038877	0\\
22.3012041036717	0\\
22.3512068034557	0\\
22.4012095032397	0\\
22.4512122030238	0\\
22.5012149028078	0\\
22.5512176025918	0\\
22.6012203023758	0\\
22.6512230021598	0\\
22.7012257019438	0\\
22.7512284017279	0\\
22.8012311015119	0\\
22.8512338012959	0\\
22.9012365010799	0\\
22.9512392008639	0\\
23.0012419006479	0\\
23.051244600432	0\\
23.101247300216	0\\
23.15125	0\\
23.201252699784	0\\
23.251255399568	0\\
23.3012580993521	0\\
23.3512607991361	0\\
23.4012634989201	0\\
23.4512661987041	0\\
23.5012688984881	0\\
23.5512715982721	0\\
23.6012742980562	0\\
23.6512769978402	0\\
23.7012796976242	0\\
23.7512823974082	0\\
23.8012850971922	0\\
23.8512877969762	0\\
23.9012904967603	0\\
23.9512931965443	0\\
24.0012958963283	0\\
24.0512985961123	0\\
24.1013012958963	0\\
24.1513039956803	0\\
24.2013066954644	0\\
24.2513093952484	0\\
24.3013120950324	0\\
24.3513147948164	0\\
24.4013174946004	0\\
24.4513201943845	0\\
24.5013228941685	0\\
24.5513255939525	0\\
24.6013282937365	0\\
24.6513309935205	0\\
24.7013336933045	0\\
24.7513363930886	0\\
24.8013390928726	0\\
24.8513417926566	0\\
24.9013444924406	0\\
24.9513471922246	0\\
25.0013498920086	0\\
25.0513525917927	0\\
25.1013552915767	0\\
25.1513579913607	0\\
25.2013606911447	0\\
25.2513633909287	0\\
25.3013660907127	0\\
25.3513687904968	0\\
25.4013714902808	0\\
25.4513741900648	0\\
25.5013768898488	0\\
25.5513795896328	0\\
25.6013822894168	0\\
25.6513849892009	0\\
25.7013876889849	0\\
25.7513903887689	0\\
25.8013930885529	0\\
25.8513957883369	0\\
25.901398488121	0\\
25.951401187905	0\\
26.001403887689	0\\
26.051406587473	0\\
26.101409287257	0\\
26.151411987041	0\\
26.2014146868251	0\\
26.2514173866091	0\\
26.3014200863931	0\\
26.3514227861771	0\\
26.4014254859611	0\\
26.4514281857451	0\\
26.5014308855292	0\\
26.5514335853132	0\\
26.6014362850972	0\\
26.6514389848812	0\\
26.7014416846652	0\\
26.7514443844492	0\\
26.8014470842333	0\\
26.8514497840173	0\\
26.9014524838013	0\\
26.9514551835853	0\\
27.0014578833693	0\\
27.0514605831534	0\\
27.1014632829374	0\\
27.1514659827214	0\\
27.2014686825054	0\\
27.2514713822894	0\\
27.3014740820734	0\\
27.3514767818575	0\\
27.4014794816415	0\\
27.4514821814255	0\\
27.5014848812095	0\\
27.5514875809935	0\\
27.6014902807775	0\\
27.6514929805616	0\\
27.7014956803456	0\\
27.7514983801296	0\\
27.8015010799136	0\\
27.8515037796976	0\\
27.9015064794816	0\\
27.9515091792657	0\\
28.0015118790497	0\\
28.0515145788337	0\\
28.1015172786177	0\\
28.1515199784017	0\\
28.2015226781857	0\\
28.2515253779698	0\\
28.3015280777538	0\\
28.3515307775378	0\\
28.4015334773218	0\\
28.4515361771058	0\\
28.5015388768899	0\\
28.5515415766739	0\\
28.6015442764579	0\\
28.6515469762419	0\\
28.7015496760259	0\\
28.7515523758099	0\\
28.801555075594	0\\
28.851557775378	0\\
28.901560475162	0\\
28.951563174946	0\\
29.00156587473	0\\
29.051568574514	0\\
29.1015712742981	0\\
29.1515739740821	0\\
29.2015766738661	0\\
29.2515793736501	0\\
29.3015820734341	0\\
29.3515847732181	0\\
29.4015874730022	0\\
29.4515901727862	0\\
29.5015928725702	0\\
29.5515955723542	0\\
29.6015982721382	0\\
29.6516009719222	0\\
29.7016036717063	0\\
29.7516063714903	0\\
29.8016090712743	0\\
29.8516117710583	0\\
29.9016144708423	0\\
29.9516171706264	0\\
30.0016198704104	0\\
30.0516225701944	0\\
30.1016252699784	0\\
30.1516279697624	0\\
30.2016306695464	0\\
30.2516333693305	0\\
30.3016360691145	0\\
30.3516387688985	0\\
30.4016414686825	0\\
30.4516441684665	0\\
30.5016468682505	0\\
30.5516495680346	0\\
30.6016522678186	0\\
30.6516549676026	0\\
30.7016576673866	0\\
30.7516603671706	0\\
30.8016630669546	0\\
30.8516657667387	0\\
30.9016684665227	0\\
30.9516711663067	0\\
31.0016738660907	0\\
31.0516765658747	0\\
31.1016792656587	0\\
31.1516819654428	0\\
31.2016846652268	0\\
31.2516873650108	0\\
31.3016900647948	0\\
31.3516927645788	0\\
31.4016954643629	0\\
31.4516981641469	0\\
31.5017008639309	0\\
31.5517035637149	0\\
31.6017062634989	0\\
31.6517089632829	0\\
31.701711663067	0\\
31.751714362851	0\\
31.801717062635	0\\
31.851719762419	0\\
31.901722462203	0\\
31.951725161987	0\\
32.0017278617711	0\\
32.0517305615551	0\\
32.1017332613391	0\\
32.1517359611231	0\\
32.2017386609071	0\\
32.2517413606911	0\\
32.3017440604752	0\\
32.3517467602592	0\\
32.4017494600432	0\\
32.4517521598272	0\\
32.5017548596112	0\\
32.5517575593952	0\\
32.6017602591793	0\\
32.6517629589633	0\\
32.7017656587473	0\\
32.7517683585313	0\\
32.8017710583153	0\\
32.8517737580993	0\\
32.9017764578834	0\\
32.9517791576674	0\\
33.0017818574514	0\\
33.0517845572354	0\\
33.1017872570194	0\\
33.1517899568035	0\\
33.2017926565875	0\\
33.2517953563715	0\\
33.3017980561555	0\\
33.3518007559395	0\\
33.4018034557235	0\\
33.4518061555076	0\\
33.5018088552916	0\\
33.5518115550756	0\\
33.6018142548596	0\\
33.6518169546436	0\\
33.7018196544277	0\\
33.7518223542117	0\\
33.8018250539957	0\\
33.8518277537797	0\\
33.9018304535637	0\\
33.9518331533477	0\\
34.0018358531318	0\\
34.0518385529158	0\\
34.1018412526998	0\\
34.1518439524838	0\\
34.2018466522678	0\\
34.2518493520518	0\\
34.3018520518359	0\\
34.3518547516199	0\\
34.4018574514039	0\\
34.4518601511879	0\\
34.5018628509719	0\\
34.5518655507559	0\\
34.60186825054	0\\
34.651870950324	0\\
34.701873650108	0\\
34.751876349892	0\\
34.801879049676	0\\
34.85188174946	0\\
34.9018844492441	0\\
34.9518871490281	0\\
35.0018898488121	0\\
35.0518925485961	0\\
35.1018952483801	0\\
35.1518979481641	0\\
35.2019006479482	0\\
35.2519033477322	0\\
35.3019060475162	0\\
35.3519087473002	0\\
35.4019114470842	0\\
35.4519141468683	0\\
35.5019168466523	0\\
35.5519195464363	0\\
35.6019222462203	0\\
35.6519249460043	0\\
35.7019276457883	0\\
35.7519303455724	0\\
35.8019330453564	0\\
35.8519357451404	0\\
35.9019384449244	0\\
35.9519411447084	0\\
36.0019438444924	0\\
36.0519465442765	0\\
36.1019492440605	0\\
36.1519519438445	0\\
36.2019546436285	0\\
36.2519573434125	0\\
36.3019600431965	0\\
36.3519627429806	0\\
36.4019654427646	0\\
36.4519681425486	0\\
36.5019708423326	0\\
36.5519735421166	0\\
36.6019762419006	0\\
36.6519789416847	0\\
36.7019816414687	0\\
36.7519843412527	0\\
36.8019870410367	0\\
36.8519897408207	0\\
36.9019924406048	0\\
36.9519951403888	0\\
37.0019978401728	0\\
37.0520005399568	0\\
37.1020032397408	0\\
37.1520059395248	0\\
37.2020086393089	0\\
37.2520113390929	0\\
37.3020140388769	0\\
37.3520167386609	0\\
37.4020194384449	0\\
37.4520221382289	0\\
37.502024838013	0\\
37.552027537797	0\\
37.602030237581	0\\
37.652032937365	0\\
37.702035637149	0\\
37.752038336933	0\\
37.8020410367171	0\\
37.8520437365011	0\\
37.9020464362851	0\\
37.9520491360691	0\\
38.0020518358531	0\\
38.0520545356372	0\\
38.1020572354212	0\\
38.1520599352052	0\\
38.2020626349892	0\\
38.2520653347732	0\\
38.3020680345572	0\\
38.3520707343413	0\\
38.4020734341253	0\\
38.4520761339093	0\\
38.5020788336933	0\\
38.5520815334773	0\\
38.6020842332613	0\\
38.6520869330454	0\\
38.7020896328294	0\\
38.7520923326134	0\\
38.8020950323974	0\\
38.8520977321814	0\\
38.9021004319654	0\\
38.9521031317495	0\\
39.0021058315335	0\\
39.0521085313175	0\\
39.1021112311015	0\\
39.1521139308855	0\\
39.2021166306696	0\\
39.2521193304536	0\\
39.3021220302376	0\\
39.3521247300216	0\\
39.4021274298056	0\\
39.4521301295896	0\\
39.5021328293737	0\\
39.5521355291577	0\\
39.6021382289417	0\\
39.6521409287257	0\\
39.7021436285097	0\\
39.7521463282937	0\\
39.8021490280778	0\\
39.8521517278618	0\\
39.9021544276458	0\\
39.9521571274298	0\\
40.0021598272138	0\\
40.0521625269978	0\\
40.1021652267819	0\\
40.1521679265659	0\\
40.2021706263499	0\\
40.2521733261339	0\\
40.3021760259179	0\\
40.3521787257019	0\\
40.402181425486	0\\
40.45218412527	0\\
40.502186825054	0\\
40.552189524838	0\\
40.602192224622	0\\
40.652194924406	0\\
40.7021976241901	0\\
40.7522003239741	0\\
40.8022030237581	0\\
40.8522057235421	0\\
40.9022084233261	0\\
40.9522111231102	0\\
41.0022138228942	0\\
41.0522165226782	0\\
41.1022192224622	0\\
41.1522219222462	0\\
41.2022246220302	0\\
41.2522273218143	0\\
41.3022300215983	0\\
41.3522327213823	0\\
41.4022354211663	0\\
41.4522381209503	0\\
41.5022408207343	0\\
41.5522435205184	0\\
41.6022462203024	0\\
41.6522489200864	0\\
41.7022516198704	0\\
41.7522543196544	0\\
41.8022570194385	0\\
41.8522597192225	0\\
41.9022624190065	0\\
41.9522651187905	0\\
42.0022678185745	0\\
42.0522705183585	0\\
42.1022732181426	0\\
42.1522759179266	0\\
42.2022786177106	0\\
42.2522813174946	0\\
42.3022840172786	0\\
42.3522867170626	0\\
42.4022894168467	0\\
42.4522921166307	0\\
42.5022948164147	0\\
42.5522975161987	0\\
42.6023002159827	0\\
42.6523029157667	0\\
42.7023056155508	0\\
42.7523083153348	0\\
42.8023110151188	0\\
42.8523137149028	0\\
42.9023164146868	0\\
42.9523191144708	0\\
43.0023218142549	0\\
43.0523245140389	0\\
43.1023272138229	0\\
43.1523299136069	0\\
43.2023326133909	0\\
43.2523353131749	0\\
43.302338012959	0\\
43.352340712743	0\\
43.402343412527	0\\
43.452346112311	0\\
43.502348812095	0\\
43.5523515118791	0\\
43.6023542116631	0\\
43.6523569114471	0\\
43.7023596112311	0\\
43.7523623110151	0\\
43.8023650107991	0\\
43.8523677105832	0\\
43.9023704103672	0\\
43.9523731101512	0\\
44.0023758099352	0\\
44.0523785097192	0\\
44.1023812095032	0\\
44.1523839092873	0\\
44.2023866090713	0\\
44.2523893088553	0\\
44.3023920086393	0\\
44.3523947084233	0\\
44.4023974082073	0\\
44.4524001079914	0\\
44.5024028077754	0\\
44.5524055075594	0\\
44.6024082073434	0\\
44.6524109071274	0\\
44.7024136069114	0\\
44.7524163066955	0\\
44.8024190064795	0\\
44.8524217062635	0\\
44.9024244060475	0\\
44.9524271058315	0\\
45.0024298056155	0\\
45.0524325053996	0\\
45.1024352051836	0\\
45.1524379049676	0\\
45.2024406047516	0\\
45.2524433045356	0\\
45.3024460043197	0\\
45.3524487041037	0\\
45.4024514038877	0\\
45.4524541036717	0\\
45.5024568034557	0\\
45.5524595032397	0\\
45.6024622030238	0\\
45.6524649028078	0\\
45.7024676025918	0\\
45.7524703023758	0\\
45.8024730021598	0\\
45.8524757019439	0\\
45.9024784017279	0\\
45.9524811015119	0\\
46.0024838012959	0\\
46.0524865010799	0\\
46.1024892008639	0\\
46.152491900648	0\\
46.202494600432	0\\
46.252497300216	0\\
46.3025	0\\
46.352502699784	0\\
46.402505399568	0\\
46.4525080993521	0\\
46.5025107991361	0\\
46.5525134989201	0\\
46.6025161987041	0\\
46.6525188984881	0\\
46.7025215982721	0\\
46.7525242980562	0\\
46.8025269978402	0\\
46.8525296976242	0\\
46.9025323974082	0\\
46.9525350971922	0\\
47.0025377969762	0\\
47.0525404967603	0\\
47.1025431965443	0\\
47.1525458963283	0\\
47.2025485961123	0\\
47.2525512958963	0\\
47.3025539956803	0\\
47.3525566954644	0\\
47.4025593952484	0\\
47.4525620950324	0\\
47.5025647948164	0\\
47.5525674946004	0\\
47.6025701943845	0\\
47.6525728941685	0\\
47.7025755939525	0\\
47.7525782937365	0\\
47.8025809935205	0\\
47.8525836933045	0\\
47.9025863930886	0\\
47.9525890928726	0\\
48.0025917926566	0\\
48.0525944924406	0\\
48.1025971922246	0\\
48.1525998920086	0\\
48.2026025917927	0\\
48.2526052915767	0\\
48.3026079913607	0\\
48.3526106911447	0\\
48.4026133909287	0\\
48.4526160907127	0\\
48.5026187904968	0\\
48.5526214902808	0\\
48.6026241900648	0\\
48.6526268898488	0\\
48.7026295896328	0\\
48.7526322894168	0\\
48.8026349892009	0\\
48.8526376889849	0\\
48.9026403887689	0\\
48.9526430885529	0\\
49.0026457883369	0\\
49.052648488121	0\\
49.102651187905	0\\
49.152653887689	0\\
49.202656587473	0\\
49.252659287257	0\\
49.302661987041	0\\
49.3526646868251	0\\
49.4026673866091	0\\
49.4526700863931	0\\
49.5026727861771	0\\
49.5526754859611	0\\
49.6026781857451	0\\
49.6526808855292	0\\
49.7026835853132	0\\
49.7526862850972	0\\
49.8026889848812	0\\
49.8526916846652	0\\
49.9026943844492	0\\
49.9526970842333	0\\
50.0026997840173	0\\
50.0527024838013	0\\
50.1027051835853	0\\
50.1527078833693	0\\
50.2027105831534	0\\
50.2527132829374	0\\
50.3027159827214	0\\
50.3527186825054	0\\
50.4027213822894	0\\
50.4527240820734	0\\
50.5027267818575	0\\
50.5527294816415	0\\
50.6027321814255	0\\
50.6527348812095	0\\
50.7027375809935	0\\
50.7527402807775	0\\
50.8027429805616	0\\
50.8527456803456	0\\
50.9027483801296	0\\
50.9527510799136	0\\
51.0027537796976	0\\
51.0527564794816	0\\
51.1027591792657	0\\
51.1527618790497	0\\
51.2027645788337	0\\
51.2527672786177	0\\
51.3027699784017	0\\
51.3527726781858	0\\
51.4027753779698	0\\
51.4527780777538	0\\
51.5027807775378	0\\
51.5527834773218	0\\
51.6027861771058	0\\
51.6527888768899	0\\
51.7027915766739	0\\
51.7527942764579	0\\
51.8027969762419	0\\
51.8527996760259	0\\
51.9028023758099	0\\
51.952805075594	0\\
52.002807775378	0\\
52.052810475162	0\\
52.102813174946	0\\
52.15281587473	0\\
52.202818574514	0\\
52.2528212742981	0\\
52.3028239740821	0\\
52.3528266738661	0\\
52.4028293736501	0\\
52.4528320734341	0\\
52.5028347732181	0\\
52.5528374730022	0\\
52.6028401727862	0\\
52.6528428725702	0\\
52.7028455723542	0\\
52.7528482721382	0\\
52.8028509719222	0\\
52.8528536717063	0\\
52.9028563714903	0\\
52.9528590712743	0\\
53.0028617710583	0\\
53.0528644708423	0\\
53.1028671706264	0\\
53.1528698704104	0\\
53.2028725701944	0\\
53.2528752699784	0\\
53.3028779697624	0\\
53.3528806695464	0\\
53.4028833693305	0\\
53.4528860691145	0\\
53.5028887688985	0\\
53.5528914686825	0\\
53.6028941684665	0\\
53.6528968682505	0\\
53.7028995680346	0\\
53.7529022678186	0\\
53.8029049676026	0\\
53.8529076673866	0\\
53.9029103671706	0\\
53.9529130669547	0\\
54.0029157667387	0\\
54.0529184665227	0\\
54.1029211663067	0\\
54.1529238660907	0\\
54.2029265658747	0\\
54.2529292656588	0\\
54.3029319654428	0\\
54.3529346652268	0\\
54.4029373650108	0\\
54.4529400647948	0\\
54.5029427645788	0\\
54.5529454643629	0\\
54.6029481641469	0\\
54.6529508639309	0\\
54.7029535637149	0\\
54.7529562634989	0\\
54.8029589632829	0\\
54.852961663067	0\\
54.902964362851	0\\
54.952967062635	0\\
55.002969762419	0\\
55.052972462203	0\\
55.102975161987	0\\
55.1529778617711	0\\
55.2029805615551	0\\
55.2529832613391	0\\
55.3029859611231	0\\
55.3529886609071	0\\
55.4029913606911	0\\
55.4529940604752	0\\
55.5029967602592	0\\
55.5529994600432	0\\
55.6030021598272	0\\
55.6530048596112	0\\
55.7030075593953	0\\
55.7530102591793	0\\
55.8030129589633	0\\
55.8530156587473	0\\
55.9030183585313	0\\
55.9530210583153	0\\
56.0030237580994	0\\
56.0530264578834	0\\
56.1030291576674	0\\
56.1530318574514	0\\
56.2030345572354	0\\
56.2530372570194	0\\
56.3030399568035	0\\
56.3530426565875	0\\
56.4030453563715	0\\
56.4530480561555	0\\
56.5030507559395	0\\
56.5530534557235	0\\
56.6030561555076	0\\
56.6530588552916	0\\
56.7030615550756	0\\
56.7530642548596	0\\
56.8030669546436	0\\
56.8530696544276	0\\
56.9030723542117	0\\
56.9530750539957	0\\
57.0030777537797	0\\
57.0530804535637	0\\
57.1030831533477	0\\
57.1530858531318	0\\
57.2030885529158	0\\
57.2530912526998	0\\
57.3030939524838	0\\
57.3530966522678	0\\
57.4030993520518	0\\
57.4531020518358	0\\
57.5031047516199	0\\
57.5531074514039	0\\
57.6031101511879	0\\
57.6531128509719	0\\
57.7031155507559	0\\
57.75311825054	0\\
57.803120950324	0\\
57.853123650108	0\\
57.903126349892	0\\
57.953129049676	0\\
58.00313174946	0\\
58.0531344492441	0\\
58.1031371490281	0\\
58.1531398488121	0\\
58.2031425485961	0\\
58.2531452483801	0\\
58.3031479481642	0\\
58.3531506479482	0\\
58.4031533477322	0\\
58.4531560475162	0\\
58.5031587473002	0\\
58.5531614470842	0\\
58.6031641468683	0\\
58.6531668466523	0\\
58.7031695464363	0\\
58.7531722462203	0\\
58.8031749460043	0\\
58.8531776457883	0\\
58.9031803455724	0\\
58.9531830453564	0\\
59.0031857451404	0\\
59.0531884449244	0\\
59.1031911447084	0\\
59.1531938444924	0\\
59.2031965442765	0\\
59.2531992440605	0\\
59.3032019438445	0\\
59.3532046436285	0\\
59.4032073434125	0\\
59.4532100431965	0\\
59.5032127429806	0\\
59.5532154427646	0\\
59.6032181425486	0\\
59.6532208423326	0\\
59.7032235421166	0\\
59.7532262419006	0\\
59.8032289416847	0\\
59.8532316414687	0\\
59.9032343412527	0\\
59.9532370410367	0\\
60.0032397408207	0\\
60.0532424406048	0\\
60.1032451403888	0\\
60.1532478401728	0\\
60.2032505399568	0\\
60.2532532397408	0\\
60.3032559395248	0\\
60.3532586393089	0\\
60.4032613390929	0\\
60.4532640388769	0\\
60.5032667386609	0\\
60.5532694384449	0\\
60.6032721382289	0\\
60.653274838013	0\\
60.703277537797	0\\
60.753280237581	0\\
60.803282937365	0\\
60.853285637149	0\\
60.9032883369331	0\\
60.9532910367171	0\\
61.0032937365011	0\\
61.0532964362851	0\\
61.1032991360691	0\\
61.1533018358531	0\\
61.2033045356371	0\\
61.2533072354212	0\\
61.3033099352052	0\\
61.3533126349892	0\\
61.4033153347732	0\\
61.4533180345572	0\\
61.5033207343413	0\\
61.5533234341253	0\\
61.6033261339093	0\\
61.6533288336933	0\\
61.7033315334773	0\\
61.7533342332613	0\\
61.8033369330454	0\\
61.8533396328294	0\\
61.9033423326134	0\\
61.9533450323974	0\\
62.0033477321814	0\\
62.0533504319654	0\\
62.1033531317495	0\\
62.1533558315335	0\\
62.2033585313175	0\\
62.2533612311015	0\\
62.3033639308855	0\\
62.3533666306695	0\\
62.4033693304536	0\\
62.4533720302376	0\\
62.5033747300216	0\\
62.5483771598272	0.2\\
62.5983798596112	0.2\\
62.6483825593953	0.2\\
62.6983852591793	0.2\\
62.7483879589633	0.2\\
62.7983906587473	0.2\\
62.8483933585313	0.2\\
62.8983960583153	0.2\\
62.9483987580994	0.2\\
62.9984014578834	0.2\\
63.0484041576674	0.2\\
63.0984068574514	0.2\\
63.1484095572354	0.2\\
63.1984122570194	0.2\\
63.2484149568035	0.2\\
63.2984176565875	0.2\\
63.3484203563715	0.2\\
63.3984230561555	0.2\\
63.4484257559395	0.2\\
63.4984284557235	0.2\\
63.5484311555076	0.2\\
63.5984338552916	0.2\\
63.6484365550756	0.2\\
63.6984392548596	0.2\\
63.7484419546436	0.2\\
63.7984446544276	0.2\\
63.8484473542117	0.2\\
63.8984500539957	0.2\\
63.9484527537797	0.2\\
63.9984554535637	0.2\\
64.0484581533477	0.2\\
64.0984608531318	0.2\\
64.1484635529158	0.2\\
64.1984662526998	0.2\\
64.2484689524838	0.2\\
64.2984716522678	0.2\\
64.3484743520518	0.2\\
64.3984770518359	0.2\\
64.4484797516199	0.2\\
64.4984824514039	0.2\\
64.5484851511879	0.2\\
64.5984878509719	0.2\\
64.648490550756	0.2\\
64.69849325054	0.2\\
64.748495950324	0.2\\
64.798498650108	0.2\\
64.848501349892	0.2\\
64.898504049676	0.2\\
64.94850674946	0.2\\
64.9985094492441	0.2\\
65.0485121490281	0.2\\
65.0985148488121	0.2\\
65.1485175485961	0.2\\
65.1985202483801	0.2\\
65.2485229481642	0.2\\
65.2985256479482	0.2\\
65.3485283477322	0.2\\
65.3985310475162	0.2\\
65.4485337473002	0.2\\
65.4985364470842	0.2\\
65.5485391468683	0.2\\
65.5985418466523	0.2\\
65.6485445464363	0.2\\
65.6985472462203	0.2\\
65.7485499460043	0.2\\
65.7985526457883	0.2\\
65.8485553455724	0.2\\
65.8985580453564	0.2\\
65.9485607451404	0.2\\
65.9985634449244	0.2\\
66.0485661447084	0.2\\
66.0985688444924	0.2\\
66.1485715442765	0.2\\
66.1985742440605	0.2\\
66.2485769438445	0.2\\
66.2985796436285	0.2\\
66.3485823434125	0.2\\
66.3985850431965	0.2\\
66.4485877429806	0.2\\
66.4985904427646	0.2\\
66.5485931425486	0.2\\
66.5985958423326	0.2\\
66.6485985421166	0.2\\
66.6986012419006	0.2\\
66.7486039416847	0.2\\
66.7986066414687	0.2\\
66.8486093412527	0.2\\
66.8986120410367	0.2\\
66.9486147408207	0.2\\
66.9986174406047	0.2\\
67.0486201403888	0.2\\
67.0986228401728	0.2\\
67.1486255399568	0.2\\
67.1986282397408	0.2\\
67.2486309395248	0.2\\
67.2986336393089	0.2\\
67.3486363390929	0.2\\
67.3986390388769	0.2\\
67.4486417386609	0.2\\
67.4986444384449	0.2\\
67.5486471382289	0.2\\
67.598649838013	0.2\\
67.648652537797	0.2\\
67.698655237581	0.2\\
67.748657937365	0.2\\
67.798660637149	0.2\\
67.8486633369331	0.2\\
67.8986660367171	0.2\\
67.9486687365011	0.2\\
67.9986714362851	0.2\\
68.0486741360691	0.2\\
68.0986768358531	0.2\\
68.1486795356372	0.2\\
68.1986822354212	0.2\\
68.2486849352052	0.2\\
68.2986876349892	0.2\\
68.3486903347732	0.2\\
68.3986930345572	0.2\\
68.4486957343413	0.2\\
68.4986984341253	0.2\\
68.5487011339093	0.2\\
68.5987038336933	0.2\\
68.6487065334773	0.2\\
68.6987092332613	0.2\\
68.7487119330454	0.2\\
68.7987146328294	0.2\\
68.8487173326134	0.2\\
68.8987200323974	0.2\\
68.9487227321814	0.2\\
68.9987254319654	0.2\\
69.0487281317495	0.2\\
69.0987308315335	0.2\\
69.1487335313175	0.2\\
69.1987362311015	0.2\\
69.2487389308855	0.2\\
69.2987416306696	0.2\\
69.3487443304536	0.2\\
69.3987470302376	0.2\\
69.4487497300216	0.2\\
69.4987524298056	0.2\\
69.5487551295896	0.2\\
69.5987578293737	0.2\\
69.6487605291577	0.2\\
69.6987632289417	0.2\\
69.7487659287257	0.2\\
69.7987686285097	0.2\\
69.8487713282937	0.2\\
69.8987740280778	0.2\\
69.9487767278618	0.2\\
69.9987794276458	0.2\\
70.0487821274298	0.2\\
70.0987848272138	0.2\\
70.1487875269979	0.2\\
70.1987902267819	0.2\\
70.2487929265659	0.2\\
70.2987956263499	0.2\\
70.3487983261339	0.2\\
70.3988010259179	0.2\\
70.4488037257019	0.2\\
70.498806425486	0.2\\
70.54880912527	0.2\\
70.598811825054	0.2\\
70.648814524838	0.2\\
70.698817224622	0.2\\
70.7488199244061	0.2\\
70.7988226241901	0.2\\
70.8488253239741	0.2\\
70.8988280237581	0.2\\
70.9488307235421	0.2\\
70.9988334233261	0.2\\
71.0488361231101	0.2\\
71.0988388228942	0.2\\
71.1488415226782	0.2\\
71.1988442224622	0.2\\
71.2488469222462	0.2\\
71.2988496220302	0.2\\
71.3488523218143	0.2\\
71.3988550215983	0.2\\
71.4488577213823	0.2\\
71.4988604211663	0.2\\
71.5488631209503	0.2\\
71.5988658207343	0.2\\
71.6488685205184	0.2\\
71.6988712203024	0.2\\
71.7488739200864	0.2\\
71.7988766198704	0.2\\
71.8488793196544	0.2\\
71.8988820194385	0.2\\
71.9488847192225	0.2\\
71.9988874190065	0.2\\
72.0488901187905	0.2\\
72.0988928185745	0.2\\
72.1488955183585	0.2\\
72.1988982181425	0.2\\
72.2489009179266	0.2\\
72.2989036177106	0.2\\
72.3489063174946	0.2\\
72.3989090172786	0.2\\
72.4489117170626	0.2\\
72.4989144168467	0.2\\
72.5489171166307	0.2\\
72.5989198164147	0.2\\
72.6489225161987	0.2\\
72.6989252159827	0.2\\
72.7489279157667	0.2\\
72.7989306155508	0.2\\
72.8489333153348	0.2\\
72.8989360151188	0.2\\
72.9489387149028	0.2\\
72.9989414146868	0.2\\
73.0489441144708	0.2\\
73.0989468142549	0.2\\
73.1489495140389	0.2\\
73.1989522138229	0.2\\
73.2489549136069	0.2\\
73.2989576133909	0.2\\
73.348960313175	0.2\\
73.398963012959	0.2\\
73.448965712743	0.2\\
73.498968412527	0.2\\
73.548971112311	0.2\\
73.598973812095	0.2\\
73.6489765118791	0.2\\
73.6989792116631	0.2\\
73.7489819114471	0.2\\
73.7989846112311	0.2\\
73.8489873110151	0.2\\
73.8989900107991	0.2\\
73.9489927105832	0.2\\
73.9989954103672	0.2\\
74.0489981101512	0.2\\
74.0990008099352	0.2\\
74.1490035097192	0.2\\
74.1990062095032	0.2\\
74.2490089092873	0.2\\
74.2990116090713	0.2\\
74.3490143088553	0.2\\
74.3990170086393	0.2\\
74.4490197084233	0.2\\
74.4990224082073	0.2\\
74.5490251079914	0.2\\
74.5990278077754	0.2\\
74.6490305075594	0.2\\
74.6990332073434	0.2\\
74.7490359071274	0.2\\
74.7990386069115	0.2\\
74.8490413066955	0.2\\
74.8990440064795	0.2\\
74.9490467062635	0.2\\
74.9990494060475	0.2\\
75.0490521058315	0.2\\
75.0990548056156	0.2\\
75.1490575053996	0.2\\
75.1990602051836	0.2\\
75.2490629049676	0.2\\
75.2990656047516	0.2\\
75.3490683045356	0.2\\
75.3990710043197	0.2\\
75.4490737041037	0.2\\
75.4990764038877	0.2\\
75.5490791036717	0.2\\
75.5990818034557	0.2\\
75.6490845032397	0.2\\
75.6990872030238	0.2\\
75.7490899028078	0.2\\
75.7990926025918	0.2\\
75.8490953023758	0.2\\
75.8990980021598	0.2\\
75.9491007019438	0.2\\
75.9991034017279	0.2\\
76.0491061015119	0.2\\
76.0991088012959	0.2\\
76.1491115010799	0.2\\
76.1991142008639	0.2\\
76.2491169006479	0.2\\
76.299119600432	0.2\\
76.349122300216	0.2\\
76.399125	0.2\\
76.449127699784	0.2\\
76.499130399568	0.2\\
76.5491330993521	0.2\\
76.5991357991361	0.2\\
76.6491384989201	0.2\\
76.6991411987041	0.2\\
76.7491438984881	0.2\\
76.7991465982721	0.2\\
76.8491492980562	0.2\\
76.8991519978402	0.2\\
76.9491546976242	0.2\\
76.9991573974082	0.2\\
77.0491600971922	0.2\\
77.0991627969762	0.2\\
77.1491654967603	0.2\\
77.1991681965443	0.2\\
77.2491708963283	0.2\\
77.2991735961123	0.2\\
77.3491762958963	0.2\\
77.3991789956804	0.2\\
77.4491816954644	0.2\\
77.4991843952484	0.2\\
77.5491870950324	0.2\\
77.5991897948164	0.2\\
77.6491924946004	0.2\\
77.6991951943844	0.2\\
77.7491978941685	0.2\\
77.7992005939525	0.2\\
77.8492032937365	0.2\\
77.8992059935205	0.2\\
77.9492086933045	0.2\\
77.9992113930886	0.2\\
78.0492140928726	0.2\\
78.0992167926566	0.2\\
78.1492194924406	0.2\\
78.1992221922246	0.2\\
78.2492248920086	0.2\\
78.2992275917927	0.2\\
78.3492302915767	0.2\\
78.3992329913607	0.2\\
78.4492356911447	0.2\\
78.4992383909287	0.2\\
78.5492410907127	0.2\\
78.5992437904968	0.2\\
78.6492464902808	0.2\\
78.6992491900648	0.2\\
78.7492518898488	0.2\\
78.7992545896328	0.2\\
78.8492572894169	0.2\\
78.8992599892009	0.2\\
78.9492626889849	0.2\\
78.9992653887689	0.2\\
79.0492680885529	0.2\\
79.0992707883369	0.2\\
79.149273488121	0.2\\
79.199276187905	0.2\\
79.249278887689	0.2\\
79.299281587473	0.2\\
79.349284287257	0.2\\
79.3992869870411	0.2\\
79.4492896868251	0.2\\
79.4992923866091	0.2\\
79.5492950863931	0.2\\
79.5992977861771	0.2\\
79.6493004859611	0.2\\
79.6993031857451	0.2\\
79.7493058855292	0.2\\
79.7993085853132	0.2\\
79.8493112850972	0.2\\
79.8993139848812	0.2\\
79.9493166846652	0.2\\
79.9993193844493	0.2\\
80.0493220842333	0.2\\
80.0993247840173	0.2\\
80.1493274838013	0.2\\
80.1993301835853	0.2\\
80.2493328833693	0.2\\
80.2993355831534	0.2\\
80.3493382829374	0.2\\
80.3993409827214	0.2\\
80.4493436825054	0.2\\
80.4993463822894	0.2\\
80.5493490820734	0.2\\
80.5993517818575	0.2\\
80.6493544816415	0.2\\
80.6993571814255	0.2\\
80.7493598812095	0.2\\
80.7993625809935	0.2\\
80.8493652807775	0.2\\
80.8993679805616	0.2\\
80.9493706803456	0.2\\
80.9993733801296	0.2\\
81.0493760799136	0.2\\
81.0993787796976	0.2\\
81.1493814794816	0.2\\
81.1993841792657	0.2\\
81.2493868790497	0.2\\
81.2993895788337	0.2\\
81.3493922786177	0.2\\
81.3993949784017	0.2\\
81.4493976781857	0.2\\
81.4994003779698	0.2\\
81.5494030777538	0.2\\
81.5994057775378	0.2\\
81.6494084773218	0.2\\
81.6994111771058	0.2\\
81.7494138768898	0.2\\
81.7994165766739	0.2\\
81.8494192764579	0.2\\
81.8994219762419	0.2\\
81.9494246760259	0.2\\
81.9994273758099	0.2\\
82.049430075594	0.2\\
82.099432775378	0.2\\
82.149435475162	0.2\\
82.199438174946	0.2\\
82.24944087473	0.2\\
82.299443574514	0.2\\
82.3494462742981	0.2\\
82.3994489740821	0.2\\
82.4494516738661	0.2\\
82.4994543736501	0.2\\
82.5494570734341	0.2\\
82.5994597732181	0.2\\
82.6294613930885	0\\
82.6794640928726	0\\
82.7294667926566	0\\
82.7794694924406	0\\
82.8294721922246	0\\
82.8794748920086	0\\
82.9294775917927	0\\
82.9794802915767	0\\
83.0294829913607	0\\
83.0794856911447	0\\
83.1294883909287	0\\
83.1794910907127	0\\
83.2294937904968	0\\
83.2794964902808	0\\
83.3294991900648	0\\
83.3795018898488	0\\
83.4295045896328	0\\
83.4795072894169	0\\
83.5295099892009	0\\
83.5795126889849	0\\
83.6295153887689	0\\
83.6795180885529	0\\
83.7295207883369	0\\
83.779523488121	0\\
83.829526187905	0\\
83.879528887689	0\\
83.929531587473	0\\
83.979534287257	0\\
84.029536987041	0\\
84.0795396868251	0\\
84.1295423866091	0\\
84.1795450863931	0\\
84.2295477861771	0\\
84.2795504859611	0\\
84.3295531857451	0\\
84.3795558855292	0\\
84.4295585853132	0\\
84.4795612850972	0\\
84.5295639848812	0\\
84.5795666846652	0\\
84.6295693844492	0\\
84.6795720842333	0\\
84.7295747840173	0\\
84.7795774838013	0\\
84.8295801835853	0\\
84.8795828833693	0\\
84.9295855831534	0\\
84.9795882829374	0\\
85.0295909827214	0\\
85.0795936825054	0\\
85.1295963822894	0\\
85.1795990820734	0\\
85.2296017818575	0\\
85.2796044816415	0\\
85.3296071814255	0\\
85.3796098812095	0\\
85.4296125809935	0\\
85.4796152807775	0\\
85.5296179805616	0\\
85.5796206803456	0\\
85.6296233801296	0\\
85.6796260799136	0\\
85.7296287796976	0\\
85.7796314794817	0\\
85.8296341792657	0\\
85.8796368790497	0\\
85.9296395788337	0\\
85.9796422786177	0\\
86.0296449784017	0\\
86.0796476781857	0\\
86.1296503779698	0\\
86.1796530777538	0\\
86.2296557775378	0\\
86.2796584773218	0\\
86.3296611771058	0\\
86.3796638768899	0\\
86.4296665766739	0\\
86.4796692764579	0\\
86.5296719762419	0\\
86.5796746760259	0\\
86.6296773758099	0\\
86.679680075594	0\\
86.729682775378	0\\
86.779685475162	0\\
86.829688174946	0\\
86.87969087473	0\\
86.929693574514	0\\
86.9796962742981	0\\
87.0296989740821	0\\
87.0797016738661	0\\
87.1297043736501	0\\
87.1797070734341	0\\
87.2297097732181	0\\
87.2797124730022	0\\
87.3297151727862	0\\
87.3797178725702	0\\
87.4297205723542	0\\
87.4797232721382	0\\
87.5297259719222	0\\
87.5797286717063	0\\
87.6297313714903	0\\
87.6797340712743	0\\
87.7297367710583	0\\
87.7797394708423	0\\
87.8297421706264	0\\
87.8797448704104	0\\
87.9297475701944	0\\
87.9797502699784	0\\
88.0297529697624	0\\
88.0797556695464	0\\
88.1297583693305	0\\
88.1797610691145	0\\
88.2297637688985	0\\
88.2797664686825	0\\
88.3297691684665	0\\
88.3797718682505	0\\
88.4297745680346	0\\
88.4797772678186	0\\
88.5297799676026	0\\
88.5797826673866	0\\
88.6297853671706	0\\
88.6797880669546	0\\
88.7297907667387	0\\
88.7797934665227	0\\
88.8297961663067	0\\
88.8797988660907	0\\
88.9298015658747	0\\
88.9798042656588	0\\
89.0298069654428	0\\
89.0798096652268	0\\
89.1298123650108	0\\
89.1798150647948	0\\
89.2298177645788	0\\
89.2798204643629	0\\
89.3298231641469	0\\
89.3798258639309	0\\
89.4298285637149	0\\
89.4798312634989	0\\
89.5298339632829	0\\
89.579836663067	0\\
89.629839362851	0\\
89.679842062635	0\\
89.729844762419	0\\
89.779847462203	0\\
89.8298501619871	0\\
89.8798528617711	0\\
89.9298555615551	0\\
89.9798582613391	0\\
90.0298609611231	0\\
90.0798636609071	0\\
90.1298663606911	0\\
90.1798690604752	0\\
90.2298717602592	0\\
90.2798744600432	0\\
90.3298771598272	0\\
90.3798798596112	0\\
90.4298825593953	0\\
90.4798852591793	0\\
90.5298879589633	0\\
90.5798906587473	0\\
90.6298933585313	0\\
90.6798960583153	0\\
90.7298987580994	0\\
90.7799014578834	0\\
90.8299041576674	0\\
90.8799068574514	0\\
90.9299095572354	0\\
90.9799122570195	0\\
91.0299149568035	0\\
91.0799176565875	0\\
91.1299203563715	0\\
91.1799230561555	0\\
91.2299257559395	0\\
91.2799284557236	0\\
91.3299311555076	0\\
91.3799338552916	0\\
91.4299365550756	0\\
91.4799392548596	0\\
91.5299419546436	0\\
91.5799446544276	0\\
91.6299473542117	0\\
91.6799500539957	0\\
91.7299527537797	0\\
91.7799554535637	0\\
91.8299581533477	0\\
91.8799608531317	0\\
91.9299635529158	0\\
91.9799662526998	0\\
92.0299689524838	0\\
92.0799716522678	0\\
92.1299743520518	0\\
92.1799770518358	0\\
92.2299797516199	0\\
92.2799824514039	0\\
92.3299851511879	0\\
92.3799878509719	0\\
92.4299905507559	0\\
92.47999325054	0\\
92.529995950324	0\\
92.579998650108	0\\
92.605	0\\
};
\addlegendentry{set point};

\addplot [color=gray,solid,line width=0.2pt]
  table[row sep=crcr]{0	0.01\\
0.0500026997840173	0.01\\
0.100005399568035	0.01\\
0.150008099352052	0.01\\
0.200010799136069	0.01\\
0.250013498920086	0.01\\
0.300016198704104	0.01\\
0.350018898488121	0.01\\
0.400021598272138	0.01\\
0.450024298056156	0.01\\
0.500026997840173	0.01\\
0.55002969762419	0.01\\
0.600032397408207	0.01\\
0.650035097192225	0.01\\
0.700037796976242	0.01\\
0.750040496760259	0.01\\
0.800043196544276	0.01\\
0.850045896328294	0.01\\
0.900048596112311	0.01\\
0.950051295896328	0.01\\
1.00005399568035	0.01\\
1.05005669546436	0.01\\
1.10005939524838	0.01\\
1.1500620950324	0.01\\
1.20006479481641	0.01\\
1.25006749460043	0.01\\
1.30007019438445	0.01\\
1.35007289416847	0.01\\
1.40007559395248	0.01\\
1.4500782937365	0.01\\
1.50008099352052	0.01\\
1.55008369330454	0.01\\
1.60008639308855	0.01\\
1.65008909287257	0.01\\
1.70009179265659	0.01\\
1.7500944924406	0.01\\
1.80009719222462	0.01\\
1.85009989200864	0.01\\
1.90010259179266	0.01\\
1.95010529157667	0.01\\
2.00010799136069	0.01\\
2.05011069114471	0.01\\
2.10011339092873	0.01\\
2.15011609071274	0.01\\
2.20011879049676	0.01\\
2.25012149028078	0.01\\
2.30012419006479	0.01\\
2.35012688984881	0.01\\
2.40012958963283	0.01\\
2.45013228941685	0.01\\
2.50013498920086	0.01\\
2.55013768898488	0.01\\
2.6001403887689	0.01\\
2.65014308855292	0.01\\
2.70014578833693	0.01\\
2.75014848812095	0.01\\
2.80015118790497	0.01\\
2.85015388768899	0.01\\
2.900156587473	0.01\\
2.95015928725702	0.01\\
3.00016198704104	0.01\\
3.05016468682505	0.01\\
3.10016738660907	0.01\\
3.15017008639309	0.01\\
3.20017278617711	0.01\\
3.25017548596112	0.01\\
3.30017818574514	0.01\\
3.35018088552916	0.01\\
3.40018358531318	0.01\\
3.45018628509719	0.01\\
3.50018898488121	0.01\\
3.55019168466523	0.01\\
3.60019438444924	0.01\\
3.65019708423326	0.01\\
3.70019978401728	0.01\\
3.7502024838013	0.01\\
3.80020518358531	0.01\\
3.85020788336933	0.01\\
3.90021058315335	0.01\\
3.95021328293736	0.01\\
4.00021598272138	0.01\\
4.0502186825054	0.01\\
4.10022138228942	0.01\\
4.15022408207343	0.01\\
4.20022678185745	0.01\\
4.25022948164147	0.01\\
4.30023218142549	0.01\\
4.3502348812095	0.01\\
4.40023758099352	0.01\\
4.45024028077754	0.01\\
4.50024298056155	0.01\\
4.55024568034557	0.01\\
4.60024838012959	0.01\\
4.65025107991361	0.01\\
4.70025377969762	0.01\\
4.75025647948164	0.01\\
4.80025917926566	0.01\\
4.85026187904968	0.01\\
4.90026457883369	0.01\\
4.95026727861771	0.01\\
5.00026997840173	0.01\\
5.05027267818575	0.01\\
5.10027537796976	0.01\\
5.15027807775378	0.01\\
5.2002807775378	0.01\\
5.25028347732181	0.01\\
5.30028617710583	0.01\\
5.35028887688985	0.01\\
5.40029157667387	0.01\\
5.45029427645788	0.01\\
5.5002969762419	0.01\\
5.55029967602592	0.01\\
5.60030237580994	0.01\\
5.65030507559395	0.01\\
5.70030777537797	0.01\\
5.75031047516199	0.01\\
5.800313174946	0.01\\
5.85031587473002	0.01\\
5.90031857451404	0.01\\
5.95032127429806	0.01\\
6.00032397408207	0.01\\
6.05032667386609	0.01\\
6.10032937365011	0.01\\
6.15033207343413	0.01\\
6.20033477321814	0.01\\
6.25033747300216	0.01\\
6.30034017278618	0.01\\
6.3503428725702	0.01\\
6.40034557235421	0.01\\
6.45034827213823	0.01\\
6.50035097192225	0.01\\
6.55035367170626	0.01\\
6.60035637149028	0.01\\
6.6503590712743	0.01\\
6.70036177105832	0.01\\
6.75036447084233	0.01\\
6.80036717062635	0.01\\
6.85036987041037	0.01\\
6.90037257019439	0.01\\
6.9503752699784	0.01\\
7.00037796976242	0.01\\
7.05038066954644	0.01\\
7.10038336933045	0.01\\
7.15038606911447	0.01\\
7.20038876889849	0.01\\
7.25039146868251	0.01\\
7.30039416846652	0.01\\
7.35039686825054	0.01\\
7.40039956803456	0.01\\
7.45040226781857	0.01\\
7.50040496760259	0.01\\
7.55040766738661	0.01\\
7.60041036717063	0.01\\
7.65041306695464	0.01\\
7.70041576673866	0.01\\
7.75041846652268	0.01\\
7.8004211663067	0.01\\
7.85042386609071	0.01\\
7.90042656587473	0.01\\
7.95042926565875	0.01\\
8.00043196544276	0.01\\
8.05043466522678	0.01\\
8.1004373650108	0.01\\
8.15044006479482	0.01\\
8.20044276457883	0.01\\
8.25044546436285	0.01\\
8.30044816414687	0.01\\
8.35045086393089	0.01\\
8.4004535637149	0.01\\
8.45045626349892	0.01\\
8.50045896328294	0.01\\
8.55046166306696	0.01\\
8.60046436285097	0.01\\
8.65046706263499	0.01\\
8.70046976241901	0.01\\
8.75047246220302	0.01\\
8.80047516198704	0.01\\
8.85047786177106	0.01\\
8.90048056155508	0.01\\
8.95048326133909	0.01\\
9.00048596112311	0.01\\
9.05048866090713	0.01\\
9.10049136069114	0.01\\
9.15049406047516	0.01\\
9.20049676025918	0.01\\
9.2504994600432	0.01\\
9.30050215982721	0.01\\
9.35050485961123	0.01\\
9.40050755939525	0.01\\
9.45051025917927	0.01\\
9.50051295896328	0.01\\
9.5505156587473	0.01\\
9.60051835853132	0.01\\
9.65052105831533	0.01\\
9.70052375809935	0.01\\
9.75052645788337	0.01\\
9.80052915766739	0.01\\
9.8505318574514	0.01\\
9.90053455723542	0.01\\
9.95053725701944	0.01\\
10.0005399568035	0.01\\
10.0505426565875	0.01\\
10.1005453563715	0.01\\
10.1505480561555	0.01\\
10.2005507559395	0.01\\
10.2505534557235	0.01\\
10.3005561555076	0.01\\
10.3505588552916	0.01\\
10.4005615550756	0.01\\
10.4505642548596	0.01\\
10.5005669546436	0.01\\
10.5505696544276	0.01\\
10.6005723542117	0.01\\
10.6505750539957	0.01\\
10.7005777537797	0.01\\
10.7505804535637	0.01\\
10.8005831533477	0.01\\
10.8505858531317	0.01\\
10.9005885529158	0.01\\
10.9505912526998	0.01\\
11.0005939524838	0.01\\
11.0505966522678	0.01\\
11.1005993520518	0.01\\
11.1506020518359	0.01\\
11.2006047516199	0.01\\
11.2506074514039	0.01\\
11.3006101511879	0.01\\
11.3506128509719	0.01\\
11.4006155507559	0.01\\
11.45061825054	0.01\\
11.500620950324	0.01\\
11.550623650108	0.01\\
11.600626349892	0.01\\
11.650629049676	0.01\\
11.70063174946	0.01\\
11.7506344492441	0.01\\
11.8006371490281	0.01\\
11.8506398488121	0.01\\
11.9006425485961	0.01\\
11.9506452483801	0.01\\
12.0006479481641	0.01\\
12.0506506479482	0.01\\
12.1006533477322	0.01\\
12.1506560475162	0.01\\
12.2006587473002	0.01\\
12.2506614470842	0.01\\
12.3006641468683	0.01\\
12.3506668466523	0.01\\
12.4006695464363	0.01\\
12.4506722462203	0.01\\
12.5006749460043	0.01\\
12.5506776457883	0.01\\
12.6006803455724	0.01\\
12.6506830453564	0.01\\
12.7006857451404	0.01\\
12.7506884449244	0.01\\
12.8006911447084	0.01\\
12.8506938444924	0.01\\
12.9006965442765	0.01\\
12.9506992440605	0.01\\
13.0007019438445	0.01\\
13.0507046436285	0.01\\
13.1007073434125	0.01\\
13.1507100431965	0.01\\
13.2007127429806	0.01\\
13.2507154427646	0.01\\
13.3007181425486	0.01\\
13.3507208423326	0.01\\
13.4007235421166	0.01\\
13.4507262419006	0.01\\
13.5007289416847	0.01\\
13.5507316414687	0.01\\
13.6007343412527	0.01\\
13.6507370410367	0.01\\
13.7007397408207	0.01\\
13.7507424406048	0.01\\
13.8007451403888	0.01\\
13.8507478401728	0.01\\
13.9007505399568	0.01\\
13.9507532397408	0.01\\
14.0007559395248	0.01\\
14.0507586393089	0.01\\
14.1007613390929	0.01\\
14.1507640388769	0.01\\
14.2007667386609	0.01\\
14.2507694384449	0.01\\
14.3007721382289	0.01\\
14.350774838013	0.01\\
14.400777537797	0.01\\
14.450780237581	0.01\\
14.500782937365	0.01\\
14.550785637149	0.01\\
14.600788336933	0.01\\
14.6507910367171	0.01\\
14.7007937365011	0.01\\
14.7507964362851	0.01\\
14.8007991360691	0.01\\
14.8508018358531	0.01\\
14.9008045356372	0.01\\
14.9508072354212	0.01\\
15.0008099352052	0.01\\
15.0508126349892	0.01\\
15.1008153347732	0.01\\
15.1508180345572	0.01\\
15.2008207343413	0.01\\
15.2508234341253	0.01\\
15.3008261339093	0.01\\
15.3508288336933	0.01\\
15.4008315334773	0.01\\
15.4508342332613	0.01\\
15.5008369330454	0.01\\
15.5508396328294	0.01\\
15.6008423326134	0.01\\
15.6508450323974	0.01\\
15.7008477321814	0.01\\
15.7508504319654	0.01\\
15.8008531317495	0.01\\
15.8508558315335	0.01\\
15.9008585313175	0.01\\
15.9508612311015	0.01\\
16.0008639308855	0.01\\
16.0508666306695	0.01\\
16.1008693304536	0.01\\
16.1508720302376	0.01\\
16.2008747300216	0.01\\
16.2508774298056	0.01\\
16.3008801295896	0.01\\
16.3508828293737	0.01\\
16.4008855291577	0.01\\
16.4508882289417	0.01\\
16.5008909287257	0.01\\
16.5508936285097	0.01\\
16.6008963282937	0.01\\
16.6508990280778	0.01\\
16.7009017278618	0.01\\
16.7509044276458	0.01\\
16.8009071274298	0.01\\
16.8509098272138	0.01\\
16.9009125269978	0.01\\
16.9509152267819	0.01\\
17.0009179265659	0.01\\
17.0509206263499	0.01\\
17.1009233261339	0.01\\
17.1509260259179	0.01\\
17.2009287257019	0.01\\
17.250931425486	0.01\\
17.30093412527	0.01\\
17.350936825054	0.01\\
17.400939524838	0.01\\
17.450942224622	0.01\\
17.500944924406	0.01\\
17.5509476241901	0.01\\
17.6009503239741	0.01\\
17.6509530237581	0.01\\
17.7009557235421	0.01\\
17.7509584233261	0.01\\
17.8009611231102	0.01\\
17.8509638228942	0.01\\
17.9009665226782	0.01\\
17.9509692224622	0.01\\
18.0009719222462	0.01\\
18.0509746220302	0.01\\
18.1009773218143	0.01\\
18.1509800215983	0.01\\
18.2009827213823	0.01\\
18.2509854211663	0.01\\
18.3009881209503	0.01\\
18.3509908207343	0.01\\
18.4009935205184	0.01\\
18.4509962203024	0.01\\
18.5009989200864	0.01\\
18.5510016198704	0.01\\
18.6010043196544	0.01\\
18.6510070194384	0.01\\
18.7010097192225	0.01\\
18.7510124190065	0.01\\
18.8010151187905	0.01\\
18.8510178185745	0.01\\
18.9010205183585	0.01\\
18.9510232181425	0.01\\
19.0010259179266	0.01\\
19.0510286177106	0.01\\
19.1010313174946	0.01\\
19.1510340172786	0.01\\
19.2010367170626	0.01\\
19.2510394168467	0.01\\
19.3010421166307	0.01\\
19.3510448164147	0.01\\
19.4010475161987	0.01\\
19.4510502159827	0.01\\
19.5010529157667	0.01\\
19.5510556155508	0.01\\
19.6010583153348	0.01\\
19.6510610151188	0.01\\
19.7010637149028	0.01\\
19.7510664146868	0.01\\
19.8010691144708	0.01\\
19.8510718142549	0.01\\
19.9010745140389	0.01\\
19.9510772138229	0.01\\
20.0010799136069	0.01\\
20.0510826133909	0.01\\
20.1010853131749	0.01\\
20.151088012959	0.01\\
20.201090712743	0.01\\
20.251093412527	0.01\\
20.301096112311	0.01\\
20.351098812095	0.01\\
20.4011015118791	0.01\\
20.4511042116631	0.01\\
20.5011069114471	0.01\\
20.5511096112311	0.01\\
20.6011123110151	0.01\\
20.6511150107991	0.01\\
20.7011177105832	0.01\\
20.7511204103672	0.01\\
20.8011231101512	0.01\\
20.8511258099352	0.01\\
20.9011285097192	0.01\\
20.9511312095032	0.01\\
21.0011339092873	0.01\\
21.0511366090713	0.01\\
21.1011393088553	0.01\\
21.1511420086393	0.01\\
21.2011447084233	0.01\\
21.2511474082073	0.01\\
21.3011501079914	0.01\\
21.3511528077754	0.01\\
21.4011555075594	0.01\\
21.4511582073434	0.01\\
21.5011609071274	0.01\\
21.5511636069114	0.01\\
21.6011663066955	0.01\\
21.6511690064795	0.01\\
21.7011717062635	0.01\\
21.7511744060475	0.01\\
21.8011771058315	0.01\\
21.8511798056156	0.01\\
21.9011825053996	0.01\\
21.9511852051836	0.01\\
22.0011879049676	0.01\\
22.0511906047516	0.01\\
22.1011933045356	0.01\\
22.1511960043197	0.01\\
22.2011987041037	0.01\\
22.2512014038877	0.01\\
22.3012041036717	0.01\\
22.3512068034557	0.01\\
22.4012095032397	0.01\\
22.4512122030238	0.01\\
22.5012149028078	0.01\\
22.5512176025918	0.01\\
22.6012203023758	0.01\\
22.6512230021598	0.01\\
22.7012257019438	0.01\\
22.7512284017279	0.01\\
22.8012311015119	0.01\\
22.8512338012959	0.01\\
22.9012365010799	0.01\\
22.9512392008639	0.01\\
23.0012419006479	0.01\\
23.051244600432	0.01\\
23.101247300216	0.01\\
23.15125	0.01\\
23.201252699784	0.01\\
23.251255399568	0.01\\
23.3012580993521	0.01\\
23.3512607991361	0.01\\
23.4012634989201	0.01\\
23.4512661987041	0.01\\
23.5012688984881	0.01\\
23.5512715982721	0.01\\
23.6012742980562	0.01\\
23.6512769978402	0.01\\
23.7012796976242	0.01\\
23.7512823974082	0.01\\
23.8012850971922	0.01\\
23.8512877969762	0.01\\
23.9012904967603	0.01\\
23.9512931965443	0.01\\
24.0012958963283	0.01\\
24.0512985961123	0.01\\
24.1013012958963	0.01\\
24.1513039956803	0.01\\
24.2013066954644	0.01\\
24.2513093952484	0.01\\
24.3013120950324	0.01\\
24.3513147948164	0.01\\
24.4013174946004	0.01\\
24.4513201943845	0.01\\
24.5013228941685	0.01\\
24.5513255939525	0.01\\
24.6013282937365	0.01\\
24.6513309935205	0.01\\
24.7013336933045	0.01\\
24.7513363930886	0.01\\
24.8013390928726	0.01\\
24.8513417926566	0.01\\
24.9013444924406	0.01\\
24.9513471922246	0.01\\
25.0013498920086	0.01\\
25.0513525917927	0.01\\
25.1013552915767	0.01\\
25.1513579913607	0.01\\
25.2013606911447	0.01\\
25.2513633909287	0.01\\
25.3013660907127	0.01\\
25.3513687904968	0.01\\
25.4013714902808	0.01\\
25.4513741900648	0.01\\
25.5013768898488	0.01\\
25.5513795896328	0.01\\
25.6013822894168	0.01\\
25.6513849892009	0.01\\
25.7013876889849	0.01\\
25.7513903887689	0.01\\
25.8013930885529	0.01\\
25.8513957883369	0.01\\
25.901398488121	0.01\\
25.951401187905	0.01\\
26.001403887689	0.01\\
26.051406587473	0.01\\
26.101409287257	0.01\\
26.151411987041	0.01\\
26.2014146868251	0.01\\
26.2514173866091	0.01\\
26.3014200863931	0.01\\
26.3514227861771	0.01\\
26.4014254859611	0.01\\
26.4514281857451	0.01\\
26.5014308855292	0.01\\
26.5514335853132	0.01\\
26.6014362850972	0.01\\
26.6514389848812	0.01\\
26.7014416846652	0.01\\
26.7514443844492	0.01\\
26.8014470842333	0.01\\
26.8514497840173	0.01\\
26.9014524838013	0.01\\
26.9514551835853	0.01\\
27.0014578833693	0.01\\
27.0514605831534	0.01\\
27.1014632829374	0.01\\
27.1514659827214	0.01\\
27.2014686825054	0.01\\
27.2514713822894	0.01\\
27.3014740820734	0.01\\
27.3514767818575	0.01\\
27.4014794816415	0.01\\
27.4514821814255	0.01\\
27.5014848812095	0.01\\
27.5514875809935	0.01\\
27.6014902807775	0.01\\
27.6514929805616	0.01\\
27.7014956803456	0.01\\
27.7514983801296	0.01\\
27.8015010799136	0.01\\
27.8515037796976	0.01\\
27.9015064794816	0.01\\
27.9515091792657	0.01\\
28.0015118790497	0.01\\
28.0515145788337	0.01\\
28.1015172786177	0.01\\
28.1515199784017	0.01\\
28.2015226781857	0.01\\
28.2515253779698	0.01\\
28.3015280777538	0.01\\
28.3515307775378	0.01\\
28.4015334773218	0.01\\
28.4515361771058	0.01\\
28.5015388768899	0.01\\
28.5515415766739	0.01\\
28.6015442764579	0.01\\
28.6515469762419	0.01\\
28.7015496760259	0.01\\
28.7515523758099	0.01\\
28.801555075594	0.01\\
28.851557775378	0.01\\
28.901560475162	0.01\\
28.951563174946	0.01\\
29.00156587473	0.01\\
29.051568574514	0.01\\
29.1015712742981	0.01\\
29.1515739740821	0.01\\
29.2015766738661	0.01\\
29.2515793736501	0.01\\
29.3015820734341	0.01\\
29.3515847732181	0.01\\
29.4015874730022	0.01\\
29.4515901727862	0.01\\
29.5015928725702	0.01\\
29.5515955723542	0.01\\
29.6015982721382	0.01\\
29.6516009719222	0.01\\
29.7016036717063	0.01\\
29.7516063714903	0.01\\
29.8016090712743	0.01\\
29.8516117710583	0.01\\
29.9016144708423	0.01\\
29.9516171706264	0.01\\
30.0016198704104	0.01\\
30.0516225701944	0.01\\
30.1016252699784	0.01\\
30.1516279697624	0.01\\
30.2016306695464	0.01\\
30.2516333693305	0.01\\
30.3016360691145	0.01\\
30.3516387688985	0.01\\
30.4016414686825	0.01\\
30.4516441684665	0.01\\
30.5016468682505	0.01\\
30.5516495680346	0.01\\
30.6016522678186	0.01\\
30.6516549676026	0.01\\
30.7016576673866	0.01\\
30.7516603671706	0.01\\
30.8016630669546	0.01\\
30.8516657667387	0.01\\
30.9016684665227	0.01\\
30.9516711663067	0.01\\
31.0016738660907	0.01\\
31.0516765658747	0.01\\
31.1016792656587	0.01\\
31.1516819654428	0.01\\
31.2016846652268	0.01\\
31.2516873650108	0.01\\
31.3016900647948	0.01\\
31.3516927645788	0.01\\
31.4016954643629	0.01\\
31.4516981641469	0.01\\
31.5017008639309	0.01\\
31.5517035637149	0.01\\
31.6017062634989	0.01\\
31.6517089632829	0.01\\
31.701711663067	0.01\\
31.751714362851	0.01\\
31.801717062635	0.01\\
31.851719762419	0.01\\
31.901722462203	0.01\\
31.951725161987	0.01\\
32.0017278617711	0.01\\
32.0517305615551	0.01\\
32.1017332613391	0.01\\
32.1517359611231	0.01\\
32.2017386609071	0.01\\
32.2517413606911	0.01\\
32.3017440604752	0.01\\
32.3517467602592	0.01\\
32.4017494600432	0.01\\
32.4517521598272	0.01\\
32.5017548596112	0.01\\
32.5517575593952	0.01\\
32.6017602591793	0.01\\
32.6517629589633	0.01\\
32.7017656587473	0.01\\
32.7517683585313	0.01\\
32.8017710583153	0.01\\
32.8517737580993	0.01\\
32.9017764578834	0.01\\
32.9517791576674	0.01\\
33.0017818574514	0.01\\
33.0517845572354	0.01\\
33.1017872570194	0.01\\
33.1517899568035	0.01\\
33.2017926565875	0.01\\
33.2517953563715	0.01\\
33.3017980561555	0.01\\
33.3518007559395	0.01\\
33.4018034557235	0.01\\
33.4518061555076	0.01\\
33.5018088552916	0.01\\
33.5518115550756	0.01\\
33.6018142548596	0.01\\
33.6518169546436	0.01\\
33.7018196544277	0.01\\
33.7518223542117	0.01\\
33.8018250539957	0.01\\
33.8518277537797	0.01\\
33.9018304535637	0.01\\
33.9518331533477	0.01\\
34.0018358531318	0.01\\
34.0518385529158	0.01\\
34.1018412526998	0.01\\
34.1518439524838	0.01\\
34.2018466522678	0.01\\
34.2518493520518	0.01\\
34.3018520518359	0.01\\
34.3518547516199	0.01\\
34.4018574514039	0.01\\
34.4518601511879	0.01\\
34.5018628509719	0.01\\
34.5518655507559	0.01\\
34.60186825054	0.01\\
34.651870950324	0.01\\
34.701873650108	0.01\\
34.751876349892	0.01\\
34.801879049676	0.01\\
34.85188174946	0.01\\
34.9018844492441	0.01\\
34.9518871490281	0.01\\
35.0018898488121	0.01\\
35.0518925485961	0.01\\
35.1018952483801	0.01\\
35.1518979481641	0.01\\
35.2019006479482	0.01\\
35.2519033477322	0.01\\
35.3019060475162	0.01\\
35.3519087473002	0.01\\
35.4019114470842	0.01\\
35.4519141468683	0.01\\
35.5019168466523	0.01\\
35.5519195464363	0.01\\
35.6019222462203	0.01\\
35.6519249460043	0.01\\
35.7019276457883	0.01\\
35.7519303455724	0.01\\
35.8019330453564	0.01\\
35.8519357451404	0.01\\
35.9019384449244	0.01\\
35.9519411447084	0.01\\
36.0019438444924	0.01\\
36.0519465442765	0.01\\
36.1019492440605	0.01\\
36.1519519438445	0.01\\
36.2019546436285	0.01\\
36.2519573434125	0.01\\
36.3019600431965	0.01\\
36.3519627429806	0.01\\
36.4019654427646	0.01\\
36.4519681425486	0.01\\
36.5019708423326	0.01\\
36.5519735421166	0.01\\
36.6019762419006	0.01\\
36.6519789416847	0.01\\
36.7019816414687	0.01\\
36.7519843412527	0.01\\
36.8019870410367	0.01\\
36.8519897408207	0.01\\
36.9019924406048	0.01\\
36.9519951403888	0.01\\
37.0019978401728	0.01\\
37.0520005399568	0.01\\
37.1020032397408	0.01\\
37.1520059395248	0.01\\
37.2020086393089	0.01\\
37.2520113390929	0.01\\
37.3020140388769	0.01\\
37.3520167386609	0.01\\
37.4020194384449	0.01\\
37.4520221382289	0.01\\
37.502024838013	0.01\\
37.552027537797	0.01\\
37.602030237581	0.01\\
37.652032937365	0.01\\
37.702035637149	0.01\\
37.752038336933	0.01\\
37.8020410367171	0.01\\
37.8520437365011	0.01\\
37.9020464362851	0.01\\
37.9520491360691	0.01\\
38.0020518358531	0.01\\
38.0520545356372	0.01\\
38.1020572354212	0.01\\
38.1520599352052	0.01\\
38.2020626349892	0.01\\
38.2520653347732	0.01\\
38.3020680345572	0.01\\
38.3520707343413	0.01\\
38.4020734341253	0.01\\
38.4520761339093	0.01\\
38.5020788336933	0.01\\
38.5520815334773	0.01\\
38.6020842332613	0.01\\
38.6520869330454	0.01\\
38.7020896328294	0.01\\
38.7520923326134	0.01\\
38.8020950323974	0.01\\
38.8520977321814	0.01\\
38.9021004319654	0.01\\
38.9521031317495	0.01\\
39.0021058315335	0.01\\
39.0521085313175	0.01\\
39.1021112311015	0.01\\
39.1521139308855	0.01\\
39.2021166306696	0.01\\
39.2521193304536	0.01\\
39.3021220302376	0.01\\
39.3521247300216	0.01\\
39.4021274298056	0.01\\
39.4521301295896	0.01\\
39.5021328293737	0.01\\
39.5521355291577	0.01\\
39.6021382289417	0.01\\
39.6521409287257	0.01\\
39.7021436285097	0.01\\
39.7521463282937	0.01\\
39.8021490280778	0.01\\
39.8521517278618	0.01\\
39.9021544276458	0.01\\
39.9521571274298	0.01\\
40.0021598272138	0.01\\
40.0521625269978	0.01\\
40.1021652267819	0.01\\
40.1521679265659	0.01\\
40.2021706263499	0.01\\
40.2521733261339	0.01\\
40.3021760259179	0.01\\
40.3521787257019	0.01\\
40.402181425486	0.01\\
40.45218412527	0.01\\
40.502186825054	0.01\\
40.552189524838	0.01\\
40.602192224622	0.01\\
40.652194924406	0.01\\
40.7021976241901	0.01\\
40.7522003239741	0.01\\
40.8022030237581	0.01\\
40.8522057235421	0.01\\
40.9022084233261	0.01\\
40.9522111231102	0.01\\
41.0022138228942	0.01\\
41.0522165226782	0.01\\
41.1022192224622	0.01\\
41.1522219222462	0.01\\
41.2022246220302	0.01\\
41.2522273218143	0.01\\
41.3022300215983	0.01\\
41.3522327213823	0.01\\
41.4022354211663	0.01\\
41.4522381209503	0.01\\
41.5022408207343	0.01\\
41.5522435205184	0.01\\
41.6022462203024	0.01\\
41.6522489200864	0.01\\
41.7022516198704	0.01\\
41.7522543196544	0.01\\
41.8022570194385	0.01\\
41.8522597192225	0.01\\
41.9022624190065	0.01\\
41.9522651187905	0.01\\
42.0022678185745	0.01\\
42.0522705183585	0.01\\
42.1022732181426	0.01\\
42.1522759179266	0.01\\
42.2022786177106	0.01\\
42.2522813174946	0.01\\
42.3022840172786	0.01\\
42.3522867170626	0.01\\
42.4022894168467	0.01\\
42.4522921166307	0.01\\
42.5022948164147	0.01\\
42.5522975161987	0.01\\
42.6023002159827	0.01\\
42.6523029157667	0.01\\
42.7023056155508	0.01\\
42.7523083153348	0.01\\
42.8023110151188	0.01\\
42.8523137149028	0.01\\
42.9023164146868	0.01\\
42.9523191144708	0.01\\
43.0023218142549	0.01\\
43.0523245140389	0.01\\
43.1023272138229	0.01\\
43.1523299136069	0.01\\
43.2023326133909	0.01\\
43.2523353131749	0.01\\
43.302338012959	0.01\\
43.352340712743	0.01\\
43.402343412527	0.01\\
43.452346112311	0.01\\
43.502348812095	0.01\\
43.5523515118791	0.01\\
43.6023542116631	0.01\\
43.6523569114471	0.01\\
43.7023596112311	0.01\\
43.7523623110151	0.01\\
43.8023650107991	0.01\\
43.8523677105832	0.01\\
43.9023704103672	0.01\\
43.9523731101512	0.01\\
44.0023758099352	0.01\\
44.0523785097192	0.01\\
44.1023812095032	0.01\\
44.1523839092873	0.01\\
44.2023866090713	0.01\\
44.2523893088553	0.01\\
44.3023920086393	0.01\\
44.3523947084233	0.01\\
44.4023974082073	0.01\\
44.4524001079914	0.01\\
44.5024028077754	0.01\\
44.5524055075594	0.01\\
44.6024082073434	0.01\\
44.6524109071274	0.01\\
44.7024136069114	0.01\\
44.7524163066955	0.01\\
44.8024190064795	0.01\\
44.8524217062635	0.01\\
44.9024244060475	0.01\\
44.9524271058315	0.01\\
45.0024298056155	0.01\\
45.0524325053996	0.01\\
45.1024352051836	0.01\\
45.1524379049676	0.01\\
45.2024406047516	0.01\\
45.2524433045356	0.01\\
45.3024460043197	0.01\\
45.3524487041037	0.01\\
45.4024514038877	0.01\\
45.4524541036717	0.01\\
45.5024568034557	0.01\\
45.5524595032397	0.01\\
45.6024622030238	0.01\\
45.6524649028078	0.01\\
45.7024676025918	0.01\\
45.7524703023758	0.01\\
45.8024730021598	0.01\\
45.8524757019439	0.01\\
45.9024784017279	0.01\\
45.9524811015119	0.01\\
46.0024838012959	0.01\\
46.0524865010799	0.01\\
46.1024892008639	0.01\\
46.152491900648	0.01\\
46.202494600432	0.01\\
46.252497300216	0.01\\
46.3025	0.01\\
46.352502699784	0.01\\
46.402505399568	0.01\\
46.4525080993521	0.01\\
46.5025107991361	0.01\\
46.5525134989201	0.01\\
46.6025161987041	0.01\\
46.6525188984881	0.01\\
46.7025215982721	0.01\\
46.7525242980562	0.01\\
46.8025269978402	0.01\\
46.8525296976242	0.01\\
46.9025323974082	0.01\\
46.9525350971922	0.01\\
47.0025377969762	0.01\\
47.0525404967603	0.01\\
47.1025431965443	0.01\\
47.1525458963283	0.01\\
47.2025485961123	0.01\\
47.2525512958963	0.01\\
47.3025539956803	0.01\\
47.3525566954644	0.01\\
47.4025593952484	0.01\\
47.4525620950324	0.01\\
47.5025647948164	0.01\\
47.5525674946004	0.01\\
47.6025701943845	0.01\\
47.6525728941685	0.01\\
47.7025755939525	0.01\\
47.7525782937365	0.01\\
47.8025809935205	0.01\\
47.8525836933045	0.01\\
47.9025863930886	0.01\\
47.9525890928726	0.01\\
48.0025917926566	0.01\\
48.0525944924406	0.01\\
48.1025971922246	0.01\\
48.1525998920086	0.01\\
48.2026025917927	0.01\\
48.2526052915767	0.01\\
48.3026079913607	0.01\\
48.3526106911447	0.01\\
48.4026133909287	0.01\\
48.4526160907127	0.01\\
48.5026187904968	0.01\\
48.5526214902808	0.01\\
48.6026241900648	0.01\\
48.6526268898488	0.01\\
48.7026295896328	0.01\\
48.7526322894168	0.01\\
48.8026349892009	0.01\\
48.8526376889849	0.01\\
48.9026403887689	0.01\\
48.9526430885529	0.01\\
49.0026457883369	0.01\\
49.052648488121	0.01\\
49.102651187905	0.01\\
49.152653887689	0.01\\
49.202656587473	0.01\\
49.252659287257	0.01\\
49.302661987041	0.01\\
49.3526646868251	0.01\\
49.4026673866091	0.01\\
49.4526700863931	0.01\\
49.5026727861771	0.01\\
49.5526754859611	0.01\\
49.6026781857451	0.01\\
49.6526808855292	0.01\\
49.7026835853132	0.01\\
49.7526862850972	0.01\\
49.8026889848812	0.01\\
49.8526916846652	0.01\\
49.9026943844492	0.01\\
49.9526970842333	0.01\\
50.0026997840173	0.01\\
50.0527024838013	0.01\\
50.1027051835853	0.01\\
50.1527078833693	0.01\\
50.2027105831534	0.01\\
50.2527132829374	0.01\\
50.3027159827214	0.01\\
50.3527186825054	0.01\\
50.4027213822894	0.01\\
50.4527240820734	0.01\\
50.5027267818575	0.01\\
50.5527294816415	0.01\\
50.6027321814255	0.01\\
50.6527348812095	0.01\\
50.7027375809935	0.01\\
50.7527402807775	0.01\\
50.8027429805616	0.01\\
50.8527456803456	0.01\\
50.9027483801296	0.01\\
50.9527510799136	0.01\\
51.0027537796976	0.01\\
51.0527564794816	0.01\\
51.1027591792657	0.01\\
51.1527618790497	0.01\\
51.2027645788337	0.01\\
51.2527672786177	0.01\\
51.3027699784017	0.01\\
51.3527726781858	0.01\\
51.4027753779698	0.01\\
51.4527780777538	0.01\\
51.5027807775378	0.01\\
51.5527834773218	0.01\\
51.6027861771058	0.01\\
51.6527888768899	0.01\\
51.7027915766739	0.01\\
51.7527942764579	0.01\\
51.8027969762419	0.01\\
51.8527996760259	0.01\\
51.9028023758099	0.01\\
51.952805075594	0.01\\
52.002807775378	0.01\\
52.052810475162	0.01\\
52.102813174946	0.01\\
52.15281587473	0.01\\
52.202818574514	0.01\\
52.2528212742981	0.01\\
52.3028239740821	0.01\\
52.3528266738661	0.01\\
52.4028293736501	0.01\\
52.4528320734341	0.01\\
52.5028347732181	0.01\\
52.5528374730022	0.01\\
52.6028401727862	0.01\\
52.6528428725702	0.01\\
52.7028455723542	0.01\\
52.7528482721382	0.01\\
52.8028509719222	0.01\\
52.8528536717063	0.01\\
52.9028563714903	0.01\\
52.9528590712743	0.01\\
53.0028617710583	0.01\\
53.0528644708423	0.01\\
53.1028671706264	0.01\\
53.1528698704104	0.01\\
53.2028725701944	0.01\\
53.2528752699784	0.01\\
53.3028779697624	0.01\\
53.3528806695464	0.01\\
53.4028833693305	0.01\\
53.4528860691145	0.01\\
53.5028887688985	0.01\\
53.5528914686825	0.01\\
53.6028941684665	0.01\\
53.6528968682505	0.01\\
53.7028995680346	0.01\\
53.7529022678186	0.01\\
53.8029049676026	0.01\\
53.8529076673866	0.01\\
53.9029103671706	0.01\\
53.9529130669547	0.01\\
54.0029157667387	0.01\\
54.0529184665227	0.01\\
54.1029211663067	0.01\\
54.1529238660907	0.01\\
54.2029265658747	0.01\\
54.2529292656588	0.01\\
54.3029319654428	0.01\\
54.3529346652268	0.01\\
54.4029373650108	0.01\\
54.4529400647948	0.01\\
54.5029427645788	0.01\\
54.5529454643629	0.01\\
54.6029481641469	0.01\\
54.6529508639309	0.01\\
54.7029535637149	0.01\\
54.7529562634989	0.01\\
54.8029589632829	0.01\\
54.852961663067	0.01\\
54.902964362851	0.01\\
54.952967062635	0.01\\
55.002969762419	0.01\\
55.052972462203	0.01\\
55.102975161987	0.01\\
55.1529778617711	0.01\\
55.2029805615551	0.01\\
55.2529832613391	0.01\\
55.3029859611231	0.01\\
55.3529886609071	0.01\\
55.4029913606911	0.01\\
55.4529940604752	0.01\\
55.5029967602592	0.01\\
55.5529994600432	0.01\\
55.6030021598272	0.01\\
55.6530048596112	0.01\\
55.7030075593953	0.01\\
55.7530102591793	0.01\\
55.8030129589633	0.01\\
55.8530156587473	0.01\\
55.9030183585313	0.01\\
55.9530210583153	0.01\\
56.0030237580994	0.01\\
56.0530264578834	0.01\\
56.1030291576674	0.01\\
56.1530318574514	0.01\\
56.2030345572354	0.01\\
56.2530372570194	0.01\\
56.3030399568035	0.01\\
56.3530426565875	0.01\\
56.4030453563715	0.01\\
56.4530480561555	0.01\\
56.5030507559395	0.01\\
56.5530534557235	0.01\\
56.6030561555076	0.01\\
56.6530588552916	0.01\\
56.7030615550756	0.01\\
56.7530642548596	0.01\\
56.8030669546436	0.01\\
56.8530696544276	0.01\\
56.9030723542117	0.01\\
56.9530750539957	0.01\\
57.0030777537797	0.01\\
57.0530804535637	0.01\\
57.1030831533477	0.01\\
57.1530858531318	0.01\\
57.2030885529158	0.01\\
57.2530912526998	0.01\\
57.3030939524838	0.01\\
57.3530966522678	0.01\\
57.4030993520518	0.01\\
57.4531020518358	0.01\\
57.5031047516199	0.01\\
57.5531074514039	0.01\\
57.6031101511879	0.01\\
57.6531128509719	0.01\\
57.7031155507559	0.01\\
57.75311825054	0.01\\
57.803120950324	0.01\\
57.853123650108	0.01\\
57.903126349892	0.01\\
57.953129049676	0.01\\
58.00313174946	0.01\\
58.0531344492441	0.01\\
58.1031371490281	0.01\\
58.1531398488121	0.01\\
58.2031425485961	0.01\\
58.2531452483801	0.01\\
58.3031479481642	0.01\\
58.3531506479482	0.01\\
58.4031533477322	0.01\\
58.4531560475162	0.01\\
58.5031587473002	0.01\\
58.5531614470842	0.01\\
58.6031641468683	0.01\\
58.6531668466523	0.01\\
58.7031695464363	0.01\\
58.7531722462203	0.01\\
58.8031749460043	0.01\\
58.8531776457883	0.01\\
58.9031803455724	0.01\\
58.9531830453564	0.01\\
59.0031857451404	0.01\\
59.0531884449244	0.01\\
59.1031911447084	0.01\\
59.1531938444924	0.01\\
59.2031965442765	0.01\\
59.2531992440605	0.01\\
59.3032019438445	0.01\\
59.3532046436285	0.01\\
59.4032073434125	0.01\\
59.4532100431965	0.01\\
59.5032127429806	0.01\\
59.5532154427646	0.01\\
59.6032181425486	0.01\\
59.6532208423326	0.01\\
59.7032235421166	0.01\\
59.7532262419006	0.01\\
59.8032289416847	0.01\\
59.8532316414687	0.01\\
59.9032343412527	0.01\\
59.9532370410367	0.01\\
60.0032397408207	0.01\\
60.0532424406048	0.01\\
60.1032451403888	0.01\\
60.1532478401728	0.01\\
60.2032505399568	0.01\\
60.2532532397408	0.01\\
60.3032559395248	0.01\\
60.3532586393089	0.01\\
60.4032613390929	0.01\\
60.4532640388769	0.01\\
60.5032667386609	0.01\\
60.5532694384449	0.01\\
60.6032721382289	0.01\\
60.653274838013	0.01\\
60.703277537797	0.01\\
60.753280237581	0.01\\
60.803282937365	0.01\\
60.853285637149	0.01\\
60.9032883369331	0.01\\
60.9532910367171	0.01\\
61.0032937365011	0.01\\
61.0532964362851	0.01\\
61.1032991360691	0.01\\
61.1533018358531	0.01\\
61.2033045356371	0.01\\
61.2533072354212	0.01\\
61.3033099352052	0.01\\
61.3533126349892	0.01\\
61.4033153347732	0.01\\
61.4533180345572	0.01\\
61.5033207343413	0.01\\
61.5533234341253	0.01\\
61.6033261339093	0.01\\
61.6533288336933	0.01\\
61.7033315334773	0.01\\
61.7533342332613	0.01\\
61.8033369330454	0.01\\
61.8533396328294	0.01\\
61.9033423326134	0.01\\
61.9533450323974	0.01\\
62.0033477321814	0.01\\
62.0533504319654	0.01\\
62.1033531317495	0.01\\
62.1533558315335	0.01\\
62.2033585313175	0.01\\
62.2533612311015	0.01\\
62.3033639308855	0.01\\
62.3533666306695	0.01\\
62.4033693304536	0.01\\
62.4533720302376	0.01\\
62.5033747300216	0.01\\
62.5483771598272	0.21\\
62.5983798596112	0.21\\
62.6483825593953	0.21\\
62.6983852591793	0.21\\
62.7483879589633	0.21\\
62.7983906587473	0.21\\
62.8483933585313	0.21\\
62.8983960583153	0.21\\
62.9483987580994	0.21\\
62.9984014578834	0.21\\
63.0484041576674	0.21\\
63.0984068574514	0.21\\
63.1484095572354	0.21\\
63.1984122570194	0.21\\
63.2484149568035	0.21\\
63.2984176565875	0.21\\
63.3484203563715	0.21\\
63.3984230561555	0.21\\
63.4484257559395	0.21\\
63.4984284557235	0.21\\
63.5484311555076	0.21\\
63.5984338552916	0.21\\
63.6484365550756	0.21\\
63.6984392548596	0.21\\
63.7484419546436	0.21\\
63.7984446544276	0.21\\
63.8484473542117	0.21\\
63.8984500539957	0.21\\
63.9484527537797	0.21\\
63.9984554535637	0.21\\
64.0484581533477	0.21\\
64.0984608531318	0.21\\
64.1484635529158	0.21\\
64.1984662526998	0.21\\
64.2484689524838	0.21\\
64.2984716522678	0.21\\
64.3484743520518	0.21\\
64.3984770518359	0.21\\
64.4484797516199	0.21\\
64.4984824514039	0.21\\
64.5484851511879	0.21\\
64.5984878509719	0.21\\
64.648490550756	0.21\\
64.69849325054	0.21\\
64.748495950324	0.21\\
64.798498650108	0.21\\
64.848501349892	0.21\\
64.898504049676	0.21\\
64.94850674946	0.21\\
64.9985094492441	0.21\\
65.0485121490281	0.21\\
65.0985148488121	0.21\\
65.1485175485961	0.21\\
65.1985202483801	0.21\\
65.2485229481642	0.21\\
65.2985256479482	0.21\\
65.3485283477322	0.21\\
65.3985310475162	0.21\\
65.4485337473002	0.21\\
65.4985364470842	0.21\\
65.5485391468683	0.21\\
65.5985418466523	0.21\\
65.6485445464363	0.21\\
65.6985472462203	0.21\\
65.7485499460043	0.21\\
65.7985526457883	0.21\\
65.8485553455724	0.21\\
65.8985580453564	0.21\\
65.9485607451404	0.21\\
65.9985634449244	0.21\\
66.0485661447084	0.21\\
66.0985688444924	0.21\\
66.1485715442765	0.21\\
66.1985742440605	0.21\\
66.2485769438445	0.21\\
66.2985796436285	0.21\\
66.3485823434125	0.21\\
66.3985850431965	0.21\\
66.4485877429806	0.21\\
66.4985904427646	0.21\\
66.5485931425486	0.21\\
66.5985958423326	0.21\\
66.6485985421166	0.21\\
66.6986012419006	0.21\\
66.7486039416847	0.21\\
66.7986066414687	0.21\\
66.8486093412527	0.21\\
66.8986120410367	0.21\\
66.9486147408207	0.21\\
66.9986174406047	0.21\\
67.0486201403888	0.21\\
67.0986228401728	0.21\\
67.1486255399568	0.21\\
67.1986282397408	0.21\\
67.2486309395248	0.21\\
67.2986336393089	0.21\\
67.3486363390929	0.21\\
67.3986390388769	0.21\\
67.4486417386609	0.21\\
67.4986444384449	0.21\\
67.5486471382289	0.21\\
67.598649838013	0.21\\
67.648652537797	0.21\\
67.698655237581	0.21\\
67.748657937365	0.21\\
67.798660637149	0.21\\
67.8486633369331	0.21\\
67.8986660367171	0.21\\
67.9486687365011	0.21\\
67.9986714362851	0.21\\
68.0486741360691	0.21\\
68.0986768358531	0.21\\
68.1486795356372	0.21\\
68.1986822354212	0.21\\
68.2486849352052	0.21\\
68.2986876349892	0.21\\
68.3486903347732	0.21\\
68.3986930345572	0.21\\
68.4486957343413	0.21\\
68.4986984341253	0.21\\
68.5487011339093	0.21\\
68.5987038336933	0.21\\
68.6487065334773	0.21\\
68.6987092332613	0.21\\
68.7487119330454	0.21\\
68.7987146328294	0.21\\
68.8487173326134	0.21\\
68.8987200323974	0.21\\
68.9487227321814	0.21\\
68.9987254319654	0.21\\
69.0487281317495	0.21\\
69.0987308315335	0.21\\
69.1487335313175	0.21\\
69.1987362311015	0.21\\
69.2487389308855	0.21\\
69.2987416306696	0.21\\
69.3487443304536	0.21\\
69.3987470302376	0.21\\
69.4487497300216	0.21\\
69.4987524298056	0.21\\
69.5487551295896	0.21\\
69.5987578293737	0.21\\
69.6487605291577	0.21\\
69.6987632289417	0.21\\
69.7487659287257	0.21\\
69.7987686285097	0.21\\
69.8487713282937	0.21\\
69.8987740280778	0.21\\
69.9487767278618	0.21\\
69.9987794276458	0.21\\
70.0487821274298	0.21\\
70.0987848272138	0.21\\
70.1487875269979	0.21\\
70.1987902267819	0.21\\
70.2487929265659	0.21\\
70.2987956263499	0.21\\
70.3487983261339	0.21\\
70.3988010259179	0.21\\
70.4488037257019	0.21\\
70.498806425486	0.21\\
70.54880912527	0.21\\
70.598811825054	0.21\\
70.648814524838	0.21\\
70.698817224622	0.21\\
70.7488199244061	0.21\\
70.7988226241901	0.21\\
70.8488253239741	0.21\\
70.8988280237581	0.21\\
70.9488307235421	0.21\\
70.9988334233261	0.21\\
71.0488361231101	0.21\\
71.0988388228942	0.21\\
71.1488415226782	0.21\\
71.1988442224622	0.21\\
71.2488469222462	0.21\\
71.2988496220302	0.21\\
71.3488523218143	0.21\\
71.3988550215983	0.21\\
71.4488577213823	0.21\\
71.4988604211663	0.21\\
71.5488631209503	0.21\\
71.5988658207343	0.21\\
71.6488685205184	0.21\\
71.6988712203024	0.21\\
71.7488739200864	0.21\\
71.7988766198704	0.21\\
71.8488793196544	0.21\\
71.8988820194385	0.21\\
71.9488847192225	0.21\\
71.9988874190065	0.21\\
72.0488901187905	0.21\\
72.0988928185745	0.21\\
72.1488955183585	0.21\\
72.1988982181425	0.21\\
72.2489009179266	0.21\\
72.2989036177106	0.21\\
72.3489063174946	0.21\\
72.3989090172786	0.21\\
72.4489117170626	0.21\\
72.4989144168467	0.21\\
72.5489171166307	0.21\\
72.5989198164147	0.21\\
72.6489225161987	0.21\\
72.6989252159827	0.21\\
72.7489279157667	0.21\\
72.7989306155508	0.21\\
72.8489333153348	0.21\\
72.8989360151188	0.21\\
72.9489387149028	0.21\\
72.9989414146868	0.21\\
73.0489441144708	0.21\\
73.0989468142549	0.21\\
73.1489495140389	0.21\\
73.1989522138229	0.21\\
73.2489549136069	0.21\\
73.2989576133909	0.21\\
73.348960313175	0.21\\
73.398963012959	0.21\\
73.448965712743	0.21\\
73.498968412527	0.21\\
73.548971112311	0.21\\
73.598973812095	0.21\\
73.6489765118791	0.21\\
73.6989792116631	0.21\\
73.7489819114471	0.21\\
73.7989846112311	0.21\\
73.8489873110151	0.21\\
73.8989900107991	0.21\\
73.9489927105832	0.21\\
73.9989954103672	0.21\\
74.0489981101512	0.21\\
74.0990008099352	0.21\\
74.1490035097192	0.21\\
74.1990062095032	0.21\\
74.2490089092873	0.21\\
74.2990116090713	0.21\\
74.3490143088553	0.21\\
74.3990170086393	0.21\\
74.4490197084233	0.21\\
74.4990224082073	0.21\\
74.5490251079914	0.21\\
74.5990278077754	0.21\\
74.6490305075594	0.21\\
74.6990332073434	0.21\\
74.7490359071274	0.21\\
74.7990386069115	0.21\\
74.8490413066955	0.21\\
74.8990440064795	0.21\\
74.9490467062635	0.21\\
74.9990494060475	0.21\\
75.0490521058315	0.21\\
75.0990548056156	0.21\\
75.1490575053996	0.21\\
75.1990602051836	0.21\\
75.2490629049676	0.21\\
75.2990656047516	0.21\\
75.3490683045356	0.21\\
75.3990710043197	0.21\\
75.4490737041037	0.21\\
75.4990764038877	0.21\\
75.5490791036717	0.21\\
75.5990818034557	0.21\\
75.6490845032397	0.21\\
75.6990872030238	0.21\\
75.7490899028078	0.21\\
75.7990926025918	0.21\\
75.8490953023758	0.21\\
75.8990980021598	0.21\\
75.9491007019438	0.21\\
75.9991034017279	0.21\\
76.0491061015119	0.21\\
76.0991088012959	0.21\\
76.1491115010799	0.21\\
76.1991142008639	0.21\\
76.2491169006479	0.21\\
76.299119600432	0.21\\
76.349122300216	0.21\\
76.399125	0.21\\
76.449127699784	0.21\\
76.499130399568	0.21\\
76.5491330993521	0.21\\
76.5991357991361	0.21\\
76.6491384989201	0.21\\
76.6991411987041	0.21\\
76.7491438984881	0.21\\
76.7991465982721	0.21\\
76.8491492980562	0.21\\
76.8991519978402	0.21\\
76.9491546976242	0.21\\
76.9991573974082	0.21\\
77.0491600971922	0.21\\
77.0991627969762	0.21\\
77.1491654967603	0.21\\
77.1991681965443	0.21\\
77.2491708963283	0.21\\
77.2991735961123	0.21\\
77.3491762958963	0.21\\
77.3991789956804	0.21\\
77.4491816954644	0.21\\
77.4991843952484	0.21\\
77.5491870950324	0.21\\
77.5991897948164	0.21\\
77.6491924946004	0.21\\
77.6991951943844	0.21\\
77.7491978941685	0.21\\
77.7992005939525	0.21\\
77.8492032937365	0.21\\
77.8992059935205	0.21\\
77.9492086933045	0.21\\
77.9992113930886	0.21\\
78.0492140928726	0.21\\
78.0992167926566	0.21\\
78.1492194924406	0.21\\
78.1992221922246	0.21\\
78.2492248920086	0.21\\
78.2992275917927	0.21\\
78.3492302915767	0.21\\
78.3992329913607	0.21\\
78.4492356911447	0.21\\
78.4992383909287	0.21\\
78.5492410907127	0.21\\
78.5992437904968	0.21\\
78.6492464902808	0.21\\
78.6992491900648	0.21\\
78.7492518898488	0.21\\
78.7992545896328	0.21\\
78.8492572894169	0.21\\
78.8992599892009	0.21\\
78.9492626889849	0.21\\
78.9992653887689	0.21\\
79.0492680885529	0.21\\
79.0992707883369	0.21\\
79.149273488121	0.21\\
79.199276187905	0.21\\
79.249278887689	0.21\\
79.299281587473	0.21\\
79.349284287257	0.21\\
79.3992869870411	0.21\\
79.4492896868251	0.21\\
79.4992923866091	0.21\\
79.5492950863931	0.21\\
79.5992977861771	0.21\\
79.6493004859611	0.21\\
79.6993031857451	0.21\\
79.7493058855292	0.21\\
79.7993085853132	0.21\\
79.8493112850972	0.21\\
79.8993139848812	0.21\\
79.9493166846652	0.21\\
79.9993193844493	0.21\\
80.0493220842333	0.21\\
80.0993247840173	0.21\\
80.1493274838013	0.21\\
80.1993301835853	0.21\\
80.2493328833693	0.21\\
80.2993355831534	0.21\\
80.3493382829374	0.21\\
80.3993409827214	0.21\\
80.4493436825054	0.21\\
80.4993463822894	0.21\\
80.5493490820734	0.21\\
80.5993517818575	0.21\\
80.6493544816415	0.21\\
80.6993571814255	0.21\\
80.7493598812095	0.21\\
80.7993625809935	0.21\\
80.8493652807775	0.21\\
80.8993679805616	0.21\\
80.9493706803456	0.21\\
80.9993733801296	0.21\\
81.0493760799136	0.21\\
81.0993787796976	0.21\\
81.1493814794816	0.21\\
81.1993841792657	0.21\\
81.2493868790497	0.21\\
81.2993895788337	0.21\\
81.3493922786177	0.21\\
81.3993949784017	0.21\\
81.4493976781857	0.21\\
81.4994003779698	0.21\\
81.5494030777538	0.21\\
81.5994057775378	0.21\\
81.6494084773218	0.21\\
81.6994111771058	0.21\\
81.7494138768898	0.21\\
81.7994165766739	0.21\\
81.8494192764579	0.21\\
81.8994219762419	0.21\\
81.9494246760259	0.21\\
81.9994273758099	0.21\\
82.049430075594	0.21\\
82.099432775378	0.21\\
82.149435475162	0.21\\
82.199438174946	0.21\\
82.24944087473	0.21\\
82.299443574514	0.21\\
82.3494462742981	0.21\\
82.3994489740821	0.21\\
82.4494516738661	0.21\\
82.4994543736501	0.21\\
82.5494570734341	0.21\\
82.5994597732181	0.21\\
82.6294613930885	0.01\\
82.6794640928726	0.01\\
82.7294667926566	0.01\\
82.7794694924406	0.01\\
82.8294721922246	0.01\\
82.8794748920086	0.01\\
82.9294775917927	0.01\\
82.9794802915767	0.01\\
83.0294829913607	0.01\\
83.0794856911447	0.01\\
83.1294883909287	0.01\\
83.1794910907127	0.01\\
83.2294937904968	0.01\\
83.2794964902808	0.01\\
83.3294991900648	0.01\\
83.3795018898488	0.01\\
83.4295045896328	0.01\\
83.4795072894169	0.01\\
83.5295099892009	0.01\\
83.5795126889849	0.01\\
83.6295153887689	0.01\\
83.6795180885529	0.01\\
83.7295207883369	0.01\\
83.779523488121	0.01\\
83.829526187905	0.01\\
83.879528887689	0.01\\
83.929531587473	0.01\\
83.979534287257	0.01\\
84.029536987041	0.01\\
84.0795396868251	0.01\\
84.1295423866091	0.01\\
84.1795450863931	0.01\\
84.2295477861771	0.01\\
84.2795504859611	0.01\\
84.3295531857451	0.01\\
84.3795558855292	0.01\\
84.4295585853132	0.01\\
84.4795612850972	0.01\\
84.5295639848812	0.01\\
84.5795666846652	0.01\\
84.6295693844492	0.01\\
84.6795720842333	0.01\\
84.7295747840173	0.01\\
84.7795774838013	0.01\\
84.8295801835853	0.01\\
84.8795828833693	0.01\\
84.9295855831534	0.01\\
84.9795882829374	0.01\\
85.0295909827214	0.01\\
85.0795936825054	0.01\\
85.1295963822894	0.01\\
85.1795990820734	0.01\\
85.2296017818575	0.01\\
85.2796044816415	0.01\\
85.3296071814255	0.01\\
85.3796098812095	0.01\\
85.4296125809935	0.01\\
85.4796152807775	0.01\\
85.5296179805616	0.01\\
85.5796206803456	0.01\\
85.6296233801296	0.01\\
85.6796260799136	0.01\\
85.7296287796976	0.01\\
85.7796314794817	0.01\\
85.8296341792657	0.01\\
85.8796368790497	0.01\\
85.9296395788337	0.01\\
85.9796422786177	0.01\\
86.0296449784017	0.01\\
86.0796476781857	0.01\\
86.1296503779698	0.01\\
86.1796530777538	0.01\\
86.2296557775378	0.01\\
86.2796584773218	0.01\\
86.3296611771058	0.01\\
86.3796638768899	0.01\\
86.4296665766739	0.01\\
86.4796692764579	0.01\\
86.5296719762419	0.01\\
86.5796746760259	0.01\\
86.6296773758099	0.01\\
86.679680075594	0.01\\
86.729682775378	0.01\\
86.779685475162	0.01\\
86.829688174946	0.01\\
86.87969087473	0.01\\
86.929693574514	0.01\\
86.9796962742981	0.01\\
87.0296989740821	0.01\\
87.0797016738661	0.01\\
87.1297043736501	0.01\\
87.1797070734341	0.01\\
87.2297097732181	0.01\\
87.2797124730022	0.01\\
87.3297151727862	0.01\\
87.3797178725702	0.01\\
87.4297205723542	0.01\\
87.4797232721382	0.01\\
87.5297259719222	0.01\\
87.5797286717063	0.01\\
87.6297313714903	0.01\\
87.6797340712743	0.01\\
87.7297367710583	0.01\\
87.7797394708423	0.01\\
87.8297421706264	0.01\\
87.8797448704104	0.01\\
87.9297475701944	0.01\\
87.9797502699784	0.01\\
88.0297529697624	0.01\\
88.0797556695464	0.01\\
88.1297583693305	0.01\\
88.1797610691145	0.01\\
88.2297637688985	0.01\\
88.2797664686825	0.01\\
88.3297691684665	0.01\\
88.3797718682505	0.01\\
88.4297745680346	0.01\\
88.4797772678186	0.01\\
88.5297799676026	0.01\\
88.5797826673866	0.01\\
88.6297853671706	0.01\\
88.6797880669546	0.01\\
88.7297907667387	0.01\\
88.7797934665227	0.01\\
88.8297961663067	0.01\\
88.8797988660907	0.01\\
88.9298015658747	0.01\\
88.9798042656588	0.01\\
89.0298069654428	0.01\\
89.0798096652268	0.01\\
89.1298123650108	0.01\\
89.1798150647948	0.01\\
89.2298177645788	0.01\\
89.2798204643629	0.01\\
89.3298231641469	0.01\\
89.3798258639309	0.01\\
89.4298285637149	0.01\\
89.4798312634989	0.01\\
89.5298339632829	0.01\\
89.579836663067	0.01\\
89.629839362851	0.01\\
89.679842062635	0.01\\
89.729844762419	0.01\\
89.779847462203	0.01\\
89.8298501619871	0.01\\
89.8798528617711	0.01\\
89.9298555615551	0.01\\
89.9798582613391	0.01\\
90.0298609611231	0.01\\
90.0798636609071	0.01\\
90.1298663606911	0.01\\
90.1798690604752	0.01\\
90.2298717602592	0.01\\
90.2798744600432	0.01\\
90.3298771598272	0.01\\
90.3798798596112	0.01\\
90.4298825593953	0.01\\
90.4798852591793	0.01\\
90.5298879589633	0.01\\
90.5798906587473	0.01\\
90.6298933585313	0.01\\
90.6798960583153	0.01\\
90.7298987580994	0.01\\
90.7799014578834	0.01\\
90.8299041576674	0.01\\
90.8799068574514	0.01\\
90.9299095572354	0.01\\
90.9799122570195	0.01\\
91.0299149568035	0.01\\
91.0799176565875	0.01\\
91.1299203563715	0.01\\
91.1799230561555	0.01\\
91.2299257559395	0.01\\
91.2799284557236	0.01\\
91.3299311555076	0.01\\
91.3799338552916	0.01\\
91.4299365550756	0.01\\
91.4799392548596	0.01\\
91.5299419546436	0.01\\
91.5799446544276	0.01\\
91.6299473542117	0.01\\
91.6799500539957	0.01\\
91.7299527537797	0.01\\
91.7799554535637	0.01\\
91.8299581533477	0.01\\
91.8799608531317	0.01\\
91.9299635529158	0.01\\
91.9799662526998	0.01\\
92.0299689524838	0.01\\
92.0799716522678	0.01\\
92.1299743520518	0.01\\
92.1799770518358	0.01\\
92.2299797516199	0.01\\
92.2799824514039	0.01\\
92.3299851511879	0.01\\
92.3799878509719	0.01\\
92.4299905507559	0.01\\
92.47999325054	0.01\\
92.529995950324	0.01\\
92.579998650108	0.01\\
92.605	0.01\\
};
\addlegendentry{+ 1cm};

\addplot [color=gray,solid,line width=0.2pt]
  table[row sep=crcr]{0	-0.01\\
0.0500026997840173	-0.01\\
0.100005399568035	-0.01\\
0.150008099352052	-0.01\\
0.200010799136069	-0.01\\
0.250013498920086	-0.01\\
0.300016198704104	-0.01\\
0.350018898488121	-0.01\\
0.400021598272138	-0.01\\
0.450024298056156	-0.01\\
0.500026997840173	-0.01\\
0.55002969762419	-0.01\\
0.600032397408207	-0.01\\
0.650035097192225	-0.01\\
0.700037796976242	-0.01\\
0.750040496760259	-0.01\\
0.800043196544276	-0.01\\
0.850045896328294	-0.01\\
0.900048596112311	-0.01\\
0.950051295896328	-0.01\\
1.00005399568035	-0.01\\
1.05005669546436	-0.01\\
1.10005939524838	-0.01\\
1.1500620950324	-0.01\\
1.20006479481641	-0.01\\
1.25006749460043	-0.01\\
1.30007019438445	-0.01\\
1.35007289416847	-0.01\\
1.40007559395248	-0.01\\
1.4500782937365	-0.01\\
1.50008099352052	-0.01\\
1.55008369330454	-0.01\\
1.60008639308855	-0.01\\
1.65008909287257	-0.01\\
1.70009179265659	-0.01\\
1.7500944924406	-0.01\\
1.80009719222462	-0.01\\
1.85009989200864	-0.01\\
1.90010259179266	-0.01\\
1.95010529157667	-0.01\\
2.00010799136069	-0.01\\
2.05011069114471	-0.01\\
2.10011339092873	-0.01\\
2.15011609071274	-0.01\\
2.20011879049676	-0.01\\
2.25012149028078	-0.01\\
2.30012419006479	-0.01\\
2.35012688984881	-0.01\\
2.40012958963283	-0.01\\
2.45013228941685	-0.01\\
2.50013498920086	-0.01\\
2.55013768898488	-0.01\\
2.6001403887689	-0.01\\
2.65014308855292	-0.01\\
2.70014578833693	-0.01\\
2.75014848812095	-0.01\\
2.80015118790497	-0.01\\
2.85015388768899	-0.01\\
2.900156587473	-0.01\\
2.95015928725702	-0.01\\
3.00016198704104	-0.01\\
3.05016468682505	-0.01\\
3.10016738660907	-0.01\\
3.15017008639309	-0.01\\
3.20017278617711	-0.01\\
3.25017548596112	-0.01\\
3.30017818574514	-0.01\\
3.35018088552916	-0.01\\
3.40018358531318	-0.01\\
3.45018628509719	-0.01\\
3.50018898488121	-0.01\\
3.55019168466523	-0.01\\
3.60019438444924	-0.01\\
3.65019708423326	-0.01\\
3.70019978401728	-0.01\\
3.7502024838013	-0.01\\
3.80020518358531	-0.01\\
3.85020788336933	-0.01\\
3.90021058315335	-0.01\\
3.95021328293736	-0.01\\
4.00021598272138	-0.01\\
4.0502186825054	-0.01\\
4.10022138228942	-0.01\\
4.15022408207343	-0.01\\
4.20022678185745	-0.01\\
4.25022948164147	-0.01\\
4.30023218142549	-0.01\\
4.3502348812095	-0.01\\
4.40023758099352	-0.01\\
4.45024028077754	-0.01\\
4.50024298056155	-0.01\\
4.55024568034557	-0.01\\
4.60024838012959	-0.01\\
4.65025107991361	-0.01\\
4.70025377969762	-0.01\\
4.75025647948164	-0.01\\
4.80025917926566	-0.01\\
4.85026187904968	-0.01\\
4.90026457883369	-0.01\\
4.95026727861771	-0.01\\
5.00026997840173	-0.01\\
5.05027267818575	-0.01\\
5.10027537796976	-0.01\\
5.15027807775378	-0.01\\
5.2002807775378	-0.01\\
5.25028347732181	-0.01\\
5.30028617710583	-0.01\\
5.35028887688985	-0.01\\
5.40029157667387	-0.01\\
5.45029427645788	-0.01\\
5.5002969762419	-0.01\\
5.55029967602592	-0.01\\
5.60030237580994	-0.01\\
5.65030507559395	-0.01\\
5.70030777537797	-0.01\\
5.75031047516199	-0.01\\
5.800313174946	-0.01\\
5.85031587473002	-0.01\\
5.90031857451404	-0.01\\
5.95032127429806	-0.01\\
6.00032397408207	-0.01\\
6.05032667386609	-0.01\\
6.10032937365011	-0.01\\
6.15033207343413	-0.01\\
6.20033477321814	-0.01\\
6.25033747300216	-0.01\\
6.30034017278618	-0.01\\
6.3503428725702	-0.01\\
6.40034557235421	-0.01\\
6.45034827213823	-0.01\\
6.50035097192225	-0.01\\
6.55035367170626	-0.01\\
6.60035637149028	-0.01\\
6.6503590712743	-0.01\\
6.70036177105832	-0.01\\
6.75036447084233	-0.01\\
6.80036717062635	-0.01\\
6.85036987041037	-0.01\\
6.90037257019439	-0.01\\
6.9503752699784	-0.01\\
7.00037796976242	-0.01\\
7.05038066954644	-0.01\\
7.10038336933045	-0.01\\
7.15038606911447	-0.01\\
7.20038876889849	-0.01\\
7.25039146868251	-0.01\\
7.30039416846652	-0.01\\
7.35039686825054	-0.01\\
7.40039956803456	-0.01\\
7.45040226781857	-0.01\\
7.50040496760259	-0.01\\
7.55040766738661	-0.01\\
7.60041036717063	-0.01\\
7.65041306695464	-0.01\\
7.70041576673866	-0.01\\
7.75041846652268	-0.01\\
7.8004211663067	-0.01\\
7.85042386609071	-0.01\\
7.90042656587473	-0.01\\
7.95042926565875	-0.01\\
8.00043196544276	-0.01\\
8.05043466522678	-0.01\\
8.1004373650108	-0.01\\
8.15044006479482	-0.01\\
8.20044276457883	-0.01\\
8.25044546436285	-0.01\\
8.30044816414687	-0.01\\
8.35045086393089	-0.01\\
8.4004535637149	-0.01\\
8.45045626349892	-0.01\\
8.50045896328294	-0.01\\
8.55046166306696	-0.01\\
8.60046436285097	-0.01\\
8.65046706263499	-0.01\\
8.70046976241901	-0.01\\
8.75047246220302	-0.01\\
8.80047516198704	-0.01\\
8.85047786177106	-0.01\\
8.90048056155508	-0.01\\
8.95048326133909	-0.01\\
9.00048596112311	-0.01\\
9.05048866090713	-0.01\\
9.10049136069114	-0.01\\
9.15049406047516	-0.01\\
9.20049676025918	-0.01\\
9.2504994600432	-0.01\\
9.30050215982721	-0.01\\
9.35050485961123	-0.01\\
9.40050755939525	-0.01\\
9.45051025917927	-0.01\\
9.50051295896328	-0.01\\
9.5505156587473	-0.01\\
9.60051835853132	-0.01\\
9.65052105831533	-0.01\\
9.70052375809935	-0.01\\
9.75052645788337	-0.01\\
9.80052915766739	-0.01\\
9.8505318574514	-0.01\\
9.90053455723542	-0.01\\
9.95053725701944	-0.01\\
10.0005399568035	-0.01\\
10.0505426565875	-0.01\\
10.1005453563715	-0.01\\
10.1505480561555	-0.01\\
10.2005507559395	-0.01\\
10.2505534557235	-0.01\\
10.3005561555076	-0.01\\
10.3505588552916	-0.01\\
10.4005615550756	-0.01\\
10.4505642548596	-0.01\\
10.5005669546436	-0.01\\
10.5505696544276	-0.01\\
10.6005723542117	-0.01\\
10.6505750539957	-0.01\\
10.7005777537797	-0.01\\
10.7505804535637	-0.01\\
10.8005831533477	-0.01\\
10.8505858531317	-0.01\\
10.9005885529158	-0.01\\
10.9505912526998	-0.01\\
11.0005939524838	-0.01\\
11.0505966522678	-0.01\\
11.1005993520518	-0.01\\
11.1506020518359	-0.01\\
11.2006047516199	-0.01\\
11.2506074514039	-0.01\\
11.3006101511879	-0.01\\
11.3506128509719	-0.01\\
11.4006155507559	-0.01\\
11.45061825054	-0.01\\
11.500620950324	-0.01\\
11.550623650108	-0.01\\
11.600626349892	-0.01\\
11.650629049676	-0.01\\
11.70063174946	-0.01\\
11.7506344492441	-0.01\\
11.8006371490281	-0.01\\
11.8506398488121	-0.01\\
11.9006425485961	-0.01\\
11.9506452483801	-0.01\\
12.0006479481641	-0.01\\
12.0506506479482	-0.01\\
12.1006533477322	-0.01\\
12.1506560475162	-0.01\\
12.2006587473002	-0.01\\
12.2506614470842	-0.01\\
12.3006641468683	-0.01\\
12.3506668466523	-0.01\\
12.4006695464363	-0.01\\
12.4506722462203	-0.01\\
12.5006749460043	-0.01\\
12.5506776457883	-0.01\\
12.6006803455724	-0.01\\
12.6506830453564	-0.01\\
12.7006857451404	-0.01\\
12.7506884449244	-0.01\\
12.8006911447084	-0.01\\
12.8506938444924	-0.01\\
12.9006965442765	-0.01\\
12.9506992440605	-0.01\\
13.0007019438445	-0.01\\
13.0507046436285	-0.01\\
13.1007073434125	-0.01\\
13.1507100431965	-0.01\\
13.2007127429806	-0.01\\
13.2507154427646	-0.01\\
13.3007181425486	-0.01\\
13.3507208423326	-0.01\\
13.4007235421166	-0.01\\
13.4507262419006	-0.01\\
13.5007289416847	-0.01\\
13.5507316414687	-0.01\\
13.6007343412527	-0.01\\
13.6507370410367	-0.01\\
13.7007397408207	-0.01\\
13.7507424406048	-0.01\\
13.8007451403888	-0.01\\
13.8507478401728	-0.01\\
13.9007505399568	-0.01\\
13.9507532397408	-0.01\\
14.0007559395248	-0.01\\
14.0507586393089	-0.01\\
14.1007613390929	-0.01\\
14.1507640388769	-0.01\\
14.2007667386609	-0.01\\
14.2507694384449	-0.01\\
14.3007721382289	-0.01\\
14.350774838013	-0.01\\
14.400777537797	-0.01\\
14.450780237581	-0.01\\
14.500782937365	-0.01\\
14.550785637149	-0.01\\
14.600788336933	-0.01\\
14.6507910367171	-0.01\\
14.7007937365011	-0.01\\
14.7507964362851	-0.01\\
14.8007991360691	-0.01\\
14.8508018358531	-0.01\\
14.9008045356372	-0.01\\
14.9508072354212	-0.01\\
15.0008099352052	-0.01\\
15.0508126349892	-0.01\\
15.1008153347732	-0.01\\
15.1508180345572	-0.01\\
15.2008207343413	-0.01\\
15.2508234341253	-0.01\\
15.3008261339093	-0.01\\
15.3508288336933	-0.01\\
15.4008315334773	-0.01\\
15.4508342332613	-0.01\\
15.5008369330454	-0.01\\
15.5508396328294	-0.01\\
15.6008423326134	-0.01\\
15.6508450323974	-0.01\\
15.7008477321814	-0.01\\
15.7508504319654	-0.01\\
15.8008531317495	-0.01\\
15.8508558315335	-0.01\\
15.9008585313175	-0.01\\
15.9508612311015	-0.01\\
16.0008639308855	-0.01\\
16.0508666306695	-0.01\\
16.1008693304536	-0.01\\
16.1508720302376	-0.01\\
16.2008747300216	-0.01\\
16.2508774298056	-0.01\\
16.3008801295896	-0.01\\
16.3508828293737	-0.01\\
16.4008855291577	-0.01\\
16.4508882289417	-0.01\\
16.5008909287257	-0.01\\
16.5508936285097	-0.01\\
16.6008963282937	-0.01\\
16.6508990280778	-0.01\\
16.7009017278618	-0.01\\
16.7509044276458	-0.01\\
16.8009071274298	-0.01\\
16.8509098272138	-0.01\\
16.9009125269978	-0.01\\
16.9509152267819	-0.01\\
17.0009179265659	-0.01\\
17.0509206263499	-0.01\\
17.1009233261339	-0.01\\
17.1509260259179	-0.01\\
17.2009287257019	-0.01\\
17.250931425486	-0.01\\
17.30093412527	-0.01\\
17.350936825054	-0.01\\
17.400939524838	-0.01\\
17.450942224622	-0.01\\
17.500944924406	-0.01\\
17.5509476241901	-0.01\\
17.6009503239741	-0.01\\
17.6509530237581	-0.01\\
17.7009557235421	-0.01\\
17.7509584233261	-0.01\\
17.8009611231102	-0.01\\
17.8509638228942	-0.01\\
17.9009665226782	-0.01\\
17.9509692224622	-0.01\\
18.0009719222462	-0.01\\
18.0509746220302	-0.01\\
18.1009773218143	-0.01\\
18.1509800215983	-0.01\\
18.2009827213823	-0.01\\
18.2509854211663	-0.01\\
18.3009881209503	-0.01\\
18.3509908207343	-0.01\\
18.4009935205184	-0.01\\
18.4509962203024	-0.01\\
18.5009989200864	-0.01\\
18.5510016198704	-0.01\\
18.6010043196544	-0.01\\
18.6510070194384	-0.01\\
18.7010097192225	-0.01\\
18.7510124190065	-0.01\\
18.8010151187905	-0.01\\
18.8510178185745	-0.01\\
18.9010205183585	-0.01\\
18.9510232181425	-0.01\\
19.0010259179266	-0.01\\
19.0510286177106	-0.01\\
19.1010313174946	-0.01\\
19.1510340172786	-0.01\\
19.2010367170626	-0.01\\
19.2510394168467	-0.01\\
19.3010421166307	-0.01\\
19.3510448164147	-0.01\\
19.4010475161987	-0.01\\
19.4510502159827	-0.01\\
19.5010529157667	-0.01\\
19.5510556155508	-0.01\\
19.6010583153348	-0.01\\
19.6510610151188	-0.01\\
19.7010637149028	-0.01\\
19.7510664146868	-0.01\\
19.8010691144708	-0.01\\
19.8510718142549	-0.01\\
19.9010745140389	-0.01\\
19.9510772138229	-0.01\\
20.0010799136069	-0.01\\
20.0510826133909	-0.01\\
20.1010853131749	-0.01\\
20.151088012959	-0.01\\
20.201090712743	-0.01\\
20.251093412527	-0.01\\
20.301096112311	-0.01\\
20.351098812095	-0.01\\
20.4011015118791	-0.01\\
20.4511042116631	-0.01\\
20.5011069114471	-0.01\\
20.5511096112311	-0.01\\
20.6011123110151	-0.01\\
20.6511150107991	-0.01\\
20.7011177105832	-0.01\\
20.7511204103672	-0.01\\
20.8011231101512	-0.01\\
20.8511258099352	-0.01\\
20.9011285097192	-0.01\\
20.9511312095032	-0.01\\
21.0011339092873	-0.01\\
21.0511366090713	-0.01\\
21.1011393088553	-0.01\\
21.1511420086393	-0.01\\
21.2011447084233	-0.01\\
21.2511474082073	-0.01\\
21.3011501079914	-0.01\\
21.3511528077754	-0.01\\
21.4011555075594	-0.01\\
21.4511582073434	-0.01\\
21.5011609071274	-0.01\\
21.5511636069114	-0.01\\
21.6011663066955	-0.01\\
21.6511690064795	-0.01\\
21.7011717062635	-0.01\\
21.7511744060475	-0.01\\
21.8011771058315	-0.01\\
21.8511798056156	-0.01\\
21.9011825053996	-0.01\\
21.9511852051836	-0.01\\
22.0011879049676	-0.01\\
22.0511906047516	-0.01\\
22.1011933045356	-0.01\\
22.1511960043197	-0.01\\
22.2011987041037	-0.01\\
22.2512014038877	-0.01\\
22.3012041036717	-0.01\\
22.3512068034557	-0.01\\
22.4012095032397	-0.01\\
22.4512122030238	-0.01\\
22.5012149028078	-0.01\\
22.5512176025918	-0.01\\
22.6012203023758	-0.01\\
22.6512230021598	-0.01\\
22.7012257019438	-0.01\\
22.7512284017279	-0.01\\
22.8012311015119	-0.01\\
22.8512338012959	-0.01\\
22.9012365010799	-0.01\\
22.9512392008639	-0.01\\
23.0012419006479	-0.01\\
23.051244600432	-0.01\\
23.101247300216	-0.01\\
23.15125	-0.01\\
23.201252699784	-0.01\\
23.251255399568	-0.01\\
23.3012580993521	-0.01\\
23.3512607991361	-0.01\\
23.4012634989201	-0.01\\
23.4512661987041	-0.01\\
23.5012688984881	-0.01\\
23.5512715982721	-0.01\\
23.6012742980562	-0.01\\
23.6512769978402	-0.01\\
23.7012796976242	-0.01\\
23.7512823974082	-0.01\\
23.8012850971922	-0.01\\
23.8512877969762	-0.01\\
23.9012904967603	-0.01\\
23.9512931965443	-0.01\\
24.0012958963283	-0.01\\
24.0512985961123	-0.01\\
24.1013012958963	-0.01\\
24.1513039956803	-0.01\\
24.2013066954644	-0.01\\
24.2513093952484	-0.01\\
24.3013120950324	-0.01\\
24.3513147948164	-0.01\\
24.4013174946004	-0.01\\
24.4513201943845	-0.01\\
24.5013228941685	-0.01\\
24.5513255939525	-0.01\\
24.6013282937365	-0.01\\
24.6513309935205	-0.01\\
24.7013336933045	-0.01\\
24.7513363930886	-0.01\\
24.8013390928726	-0.01\\
24.8513417926566	-0.01\\
24.9013444924406	-0.01\\
24.9513471922246	-0.01\\
25.0013498920086	-0.01\\
25.0513525917927	-0.01\\
25.1013552915767	-0.01\\
25.1513579913607	-0.01\\
25.2013606911447	-0.01\\
25.2513633909287	-0.01\\
25.3013660907127	-0.01\\
25.3513687904968	-0.01\\
25.4013714902808	-0.01\\
25.4513741900648	-0.01\\
25.5013768898488	-0.01\\
25.5513795896328	-0.01\\
25.6013822894168	-0.01\\
25.6513849892009	-0.01\\
25.7013876889849	-0.01\\
25.7513903887689	-0.01\\
25.8013930885529	-0.01\\
25.8513957883369	-0.01\\
25.901398488121	-0.01\\
25.951401187905	-0.01\\
26.001403887689	-0.01\\
26.051406587473	-0.01\\
26.101409287257	-0.01\\
26.151411987041	-0.01\\
26.2014146868251	-0.01\\
26.2514173866091	-0.01\\
26.3014200863931	-0.01\\
26.3514227861771	-0.01\\
26.4014254859611	-0.01\\
26.4514281857451	-0.01\\
26.5014308855292	-0.01\\
26.5514335853132	-0.01\\
26.6014362850972	-0.01\\
26.6514389848812	-0.01\\
26.7014416846652	-0.01\\
26.7514443844492	-0.01\\
26.8014470842333	-0.01\\
26.8514497840173	-0.01\\
26.9014524838013	-0.01\\
26.9514551835853	-0.01\\
27.0014578833693	-0.01\\
27.0514605831534	-0.01\\
27.1014632829374	-0.01\\
27.1514659827214	-0.01\\
27.2014686825054	-0.01\\
27.2514713822894	-0.01\\
27.3014740820734	-0.01\\
27.3514767818575	-0.01\\
27.4014794816415	-0.01\\
27.4514821814255	-0.01\\
27.5014848812095	-0.01\\
27.5514875809935	-0.01\\
27.6014902807775	-0.01\\
27.6514929805616	-0.01\\
27.7014956803456	-0.01\\
27.7514983801296	-0.01\\
27.8015010799136	-0.01\\
27.8515037796976	-0.01\\
27.9015064794816	-0.01\\
27.9515091792657	-0.01\\
28.0015118790497	-0.01\\
28.0515145788337	-0.01\\
28.1015172786177	-0.01\\
28.1515199784017	-0.01\\
28.2015226781857	-0.01\\
28.2515253779698	-0.01\\
28.3015280777538	-0.01\\
28.3515307775378	-0.01\\
28.4015334773218	-0.01\\
28.4515361771058	-0.01\\
28.5015388768899	-0.01\\
28.5515415766739	-0.01\\
28.6015442764579	-0.01\\
28.6515469762419	-0.01\\
28.7015496760259	-0.01\\
28.7515523758099	-0.01\\
28.801555075594	-0.01\\
28.851557775378	-0.01\\
28.901560475162	-0.01\\
28.951563174946	-0.01\\
29.00156587473	-0.01\\
29.051568574514	-0.01\\
29.1015712742981	-0.01\\
29.1515739740821	-0.01\\
29.2015766738661	-0.01\\
29.2515793736501	-0.01\\
29.3015820734341	-0.01\\
29.3515847732181	-0.01\\
29.4015874730022	-0.01\\
29.4515901727862	-0.01\\
29.5015928725702	-0.01\\
29.5515955723542	-0.01\\
29.6015982721382	-0.01\\
29.6516009719222	-0.01\\
29.7016036717063	-0.01\\
29.7516063714903	-0.01\\
29.8016090712743	-0.01\\
29.8516117710583	-0.01\\
29.9016144708423	-0.01\\
29.9516171706264	-0.01\\
30.0016198704104	-0.01\\
30.0516225701944	-0.01\\
30.1016252699784	-0.01\\
30.1516279697624	-0.01\\
30.2016306695464	-0.01\\
30.2516333693305	-0.01\\
30.3016360691145	-0.01\\
30.3516387688985	-0.01\\
30.4016414686825	-0.01\\
30.4516441684665	-0.01\\
30.5016468682505	-0.01\\
30.5516495680346	-0.01\\
30.6016522678186	-0.01\\
30.6516549676026	-0.01\\
30.7016576673866	-0.01\\
30.7516603671706	-0.01\\
30.8016630669546	-0.01\\
30.8516657667387	-0.01\\
30.9016684665227	-0.01\\
30.9516711663067	-0.01\\
31.0016738660907	-0.01\\
31.0516765658747	-0.01\\
31.1016792656587	-0.01\\
31.1516819654428	-0.01\\
31.2016846652268	-0.01\\
31.2516873650108	-0.01\\
31.3016900647948	-0.01\\
31.3516927645788	-0.01\\
31.4016954643629	-0.01\\
31.4516981641469	-0.01\\
31.5017008639309	-0.01\\
31.5517035637149	-0.01\\
31.6017062634989	-0.01\\
31.6517089632829	-0.01\\
31.701711663067	-0.01\\
31.751714362851	-0.01\\
31.801717062635	-0.01\\
31.851719762419	-0.01\\
31.901722462203	-0.01\\
31.951725161987	-0.01\\
32.0017278617711	-0.01\\
32.0517305615551	-0.01\\
32.1017332613391	-0.01\\
32.1517359611231	-0.01\\
32.2017386609071	-0.01\\
32.2517413606911	-0.01\\
32.3017440604752	-0.01\\
32.3517467602592	-0.01\\
32.4017494600432	-0.01\\
32.4517521598272	-0.01\\
32.5017548596112	-0.01\\
32.5517575593952	-0.01\\
32.6017602591793	-0.01\\
32.6517629589633	-0.01\\
32.7017656587473	-0.01\\
32.7517683585313	-0.01\\
32.8017710583153	-0.01\\
32.8517737580993	-0.01\\
32.9017764578834	-0.01\\
32.9517791576674	-0.01\\
33.0017818574514	-0.01\\
33.0517845572354	-0.01\\
33.1017872570194	-0.01\\
33.1517899568035	-0.01\\
33.2017926565875	-0.01\\
33.2517953563715	-0.01\\
33.3017980561555	-0.01\\
33.3518007559395	-0.01\\
33.4018034557235	-0.01\\
33.4518061555076	-0.01\\
33.5018088552916	-0.01\\
33.5518115550756	-0.01\\
33.6018142548596	-0.01\\
33.6518169546436	-0.01\\
33.7018196544277	-0.01\\
33.7518223542117	-0.01\\
33.8018250539957	-0.01\\
33.8518277537797	-0.01\\
33.9018304535637	-0.01\\
33.9518331533477	-0.01\\
34.0018358531318	-0.01\\
34.0518385529158	-0.01\\
34.1018412526998	-0.01\\
34.1518439524838	-0.01\\
34.2018466522678	-0.01\\
34.2518493520518	-0.01\\
34.3018520518359	-0.01\\
34.3518547516199	-0.01\\
34.4018574514039	-0.01\\
34.4518601511879	-0.01\\
34.5018628509719	-0.01\\
34.5518655507559	-0.01\\
34.60186825054	-0.01\\
34.651870950324	-0.01\\
34.701873650108	-0.01\\
34.751876349892	-0.01\\
34.801879049676	-0.01\\
34.85188174946	-0.01\\
34.9018844492441	-0.01\\
34.9518871490281	-0.01\\
35.0018898488121	-0.01\\
35.0518925485961	-0.01\\
35.1018952483801	-0.01\\
35.1518979481641	-0.01\\
35.2019006479482	-0.01\\
35.2519033477322	-0.01\\
35.3019060475162	-0.01\\
35.3519087473002	-0.01\\
35.4019114470842	-0.01\\
35.4519141468683	-0.01\\
35.5019168466523	-0.01\\
35.5519195464363	-0.01\\
35.6019222462203	-0.01\\
35.6519249460043	-0.01\\
35.7019276457883	-0.01\\
35.7519303455724	-0.01\\
35.8019330453564	-0.01\\
35.8519357451404	-0.01\\
35.9019384449244	-0.01\\
35.9519411447084	-0.01\\
36.0019438444924	-0.01\\
36.0519465442765	-0.01\\
36.1019492440605	-0.01\\
36.1519519438445	-0.01\\
36.2019546436285	-0.01\\
36.2519573434125	-0.01\\
36.3019600431965	-0.01\\
36.3519627429806	-0.01\\
36.4019654427646	-0.01\\
36.4519681425486	-0.01\\
36.5019708423326	-0.01\\
36.5519735421166	-0.01\\
36.6019762419006	-0.01\\
36.6519789416847	-0.01\\
36.7019816414687	-0.01\\
36.7519843412527	-0.01\\
36.8019870410367	-0.01\\
36.8519897408207	-0.01\\
36.9019924406048	-0.01\\
36.9519951403888	-0.01\\
37.0019978401728	-0.01\\
37.0520005399568	-0.01\\
37.1020032397408	-0.01\\
37.1520059395248	-0.01\\
37.2020086393089	-0.01\\
37.2520113390929	-0.01\\
37.3020140388769	-0.01\\
37.3520167386609	-0.01\\
37.4020194384449	-0.01\\
37.4520221382289	-0.01\\
37.502024838013	-0.01\\
37.552027537797	-0.01\\
37.602030237581	-0.01\\
37.652032937365	-0.01\\
37.702035637149	-0.01\\
37.752038336933	-0.01\\
37.8020410367171	-0.01\\
37.8520437365011	-0.01\\
37.9020464362851	-0.01\\
37.9520491360691	-0.01\\
38.0020518358531	-0.01\\
38.0520545356372	-0.01\\
38.1020572354212	-0.01\\
38.1520599352052	-0.01\\
38.2020626349892	-0.01\\
38.2520653347732	-0.01\\
38.3020680345572	-0.01\\
38.3520707343413	-0.01\\
38.4020734341253	-0.01\\
38.4520761339093	-0.01\\
38.5020788336933	-0.01\\
38.5520815334773	-0.01\\
38.6020842332613	-0.01\\
38.6520869330454	-0.01\\
38.7020896328294	-0.01\\
38.7520923326134	-0.01\\
38.8020950323974	-0.01\\
38.8520977321814	-0.01\\
38.9021004319654	-0.01\\
38.9521031317495	-0.01\\
39.0021058315335	-0.01\\
39.0521085313175	-0.01\\
39.1021112311015	-0.01\\
39.1521139308855	-0.01\\
39.2021166306696	-0.01\\
39.2521193304536	-0.01\\
39.3021220302376	-0.01\\
39.3521247300216	-0.01\\
39.4021274298056	-0.01\\
39.4521301295896	-0.01\\
39.5021328293737	-0.01\\
39.5521355291577	-0.01\\
39.6021382289417	-0.01\\
39.6521409287257	-0.01\\
39.7021436285097	-0.01\\
39.7521463282937	-0.01\\
39.8021490280778	-0.01\\
39.8521517278618	-0.01\\
39.9021544276458	-0.01\\
39.9521571274298	-0.01\\
40.0021598272138	-0.01\\
40.0521625269978	-0.01\\
40.1021652267819	-0.01\\
40.1521679265659	-0.01\\
40.2021706263499	-0.01\\
40.2521733261339	-0.01\\
40.3021760259179	-0.01\\
40.3521787257019	-0.01\\
40.402181425486	-0.01\\
40.45218412527	-0.01\\
40.502186825054	-0.01\\
40.552189524838	-0.01\\
40.602192224622	-0.01\\
40.652194924406	-0.01\\
40.7021976241901	-0.01\\
40.7522003239741	-0.01\\
40.8022030237581	-0.01\\
40.8522057235421	-0.01\\
40.9022084233261	-0.01\\
40.9522111231102	-0.01\\
41.0022138228942	-0.01\\
41.0522165226782	-0.01\\
41.1022192224622	-0.01\\
41.1522219222462	-0.01\\
41.2022246220302	-0.01\\
41.2522273218143	-0.01\\
41.3022300215983	-0.01\\
41.3522327213823	-0.01\\
41.4022354211663	-0.01\\
41.4522381209503	-0.01\\
41.5022408207343	-0.01\\
41.5522435205184	-0.01\\
41.6022462203024	-0.01\\
41.6522489200864	-0.01\\
41.7022516198704	-0.01\\
41.7522543196544	-0.01\\
41.8022570194385	-0.01\\
41.8522597192225	-0.01\\
41.9022624190065	-0.01\\
41.9522651187905	-0.01\\
42.0022678185745	-0.01\\
42.0522705183585	-0.01\\
42.1022732181426	-0.01\\
42.1522759179266	-0.01\\
42.2022786177106	-0.01\\
42.2522813174946	-0.01\\
42.3022840172786	-0.01\\
42.3522867170626	-0.01\\
42.4022894168467	-0.01\\
42.4522921166307	-0.01\\
42.5022948164147	-0.01\\
42.5522975161987	-0.01\\
42.6023002159827	-0.01\\
42.6523029157667	-0.01\\
42.7023056155508	-0.01\\
42.7523083153348	-0.01\\
42.8023110151188	-0.01\\
42.8523137149028	-0.01\\
42.9023164146868	-0.01\\
42.9523191144708	-0.01\\
43.0023218142549	-0.01\\
43.0523245140389	-0.01\\
43.1023272138229	-0.01\\
43.1523299136069	-0.01\\
43.2023326133909	-0.01\\
43.2523353131749	-0.01\\
43.302338012959	-0.01\\
43.352340712743	-0.01\\
43.402343412527	-0.01\\
43.452346112311	-0.01\\
43.502348812095	-0.01\\
43.5523515118791	-0.01\\
43.6023542116631	-0.01\\
43.6523569114471	-0.01\\
43.7023596112311	-0.01\\
43.7523623110151	-0.01\\
43.8023650107991	-0.01\\
43.8523677105832	-0.01\\
43.9023704103672	-0.01\\
43.9523731101512	-0.01\\
44.0023758099352	-0.01\\
44.0523785097192	-0.01\\
44.1023812095032	-0.01\\
44.1523839092873	-0.01\\
44.2023866090713	-0.01\\
44.2523893088553	-0.01\\
44.3023920086393	-0.01\\
44.3523947084233	-0.01\\
44.4023974082073	-0.01\\
44.4524001079914	-0.01\\
44.5024028077754	-0.01\\
44.5524055075594	-0.01\\
44.6024082073434	-0.01\\
44.6524109071274	-0.01\\
44.7024136069114	-0.01\\
44.7524163066955	-0.01\\
44.8024190064795	-0.01\\
44.8524217062635	-0.01\\
44.9024244060475	-0.01\\
44.9524271058315	-0.01\\
45.0024298056155	-0.01\\
45.0524325053996	-0.01\\
45.1024352051836	-0.01\\
45.1524379049676	-0.01\\
45.2024406047516	-0.01\\
45.2524433045356	-0.01\\
45.3024460043197	-0.01\\
45.3524487041037	-0.01\\
45.4024514038877	-0.01\\
45.4524541036717	-0.01\\
45.5024568034557	-0.01\\
45.5524595032397	-0.01\\
45.6024622030238	-0.01\\
45.6524649028078	-0.01\\
45.7024676025918	-0.01\\
45.7524703023758	-0.01\\
45.8024730021598	-0.01\\
45.8524757019439	-0.01\\
45.9024784017279	-0.01\\
45.9524811015119	-0.01\\
46.0024838012959	-0.01\\
46.0524865010799	-0.01\\
46.1024892008639	-0.01\\
46.152491900648	-0.01\\
46.202494600432	-0.01\\
46.252497300216	-0.01\\
46.3025	-0.01\\
46.352502699784	-0.01\\
46.402505399568	-0.01\\
46.4525080993521	-0.01\\
46.5025107991361	-0.01\\
46.5525134989201	-0.01\\
46.6025161987041	-0.01\\
46.6525188984881	-0.01\\
46.7025215982721	-0.01\\
46.7525242980562	-0.01\\
46.8025269978402	-0.01\\
46.8525296976242	-0.01\\
46.9025323974082	-0.01\\
46.9525350971922	-0.01\\
47.0025377969762	-0.01\\
47.0525404967603	-0.01\\
47.1025431965443	-0.01\\
47.1525458963283	-0.01\\
47.2025485961123	-0.01\\
47.2525512958963	-0.01\\
47.3025539956803	-0.01\\
47.3525566954644	-0.01\\
47.4025593952484	-0.01\\
47.4525620950324	-0.01\\
47.5025647948164	-0.01\\
47.5525674946004	-0.01\\
47.6025701943845	-0.01\\
47.6525728941685	-0.01\\
47.7025755939525	-0.01\\
47.7525782937365	-0.01\\
47.8025809935205	-0.01\\
47.8525836933045	-0.01\\
47.9025863930886	-0.01\\
47.9525890928726	-0.01\\
48.0025917926566	-0.01\\
48.0525944924406	-0.01\\
48.1025971922246	-0.01\\
48.1525998920086	-0.01\\
48.2026025917927	-0.01\\
48.2526052915767	-0.01\\
48.3026079913607	-0.01\\
48.3526106911447	-0.01\\
48.4026133909287	-0.01\\
48.4526160907127	-0.01\\
48.5026187904968	-0.01\\
48.5526214902808	-0.01\\
48.6026241900648	-0.01\\
48.6526268898488	-0.01\\
48.7026295896328	-0.01\\
48.7526322894168	-0.01\\
48.8026349892009	-0.01\\
48.8526376889849	-0.01\\
48.9026403887689	-0.01\\
48.9526430885529	-0.01\\
49.0026457883369	-0.01\\
49.052648488121	-0.01\\
49.102651187905	-0.01\\
49.152653887689	-0.01\\
49.202656587473	-0.01\\
49.252659287257	-0.01\\
49.302661987041	-0.01\\
49.3526646868251	-0.01\\
49.4026673866091	-0.01\\
49.4526700863931	-0.01\\
49.5026727861771	-0.01\\
49.5526754859611	-0.01\\
49.6026781857451	-0.01\\
49.6526808855292	-0.01\\
49.7026835853132	-0.01\\
49.7526862850972	-0.01\\
49.8026889848812	-0.01\\
49.8526916846652	-0.01\\
49.9026943844492	-0.01\\
49.9526970842333	-0.01\\
50.0026997840173	-0.01\\
50.0527024838013	-0.01\\
50.1027051835853	-0.01\\
50.1527078833693	-0.01\\
50.2027105831534	-0.01\\
50.2527132829374	-0.01\\
50.3027159827214	-0.01\\
50.3527186825054	-0.01\\
50.4027213822894	-0.01\\
50.4527240820734	-0.01\\
50.5027267818575	-0.01\\
50.5527294816415	-0.01\\
50.6027321814255	-0.01\\
50.6527348812095	-0.01\\
50.7027375809935	-0.01\\
50.7527402807775	-0.01\\
50.8027429805616	-0.01\\
50.8527456803456	-0.01\\
50.9027483801296	-0.01\\
50.9527510799136	-0.01\\
51.0027537796976	-0.01\\
51.0527564794816	-0.01\\
51.1027591792657	-0.01\\
51.1527618790497	-0.01\\
51.2027645788337	-0.01\\
51.2527672786177	-0.01\\
51.3027699784017	-0.01\\
51.3527726781858	-0.01\\
51.4027753779698	-0.01\\
51.4527780777538	-0.01\\
51.5027807775378	-0.01\\
51.5527834773218	-0.01\\
51.6027861771058	-0.01\\
51.6527888768899	-0.01\\
51.7027915766739	-0.01\\
51.7527942764579	-0.01\\
51.8027969762419	-0.01\\
51.8527996760259	-0.01\\
51.9028023758099	-0.01\\
51.952805075594	-0.01\\
52.002807775378	-0.01\\
52.052810475162	-0.01\\
52.102813174946	-0.01\\
52.15281587473	-0.01\\
52.202818574514	-0.01\\
52.2528212742981	-0.01\\
52.3028239740821	-0.01\\
52.3528266738661	-0.01\\
52.4028293736501	-0.01\\
52.4528320734341	-0.01\\
52.5028347732181	-0.01\\
52.5528374730022	-0.01\\
52.6028401727862	-0.01\\
52.6528428725702	-0.01\\
52.7028455723542	-0.01\\
52.7528482721382	-0.01\\
52.8028509719222	-0.01\\
52.8528536717063	-0.01\\
52.9028563714903	-0.01\\
52.9528590712743	-0.01\\
53.0028617710583	-0.01\\
53.0528644708423	-0.01\\
53.1028671706264	-0.01\\
53.1528698704104	-0.01\\
53.2028725701944	-0.01\\
53.2528752699784	-0.01\\
53.3028779697624	-0.01\\
53.3528806695464	-0.01\\
53.4028833693305	-0.01\\
53.4528860691145	-0.01\\
53.5028887688985	-0.01\\
53.5528914686825	-0.01\\
53.6028941684665	-0.01\\
53.6528968682505	-0.01\\
53.7028995680346	-0.01\\
53.7529022678186	-0.01\\
53.8029049676026	-0.01\\
53.8529076673866	-0.01\\
53.9029103671706	-0.01\\
53.9529130669547	-0.01\\
54.0029157667387	-0.01\\
54.0529184665227	-0.01\\
54.1029211663067	-0.01\\
54.1529238660907	-0.01\\
54.2029265658747	-0.01\\
54.2529292656588	-0.01\\
54.3029319654428	-0.01\\
54.3529346652268	-0.01\\
54.4029373650108	-0.01\\
54.4529400647948	-0.01\\
54.5029427645788	-0.01\\
54.5529454643629	-0.01\\
54.6029481641469	-0.01\\
54.6529508639309	-0.01\\
54.7029535637149	-0.01\\
54.7529562634989	-0.01\\
54.8029589632829	-0.01\\
54.852961663067	-0.01\\
54.902964362851	-0.01\\
54.952967062635	-0.01\\
55.002969762419	-0.01\\
55.052972462203	-0.01\\
55.102975161987	-0.01\\
55.1529778617711	-0.01\\
55.2029805615551	-0.01\\
55.2529832613391	-0.01\\
55.3029859611231	-0.01\\
55.3529886609071	-0.01\\
55.4029913606911	-0.01\\
55.4529940604752	-0.01\\
55.5029967602592	-0.01\\
55.5529994600432	-0.01\\
55.6030021598272	-0.01\\
55.6530048596112	-0.01\\
55.7030075593953	-0.01\\
55.7530102591793	-0.01\\
55.8030129589633	-0.01\\
55.8530156587473	-0.01\\
55.9030183585313	-0.01\\
55.9530210583153	-0.01\\
56.0030237580994	-0.01\\
56.0530264578834	-0.01\\
56.1030291576674	-0.01\\
56.1530318574514	-0.01\\
56.2030345572354	-0.01\\
56.2530372570194	-0.01\\
56.3030399568035	-0.01\\
56.3530426565875	-0.01\\
56.4030453563715	-0.01\\
56.4530480561555	-0.01\\
56.5030507559395	-0.01\\
56.5530534557235	-0.01\\
56.6030561555076	-0.01\\
56.6530588552916	-0.01\\
56.7030615550756	-0.01\\
56.7530642548596	-0.01\\
56.8030669546436	-0.01\\
56.8530696544276	-0.01\\
56.9030723542117	-0.01\\
56.9530750539957	-0.01\\
57.0030777537797	-0.01\\
57.0530804535637	-0.01\\
57.1030831533477	-0.01\\
57.1530858531318	-0.01\\
57.2030885529158	-0.01\\
57.2530912526998	-0.01\\
57.3030939524838	-0.01\\
57.3530966522678	-0.01\\
57.4030993520518	-0.01\\
57.4531020518358	-0.01\\
57.5031047516199	-0.01\\
57.5531074514039	-0.01\\
57.6031101511879	-0.01\\
57.6531128509719	-0.01\\
57.7031155507559	-0.01\\
57.75311825054	-0.01\\
57.803120950324	-0.01\\
57.853123650108	-0.01\\
57.903126349892	-0.01\\
57.953129049676	-0.01\\
58.00313174946	-0.01\\
58.0531344492441	-0.01\\
58.1031371490281	-0.01\\
58.1531398488121	-0.01\\
58.2031425485961	-0.01\\
58.2531452483801	-0.01\\
58.3031479481642	-0.01\\
58.3531506479482	-0.01\\
58.4031533477322	-0.01\\
58.4531560475162	-0.01\\
58.5031587473002	-0.01\\
58.5531614470842	-0.01\\
58.6031641468683	-0.01\\
58.6531668466523	-0.01\\
58.7031695464363	-0.01\\
58.7531722462203	-0.01\\
58.8031749460043	-0.01\\
58.8531776457883	-0.01\\
58.9031803455724	-0.01\\
58.9531830453564	-0.01\\
59.0031857451404	-0.01\\
59.0531884449244	-0.01\\
59.1031911447084	-0.01\\
59.1531938444924	-0.01\\
59.2031965442765	-0.01\\
59.2531992440605	-0.01\\
59.3032019438445	-0.01\\
59.3532046436285	-0.01\\
59.4032073434125	-0.01\\
59.4532100431965	-0.01\\
59.5032127429806	-0.01\\
59.5532154427646	-0.01\\
59.6032181425486	-0.01\\
59.6532208423326	-0.01\\
59.7032235421166	-0.01\\
59.7532262419006	-0.01\\
59.8032289416847	-0.01\\
59.8532316414687	-0.01\\
59.9032343412527	-0.01\\
59.9532370410367	-0.01\\
60.0032397408207	-0.01\\
60.0532424406048	-0.01\\
60.1032451403888	-0.01\\
60.1532478401728	-0.01\\
60.2032505399568	-0.01\\
60.2532532397408	-0.01\\
60.3032559395248	-0.01\\
60.3532586393089	-0.01\\
60.4032613390929	-0.01\\
60.4532640388769	-0.01\\
60.5032667386609	-0.01\\
60.5532694384449	-0.01\\
60.6032721382289	-0.01\\
60.653274838013	-0.01\\
60.703277537797	-0.01\\
60.753280237581	-0.01\\
60.803282937365	-0.01\\
60.853285637149	-0.01\\
60.9032883369331	-0.01\\
60.9532910367171	-0.01\\
61.0032937365011	-0.01\\
61.0532964362851	-0.01\\
61.1032991360691	-0.01\\
61.1533018358531	-0.01\\
61.2033045356371	-0.01\\
61.2533072354212	-0.01\\
61.3033099352052	-0.01\\
61.3533126349892	-0.01\\
61.4033153347732	-0.01\\
61.4533180345572	-0.01\\
61.5033207343413	-0.01\\
61.5533234341253	-0.01\\
61.6033261339093	-0.01\\
61.6533288336933	-0.01\\
61.7033315334773	-0.01\\
61.7533342332613	-0.01\\
61.8033369330454	-0.01\\
61.8533396328294	-0.01\\
61.9033423326134	-0.01\\
61.9533450323974	-0.01\\
62.0033477321814	-0.01\\
62.0533504319654	-0.01\\
62.1033531317495	-0.01\\
62.1533558315335	-0.01\\
62.2033585313175	-0.01\\
62.2533612311015	-0.01\\
62.3033639308855	-0.01\\
62.3533666306695	-0.01\\
62.4033693304536	-0.01\\
62.4533720302376	-0.01\\
62.5033747300216	-0.01\\
62.5483771598272	0.19\\
62.5983798596112	0.19\\
62.6483825593953	0.19\\
62.6983852591793	0.19\\
62.7483879589633	0.19\\
62.7983906587473	0.19\\
62.8483933585313	0.19\\
62.8983960583153	0.19\\
62.9483987580994	0.19\\
62.9984014578834	0.19\\
63.0484041576674	0.19\\
63.0984068574514	0.19\\
63.1484095572354	0.19\\
63.1984122570194	0.19\\
63.2484149568035	0.19\\
63.2984176565875	0.19\\
63.3484203563715	0.19\\
63.3984230561555	0.19\\
63.4484257559395	0.19\\
63.4984284557235	0.19\\
63.5484311555076	0.19\\
63.5984338552916	0.19\\
63.6484365550756	0.19\\
63.6984392548596	0.19\\
63.7484419546436	0.19\\
63.7984446544276	0.19\\
63.8484473542117	0.19\\
63.8984500539957	0.19\\
63.9484527537797	0.19\\
63.9984554535637	0.19\\
64.0484581533477	0.19\\
64.0984608531318	0.19\\
64.1484635529158	0.19\\
64.1984662526998	0.19\\
64.2484689524838	0.19\\
64.2984716522678	0.19\\
64.3484743520518	0.19\\
64.3984770518359	0.19\\
64.4484797516199	0.19\\
64.4984824514039	0.19\\
64.5484851511879	0.19\\
64.5984878509719	0.19\\
64.648490550756	0.19\\
64.69849325054	0.19\\
64.748495950324	0.19\\
64.798498650108	0.19\\
64.848501349892	0.19\\
64.898504049676	0.19\\
64.94850674946	0.19\\
64.9985094492441	0.19\\
65.0485121490281	0.19\\
65.0985148488121	0.19\\
65.1485175485961	0.19\\
65.1985202483801	0.19\\
65.2485229481642	0.19\\
65.2985256479482	0.19\\
65.3485283477322	0.19\\
65.3985310475162	0.19\\
65.4485337473002	0.19\\
65.4985364470842	0.19\\
65.5485391468683	0.19\\
65.5985418466523	0.19\\
65.6485445464363	0.19\\
65.6985472462203	0.19\\
65.7485499460043	0.19\\
65.7985526457883	0.19\\
65.8485553455724	0.19\\
65.8985580453564	0.19\\
65.9485607451404	0.19\\
65.9985634449244	0.19\\
66.0485661447084	0.19\\
66.0985688444924	0.19\\
66.1485715442765	0.19\\
66.1985742440605	0.19\\
66.2485769438445	0.19\\
66.2985796436285	0.19\\
66.3485823434125	0.19\\
66.3985850431965	0.19\\
66.4485877429806	0.19\\
66.4985904427646	0.19\\
66.5485931425486	0.19\\
66.5985958423326	0.19\\
66.6485985421166	0.19\\
66.6986012419006	0.19\\
66.7486039416847	0.19\\
66.7986066414687	0.19\\
66.8486093412527	0.19\\
66.8986120410367	0.19\\
66.9486147408207	0.19\\
66.9986174406047	0.19\\
67.0486201403888	0.19\\
67.0986228401728	0.19\\
67.1486255399568	0.19\\
67.1986282397408	0.19\\
67.2486309395248	0.19\\
67.2986336393089	0.19\\
67.3486363390929	0.19\\
67.3986390388769	0.19\\
67.4486417386609	0.19\\
67.4986444384449	0.19\\
67.5486471382289	0.19\\
67.598649838013	0.19\\
67.648652537797	0.19\\
67.698655237581	0.19\\
67.748657937365	0.19\\
67.798660637149	0.19\\
67.8486633369331	0.19\\
67.8986660367171	0.19\\
67.9486687365011	0.19\\
67.9986714362851	0.19\\
68.0486741360691	0.19\\
68.0986768358531	0.19\\
68.1486795356372	0.19\\
68.1986822354212	0.19\\
68.2486849352052	0.19\\
68.2986876349892	0.19\\
68.3486903347732	0.19\\
68.3986930345572	0.19\\
68.4486957343413	0.19\\
68.4986984341253	0.19\\
68.5487011339093	0.19\\
68.5987038336933	0.19\\
68.6487065334773	0.19\\
68.6987092332613	0.19\\
68.7487119330454	0.19\\
68.7987146328294	0.19\\
68.8487173326134	0.19\\
68.8987200323974	0.19\\
68.9487227321814	0.19\\
68.9987254319654	0.19\\
69.0487281317495	0.19\\
69.0987308315335	0.19\\
69.1487335313175	0.19\\
69.1987362311015	0.19\\
69.2487389308855	0.19\\
69.2987416306696	0.19\\
69.3487443304536	0.19\\
69.3987470302376	0.19\\
69.4487497300216	0.19\\
69.4987524298056	0.19\\
69.5487551295896	0.19\\
69.5987578293737	0.19\\
69.6487605291577	0.19\\
69.6987632289417	0.19\\
69.7487659287257	0.19\\
69.7987686285097	0.19\\
69.8487713282937	0.19\\
69.8987740280778	0.19\\
69.9487767278618	0.19\\
69.9987794276458	0.19\\
70.0487821274298	0.19\\
70.0987848272138	0.19\\
70.1487875269979	0.19\\
70.1987902267819	0.19\\
70.2487929265659	0.19\\
70.2987956263499	0.19\\
70.3487983261339	0.19\\
70.3988010259179	0.19\\
70.4488037257019	0.19\\
70.498806425486	0.19\\
70.54880912527	0.19\\
70.598811825054	0.19\\
70.648814524838	0.19\\
70.698817224622	0.19\\
70.7488199244061	0.19\\
70.7988226241901	0.19\\
70.8488253239741	0.19\\
70.8988280237581	0.19\\
70.9488307235421	0.19\\
70.9988334233261	0.19\\
71.0488361231101	0.19\\
71.0988388228942	0.19\\
71.1488415226782	0.19\\
71.1988442224622	0.19\\
71.2488469222462	0.19\\
71.2988496220302	0.19\\
71.3488523218143	0.19\\
71.3988550215983	0.19\\
71.4488577213823	0.19\\
71.4988604211663	0.19\\
71.5488631209503	0.19\\
71.5988658207343	0.19\\
71.6488685205184	0.19\\
71.6988712203024	0.19\\
71.7488739200864	0.19\\
71.7988766198704	0.19\\
71.8488793196544	0.19\\
71.8988820194385	0.19\\
71.9488847192225	0.19\\
71.9988874190065	0.19\\
72.0488901187905	0.19\\
72.0988928185745	0.19\\
72.1488955183585	0.19\\
72.1988982181425	0.19\\
72.2489009179266	0.19\\
72.2989036177106	0.19\\
72.3489063174946	0.19\\
72.3989090172786	0.19\\
72.4489117170626	0.19\\
72.4989144168467	0.19\\
72.5489171166307	0.19\\
72.5989198164147	0.19\\
72.6489225161987	0.19\\
72.6989252159827	0.19\\
72.7489279157667	0.19\\
72.7989306155508	0.19\\
72.8489333153348	0.19\\
72.8989360151188	0.19\\
72.9489387149028	0.19\\
72.9989414146868	0.19\\
73.0489441144708	0.19\\
73.0989468142549	0.19\\
73.1489495140389	0.19\\
73.1989522138229	0.19\\
73.2489549136069	0.19\\
73.2989576133909	0.19\\
73.348960313175	0.19\\
73.398963012959	0.19\\
73.448965712743	0.19\\
73.498968412527	0.19\\
73.548971112311	0.19\\
73.598973812095	0.19\\
73.6489765118791	0.19\\
73.6989792116631	0.19\\
73.7489819114471	0.19\\
73.7989846112311	0.19\\
73.8489873110151	0.19\\
73.8989900107991	0.19\\
73.9489927105832	0.19\\
73.9989954103672	0.19\\
74.0489981101512	0.19\\
74.0990008099352	0.19\\
74.1490035097192	0.19\\
74.1990062095032	0.19\\
74.2490089092873	0.19\\
74.2990116090713	0.19\\
74.3490143088553	0.19\\
74.3990170086393	0.19\\
74.4490197084233	0.19\\
74.4990224082073	0.19\\
74.5490251079914	0.19\\
74.5990278077754	0.19\\
74.6490305075594	0.19\\
74.6990332073434	0.19\\
74.7490359071274	0.19\\
74.7990386069115	0.19\\
74.8490413066955	0.19\\
74.8990440064795	0.19\\
74.9490467062635	0.19\\
74.9990494060475	0.19\\
75.0490521058315	0.19\\
75.0990548056156	0.19\\
75.1490575053996	0.19\\
75.1990602051836	0.19\\
75.2490629049676	0.19\\
75.2990656047516	0.19\\
75.3490683045356	0.19\\
75.3990710043197	0.19\\
75.4490737041037	0.19\\
75.4990764038877	0.19\\
75.5490791036717	0.19\\
75.5990818034557	0.19\\
75.6490845032397	0.19\\
75.6990872030238	0.19\\
75.7490899028078	0.19\\
75.7990926025918	0.19\\
75.8490953023758	0.19\\
75.8990980021598	0.19\\
75.9491007019438	0.19\\
75.9991034017279	0.19\\
76.0491061015119	0.19\\
76.0991088012959	0.19\\
76.1491115010799	0.19\\
76.1991142008639	0.19\\
76.2491169006479	0.19\\
76.299119600432	0.19\\
76.349122300216	0.19\\
76.399125	0.19\\
76.449127699784	0.19\\
76.499130399568	0.19\\
76.5491330993521	0.19\\
76.5991357991361	0.19\\
76.6491384989201	0.19\\
76.6991411987041	0.19\\
76.7491438984881	0.19\\
76.7991465982721	0.19\\
76.8491492980562	0.19\\
76.8991519978402	0.19\\
76.9491546976242	0.19\\
76.9991573974082	0.19\\
77.0491600971922	0.19\\
77.0991627969762	0.19\\
77.1491654967603	0.19\\
77.1991681965443	0.19\\
77.2491708963283	0.19\\
77.2991735961123	0.19\\
77.3491762958963	0.19\\
77.3991789956804	0.19\\
77.4491816954644	0.19\\
77.4991843952484	0.19\\
77.5491870950324	0.19\\
77.5991897948164	0.19\\
77.6491924946004	0.19\\
77.6991951943844	0.19\\
77.7491978941685	0.19\\
77.7992005939525	0.19\\
77.8492032937365	0.19\\
77.8992059935205	0.19\\
77.9492086933045	0.19\\
77.9992113930886	0.19\\
78.0492140928726	0.19\\
78.0992167926566	0.19\\
78.1492194924406	0.19\\
78.1992221922246	0.19\\
78.2492248920086	0.19\\
78.2992275917927	0.19\\
78.3492302915767	0.19\\
78.3992329913607	0.19\\
78.4492356911447	0.19\\
78.4992383909287	0.19\\
78.5492410907127	0.19\\
78.5992437904968	0.19\\
78.6492464902808	0.19\\
78.6992491900648	0.19\\
78.7492518898488	0.19\\
78.7992545896328	0.19\\
78.8492572894169	0.19\\
78.8992599892009	0.19\\
78.9492626889849	0.19\\
78.9992653887689	0.19\\
79.0492680885529	0.19\\
79.0992707883369	0.19\\
79.149273488121	0.19\\
79.199276187905	0.19\\
79.249278887689	0.19\\
79.299281587473	0.19\\
79.349284287257	0.19\\
79.3992869870411	0.19\\
79.4492896868251	0.19\\
79.4992923866091	0.19\\
79.5492950863931	0.19\\
79.5992977861771	0.19\\
79.6493004859611	0.19\\
79.6993031857451	0.19\\
79.7493058855292	0.19\\
79.7993085853132	0.19\\
79.8493112850972	0.19\\
79.8993139848812	0.19\\
79.9493166846652	0.19\\
79.9993193844493	0.19\\
80.0493220842333	0.19\\
80.0993247840173	0.19\\
80.1493274838013	0.19\\
80.1993301835853	0.19\\
80.2493328833693	0.19\\
80.2993355831534	0.19\\
80.3493382829374	0.19\\
80.3993409827214	0.19\\
80.4493436825054	0.19\\
80.4993463822894	0.19\\
80.5493490820734	0.19\\
80.5993517818575	0.19\\
80.6493544816415	0.19\\
80.6993571814255	0.19\\
80.7493598812095	0.19\\
80.7993625809935	0.19\\
80.8493652807775	0.19\\
80.8993679805616	0.19\\
80.9493706803456	0.19\\
80.9993733801296	0.19\\
81.0493760799136	0.19\\
81.0993787796976	0.19\\
81.1493814794816	0.19\\
81.1993841792657	0.19\\
81.2493868790497	0.19\\
81.2993895788337	0.19\\
81.3493922786177	0.19\\
81.3993949784017	0.19\\
81.4493976781857	0.19\\
81.4994003779698	0.19\\
81.5494030777538	0.19\\
81.5994057775378	0.19\\
81.6494084773218	0.19\\
81.6994111771058	0.19\\
81.7494138768898	0.19\\
81.7994165766739	0.19\\
81.8494192764579	0.19\\
81.8994219762419	0.19\\
81.9494246760259	0.19\\
81.9994273758099	0.19\\
82.049430075594	0.19\\
82.099432775378	0.19\\
82.149435475162	0.19\\
82.199438174946	0.19\\
82.24944087473	0.19\\
82.299443574514	0.19\\
82.3494462742981	0.19\\
82.3994489740821	0.19\\
82.4494516738661	0.19\\
82.4994543736501	0.19\\
82.5494570734341	0.19\\
82.5994597732181	0.19\\
82.6294613930885	-0.01\\
82.6794640928726	-0.01\\
82.7294667926566	-0.01\\
82.7794694924406	-0.01\\
82.8294721922246	-0.01\\
82.8794748920086	-0.01\\
82.9294775917927	-0.01\\
82.9794802915767	-0.01\\
83.0294829913607	-0.01\\
83.0794856911447	-0.01\\
83.1294883909287	-0.01\\
83.1794910907127	-0.01\\
83.2294937904968	-0.01\\
83.2794964902808	-0.01\\
83.3294991900648	-0.01\\
83.3795018898488	-0.01\\
83.4295045896328	-0.01\\
83.4795072894169	-0.01\\
83.5295099892009	-0.01\\
83.5795126889849	-0.01\\
83.6295153887689	-0.01\\
83.6795180885529	-0.01\\
83.7295207883369	-0.01\\
83.779523488121	-0.01\\
83.829526187905	-0.01\\
83.879528887689	-0.01\\
83.929531587473	-0.01\\
83.979534287257	-0.01\\
84.029536987041	-0.01\\
84.0795396868251	-0.01\\
84.1295423866091	-0.01\\
84.1795450863931	-0.01\\
84.2295477861771	-0.01\\
84.2795504859611	-0.01\\
84.3295531857451	-0.01\\
84.3795558855292	-0.01\\
84.4295585853132	-0.01\\
84.4795612850972	-0.01\\
84.5295639848812	-0.01\\
84.5795666846652	-0.01\\
84.6295693844492	-0.01\\
84.6795720842333	-0.01\\
84.7295747840173	-0.01\\
84.7795774838013	-0.01\\
84.8295801835853	-0.01\\
84.8795828833693	-0.01\\
84.9295855831534	-0.01\\
84.9795882829374	-0.01\\
85.0295909827214	-0.01\\
85.0795936825054	-0.01\\
85.1295963822894	-0.01\\
85.1795990820734	-0.01\\
85.2296017818575	-0.01\\
85.2796044816415	-0.01\\
85.3296071814255	-0.01\\
85.3796098812095	-0.01\\
85.4296125809935	-0.01\\
85.4796152807775	-0.01\\
85.5296179805616	-0.01\\
85.5796206803456	-0.01\\
85.6296233801296	-0.01\\
85.6796260799136	-0.01\\
85.7296287796976	-0.01\\
85.7796314794817	-0.01\\
85.8296341792657	-0.01\\
85.8796368790497	-0.01\\
85.9296395788337	-0.01\\
85.9796422786177	-0.01\\
86.0296449784017	-0.01\\
86.0796476781857	-0.01\\
86.1296503779698	-0.01\\
86.1796530777538	-0.01\\
86.2296557775378	-0.01\\
86.2796584773218	-0.01\\
86.3296611771058	-0.01\\
86.3796638768899	-0.01\\
86.4296665766739	-0.01\\
86.4796692764579	-0.01\\
86.5296719762419	-0.01\\
86.5796746760259	-0.01\\
86.6296773758099	-0.01\\
86.679680075594	-0.01\\
86.729682775378	-0.01\\
86.779685475162	-0.01\\
86.829688174946	-0.01\\
86.87969087473	-0.01\\
86.929693574514	-0.01\\
86.9796962742981	-0.01\\
87.0296989740821	-0.01\\
87.0797016738661	-0.01\\
87.1297043736501	-0.01\\
87.1797070734341	-0.01\\
87.2297097732181	-0.01\\
87.2797124730022	-0.01\\
87.3297151727862	-0.01\\
87.3797178725702	-0.01\\
87.4297205723542	-0.01\\
87.4797232721382	-0.01\\
87.5297259719222	-0.01\\
87.5797286717063	-0.01\\
87.6297313714903	-0.01\\
87.6797340712743	-0.01\\
87.7297367710583	-0.01\\
87.7797394708423	-0.01\\
87.8297421706264	-0.01\\
87.8797448704104	-0.01\\
87.9297475701944	-0.01\\
87.9797502699784	-0.01\\
88.0297529697624	-0.01\\
88.0797556695464	-0.01\\
88.1297583693305	-0.01\\
88.1797610691145	-0.01\\
88.2297637688985	-0.01\\
88.2797664686825	-0.01\\
88.3297691684665	-0.01\\
88.3797718682505	-0.01\\
88.4297745680346	-0.01\\
88.4797772678186	-0.01\\
88.5297799676026	-0.01\\
88.5797826673866	-0.01\\
88.6297853671706	-0.01\\
88.6797880669546	-0.01\\
88.7297907667387	-0.01\\
88.7797934665227	-0.01\\
88.8297961663067	-0.01\\
88.8797988660907	-0.01\\
88.9298015658747	-0.01\\
88.9798042656588	-0.01\\
89.0298069654428	-0.01\\
89.0798096652268	-0.01\\
89.1298123650108	-0.01\\
89.1798150647948	-0.01\\
89.2298177645788	-0.01\\
89.2798204643629	-0.01\\
89.3298231641469	-0.01\\
89.3798258639309	-0.01\\
89.4298285637149	-0.01\\
89.4798312634989	-0.01\\
89.5298339632829	-0.01\\
89.579836663067	-0.01\\
89.629839362851	-0.01\\
89.679842062635	-0.01\\
89.729844762419	-0.01\\
89.779847462203	-0.01\\
89.8298501619871	-0.01\\
89.8798528617711	-0.01\\
89.9298555615551	-0.01\\
89.9798582613391	-0.01\\
90.0298609611231	-0.01\\
90.0798636609071	-0.01\\
90.1298663606911	-0.01\\
90.1798690604752	-0.01\\
90.2298717602592	-0.01\\
90.2798744600432	-0.01\\
90.3298771598272	-0.01\\
90.3798798596112	-0.01\\
90.4298825593953	-0.01\\
90.4798852591793	-0.01\\
90.5298879589633	-0.01\\
90.5798906587473	-0.01\\
90.6298933585313	-0.01\\
90.6798960583153	-0.01\\
90.7298987580994	-0.01\\
90.7799014578834	-0.01\\
90.8299041576674	-0.01\\
90.8799068574514	-0.01\\
90.9299095572354	-0.01\\
90.9799122570195	-0.01\\
91.0299149568035	-0.01\\
91.0799176565875	-0.01\\
91.1299203563715	-0.01\\
91.1799230561555	-0.01\\
91.2299257559395	-0.01\\
91.2799284557236	-0.01\\
91.3299311555076	-0.01\\
91.3799338552916	-0.01\\
91.4299365550756	-0.01\\
91.4799392548596	-0.01\\
91.5299419546436	-0.01\\
91.5799446544276	-0.01\\
91.6299473542117	-0.01\\
91.6799500539957	-0.01\\
91.7299527537797	-0.01\\
91.7799554535637	-0.01\\
91.8299581533477	-0.01\\
91.8799608531317	-0.01\\
91.9299635529158	-0.01\\
91.9799662526998	-0.01\\
92.0299689524838	-0.01\\
92.0799716522678	-0.01\\
92.1299743520518	-0.01\\
92.1799770518358	-0.01\\
92.2299797516199	-0.01\\
92.2799824514039	-0.01\\
92.3299851511879	-0.01\\
92.3799878509719	-0.01\\
92.4299905507559	-0.01\\
92.47999325054	-0.01\\
92.529995950324	-0.01\\
92.579998650108	-0.01\\
92.605	-0.01\\
};
\addlegendentry{- 1cm};

\addplot [color=red,solid,line width=0.2pt]
  table[row sep=crcr]{0	-0.075003\\
0.0500026997840173	-0.074983\\
0.100005399568035	-0.074955\\
0.150008099352052	-0.073954\\
0.200010799136069	-0.075066\\
0.250013498920086	-0.074146\\
0.300016198704104	-0.073995\\
0.350018898488121	-0.073456\\
0.400021598272138	-0.073304\\
0.450024298056156	-0.073288\\
0.500026997840173	-0.073859\\
0.55002969762419	-0.073271\\
0.600032397408207	-0.075337\\
0.650035097192225	-0.073829\\
0.700037796976242	-0.073349\\
0.750040496760259	-0.073073\\
0.800043196544276	-0.073011\\
0.850045896328294	-0.074833\\
0.900048596112311	-0.073879\\
0.950051295896328	-0.074291\\
1.00005399568035	-0.073886\\
1.05005669546436	-0.073226\\
1.10005939524838	-0.07297\\
1.1500620950324	-0.07345\\
1.20006479481641	-0.073253\\
1.25006749460043	-0.073608\\
1.30007019438445	-0.073357\\
1.35007289416847	-0.072791\\
1.40007559395248	-0.072473\\
1.4500782937365	-0.072299\\
1.50008099352052	-0.071944\\
1.55008369330454	-0.071813\\
1.60008639308855	-0.072657\\
1.65008909287257	-0.072566\\
1.70009179265659	-0.072121\\
1.7500944924406	-0.071908\\
1.80009719222462	-0.071798\\
1.85009989200864	-0.072185\\
1.90010259179266	-0.072439\\
1.95010529157667	-0.072051\\
2.00010799136069	-0.072449\\
2.05011069114471	-0.070773\\
2.10011339092873	-0.066297\\
2.15011609071274	-0.060216\\
2.20011879049676	-0.049576\\
2.25012149028078	-0.038458\\
2.30012419006479	-0.030138\\
2.35012688984881	-0.02659\\
2.40012958963283	-0.024174\\
2.45013228941685	-0.021831\\
2.50013498920086	-0.019046\\
2.55013768898488	-0.016436\\
2.6001403887689	-0.013247\\
2.65014308855292	-0.012173\\
2.70014578833693	-0.011279\\
2.75014848812095	-0.010226\\
2.80015118790497	-0.009076\\
2.85015388768899	-0.007366\\
2.900156587473	-0.007382\\
2.95015928725702	-0.006207\\
3.00016198704104	-0.006207\\
3.05016468682505	-0.00545\\
3.10016738660907	-0.003491\\
3.15017008639309	-0.003284\\
3.20017278617711	-0.003722\\
3.25017548596112	-0.004076\\
3.30017818574514	-0.004393\\
3.35018088552916	-0.003226\\
3.40018358531318	-0.002523\\
3.45018628509719	-0.003838\\
3.50018898488121	-0.004748\\
3.55019168466523	-0.005132\\
3.60019438444924	-0.004096\\
3.65019708423326	-0.004085\\
3.70019978401728	-0.00422\\
3.7502024838013	-0.00456\\
3.80020518358531	-0.004908\\
3.85020788336933	-0.0044\\
3.90021058315335	-0.003887\\
3.95021328293736	-0.002921\\
4.00021598272138	-0.001625\\
4.0502186825054	-0.001229\\
4.10022138228942	-0.002481\\
4.15022408207343	-0.003454\\
4.20022678185745	-0.003244\\
4.25022948164147	-0.003109\\
4.30023218142549	-0.00209\\
4.3502348812095	-0.002396\\
4.40023758099352	-0.002751\\
4.45024028077754	-0.001355\\
4.50024298056155	-0.001316\\
4.55024568034557	-0.001413\\
4.60024838012959	-0.001619\\
4.65025107991361	-0.001125\\
4.70025377969762	-0.001186\\
4.75025647948164	-0.000338\\
4.80025917926566	-0.000139\\
4.85026187904968	-0.000115\\
4.90026457883369	-0.00022\\
4.95026727861771	7.6e-05\\
5.00026997840173	-0.000148\\
5.05027267818575	-0.000546\\
5.10027537796976	-0.000892\\
5.15027807775378	-0.001127\\
5.2002807775378	-0.000283\\
5.25028347732181	0.000874\\
5.30028617710583	0.000295\\
5.35028887688985	8.6e-05\\
5.40029157667387	0.000259\\
5.45029427645788	-0.000391\\
5.5002969762419	-0.001348\\
5.55029967602592	-0.000577\\
5.60030237580994	-0.000573\\
5.65030507559395	-0.000616\\
5.70030777537797	-0.000678\\
5.75031047516199	-0.000872\\
5.800313174946	-0.00113\\
5.85031587473002	-0.000484\\
5.90031857451404	-0.000644\\
5.95032127429806	-0.000821\\
6.00032397408207	-0.00062\\
6.05032667386609	-0.00029\\
6.10032937365011	0.000829\\
6.15033207343413	0.000662\\
6.20033477321814	0.000289\\
6.25033747300216	0.001395\\
6.30034017278618	0.00207\\
6.3503428725702	0.001229\\
6.40034557235421	0.000862\\
6.45034827213823	0.00131\\
6.50035097192225	0.000227\\
6.55035367170626	0.001733\\
6.60035637149028	0.001414\\
6.6503590712743	-0.000218\\
6.70036177105832	-0.001463\\
6.75036447084233	-0.001581\\
6.80036717062635	-0.000933\\
6.85036987041037	0.001031\\
6.90037257019439	-0.000786\\
6.9503752699784	-0.002603\\
7.00037796976242	-0.003766\\
7.05038066954644	-0.004201\\
7.10038336933045	-0.003364\\
7.15038606911447	-0.003124\\
7.20038876889849	-0.003674\\
7.25039146868251	-0.002977\\
7.30039416846652	-0.003123\\
7.35039686825054	-0.003868\\
7.40039956803456	-0.003522\\
7.45040226781857	-0.003455\\
7.50040496760259	-0.002184\\
7.55040766738661	-0.00125\\
7.60041036717063	0.000722\\
7.65041306695464	-0.000264\\
7.70041576673866	-0.000339\\
7.75041846652268	-0.001041\\
7.8004211663067	-0.000131\\
7.85042386609071	0.00348\\
7.90042656587473	0.002925\\
7.95042926565875	0.001898\\
8.00043196544276	0.001288\\
8.05043466522678	0.000726\\
8.1004373650108	0.001815\\
8.15044006479482	0.001858\\
8.20044276457883	0.001495\\
8.25044546436285	0.000999\\
8.30044816414687	-0.000346\\
8.35045086393089	-0.003011\\
8.4004535637149	-0.004537\\
8.45045626349892	-0.004557\\
8.50045896328294	-0.002099\\
8.55046166306696	-0.002409\\
8.60046436285097	-0.003178\\
8.65046706263499	-0.004498\\
8.70046976241901	-0.005089\\
8.75047246220302	-0.004953\\
8.80047516198704	-0.004918\\
8.85047786177106	-0.0049\\
8.90048056155508	-0.00396\\
8.95048326133909	-0.00345\\
9.00048596112311	-0.001789\\
9.05048866090713	-0.001724\\
9.10049136069114	-0.001457\\
9.15049406047516	-0.000356\\
9.20049676025918	-0.000451\\
9.2504994600432	-0.000413\\
9.30050215982721	-0.000287\\
9.35050485961123	0.000213\\
9.40050755939525	0.000973\\
9.45051025917927	0.000947\\
9.50051295896328	0.002084\\
9.5505156587473	0.001466\\
9.60051835853132	0.001353\\
9.65052105831533	0.000178\\
9.70052375809935	-0.000551\\
9.75052645788337	-0.001248\\
9.80052915766739	-0.00129\\
9.8505318574514	-0.001355\\
9.90053455723542	-0.00132\\
9.95053725701944	-0.000797\\
10.0005399568035	-0.000423\\
10.0505426565875	-0.000156\\
10.1005453563715	0.000418\\
10.1505480561555	0.00023\\
10.2005507559395	0.000394\\
10.2505534557235	0.001086\\
10.3005561555076	0.000637\\
10.3505588552916	0.000321\\
10.4005615550756	-4.1e-05\\
10.4505642548596	-0.000723\\
10.5005669546436	-0.000904\\
10.5505696544276	-0.000589\\
10.6005723542117	-0.00056\\
10.6505750539957	-0.000241\\
10.7005777537797	-0.000559\\
10.7505804535637	0.00028\\
10.8005831533477	0.001932\\
10.8505858531317	0.001772\\
10.9005885529158	0.000876\\
10.9505912526998	0.001343\\
11.0005939524838	0.001237\\
11.0505966522678	0.000811\\
11.1005993520518	0.001141\\
11.1506020518359	0.000993\\
11.2006047516199	0.001048\\
11.2506074514039	0.001857\\
11.3006101511879	0.001288\\
11.3506128509719	0.000592\\
11.4006155507559	0.00039\\
11.45061825054	0.000264\\
11.500620950324	-4e-05\\
11.550623650108	0.000601\\
11.600626349892	0.000572\\
11.650629049676	0.000258\\
11.70063174946	0.000704\\
11.7506344492441	0.000452\\
11.8006371490281	-0.000178\\
11.8506398488121	-0.000898\\
11.9006425485961	-0.001077\\
11.9506452483801	-0.000951\\
12.0006479481641	-0.000318\\
12.0506506479482	-5.8e-05\\
12.1006533477322	-9.6e-05\\
12.1506560475162	-3e-06\\
12.2006587473002	0.000441\\
12.2506614470842	0.001098\\
12.3006641468683	-2.6e-05\\
12.3506668466523	-0.000746\\
12.4006695464363	-0.000981\\
12.4506722462203	-0.000914\\
12.5006749460043	-0.000464\\
12.5506776457883	-0.000378\\
12.6006803455724	-0.000711\\
12.6506830453564	-0.000839\\
12.7006857451404	0.00066\\
12.7506884449244	0.000303\\
12.8006911447084	0.000347\\
12.8506938444924	0.000356\\
12.9006965442765	-0.001082\\
12.9506992440605	-0.001425\\
13.0007019438445	-0.001715\\
13.0507046436285	-0.001476\\
13.1007073434125	-0.001141\\
13.1507100431965	-0.001183\\
13.2007127429806	-0.001379\\
13.2507154427646	-0.000768\\
13.3007181425486	-0.001726\\
13.3507208423326	-0.002396\\
13.4007235421166	-0.00239\\
13.4507262419006	-0.001559\\
13.5007289416847	-0.002046\\
13.5507316414687	-0.001881\\
13.6007343412527	-0.001983\\
13.6507370410367	-0.001245\\
13.7007397408207	-0.001056\\
13.7507424406048	-0.001496\\
13.8007451403888	-0.001744\\
13.8507478401728	-0.002279\\
13.9007505399568	-0.00252\\
13.9507532397408	-0.002517\\
14.0007559395248	-0.001835\\
14.0507586393089	-0.000818\\
14.1007613390929	-0.000685\\
14.1507640388769	-0.000948\\
14.2007667386609	-0.001333\\
14.2507694384449	-0.001336\\
14.3007721382289	-0.000471\\
14.350774838013	0.000806\\
14.400777537797	0.001366\\
14.450780237581	0.001269\\
14.500782937365	0.000187\\
14.550785637149	0.000404\\
14.600788336933	-0.000773\\
14.6507910367171	-0.001171\\
14.7007937365011	0.00059\\
14.7507964362851	-2.8e-05\\
14.8007991360691	-0.000833\\
14.8508018358531	-0.001261\\
14.9008045356372	-0.002331\\
14.9508072354212	-0.003037\\
15.0008099352052	-0.003226\\
15.0508126349892	-0.002441\\
15.1008153347732	-0.001735\\
15.1508180345572	-0.002867\\
15.2008207343413	-0.00366\\
15.2508234341253	-0.003202\\
15.3008261339093	-0.002875\\
15.3508288336933	-0.002258\\
15.4008315334773	-0.000562\\
15.4508342332613	-0.000544\\
15.5008369330454	-0.000525\\
15.5508396328294	-0.0013\\
15.6008423326134	-0.002581\\
15.6508450323974	-0.002729\\
15.7008477321814	-0.002326\\
15.7508504319654	-0.002416\\
15.8008531317495	-0.000636\\
15.8508558315335	0.000348\\
15.9008585313175	-0.000927\\
15.9508612311015	-0.000598\\
16.0008639308855	-0.000968\\
16.0508666306695	-0.000954\\
16.1008693304536	-0.001695\\
16.1508720302376	-0.00107\\
16.2008747300216	-0.00211\\
16.2508774298056	-0.00295\\
16.3008801295896	-0.003853\\
16.3508828293737	-0.003772\\
16.4008855291577	-0.003106\\
16.4508882289417	-0.002199\\
16.5008909287257	-0.000958\\
16.5508936285097	-1.2e-05\\
16.6008963282937	0.000271\\
16.6508990280778	-0.000899\\
16.7009017278618	-0.001266\\
16.7509044276458	-7.7e-05\\
16.8009071274298	-0.000147\\
16.8509098272138	-0.001036\\
16.9009125269978	-0.000567\\
16.9509152267819	-0.001218\\
17.0009179265659	-0.001096\\
17.0509206263499	-0.000867\\
17.1009233261339	-0.001656\\
17.1509260259179	-0.002278\\
17.2009287257019	-0.002688\\
17.250931425486	-0.002569\\
17.30093412527	-0.002739\\
17.350936825054	-0.001736\\
17.400939524838	-0.001091\\
17.450942224622	-0.000953\\
17.500944924406	-0.000666\\
17.5509476241901	-0.000664\\
17.6009503239741	-0.0001\\
17.6509530237581	0.000654\\
17.7009557235421	0.00044\\
17.7509584233261	0.000312\\
17.8009611231102	0.000842\\
17.8509638228942	0.001009\\
17.9009665226782	0.000912\\
17.9509692224622	0.001315\\
18.0009719222462	0.001191\\
18.0509746220302	0.000482\\
18.1009773218143	-0.000669\\
18.1509800215983	-0.000821\\
18.2009827213823	0.00016\\
18.2509854211663	0.00052\\
18.3009881209503	0.000571\\
18.3509908207343	0.000527\\
18.4009935205184	0.000782\\
18.4509962203024	0.00124\\
18.5009989200864	0.001153\\
18.5510016198704	0.002918\\
18.6010043196544	0.003492\\
18.6510070194384	0.001586\\
18.7010097192225	0.001251\\
18.7510124190065	0.000436\\
18.8010151187905	-4.6e-05\\
18.8510178185745	-0.000634\\
18.9010205183585	-0.000461\\
18.9510232181425	-0.001073\\
19.0010259179266	-0.000593\\
19.0510286177106	-0.000474\\
19.1010313174946	-0.00037\\
19.1510340172786	-0.000341\\
19.2010367170626	-0.000302\\
19.2510394168467	-0.000211\\
19.3010421166307	-0.000801\\
19.3510448164147	-0.001197\\
19.4010475161987	-0.001381\\
19.4510502159827	-0.001101\\
19.5010529157667	-0.001062\\
19.5510556155508	-0.000857\\
19.6010583153348	-0.001064\\
19.6510610151188	-0.001314\\
19.7010637149028	-0.000624\\
19.7510664146868	-0.000732\\
19.8010691144708	-0.000762\\
19.8510718142549	-0.000849\\
19.9010745140389	-0.000421\\
19.9510772138229	-0.000378\\
20.0010799136069	-0.000334\\
20.0510826133909	-0.000315\\
20.1010853131749	6.7e-05\\
20.151088012959	-9e-05\\
20.201090712743	-0.000298\\
20.251093412527	-0.000804\\
20.301096112311	-0.00085\\
20.351098812095	-0.000548\\
20.4011015118791	-0.000628\\
20.4511042116631	-0.000562\\
20.5011069114471	-0.000286\\
20.5511096112311	-0.000442\\
20.6011123110151	-0.00102\\
20.6511150107991	-0.000222\\
20.7011177105832	0.000443\\
20.7511204103672	0.000409\\
20.8011231101512	0.001064\\
20.8511258099352	0.001397\\
20.9011285097192	0.003367\\
20.9511312095032	0.004354\\
21.0011339092873	0.002344\\
21.0511366090713	0.000658\\
21.1011393088553	0.000298\\
21.1511420086393	0.000475\\
21.2011447084233	0.000384\\
21.2511474082073	0.000978\\
21.3011501079914	0.00066\\
21.3511528077754	0.000426\\
21.4011555075594	0.000426\\
21.4511582073434	0.00043\\
21.5011609071274	0.000806\\
21.5511636069114	0.00172\\
21.6011663066955	0.001087\\
21.6511690064795	0.00035\\
21.7011717062635	0.003488\\
21.7511744060475	0.003507\\
21.8011771058315	0.00285\\
21.8511798056156	0.002122\\
21.9011825053996	0.001713\\
21.9511852051836	0.001712\\
22.0011879049676	0.001462\\
22.0511906047516	0.001087\\
22.1011933045356	0.000533\\
22.1511960043197	8.3e-05\\
22.2011987041037	-0.000132\\
22.2512014038877	-0.001331\\
22.3012041036717	-0.004184\\
22.3512068034557	-0.005679\\
22.4012095032397	-0.005819\\
22.4512122030238	-0.004488\\
22.5012149028078	-0.003657\\
22.5512176025918	-0.003425\\
22.6012203023758	-0.002426\\
22.6512230021598	-0.002608\\
22.7012257019438	-0.003989\\
22.7512284017279	-0.00293\\
22.8012311015119	-0.001825\\
22.8512338012959	-0.001131\\
22.9012365010799	-0.000975\\
22.9512392008639	-0.000849\\
23.0012419006479	-0.000703\\
23.051244600432	-0.001702\\
23.101247300216	-0.002215\\
23.15125	-0.001022\\
23.201252699784	0.001431\\
23.251255399568	-0.000626\\
23.3012580993521	-0.001549\\
23.3512607991361	-0.000209\\
23.4012634989201	-0.000746\\
23.4512661987041	0.000256\\
23.5012688984881	0.001305\\
23.5512715982721	0.000309\\
23.6012742980562	9.7e-05\\
23.6512769978402	0.001325\\
23.7012796976242	-0.000449\\
23.7512823974082	-0.000801\\
23.8012850971922	-0.001959\\
23.8512877969762	-0.002036\\
23.9012904967603	-0.002507\\
23.9512931965443	-0.00281\\
24.0012958963283	-0.0019\\
24.0512985961123	0.000461\\
24.1013012958963	0.000797\\
24.1513039956803	-9.7e-05\\
24.2013066954644	-0.000129\\
24.2513093952484	0.000708\\
24.3013120950324	0.000809\\
24.3513147948164	0.000495\\
24.4013174946004	0.000436\\
24.4513201943845	0.000767\\
24.5013228941685	0.000331\\
24.5513255939525	0.000577\\
24.6013282937365	-0.00065\\
24.6513309935205	-0.001352\\
24.7013336933045	-0.00186\\
24.7513363930886	-0.000869\\
24.8013390928726	-0.000372\\
24.8513417926566	-0.000186\\
24.9013444924406	0.00017\\
24.9513471922246	-0.000184\\
25.0013498920086	-0.000705\\
25.0513525917927	-0.000624\\
25.1013552915767	-0.000371\\
25.1513579913607	0.000798\\
25.2013606911447	0.000267\\
25.2513633909287	0.002109\\
25.3013660907127	0.001984\\
25.3513687904968	0.001681\\
25.4013714902808	0.001423\\
25.4513741900648	0.000817\\
25.5013768898488	0.000909\\
25.5513795896328	0.000249\\
25.6013822894168	-0.0002\\
25.6513849892009	0.001483\\
25.7013876889849	0.001361\\
25.7513903887689	0.000378\\
25.8013930885529	0.000234\\
25.8513957883369	0.001121\\
25.901398488121	0.002784\\
25.951401187905	0.00306\\
26.001403887689	0.00249\\
26.051406587473	0.00134\\
26.101409287257	0.00015\\
26.151411987041	-0.00106\\
26.2014146868251	0.00017\\
26.2514173866091	0.000122\\
26.3014200863931	0.000275\\
26.3514227861771	0.001988\\
26.4014254859611	0.002515\\
26.4514281857451	0.000666\\
26.5014308855292	-0.000814\\
26.5514335853132	-0.000797\\
26.6014362850972	-0.000399\\
26.6514389848812	0.000624\\
26.7014416846652	-0.000467\\
26.7514443844492	-0.000968\\
26.8014470842333	-0.001945\\
26.8514497840173	-0.001372\\
26.9014524838013	-0.00144\\
26.9514551835853	-0.00101\\
27.0014578833693	0.000302\\
27.0514605831534	-0.00084\\
27.1014632829374	-0.001684\\
27.1514659827214	-0.001495\\
27.2014686825054	-0.000821\\
27.2514713822894	2.1e-05\\
27.3014740820734	9.8e-05\\
27.3514767818575	-5.5e-05\\
27.4014794816415	-0.000255\\
27.4514821814255	0.000702\\
27.5014848812095	0.002969\\
27.5514875809935	0.001437\\
27.6014902807775	-0.000664\\
27.6514929805616	-0.000659\\
27.7014956803456	0.00022\\
27.7514983801296	0.000681\\
27.8015010799136	0.001145\\
27.8515037796976	0.001482\\
27.9015064794816	0.002174\\
27.9515091792657	0.002821\\
28.0015118790497	0.002352\\
28.0515145788337	0.003329\\
28.1015172786177	0.003065\\
28.1515199784017	0.001469\\
28.2015226781857	0.001404\\
28.2515253779698	0.002168\\
28.3015280777538	0.002557\\
28.3515307775378	0.002174\\
28.4015334773218	0.002307\\
28.4515361771058	0.001834\\
28.5015388768899	0.001462\\
28.5515415766739	0.001309\\
28.6015442764579	0.000767\\
28.6515469762419	0.000287\\
28.7015496760259	-7.1e-05\\
28.7515523758099	0.000532\\
28.801555075594	0.000476\\
28.851557775378	0.000184\\
28.901560475162	0.00087\\
28.951563174946	0.000492\\
29.00156587473	0.000195\\
29.051568574514	4.2e-05\\
29.1015712742981	0.000132\\
29.1515739740821	0.000536\\
29.2015766738661	-0.001627\\
29.2515793736501	-0.002352\\
29.3015820734341	6.8e-05\\
29.3515847732181	0.002701\\
29.4015874730022	0.001543\\
29.4515901727862	-0.000644\\
29.5015928725702	-0.001642\\
29.5515955723542	-0.002692\\
29.6015982721382	-0.003291\\
29.6516009719222	-0.004052\\
29.7016036717063	-0.003789\\
29.7516063714903	-0.004365\\
29.8016090712743	-0.003849\\
29.8516117710583	-0.002884\\
29.9016144708423	-0.000968\\
29.9516171706264	-0.000265\\
30.0016198704104	-0.00035\\
30.0516225701944	-7.9e-05\\
30.1016252699784	5.9e-05\\
30.1516279697624	8e-05\\
30.2016306695464	-0.000124\\
30.2516333693305	-0.001495\\
30.3016360691145	-0.001724\\
30.3516387688985	-0.001759\\
30.4016414686825	-0.002406\\
30.4516441684665	-0.000678\\
30.5016468682505	0.000659\\
30.5516495680346	0.000427\\
30.6016522678186	0.000386\\
30.6516549676026	-0.000313\\
30.7016576673866	-6.7e-05\\
30.7516603671706	6.2e-05\\
30.8016630669546	0.000122\\
30.8516657667387	8e-06\\
30.9016684665227	-0.002042\\
30.9516711663067	-0.000988\\
31.0016738660907	0.000239\\
31.0516765658747	0.001477\\
31.1016792656587	0.002072\\
31.1516819654428	0.002352\\
31.2016846652268	0.001227\\
31.2516873650108	0.000156\\
31.3016900647948	-0.001407\\
31.3516927645788	-0.002247\\
31.4016954643629	-0.001753\\
31.4516981641469	-0.001927\\
31.5017008639309	-0.000482\\
31.5517035637149	-0.001014\\
31.6017062634989	-0.002057\\
31.6517089632829	-0.002118\\
31.701711663067	-0.001594\\
31.751714362851	-0.001417\\
31.801717062635	-0.0008\\
31.851719762419	-7.4e-05\\
31.901722462203	0.00142\\
31.951725161987	0.001301\\
32.0017278617711	0.001576\\
32.0517305615551	0.003084\\
32.1017332613391	0.003937\\
32.1517359611231	0.003751\\
32.2017386609071	0.004106\\
32.2517413606911	0.003356\\
32.3017440604752	0.003239\\
32.3517467602592	0.003579\\
32.4017494600432	0.004336\\
32.4517521598272	0.00576\\
32.5017548596112	0.004938\\
32.5517575593952	0.003938\\
32.6017602591793	0.003459\\
32.6517629589633	0.004388\\
32.7017656587473	0.002456\\
32.7517683585313	0.001991\\
32.8017710583153	0.000901\\
32.8517737580993	-0.000228\\
32.9017764578834	-0.000426\\
32.9517791576674	-0.000698\\
33.0017818574514	-0.000773\\
33.0517845572354	-1e-06\\
33.1017872570194	0.001055\\
33.1517899568035	0.001259\\
33.2017926565875	0.001874\\
33.2517953563715	0.001415\\
33.3017980561555	0.000956\\
33.3518007559395	0.001491\\
33.4018034557235	0.000896\\
33.4518061555076	0.000446\\
33.5018088552916	-0.000154\\
33.5518115550756	-0.00028\\
33.6018142548596	0.000636\\
33.6518169546436	0.000594\\
33.7018196544277	0.001963\\
33.7518223542117	0.001605\\
33.8018250539957	-0.000556\\
33.8518277537797	-0.000975\\
33.9018304535637	-0.001765\\
33.9518331533477	-0.001546\\
34.0018358531318	-0.000955\\
34.0518385529158	-0.000721\\
34.1018412526998	-0.000503\\
34.1518439524838	0.000312\\
34.2018466522678	0.000513\\
34.2518493520518	-0.000383\\
34.3018520518359	-0.001435\\
34.3518547516199	-0.001256\\
34.4018574514039	-0.000319\\
34.4518601511879	0.000336\\
34.5018628509719	0.000396\\
34.5518655507559	0.00021\\
34.60186825054	0.000896\\
34.651870950324	0.000974\\
34.701873650108	0.000541\\
34.751876349892	0.000916\\
34.801879049676	0.001037\\
34.85188174946	0.000144\\
34.9018844492441	-0.000927\\
34.9518871490281	-0.000795\\
35.0018898488121	-0.000744\\
35.0518925485961	-0.000466\\
35.1018952483801	-0.000585\\
35.1518979481641	-0.000563\\
35.2019006479482	-0.000468\\
35.2519033477322	-7.3e-05\\
35.3019060475162	0.000127\\
35.3519087473002	0.000304\\
35.4019114470842	0.000302\\
35.4519141468683	0.000382\\
35.5019168466523	0.0006\\
35.5519195464363	0.000268\\
35.6019222462203	0.001058\\
35.6519249460043	-0.000153\\
35.7019276457883	-0.000187\\
35.7519303455724	-9.9e-05\\
35.8019330453564	-0.000302\\
35.8519357451404	-0.000503\\
35.9019384449244	-0.000419\\
35.9519411447084	-0.00054\\
36.0019438444924	-0.000704\\
36.0519465442765	-0.000754\\
36.1019492440605	-0.000678\\
36.1519519438445	-0.000524\\
36.2019546436285	-0.00027\\
36.2519573434125	0.00022\\
36.3019600431965	0.000374\\
36.3519627429806	0.00045\\
36.4019654427646	0.000601\\
36.4519681425486	0.001176\\
36.5019708423326	0.001753\\
36.5519735421166	0.001199\\
36.6019762419006	0.000332\\
36.6519789416847	-0.000237\\
36.7019816414687	-0.000368\\
36.7519843412527	-0.00011\\
36.8019870410367	-0.000386\\
36.8519897408207	-0.000655\\
36.9019924406048	4.7e-05\\
36.9519951403888	0.003273\\
37.0019978401728	0.002291\\
37.0520005399568	0.000155\\
37.1020032397408	-0.000589\\
37.1520059395248	-0.000857\\
37.2020086393089	-0.001545\\
37.2520113390929	-0.002093\\
37.3020140388769	-0.002181\\
37.3520167386609	-0.002056\\
37.4020194384449	-0.001086\\
37.4520221382289	0.000163\\
37.502024838013	0.000101\\
37.552027537797	-0.001578\\
37.602030237581	-0.002819\\
37.652032937365	-0.002939\\
37.702035637149	-0.002286\\
37.752038336933	-0.003363\\
37.8020410367171	-0.002636\\
37.8520437365011	-0.002885\\
37.9020464362851	-0.002836\\
37.9520491360691	-0.00196\\
38.0020518358531	-0.001381\\
38.0520545356372	-0.001641\\
38.1020572354212	-0.002199\\
38.1520599352052	-0.002522\\
38.2020626349892	-0.001957\\
38.2520653347732	-0.000627\\
38.3020680345572	-0.000366\\
38.3520707343413	-0.000876\\
38.4020734341253	-0.000837\\
38.4520761339093	-0.001529\\
38.5020788336933	-0.002187\\
38.5520815334773	-0.001354\\
38.6020842332613	-0.001315\\
38.6520869330454	-0.001128\\
38.7020896328294	-0.001426\\
38.7520923326134	-0.001604\\
38.8020950323974	-0.000835\\
38.8520977321814	-0.001468\\
38.9021004319654	-0.000856\\
38.9521031317495	-0.001313\\
39.0021058315335	-0.001383\\
39.0521085313175	-0.001014\\
39.1021112311015	-0.000163\\
39.1521139308855	-2.4e-05\\
39.2021166306696	-0.00022\\
39.2521193304536	0.001562\\
39.3021220302376	0.001112\\
39.3521247300216	0.000787\\
39.4021274298056	0.00132\\
39.4521301295896	0.003684\\
39.5021328293737	0.002306\\
39.5521355291577	0.001448\\
39.6021382289417	0.000143\\
39.6521409287257	-7e-05\\
39.7021436285097	-0.000516\\
39.7521463282937	-0.001232\\
39.8021490280778	-0.002899\\
39.8521517278618	-0.002467\\
39.9021544276458	-0.001196\\
39.9521571274298	-0.000397\\
40.0021598272138	0.000115\\
40.0521625269978	-0.000184\\
40.1021652267819	-0.000202\\
40.1521679265659	-0.000804\\
40.2021706263499	-0.000118\\
40.2521733261339	-0.001098\\
40.3021760259179	-0.001513\\
40.3521787257019	-0.001057\\
40.402181425486	-0.000858\\
40.45218412527	-0.000206\\
40.502186825054	0.000655\\
40.552189524838	0.001291\\
40.602192224622	0.000446\\
40.652194924406	0.000979\\
40.7021976241901	0.000357\\
40.7522003239741	0.001848\\
40.8022030237581	0.004324\\
40.8522057235421	0.00229\\
40.9022084233261	9.5e-05\\
40.9522111231102	-0.001619\\
41.0022138228942	-0.000867\\
41.0522165226782	3.4e-05\\
41.1022192224622	0.00111\\
41.1522219222462	0.000494\\
41.2022246220302	0.001196\\
41.2522273218143	0.000845\\
41.3022300215983	0.000868\\
41.3522327213823	0.000314\\
41.4022354211663	-0.000309\\
41.4522381209503	-0.000806\\
41.5022408207343	-0.001845\\
41.5522435205184	-0.002723\\
41.6022462203024	-0.00334\\
41.6522489200864	-0.002688\\
41.7022516198704	-0.002991\\
41.7522543196544	-0.003819\\
41.8022570194385	-0.002887\\
41.8522597192225	-0.001264\\
41.9022624190065	-0.000734\\
41.9522651187905	-0.001823\\
42.0022678185745	-0.001766\\
42.0522705183585	-0.002119\\
42.1022732181426	-0.002127\\
42.1522759179266	-0.001823\\
42.2022786177106	-0.000605\\
42.2522813174946	-0.001301\\
42.3022840172786	-0.002154\\
42.3522867170626	-0.001791\\
42.4022894168467	-0.001799\\
42.4522921166307	-0.00134\\
42.5022948164147	-0.000558\\
42.5522975161987	-0.001274\\
42.6023002159827	-0.000149\\
42.6523029157667	0.000437\\
42.7023056155508	9.9e-05\\
42.7523083153348	-0.000723\\
42.8023110151188	-0.001561\\
42.8523137149028	-0.001574\\
42.9023164146868	-0.00024\\
42.9523191144708	-0.000911\\
43.0023218142549	0.000252\\
43.0523245140389	0.001542\\
43.1023272138229	0.003832\\
43.1523299136069	0.002894\\
43.2023326133909	4.8e-05\\
43.2523353131749	-0.000508\\
43.302338012959	-0.000639\\
43.352340712743	-0.000619\\
43.402343412527	-8e-06\\
43.452346112311	-0.001281\\
43.502348812095	-0.00331\\
43.5523515118791	-0.003527\\
43.6023542116631	-0.003863\\
43.6523569114471	-0.003416\\
43.7023596112311	-0.003178\\
43.7523623110151	-0.003327\\
43.8023650107991	-0.002505\\
43.8523677105832	-0.002666\\
43.9023704103672	-0.003542\\
43.9523731101512	-0.004918\\
44.0023758099352	-0.002447\\
44.0523785097192	-0.002436\\
44.1023812095032	-0.001803\\
44.1523839092873	-0.0018\\
44.2023866090713	-0.001768\\
44.2523893088553	-0.001645\\
44.3023920086393	-0.000576\\
44.3523947084233	-0.001367\\
44.4023974082073	-0.002905\\
44.4524001079914	-0.004984\\
44.5024028077754	-0.003465\\
44.5524055075594	-0.004206\\
44.6024082073434	-0.002896\\
44.6524109071274	-0.003545\\
44.7024136069114	-0.003577\\
44.7524163066955	-0.003469\\
44.8024190064795	-0.002618\\
44.8524217062635	-0.002806\\
44.9024244060475	-0.003069\\
44.9524271058315	-0.002938\\
45.0024298056155	-0.002644\\
45.0524325053996	-0.002632\\
45.1024352051836	-0.00244\\
45.1524379049676	-0.002071\\
45.2024406047516	-0.002047\\
45.2524433045356	-0.002471\\
45.3024460043197	-0.00216\\
45.3524487041037	-0.00199\\
45.4024514038877	-0.001866\\
45.4524541036717	-0.001641\\
45.5024568034557	-0.002111\\
45.5524595032397	-0.002515\\
45.6024622030238	-0.001963\\
45.6524649028078	-0.002063\\
45.7024676025918	-0.002239\\
45.7524703023758	-0.002363\\
45.8024730021598	-0.001057\\
45.8524757019439	-0.000384\\
45.9024784017279	-0.000584\\
45.9524811015119	-0.000986\\
46.0024838012959	-0.001498\\
46.0524865010799	-0.002185\\
46.1024892008639	-0.002259\\
46.152491900648	-0.002252\\
46.202494600432	-0.001481\\
46.252497300216	-0.002611\\
46.3025	-0.003405\\
46.352502699784	-0.001891\\
46.402505399568	-0.001959\\
46.4525080993521	-0.002243\\
46.5025107991361	-0.00304\\
46.5525134989201	-0.003185\\
46.6025161987041	-0.003029\\
46.6525188984881	-0.002869\\
46.7025215982721	-0.002152\\
46.7525242980562	-0.000599\\
46.8025269978402	0.000168\\
46.8525296976242	0.000693\\
46.9025323974082	-0.000146\\
46.9525350971922	-0.00033\\
47.0025377969762	-0.000447\\
47.0525404967603	-0.000408\\
47.1025431965443	-0.000201\\
47.1525458963283	-0.00068\\
47.2025485961123	-0.000731\\
47.2525512958963	-0.000779\\
47.3025539956803	-7.9e-05\\
47.3525566954644	-0.000812\\
47.4025593952484	-0.000704\\
47.4525620950324	0.000393\\
47.5025647948164	-0.001894\\
47.5525674946004	-0.003063\\
47.6025701943845	-0.004501\\
47.6525728941685	-0.004738\\
47.7025755939525	-0.005063\\
47.7525782937365	-0.005582\\
47.8025809935205	-0.005517\\
47.8525836933045	-0.004928\\
47.9025863930886	-0.004895\\
47.9525890928726	-0.003785\\
48.0025917926566	-0.00371\\
48.0525944924406	-0.004048\\
48.1025971922246	-0.004175\\
48.1525998920086	-0.003263\\
48.2026025917927	-0.001863\\
48.2526052915767	-0.001425\\
48.3026079913607	-0.001444\\
48.3526106911447	-0.00173\\
48.4026133909287	-0.00149\\
48.4526160907127	-0.000102\\
48.5026187904968	0.000626\\
48.5526214902808	-0.000229\\
48.6026241900648	-0.000453\\
48.6526268898488	-0.00125\\
48.7026295896328	-0.001913\\
48.7526322894168	-0.001937\\
48.8026349892009	-0.002132\\
48.8526376889849	-0.001852\\
48.9026403887689	-0.000987\\
48.9526430885529	0.000231\\
49.0026457883369	0.000141\\
49.052648488121	-4.1e-05\\
49.102651187905	-0.000822\\
49.152653887689	0.000427\\
49.202656587473	0.000446\\
49.252659287257	-0.000201\\
49.302661987041	-0.000481\\
49.3526646868251	-0.001367\\
49.4026673866091	-0.001303\\
49.4526700863931	-0.000603\\
49.5026727861771	-0.000909\\
49.5526754859611	-0.001499\\
49.6026781857451	-0.00151\\
49.6526808855292	-0.001485\\
49.7026835853132	-0.000807\\
49.7526862850972	-0.001407\\
49.8026889848812	-0.001112\\
49.8526916846652	-0.001677\\
49.9026943844492	-0.001922\\
49.9526970842333	-0.001382\\
50.0026997840173	8.8e-05\\
50.0527024838013	0.001873\\
50.1027051835853	0.001002\\
50.1527078833693	0.001078\\
50.2027105831534	0.000939\\
50.2527132829374	0.000328\\
50.3027159827214	-0.000237\\
50.3527186825054	-0.000787\\
50.4027213822894	-0.00029\\
50.4527240820734	-0.000632\\
50.5027267818575	-0.000151\\
50.5527294816415	0.000624\\
50.6027321814255	0.000355\\
50.6527348812095	5.1e-05\\
50.7027375809935	-4.1e-05\\
50.7527402807775	-0.000506\\
50.8027429805616	-0.001314\\
50.8527456803456	0.000359\\
50.9027483801296	-0.00096\\
50.9527510799136	-0.001901\\
51.0027537796976	-0.002947\\
51.0527564794816	-0.002002\\
51.1027591792657	-0.00142\\
51.1527618790497	-0.001424\\
51.2027645788337	-0.001347\\
51.2527672786177	0.000382\\
51.3027699784017	-0.000836\\
51.3527726781858	-0.002007\\
51.4027753779698	-0.003193\\
51.4527780777538	-0.002022\\
51.5027807775378	-0.001807\\
51.5527834773218	-0.001439\\
51.6027861771058	-0.002094\\
51.6527888768899	-0.000734\\
51.7027915766739	-0.001545\\
51.7527942764579	-0.001464\\
51.8027969762419	-0.001235\\
51.8527996760259	-0.001228\\
51.9028023758099	-0.001339\\
51.952805075594	-0.001694\\
52.002807775378	-0.002643\\
52.052810475162	-0.002978\\
52.102813174946	-0.003149\\
52.15281587473	-0.003454\\
52.202818574514	-0.003025\\
52.2528212742981	-0.00172\\
52.3028239740821	-0.001405\\
52.3528266738661	-0.001853\\
52.4028293736501	-0.002155\\
52.4528320734341	-0.003213\\
52.5028347732181	-0.002219\\
52.5528374730022	-0.00234\\
52.6028401727862	-0.002304\\
52.6528428725702	-0.001948\\
52.7028455723542	-0.001393\\
52.7528482721382	-0.000995\\
52.8028509719222	-0.000534\\
52.8528536717063	0.000495\\
52.9028563714903	0.000636\\
52.9528590712743	-0.000811\\
53.0028617710583	-0.000674\\
53.0528644708423	-0.000553\\
53.1028671706264	-0.001695\\
53.1528698704104	-0.002949\\
53.2028725701944	-0.002658\\
53.2528752699784	-0.000867\\
53.3028779697624	-0.001431\\
53.3528806695464	-0.002689\\
53.4028833693305	-0.003605\\
53.4528860691145	-0.003788\\
53.5028887688985	-0.003629\\
53.5528914686825	-0.003284\\
53.6028941684665	-0.002468\\
53.6528968682505	-0.002477\\
53.7028995680346	-0.001863\\
53.7529022678186	0.00067\\
53.8029049676026	0.001978\\
53.8529076673866	0.001269\\
53.9029103671706	-0.000812\\
53.9529130669547	-0.001214\\
54.0029157667387	-5.1e-05\\
54.0529184665227	0.000514\\
54.1029211663067	0.001192\\
54.1529238660907	0.002882\\
54.2029265658747	0.003517\\
54.2529292656588	0.001256\\
54.3029319654428	-0.001867\\
54.3529346652268	-0.001594\\
54.4029373650108	0.000661\\
54.4529400647948	-8e-05\\
54.5029427645788	-0.002743\\
54.5529454643629	-0.003544\\
54.6029481641469	-0.004488\\
54.6529508639309	-0.003031\\
54.7029535637149	-0.000565\\
54.7529562634989	-0.000271\\
54.8029589632829	-0.002065\\
54.852961663067	-0.003111\\
54.902964362851	-0.002274\\
54.952967062635	-0.001285\\
55.002969762419	-0.000808\\
55.052972462203	-0.000772\\
55.102975161987	-0.001004\\
55.1529778617711	-0.003166\\
55.2029805615551	-0.004529\\
55.2529832613391	-0.003937\\
55.3029859611231	-0.002646\\
55.3529886609071	-0.003674\\
55.4029913606911	-0.002948\\
55.4529940604752	-0.001488\\
55.5029967602592	-0.000413\\
55.5529994600432	-0.000574\\
55.6030021598272	-0.001775\\
55.6530048596112	-0.002\\
55.7030075593953	-0.002148\\
55.7530102591793	-0.001529\\
55.8030129589633	-0.00099\\
55.8530156587473	-0.001782\\
55.9030183585313	-0.002284\\
55.9530210583153	-0.002648\\
56.0030237580994	-0.002479\\
56.0530264578834	-0.001724\\
56.1030291576674	0.000197\\
56.1530318574514	0.000194\\
56.2030345572354	0.000278\\
56.2530372570194	-0.000147\\
56.3030399568035	-0.000184\\
56.3530426565875	0.00076\\
56.4030453563715	0.000574\\
56.4530480561555	0.000281\\
56.5030507559395	-0.000198\\
56.5530534557235	-0.000185\\
56.6030561555076	0.001157\\
56.6530588552916	0.001644\\
56.7030615550756	0.000788\\
56.7530642548596	0.000483\\
56.8030669546436	0.000289\\
56.8530696544276	0.000377\\
56.9030723542117	0.000883\\
56.9530750539957	0.000512\\
57.0030777537797	-0.000376\\
57.0530804535637	-0.001051\\
57.1030831533477	-0.001333\\
57.1530858531318	-0.000678\\
57.2030885529158	0.000921\\
57.2530912526998	0.00054\\
57.3030939524838	-0.00012\\
57.3530966522678	0.000372\\
57.4030993520518	7.6e-05\\
57.4531020518358	-7.2e-05\\
57.5031047516199	0.000286\\
57.5531074514039	-0.000418\\
57.6031101511879	-0.000726\\
57.6531128509719	-0.000283\\
57.7031155507559	-0.001326\\
57.75311825054	-0.002458\\
57.803120950324	-0.002663\\
57.853123650108	-0.002176\\
57.903126349892	-0.001027\\
57.953129049676	-0.001172\\
58.00313174946	0.000723\\
58.0531344492441	-0.00106\\
58.1031371490281	-0.002301\\
58.1531398488121	-0.002903\\
58.2031425485961	-0.002291\\
58.2531452483801	-0.000757\\
58.3031479481642	-0.000762\\
58.3531506479482	0.000244\\
58.4031533477322	-0.000239\\
58.4531560475162	0.001046\\
58.5031587473002	0.000343\\
58.5531614470842	-0.0001\\
58.6031641468683	-0.00076\\
58.6531668466523	-0.001297\\
58.7031695464363	-2.3e-05\\
58.7531722462203	0.000405\\
58.8031749460043	0.000761\\
58.8531776457883	-0.000399\\
58.9031803455724	-0.000771\\
58.9531830453564	-0.002974\\
59.0031857451404	-0.004057\\
59.0531884449244	-0.004759\\
59.1031911447084	-0.005312\\
59.1531938444924	-0.005328\\
59.2031965442765	-0.005358\\
59.2531992440605	-0.004697\\
59.3032019438445	-0.004715\\
59.3532046436285	-0.0044\\
59.4032073434125	-0.002977\\
59.4532100431965	-0.00188\\
59.5032127429806	-0.001484\\
59.5532154427646	-0.001\\
59.6032181425486	-0.001507\\
59.6532208423326	-0.001765\\
59.7032235421166	-0.000578\\
59.7532262419006	-0.000879\\
59.8032289416847	-1.7e-05\\
59.8532316414687	0.000717\\
59.9032343412527	0.000141\\
59.9532370410367	-0.000806\\
60.0032397408207	-0.002428\\
60.0532424406048	-0.002233\\
60.1032451403888	-0.002036\\
60.1532478401728	-0.001703\\
60.2032505399568	-0.001437\\
60.2532532397408	-0.001439\\
60.3032559395248	-0.000312\\
60.3532586393089	0.000129\\
60.4032613390929	-0.000573\\
60.4532640388769	0.000139\\
60.5032667386609	-0.001185\\
60.5532694384449	-0.002569\\
60.6032721382289	-0.003136\\
60.653274838013	-0.002828\\
60.703277537797	-0.003\\
60.753280237581	-0.003516\\
60.803282937365	-0.002424\\
60.853285637149	-0.001302\\
60.9032883369331	-0.001825\\
60.9532910367171	-0.002181\\
61.0032937365011	-0.001629\\
61.0532964362851	-0.001031\\
61.1032991360691	0.000179\\
61.1533018358531	0.000829\\
61.2033045356371	0.000852\\
61.2533072354212	3.6e-05\\
61.3033099352052	-0.000191\\
61.3533126349892	-0.000515\\
61.4033153347732	-0.000245\\
61.4533180345572	0.000949\\
61.5033207343413	0.000657\\
61.5533234341253	0.000554\\
61.6033261339093	0.000466\\
61.6533288336933	0.000853\\
61.7033315334773	0.000969\\
61.7533342332613	0.001278\\
61.8033369330454	0.001363\\
61.8533396328294	0.000617\\
61.9033423326134	0.000165\\
61.9533450323974	-0.000305\\
62.0033477321814	-0.000307\\
62.0533504319654	0.000556\\
62.1033531317495	0.000563\\
62.1533558315335	0.000515\\
62.2033585313175	0.000779\\
62.2533612311015	-1.7e-05\\
62.3033639308855	0.002252\\
62.3533666306695	0.002984\\
62.4033693304536	0.001301\\
62.4533720302376	-0.000137\\
62.5033747300216	6.5e-05\\
62.5533774298056	0.00034\\
62.6033801295896	0.002642\\
62.6533828293737	0.014417\\
62.6983852591793	0.036296\\
62.7433876889849	0.066141\\
62.7883901187905	0.09605\\
62.8333925485961	0.121222\\
62.8833952483801	0.141309\\
62.9333979481641	0.151022\\
62.9834006479482	0.155142\\
63.0334033477322	0.159414\\
63.0834060475162	0.166637\\
63.1334087473002	0.173664\\
63.1834114470842	0.178592\\
63.2334141468683	0.18183\\
63.2834168466523	0.184875\\
63.3334195464363	0.18768\\
63.3834222462203	0.190456\\
63.4334249460043	0.193105\\
63.4834276457883	0.195156\\
63.5334303455724	0.196708\\
63.5834330453564	0.197873\\
63.6334357451404	0.199034\\
63.6834384449244	0.19983\\
63.7334411447084	0.200343\\
63.7834438444924	0.200503\\
63.8334465442765	0.201718\\
63.8834492440605	0.202574\\
63.9334519438445	0.202726\\
63.9834546436285	0.202862\\
64.0334573434125	0.20272\\
64.0834600431965	0.203007\\
64.1334627429806	0.203288\\
64.1834654427646	0.203454\\
64.2334681425486	0.203445\\
64.2834708423326	0.203459\\
64.3334735421166	0.203516\\
64.3834762419007	0.202874\\
64.4334789416847	0.202032\\
64.4834816414687	0.201554\\
64.5334843412527	0.201426\\
64.5834870410367	0.201694\\
64.6334897408207	0.201893\\
64.6834924406048	0.202098\\
64.7334951403888	0.202161\\
64.7834978401728	0.201787\\
64.8335005399568	0.201817\\
64.8835032397408	0.201642\\
64.9335059395248	0.201283\\
64.9835086393089	0.200854\\
65.0335113390929	0.200952\\
65.0835140388769	0.201085\\
65.1335167386609	0.201288\\
65.1835194384449	0.201408\\
65.2335221382289	0.201285\\
65.283524838013	0.201574\\
65.333527537797	0.201591\\
65.383530237581	0.201806\\
65.433532937365	0.201915\\
65.483535637149	0.202041\\
65.533538336933	0.202123\\
65.5835410367171	0.20231\\
65.6335437365011	0.202297\\
65.6835464362851	0.202037\\
65.7335491360691	0.201688\\
65.7835518358531	0.201264\\
65.8335545356372	0.200999\\
65.8835572354212	0.200845\\
65.9335599352052	0.200777\\
65.9835626349892	0.200565\\
66.0335653347732	0.200237\\
66.0835680345572	0.200148\\
66.1335707343412	0.200118\\
66.1835734341253	0.200189\\
66.2335761339093	0.200502\\
66.2835788336933	0.200789\\
66.3335815334773	0.200927\\
66.3835842332613	0.200981\\
66.4335869330454	0.201087\\
66.4835896328294	0.201103\\
66.5335923326134	0.201112\\
66.5835950323974	0.200848\\
66.6335977321814	0.200899\\
66.6836004319654	0.201008\\
66.7336031317495	0.201146\\
66.7836058315335	0.201239\\
66.8336085313175	0.201274\\
66.8836112311015	0.201318\\
66.9336139308855	0.201404\\
66.9836166306695	0.201522\\
67.0336193304536	0.201643\\
67.0836220302376	0.201592\\
67.1336247300216	0.20131\\
67.1836274298056	0.201433\\
67.2336301295896	0.201725\\
67.2836328293737	0.201901\\
67.3336355291577	0.202029\\
67.3836382289417	0.201844\\
67.4336409287257	0.201894\\
67.4836436285097	0.201917\\
67.5336463282937	0.201727\\
67.5836490280778	0.201856\\
67.6336517278618	0.201946\\
67.6836544276458	0.202175\\
67.7336571274298	0.202373\\
67.7836598272138	0.202331\\
67.8336625269979	0.202191\\
67.8836652267819	0.202154\\
67.9336679265659	0.202235\\
67.9836706263499	0.202324\\
68.0336733261339	0.20307\\
68.0836760259179	0.203558\\
68.1336787257019	0.2039\\
68.183681425486	0.204087\\
68.23368412527	0.204074\\
68.283686825054	0.203999\\
68.333689524838	0.203588\\
68.383692224622	0.202999\\
68.4336949244061	0.202441\\
68.4836976241901	0.202203\\
68.5337003239741	0.202151\\
68.5837030237581	0.201865\\
68.6337057235421	0.201796\\
68.6837084233261	0.201512\\
68.7337111231102	0.201313\\
68.7837138228942	0.201374\\
68.8337165226782	0.201687\\
68.8837192224622	0.201784\\
68.9337219222462	0.201976\\
68.9837246220302	0.202052\\
69.0337273218143	0.2021\\
69.0837300215983	0.202204\\
69.1337327213823	0.202518\\
69.1837354211663	0.202701\\
69.2337381209503	0.202252\\
69.2837408207344	0.202488\\
69.3337435205184	0.203084\\
69.3837462203024	0.203379\\
69.4337489200864	0.203494\\
69.4837516198704	0.203539\\
69.5337543196544	0.203714\\
69.5837570194385	0.2032\\
69.6337597192225	0.202855\\
69.6837624190065	0.202469\\
69.7337651187905	0.202526\\
69.7837678185745	0.202685\\
69.8337705183585	0.20253\\
69.8837732181426	0.202437\\
69.9337759179266	0.202234\\
69.9837786177106	0.202711\\
70.0337813174946	0.203225\\
70.0837840172786	0.203595\\
70.1337867170626	0.203766\\
70.1837894168467	0.20321\\
70.2337921166307	0.202877\\
70.2837948164147	0.202761\\
70.3337975161987	0.202732\\
70.3838002159827	0.202549\\
70.4338029157667	0.202445\\
70.4838056155508	0.202154\\
70.5338083153348	0.202153\\
70.5838110151188	0.202065\\
70.6338137149028	0.202323\\
70.6838164146868	0.2023\\
70.7338191144708	0.202156\\
70.7838218142549	0.202008\\
70.8338245140389	0.201922\\
70.8838272138229	0.201973\\
70.9338299136069	0.202038\\
70.9838326133909	0.20207\\
71.0338353131749	0.202123\\
71.083838012959	0.201951\\
71.133840712743	0.202631\\
71.183843412527	0.203185\\
71.233846112311	0.202786\\
71.283848812095	0.203137\\
71.3338515118791	0.203348\\
71.3838542116631	0.203153\\
71.4338569114471	0.202946\\
71.4838596112311	0.202923\\
71.5338623110151	0.203298\\
71.5838650107991	0.203595\\
71.6338677105831	0.203618\\
71.6838704103672	0.203693\\
71.7338731101512	0.203516\\
71.7838758099352	0.20359\\
71.8338785097192	0.20356\\
71.8838812095032	0.203032\\
71.9338839092873	0.202657\\
71.9838866090713	0.202482\\
72.0338893088553	0.203069\\
72.0838920086393	0.203427\\
72.1338947084233	0.203588\\
72.1838974082073	0.20369\\
72.2339001079914	0.203769\\
72.2839028077754	0.203861\\
72.3339055075594	0.203207\\
72.3839082073434	0.202745\\
72.4339109071274	0.202324\\
72.4839136069115	0.20239\\
72.5339163066955	0.203073\\
72.5839190064795	0.203404\\
72.6339217062635	0.20338\\
72.6839244060475	0.203528\\
72.7339271058315	0.203598\\
72.7839298056156	0.203513\\
72.8339325053996	0.203075\\
72.8839352051836	0.202633\\
72.9339379049676	0.20241\\
72.9839406047516	0.202839\\
73.0339433045356	0.203345\\
73.0839460043197	0.203567\\
73.1339487041037	0.203682\\
73.1839514038877	0.203158\\
73.2339541036717	0.202369\\
73.2839568034557	0.202201\\
73.3339595032398	0.202239\\
73.3839622030238	0.202887\\
73.4339649028078	0.203255\\
73.4839676025918	0.203435\\
73.5339703023758	0.203239\\
73.5839730021598	0.20263\\
73.6339757019438	0.202221\\
73.6839784017279	0.202298\\
73.7339811015119	0.201874\\
73.7839838012959	0.202039\\
73.8339865010799	0.201898\\
73.8839892008639	0.201797\\
73.933991900648	0.201809\\
73.983994600432	0.201514\\
74.033997300216	0.201534\\
74.084	0.201587\\
74.134002699784	0.201616\\
74.184005399568	0.201402\\
74.2340080993521	0.201831\\
74.2840107991361	0.202178\\
74.3340134989201	0.202252\\
74.3840161987041	0.202331\\
74.4340188984881	0.202158\\
74.4840215982721	0.20227\\
74.5340242980562	0.20234\\
74.5840269978402	0.20268\\
74.6340296976242	0.203319\\
74.6840323974082	0.203421\\
74.7340350971922	0.20373\\
74.7840377969763	0.203927\\
74.8340404967603	0.203381\\
74.8840431965443	0.202858\\
74.9340458963283	0.202607\\
74.9840485961123	0.202408\\
75.0340512958963	0.201879\\
75.0840539956804	0.202421\\
75.1340566954644	0.202154\\
75.1840593952484	0.202187\\
75.2340620950324	0.201902\\
75.2840647948164	0.201807\\
75.3340674946004	0.201822\\
75.3840701943844	0.201893\\
75.4340728941685	0.201953\\
75.4840755939525	0.201979\\
75.5340782937365	0.20215\\
75.5840809935205	0.201999\\
75.6340836933045	0.201869\\
75.6840863930885	0.201785\\
75.7340890928726	0.201688\\
75.7840917926566	0.20164\\
75.8340944924406	0.201842\\
75.8840971922246	0.202001\\
75.9340998920086	0.20186\\
75.9841025917927	0.201815\\
76.0341052915767	0.201663\\
76.0841079913607	0.202289\\
76.1341106911447	0.202301\\
76.1841133909287	0.20228\\
76.2341160907127	0.201849\\
76.2841187904968	0.20173\\
76.3341214902808	0.20174\\
76.3841241900648	0.201855\\
76.4341268898488	0.201885\\
76.4841295896328	0.202387\\
76.5341322894169	0.202987\\
76.5841349892009	0.203171\\
76.6341376889849	0.202951\\
76.6841403887689	0.202687\\
76.7341430885529	0.202423\\
76.7841457883369	0.201811\\
76.834148488121	0.201889\\
76.884151187905	0.202296\\
76.934153887689	0.203021\\
76.984156587473	0.203057\\
77.034159287257	0.202906\\
77.084161987041	0.202708\\
77.1341646868251	0.202439\\
77.1841673866091	0.202025\\
77.2341700863931	0.201916\\
77.2841727861771	0.20203\\
77.3341754859611	0.202516\\
77.3841781857451	0.203135\\
77.4341808855292	0.20343\\
77.4841835853132	0.203513\\
77.5341862850972	0.202987\\
77.5841889848812	0.202095\\
77.6341916846652	0.201628\\
77.6841943844492	0.201458\\
77.7341970842333	0.200586\\
77.7841997840173	0.201127\\
77.8342024838013	0.202021\\
77.8842051835853	0.202771\\
77.9342078833693	0.203168\\
77.9842105831534	0.203315\\
78.0342132829374	0.202659\\
78.0842159827214	0.202179\\
78.1342186825054	0.201603\\
78.1842213822894	0.201541\\
78.2342240820734	0.201471\\
78.2842267818575	0.201387\\
78.3342294816415	0.201095\\
78.3842321814255	0.201105\\
78.4342348812095	0.200949\\
78.4842375809935	0.201073\\
78.5342402807775	0.201024\\
78.5842429805616	0.200975\\
78.6342456803456	0.201445\\
78.6842483801296	0.201945\\
78.7342510799136	0.201516\\
78.7842537796976	0.201501\\
78.8342564794817	0.201463\\
78.8842591792657	0.201274\\
78.9342618790497	0.201107\\
78.9842645788337	0.20094\\
79.0342672786177	0.2008\\
79.0842699784017	0.200691\\
79.1342726781857	0.200386\\
79.1842753779698	0.200392\\
79.2342780777538	0.200434\\
79.2842807775378	0.200567\\
79.3342834773218	0.20061\\
79.3842861771058	0.200759\\
79.4342888768899	0.200873\\
79.4842915766739	0.200876\\
79.5342942764579	0.200423\\
79.5842969762419	0.200559\\
79.6342996760259	0.200762\\
79.6843023758099	0.200911\\
79.734305075594	0.200966\\
79.784307775378	0.200924\\
79.834310475162	0.200608\\
79.884313174946	0.20082\\
79.93431587473	0.200875\\
79.984318574514	0.200685\\
80.0343212742981	0.200811\\
80.0843239740821	0.200996\\
80.1343266738661	0.201148\\
80.1843293736501	0.200803\\
80.2343320734341	0.201082\\
80.2843347732181	0.20139\\
80.3343374730022	0.20161\\
80.3843401727862	0.201675\\
80.4343428725702	0.201667\\
80.4843455723542	0.201527\\
80.5343482721382	0.201224\\
80.5843509719223	0.201313\\
80.6343536717063	0.20135\\
80.6843563714903	0.201304\\
80.7343590712743	0.201125\\
80.7843617710583	0.20136\\
80.8343644708423	0.20147\\
80.8843671706263	0.201202\\
80.9343698704104	0.201256\\
80.9843725701944	0.201153\\
81.0343752699784	0.201345\\
81.0843779697624	0.201435\\
81.1343806695464	0.20121\\
81.1843833693305	0.201289\\
81.2343860691145	0.201418\\
81.2843887688985	0.201394\\
81.3343914686825	0.201603\\
81.3843941684665	0.20174\\
81.4343968682505	0.201848\\
81.4843995680346	0.201917\\
81.5344022678186	0.201957\\
81.5844049676026	0.201923\\
81.6344076673866	0.201776\\
81.6844103671706	0.201694\\
81.7344130669546	0.20143\\
81.7844157667387	0.201274\\
81.8344184665227	0.201379\\
81.8844211663067	0.201832\\
81.9344238660907	0.202128\\
81.9844265658747	0.201958\\
82.0344292656588	0.201554\\
82.0844319654428	0.201199\\
82.1344346652268	0.200967\\
82.1844373650108	0.201212\\
82.2344400647948	0.20136\\
82.2844427645788	0.201626\\
82.3344454643629	0.201692\\
82.3844481641469	0.20138\\
82.4344508639309	0.201146\\
82.4844535637149	0.201043\\
82.5344562634989	0.200973\\
82.5844589632829	0.20116\\
82.634461663067	0.200959\\
82.684464362851	0.199022\\
82.734467062635	0.189471\\
82.784469762419	0.168776\\
82.8294721922246	0.139521\\
82.8744746220302	0.108195\\
82.9194770518359	0.083092\\
82.9694797516199	0.06655\\
83.0194824514039	0.057887\\
83.0694851511879	0.050696\\
83.1194878509719	0.041472\\
83.1694905507559	0.033\\
83.21949325054	0.026332\\
83.269495950324	0.022057\\
83.319498650108	0.018593\\
83.369501349892	0.014812\\
83.419504049676	0.01088\\
83.4695067494601	0.008768\\
83.5195094492441	0.008386\\
83.5695121490281	0.007319\\
83.6195148488121	0.005662\\
83.6695175485961	0.003801\\
83.7195202483801	0.003023\\
83.7695229481642	0.002741\\
83.8195256479482	0.002743\\
83.8695283477322	0.001931\\
83.9195310475162	0.001214\\
83.9695337473002	0.000596\\
84.0195364470842	0.00024\\
84.0695391468683	0.001009\\
84.1195418466523	0.001001\\
84.1695445464363	-0.000389\\
84.2195472462203	-0.002576\\
84.2695499460043	-0.002941\\
84.3195526457883	-0.001291\\
84.3695553455724	-0.000597\\
84.4195580453564	-0.000677\\
84.4695607451404	-0.001232\\
84.5195634449244	-0.001129\\
84.5695661447084	-0.000848\\
84.6195688444924	-0.000569\\
84.6695715442765	-0.00054\\
84.7195742440605	0.000295\\
84.7695769438445	0.000536\\
84.8195796436285	0.000497\\
84.8695823434125	0.000329\\
84.9195850431965	0.000358\\
84.9695877429806	0.001734\\
85.0195904427646	0.001105\\
85.0695931425486	0.000672\\
85.1195958423326	-0.000331\\
85.1695985421166	0.000362\\
85.2196012419006	0.000912\\
85.2696039416847	0.002235\\
85.3196066414687	0.002055\\
85.3696093412527	0.001953\\
85.4196120410367	0.002127\\
85.4696147408207	0.001322\\
85.5196174406048	0.000747\\
85.5696201403888	0.001082\\
85.6196228401728	0.000829\\
85.6696255399568	0.001906\\
85.7196282397408	0.001978\\
85.7696309395248	0.001035\\
85.8196336393089	0.000934\\
85.8696363390929	0.00222\\
85.9196390388769	0.002693\\
85.9696417386609	0.001568\\
86.0196444384449	0.000773\\
86.0696471382289	-0.000723\\
86.119649838013	-0.001303\\
86.169652537797	-0.000326\\
86.219655237581	-0.000527\\
86.269657937365	-0.000521\\
86.319660637149	-0.000839\\
86.369663336933	-0.001169\\
86.4196660367171	-0.001302\\
86.4696687365011	-0.000596\\
86.5196714362851	-0.000136\\
86.5696741360691	0.001274\\
86.6196768358531	0.001077\\
86.6696795356372	0.000311\\
86.7196822354212	0.000236\\
86.7696849352052	0.000744\\
86.8196876349892	0.000964\\
86.8696903347732	0.001639\\
86.9196930345572	0.001317\\
86.9696957343413	9.5e-05\\
87.0196984341253	-0.000386\\
87.0697011339093	0.000913\\
87.1197038336933	0.002098\\
87.1697065334773	0.001729\\
87.2197092332613	0.000379\\
87.2697119330454	-0.000594\\
87.3197146328294	-0.000562\\
87.3697173326134	-0.000397\\
87.4197200323974	-0.000693\\
87.4697227321814	-6.9e-05\\
87.5197254319655	-0.000607\\
87.5697281317495	-0.000568\\
87.6197308315335	0.001747\\
87.6697335313175	0.001617\\
87.7197362311015	0.000508\\
87.7697389308855	0.000241\\
87.8197416306695	0.00023\\
87.8697443304536	0.0005\\
87.9197470302376	-0.000248\\
87.9697497300216	-0.000481\\
88.0197524298056	-0.000898\\
88.0697551295896	-0.00091\\
88.1197578293737	-0.001023\\
88.1697605291577	-0.001069\\
88.2197632289417	-0.000959\\
88.2697659287257	-0.000366\\
88.3197686285097	0.000267\\
88.3697713282937	0.001655\\
88.4197740280778	0.001731\\
88.4697767278618	0.003801\\
88.5197794276458	0.00382\\
88.5697821274298	0.002682\\
88.6197848272138	0.000945\\
88.6697875269979	0.000375\\
88.7197902267819	4.5e-05\\
88.7697929265659	0.001715\\
88.8197956263499	0.00197\\
88.8697983261339	0.001677\\
88.9198010259179	0.001338\\
88.969803725702	0.000488\\
89.019806425486	0.0009\\
89.06980912527	0.000766\\
89.119811825054	0.000675\\
89.169814524838	0.001351\\
89.219817224622	0.002006\\
89.2698199244061	0.002146\\
89.3198226241901	0.002362\\
89.3698253239741	0.002307\\
89.4198280237581	0.001465\\
89.4698307235421	0.00156\\
89.5198334233261	0.001772\\
89.5698361231102	0.001424\\
89.6198388228942	0.001852\\
89.6698415226782	0.001046\\
89.7198442224622	0.00183\\
89.7698469222462	0.001939\\
89.8198496220302	0.002001\\
89.8698523218143	0.00098\\
89.9198550215983	-5.9e-05\\
89.9698577213823	-0.000762\\
90.0198604211663	8.2e-05\\
90.0698631209503	0.00051\\
90.1198658207343	-0.000239\\
90.1698685205184	-0.00082\\
90.2198712203024	-0.00081\\
90.2698739200864	0.000115\\
90.3198766198704	0.003246\\
90.3698793196544	0.00488\\
90.4198820194384	0.005064\\
90.4698847192225	0.002435\\
90.5198874190065	0.00157\\
90.5698901187905	0.000199\\
90.6198928185745	-7e-06\\
90.6698955183585	-0.000211\\
90.7198982181426	0.000249\\
90.7699009179266	-0.000424\\
90.8199036177106	-0.000941\\
90.8699063174946	-0.001261\\
90.9199090172786	-0.000779\\
90.9699117170626	-0.002059\\
91.0199144168467	-0.002635\\
91.0699171166307	-0.001968\\
91.1199198164147	-0.000395\\
91.1699225161987	0.000681\\
91.2199252159827	0.000896\\
91.2699279157667	0.001861\\
91.3199306155508	0.001365\\
91.3699333153348	0.00051\\
91.4199360151188	0.000113\\
91.4699387149028	0.000488\\
91.5199414146868	0.000654\\
91.5699441144708	0.001186\\
91.6199468142549	0.000824\\
91.6699495140389	0.001013\\
91.7199522138229	0.00069\\
91.7699549136069	0.002648\\
91.8199576133909	0.002392\\
91.8699603131749	0.002036\\
91.919963012959	0.001627\\
91.969965712743	0.001747\\
92.019968412527	0.002095\\
92.069971112311	0.001551\\
92.119973812095	0.002309\\
92.1699765118791	0.002372\\
92.2199792116631	0.001649\\
92.2699819114471	0.002331\\
92.3199846112311	0.001723\\
92.3699873110151	0.001579\\
92.4199900107991	0.002541\\
92.4699927105832	0.002829\\
92.5199954103672	0.001885\\
92.5699981101512	0.003252\\
92.605	0.003252\\
};
\addlegendentry{Estimated position [m]};

\addplot [color=blue,solid,line width=0.2pt]
  table[row sep=crcr]{0	-0.0749067719101089\\
0.0500026997840173	-0.0756575527496417\\
0.100005399568035	-0.0755463917366384\\
0.150008099352052	-0.0754819939799291\\
0.200010799136069	-0.0754424823058127\\
0.250013498920086	-0.0755106348605658\\
0.300016198704104	-0.0755070659702472\\
0.350018898488121	-0.0754657842409125\\
0.400021598272138	-0.0753940467724746\\
0.450024298056156	-0.0751573862684106\\
0.500026997840173	-0.0753222664025992\\
0.55002969762419	-0.075684518712868\\
0.600032397408207	-0.0756881669400238\\
0.650035097192225	-0.0755985775994107\\
0.700037796976242	-0.07560929083947\\
0.750040496760259	-0.0754711685200185\\
0.800043196544276	-0.0755896288163063\\
0.850045896328294	-0.0755178348716048\\
0.900048596112311	-0.0754425418342413\\
0.950051295896328	-0.0753779265865785\\
1.00005399568035	-0.0755356821504329\\
1.05005669546436	-0.0752613055781661\\
1.10005939524838	-0.075397633537239\\
1.1500620950324	-0.0757400724448426\\
1.20006479481641	-0.0756577405276306\\
1.25006749460043	-0.0755986035211804\\
1.30007019438445	-0.0754801426142594\\
1.35007289416847	-0.075196859228696\\
1.40007559395248	-0.0751913834723345\\
1.4500782937365	-0.075324119572121\\
1.50008099352052	-0.07524167792677\\
1.55008369330454	-0.0756630542406475\\
1.60008639308855	-0.0758100507059259\\
1.65008909287257	-0.0757096105790356\\
1.70009179265659	-0.0755895975825141\\
1.7500944924406	-0.0755322125550733\\
1.80009719222462	-0.0755231490336972\\
1.85009989200864	-0.0753617467508887\\
1.90010259179266	-0.0752578490561568\\
1.95010529157667	-0.0757687928378589\\
2.00010799136069	-0.0755965601809048\\
2.05011069114471	-0.0762455577957876\\
2.10011339092873	-0.0726628665586936\\
2.15011609071274	-0.0653693271160434\\
2.20011879049676	-0.0544194462585877\\
2.25012149028078	-0.0428974159908262\\
2.30012419006479	-0.0345433641546035\\
2.35012688984881	-0.0307157467317714\\
2.40012958963283	-0.0281870673800009\\
2.45013228941685	-0.0249247063455462\\
2.50013498920086	-0.0209469889515255\\
2.55013768898488	-0.0177849845374986\\
2.6001403887689	-0.0150774642480426\\
2.65014308855292	-0.0137183512275043\\
2.70014578833693	-0.0131682526312357\\
2.75014848812095	-0.0117226205947694\\
2.80015118790497	-0.0103904380092325\\
2.85015388768899	-0.00922714362317916\\
2.900156587473	-0.00824569123136266\\
2.95015928725702	-0.00735734185180289\\
3.00016198704104	-0.00691141401176108\\
3.05016468682505	-0.0064314624708795\\
3.10016738660907	-0.00584883465341678\\
3.15017008639309	-0.00575186768904735\\
3.20017278617711	-0.00563135917754207\\
3.25017548596112	-0.00571034804003099\\
3.30017818574514	-0.00511706579560307\\
3.35018088552916	-0.00470346959643639\\
3.40018358531318	-0.00483495896974648\\
3.45018628509719	-0.00509385439619457\\
3.50018898488121	-0.00537774069671827\\
3.55019168466523	-0.0048240401651585\\
3.60019438444924	-0.00405979815827595\\
3.65019708423326	-0.00367144993652071\\
3.70019978401728	-0.00355107045091932\\
3.7502024838013	-0.00313041447951463\\
3.80020518358531	-0.00247219680868018\\
3.85020788336933	-0.00202998694297108\\
3.90021058315335	-0.00139888902555756\\
3.95021328293736	-0.000708429778284784\\
4.00021598272138	-0.000312869911554552\\
4.0502186825054	-0.0005790396553982\\
4.10022138228942	-0.000836049662513396\\
4.15022408207343	-0.000981789379811676\\
4.20022678185745	-0.000246379813992724\\
4.25022948164147	-6.30099922993897e-05\\
4.30023218142549	0.000445929894498322\\
4.3502348812095	0.000174450119761246\\
4.40023758099352	0.000629349867364495\\
4.45024028077754	0.00117220930130371\\
4.50024298056155	0.00143476853460119\\
4.55024568034557	0.00150141914709917\\
4.60024838012959	0.00134848852764166\\
4.65025107991361	0.00163623835515118\\
4.70025377969762	0.00206417786312853\\
4.75025647948164	0.00248130669648662\\
4.80025917926566	0.00256577538035016\\
4.85026187904968	0.00252798535382201\\
4.90026457883369	0.00221510736971197\\
4.95026727861771	0.00188237788011579\\
5.00026997840173	0.00183204847945183\\
5.05027267818575	0.00191484830919665\\
5.10027537796976	0.00229254728696066\\
5.15027807775378	0.00278156379283156\\
5.2002807775378	0.00344854110048071\\
5.25028347732181	0.00381360960967787\\
5.30028617710583	0.00357973081268452\\
5.35028887688985	0.00345752188828196\\
5.40029157667387	0.0034863097553107\\
5.45029427645788	0.00338022158927959\\
5.5002969762419	0.00356714089445493\\
5.55029967602592	0.00337307059732386\\
5.60030237580994	0.0037505896846448\\
5.65030507559395	0.00366425036547586\\
5.70030777537797	0.00363008083642352\\
5.75031047516199	0.00352586078444241\\
5.800313174946	0.00376495747464059\\
5.85031587473002	0.00409410721593653\\
5.90031857451404	0.00397720776327857\\
5.95032127429806	0.00414442508379436\\
6.00032397408207	0.00434752436859698\\
6.05032667386609	0.00469445103656204\\
6.10032937365011	0.00470349107706353\\
6.15033207343413	0.00436377464668437\\
6.20033477321814	0.00379735889525196\\
6.25033747300216	0.00334426208278469\\
6.30034017278618	0.00301890337604044\\
6.3503428725702	0.00233376605724472\\
6.40034557235421	0.00173148828123556\\
6.45034827213823	0.00121363937391702\\
6.50035097192225	0.00113083954294815\\
6.55035367170626	0.00067608976409079\\
6.60035637149028	0.000359649799037169\\
6.6503590712743	-0.000469319777677215\\
6.70036177105832	-0.000604169807233929\\
6.75036447084233	-0.00067971958597613\\
6.80036717062635	-0.000582659069966439\\
6.85036987041037	-0.000665349433361255\\
6.90037257019439	-0.00103382926105933\\
6.9503752699784	-0.00170098892628634\\
7.00037796976242	-0.00155522873620708\\
7.05038066954644	-0.00108236901358709\\
7.10038336933045	-0.000677849513434459\\
7.15038606911447	-0.000179829947778432\\
7.20038876889849	-0.000267909928984278\\
7.25039146868251	-0.000127650031092487\\
7.30039416846652	0.000348829778529024\\
7.35039686825054	0.000639989738128178\\
7.40039956803456	0.00116678927838821\\
7.45040226781857	0.00183380797875145\\
7.50040496760259	0.00240929586277716\\
7.55040766738661	0.0026700456071208\\
7.60041036717063	0.00252429669239999\\
7.65041306695464	0.00199229810402074\\
7.70041576673866	0.00181060854596582\\
7.75041846652268	0.00188062790981115\\
7.8004211663067	0.00212343738400876\\
7.85042386609071	0.00192380731266607\\
7.90042656587473	0.00128923884119827\\
7.95042926565875	7.54799206101253e-05\\
8.00043196544276	-0.000650799634913074\\
8.05043466522678	-0.00117408888928762\\
8.1004373650108	-0.00150304833092228\\
8.15044006479482	-0.0021522171344814\\
8.20044276457883	-0.00317885347681124\\
8.25044546436285	-0.00416582620039677\\
8.30044816414687	-0.00477706030430347\\
8.35045086393089	-0.00531836128852041\\
8.4004535637149	-0.00496236401241241\\
8.45045626349892	-0.00413174634732054\\
8.50045896328294	-0.00310860412400277\\
8.55046166306696	-0.00258543608967034\\
8.60046436285097	-0.00263760475627263\\
8.65046706263499	-0.00247044629153745\\
8.70046976241901	-0.00192370799300143\\
8.75047246220302	-0.000911629516475116\\
8.80047516198704	-0.000181829699903031\\
8.85047786177106	0.000900789042198345\\
8.90048056155508	0.0019095668440927\\
8.95048326133909	0.00297379445880644\\
9.00048596112311	0.00333358285258419\\
9.05048866090713	0.00390355873692718\\
9.10049136069114	0.00434209461348836\\
9.15049406047516	0.0044105032495615\\
9.20049676025918	0.00468561960531963\\
9.2504994600432	0.00498053741970767\\
9.30050215982721	0.0053994223718181\\
9.35050485961123	0.00584706400516743\\
9.40050755939525	0.00575896506734892\\
9.45051025917927	0.00546941015793189\\
9.50051295896328	0.00544436001910002\\
9.5505156587473	0.00486555954972894\\
9.60051835853132	0.00413175614455598\\
9.65052105831533	0.00342335070781257\\
9.70052375809935	0.00345754965190532\\
9.75052645788337	0.00369661016272354\\
9.80052915766739	0.00429183598888206\\
9.8505318574514	0.0045003438742042\\
9.90053455723542	0.00469107012490006\\
9.95053725701944	0.00522869424075406\\
10.0005399568035	0.00566909859974647\\
10.0505426565875	0.00589197378142339\\
10.1005453563715	0.00578413634413553\\
10.1505480561555	0.0058291750820762\\
10.2005507559395	0.00595859153565075\\
10.2505534557235	0.00591901257978793\\
10.3005561555076	0.0056367267987096\\
10.3505588552916	0.00568357658257157\\
10.4005615550756	0.00562056899694706\\
10.4505642548596	0.00578943496028602\\
10.5005669546436	0.00605552949430342\\
10.5505696544276	0.00661300648173535\\
10.6005723542117	0.0068268715316075\\
10.6505750539957	0.00700321051556172\\
10.7005777537797	0.00739513071690921\\
10.7505804535637	0.00798303342142096\\
10.8005831533477	0.00830837154789158\\
10.8505858531317	0.00813405526012492\\
10.9005885529158	0.00801889732467205\\
10.9505912526998	0.00797936297252162\\
11.0005939524838	0.00805489127720868\\
11.0505966522678	0.00793811355732946\\
11.1005993520518	0.00787696459743851\\
11.1506020518359	0.0078374366830522\\
11.2006047516199	0.0081628350039113\\
11.2506074514039	0.00772597972670861\\
11.3006101511879	0.00742744927726425\\
11.3506128509719	0.00732139194216939\\
11.4006155507559	0.00708591878622384\\
11.45061825054	0.00698338943130212\\
11.500620950324	0.00712355846792163\\
11.550623650108	0.00656806874034679\\
11.600626349892	0.00668498784198731\\
11.650629049676	0.00605758015371286\\
11.70063174946	0.0057444448252748\\
11.7506344492441	0.00553598803698523\\
11.8006371490281	0.005395769644015\\
11.8506398488121	0.00518899538249185\\
11.9006425485961	0.0053940317341587\\
11.9506452483801	0.00557559854024851\\
12.0006479481641	0.00576629435182912\\
12.0506506479482	0.00589573367379907\\
12.1006533477322	0.00606812103310179\\
12.1506560475162	0.00593505363028023\\
12.2006587473002	0.00600351233950317\\
12.2506614470842	0.00616705837377553\\
12.3006641468683	0.00579851341459548\\
12.3506668466523	0.00545147024646954\\
12.4006695464363	0.005625879084251\\
12.4506722462203	0.00584712573548209\\
12.5006749460043	0.00574634603964122\\
12.5506776457883	0.00552335895131969\\
12.6006803455724	0.00538859074619017\\
12.6506830453564	0.00503785539425297\\
12.7006857451404	0.00503243801140823\\
12.7506884449244	0.00468189101033363\\
12.8006911447084	0.00430082299806487\\
12.8506938444924	0.00350421032530727\\
12.9006965442765	0.0024614864107951\\
12.9506992440605	0.00234460741733558\\
13.0007019438445	0.00285901489311952\\
13.0507046436285	0.00261620546990234\\
13.1007073434125	0.00268442514492828\\
13.1507100431965	0.00264306516826795\\
13.2007127429806	0.00230872720458827\\
13.2507154427646	0.0021827977121206\\
13.3007181425486	0.00243816698895168\\
13.3507208423326	0.00209289777212395\\
13.4007235421166	0.00233740635218109\\
13.4507262419006	0.00280305366232335\\
13.5007289416847	0.00277067529286448\\
13.5507316414687	0.00275458573435396\\
13.6007343412527	0.00283182547837866\\
13.6507370410367	0.0030638337666442\\
13.7007397408207	0.00323283219434706\\
13.7507424406048	0.00343057140178159\\
13.8007451403888	0.0031969534056526\\
13.8507478401728	0.00319861301619651\\
13.9007505399568	0.00300266421901345\\
13.9507532397408	0.00349168123155314\\
14.0007559395248	0.00376333873079369\\
14.0507586393089	0.00423608383689138\\
14.1007613390929	0.00472153760741455\\
14.1507640388769	0.00462442174792232\\
14.2007667386609	0.00448792325658788\\
14.2507694384449	0.0045004142193252\\
14.3007721382289	0.00475033028788265\\
14.350774838013	0.00511162397542966\\
14.400777537797	0.00503249685014477\\
14.450780237581	0.00418400617367782\\
14.500782937365	0.00336227124020106\\
14.550785637149	0.00272585462142152\\
14.600788336933	0.00223324679498001\\
14.6507910367171	0.00165782884638781\\
14.7007937365011	0.00165960857889543\\
14.7507964362851	0.000891809273292958\\
14.8007991360691	0.000503449821803863\\
14.8508018358531	-7.72598263329883e-05\\
14.9008045356372	-0.000524989729380052\\
14.9508072354212	-0.000350539489990333\\
15.0008099352052	-0.000242639592597299\\
15.0508126349892	0.000131270023620464\\
15.1008153347732	9.00999778047551e-06\\
15.1508180345572	-0.000213869846729365\\
15.2008207343413	-0.000381029714717765\\
15.2508234341253	0.000187029856140875\\
15.3008261339093	0.000985239164585737\\
15.3508288336933	0.00140957898752667\\
15.4008315334773	0.00147436905167427\\
15.4508342332613	0.00127646918384729\\
15.5008369330454	0.000679649669343693\\
15.5508396328294	-0.000183399887470766\\
15.6008423326134	-0.000359479889720392\\
15.6508450323974	-0.00032535959877671\\
15.7008477321814	-0.000523129613549287\\
15.7508504319654	-0.000783859553213259\\
15.8008531317495	-0.000789299707794974\\
15.8508558315335	-0.0010770294225889\\
15.9008585313175	-0.00144031864304545\\
15.9508612311015	-0.00185015790468893\\
16.0008639308855	-0.0020892474356563\\
16.0508666306695	-0.00203706797581211\\
16.1008693304536	-0.00175489868893561\\
16.1508720302376	-0.00142232869665297\\
16.2008747300216	-0.00133941854281566\\
16.2508774298056	-0.00148692834626524\\
16.3008801295896	-0.000890019669062207\\
16.3508828293737	-6.8370009657087e-05\\
16.4008855291577	0.00105725910850375\\
16.4508882289417	0.00168650850110484\\
16.5008909287257	0.00248124652279106\\
16.5508936285097	0.00268616607250207\\
16.6008963282937	0.00232291672702059\\
16.6508990280778	0.00225649710095432\\
16.7009017278618	0.00197054781448886\\
16.7509044276458	0.00184827845682872\\
16.8009071274298	0.00221687740795353\\
16.8509098272138	0.00213964702713628\\
16.9009125269978	0.00194718748019252\\
16.9509152267819	0.00184115847517189\\
17.0009179265659	0.00220450786234546\\
17.0509206263499	0.00254062624565287\\
17.1009233261339	0.00237705640968658\\
17.1509260259179	0.00266475514939011\\
17.2009287257019	0.00317537289540647\\
17.250931425486	0.00359416046776427\\
17.30093412527	0.00425771425306453\\
17.350936825054	0.00480415935936595\\
17.400939524838	0.00509197552660602\\
17.450942224622	0.00579136319276883\\
17.500944924406	0.0060322496742723\\
17.5509476241901	0.00591002400454603\\
17.6009503239741	0.00608798036753664\\
17.6509530237581	0.00641518345200116\\
17.7009557235421	0.00635413528889422\\
17.7509584233261	0.00613656914244406\\
17.8009611231102	0.006064571266275\\
17.8509638228942	0.00641335298189671\\
17.9009665226782	0.00658421083202115\\
17.9509692224622	0.00639913420711237\\
18.0009719222462	0.00590279374376315\\
18.0509746220302	0.00550003134815767\\
18.1009773218143	0.00530760269749266\\
18.1509800215983	0.00540107205864101\\
18.2009827213823	0.00559540596778667\\
18.2509854211663	0.00585971528192784\\
18.3009881209503	0.00592805169032435\\
18.3509908207343	0.00612030957310163\\
18.4009935205184	0.00620305797447331\\
18.4509962203024	0.00616359983378497\\
18.5009989200864	0.00570686372800143\\
18.5510016198704	0.00566191800160119\\
18.6010043196544	0.0052393436129763\\
18.6510070194384	0.00389245802818534\\
18.7010097192225	0.00317890120858389\\
18.7510124190065	0.0031464731119914\\
18.8010151187905	0.00283908459484358\\
18.8510178185745	0.00238783695361686\\
18.9010205183585	0.0026700062130619\\
18.9510232181425	0.00294156337315796\\
19.0010259179266	0.00334059319065943\\
19.0510286177106	0.00361032996086967\\
19.1010313174946	0.00394298696075862\\
19.1510340172786	0.00443748170465587\\
19.2010367170626	0.00424872462478818\\
19.2510394168467	0.00418217586376731\\
19.3010421166307	0.00448230419781984\\
19.3510448164147	0.00432765350822959\\
19.4010475161987	0.00418394609104758\\
19.4510502159827	0.0045057941798455\\
19.5010529157667	0.00486891723413525\\
19.5510556155508	0.00500374545666346\\
19.6010583153348	0.00508652731719281\\
19.6510610151188	0.00515840628204952\\
19.7010637149028	0.00532025234140701\\
19.7510664146868	0.00539558075737698\\
19.8010691144708	0.00571216671417141\\
19.8510718142549	0.0057804651641632\\
19.9010745140389	0.00538855912060207\\
19.9510772138229	0.00591895272869695\\
20.0010799136069	0.00594415387329483\\
20.0510826133909	0.00593514434938282\\
20.1010853131749	0.00569951505860091\\
20.151088012959	0.00604298666088811\\
20.201090712743	0.00586682504438395\\
20.251093412527	0.00545688235042049\\
20.301096112311	0.00517989625426309\\
20.351098812095	0.00530579272019762\\
20.4011015118791	0.00556477176752429\\
20.4511042116631	0.00552887989637731\\
20.5011069114471	0.00564928846398874\\
20.5511096112311	0.00527721051564733\\
20.6011123110151	0.00492290827400756\\
20.6511150107991	0.00480236039841602\\
20.7011177105832	0.00484381911603742\\
20.7511204103672	0.00517285921232203\\
20.8011231101512	0.00485642666793257\\
20.8511258099352	0.00454349227841524\\
20.9011285097192	0.00469630905831151\\
20.9511312095032	0.00431341548690792\\
21.0011339092873	0.00313386332241982\\
21.0511366090713	0.00230503700012473\\
21.1011393088553	0.00199932797195455\\
21.1511420086393	0.00252246684826434\\
21.2011447084233	0.00280296510736208\\
21.2511474082073	0.0020533269712745\\
21.3011501079914	0.00190418829130572\\
21.3511528077754	0.00165949875233573\\
21.4011555075594	0.00147428921674325\\
21.4511582073434	0.00148868884403736\\
21.5011609071274	0.00106613957271884\\
21.5511636069114	0.000719289688580044\\
21.6011663066955	0.000605909772445469\\
21.6511690064795	0.000185270072122915\\
21.7011717062635	-4.49099890778777e-05\\
21.7511744060475	-0.000634779446769759\\
21.8011771058315	-0.00176023783081997\\
21.8511798056156	-0.00293267437337592\\
21.9011825053996	-0.00377221723904167\\
21.9511852051836	-0.00392322776549649\\
22.0011879049676	-0.00446439887302389\\
22.0511906047516	-0.00562047359753456\\
22.1011933045356	-0.00668668864518672\\
22.1511960043197	-0.00756234226152629\\
22.2011987041037	-0.00842347693920958\\
22.2512014038877	-0.00948788368421323\\
22.3012041036717	-0.00959768103822648\\
22.3512068034557	-0.00950410401824841\\
22.4012095032397	-0.00864279957339023\\
22.4512122030238	-0.00777082862433696\\
22.5012149028078	-0.00694188627691208\\
22.5512176025918	-0.00667058764974037\\
22.6012203023758	-0.00594237331347719\\
22.6512230021598	-0.00569424725881398\\
22.7012257019438	-0.0053166225840675\\
22.7512284017279	-0.00435273282540516\\
22.8012311015119	-0.00330650142214388\\
22.8512338012959	-0.00249736673089154\\
22.9012365010799	-0.00231208694931201\\
22.9512392008639	-0.00219708765909093\\
23.0012419006479	-0.0017601987147139\\
23.051244600432	-0.00180878676022762\\
23.101247300216	-0.00138446879408728\\
23.15125	-0.000683279811616574\\
23.201252699784	-0.000426209596171138\\
23.251255399568	-0.000922379206049305\\
23.3012580993521	-0.000904539643661433\\
23.3512607991361	-0.000703059733203253\\
23.4012634989201	0.000291279830132503\\
23.4512661987041	0.000293009994653618\\
23.5012688984881	0.000183499837065832\\
23.5512715982721	0.000199669869154975\\
23.6012742980562	0.000330829907350269\\
23.6512769978402	0.000877299291191873\\
23.7012796976242	0.000882659655705361\\
23.7512823974082	0.000339829715217881\\
23.8012850971922	0.000323579872628062\\
23.8512877969762	0.000560989592868841\\
23.9012904967603	0.00133424928494087\\
23.9512931965443	0.00199208768806045\\
24.0012958963283	0.00304746347917761\\
24.0512985961123	0.00343963177213336\\
24.1013012958963	0.00338388182503185\\
24.1513039956803	0.00302422433184002\\
24.2013066954644	0.00279405551608377\\
24.2513093952484	0.00304029454027128\\
24.3013120950324	0.00327766276802533\\
24.3513147948164	0.00339289216229332\\
24.4013174946004	0.00350793142998796\\
24.4513201943845	0.0029900336611875\\
24.5013228941685	0.00274912550788514\\
24.5513255939525	0.00234646628786219\\
24.6013282937365	0.00222762717366508\\
24.6513309935205	0.00245069656379573\\
24.7013336933045	0.00272392556471188\\
24.7513363930886	0.00287312434467878\\
24.8013390928726	0.00364467993278136\\
24.8513417926566	0.00426846482277529\\
24.9013444924406	0.00425949406098344\\
24.9513471922246	0.00431334501480859\\
25.0013498920086	0.00439605413842229\\
25.0513525917927	0.00491204734217163\\
25.1013552915767	0.00524463370145238\\
25.1513579913607	0.00582552436931003\\
25.2013606911447	0.00567985717112449\\
25.2513633909287	0.00601967972365235\\
25.3013660907127	0.00603233176534936\\
25.3513687904968	0.00574471711316518\\
25.4013714902808	0.00558992622889527\\
25.4513741900648	0.00538133065885374\\
25.5013768898488	0.00502899628228773\\
25.5513795896328	0.00533467153965397\\
25.6013822894168	0.00542623111119736\\
25.6513849892009	0.00538315287928594\\
25.7013876889849	0.0054929397081371\\
25.7513903887689	0.00533828104981426\\
25.8013930885529	0.00513690557329559\\
25.8513957883369	0.00527888260450313\\
25.901398488121	0.00542457057739379\\
25.951401187905	0.00446086147071748\\
26.001403887689	0.00294862249540413\\
26.051406587473	0.00242372585260081\\
26.101409287257	0.00212707633642342\\
26.151411987041	0.00194183728102899\\
26.2014146868251	0.00255850590450707\\
26.2514173866091	0.00309269313183119\\
26.3014200863931	0.00300089453768856\\
26.3514227861771	0.00286612542595932\\
26.4014254859611	0.00237698688890595\\
26.4514281857451	0.00149958874320624\\
26.5014308855292	0.000458579703979448\\
26.5514335853132	0.000214039696794486\\
26.6014362850972	0.000550079369976169\\
26.6514389848812	0.000562609641191814\\
26.7014416846652	0.00037754985172562\\
26.7514443844492	0.000235649862124914\\
26.8014470842333	0.000427969665200734\\
26.8514497840173	0.000845139576806593\\
26.9014524838013	0.00123508888908566\\
26.9514551835853	0.00133940848276814\\
27.0014578833693	0.00112195872085287\\
27.0514605831534	0.000854009321760929\\
27.1014632829374	0.00110230915982468\\
27.1514659827214	0.00161652876177592\\
27.2014686825054	0.00227620723155335\\
27.2514713822894	0.00249744600693547\\
27.3014740820734	0.00243101613316888\\
27.3514767818575	0.00252445588599725\\
27.4014794816415	0.00274185542076253\\
27.4514821814255	0.00320938252485714\\
27.5014848812095	0.00343243152494549\\
27.5514875809935	0.0033281820415451\\
27.6014902807775	0.0033676021510743\\
27.6514929805616	0.00383869920534139\\
27.7014956803456	0.00519795415252214\\
27.7514983801296	0.00604481093774999\\
27.8015010799136	0.00632357652892712\\
27.8515037796976	0.00637205482835713\\
27.9015064794816	0.00639906393372067\\
27.9515091792657	0.00618496788575335\\
28.0015118790497	0.00605011039369309\\
28.0515145788337	0.00590102358395891\\
28.1015172786177	0.00539577070385783\\
28.1515199784017	0.00494093748665625\\
28.2015226781857	0.00485271951563599\\
28.2515253779698	0.00494633824803935\\
28.3015280777538	0.00509017649197831\\
28.3515307775378	0.0048707888261675\\
28.4015334773218	0.00417305663539301\\
28.4515361771058	0.00346645126854507\\
28.5015388768899	0.00277430430457159\\
28.5515415766739	0.00219903654294797\\
28.6015442764579	0.0013628586944874\\
28.6515469762419	0.000609399738529948\\
28.7015496760259	0.000424409897677328\\
28.7515523758099	0.000257199926784578\\
28.801555075594	-0.000300269713257299\\
28.851557775378	-0.000807279357423429\\
28.901560475162	-0.00147437882891631\\
28.951563174946	-0.00206219756514822\\
29.00156587473	-0.00242730643745416\\
29.051568574514	-0.00304584352863491\\
29.1015712742981	-0.00336942223407156\\
29.1515739740821	-0.00356720042703173\\
29.2015766738661	-0.00369660917123532\\
29.2515793736501	-0.00409044549999944\\
29.3015820734341	-0.00389077887153352\\
29.3515847732181	-0.00391068833624874\\
29.4015874730022	-0.00457968112662776\\
29.4515901727862	-0.00555944763101826\\
29.5015928725702	-0.0055683483910907\\
29.5515955723542	-0.00528610144531621\\
29.6015982721382	-0.00494642733178722\\
29.6516009719222	-0.00434203406009637\\
29.7016036717063	-0.00401487669103709\\
29.7516063714903	-0.00369494954444185\\
29.8016090712743	-0.00344680167686239\\
29.8516117710583	-0.00285152434063921\\
29.9016144708423	-0.00192738709972887\\
29.9516171706264	-0.0015462190395103\\
30.0016198704104	-0.00208903667733709\\
30.0516225701944	-0.00188240829751046\\
30.1016252699784	-0.00164520795980134\\
30.1516279697624	-0.00166853848167276\\
30.2016306695464	-0.00180690836465789\\
30.2516333693305	-0.00209106811928193\\
30.3016360691145	-0.00164347866771455\\
30.3516387688985	-0.00107523931664185\\
30.4016414686825	-0.000661679621226868\\
30.4516441684665	0.000224689915381493\\
30.5016468682505	0.000598779899376188\\
30.5516495680346	0.000724669758866384\\
30.6016522678186	0.000738919666674803\\
30.6516549676026	0.000496279637794361\\
30.7016576673866	0.000773259692716715\\
30.7516603671706	0.00108968933440783\\
30.8016630669546	0.00146538886216005\\
30.8516657667387	0.00137005901465024\\
30.9016684665227	0.00127660939645029\\
30.9516711663067	0.00172248855634045\\
31.0016738660907	0.00229426705440814\\
31.0516765658747	0.00248131679196432\\
31.1016792656587	0.00233740666053548\\
31.1516819654428	0.00226543748952018\\
31.2016846652268	0.0015731583037256\\
31.2516873650108	0.00119392929768067\\
31.3016900647948	0.000787459672460606\\
31.3516927645788	0.00099778936737817\\
31.4016954643629	0.00200842757799059\\
31.4516981641469	0.00224397723492307\\
31.5017008639309	0.00273642559510133\\
31.5517035637149	0.00327953303227104\\
31.6017062634989	0.0033982419296411\\
31.6517089632829	0.00432236363166546\\
31.701711663067	0.00492645716808901\\
31.751714362851	0.00577515532821822\\
31.801717062635	0.00645297169425284\\
31.851719762419	0.0074238991021229\\
31.901722462203	0.00821492380230794\\
31.951725161987	0.00870037637713264\\
32.0017278617711	0.0089466457051069\\
32.0517305615551	0.00927034303882632\\
32.1017332613391	0.00942674730853918\\
32.1517359611231	0.00950397299057843\\
32.2017386609071	0.00901486351539632\\
32.2517413606911	0.0085601513426075\\
32.3017440604752	0.0082276013610388\\
32.3517467602592	0.00794348029564186\\
32.4017494600432	0.00788971330447455\\
32.4517521598272	0.00734668058175393\\
32.5017548596112	0.00667606701884247\\
32.5517575593952	0.00589210217084455\\
32.6017602591793	0.00547140024737316\\
32.6517629589633	0.00522669404171772\\
32.7017656587473	0.00492651539021049\\
32.7517683585313	0.0042180651981794\\
32.8017710583153	0.00388183884366414\\
32.8517737580993	0.00379196929803827\\
32.9017764578834	0.00387828916932585\\
32.9517791576674	0.00433307405089926\\
33.0017818574514	0.0048526288558031\\
33.0517845572354	0.00520522355454912\\
33.1017872570194	0.00517638545550738\\
33.1517899568035	0.00520337364677479\\
33.2017926565875	0.00446444246368493\\
33.2517953563715	0.00404539708232453\\
33.3017980561555	0.0032417523116352\\
33.3518007559395	0.00330108317583371\\
33.4018034557235	0.00292892451958916\\
33.4518061555076	0.00238063661839234\\
33.5018088552916	0.00222242752488762\\
33.5518115550756	0.00201735793342532\\
33.6018142548596	0.00242012549989068\\
33.6518169546436	0.00208747721211707\\
33.7018196544277	0.00153909863093976\\
33.7518223542117	0.0012351589310057\\
33.8018250539957	0.000526670060462438\\
33.8518277537797	0.000640149741576534\\
33.9018304535637	0.00090620950017878\\
33.9518331533477	0.00139886917636172\\
34.0018358531318	0.00180696827719675\\
34.0518385529158	0.00235884698452354\\
34.1018412526998	0.00238757680239016\\
34.1518439524838	0.00281214485910528\\
34.2018466522678	0.00271138573153159\\
34.2518493520518	0.00290553470997876\\
34.3018520518359	0.00302783421751983\\
34.3518547516199	0.00357968144832602\\
34.4018574514039	0.00433315272128077\\
34.4518601511879	0.00491932620099826\\
34.5018628509719	0.00490491936017063\\
34.5518655507559	0.00471797025052832\\
34.60186825054	0.00496784784030365\\
34.651870950324	0.00512426575760255\\
34.701873650108	0.00540302101685182\\
34.751876349892	0.00511719487777197\\
34.801879049676	0.00530969294837072\\
34.85188174946	0.00512424570014062\\
34.9018844492441	0.00481493047084053\\
34.9518871490281	0.00505949704861202\\
35.0018898488121	0.00543526173109744\\
35.0518925485961	0.00566374789012745\\
35.1018952483801	0.00591905410253261\\
35.1518979481641	0.00580752530239062\\
35.2019006479482	0.00595328317091551\\
35.2519033477322	0.0062642769004534\\
35.3019060475162	0.00648724067393887\\
35.3519087473002	0.00639002294669871\\
35.4019114470842	0.00619579773228755\\
35.4519141468683	0.00621746646880665\\
35.5019168466523	0.00631642569470169\\
35.5519195464363	0.00613473801544346\\
35.6019222462203	0.00598012124262992\\
35.6519249460043	0.00601790050461425\\
35.7019276457883	0.00604495165217911\\
35.7519303455724	0.00597649304155158\\
35.8019330453564	0.00583632559660245\\
35.8519357451404	0.00573919489269142\\
35.9019384449244	0.00558110801468417\\
35.9519411447084	0.00534008234113301\\
36.0019438444924	0.00549458078840487\\
36.0519465442765	0.0056888674134802\\
36.1019492440605	0.00590288198363911\\
36.1519519438445	0.00575708511573964\\
36.2019546436285	0.00587777304969314\\
36.2519573434125	0.00590830377927248\\
36.3019600431965	0.00595141198642355\\
36.3519627429806	0.00590285152312133\\
36.4019654427646	0.00555046935881876\\
36.4519681425486	0.00521422526520973\\
36.5019708423326	0.00496791807295861\\
36.5519735421166	0.0045326223762739\\
36.6019762419006	0.00424491525572717\\
36.6519789416847	0.00334608173831331\\
36.7019816414687	0.00309965385426933\\
36.7519843412527	0.00261598495283963\\
36.8019870410367	0.00237695678537874\\
36.8519897408207	0.00262503582656922\\
36.9019924406048	0.00253864615488404\\
36.9519951403888	0.00228880717450668\\
37.0019978401728	0.00142751835636407\\
37.0520005399568	0.0003757201082467\\
37.1020032397408	5.75202021288718e-05\\
37.1520059395248	0.000384709809858219\\
37.2020086393089	0.000693959637150849\\
37.2520113390929	0.00122252920905154\\
37.3020140388769	0.00169184790198186\\
37.3520167386609	0.00179806827677754\\
37.4020194384449	0.00192208834049123\\
37.4520221382289	0.0018986682249785\\
37.502024838013	0.00126213895347824\\
37.552027537797	0.000356029868759906\\
37.602030237581	0.000438529513249345\\
37.652032937365	0.000449649749579412\\
37.702035637149	0.000886429507537473\\
37.752038336933	0.00130895867259276\\
37.8020410367171	0.00198507780632461\\
37.8520437365011	0.00234274643857579\\
37.9020464362851	0.00278506437823052\\
37.9520491360691	0.00326860236335206\\
38.0020518358531	0.00379371949994972\\
38.0520545356372	0.00369478929009691\\
38.1020572354212	0.00352582990986555\\
38.1520599352052	0.00375593894846816\\
38.2020626349892	0.00419286568241482\\
38.2520653347732	0.00461532150735514\\
38.3020680345572	0.00433489392248629\\
38.3520707343413	0.00456151308378561\\
38.4020734341253	0.00469088053628568\\
38.4520761339093	0.00444277250122623\\
38.5020788336933	0.00443743371814253\\
38.5520815334773	0.00459398125886192\\
38.6020842332613	0.00514222587576821\\
38.6520869330454	0.00477718030834715\\
38.7020896328294	0.00442306361084213\\
38.7520923326134	0.00467309062272553\\
38.8020950323974	0.00490486770907312\\
38.8520977321814	0.00480419918584166\\
38.9021004319654	0.00456693245711129\\
38.9521031317495	0.00471970034402126\\
39.0021058315335	0.00443915103149122\\
39.0521085313175	0.00459197128609839\\
39.1021112311015	0.00471787996161244\\
39.1521139308855	0.0046747610728386\\
39.2021166306696	0.00471435000436613\\
39.2521193304536	0.00401149688417061\\
39.3021220302376	0.00425594534659075\\
39.3521247300216	0.00389089701142924\\
39.4021274298056	0.00322913327345712\\
39.4521301295896	0.00265206543182994\\
39.5021328293737	0.00147975836109475\\
39.5521355291577	0.000251629811654192\\
39.6021382289417	-0.000195959896192169\\
39.6521409287257	-0.000508939605322318\\
39.7021436285097	-0.000794769295509856\\
39.7521463282937	-0.00109690892743403\\
39.8021490280778	-0.000618519511049038\\
39.8521517278618	-0.000373959801902219\\
39.9021544276458	0.000341629886748533\\
39.9521571274298	0.0015084487992411\\
40.0021598272138	0.00201369789093214\\
40.0521625269978	0.00221148762521484\\
40.1021652267819	0.00236620695748794\\
40.1521679265659	0.00216486756244311\\
40.2021706263499	0.00175126853189745\\
40.2521733261339	0.00172249782426752\\
40.3021760259179	0.00251180601809696\\
40.3521787257019	0.003137403298325\\
40.402181425486	0.00351323094244012\\
40.45218412527	0.00310880381685368\\
40.502186825054	0.00300264396792672\\
40.552189524838	0.00280841529006896\\
40.602192224622	0.00265751564958432\\
40.652194924406	0.00280850529174357\\
40.7021976241901	0.00285356538104834\\
40.7522003239741	0.00305294340439655\\
40.8022030237581	0.00285875509664806\\
40.8522057235421	0.00214331697248389\\
40.9022084233261	0.00129272876301639\\
40.9522111231102	0.00143475891559517\\
41.0022138228942	0.00200474739997376\\
41.0522165226782	0.00270767428946571\\
41.1022192224622	0.00294687302766485\\
41.1522219222462	0.00246319656078214\\
41.2022246220302	0.00187699806489323\\
41.2522273218143	0.0017888983648984\\
41.3022300215983	0.00143126902524554\\
41.3522327213823	0.000769619582045679\\
41.4022354211663	0.000237249876907835\\
41.4522381209503	-0.000271599753943099\\
41.5022408207343	-0.000649229505605303\\
41.5522435205184	-0.000127719768687107\\
41.6022462203024	0.000683219626375215\\
41.6522489200864	0.00128566892514395\\
41.7022516198704	0.00148513855018477\\
41.7522543196544	0.00176395790543268\\
41.8022570194385	0.00302055329641319\\
41.8522597192225	0.0038064287574006\\
41.9022624190065	0.00413356671983891\\
41.9522651187905	0.00368573894300746\\
42.0022678185745	0.00357445035956827\\
42.0522705183585	0.00386024946583087\\
42.1022732181426	0.00422526609184498\\
42.1522759179266	0.00457776281849368\\
42.2022786177106	0.0043638137093383\\
42.2522813174946	0.00454349276497787\\
42.3022840172786	0.00451663390717768\\
42.3522867170626	0.00444112130882181\\
42.4022894168467	0.00416953536806037\\
42.4522921166307	0.00466582111658377\\
42.5022948164147	0.00450770192757692\\
42.5522975161987	0.00433500458469667\\
42.6023002159827	0.00428276462297503\\
42.6523029157667	0.00388724781381283\\
42.7023056155508	0.00337306249996738\\
42.7523083153348	0.00276715545650488\\
42.8023110151188	0.00262315652112114\\
42.8523137149028	0.00293435382815401\\
42.9023164146868	0.00254779535671025\\
42.9523191144708	0.00235898696223635\\
43.0023218142549	0.00186078830750413\\
43.0523245140389	0.00128192916081836\\
43.1023272138229	0.0006508696609716\\
43.1523299136069	-0.000582509590815581\\
43.2023326133909	-0.0018248774012112\\
43.2523353131749	-0.00226916667362878\\
43.302338012959	-0.00278865373355642\\
43.352340712743	-0.00279585459994314\\
43.402343412527	-0.0032885824522709\\
43.452346112311	-0.00396999868065576\\
43.502348812095	-0.00437817383077025\\
43.5523515118791	-0.00423973618199759\\
43.6023542116631	-0.00408136703572331\\
43.6523569114471	-0.00336220223904603\\
43.7023596112311	-0.00351497065544172\\
43.7523623110151	-0.00337489255934782\\
43.8023650107991	-0.00363746065500374\\
43.8523677105832	-0.00337667161880986\\
43.9023704103672	-0.00326692266930452\\
43.9523731101512	-0.00357073113031836\\
44.0023758099352	-0.00311583306423877\\
44.0523785097192	-0.00268971583403272\\
44.1023812095032	-0.00257650581368028\\
44.1523839092873	-0.00292723440316876\\
44.2023866090713	-0.00281032574431531\\
44.2523893088553	-0.0027112950860048\\
44.3023920086393	-0.00293085451976324\\
44.3523947084233	-0.00337111189129592\\
44.4023974082073	-0.00352588037371459\\
44.4524001079914	-0.00404735663034087\\
44.5024028077754	-0.00330119203658418\\
44.5524055075594	-0.00239498619890092\\
44.6024082073434	-0.00182318822435234\\
44.6524109071274	-0.0016327186619111\\
44.7024136069114	-0.00154450897065468\\
44.7524163066955	-0.00100499952045453\\
44.8024190064795	-0.000591479774582077\\
44.8524217062635	-0.000424429839627637\\
44.9024244060475	-0.000363209716596425\\
44.9524271058315	-0.000264309720565549\\
45.0024298056155	-3.60698305887373e-05\\
45.0524325053996	0.000188700094289764\\
45.1024352051836	0.000176160282847857\\
45.1524379049676	0.000507079858725142\\
45.2024406047516	0.00025539989767019\\
45.2524433045356	0.000361489825815174\\
45.3024460043197	0.000476629871251091\\
45.3524487041037	0.000620299735768219\\
45.4024514038877	0.000632889640619591\\
45.4524541036717	0.000433339968929961\\
45.5024568034557	0.000190569984372465\\
45.5524595032397	-0.000133069933037499\\
45.6024622030238	5.57802689794284e-05\\
45.6524649028078	0.000744339485824126\\
45.7024676025918	0.00100868948637548\\
45.7524703023758	0.00126219910546192\\
45.8024730021598	0.0016721476023496\\
45.8524757019439	0.00148343854339895\\
45.9024784017279	0.000924289520510235\\
45.9524811015119	0.000226419996922166\\
46.0024838012959	6.11400047985896e-05\\
46.0524865010799	-7.55499949421792e-05\\
46.1024892008639	0.00013316998789784\\
46.152491900648	-0.000352339658275781\\
46.202494600432	-0.000474599764417297\\
46.252497300216	-0.00074621931414939\\
46.3025	-0.000879259477395778\\
46.352502699784	-0.000658109858804642\\
46.402505399568	-0.000593219186858262\\
46.4525080993521	-0.000920599231488993\\
46.5025107991361	-0.00113992946063732\\
46.5525134989201	-0.0011236692959578\\
46.6025161987041	-0.00068867959171149\\
46.6525188984881	-0.000535759779240732\\
46.7025215982721	-6.30801608565412e-05\\
46.7525242980562	-7.91399981448291e-05\\
46.8025269978402	-0.00033810964329721\\
46.8525296976242	-0.000600489652576696\\
46.9025323974082	-0.00124602908456639\\
46.9525350971922	-0.00189149793775259\\
47.0025377969762	-0.00201906783278563\\
47.0525404967603	-0.00210016748770153\\
47.1025431965443	-0.00227435698662607\\
47.1525458963283	-0.00259089534395604\\
47.2025485961123	-0.00307990301636289\\
47.2525512958963	-0.00293450481358719\\
47.3025539956803	-0.00323996250365947\\
47.3525566954644	-0.00352579036346967\\
47.4025593952484	-0.00439960356811501\\
47.4525620950324	-0.0045058513960571\\
47.5025647948164	-0.00433492415988929\\
47.5525674946004	-0.00444105340959154\\
47.6025701943845	-0.00409048697860514\\
47.6525728941685	-0.00334429131692292\\
47.7025755939525	-0.00309435393979714\\
47.7525782937365	-0.00230316641359661\\
47.8025809935205	-0.00127289906527233\\
47.8525836933045	-0.000674209783795028\\
47.9025863930886	-1.43500385224513e-05\\
47.9525890928726	0.000548359842153551\\
48.0025917926566	0.0010967493762624\\
48.0525944924406	0.00171708765235766\\
48.1025971922246	0.00230311651108682\\
48.1525998920086	0.00292723418306\\
48.2026025917927	0.00389445851159452\\
48.2526052915767	0.00453987222823123\\
48.3026079913607	0.00459211215953575\\
48.3526106911447	0.00442126353425929\\
48.4026133909287	0.0048653394910996\\
48.4526160907127	0.00494633766804388\\
48.5026187904968	0.00502708705030539\\
48.5526214902808	0.0046747206174938\\
48.6026241900648	0.00442130486781359\\
48.6526268898488	0.00413008564644757\\
48.7026295896328	0.00421450427938129\\
48.7526322894168	0.00449678219488376\\
48.8026349892009	0.00471438962464723\\
48.8526376889849	0.00428818556102667\\
48.9026403887689	0.00416593589708221\\
48.9526430885529	0.00383509717116442\\
49.0026457883369	0.00307453374203581\\
49.052648488121	0.00218451707970831\\
49.102651187905	0.00214313691955187\\
49.152653887689	0.00211085727990739\\
49.202656587473	0.00201914748782639\\
49.252659287257	0.00187535706382901\\
49.302661987041	0.000960089537344542\\
49.3526646868251	0.00115785921459375\\
49.4026673866091	0.0017782683235262\\
49.4526700863931	0.00162898853573733\\
49.5026727861771	0.00169179854438567\\
49.5526754859611	0.00168833794459515\\
49.6026781857451	0.0017889571952132\\
49.6526808855292	0.00165040867697787\\
49.7026835853132	0.00194722806751218\\
49.7526862850972	0.00211628692888515\\
49.8026889848812	0.00171522838953027\\
49.8526916846652	0.00194180735960732\\
49.9026943844492	0.00174578832698388\\
49.9526970842333	0.00221501750009263\\
50.0026997840173	0.00219347745321166\\
50.0527024838013	0.00222415708389657\\
50.1027051835853	0.00175305834118241\\
50.1527078833693	0.00114002951328118\\
50.2027105831534	0.000843169408893043\\
50.2527132829374	0.000647179529573017\\
50.3027159827214	0.00134115851637777\\
50.3527186825054	0.00198855791893899\\
50.4027213822894	0.00216669643758935\\
50.4527240820734	0.00252440498362477\\
50.5027267818575	0.00259627604106286\\
50.5527294816415	0.00224392730385893\\
50.6027321814255	0.001884448080172\\
50.6527348812095	0.00110927936035129\\
50.7027375809935	0.0014690290298229\\
50.7527402807775	0.00125861900930967\\
50.8027429805616	0.000897399358710261\\
50.8527456803456	0.000490919775444955\\
50.9027483801296	7.00799181265671e-05\\
50.9527510799136	-0.000510569831979365\\
51.0027537796976	-0.000631129729216842\\
51.0527564794816	-0.000600579690294316\\
51.1027591792657	-0.000348719656193978\\
51.1527618790497	-0.000327179476639057\\
51.2027645788337	-0.000221189907494259\\
51.2527672786177	-0.000151029998258198\\
51.3027699784017	-0.000129409950710511\\
51.3527726781858	-0.000264249938517082\\
51.4027753779698	-0.000233829978992817\\
51.4527780777538	0.000154569961317225\\
51.5027807775378	0.000332540004261411\\
51.5527834773218	6.47201584153381e-05\\
51.6027861771058	0.000177979952722506\\
51.6527888768899	0.000231919916805988\\
51.7027915766739	0.000280469925258247\\
51.7527942764579	0.000239099971425318\\
51.8027969762419	0.000602249595044983\\
51.8527996760259	0.00070300952998984\\
51.9028023758099	0.000348879852118325\\
51.952805075594	3.41200151886253e-05\\
52.002807775378	-0.000170919989100204\\
52.052810475162	-5.2190005358168e-05\\
52.102813174946	0.000474719858712922\\
52.15281587473	0.00111285935688178\\
52.202818574514	0.00163623845600037\\
52.2528212742981	0.00236976668816891\\
52.3028239740821	0.00227454682930199\\
52.3528266738661	0.00218441689787237\\
52.4028293736501	0.00205505740265017\\
52.4528320734341	0.00232676685205987\\
52.5028347732181	0.00283374471294499\\
52.5528374730022	0.00338213090457357\\
52.6028401727862	0.00347010096198404\\
52.6528428725702	0.00308363367616927\\
52.7028455723542	0.00330110178975981\\
52.7528482721382	0.00345581129041286\\
52.8028509719222	0.00290540309004659\\
52.8528536717063	0.00266814499704332\\
52.9028563714903	0.00281565500927903\\
52.9528590712743	0.00208384690586102\\
53.0028617710583	0.00133410872375079\\
53.0528644708423	0.00116699886135287\\
53.1028671706264	0.000284220353132462\\
53.1528698704104	0.000181679927191552\\
53.2028725701944	-0.000203129701990933\\
53.2528752699784	0.000320109924441037\\
53.3028779697624	-8.99198129256683e-05\\
53.3528806695464	-0.00049986963067135\\
53.4028833693305	-0.000920459536657\\
53.4528860691145	-0.000640039823458378\\
53.5028887688985	-0.000237349937412331\\
53.5528914686825	0.000255409876868627\\
53.6028941684665	0.000706639600517943\\
53.6528968682505	0.00133583905157533\\
53.7028995680346	0.0014815977191148\\
53.7529022678186	0.00178550788763731\\
53.8029049676026	0.00153726826925826\\
53.8529076673866	0.000285819692207322\\
53.9029103671706	-0.000426129737918173\\
53.9529130669547	-0.000881039637203463\\
54.0029157667387	-0.000476469864959895\\
54.0529184665227	-0.000751669435273273\\
54.1029211663067	-0.000767869542033553\\
54.1529238660907	-0.00129278914574383\\
54.2029265658747	-0.00202628779789695\\
54.2529292656588	-0.00297752400697056\\
54.3029319654428	-0.00382801884888528\\
54.3529346652268	-0.00415162556875782\\
54.4029373650108	-0.00368052933594275\\
54.4529400647948	-0.00365698995329947\\
54.5029427645788	-0.0040563154700856\\
54.5529454643629	-0.00472141964214133\\
54.6029481641469	-0.00404721708616086\\
54.6529508639309	-0.00301516405810238\\
54.7029535637149	-0.00285332511483205\\
54.7529562634989	-0.00294329469427896\\
54.8029589632829	-0.00335507120949101\\
54.852961663067	-0.00390703721405686\\
54.902964362851	-0.00331005317330301\\
54.952967062635	-0.00235719604092604\\
55.002969762419	-0.00212173744220744\\
55.052972462203	-0.00242370617805109\\
55.102975161987	-0.00293063403253336\\
55.1529778617711	-0.00344495182112723\\
55.2029805615551	-0.00274186402208192\\
55.2529832613391	-0.00175833802991206\\
55.3029859611231	-0.000906179602444737\\
55.3529886609071	-0.000458479794210678\\
55.4029913606911	-0.000215609801766176\\
55.4529940604752	0.000165430070748296\\
55.5029967602592	0.000587839904501372\\
55.5529994600432	0.000790949731627559\\
55.6030021598272	0.000872039416211696\\
55.6530048596112	0.00105366936145565\\
55.7030075593953	0.00144913885979256\\
55.7530102591793	0.00197593800268725\\
55.8030129589633	0.00262327636080257\\
55.8530156587473	0.00260170585973311\\
55.9030183585313	0.00273293472677179\\
55.9530210583153	0.00307640399613353\\
56.0030237580994	0.00415328535223158\\
56.0530264578834	0.00505965552493703\\
56.1030291576674	0.00570499553413222\\
56.1530318574514	0.00589560277591765\\
56.2030345572354	0.00537074122625559\\
56.2530372570194	0.00525934398197636\\
56.3030399568035	0.00522145339648332\\
56.3530426565875	0.00563142687374216\\
56.4030453563715	0.00548936738991504\\
56.4530480561555	0.00533651283469107\\
56.5030507559395	0.00500932686318114\\
56.5530534557235	0.00546411086599676\\
56.6030561555076	0.00576801596854608\\
56.6530588552916	0.00547673115483838\\
56.7030615550756	0.0049319077870376\\
56.7530642548596	0.00437278341398136\\
56.8030669546436	0.00471426076751116\\
56.8530696544276	0.00467650120627834\\
56.9030723542117	0.00489054565877607\\
56.9530750539957	0.00463889162594013\\
57.0030777537797	0.00403651646423361\\
57.0530804535637	0.00375594958531539\\
57.1030831533477	0.00435108558318214\\
57.1530858531318	0.00438002487969307\\
57.2030885529158	0.0045040127206874\\
57.2530912526998	0.00404909804593291\\
57.3030939524838	0.00281913479748756\\
57.3530966522678	0.00220237636332837\\
57.4030993520518	0.00210723768347943\\
57.4531020518358	0.00185902755558906\\
57.5031047516199	0.00191483753913722\\
57.5531074514039	0.0013863389442798\\
57.6031101511879	0.00138267898508889\\
57.6531128509719	0.0013412389205864\\
57.7031155507559	0.000981709447922823\\
57.75311825054	0.00118835933671663\\
57.803120950324	0.00133763876707803\\
57.853123650108	0.00165602810473862\\
57.903126349892	0.00183218751423801\\
57.953129049676	0.00221506753286109\\
58.00313174946	0.00194900816273889\\
58.0531344492441	0.00147251872789674\\
58.1031371490281	0.00107687948746148\\
58.1531398488121	0.00164340888797416\\
58.2031425485961	0.00245072697592513\\
58.2531452483801	0.00297391371142186\\
58.3031479481642	0.00314095278610468\\
58.3531506479482	0.00270412551897773\\
58.4031533477322	0.00186990680233634\\
58.4531560475162	0.00137737923772818\\
58.5031587473002	0.00093324972465021\\
58.5531614470842	0.000584339862454044\\
58.6031641468683	0.000127670019611553\\
58.6531668466523	-0.00039733978316605\\
58.7031695464363	-0.000809009666513718\\
58.7531722462203	-0.00133241829414712\\
58.8031749460043	-0.00204433752864902\\
58.8531776457883	-0.00247040625326514\\
58.9031803455724	-0.00342701154484009\\
58.9531830453564	-0.00386749702959152\\
59.0031857451404	-0.00373987950651004\\
59.0531884449244	-0.00343410228485632\\
59.1031911447084	-0.00274552400491497\\
59.1531938444924	-0.0020983272769289\\
59.2031965442765	-0.00173871832367263\\
59.2531992440605	-0.0013862688991717\\
59.3032019438445	-0.000987209512049864\\
59.3532046436285	-0.00041356979463572\\
59.4032073434125	0.000384719935650402\\
59.4532100431965	0.000909779325083631\\
59.5032127429806	0.000733559680435983\\
59.5532154427646	0.00101956934750335\\
59.6032181425486	0.00108060929596085\\
59.6532208423326	0.00129271918405674\\
59.7032235421166	0.00158398868364583\\
59.7532262419006	0.00181954828606523\\
59.8032289416847	0.00159659872175605\\
59.8532316414687	0.00103551927073854\\
59.9032343412527	0.000773049442948285\\
59.9532370410367	0.000571769714363663\\
60.0032397408207	0.000521469847180079\\
60.0532424406048	0.000780319318524992\\
60.1032451403888	0.00151751876226227\\
60.1532478401728	0.00153545856836991\\
60.2032505399568	0.001341248983216\\
60.2532532397408	0.00123160900508831\\
60.3032559395248	0.00131412870617014\\
60.3532586393089	0.00125492900510075\\
60.4032613390929	0.00059331967283991\\
60.4532640388769	-0.000282289746829835\\
60.5032667386609	-0.000958349166287875\\
60.5532694384449	-0.00125128937347248\\
60.6032721382289	-0.00114527931822214\\
60.653274838013	-0.000695939463907662\\
60.703277537797	-5.56900749343967e-05\\
60.753280237581	0.000436939442653485\\
60.803282937365	0.000906199531019596\\
60.853285637149	0.00148501889647518\\
60.9032883369331	0.00202448753273262\\
60.9532910367171	0.00252433641711433\\
61.0032937365011	0.00288764507782959\\
61.0532964362851	0.00357087079043846\\
61.1032991360691	0.00439795442000325\\
61.1533018358531	0.00448411290875431\\
61.2033045356371	0.00456507284757473\\
61.2533072354212	0.00451116279979938\\
61.3033099352052	0.00498580631086683\\
61.3533126349892	0.00516917418032289\\
61.4033153347732	0.00522297393019774\\
61.4533180345572	0.00566893766234594\\
61.5033207343413	0.00568702732932883\\
61.5533234341253	0.00577339605712297\\
61.6033261339093	0.00581823398624873\\
61.6533288336933	0.00583261364872622\\
61.7033315334773	0.00642425375363245\\
61.7533342332613	0.00638108299943514\\
61.8033369330454	0.00587392427279314\\
61.8533396328294	0.00558984756623299\\
61.9033423326134	0.00532385241643266\\
61.9533450323974	0.00534015075260241\\
62.0033477321814	0.00525371340870952\\
62.0533504319654	0.00533467286513017\\
62.1033531317495	0.00570509489316111\\
62.1533558315335	0.00513508431931235\\
62.2033585313175	0.00458495184243939\\
62.2533612311015	0.00428279313849757\\
62.3033639308855	0.00410131657483443\\
62.3533666306695	0.0038693381282825\\
62.4033693304536	0.00290382430229006\\
62.4533720302376	0.00248838663485624\\
62.5033747300216	0.00256201674308039\\
62.5533774298056	0.00293969373360033\\
62.6033801295896	0.00730158037406515\\
62.6533828293737	0.0209804584179161\\
62.6983852591793	0.0443165373128708\\
62.7433876889849	0.0773065720009305\\
62.7883901187905	0.107975776811472\\
62.8333925485961	0.132401873201575\\
62.8833952483801	0.151120863574408\\
62.9333979481641	0.161097951541105\\
62.9834006479482	0.167220659514066\\
63.0334033477322	0.174147197072698\\
63.0834060475162	0.183101929436818\\
63.1334087473002	0.191018891849376\\
63.1834114470842	0.196952668310739\\
63.2334141468683	0.20137008400624\\
63.2834168466523	0.2055281673067\\
63.3334195464363	0.209942728578313\\
63.3834222462203	0.213768035539077\\
63.4334249460043	0.21765871193351\\
63.4834276457883	0.220498912423392\\
63.5334303455724	0.222248807200975\\
63.5834330453564	0.223994402523092\\
63.6334357451404	0.225933825208333\\
63.6834384449244	0.226659074170141\\
63.7334411447084	0.227287712377391\\
63.7834438444924	0.228044111394442\\
63.8334465442765	0.228850803410512\\
63.8834492440605	0.228639097044047\\
63.9334519438445	0.22836246570446\\
63.9834546436285	0.228726745914892\\
64.0334573434125	0.228702082984959\\
64.0834600431965	0.228714222501722\\
64.1334627429806	0.228775612763585\\
64.1834654427646	0.22849902076264\\
64.2334681425486	0.227783169831619\\
64.2834708423326	0.227508275400174\\
64.3334735421166	0.227084606585917\\
64.3834762419007	0.226946348299605\\
64.4334789416847	0.226387543666201\\
64.4834816414687	0.226709732458104\\
64.5334843412527	0.227109004501926\\
64.5834870410367	0.227324459452586\\
64.6334897408207	0.227436576353541\\
64.6834924406048	0.227529219394336\\
64.7334951403888	0.226932039512644\\
64.7834978401728	0.226926746519794\\
64.8335005399568	0.226450733094179\\
64.8835032397408	0.227110785782134\\
64.9335059395248	0.22693746069593\\
64.9835086393089	0.226872687185046\\
65.0335113390929	0.226907733881642\\
65.0835140388769	0.227067041086207\\
65.1335167386609	0.227273712781603\\
65.1835194384449	0.227434715332753\\
65.2335221382289	0.227522273562301\\
65.283524838013	0.227471527756065\\
65.333527537797	0.227376846635155\\
65.383530237581	0.226872572709928\\
65.433532937365	0.227201924381569\\
65.483535637149	0.227235125397334\\
65.533538336933	0.227123026696113\\
65.5835410367171	0.227219347096309\\
65.6335437365011	0.226926821155741\\
65.6835464362851	0.226165088597998\\
65.7335491360691	0.225879490673085\\
65.7835518358531	0.225332937617479\\
65.8335545356372	0.225282246640105\\
65.8835572354212	0.224537774169903\\
65.9335599352052	0.224637606495278\\
65.9835626349892	0.224721799161723\\
66.0335653347732	0.22449215129886\\
66.0835680345572	0.224555079092082\\
66.1335707343412	0.225028024457346\\
66.1835734341253	0.225566080452077\\
66.2335761339093	0.225658819646868\\
66.2835788336933	0.225650094448491\\
66.3335815334773	0.225729092744312\\
66.3835842332613	0.225558787218479\\
66.4335869330454	0.225280391134651\\
66.4835896328294	0.225583456119156\\
66.5335923326134	0.225373412038953\\
66.5835950323974	0.225396041968847\\
66.6335977321814	0.225467957918357\\
66.6836004319654	0.225781515901777\\
66.7336031317495	0.225864009053866\\
66.7836058315335	0.225877951538374\\
66.8336085313175	0.225930399908062\\
66.8836112311015	0.225872458582879\\
66.9336139308855	0.226042467536164\\
66.9836166306695	0.226017913809417\\
67.0336193304536	0.226476886042622\\
67.0836220302376	0.226872527843495\\
67.1336247300216	0.226732422861491\\
67.1836274298056	0.22703012353998\\
67.2336301295896	0.227443163443539\\
67.2836328293737	0.227122988095744\\
67.3336355291577	0.227364690212217\\
67.3836382289417	0.227038910309663\\
67.4336409287257	0.227480376230828\\
67.4836436285097	0.227207120462992\\
67.5336463282937	0.22724375202725\\
67.5836490280778	0.227257646508258\\
67.6336517278618	0.227499491978991\\
67.6836544276458	0.227250789714039\\
67.7336571274298	0.227385699959371\\
67.7836598272138	0.227154650139796\\
67.8336625269979	0.227406620744997\\
67.8836652267819	0.22714934087652\\
67.9336679265659	0.227618602463614\\
67.9836706263499	0.227690235351587\\
68.0336733261339	0.227837173379935\\
68.0836760259179	0.228408125072835\\
68.1336787257019	0.228448291047751\\
68.183681425486	0.228079093063157\\
68.23368412527	0.227327815626012\\
68.283686825054	0.227052969375949\\
68.333689524838	0.226746614721429\\
68.383692224622	0.226434829804418\\
68.4336949244061	0.226610061979202\\
68.4836976241901	0.226340232830462\\
68.5337003239741	0.226646742762452\\
68.5837030237581	0.226580297187269\\
68.6337057235421	0.226695794527674\\
68.6837084233261	0.226257919852844\\
68.7337111231102	0.226205363321338\\
68.7837138228942	0.226515440713694\\
68.8337165226782	0.227003758094325\\
68.8837192224622	0.22711941622174\\
68.9337219222462	0.226741358759102\\
68.9837246220302	0.226956661451517\\
69.0337273218143	0.227177204943962\\
69.0837300215983	0.227655213972991\\
69.1337327213823	0.227326048888493\\
69.1837354211663	0.227376923985825\\
69.2337381209503	0.227319130717815\\
69.2837408207344	0.227578138183414\\
69.3337435205184	0.227347142970338\\
69.3837462203024	0.227541600837216\\
69.4337489200864	0.227370026449022\\
69.4837516198704	0.227312096753621\\
69.5337543196544	0.227441571157644\\
69.5837570194385	0.226905672147285\\
69.6337597192225	0.226669433326853\\
69.6837624190065	0.227047662009919\\
69.7337651187905	0.227348916321184\\
69.7837678185745	0.227159729101177\\
69.8337705183585	0.227154504937698\\
69.8837732181426	0.227024971077747\\
69.9337759179266	0.227494328659743\\
69.9837786177106	0.228135054219434\\
70.0337813174946	0.228261066731367\\
70.0837840172786	0.228077234370484\\
70.1337867170626	0.227301634220992\\
70.1837894168467	0.227028491548167\\
70.2337921166307	0.227103847987583\\
70.2837948164147	0.226867610939524\\
70.3337975161987	0.226723826073888\\
70.3838002159827	0.226595946763771\\
70.4338029157667	0.226816707711521\\
70.4838056155508	0.226930546123243\\
70.5338083153348	0.226906095921723\\
70.5838110151188	0.226709811050348\\
70.6338137149028	0.226830568400871\\
70.6838164146868	0.226718556109747\\
70.7338191144708	0.226506590256718\\
70.7838218142549	0.22648717111665\\
70.8338245140389	0.226750052453913\\
70.8838272138229	0.22698999273731\\
70.9338299136069	0.226867371123008\\
70.9838326133909	0.227039039599266\\
71.0338353131749	0.227427616418656\\
71.083838012959	0.227355957791133\\
71.133840712743	0.227515182888692\\
71.183843412527	0.227588768293458\\
71.233846112311	0.22743666743276\\
71.283848812095	0.227453878608007\\
71.3338515118791	0.227035431167785\\
71.3838542116631	0.227163287213078\\
71.4338569114471	0.227007489486918\\
71.4838596112311	0.227107179704546\\
71.5338623110151	0.227576491447949\\
71.5838650107991	0.227963450897876\\
71.6338677105831	0.22733665864003\\
71.6838704103672	0.227156352042792\\
71.7338731101512	0.22706339700041\\
71.7838758099352	0.22723145512094\\
71.8338785097192	0.226988198986808\\
71.8838812095032	0.2270477962494\\
71.9338839092873	0.227231715816556\\
71.9838866090713	0.227184321815122\\
72.0338893088553	0.227410199135695\\
72.0838920086393	0.22780775648971\\
72.1338947084233	0.227720129519331\\
72.1838974082073	0.227544977516418\\
72.2339001079914	0.227557131988372\\
72.2839028077754	0.227385664799045\\
72.3339055075594	0.227179060240081\\
72.3839082073434	0.226925134230673\\
72.4339109071274	0.227417117025614\\
72.4839136069115	0.227709582878879\\
72.5339163066955	0.227914323622725\\
72.5839190064795	0.227886356675259\\
72.6339217062635	0.227930154822992\\
72.6839244060475	0.227623901936983\\
72.7339271058315	0.227476676062433\\
72.7839298056156	0.227126553568641\\
72.8339325053996	0.227140658575291\\
72.8839352051836	0.227156236381577\\
72.9339379049676	0.22750823894483\\
72.9839406047516	0.227685203287089\\
73.0339433045356	0.227965323619949\\
73.0839460043197	0.227651878075114\\
73.1339487041037	0.227261425086144\\
73.1839514038877	0.226797461698874\\
73.2339541036717	0.226683531397192\\
73.2839568034557	0.227082832842588\\
73.3339595032398	0.227469901691643\\
73.3839622030238	0.227735821254563\\
73.4339649028078	0.22799158539563\\
73.4839676025918	0.227291128335084\\
73.5339703023758	0.226709768563354\\
73.5839730021598	0.226578654098223\\
73.6339757019438	0.226060103194358\\
73.6839784017279	0.226079312896566\\
73.7339811015119	0.226103898691942\\
73.7839838012959	0.226361325420596\\
73.8339865010799	0.226210754210587\\
73.8839892008639	0.22654875189556\\
73.933991900648	0.226375154311748\\
73.983994600432	0.226068778024761\\
74.033997300216	0.226187827759521\\
74.084	0.226275517337194\\
74.134002699784	0.226534628114258\\
74.184005399568	0.226536490918794\\
74.2340080993521	0.226983050977593\\
74.2840107991361	0.227200203238592\\
74.3340134989201	0.227219376662465\\
74.3840161987041	0.227357807776448\\
74.4340188984881	0.226969115053647\\
74.4840215982721	0.226541760186137\\
74.5340242980562	0.226678328429215\\
74.5840269978402	0.227186166227889\\
74.6340296976242	0.227532753582388\\
74.6840323974082	0.22711949294522\\
74.7340350971922	0.227131880975317\\
74.7840377969763	0.226886778470079\\
74.8340404967603	0.226618809860777\\
74.8840431965443	0.226103943890954\\
74.9340458963283	0.225951650651653\\
74.9840485961123	0.22597612800623\\
75.0340512958963	0.226012857081478\\
75.0840539956804	0.226372025233828\\
75.1340566954644	0.226319454436018\\
75.1840593952484	0.22624228638923\\
75.2340620950324	0.226079456372338\\
75.2840647948164	0.226047846078021\\
75.3340674946004	0.226051481237267\\
75.3840701943844	0.226291184471901\\
75.4340728941685	0.226352576639098\\
75.4840755939525	0.226352466419687\\
75.5340782937365	0.22633845015395\\
75.5840809935205	0.225923416112247\\
75.6340836933045	0.225732483505411\\
75.6840863930885	0.225742954972336\\
75.7340890928726	0.225862195875967\\
75.7840917926566	0.225758868919044\\
75.8340944924406	0.225594184963269\\
75.8840971922246	0.226042576618594\\
75.9340998920086	0.226207176563544\\
75.9841025917927	0.226674869987872\\
76.0341052915767	0.226569705227704\\
76.0841079913607	0.227065315844649\\
76.1341106911447	0.227263319470776\\
76.1841133909287	0.226709958379363\\
76.2341160907127	0.226834205941133\\
76.2841187904968	0.226527585686737\\
76.3341214902808	0.226814751098542\\
76.3841241900648	0.227170267270074\\
76.4341268898488	0.227326137955795\\
76.4841295896328	0.227769222171422\\
76.5341322894169	0.227658930105112\\
76.5841349892009	0.226972365566557\\
76.6341376889849	0.227063461430736\\
76.6841403887689	0.227070619105458\\
76.7341430885529	0.226737788411929\\
76.7841457883369	0.226785103425983\\
76.834148488121	0.227254352535118\\
76.884151187905	0.227622100696805\\
76.934153887689	0.227966967802951\\
76.984156587473	0.227459260151877\\
77.034159287257	0.227060126490483\\
77.084161987041	0.227142345400423\\
77.1341646868251	0.227147709389291\\
77.1841673866091	0.227287597684309\\
77.2341700863931	0.227809399484808\\
77.2841727861771	0.228040580442081\\
77.3341754859611	0.228240115045913\\
77.3841781857451	0.228287217031012\\
77.4341808855292	0.227583506489992\\
77.4841835853132	0.226907613790637\\
77.5341862850972	0.226198471084379\\
77.5841889848812	0.225800863252245\\
77.6341916846652	0.226292992588555\\
77.6841943844492	0.22652577233064\\
77.7341970842333	0.227168435203466\\
77.7841997840173	0.227560737529728\\
77.8342024838013	0.228019517279264\\
77.8842051835853	0.228166585510011\\
77.9342078833693	0.227392672471546\\
77.9842105831534	0.226646741080907\\
78.0342132829374	0.226371912043483\\
78.0842159827214	0.226004089530256\\
78.1342186825054	0.225835870025411\\
78.1842213822894	0.225839347922292\\
78.2342240820734	0.226100361072936\\
78.2842267818575	0.225748386030993\\
78.3342294816415	0.22581482153241\\
78.3842321814255	0.225532996866684\\
78.4342348812095	0.225489135494228\\
78.4842375809935	0.225728958328691\\
78.5342402807775	0.225883030352626\\
78.5842429805616	0.225883007659046\\
78.6342456803456	0.225860322397814\\
78.6842483801296	0.225615203108192\\
78.7342510799136	0.225350777965655\\
78.7842537796976	0.224830278935497\\
78.8342564794817	0.224536129369535\\
78.8842591792657	0.224483340143306\\
78.9342618790497	0.224557023322254\\
78.9842645788337	0.224467732484615\\
79.0342672786177	0.224190881247234\\
79.0842699784017	0.223959488736878\\
79.1342726781857	0.224111956333133\\
79.1842753779698	0.224388786617838\\
79.2342780777538	0.224527287664327\\
79.2842807775378	0.225043970277815\\
79.3342834773218	0.225284168587219\\
79.3842861771058	0.225355935248789\\
79.4342888768899	0.225492677255886\\
79.4842915766739	0.225170297187574\\
79.5342942764579	0.22496006401053\\
79.5842969762419	0.225044109560655\\
79.6342996760259	0.225352387514199\\
79.6843023758099	0.225422355400427\\
79.734305075594	0.225089612561671\\
79.784307775378	0.225105532850016\\
79.834310475162	0.224742923307069\\
79.884313174946	0.224753243234981\\
79.93431587473	0.224961699124244\\
79.984318574514	0.225319050886594\\
80.0343212742981	0.225415481210633\\
80.0843239740821	0.225853359755869\\
80.1343266738661	0.225639547277676\\
80.1843293736501	0.225744644360478\\
80.2343320734341	0.225858624266574\\
80.2843347732181	0.226373645830824\\
80.3343374730022	0.22670633837153\\
80.3843401727862	0.226366543418611\\
80.4343428725702	0.22560990033447\\
80.4843455723542	0.225361187951978\\
80.5343482721382	0.225508190431281\\
80.5843509719223	0.225769249165602\\
80.6343536717063	0.225797450016728\\
80.6843563714903	0.225779892118661\\
80.7343590712743	0.226098417023271\\
80.7843617710583	0.226525720871782\\
80.8343644708423	0.22626657100547\\
80.8843671706263	0.22615298150602\\
80.9343698704104	0.226177463583559\\
80.9843725701944	0.226345501555486\\
81.0343752699784	0.22653997323741\\
81.0843779697624	0.226518858993858\\
81.1343806695464	0.226487330802165\\
81.1843833693305	0.227072265561315\\
81.2343860691145	0.227279066251018\\
81.2843887688985	0.227531120229253\\
81.3343914686825	0.227711366428718\\
81.3843941684665	0.227644809087872\\
81.4343968682505	0.227389222295694\\
81.4843995680346	0.226760493708642\\
81.5344022678186	0.226371815538025\\
81.5844049676026	0.226065362503017\\
81.6344076673866	0.226237042191789\\
81.6844103671706	0.226089931916124\\
81.7344130669546	0.225720230860688\\
81.7844157667387	0.22549262045075\\
81.8344184665227	0.225643221108665\\
81.8844211663067	0.226023165353749\\
81.9344238660907	0.225974189084588\\
81.9844265658747	0.22573951252602\\
82.0344292656588	0.225089765593038\\
82.0844319654428	0.225072055829493\\
82.1344346652268	0.225434593042995\\
82.1844373650108	0.225707920790958\\
82.2344400647948	0.226284179412601\\
82.2844427645788	0.226189574606463\\
82.3344454643629	0.225515362037069\\
82.3844481641469	0.224933759103168\\
82.4344508639309	0.224837505188081\\
82.4844535637149	0.224968693661684\\
82.5344562634989	0.224891617880986\\
82.5844589632829	0.224318602284448\\
82.634461663067	0.223935175873868\\
82.684464362851	0.22035270169412\\
82.734467062635	0.207916524503569\\
82.784469762419	0.183328983417589\\
82.8294721922246	0.153412554190566\\
82.8744746220302	0.12306992036011\\
82.9194770518359	0.0996001673622788\\
82.9694797516199	0.0835282550061769\\
83.0194824514039	0.0740236799496861\\
83.0694851511879	0.0640471313885887\\
83.1194878509719	0.0531646232911571\\
83.1694905507559	0.0428059383998715\\
83.21949325054	0.0352712113741453\\
83.269495950324	0.0305683245920346\\
83.319498650108	0.0265279956712241\\
83.369501349892	0.0224460666228536\\
83.419504049676	0.0182972973913731\\
83.4695067494601	0.0153471979930889\\
83.5195094492441	0.0136158592903449\\
83.5695121490281	0.0119780296127673\\
83.6195148488121	0.0104244701914272\\
83.6695175485961	0.00904560210827476\\
83.7195202483801	0.00828147419313903\\
83.7695229481642	0.00735016887896697\\
83.8195256479482	0.007080625498911\\
83.8695283477322	0.00631999370314961\\
83.9195310475162	0.00531497205592703\\
83.9695337473002	0.00489239725917009\\
84.0195364470842	0.00448783180672328\\
84.0695391468683	0.00415345575496839\\
84.1195418466523	0.00358529011788749\\
84.1695445464363	0.00279418464566026\\
84.2195472462203	0.00212527764728735\\
84.2695499460043	0.00263225599875939\\
84.3195526457883	0.00377037962636581\\
84.3695553455724	0.00448961307073124\\
84.4195580453564	0.00433671471979191\\
84.4695607451404	0.00421999477444365\\
84.5195634449244	0.00443564368657821\\
84.5695661447084	0.00436913291121546\\
84.6195688444924	0.00520353342184274\\
84.6695715442765	0.00544432980749576\\
84.7195742440605	0.0056420470379737\\
84.7695769438445	0.00569967692191078\\
84.8195796436285	0.00574993661657542\\
84.8695823434125	0.00574649726373412\\
84.9195850431965	0.00573732698007875\\
84.9695877429806	0.00569234759153559\\
85.0195904427646	0.00564924806138927\\
85.0695931425486	0.00554143978801434\\
85.1195958423326	0.00566186775627214\\
85.1695985421166	0.00587409438145352\\
85.2196012419006	0.00582723545126923\\
85.2696039416847	0.00558444789600639\\
85.3196066414687	0.00548398075420764\\
85.3696093412527	0.00536334303105489\\
85.4196120410367	0.00559358981062458\\
85.4696147408207	0.00530966112351205\\
85.5196174406048	0.00533113191316668\\
85.5696201403888	0.0051675845428977\\
85.6196228401728	0.00550733916916022\\
85.6696255399568	0.00558094940450933\\
85.7196282397408	0.00563136891314282\\
85.7696309395248	0.0055774186954233\\
85.8196336393089	0.00566194855139695\\
85.8696363390929	0.00567428720235431\\
85.9196390388769	0.0053757925867606\\
85.9696417386609	0.00508650600980657\\
86.0196444384449	0.00411563611157582\\
86.0696471382289	0.00420916470850511\\
86.119649838013	0.00469822052844111\\
86.169652537797	0.00515118574319424\\
86.219655237581	0.00558447830372627\\
86.269657937365	0.00565826705706104\\
86.319660637149	0.00565469723049222\\
86.369663336933	0.00594239166309741\\
86.4196660367171	0.00647463328721985\\
86.4696687365011	0.00692219203157614\\
86.5196714362851	0.00714340651498583\\
86.5696741360691	0.00701212803431326\\
86.6196768358531	0.00684671257649742\\
86.6696795356372	0.0063954935961842\\
86.7196822354212	0.00634147460325669\\
86.7696849352052	0.00645842249476133\\
86.8196876349892	0.00644944239785992\\
86.8696903347732	0.00632904538683023\\
86.9196930345572	0.00592630228214516\\
86.9696957343413	0.00569604666040618\\
87.0196984341253	0.00574103695938795\\
87.0697011339093	0.00593697255520646\\
87.1197038336933	0.00630010605433688\\
87.1697065334773	0.00576972556807831\\
87.2197092332613	0.0052591332004222\\
87.2697119330454	0.00507735581727484\\
87.3197146328294	0.00518545437518709\\
87.3697173326134	0.00569443521090498\\
87.4197200323974	0.00558627965336371\\
87.4697227321814	0.00554142821307241\\
87.5197254319655	0.00556125918506151\\
87.5697281317495	0.00574109659625791\\
87.6197308315335	0.00598550313166618\\
87.6697335313175	0.00605017032841072\\
87.7197362311015	0.00582539429039633\\
87.7697389308855	0.00547856050937998\\
87.8197416306695	0.00597107952395287\\
87.8697443304536	0.00575900556352167\\
87.9197470302376	0.00560434771982412\\
87.9697497300216	0.00531131212919511\\
88.0197524298056	0.00559360949780868\\
88.0697551295896	0.00571385787175204\\
88.1197578293737	0.00573193635581792\\
88.1697605291577	0.00591189395675378\\
88.2197632289417	0.00610056030840477\\
88.2697659287257	0.00631454623234289\\
88.3197686285097	0.00639546412215314\\
88.3697713282937	0.00615614753873913\\
88.4197740280778	0.00597483097353473\\
88.4697767278618	0.005645726602368\\
88.5197794276458	0.00506311604119508\\
88.5697821274298	0.00440143372962299\\
88.6197848272138	0.00386395829387582\\
88.6697875269979	0.00376686943519728\\
88.7197902267819	0.00443562269655846\\
88.7697929265659	0.00509179384860369\\
88.8197956263499	0.00489408742832628\\
88.8697983261339	0.0045704517534367\\
88.9198010259179	0.00470699967205324\\
88.969803725702	0.00452548325542936\\
89.019806425486	0.0046945306376808\\
89.06980912527	0.00500741712408223\\
89.119811825054	0.0052375347252177\\
89.169814524838	0.0053255827683078\\
89.219817224622	0.00536343111505793\\
89.2698199244061	0.00518547549966283\\
89.3198226241901	0.00481851024787414\\
89.3698253239741	0.00480767921814178\\
89.4198280237581	0.00476294109881334\\
89.4698307235421	0.0047289004136007\\
89.5198334233261	0.00484561571415186\\
89.5698361231102	0.00483838946767312\\
89.6198388228942	0.00469281112172522\\
89.6698415226782	0.00442127429495032\\
89.7198442224622	0.00458299225642242\\
89.7698469222462	0.00455964087958895\\
89.8198496220302	0.0043961344652495\\
89.8698523218143	0.0038891471054401\\
89.9198550215983	0.00355650045240237\\
89.9698577213823	0.00360313053698644\\
90.0198604211663	0.00388185816533716\\
90.0698631209503	0.00404547601205708\\
90.1198658207343	0.00398791813281693\\
90.1698685205184	0.00413541695858906\\
90.2198712203024	0.00427017546977268\\
90.2698739200864	0.00488698903273049\\
90.3198766198704	0.00526268239567088\\
90.3698793196544	0.00478797758899237\\
90.4198820194384	0.00294157309306403\\
90.4698847192225	0.00150843833897814\\
90.5198874190065	0.000742709661971891\\
90.5698901187905	0.00087552959325023\\
90.6198928185745	0.00152655883007551\\
90.6698955183585	0.00217370736264713\\
90.7198982181426	0.00288207520349131\\
90.7699009179266	0.00305294427411397\\
90.8199036177106	0.00334256250167673\\
90.8699063174946	0.00425227544946609\\
90.9199090172786	0.00477191882215444\\
90.9699117170626	0.00515125500376009\\
91.0199144168467	0.0056366180481516\\
91.0699171166307	0.00667959577381927\\
91.1199198164147	0.0073322011123369\\
91.1699225161987	0.00780138910052387\\
91.2199252159827	0.00789672627052344\\
91.2699279157667	0.00738978010240725\\
91.3199306155508	0.00739852858948746\\
91.3699333153348	0.00751375742176313\\
91.4199360151188	0.00749391645678491\\
91.4699387149028	0.00777444750897286\\
91.5199414146868	0.00818815406393797\\
91.5699441144708	0.00828143290751964\\
91.6199468142549	0.00830476184911301\\
91.6699495140389	0.00843438619274514\\
91.7199522138229	0.00866986886449887\\
91.7699549136069	0.00859065090246094\\
91.8199576133909	0.00868423634309281\\
91.8699603131749	0.00799196061403987\\
91.919963012959	0.00767013220219464\\
91.969965712743	0.0074958261159595\\
92.019968412527	0.00745080583998396\\
92.069971112311	0.00739513043255994\\
92.119973812095	0.00740409904751166\\
92.1699765118791	0.00702485917642045\\
92.2199792116631	0.00649975256353147\\
92.2699819114471	0.00651950206378776\\
92.3199846112311	0.00627858656317666\\
92.3699873110151	0.00617968769436338\\
92.4199900107991	0.00621938835006525\\
92.4699927105832	0.00579686477326045\\
92.5199954103672	0.00524848408128662\\
92.5699981101512	0.00525540557448442\\
92.605	0.00521397637557539\\
};
\addlegendentry{Measured angle [rad]};

\end{axis}
\end{tikzpicture}%
	}
	\caption{(a): Flight test of our implementation using the potentiometer as reference and sampling at 20Hz versus (b):
	Hermansen's implementation using the potentiometer as reference and sampling at 200Hz. (a) exhibits a shorter rise time and less
	overshoot on a step input and is never more than 1cm from the set point while (b) has an overshoot of slighly more than 1cm.}
	\label{fig:pottest}
	\vspace{3pt}
	\hrulefill
\end{figure*}
\begin{figure*}
	\centering
	\subfloat[][]{\setlength\figureheight{4cm}
		\setlength\figurewidth{6cm}
		% This file was created by matlab2tikz v0.4.7 running on MATLAB 8.0.
% Copyright (c) 2008--2014, Nico Schlömer <nico.schloemer@gmail.com>
% All rights reserved.
% Minimal pgfplots version: 1.3
% 
% The latest updates can be retrieved from
%   http://www.mathworks.com/matlabcentral/fileexchange/22022-matlab2tikz
% where you can also make suggestions and rate matlab2tikz.
% 
\begin{tikzpicture}

\begin{axis}[%
width=\figurewidth,
height=\figureheight,
scale only axis,
xmin=0,
xmax=92.605,
xlabel={time [s]},
ymin=-0.1,
ymax=0.25,
ylabel={distance [m] / angle [rad]},
axis x line*=bottom,
axis y line*=left,
legend style={at={(0.03,0.97)},anchor=north west,draw=black,fill=white,legend cell align=left},
%legend columns = 2
]
\addplot [color=black!50!green,solid,line width=0.2pt]
  table[row sep=crcr]{0	0\\
0.0500026997840173	0\\
0.100005399568035	0\\
0.150008099352052	0\\
0.200010799136069	0\\
0.250013498920086	0\\
0.300016198704104	0\\
0.350018898488121	0\\
0.400021598272138	0\\
0.450024298056156	0\\
0.500026997840173	0\\
0.55002969762419	0\\
0.600032397408207	0\\
0.650035097192225	0\\
0.700037796976242	0\\
0.750040496760259	0\\
0.800043196544276	0\\
0.850045896328294	0\\
0.900048596112311	0\\
0.950051295896328	0\\
1.00005399568035	0\\
1.05005669546436	0\\
1.10005939524838	0\\
1.1500620950324	0\\
1.20006479481641	0\\
1.25006749460043	0\\
1.30007019438445	0\\
1.35007289416847	0\\
1.40007559395248	0\\
1.4500782937365	0\\
1.50008099352052	0\\
1.55008369330454	0\\
1.60008639308855	0\\
1.65008909287257	0\\
1.70009179265659	0\\
1.7500944924406	0\\
1.80009719222462	0\\
1.85009989200864	0\\
1.90010259179266	0\\
1.95010529157667	0\\
2.00010799136069	0\\
2.05011069114471	0\\
2.10011339092873	0\\
2.15011609071274	0\\
2.20011879049676	0\\
2.25012149028078	0\\
2.30012419006479	0\\
2.35012688984881	0\\
2.40012958963283	0\\
2.45013228941685	0\\
2.50013498920086	0\\
2.55013768898488	0\\
2.6001403887689	0\\
2.65014308855292	0\\
2.70014578833693	0\\
2.75014848812095	0\\
2.80015118790497	0\\
2.85015388768899	0\\
2.900156587473	0\\
2.95015928725702	0\\
3.00016198704104	0\\
3.05016468682505	0\\
3.10016738660907	0\\
3.15017008639309	0\\
3.20017278617711	0\\
3.25017548596112	0\\
3.30017818574514	0\\
3.35018088552916	0\\
3.40018358531318	0\\
3.45018628509719	0\\
3.50018898488121	0\\
3.55019168466523	0\\
3.60019438444924	0\\
3.65019708423326	0\\
3.70019978401728	0\\
3.7502024838013	0\\
3.80020518358531	0\\
3.85020788336933	0\\
3.90021058315335	0\\
3.95021328293736	0\\
4.00021598272138	0\\
4.0502186825054	0\\
4.10022138228942	0\\
4.15022408207343	0\\
4.20022678185745	0\\
4.25022948164147	0\\
4.30023218142549	0\\
4.3502348812095	0\\
4.40023758099352	0\\
4.45024028077754	0\\
4.50024298056155	0\\
4.55024568034557	0\\
4.60024838012959	0\\
4.65025107991361	0\\
4.70025377969762	0\\
4.75025647948164	0\\
4.80025917926566	0\\
4.85026187904968	0\\
4.90026457883369	0\\
4.95026727861771	0\\
5.00026997840173	0\\
5.05027267818575	0\\
5.10027537796976	0\\
5.15027807775378	0\\
5.2002807775378	0\\
5.25028347732181	0\\
5.30028617710583	0\\
5.35028887688985	0\\
5.40029157667387	0\\
5.45029427645788	0\\
5.5002969762419	0\\
5.55029967602592	0\\
5.60030237580994	0\\
5.65030507559395	0\\
5.70030777537797	0\\
5.75031047516199	0\\
5.800313174946	0\\
5.85031587473002	0\\
5.90031857451404	0\\
5.95032127429806	0\\
6.00032397408207	0\\
6.05032667386609	0\\
6.10032937365011	0\\
6.15033207343413	0\\
6.20033477321814	0\\
6.25033747300216	0\\
6.30034017278618	0\\
6.3503428725702	0\\
6.40034557235421	0\\
6.45034827213823	0\\
6.50035097192225	0\\
6.55035367170626	0\\
6.60035637149028	0\\
6.6503590712743	0\\
6.70036177105832	0\\
6.75036447084233	0\\
6.80036717062635	0\\
6.85036987041037	0\\
6.90037257019439	0\\
6.9503752699784	0\\
7.00037796976242	0\\
7.05038066954644	0\\
7.10038336933045	0\\
7.15038606911447	0\\
7.20038876889849	0\\
7.25039146868251	0\\
7.30039416846652	0\\
7.35039686825054	0\\
7.40039956803456	0\\
7.45040226781857	0\\
7.50040496760259	0\\
7.55040766738661	0\\
7.60041036717063	0\\
7.65041306695464	0\\
7.70041576673866	0\\
7.75041846652268	0\\
7.8004211663067	0\\
7.85042386609071	0\\
7.90042656587473	0\\
7.95042926565875	0\\
8.00043196544276	0\\
8.05043466522678	0\\
8.1004373650108	0\\
8.15044006479482	0\\
8.20044276457883	0\\
8.25044546436285	0\\
8.30044816414687	0\\
8.35045086393089	0\\
8.4004535637149	0\\
8.45045626349892	0\\
8.50045896328294	0\\
8.55046166306696	0\\
8.60046436285097	0\\
8.65046706263499	0\\
8.70046976241901	0\\
8.75047246220302	0\\
8.80047516198704	0\\
8.85047786177106	0\\
8.90048056155508	0\\
8.95048326133909	0\\
9.00048596112311	0\\
9.05048866090713	0\\
9.10049136069114	0\\
9.15049406047516	0\\
9.20049676025918	0\\
9.2504994600432	0\\
9.30050215982721	0\\
9.35050485961123	0\\
9.40050755939525	0\\
9.45051025917927	0\\
9.50051295896328	0\\
9.5505156587473	0\\
9.60051835853132	0\\
9.65052105831533	0\\
9.70052375809935	0\\
9.75052645788337	0\\
9.80052915766739	0\\
9.8505318574514	0\\
9.90053455723542	0\\
9.95053725701944	0\\
10.0005399568035	0\\
10.0505426565875	0\\
10.1005453563715	0\\
10.1505480561555	0\\
10.2005507559395	0\\
10.2505534557235	0\\
10.3005561555076	0\\
10.3505588552916	0\\
10.4005615550756	0\\
10.4505642548596	0\\
10.5005669546436	0\\
10.5505696544276	0\\
10.6005723542117	0\\
10.6505750539957	0\\
10.7005777537797	0\\
10.7505804535637	0\\
10.8005831533477	0\\
10.8505858531317	0\\
10.9005885529158	0\\
10.9505912526998	0\\
11.0005939524838	0\\
11.0505966522678	0\\
11.1005993520518	0\\
11.1506020518359	0\\
11.2006047516199	0\\
11.2506074514039	0\\
11.3006101511879	0\\
11.3506128509719	0\\
11.4006155507559	0\\
11.45061825054	0\\
11.500620950324	0\\
11.550623650108	0\\
11.600626349892	0\\
11.650629049676	0\\
11.70063174946	0\\
11.7506344492441	0\\
11.8006371490281	0\\
11.8506398488121	0\\
11.9006425485961	0\\
11.9506452483801	0\\
12.0006479481641	0\\
12.0506506479482	0\\
12.1006533477322	0\\
12.1506560475162	0\\
12.2006587473002	0\\
12.2506614470842	0\\
12.3006641468683	0\\
12.3506668466523	0\\
12.4006695464363	0\\
12.4506722462203	0\\
12.5006749460043	0\\
12.5506776457883	0\\
12.6006803455724	0\\
12.6506830453564	0\\
12.7006857451404	0\\
12.7506884449244	0\\
12.8006911447084	0\\
12.8506938444924	0\\
12.9006965442765	0\\
12.9506992440605	0\\
13.0007019438445	0\\
13.0507046436285	0\\
13.1007073434125	0\\
13.1507100431965	0\\
13.2007127429806	0\\
13.2507154427646	0\\
13.3007181425486	0\\
13.3507208423326	0\\
13.4007235421166	0\\
13.4507262419006	0\\
13.5007289416847	0\\
13.5507316414687	0\\
13.6007343412527	0\\
13.6507370410367	0\\
13.7007397408207	0\\
13.7507424406048	0\\
13.8007451403888	0\\
13.8507478401728	0\\
13.9007505399568	0\\
13.9507532397408	0\\
14.0007559395248	0\\
14.0507586393089	0\\
14.1007613390929	0\\
14.1507640388769	0\\
14.2007667386609	0\\
14.2507694384449	0\\
14.3007721382289	0\\
14.350774838013	0\\
14.400777537797	0\\
14.450780237581	0\\
14.500782937365	0\\
14.550785637149	0\\
14.600788336933	0\\
14.6507910367171	0\\
14.7007937365011	0\\
14.7507964362851	0\\
14.8007991360691	0\\
14.8508018358531	0\\
14.9008045356372	0\\
14.9508072354212	0\\
15.0008099352052	0\\
15.0508126349892	0\\
15.1008153347732	0\\
15.1508180345572	0\\
15.2008207343413	0\\
15.2508234341253	0\\
15.3008261339093	0\\
15.3508288336933	0\\
15.4008315334773	0\\
15.4508342332613	0\\
15.5008369330454	0\\
15.5508396328294	0\\
15.6008423326134	0\\
15.6508450323974	0\\
15.7008477321814	0\\
15.7508504319654	0\\
15.8008531317495	0\\
15.8508558315335	0\\
15.9008585313175	0\\
15.9508612311015	0\\
16.0008639308855	0\\
16.0508666306695	0\\
16.1008693304536	0\\
16.1508720302376	0\\
16.2008747300216	0\\
16.2508774298056	0\\
16.3008801295896	0\\
16.3508828293737	0\\
16.4008855291577	0\\
16.4508882289417	0\\
16.5008909287257	0\\
16.5508936285097	0\\
16.6008963282937	0\\
16.6508990280778	0\\
16.7009017278618	0\\
16.7509044276458	0\\
16.8009071274298	0\\
16.8509098272138	0\\
16.9009125269978	0\\
16.9509152267819	0\\
17.0009179265659	0\\
17.0509206263499	0\\
17.1009233261339	0\\
17.1509260259179	0\\
17.2009287257019	0\\
17.250931425486	0\\
17.30093412527	0\\
17.350936825054	0\\
17.400939524838	0\\
17.450942224622	0\\
17.500944924406	0\\
17.5509476241901	0\\
17.6009503239741	0\\
17.6509530237581	0\\
17.7009557235421	0\\
17.7509584233261	0\\
17.8009611231102	0\\
17.8509638228942	0\\
17.9009665226782	0\\
17.9509692224622	0\\
18.0009719222462	0\\
18.0509746220302	0\\
18.1009773218143	0\\
18.1509800215983	0\\
18.2009827213823	0\\
18.2509854211663	0\\
18.3009881209503	0\\
18.3509908207343	0\\
18.4009935205184	0\\
18.4509962203024	0\\
18.5009989200864	0\\
18.5510016198704	0\\
18.6010043196544	0\\
18.6510070194384	0\\
18.7010097192225	0\\
18.7510124190065	0\\
18.8010151187905	0\\
18.8510178185745	0\\
18.9010205183585	0\\
18.9510232181425	0\\
19.0010259179266	0\\
19.0510286177106	0\\
19.1010313174946	0\\
19.1510340172786	0\\
19.2010367170626	0\\
19.2510394168467	0\\
19.3010421166307	0\\
19.3510448164147	0\\
19.4010475161987	0\\
19.4510502159827	0\\
19.5010529157667	0\\
19.5510556155508	0\\
19.6010583153348	0\\
19.6510610151188	0\\
19.7010637149028	0\\
19.7510664146868	0\\
19.8010691144708	0\\
19.8510718142549	0\\
19.9010745140389	0\\
19.9510772138229	0\\
20.0010799136069	0\\
20.0510826133909	0\\
20.1010853131749	0\\
20.151088012959	0\\
20.201090712743	0\\
20.251093412527	0\\
20.301096112311	0\\
20.351098812095	0\\
20.4011015118791	0\\
20.4511042116631	0\\
20.5011069114471	0\\
20.5511096112311	0\\
20.6011123110151	0\\
20.6511150107991	0\\
20.7011177105832	0\\
20.7511204103672	0\\
20.8011231101512	0\\
20.8511258099352	0\\
20.9011285097192	0\\
20.9511312095032	0\\
21.0011339092873	0\\
21.0511366090713	0\\
21.1011393088553	0\\
21.1511420086393	0\\
21.2011447084233	0\\
21.2511474082073	0\\
21.3011501079914	0\\
21.3511528077754	0\\
21.4011555075594	0\\
21.4511582073434	0\\
21.5011609071274	0\\
21.5511636069114	0\\
21.6011663066955	0\\
21.6511690064795	0\\
21.7011717062635	0\\
21.7511744060475	0\\
21.8011771058315	0\\
21.8511798056156	0\\
21.9011825053996	0\\
21.9511852051836	0\\
22.0011879049676	0\\
22.0511906047516	0\\
22.1011933045356	0\\
22.1511960043197	0\\
22.2011987041037	0\\
22.2512014038877	0\\
22.3012041036717	0\\
22.3512068034557	0\\
22.4012095032397	0\\
22.4512122030238	0\\
22.5012149028078	0\\
22.5512176025918	0\\
22.6012203023758	0\\
22.6512230021598	0\\
22.7012257019438	0\\
22.7512284017279	0\\
22.8012311015119	0\\
22.8512338012959	0\\
22.9012365010799	0\\
22.9512392008639	0\\
23.0012419006479	0\\
23.051244600432	0\\
23.101247300216	0\\
23.15125	0\\
23.201252699784	0\\
23.251255399568	0\\
23.3012580993521	0\\
23.3512607991361	0\\
23.4012634989201	0\\
23.4512661987041	0\\
23.5012688984881	0\\
23.5512715982721	0\\
23.6012742980562	0\\
23.6512769978402	0\\
23.7012796976242	0\\
23.7512823974082	0\\
23.8012850971922	0\\
23.8512877969762	0\\
23.9012904967603	0\\
23.9512931965443	0\\
24.0012958963283	0\\
24.0512985961123	0\\
24.1013012958963	0\\
24.1513039956803	0\\
24.2013066954644	0\\
24.2513093952484	0\\
24.3013120950324	0\\
24.3513147948164	0\\
24.4013174946004	0\\
24.4513201943845	0\\
24.5013228941685	0\\
24.5513255939525	0\\
24.6013282937365	0\\
24.6513309935205	0\\
24.7013336933045	0\\
24.7513363930886	0\\
24.8013390928726	0\\
24.8513417926566	0\\
24.9013444924406	0\\
24.9513471922246	0\\
25.0013498920086	0\\
25.0513525917927	0\\
25.1013552915767	0\\
25.1513579913607	0\\
25.2013606911447	0\\
25.2513633909287	0\\
25.3013660907127	0\\
25.3513687904968	0\\
25.4013714902808	0\\
25.4513741900648	0\\
25.5013768898488	0\\
25.5513795896328	0\\
25.6013822894168	0\\
25.6513849892009	0\\
25.7013876889849	0\\
25.7513903887689	0\\
25.8013930885529	0\\
25.8513957883369	0\\
25.901398488121	0\\
25.951401187905	0\\
26.001403887689	0\\
26.051406587473	0\\
26.101409287257	0\\
26.151411987041	0\\
26.2014146868251	0\\
26.2514173866091	0\\
26.3014200863931	0\\
26.3514227861771	0\\
26.4014254859611	0\\
26.4514281857451	0\\
26.5014308855292	0\\
26.5514335853132	0\\
26.6014362850972	0\\
26.6514389848812	0\\
26.7014416846652	0\\
26.7514443844492	0\\
26.8014470842333	0\\
26.8514497840173	0\\
26.9014524838013	0\\
26.9514551835853	0\\
27.0014578833693	0\\
27.0514605831534	0\\
27.1014632829374	0\\
27.1514659827214	0\\
27.2014686825054	0\\
27.2514713822894	0\\
27.3014740820734	0\\
27.3514767818575	0\\
27.4014794816415	0\\
27.4514821814255	0\\
27.5014848812095	0\\
27.5514875809935	0\\
27.6014902807775	0\\
27.6514929805616	0\\
27.7014956803456	0\\
27.7514983801296	0\\
27.8015010799136	0\\
27.8515037796976	0\\
27.9015064794816	0\\
27.9515091792657	0\\
28.0015118790497	0\\
28.0515145788337	0\\
28.1015172786177	0\\
28.1515199784017	0\\
28.2015226781857	0\\
28.2515253779698	0\\
28.3015280777538	0\\
28.3515307775378	0\\
28.4015334773218	0\\
28.4515361771058	0\\
28.5015388768899	0\\
28.5515415766739	0\\
28.6015442764579	0\\
28.6515469762419	0\\
28.7015496760259	0\\
28.7515523758099	0\\
28.801555075594	0\\
28.851557775378	0\\
28.901560475162	0\\
28.951563174946	0\\
29.00156587473	0\\
29.051568574514	0\\
29.1015712742981	0\\
29.1515739740821	0\\
29.2015766738661	0\\
29.2515793736501	0\\
29.3015820734341	0\\
29.3515847732181	0\\
29.4015874730022	0\\
29.4515901727862	0\\
29.5015928725702	0\\
29.5515955723542	0\\
29.6015982721382	0\\
29.6516009719222	0\\
29.7016036717063	0\\
29.7516063714903	0\\
29.8016090712743	0\\
29.8516117710583	0\\
29.9016144708423	0\\
29.9516171706264	0\\
30.0016198704104	0\\
30.0516225701944	0\\
30.1016252699784	0\\
30.1516279697624	0\\
30.2016306695464	0\\
30.2516333693305	0\\
30.3016360691145	0\\
30.3516387688985	0\\
30.4016414686825	0\\
30.4516441684665	0\\
30.5016468682505	0\\
30.5516495680346	0\\
30.6016522678186	0\\
30.6516549676026	0\\
30.7016576673866	0\\
30.7516603671706	0\\
30.8016630669546	0\\
30.8516657667387	0\\
30.9016684665227	0\\
30.9516711663067	0\\
31.0016738660907	0\\
31.0516765658747	0\\
31.1016792656587	0\\
31.1516819654428	0\\
31.2016846652268	0\\
31.2516873650108	0\\
31.3016900647948	0\\
31.3516927645788	0\\
31.4016954643629	0\\
31.4516981641469	0\\
31.5017008639309	0\\
31.5517035637149	0\\
31.6017062634989	0\\
31.6517089632829	0\\
31.701711663067	0\\
31.751714362851	0\\
31.801717062635	0\\
31.851719762419	0\\
31.901722462203	0\\
31.951725161987	0\\
32.0017278617711	0\\
32.0517305615551	0\\
32.1017332613391	0\\
32.1517359611231	0\\
32.2017386609071	0\\
32.2517413606911	0\\
32.3017440604752	0\\
32.3517467602592	0\\
32.4017494600432	0\\
32.4517521598272	0\\
32.5017548596112	0\\
32.5517575593952	0\\
32.6017602591793	0\\
32.6517629589633	0\\
32.7017656587473	0\\
32.7517683585313	0\\
32.8017710583153	0\\
32.8517737580993	0\\
32.9017764578834	0\\
32.9517791576674	0\\
33.0017818574514	0\\
33.0517845572354	0\\
33.1017872570194	0\\
33.1517899568035	0\\
33.2017926565875	0\\
33.2517953563715	0\\
33.3017980561555	0\\
33.3518007559395	0\\
33.4018034557235	0\\
33.4518061555076	0\\
33.5018088552916	0\\
33.5518115550756	0\\
33.6018142548596	0\\
33.6518169546436	0\\
33.7018196544277	0\\
33.7518223542117	0\\
33.8018250539957	0\\
33.8518277537797	0\\
33.9018304535637	0\\
33.9518331533477	0\\
34.0018358531318	0\\
34.0518385529158	0\\
34.1018412526998	0\\
34.1518439524838	0\\
34.2018466522678	0\\
34.2518493520518	0\\
34.3018520518359	0\\
34.3518547516199	0\\
34.4018574514039	0\\
34.4518601511879	0\\
34.5018628509719	0\\
34.5518655507559	0\\
34.60186825054	0\\
34.651870950324	0\\
34.701873650108	0\\
34.751876349892	0\\
34.801879049676	0\\
34.85188174946	0\\
34.9018844492441	0\\
34.9518871490281	0\\
35.0018898488121	0\\
35.0518925485961	0\\
35.1018952483801	0\\
35.1518979481641	0\\
35.2019006479482	0\\
35.2519033477322	0\\
35.3019060475162	0\\
35.3519087473002	0\\
35.4019114470842	0\\
35.4519141468683	0\\
35.5019168466523	0\\
35.5519195464363	0\\
35.6019222462203	0\\
35.6519249460043	0\\
35.7019276457883	0\\
35.7519303455724	0\\
35.8019330453564	0\\
35.8519357451404	0\\
35.9019384449244	0\\
35.9519411447084	0\\
36.0019438444924	0\\
36.0519465442765	0\\
36.1019492440605	0\\
36.1519519438445	0\\
36.2019546436285	0\\
36.2519573434125	0\\
36.3019600431965	0\\
36.3519627429806	0\\
36.4019654427646	0\\
36.4519681425486	0\\
36.5019708423326	0\\
36.5519735421166	0\\
36.6019762419006	0\\
36.6519789416847	0\\
36.7019816414687	0\\
36.7519843412527	0\\
36.8019870410367	0\\
36.8519897408207	0\\
36.9019924406048	0\\
36.9519951403888	0\\
37.0019978401728	0\\
37.0520005399568	0\\
37.1020032397408	0\\
37.1520059395248	0\\
37.2020086393089	0\\
37.2520113390929	0\\
37.3020140388769	0\\
37.3520167386609	0\\
37.4020194384449	0\\
37.4520221382289	0\\
37.502024838013	0\\
37.552027537797	0\\
37.602030237581	0\\
37.652032937365	0\\
37.702035637149	0\\
37.752038336933	0\\
37.8020410367171	0\\
37.8520437365011	0\\
37.9020464362851	0\\
37.9520491360691	0\\
38.0020518358531	0\\
38.0520545356372	0\\
38.1020572354212	0\\
38.1520599352052	0\\
38.2020626349892	0\\
38.2520653347732	0\\
38.3020680345572	0\\
38.3520707343413	0\\
38.4020734341253	0\\
38.4520761339093	0\\
38.5020788336933	0\\
38.5520815334773	0\\
38.6020842332613	0\\
38.6520869330454	0\\
38.7020896328294	0\\
38.7520923326134	0\\
38.8020950323974	0\\
38.8520977321814	0\\
38.9021004319654	0\\
38.9521031317495	0\\
39.0021058315335	0\\
39.0521085313175	0\\
39.1021112311015	0\\
39.1521139308855	0\\
39.2021166306696	0\\
39.2521193304536	0\\
39.3021220302376	0\\
39.3521247300216	0\\
39.4021274298056	0\\
39.4521301295896	0\\
39.5021328293737	0\\
39.5521355291577	0\\
39.6021382289417	0\\
39.6521409287257	0\\
39.7021436285097	0\\
39.7521463282937	0\\
39.8021490280778	0\\
39.8521517278618	0\\
39.9021544276458	0\\
39.9521571274298	0\\
40.0021598272138	0\\
40.0521625269978	0\\
40.1021652267819	0\\
40.1521679265659	0\\
40.2021706263499	0\\
40.2521733261339	0\\
40.3021760259179	0\\
40.3521787257019	0\\
40.402181425486	0\\
40.45218412527	0\\
40.502186825054	0\\
40.552189524838	0\\
40.602192224622	0\\
40.652194924406	0\\
40.7021976241901	0\\
40.7522003239741	0\\
40.8022030237581	0\\
40.8522057235421	0\\
40.9022084233261	0\\
40.9522111231102	0\\
41.0022138228942	0\\
41.0522165226782	0\\
41.1022192224622	0\\
41.1522219222462	0\\
41.2022246220302	0\\
41.2522273218143	0\\
41.3022300215983	0\\
41.3522327213823	0\\
41.4022354211663	0\\
41.4522381209503	0\\
41.5022408207343	0\\
41.5522435205184	0\\
41.6022462203024	0\\
41.6522489200864	0\\
41.7022516198704	0\\
41.7522543196544	0\\
41.8022570194385	0\\
41.8522597192225	0\\
41.9022624190065	0\\
41.9522651187905	0\\
42.0022678185745	0\\
42.0522705183585	0\\
42.1022732181426	0\\
42.1522759179266	0\\
42.2022786177106	0\\
42.2522813174946	0\\
42.3022840172786	0\\
42.3522867170626	0\\
42.4022894168467	0\\
42.4522921166307	0\\
42.5022948164147	0\\
42.5522975161987	0\\
42.6023002159827	0\\
42.6523029157667	0\\
42.7023056155508	0\\
42.7523083153348	0\\
42.8023110151188	0\\
42.8523137149028	0\\
42.9023164146868	0\\
42.9523191144708	0\\
43.0023218142549	0\\
43.0523245140389	0\\
43.1023272138229	0\\
43.1523299136069	0\\
43.2023326133909	0\\
43.2523353131749	0\\
43.302338012959	0\\
43.352340712743	0\\
43.402343412527	0\\
43.452346112311	0\\
43.502348812095	0\\
43.5523515118791	0\\
43.6023542116631	0\\
43.6523569114471	0\\
43.7023596112311	0\\
43.7523623110151	0\\
43.8023650107991	0\\
43.8523677105832	0\\
43.9023704103672	0\\
43.9523731101512	0\\
44.0023758099352	0\\
44.0523785097192	0\\
44.1023812095032	0\\
44.1523839092873	0\\
44.2023866090713	0\\
44.2523893088553	0\\
44.3023920086393	0\\
44.3523947084233	0\\
44.4023974082073	0\\
44.4524001079914	0\\
44.5024028077754	0\\
44.5524055075594	0\\
44.6024082073434	0\\
44.6524109071274	0\\
44.7024136069114	0\\
44.7524163066955	0\\
44.8024190064795	0\\
44.8524217062635	0\\
44.9024244060475	0\\
44.9524271058315	0\\
45.0024298056155	0\\
45.0524325053996	0\\
45.1024352051836	0\\
45.1524379049676	0\\
45.2024406047516	0\\
45.2524433045356	0\\
45.3024460043197	0\\
45.3524487041037	0\\
45.4024514038877	0\\
45.4524541036717	0\\
45.5024568034557	0\\
45.5524595032397	0\\
45.6024622030238	0\\
45.6524649028078	0\\
45.7024676025918	0\\
45.7524703023758	0\\
45.8024730021598	0\\
45.8524757019439	0\\
45.9024784017279	0\\
45.9524811015119	0\\
46.0024838012959	0\\
46.0524865010799	0\\
46.1024892008639	0\\
46.152491900648	0\\
46.202494600432	0\\
46.252497300216	0\\
46.3025	0\\
46.352502699784	0\\
46.402505399568	0\\
46.4525080993521	0\\
46.5025107991361	0\\
46.5525134989201	0\\
46.6025161987041	0\\
46.6525188984881	0\\
46.7025215982721	0\\
46.7525242980562	0\\
46.8025269978402	0\\
46.8525296976242	0\\
46.9025323974082	0\\
46.9525350971922	0\\
47.0025377969762	0\\
47.0525404967603	0\\
47.1025431965443	0\\
47.1525458963283	0\\
47.2025485961123	0\\
47.2525512958963	0\\
47.3025539956803	0\\
47.3525566954644	0\\
47.4025593952484	0\\
47.4525620950324	0\\
47.5025647948164	0\\
47.5525674946004	0\\
47.6025701943845	0\\
47.6525728941685	0\\
47.7025755939525	0\\
47.7525782937365	0\\
47.8025809935205	0\\
47.8525836933045	0\\
47.9025863930886	0\\
47.9525890928726	0\\
48.0025917926566	0\\
48.0525944924406	0\\
48.1025971922246	0\\
48.1525998920086	0\\
48.2026025917927	0\\
48.2526052915767	0\\
48.3026079913607	0\\
48.3526106911447	0\\
48.4026133909287	0\\
48.4526160907127	0\\
48.5026187904968	0\\
48.5526214902808	0\\
48.6026241900648	0\\
48.6526268898488	0\\
48.7026295896328	0\\
48.7526322894168	0\\
48.8026349892009	0\\
48.8526376889849	0\\
48.9026403887689	0\\
48.9526430885529	0\\
49.0026457883369	0\\
49.052648488121	0\\
49.102651187905	0\\
49.152653887689	0\\
49.202656587473	0\\
49.252659287257	0\\
49.302661987041	0\\
49.3526646868251	0\\
49.4026673866091	0\\
49.4526700863931	0\\
49.5026727861771	0\\
49.5526754859611	0\\
49.6026781857451	0\\
49.6526808855292	0\\
49.7026835853132	0\\
49.7526862850972	0\\
49.8026889848812	0\\
49.8526916846652	0\\
49.9026943844492	0\\
49.9526970842333	0\\
50.0026997840173	0\\
50.0527024838013	0\\
50.1027051835853	0\\
50.1527078833693	0\\
50.2027105831534	0\\
50.2527132829374	0\\
50.3027159827214	0\\
50.3527186825054	0\\
50.4027213822894	0\\
50.4527240820734	0\\
50.5027267818575	0\\
50.5527294816415	0\\
50.6027321814255	0\\
50.6527348812095	0\\
50.7027375809935	0\\
50.7527402807775	0\\
50.8027429805616	0\\
50.8527456803456	0\\
50.9027483801296	0\\
50.9527510799136	0\\
51.0027537796976	0\\
51.0527564794816	0\\
51.1027591792657	0\\
51.1527618790497	0\\
51.2027645788337	0\\
51.2527672786177	0\\
51.3027699784017	0\\
51.3527726781858	0\\
51.4027753779698	0\\
51.4527780777538	0\\
51.5027807775378	0\\
51.5527834773218	0\\
51.6027861771058	0\\
51.6527888768899	0\\
51.7027915766739	0\\
51.7527942764579	0\\
51.8027969762419	0\\
51.8527996760259	0\\
51.9028023758099	0\\
51.952805075594	0\\
52.002807775378	0\\
52.052810475162	0\\
52.102813174946	0\\
52.15281587473	0\\
52.202818574514	0\\
52.2528212742981	0\\
52.3028239740821	0\\
52.3528266738661	0\\
52.4028293736501	0\\
52.4528320734341	0\\
52.5028347732181	0\\
52.5528374730022	0\\
52.6028401727862	0\\
52.6528428725702	0\\
52.7028455723542	0\\
52.7528482721382	0\\
52.8028509719222	0\\
52.8528536717063	0\\
52.9028563714903	0\\
52.9528590712743	0\\
53.0028617710583	0\\
53.0528644708423	0\\
53.1028671706264	0\\
53.1528698704104	0\\
53.2028725701944	0\\
53.2528752699784	0\\
53.3028779697624	0\\
53.3528806695464	0\\
53.4028833693305	0\\
53.4528860691145	0\\
53.5028887688985	0\\
53.5528914686825	0\\
53.6028941684665	0\\
53.6528968682505	0\\
53.7028995680346	0\\
53.7529022678186	0\\
53.8029049676026	0\\
53.8529076673866	0\\
53.9029103671706	0\\
53.9529130669547	0\\
54.0029157667387	0\\
54.0529184665227	0\\
54.1029211663067	0\\
54.1529238660907	0\\
54.2029265658747	0\\
54.2529292656588	0\\
54.3029319654428	0\\
54.3529346652268	0\\
54.4029373650108	0\\
54.4529400647948	0\\
54.5029427645788	0\\
54.5529454643629	0\\
54.6029481641469	0\\
54.6529508639309	0\\
54.7029535637149	0\\
54.7529562634989	0\\
54.8029589632829	0\\
54.852961663067	0\\
54.902964362851	0\\
54.952967062635	0\\
55.002969762419	0\\
55.052972462203	0\\
55.102975161987	0\\
55.1529778617711	0\\
55.2029805615551	0\\
55.2529832613391	0\\
55.3029859611231	0\\
55.3529886609071	0\\
55.4029913606911	0\\
55.4529940604752	0\\
55.5029967602592	0\\
55.5529994600432	0\\
55.6030021598272	0\\
55.6530048596112	0\\
55.7030075593953	0\\
55.7530102591793	0\\
55.8030129589633	0\\
55.8530156587473	0\\
55.9030183585313	0\\
55.9530210583153	0\\
56.0030237580994	0\\
56.0530264578834	0\\
56.1030291576674	0\\
56.1530318574514	0\\
56.2030345572354	0\\
56.2530372570194	0\\
56.3030399568035	0\\
56.3530426565875	0\\
56.4030453563715	0\\
56.4530480561555	0\\
56.5030507559395	0\\
56.5530534557235	0\\
56.6030561555076	0\\
56.6530588552916	0\\
56.7030615550756	0\\
56.7530642548596	0\\
56.8030669546436	0\\
56.8530696544276	0\\
56.9030723542117	0\\
56.9530750539957	0\\
57.0030777537797	0\\
57.0530804535637	0\\
57.1030831533477	0\\
57.1530858531318	0\\
57.2030885529158	0\\
57.2530912526998	0\\
57.3030939524838	0\\
57.3530966522678	0\\
57.4030993520518	0\\
57.4531020518358	0\\
57.5031047516199	0\\
57.5531074514039	0\\
57.6031101511879	0\\
57.6531128509719	0\\
57.7031155507559	0\\
57.75311825054	0\\
57.803120950324	0\\
57.853123650108	0\\
57.903126349892	0\\
57.953129049676	0\\
58.00313174946	0\\
58.0531344492441	0\\
58.1031371490281	0\\
58.1531398488121	0\\
58.2031425485961	0\\
58.2531452483801	0\\
58.3031479481642	0\\
58.3531506479482	0\\
58.4031533477322	0\\
58.4531560475162	0\\
58.5031587473002	0\\
58.5531614470842	0\\
58.6031641468683	0\\
58.6531668466523	0\\
58.7031695464363	0\\
58.7531722462203	0\\
58.8031749460043	0\\
58.8531776457883	0\\
58.9031803455724	0\\
58.9531830453564	0\\
59.0031857451404	0\\
59.0531884449244	0\\
59.1031911447084	0\\
59.1531938444924	0\\
59.2031965442765	0\\
59.2531992440605	0\\
59.3032019438445	0\\
59.3532046436285	0\\
59.4032073434125	0\\
59.4532100431965	0\\
59.5032127429806	0\\
59.5532154427646	0\\
59.6032181425486	0\\
59.6532208423326	0\\
59.7032235421166	0\\
59.7532262419006	0\\
59.8032289416847	0\\
59.8532316414687	0\\
59.9032343412527	0\\
59.9532370410367	0\\
60.0032397408207	0\\
60.0532424406048	0\\
60.1032451403888	0\\
60.1532478401728	0\\
60.2032505399568	0\\
60.2532532397408	0\\
60.3032559395248	0\\
60.3532586393089	0\\
60.4032613390929	0\\
60.4532640388769	0\\
60.5032667386609	0\\
60.5532694384449	0\\
60.6032721382289	0\\
60.653274838013	0\\
60.703277537797	0\\
60.753280237581	0\\
60.803282937365	0\\
60.853285637149	0\\
60.9032883369331	0\\
60.9532910367171	0\\
61.0032937365011	0\\
61.0532964362851	0\\
61.1032991360691	0\\
61.1533018358531	0\\
61.2033045356371	0\\
61.2533072354212	0\\
61.3033099352052	0\\
61.3533126349892	0\\
61.4033153347732	0\\
61.4533180345572	0\\
61.5033207343413	0\\
61.5533234341253	0\\
61.6033261339093	0\\
61.6533288336933	0\\
61.7033315334773	0\\
61.7533342332613	0\\
61.8033369330454	0\\
61.8533396328294	0\\
61.9033423326134	0\\
61.9533450323974	0\\
62.0033477321814	0\\
62.0533504319654	0\\
62.1033531317495	0\\
62.1533558315335	0\\
62.2033585313175	0\\
62.2533612311015	0\\
62.3033639308855	0\\
62.3533666306695	0\\
62.4033693304536	0\\
62.4533720302376	0\\
62.5033747300216	0\\
62.5483771598272	0.2\\
62.5983798596112	0.2\\
62.6483825593953	0.2\\
62.6983852591793	0.2\\
62.7483879589633	0.2\\
62.7983906587473	0.2\\
62.8483933585313	0.2\\
62.8983960583153	0.2\\
62.9483987580994	0.2\\
62.9984014578834	0.2\\
63.0484041576674	0.2\\
63.0984068574514	0.2\\
63.1484095572354	0.2\\
63.1984122570194	0.2\\
63.2484149568035	0.2\\
63.2984176565875	0.2\\
63.3484203563715	0.2\\
63.3984230561555	0.2\\
63.4484257559395	0.2\\
63.4984284557235	0.2\\
63.5484311555076	0.2\\
63.5984338552916	0.2\\
63.6484365550756	0.2\\
63.6984392548596	0.2\\
63.7484419546436	0.2\\
63.7984446544276	0.2\\
63.8484473542117	0.2\\
63.8984500539957	0.2\\
63.9484527537797	0.2\\
63.9984554535637	0.2\\
64.0484581533477	0.2\\
64.0984608531318	0.2\\
64.1484635529158	0.2\\
64.1984662526998	0.2\\
64.2484689524838	0.2\\
64.2984716522678	0.2\\
64.3484743520518	0.2\\
64.3984770518359	0.2\\
64.4484797516199	0.2\\
64.4984824514039	0.2\\
64.5484851511879	0.2\\
64.5984878509719	0.2\\
64.648490550756	0.2\\
64.69849325054	0.2\\
64.748495950324	0.2\\
64.798498650108	0.2\\
64.848501349892	0.2\\
64.898504049676	0.2\\
64.94850674946	0.2\\
64.9985094492441	0.2\\
65.0485121490281	0.2\\
65.0985148488121	0.2\\
65.1485175485961	0.2\\
65.1985202483801	0.2\\
65.2485229481642	0.2\\
65.2985256479482	0.2\\
65.3485283477322	0.2\\
65.3985310475162	0.2\\
65.4485337473002	0.2\\
65.4985364470842	0.2\\
65.5485391468683	0.2\\
65.5985418466523	0.2\\
65.6485445464363	0.2\\
65.6985472462203	0.2\\
65.7485499460043	0.2\\
65.7985526457883	0.2\\
65.8485553455724	0.2\\
65.8985580453564	0.2\\
65.9485607451404	0.2\\
65.9985634449244	0.2\\
66.0485661447084	0.2\\
66.0985688444924	0.2\\
66.1485715442765	0.2\\
66.1985742440605	0.2\\
66.2485769438445	0.2\\
66.2985796436285	0.2\\
66.3485823434125	0.2\\
66.3985850431965	0.2\\
66.4485877429806	0.2\\
66.4985904427646	0.2\\
66.5485931425486	0.2\\
66.5985958423326	0.2\\
66.6485985421166	0.2\\
66.6986012419006	0.2\\
66.7486039416847	0.2\\
66.7986066414687	0.2\\
66.8486093412527	0.2\\
66.8986120410367	0.2\\
66.9486147408207	0.2\\
66.9986174406047	0.2\\
67.0486201403888	0.2\\
67.0986228401728	0.2\\
67.1486255399568	0.2\\
67.1986282397408	0.2\\
67.2486309395248	0.2\\
67.2986336393089	0.2\\
67.3486363390929	0.2\\
67.3986390388769	0.2\\
67.4486417386609	0.2\\
67.4986444384449	0.2\\
67.5486471382289	0.2\\
67.598649838013	0.2\\
67.648652537797	0.2\\
67.698655237581	0.2\\
67.748657937365	0.2\\
67.798660637149	0.2\\
67.8486633369331	0.2\\
67.8986660367171	0.2\\
67.9486687365011	0.2\\
67.9986714362851	0.2\\
68.0486741360691	0.2\\
68.0986768358531	0.2\\
68.1486795356372	0.2\\
68.1986822354212	0.2\\
68.2486849352052	0.2\\
68.2986876349892	0.2\\
68.3486903347732	0.2\\
68.3986930345572	0.2\\
68.4486957343413	0.2\\
68.4986984341253	0.2\\
68.5487011339093	0.2\\
68.5987038336933	0.2\\
68.6487065334773	0.2\\
68.6987092332613	0.2\\
68.7487119330454	0.2\\
68.7987146328294	0.2\\
68.8487173326134	0.2\\
68.8987200323974	0.2\\
68.9487227321814	0.2\\
68.9987254319654	0.2\\
69.0487281317495	0.2\\
69.0987308315335	0.2\\
69.1487335313175	0.2\\
69.1987362311015	0.2\\
69.2487389308855	0.2\\
69.2987416306696	0.2\\
69.3487443304536	0.2\\
69.3987470302376	0.2\\
69.4487497300216	0.2\\
69.4987524298056	0.2\\
69.5487551295896	0.2\\
69.5987578293737	0.2\\
69.6487605291577	0.2\\
69.6987632289417	0.2\\
69.7487659287257	0.2\\
69.7987686285097	0.2\\
69.8487713282937	0.2\\
69.8987740280778	0.2\\
69.9487767278618	0.2\\
69.9987794276458	0.2\\
70.0487821274298	0.2\\
70.0987848272138	0.2\\
70.1487875269979	0.2\\
70.1987902267819	0.2\\
70.2487929265659	0.2\\
70.2987956263499	0.2\\
70.3487983261339	0.2\\
70.3988010259179	0.2\\
70.4488037257019	0.2\\
70.498806425486	0.2\\
70.54880912527	0.2\\
70.598811825054	0.2\\
70.648814524838	0.2\\
70.698817224622	0.2\\
70.7488199244061	0.2\\
70.7988226241901	0.2\\
70.8488253239741	0.2\\
70.8988280237581	0.2\\
70.9488307235421	0.2\\
70.9988334233261	0.2\\
71.0488361231101	0.2\\
71.0988388228942	0.2\\
71.1488415226782	0.2\\
71.1988442224622	0.2\\
71.2488469222462	0.2\\
71.2988496220302	0.2\\
71.3488523218143	0.2\\
71.3988550215983	0.2\\
71.4488577213823	0.2\\
71.4988604211663	0.2\\
71.5488631209503	0.2\\
71.5988658207343	0.2\\
71.6488685205184	0.2\\
71.6988712203024	0.2\\
71.7488739200864	0.2\\
71.7988766198704	0.2\\
71.8488793196544	0.2\\
71.8988820194385	0.2\\
71.9488847192225	0.2\\
71.9988874190065	0.2\\
72.0488901187905	0.2\\
72.0988928185745	0.2\\
72.1488955183585	0.2\\
72.1988982181425	0.2\\
72.2489009179266	0.2\\
72.2989036177106	0.2\\
72.3489063174946	0.2\\
72.3989090172786	0.2\\
72.4489117170626	0.2\\
72.4989144168467	0.2\\
72.5489171166307	0.2\\
72.5989198164147	0.2\\
72.6489225161987	0.2\\
72.6989252159827	0.2\\
72.7489279157667	0.2\\
72.7989306155508	0.2\\
72.8489333153348	0.2\\
72.8989360151188	0.2\\
72.9489387149028	0.2\\
72.9989414146868	0.2\\
73.0489441144708	0.2\\
73.0989468142549	0.2\\
73.1489495140389	0.2\\
73.1989522138229	0.2\\
73.2489549136069	0.2\\
73.2989576133909	0.2\\
73.348960313175	0.2\\
73.398963012959	0.2\\
73.448965712743	0.2\\
73.498968412527	0.2\\
73.548971112311	0.2\\
73.598973812095	0.2\\
73.6489765118791	0.2\\
73.6989792116631	0.2\\
73.7489819114471	0.2\\
73.7989846112311	0.2\\
73.8489873110151	0.2\\
73.8989900107991	0.2\\
73.9489927105832	0.2\\
73.9989954103672	0.2\\
74.0489981101512	0.2\\
74.0990008099352	0.2\\
74.1490035097192	0.2\\
74.1990062095032	0.2\\
74.2490089092873	0.2\\
74.2990116090713	0.2\\
74.3490143088553	0.2\\
74.3990170086393	0.2\\
74.4490197084233	0.2\\
74.4990224082073	0.2\\
74.5490251079914	0.2\\
74.5990278077754	0.2\\
74.6490305075594	0.2\\
74.6990332073434	0.2\\
74.7490359071274	0.2\\
74.7990386069115	0.2\\
74.8490413066955	0.2\\
74.8990440064795	0.2\\
74.9490467062635	0.2\\
74.9990494060475	0.2\\
75.0490521058315	0.2\\
75.0990548056156	0.2\\
75.1490575053996	0.2\\
75.1990602051836	0.2\\
75.2490629049676	0.2\\
75.2990656047516	0.2\\
75.3490683045356	0.2\\
75.3990710043197	0.2\\
75.4490737041037	0.2\\
75.4990764038877	0.2\\
75.5490791036717	0.2\\
75.5990818034557	0.2\\
75.6490845032397	0.2\\
75.6990872030238	0.2\\
75.7490899028078	0.2\\
75.7990926025918	0.2\\
75.8490953023758	0.2\\
75.8990980021598	0.2\\
75.9491007019438	0.2\\
75.9991034017279	0.2\\
76.0491061015119	0.2\\
76.0991088012959	0.2\\
76.1491115010799	0.2\\
76.1991142008639	0.2\\
76.2491169006479	0.2\\
76.299119600432	0.2\\
76.349122300216	0.2\\
76.399125	0.2\\
76.449127699784	0.2\\
76.499130399568	0.2\\
76.5491330993521	0.2\\
76.5991357991361	0.2\\
76.6491384989201	0.2\\
76.6991411987041	0.2\\
76.7491438984881	0.2\\
76.7991465982721	0.2\\
76.8491492980562	0.2\\
76.8991519978402	0.2\\
76.9491546976242	0.2\\
76.9991573974082	0.2\\
77.0491600971922	0.2\\
77.0991627969762	0.2\\
77.1491654967603	0.2\\
77.1991681965443	0.2\\
77.2491708963283	0.2\\
77.2991735961123	0.2\\
77.3491762958963	0.2\\
77.3991789956804	0.2\\
77.4491816954644	0.2\\
77.4991843952484	0.2\\
77.5491870950324	0.2\\
77.5991897948164	0.2\\
77.6491924946004	0.2\\
77.6991951943844	0.2\\
77.7491978941685	0.2\\
77.7992005939525	0.2\\
77.8492032937365	0.2\\
77.8992059935205	0.2\\
77.9492086933045	0.2\\
77.9992113930886	0.2\\
78.0492140928726	0.2\\
78.0992167926566	0.2\\
78.1492194924406	0.2\\
78.1992221922246	0.2\\
78.2492248920086	0.2\\
78.2992275917927	0.2\\
78.3492302915767	0.2\\
78.3992329913607	0.2\\
78.4492356911447	0.2\\
78.4992383909287	0.2\\
78.5492410907127	0.2\\
78.5992437904968	0.2\\
78.6492464902808	0.2\\
78.6992491900648	0.2\\
78.7492518898488	0.2\\
78.7992545896328	0.2\\
78.8492572894169	0.2\\
78.8992599892009	0.2\\
78.9492626889849	0.2\\
78.9992653887689	0.2\\
79.0492680885529	0.2\\
79.0992707883369	0.2\\
79.149273488121	0.2\\
79.199276187905	0.2\\
79.249278887689	0.2\\
79.299281587473	0.2\\
79.349284287257	0.2\\
79.3992869870411	0.2\\
79.4492896868251	0.2\\
79.4992923866091	0.2\\
79.5492950863931	0.2\\
79.5992977861771	0.2\\
79.6493004859611	0.2\\
79.6993031857451	0.2\\
79.7493058855292	0.2\\
79.7993085853132	0.2\\
79.8493112850972	0.2\\
79.8993139848812	0.2\\
79.9493166846652	0.2\\
79.9993193844493	0.2\\
80.0493220842333	0.2\\
80.0993247840173	0.2\\
80.1493274838013	0.2\\
80.1993301835853	0.2\\
80.2493328833693	0.2\\
80.2993355831534	0.2\\
80.3493382829374	0.2\\
80.3993409827214	0.2\\
80.4493436825054	0.2\\
80.4993463822894	0.2\\
80.5493490820734	0.2\\
80.5993517818575	0.2\\
80.6493544816415	0.2\\
80.6993571814255	0.2\\
80.7493598812095	0.2\\
80.7993625809935	0.2\\
80.8493652807775	0.2\\
80.8993679805616	0.2\\
80.9493706803456	0.2\\
80.9993733801296	0.2\\
81.0493760799136	0.2\\
81.0993787796976	0.2\\
81.1493814794816	0.2\\
81.1993841792657	0.2\\
81.2493868790497	0.2\\
81.2993895788337	0.2\\
81.3493922786177	0.2\\
81.3993949784017	0.2\\
81.4493976781857	0.2\\
81.4994003779698	0.2\\
81.5494030777538	0.2\\
81.5994057775378	0.2\\
81.6494084773218	0.2\\
81.6994111771058	0.2\\
81.7494138768898	0.2\\
81.7994165766739	0.2\\
81.8494192764579	0.2\\
81.8994219762419	0.2\\
81.9494246760259	0.2\\
81.9994273758099	0.2\\
82.049430075594	0.2\\
82.099432775378	0.2\\
82.149435475162	0.2\\
82.199438174946	0.2\\
82.24944087473	0.2\\
82.299443574514	0.2\\
82.3494462742981	0.2\\
82.3994489740821	0.2\\
82.4494516738661	0.2\\
82.4994543736501	0.2\\
82.5494570734341	0.2\\
82.5994597732181	0.2\\
82.6294613930885	0\\
82.6794640928726	0\\
82.7294667926566	0\\
82.7794694924406	0\\
82.8294721922246	0\\
82.8794748920086	0\\
82.9294775917927	0\\
82.9794802915767	0\\
83.0294829913607	0\\
83.0794856911447	0\\
83.1294883909287	0\\
83.1794910907127	0\\
83.2294937904968	0\\
83.2794964902808	0\\
83.3294991900648	0\\
83.3795018898488	0\\
83.4295045896328	0\\
83.4795072894169	0\\
83.5295099892009	0\\
83.5795126889849	0\\
83.6295153887689	0\\
83.6795180885529	0\\
83.7295207883369	0\\
83.779523488121	0\\
83.829526187905	0\\
83.879528887689	0\\
83.929531587473	0\\
83.979534287257	0\\
84.029536987041	0\\
84.0795396868251	0\\
84.1295423866091	0\\
84.1795450863931	0\\
84.2295477861771	0\\
84.2795504859611	0\\
84.3295531857451	0\\
84.3795558855292	0\\
84.4295585853132	0\\
84.4795612850972	0\\
84.5295639848812	0\\
84.5795666846652	0\\
84.6295693844492	0\\
84.6795720842333	0\\
84.7295747840173	0\\
84.7795774838013	0\\
84.8295801835853	0\\
84.8795828833693	0\\
84.9295855831534	0\\
84.9795882829374	0\\
85.0295909827214	0\\
85.0795936825054	0\\
85.1295963822894	0\\
85.1795990820734	0\\
85.2296017818575	0\\
85.2796044816415	0\\
85.3296071814255	0\\
85.3796098812095	0\\
85.4296125809935	0\\
85.4796152807775	0\\
85.5296179805616	0\\
85.5796206803456	0\\
85.6296233801296	0\\
85.6796260799136	0\\
85.7296287796976	0\\
85.7796314794817	0\\
85.8296341792657	0\\
85.8796368790497	0\\
85.9296395788337	0\\
85.9796422786177	0\\
86.0296449784017	0\\
86.0796476781857	0\\
86.1296503779698	0\\
86.1796530777538	0\\
86.2296557775378	0\\
86.2796584773218	0\\
86.3296611771058	0\\
86.3796638768899	0\\
86.4296665766739	0\\
86.4796692764579	0\\
86.5296719762419	0\\
86.5796746760259	0\\
86.6296773758099	0\\
86.679680075594	0\\
86.729682775378	0\\
86.779685475162	0\\
86.829688174946	0\\
86.87969087473	0\\
86.929693574514	0\\
86.9796962742981	0\\
87.0296989740821	0\\
87.0797016738661	0\\
87.1297043736501	0\\
87.1797070734341	0\\
87.2297097732181	0\\
87.2797124730022	0\\
87.3297151727862	0\\
87.3797178725702	0\\
87.4297205723542	0\\
87.4797232721382	0\\
87.5297259719222	0\\
87.5797286717063	0\\
87.6297313714903	0\\
87.6797340712743	0\\
87.7297367710583	0\\
87.7797394708423	0\\
87.8297421706264	0\\
87.8797448704104	0\\
87.9297475701944	0\\
87.9797502699784	0\\
88.0297529697624	0\\
88.0797556695464	0\\
88.1297583693305	0\\
88.1797610691145	0\\
88.2297637688985	0\\
88.2797664686825	0\\
88.3297691684665	0\\
88.3797718682505	0\\
88.4297745680346	0\\
88.4797772678186	0\\
88.5297799676026	0\\
88.5797826673866	0\\
88.6297853671706	0\\
88.6797880669546	0\\
88.7297907667387	0\\
88.7797934665227	0\\
88.8297961663067	0\\
88.8797988660907	0\\
88.9298015658747	0\\
88.9798042656588	0\\
89.0298069654428	0\\
89.0798096652268	0\\
89.1298123650108	0\\
89.1798150647948	0\\
89.2298177645788	0\\
89.2798204643629	0\\
89.3298231641469	0\\
89.3798258639309	0\\
89.4298285637149	0\\
89.4798312634989	0\\
89.5298339632829	0\\
89.579836663067	0\\
89.629839362851	0\\
89.679842062635	0\\
89.729844762419	0\\
89.779847462203	0\\
89.8298501619871	0\\
89.8798528617711	0\\
89.9298555615551	0\\
89.9798582613391	0\\
90.0298609611231	0\\
90.0798636609071	0\\
90.1298663606911	0\\
90.1798690604752	0\\
90.2298717602592	0\\
90.2798744600432	0\\
90.3298771598272	0\\
90.3798798596112	0\\
90.4298825593953	0\\
90.4798852591793	0\\
90.5298879589633	0\\
90.5798906587473	0\\
90.6298933585313	0\\
90.6798960583153	0\\
90.7298987580994	0\\
90.7799014578834	0\\
90.8299041576674	0\\
90.8799068574514	0\\
90.9299095572354	0\\
90.9799122570195	0\\
91.0299149568035	0\\
91.0799176565875	0\\
91.1299203563715	0\\
91.1799230561555	0\\
91.2299257559395	0\\
91.2799284557236	0\\
91.3299311555076	0\\
91.3799338552916	0\\
91.4299365550756	0\\
91.4799392548596	0\\
91.5299419546436	0\\
91.5799446544276	0\\
91.6299473542117	0\\
91.6799500539957	0\\
91.7299527537797	0\\
91.7799554535637	0\\
91.8299581533477	0\\
91.8799608531317	0\\
91.9299635529158	0\\
91.9799662526998	0\\
92.0299689524838	0\\
92.0799716522678	0\\
92.1299743520518	0\\
92.1799770518358	0\\
92.2299797516199	0\\
92.2799824514039	0\\
92.3299851511879	0\\
92.3799878509719	0\\
92.4299905507559	0\\
92.47999325054	0\\
92.529995950324	0\\
92.579998650108	0\\
92.605	0\\
};
\addlegendentry{set point};

\addplot [color=gray,solid,line width=0.2pt]
  table[row sep=crcr]{0	0.01\\
0.0500026997840173	0.01\\
0.100005399568035	0.01\\
0.150008099352052	0.01\\
0.200010799136069	0.01\\
0.250013498920086	0.01\\
0.300016198704104	0.01\\
0.350018898488121	0.01\\
0.400021598272138	0.01\\
0.450024298056156	0.01\\
0.500026997840173	0.01\\
0.55002969762419	0.01\\
0.600032397408207	0.01\\
0.650035097192225	0.01\\
0.700037796976242	0.01\\
0.750040496760259	0.01\\
0.800043196544276	0.01\\
0.850045896328294	0.01\\
0.900048596112311	0.01\\
0.950051295896328	0.01\\
1.00005399568035	0.01\\
1.05005669546436	0.01\\
1.10005939524838	0.01\\
1.1500620950324	0.01\\
1.20006479481641	0.01\\
1.25006749460043	0.01\\
1.30007019438445	0.01\\
1.35007289416847	0.01\\
1.40007559395248	0.01\\
1.4500782937365	0.01\\
1.50008099352052	0.01\\
1.55008369330454	0.01\\
1.60008639308855	0.01\\
1.65008909287257	0.01\\
1.70009179265659	0.01\\
1.7500944924406	0.01\\
1.80009719222462	0.01\\
1.85009989200864	0.01\\
1.90010259179266	0.01\\
1.95010529157667	0.01\\
2.00010799136069	0.01\\
2.05011069114471	0.01\\
2.10011339092873	0.01\\
2.15011609071274	0.01\\
2.20011879049676	0.01\\
2.25012149028078	0.01\\
2.30012419006479	0.01\\
2.35012688984881	0.01\\
2.40012958963283	0.01\\
2.45013228941685	0.01\\
2.50013498920086	0.01\\
2.55013768898488	0.01\\
2.6001403887689	0.01\\
2.65014308855292	0.01\\
2.70014578833693	0.01\\
2.75014848812095	0.01\\
2.80015118790497	0.01\\
2.85015388768899	0.01\\
2.900156587473	0.01\\
2.95015928725702	0.01\\
3.00016198704104	0.01\\
3.05016468682505	0.01\\
3.10016738660907	0.01\\
3.15017008639309	0.01\\
3.20017278617711	0.01\\
3.25017548596112	0.01\\
3.30017818574514	0.01\\
3.35018088552916	0.01\\
3.40018358531318	0.01\\
3.45018628509719	0.01\\
3.50018898488121	0.01\\
3.55019168466523	0.01\\
3.60019438444924	0.01\\
3.65019708423326	0.01\\
3.70019978401728	0.01\\
3.7502024838013	0.01\\
3.80020518358531	0.01\\
3.85020788336933	0.01\\
3.90021058315335	0.01\\
3.95021328293736	0.01\\
4.00021598272138	0.01\\
4.0502186825054	0.01\\
4.10022138228942	0.01\\
4.15022408207343	0.01\\
4.20022678185745	0.01\\
4.25022948164147	0.01\\
4.30023218142549	0.01\\
4.3502348812095	0.01\\
4.40023758099352	0.01\\
4.45024028077754	0.01\\
4.50024298056155	0.01\\
4.55024568034557	0.01\\
4.60024838012959	0.01\\
4.65025107991361	0.01\\
4.70025377969762	0.01\\
4.75025647948164	0.01\\
4.80025917926566	0.01\\
4.85026187904968	0.01\\
4.90026457883369	0.01\\
4.95026727861771	0.01\\
5.00026997840173	0.01\\
5.05027267818575	0.01\\
5.10027537796976	0.01\\
5.15027807775378	0.01\\
5.2002807775378	0.01\\
5.25028347732181	0.01\\
5.30028617710583	0.01\\
5.35028887688985	0.01\\
5.40029157667387	0.01\\
5.45029427645788	0.01\\
5.5002969762419	0.01\\
5.55029967602592	0.01\\
5.60030237580994	0.01\\
5.65030507559395	0.01\\
5.70030777537797	0.01\\
5.75031047516199	0.01\\
5.800313174946	0.01\\
5.85031587473002	0.01\\
5.90031857451404	0.01\\
5.95032127429806	0.01\\
6.00032397408207	0.01\\
6.05032667386609	0.01\\
6.10032937365011	0.01\\
6.15033207343413	0.01\\
6.20033477321814	0.01\\
6.25033747300216	0.01\\
6.30034017278618	0.01\\
6.3503428725702	0.01\\
6.40034557235421	0.01\\
6.45034827213823	0.01\\
6.50035097192225	0.01\\
6.55035367170626	0.01\\
6.60035637149028	0.01\\
6.6503590712743	0.01\\
6.70036177105832	0.01\\
6.75036447084233	0.01\\
6.80036717062635	0.01\\
6.85036987041037	0.01\\
6.90037257019439	0.01\\
6.9503752699784	0.01\\
7.00037796976242	0.01\\
7.05038066954644	0.01\\
7.10038336933045	0.01\\
7.15038606911447	0.01\\
7.20038876889849	0.01\\
7.25039146868251	0.01\\
7.30039416846652	0.01\\
7.35039686825054	0.01\\
7.40039956803456	0.01\\
7.45040226781857	0.01\\
7.50040496760259	0.01\\
7.55040766738661	0.01\\
7.60041036717063	0.01\\
7.65041306695464	0.01\\
7.70041576673866	0.01\\
7.75041846652268	0.01\\
7.8004211663067	0.01\\
7.85042386609071	0.01\\
7.90042656587473	0.01\\
7.95042926565875	0.01\\
8.00043196544276	0.01\\
8.05043466522678	0.01\\
8.1004373650108	0.01\\
8.15044006479482	0.01\\
8.20044276457883	0.01\\
8.25044546436285	0.01\\
8.30044816414687	0.01\\
8.35045086393089	0.01\\
8.4004535637149	0.01\\
8.45045626349892	0.01\\
8.50045896328294	0.01\\
8.55046166306696	0.01\\
8.60046436285097	0.01\\
8.65046706263499	0.01\\
8.70046976241901	0.01\\
8.75047246220302	0.01\\
8.80047516198704	0.01\\
8.85047786177106	0.01\\
8.90048056155508	0.01\\
8.95048326133909	0.01\\
9.00048596112311	0.01\\
9.05048866090713	0.01\\
9.10049136069114	0.01\\
9.15049406047516	0.01\\
9.20049676025918	0.01\\
9.2504994600432	0.01\\
9.30050215982721	0.01\\
9.35050485961123	0.01\\
9.40050755939525	0.01\\
9.45051025917927	0.01\\
9.50051295896328	0.01\\
9.5505156587473	0.01\\
9.60051835853132	0.01\\
9.65052105831533	0.01\\
9.70052375809935	0.01\\
9.75052645788337	0.01\\
9.80052915766739	0.01\\
9.8505318574514	0.01\\
9.90053455723542	0.01\\
9.95053725701944	0.01\\
10.0005399568035	0.01\\
10.0505426565875	0.01\\
10.1005453563715	0.01\\
10.1505480561555	0.01\\
10.2005507559395	0.01\\
10.2505534557235	0.01\\
10.3005561555076	0.01\\
10.3505588552916	0.01\\
10.4005615550756	0.01\\
10.4505642548596	0.01\\
10.5005669546436	0.01\\
10.5505696544276	0.01\\
10.6005723542117	0.01\\
10.6505750539957	0.01\\
10.7005777537797	0.01\\
10.7505804535637	0.01\\
10.8005831533477	0.01\\
10.8505858531317	0.01\\
10.9005885529158	0.01\\
10.9505912526998	0.01\\
11.0005939524838	0.01\\
11.0505966522678	0.01\\
11.1005993520518	0.01\\
11.1506020518359	0.01\\
11.2006047516199	0.01\\
11.2506074514039	0.01\\
11.3006101511879	0.01\\
11.3506128509719	0.01\\
11.4006155507559	0.01\\
11.45061825054	0.01\\
11.500620950324	0.01\\
11.550623650108	0.01\\
11.600626349892	0.01\\
11.650629049676	0.01\\
11.70063174946	0.01\\
11.7506344492441	0.01\\
11.8006371490281	0.01\\
11.8506398488121	0.01\\
11.9006425485961	0.01\\
11.9506452483801	0.01\\
12.0006479481641	0.01\\
12.0506506479482	0.01\\
12.1006533477322	0.01\\
12.1506560475162	0.01\\
12.2006587473002	0.01\\
12.2506614470842	0.01\\
12.3006641468683	0.01\\
12.3506668466523	0.01\\
12.4006695464363	0.01\\
12.4506722462203	0.01\\
12.5006749460043	0.01\\
12.5506776457883	0.01\\
12.6006803455724	0.01\\
12.6506830453564	0.01\\
12.7006857451404	0.01\\
12.7506884449244	0.01\\
12.8006911447084	0.01\\
12.8506938444924	0.01\\
12.9006965442765	0.01\\
12.9506992440605	0.01\\
13.0007019438445	0.01\\
13.0507046436285	0.01\\
13.1007073434125	0.01\\
13.1507100431965	0.01\\
13.2007127429806	0.01\\
13.2507154427646	0.01\\
13.3007181425486	0.01\\
13.3507208423326	0.01\\
13.4007235421166	0.01\\
13.4507262419006	0.01\\
13.5007289416847	0.01\\
13.5507316414687	0.01\\
13.6007343412527	0.01\\
13.6507370410367	0.01\\
13.7007397408207	0.01\\
13.7507424406048	0.01\\
13.8007451403888	0.01\\
13.8507478401728	0.01\\
13.9007505399568	0.01\\
13.9507532397408	0.01\\
14.0007559395248	0.01\\
14.0507586393089	0.01\\
14.1007613390929	0.01\\
14.1507640388769	0.01\\
14.2007667386609	0.01\\
14.2507694384449	0.01\\
14.3007721382289	0.01\\
14.350774838013	0.01\\
14.400777537797	0.01\\
14.450780237581	0.01\\
14.500782937365	0.01\\
14.550785637149	0.01\\
14.600788336933	0.01\\
14.6507910367171	0.01\\
14.7007937365011	0.01\\
14.7507964362851	0.01\\
14.8007991360691	0.01\\
14.8508018358531	0.01\\
14.9008045356372	0.01\\
14.9508072354212	0.01\\
15.0008099352052	0.01\\
15.0508126349892	0.01\\
15.1008153347732	0.01\\
15.1508180345572	0.01\\
15.2008207343413	0.01\\
15.2508234341253	0.01\\
15.3008261339093	0.01\\
15.3508288336933	0.01\\
15.4008315334773	0.01\\
15.4508342332613	0.01\\
15.5008369330454	0.01\\
15.5508396328294	0.01\\
15.6008423326134	0.01\\
15.6508450323974	0.01\\
15.7008477321814	0.01\\
15.7508504319654	0.01\\
15.8008531317495	0.01\\
15.8508558315335	0.01\\
15.9008585313175	0.01\\
15.9508612311015	0.01\\
16.0008639308855	0.01\\
16.0508666306695	0.01\\
16.1008693304536	0.01\\
16.1508720302376	0.01\\
16.2008747300216	0.01\\
16.2508774298056	0.01\\
16.3008801295896	0.01\\
16.3508828293737	0.01\\
16.4008855291577	0.01\\
16.4508882289417	0.01\\
16.5008909287257	0.01\\
16.5508936285097	0.01\\
16.6008963282937	0.01\\
16.6508990280778	0.01\\
16.7009017278618	0.01\\
16.7509044276458	0.01\\
16.8009071274298	0.01\\
16.8509098272138	0.01\\
16.9009125269978	0.01\\
16.9509152267819	0.01\\
17.0009179265659	0.01\\
17.0509206263499	0.01\\
17.1009233261339	0.01\\
17.1509260259179	0.01\\
17.2009287257019	0.01\\
17.250931425486	0.01\\
17.30093412527	0.01\\
17.350936825054	0.01\\
17.400939524838	0.01\\
17.450942224622	0.01\\
17.500944924406	0.01\\
17.5509476241901	0.01\\
17.6009503239741	0.01\\
17.6509530237581	0.01\\
17.7009557235421	0.01\\
17.7509584233261	0.01\\
17.8009611231102	0.01\\
17.8509638228942	0.01\\
17.9009665226782	0.01\\
17.9509692224622	0.01\\
18.0009719222462	0.01\\
18.0509746220302	0.01\\
18.1009773218143	0.01\\
18.1509800215983	0.01\\
18.2009827213823	0.01\\
18.2509854211663	0.01\\
18.3009881209503	0.01\\
18.3509908207343	0.01\\
18.4009935205184	0.01\\
18.4509962203024	0.01\\
18.5009989200864	0.01\\
18.5510016198704	0.01\\
18.6010043196544	0.01\\
18.6510070194384	0.01\\
18.7010097192225	0.01\\
18.7510124190065	0.01\\
18.8010151187905	0.01\\
18.8510178185745	0.01\\
18.9010205183585	0.01\\
18.9510232181425	0.01\\
19.0010259179266	0.01\\
19.0510286177106	0.01\\
19.1010313174946	0.01\\
19.1510340172786	0.01\\
19.2010367170626	0.01\\
19.2510394168467	0.01\\
19.3010421166307	0.01\\
19.3510448164147	0.01\\
19.4010475161987	0.01\\
19.4510502159827	0.01\\
19.5010529157667	0.01\\
19.5510556155508	0.01\\
19.6010583153348	0.01\\
19.6510610151188	0.01\\
19.7010637149028	0.01\\
19.7510664146868	0.01\\
19.8010691144708	0.01\\
19.8510718142549	0.01\\
19.9010745140389	0.01\\
19.9510772138229	0.01\\
20.0010799136069	0.01\\
20.0510826133909	0.01\\
20.1010853131749	0.01\\
20.151088012959	0.01\\
20.201090712743	0.01\\
20.251093412527	0.01\\
20.301096112311	0.01\\
20.351098812095	0.01\\
20.4011015118791	0.01\\
20.4511042116631	0.01\\
20.5011069114471	0.01\\
20.5511096112311	0.01\\
20.6011123110151	0.01\\
20.6511150107991	0.01\\
20.7011177105832	0.01\\
20.7511204103672	0.01\\
20.8011231101512	0.01\\
20.8511258099352	0.01\\
20.9011285097192	0.01\\
20.9511312095032	0.01\\
21.0011339092873	0.01\\
21.0511366090713	0.01\\
21.1011393088553	0.01\\
21.1511420086393	0.01\\
21.2011447084233	0.01\\
21.2511474082073	0.01\\
21.3011501079914	0.01\\
21.3511528077754	0.01\\
21.4011555075594	0.01\\
21.4511582073434	0.01\\
21.5011609071274	0.01\\
21.5511636069114	0.01\\
21.6011663066955	0.01\\
21.6511690064795	0.01\\
21.7011717062635	0.01\\
21.7511744060475	0.01\\
21.8011771058315	0.01\\
21.8511798056156	0.01\\
21.9011825053996	0.01\\
21.9511852051836	0.01\\
22.0011879049676	0.01\\
22.0511906047516	0.01\\
22.1011933045356	0.01\\
22.1511960043197	0.01\\
22.2011987041037	0.01\\
22.2512014038877	0.01\\
22.3012041036717	0.01\\
22.3512068034557	0.01\\
22.4012095032397	0.01\\
22.4512122030238	0.01\\
22.5012149028078	0.01\\
22.5512176025918	0.01\\
22.6012203023758	0.01\\
22.6512230021598	0.01\\
22.7012257019438	0.01\\
22.7512284017279	0.01\\
22.8012311015119	0.01\\
22.8512338012959	0.01\\
22.9012365010799	0.01\\
22.9512392008639	0.01\\
23.0012419006479	0.01\\
23.051244600432	0.01\\
23.101247300216	0.01\\
23.15125	0.01\\
23.201252699784	0.01\\
23.251255399568	0.01\\
23.3012580993521	0.01\\
23.3512607991361	0.01\\
23.4012634989201	0.01\\
23.4512661987041	0.01\\
23.5012688984881	0.01\\
23.5512715982721	0.01\\
23.6012742980562	0.01\\
23.6512769978402	0.01\\
23.7012796976242	0.01\\
23.7512823974082	0.01\\
23.8012850971922	0.01\\
23.8512877969762	0.01\\
23.9012904967603	0.01\\
23.9512931965443	0.01\\
24.0012958963283	0.01\\
24.0512985961123	0.01\\
24.1013012958963	0.01\\
24.1513039956803	0.01\\
24.2013066954644	0.01\\
24.2513093952484	0.01\\
24.3013120950324	0.01\\
24.3513147948164	0.01\\
24.4013174946004	0.01\\
24.4513201943845	0.01\\
24.5013228941685	0.01\\
24.5513255939525	0.01\\
24.6013282937365	0.01\\
24.6513309935205	0.01\\
24.7013336933045	0.01\\
24.7513363930886	0.01\\
24.8013390928726	0.01\\
24.8513417926566	0.01\\
24.9013444924406	0.01\\
24.9513471922246	0.01\\
25.0013498920086	0.01\\
25.0513525917927	0.01\\
25.1013552915767	0.01\\
25.1513579913607	0.01\\
25.2013606911447	0.01\\
25.2513633909287	0.01\\
25.3013660907127	0.01\\
25.3513687904968	0.01\\
25.4013714902808	0.01\\
25.4513741900648	0.01\\
25.5013768898488	0.01\\
25.5513795896328	0.01\\
25.6013822894168	0.01\\
25.6513849892009	0.01\\
25.7013876889849	0.01\\
25.7513903887689	0.01\\
25.8013930885529	0.01\\
25.8513957883369	0.01\\
25.901398488121	0.01\\
25.951401187905	0.01\\
26.001403887689	0.01\\
26.051406587473	0.01\\
26.101409287257	0.01\\
26.151411987041	0.01\\
26.2014146868251	0.01\\
26.2514173866091	0.01\\
26.3014200863931	0.01\\
26.3514227861771	0.01\\
26.4014254859611	0.01\\
26.4514281857451	0.01\\
26.5014308855292	0.01\\
26.5514335853132	0.01\\
26.6014362850972	0.01\\
26.6514389848812	0.01\\
26.7014416846652	0.01\\
26.7514443844492	0.01\\
26.8014470842333	0.01\\
26.8514497840173	0.01\\
26.9014524838013	0.01\\
26.9514551835853	0.01\\
27.0014578833693	0.01\\
27.0514605831534	0.01\\
27.1014632829374	0.01\\
27.1514659827214	0.01\\
27.2014686825054	0.01\\
27.2514713822894	0.01\\
27.3014740820734	0.01\\
27.3514767818575	0.01\\
27.4014794816415	0.01\\
27.4514821814255	0.01\\
27.5014848812095	0.01\\
27.5514875809935	0.01\\
27.6014902807775	0.01\\
27.6514929805616	0.01\\
27.7014956803456	0.01\\
27.7514983801296	0.01\\
27.8015010799136	0.01\\
27.8515037796976	0.01\\
27.9015064794816	0.01\\
27.9515091792657	0.01\\
28.0015118790497	0.01\\
28.0515145788337	0.01\\
28.1015172786177	0.01\\
28.1515199784017	0.01\\
28.2015226781857	0.01\\
28.2515253779698	0.01\\
28.3015280777538	0.01\\
28.3515307775378	0.01\\
28.4015334773218	0.01\\
28.4515361771058	0.01\\
28.5015388768899	0.01\\
28.5515415766739	0.01\\
28.6015442764579	0.01\\
28.6515469762419	0.01\\
28.7015496760259	0.01\\
28.7515523758099	0.01\\
28.801555075594	0.01\\
28.851557775378	0.01\\
28.901560475162	0.01\\
28.951563174946	0.01\\
29.00156587473	0.01\\
29.051568574514	0.01\\
29.1015712742981	0.01\\
29.1515739740821	0.01\\
29.2015766738661	0.01\\
29.2515793736501	0.01\\
29.3015820734341	0.01\\
29.3515847732181	0.01\\
29.4015874730022	0.01\\
29.4515901727862	0.01\\
29.5015928725702	0.01\\
29.5515955723542	0.01\\
29.6015982721382	0.01\\
29.6516009719222	0.01\\
29.7016036717063	0.01\\
29.7516063714903	0.01\\
29.8016090712743	0.01\\
29.8516117710583	0.01\\
29.9016144708423	0.01\\
29.9516171706264	0.01\\
30.0016198704104	0.01\\
30.0516225701944	0.01\\
30.1016252699784	0.01\\
30.1516279697624	0.01\\
30.2016306695464	0.01\\
30.2516333693305	0.01\\
30.3016360691145	0.01\\
30.3516387688985	0.01\\
30.4016414686825	0.01\\
30.4516441684665	0.01\\
30.5016468682505	0.01\\
30.5516495680346	0.01\\
30.6016522678186	0.01\\
30.6516549676026	0.01\\
30.7016576673866	0.01\\
30.7516603671706	0.01\\
30.8016630669546	0.01\\
30.8516657667387	0.01\\
30.9016684665227	0.01\\
30.9516711663067	0.01\\
31.0016738660907	0.01\\
31.0516765658747	0.01\\
31.1016792656587	0.01\\
31.1516819654428	0.01\\
31.2016846652268	0.01\\
31.2516873650108	0.01\\
31.3016900647948	0.01\\
31.3516927645788	0.01\\
31.4016954643629	0.01\\
31.4516981641469	0.01\\
31.5017008639309	0.01\\
31.5517035637149	0.01\\
31.6017062634989	0.01\\
31.6517089632829	0.01\\
31.701711663067	0.01\\
31.751714362851	0.01\\
31.801717062635	0.01\\
31.851719762419	0.01\\
31.901722462203	0.01\\
31.951725161987	0.01\\
32.0017278617711	0.01\\
32.0517305615551	0.01\\
32.1017332613391	0.01\\
32.1517359611231	0.01\\
32.2017386609071	0.01\\
32.2517413606911	0.01\\
32.3017440604752	0.01\\
32.3517467602592	0.01\\
32.4017494600432	0.01\\
32.4517521598272	0.01\\
32.5017548596112	0.01\\
32.5517575593952	0.01\\
32.6017602591793	0.01\\
32.6517629589633	0.01\\
32.7017656587473	0.01\\
32.7517683585313	0.01\\
32.8017710583153	0.01\\
32.8517737580993	0.01\\
32.9017764578834	0.01\\
32.9517791576674	0.01\\
33.0017818574514	0.01\\
33.0517845572354	0.01\\
33.1017872570194	0.01\\
33.1517899568035	0.01\\
33.2017926565875	0.01\\
33.2517953563715	0.01\\
33.3017980561555	0.01\\
33.3518007559395	0.01\\
33.4018034557235	0.01\\
33.4518061555076	0.01\\
33.5018088552916	0.01\\
33.5518115550756	0.01\\
33.6018142548596	0.01\\
33.6518169546436	0.01\\
33.7018196544277	0.01\\
33.7518223542117	0.01\\
33.8018250539957	0.01\\
33.8518277537797	0.01\\
33.9018304535637	0.01\\
33.9518331533477	0.01\\
34.0018358531318	0.01\\
34.0518385529158	0.01\\
34.1018412526998	0.01\\
34.1518439524838	0.01\\
34.2018466522678	0.01\\
34.2518493520518	0.01\\
34.3018520518359	0.01\\
34.3518547516199	0.01\\
34.4018574514039	0.01\\
34.4518601511879	0.01\\
34.5018628509719	0.01\\
34.5518655507559	0.01\\
34.60186825054	0.01\\
34.651870950324	0.01\\
34.701873650108	0.01\\
34.751876349892	0.01\\
34.801879049676	0.01\\
34.85188174946	0.01\\
34.9018844492441	0.01\\
34.9518871490281	0.01\\
35.0018898488121	0.01\\
35.0518925485961	0.01\\
35.1018952483801	0.01\\
35.1518979481641	0.01\\
35.2019006479482	0.01\\
35.2519033477322	0.01\\
35.3019060475162	0.01\\
35.3519087473002	0.01\\
35.4019114470842	0.01\\
35.4519141468683	0.01\\
35.5019168466523	0.01\\
35.5519195464363	0.01\\
35.6019222462203	0.01\\
35.6519249460043	0.01\\
35.7019276457883	0.01\\
35.7519303455724	0.01\\
35.8019330453564	0.01\\
35.8519357451404	0.01\\
35.9019384449244	0.01\\
35.9519411447084	0.01\\
36.0019438444924	0.01\\
36.0519465442765	0.01\\
36.1019492440605	0.01\\
36.1519519438445	0.01\\
36.2019546436285	0.01\\
36.2519573434125	0.01\\
36.3019600431965	0.01\\
36.3519627429806	0.01\\
36.4019654427646	0.01\\
36.4519681425486	0.01\\
36.5019708423326	0.01\\
36.5519735421166	0.01\\
36.6019762419006	0.01\\
36.6519789416847	0.01\\
36.7019816414687	0.01\\
36.7519843412527	0.01\\
36.8019870410367	0.01\\
36.8519897408207	0.01\\
36.9019924406048	0.01\\
36.9519951403888	0.01\\
37.0019978401728	0.01\\
37.0520005399568	0.01\\
37.1020032397408	0.01\\
37.1520059395248	0.01\\
37.2020086393089	0.01\\
37.2520113390929	0.01\\
37.3020140388769	0.01\\
37.3520167386609	0.01\\
37.4020194384449	0.01\\
37.4520221382289	0.01\\
37.502024838013	0.01\\
37.552027537797	0.01\\
37.602030237581	0.01\\
37.652032937365	0.01\\
37.702035637149	0.01\\
37.752038336933	0.01\\
37.8020410367171	0.01\\
37.8520437365011	0.01\\
37.9020464362851	0.01\\
37.9520491360691	0.01\\
38.0020518358531	0.01\\
38.0520545356372	0.01\\
38.1020572354212	0.01\\
38.1520599352052	0.01\\
38.2020626349892	0.01\\
38.2520653347732	0.01\\
38.3020680345572	0.01\\
38.3520707343413	0.01\\
38.4020734341253	0.01\\
38.4520761339093	0.01\\
38.5020788336933	0.01\\
38.5520815334773	0.01\\
38.6020842332613	0.01\\
38.6520869330454	0.01\\
38.7020896328294	0.01\\
38.7520923326134	0.01\\
38.8020950323974	0.01\\
38.8520977321814	0.01\\
38.9021004319654	0.01\\
38.9521031317495	0.01\\
39.0021058315335	0.01\\
39.0521085313175	0.01\\
39.1021112311015	0.01\\
39.1521139308855	0.01\\
39.2021166306696	0.01\\
39.2521193304536	0.01\\
39.3021220302376	0.01\\
39.3521247300216	0.01\\
39.4021274298056	0.01\\
39.4521301295896	0.01\\
39.5021328293737	0.01\\
39.5521355291577	0.01\\
39.6021382289417	0.01\\
39.6521409287257	0.01\\
39.7021436285097	0.01\\
39.7521463282937	0.01\\
39.8021490280778	0.01\\
39.8521517278618	0.01\\
39.9021544276458	0.01\\
39.9521571274298	0.01\\
40.0021598272138	0.01\\
40.0521625269978	0.01\\
40.1021652267819	0.01\\
40.1521679265659	0.01\\
40.2021706263499	0.01\\
40.2521733261339	0.01\\
40.3021760259179	0.01\\
40.3521787257019	0.01\\
40.402181425486	0.01\\
40.45218412527	0.01\\
40.502186825054	0.01\\
40.552189524838	0.01\\
40.602192224622	0.01\\
40.652194924406	0.01\\
40.7021976241901	0.01\\
40.7522003239741	0.01\\
40.8022030237581	0.01\\
40.8522057235421	0.01\\
40.9022084233261	0.01\\
40.9522111231102	0.01\\
41.0022138228942	0.01\\
41.0522165226782	0.01\\
41.1022192224622	0.01\\
41.1522219222462	0.01\\
41.2022246220302	0.01\\
41.2522273218143	0.01\\
41.3022300215983	0.01\\
41.3522327213823	0.01\\
41.4022354211663	0.01\\
41.4522381209503	0.01\\
41.5022408207343	0.01\\
41.5522435205184	0.01\\
41.6022462203024	0.01\\
41.6522489200864	0.01\\
41.7022516198704	0.01\\
41.7522543196544	0.01\\
41.8022570194385	0.01\\
41.8522597192225	0.01\\
41.9022624190065	0.01\\
41.9522651187905	0.01\\
42.0022678185745	0.01\\
42.0522705183585	0.01\\
42.1022732181426	0.01\\
42.1522759179266	0.01\\
42.2022786177106	0.01\\
42.2522813174946	0.01\\
42.3022840172786	0.01\\
42.3522867170626	0.01\\
42.4022894168467	0.01\\
42.4522921166307	0.01\\
42.5022948164147	0.01\\
42.5522975161987	0.01\\
42.6023002159827	0.01\\
42.6523029157667	0.01\\
42.7023056155508	0.01\\
42.7523083153348	0.01\\
42.8023110151188	0.01\\
42.8523137149028	0.01\\
42.9023164146868	0.01\\
42.9523191144708	0.01\\
43.0023218142549	0.01\\
43.0523245140389	0.01\\
43.1023272138229	0.01\\
43.1523299136069	0.01\\
43.2023326133909	0.01\\
43.2523353131749	0.01\\
43.302338012959	0.01\\
43.352340712743	0.01\\
43.402343412527	0.01\\
43.452346112311	0.01\\
43.502348812095	0.01\\
43.5523515118791	0.01\\
43.6023542116631	0.01\\
43.6523569114471	0.01\\
43.7023596112311	0.01\\
43.7523623110151	0.01\\
43.8023650107991	0.01\\
43.8523677105832	0.01\\
43.9023704103672	0.01\\
43.9523731101512	0.01\\
44.0023758099352	0.01\\
44.0523785097192	0.01\\
44.1023812095032	0.01\\
44.1523839092873	0.01\\
44.2023866090713	0.01\\
44.2523893088553	0.01\\
44.3023920086393	0.01\\
44.3523947084233	0.01\\
44.4023974082073	0.01\\
44.4524001079914	0.01\\
44.5024028077754	0.01\\
44.5524055075594	0.01\\
44.6024082073434	0.01\\
44.6524109071274	0.01\\
44.7024136069114	0.01\\
44.7524163066955	0.01\\
44.8024190064795	0.01\\
44.8524217062635	0.01\\
44.9024244060475	0.01\\
44.9524271058315	0.01\\
45.0024298056155	0.01\\
45.0524325053996	0.01\\
45.1024352051836	0.01\\
45.1524379049676	0.01\\
45.2024406047516	0.01\\
45.2524433045356	0.01\\
45.3024460043197	0.01\\
45.3524487041037	0.01\\
45.4024514038877	0.01\\
45.4524541036717	0.01\\
45.5024568034557	0.01\\
45.5524595032397	0.01\\
45.6024622030238	0.01\\
45.6524649028078	0.01\\
45.7024676025918	0.01\\
45.7524703023758	0.01\\
45.8024730021598	0.01\\
45.8524757019439	0.01\\
45.9024784017279	0.01\\
45.9524811015119	0.01\\
46.0024838012959	0.01\\
46.0524865010799	0.01\\
46.1024892008639	0.01\\
46.152491900648	0.01\\
46.202494600432	0.01\\
46.252497300216	0.01\\
46.3025	0.01\\
46.352502699784	0.01\\
46.402505399568	0.01\\
46.4525080993521	0.01\\
46.5025107991361	0.01\\
46.5525134989201	0.01\\
46.6025161987041	0.01\\
46.6525188984881	0.01\\
46.7025215982721	0.01\\
46.7525242980562	0.01\\
46.8025269978402	0.01\\
46.8525296976242	0.01\\
46.9025323974082	0.01\\
46.9525350971922	0.01\\
47.0025377969762	0.01\\
47.0525404967603	0.01\\
47.1025431965443	0.01\\
47.1525458963283	0.01\\
47.2025485961123	0.01\\
47.2525512958963	0.01\\
47.3025539956803	0.01\\
47.3525566954644	0.01\\
47.4025593952484	0.01\\
47.4525620950324	0.01\\
47.5025647948164	0.01\\
47.5525674946004	0.01\\
47.6025701943845	0.01\\
47.6525728941685	0.01\\
47.7025755939525	0.01\\
47.7525782937365	0.01\\
47.8025809935205	0.01\\
47.8525836933045	0.01\\
47.9025863930886	0.01\\
47.9525890928726	0.01\\
48.0025917926566	0.01\\
48.0525944924406	0.01\\
48.1025971922246	0.01\\
48.1525998920086	0.01\\
48.2026025917927	0.01\\
48.2526052915767	0.01\\
48.3026079913607	0.01\\
48.3526106911447	0.01\\
48.4026133909287	0.01\\
48.4526160907127	0.01\\
48.5026187904968	0.01\\
48.5526214902808	0.01\\
48.6026241900648	0.01\\
48.6526268898488	0.01\\
48.7026295896328	0.01\\
48.7526322894168	0.01\\
48.8026349892009	0.01\\
48.8526376889849	0.01\\
48.9026403887689	0.01\\
48.9526430885529	0.01\\
49.0026457883369	0.01\\
49.052648488121	0.01\\
49.102651187905	0.01\\
49.152653887689	0.01\\
49.202656587473	0.01\\
49.252659287257	0.01\\
49.302661987041	0.01\\
49.3526646868251	0.01\\
49.4026673866091	0.01\\
49.4526700863931	0.01\\
49.5026727861771	0.01\\
49.5526754859611	0.01\\
49.6026781857451	0.01\\
49.6526808855292	0.01\\
49.7026835853132	0.01\\
49.7526862850972	0.01\\
49.8026889848812	0.01\\
49.8526916846652	0.01\\
49.9026943844492	0.01\\
49.9526970842333	0.01\\
50.0026997840173	0.01\\
50.0527024838013	0.01\\
50.1027051835853	0.01\\
50.1527078833693	0.01\\
50.2027105831534	0.01\\
50.2527132829374	0.01\\
50.3027159827214	0.01\\
50.3527186825054	0.01\\
50.4027213822894	0.01\\
50.4527240820734	0.01\\
50.5027267818575	0.01\\
50.5527294816415	0.01\\
50.6027321814255	0.01\\
50.6527348812095	0.01\\
50.7027375809935	0.01\\
50.7527402807775	0.01\\
50.8027429805616	0.01\\
50.8527456803456	0.01\\
50.9027483801296	0.01\\
50.9527510799136	0.01\\
51.0027537796976	0.01\\
51.0527564794816	0.01\\
51.1027591792657	0.01\\
51.1527618790497	0.01\\
51.2027645788337	0.01\\
51.2527672786177	0.01\\
51.3027699784017	0.01\\
51.3527726781858	0.01\\
51.4027753779698	0.01\\
51.4527780777538	0.01\\
51.5027807775378	0.01\\
51.5527834773218	0.01\\
51.6027861771058	0.01\\
51.6527888768899	0.01\\
51.7027915766739	0.01\\
51.7527942764579	0.01\\
51.8027969762419	0.01\\
51.8527996760259	0.01\\
51.9028023758099	0.01\\
51.952805075594	0.01\\
52.002807775378	0.01\\
52.052810475162	0.01\\
52.102813174946	0.01\\
52.15281587473	0.01\\
52.202818574514	0.01\\
52.2528212742981	0.01\\
52.3028239740821	0.01\\
52.3528266738661	0.01\\
52.4028293736501	0.01\\
52.4528320734341	0.01\\
52.5028347732181	0.01\\
52.5528374730022	0.01\\
52.6028401727862	0.01\\
52.6528428725702	0.01\\
52.7028455723542	0.01\\
52.7528482721382	0.01\\
52.8028509719222	0.01\\
52.8528536717063	0.01\\
52.9028563714903	0.01\\
52.9528590712743	0.01\\
53.0028617710583	0.01\\
53.0528644708423	0.01\\
53.1028671706264	0.01\\
53.1528698704104	0.01\\
53.2028725701944	0.01\\
53.2528752699784	0.01\\
53.3028779697624	0.01\\
53.3528806695464	0.01\\
53.4028833693305	0.01\\
53.4528860691145	0.01\\
53.5028887688985	0.01\\
53.5528914686825	0.01\\
53.6028941684665	0.01\\
53.6528968682505	0.01\\
53.7028995680346	0.01\\
53.7529022678186	0.01\\
53.8029049676026	0.01\\
53.8529076673866	0.01\\
53.9029103671706	0.01\\
53.9529130669547	0.01\\
54.0029157667387	0.01\\
54.0529184665227	0.01\\
54.1029211663067	0.01\\
54.1529238660907	0.01\\
54.2029265658747	0.01\\
54.2529292656588	0.01\\
54.3029319654428	0.01\\
54.3529346652268	0.01\\
54.4029373650108	0.01\\
54.4529400647948	0.01\\
54.5029427645788	0.01\\
54.5529454643629	0.01\\
54.6029481641469	0.01\\
54.6529508639309	0.01\\
54.7029535637149	0.01\\
54.7529562634989	0.01\\
54.8029589632829	0.01\\
54.852961663067	0.01\\
54.902964362851	0.01\\
54.952967062635	0.01\\
55.002969762419	0.01\\
55.052972462203	0.01\\
55.102975161987	0.01\\
55.1529778617711	0.01\\
55.2029805615551	0.01\\
55.2529832613391	0.01\\
55.3029859611231	0.01\\
55.3529886609071	0.01\\
55.4029913606911	0.01\\
55.4529940604752	0.01\\
55.5029967602592	0.01\\
55.5529994600432	0.01\\
55.6030021598272	0.01\\
55.6530048596112	0.01\\
55.7030075593953	0.01\\
55.7530102591793	0.01\\
55.8030129589633	0.01\\
55.8530156587473	0.01\\
55.9030183585313	0.01\\
55.9530210583153	0.01\\
56.0030237580994	0.01\\
56.0530264578834	0.01\\
56.1030291576674	0.01\\
56.1530318574514	0.01\\
56.2030345572354	0.01\\
56.2530372570194	0.01\\
56.3030399568035	0.01\\
56.3530426565875	0.01\\
56.4030453563715	0.01\\
56.4530480561555	0.01\\
56.5030507559395	0.01\\
56.5530534557235	0.01\\
56.6030561555076	0.01\\
56.6530588552916	0.01\\
56.7030615550756	0.01\\
56.7530642548596	0.01\\
56.8030669546436	0.01\\
56.8530696544276	0.01\\
56.9030723542117	0.01\\
56.9530750539957	0.01\\
57.0030777537797	0.01\\
57.0530804535637	0.01\\
57.1030831533477	0.01\\
57.1530858531318	0.01\\
57.2030885529158	0.01\\
57.2530912526998	0.01\\
57.3030939524838	0.01\\
57.3530966522678	0.01\\
57.4030993520518	0.01\\
57.4531020518358	0.01\\
57.5031047516199	0.01\\
57.5531074514039	0.01\\
57.6031101511879	0.01\\
57.6531128509719	0.01\\
57.7031155507559	0.01\\
57.75311825054	0.01\\
57.803120950324	0.01\\
57.853123650108	0.01\\
57.903126349892	0.01\\
57.953129049676	0.01\\
58.00313174946	0.01\\
58.0531344492441	0.01\\
58.1031371490281	0.01\\
58.1531398488121	0.01\\
58.2031425485961	0.01\\
58.2531452483801	0.01\\
58.3031479481642	0.01\\
58.3531506479482	0.01\\
58.4031533477322	0.01\\
58.4531560475162	0.01\\
58.5031587473002	0.01\\
58.5531614470842	0.01\\
58.6031641468683	0.01\\
58.6531668466523	0.01\\
58.7031695464363	0.01\\
58.7531722462203	0.01\\
58.8031749460043	0.01\\
58.8531776457883	0.01\\
58.9031803455724	0.01\\
58.9531830453564	0.01\\
59.0031857451404	0.01\\
59.0531884449244	0.01\\
59.1031911447084	0.01\\
59.1531938444924	0.01\\
59.2031965442765	0.01\\
59.2531992440605	0.01\\
59.3032019438445	0.01\\
59.3532046436285	0.01\\
59.4032073434125	0.01\\
59.4532100431965	0.01\\
59.5032127429806	0.01\\
59.5532154427646	0.01\\
59.6032181425486	0.01\\
59.6532208423326	0.01\\
59.7032235421166	0.01\\
59.7532262419006	0.01\\
59.8032289416847	0.01\\
59.8532316414687	0.01\\
59.9032343412527	0.01\\
59.9532370410367	0.01\\
60.0032397408207	0.01\\
60.0532424406048	0.01\\
60.1032451403888	0.01\\
60.1532478401728	0.01\\
60.2032505399568	0.01\\
60.2532532397408	0.01\\
60.3032559395248	0.01\\
60.3532586393089	0.01\\
60.4032613390929	0.01\\
60.4532640388769	0.01\\
60.5032667386609	0.01\\
60.5532694384449	0.01\\
60.6032721382289	0.01\\
60.653274838013	0.01\\
60.703277537797	0.01\\
60.753280237581	0.01\\
60.803282937365	0.01\\
60.853285637149	0.01\\
60.9032883369331	0.01\\
60.9532910367171	0.01\\
61.0032937365011	0.01\\
61.0532964362851	0.01\\
61.1032991360691	0.01\\
61.1533018358531	0.01\\
61.2033045356371	0.01\\
61.2533072354212	0.01\\
61.3033099352052	0.01\\
61.3533126349892	0.01\\
61.4033153347732	0.01\\
61.4533180345572	0.01\\
61.5033207343413	0.01\\
61.5533234341253	0.01\\
61.6033261339093	0.01\\
61.6533288336933	0.01\\
61.7033315334773	0.01\\
61.7533342332613	0.01\\
61.8033369330454	0.01\\
61.8533396328294	0.01\\
61.9033423326134	0.01\\
61.9533450323974	0.01\\
62.0033477321814	0.01\\
62.0533504319654	0.01\\
62.1033531317495	0.01\\
62.1533558315335	0.01\\
62.2033585313175	0.01\\
62.2533612311015	0.01\\
62.3033639308855	0.01\\
62.3533666306695	0.01\\
62.4033693304536	0.01\\
62.4533720302376	0.01\\
62.5033747300216	0.01\\
62.5483771598272	0.21\\
62.5983798596112	0.21\\
62.6483825593953	0.21\\
62.6983852591793	0.21\\
62.7483879589633	0.21\\
62.7983906587473	0.21\\
62.8483933585313	0.21\\
62.8983960583153	0.21\\
62.9483987580994	0.21\\
62.9984014578834	0.21\\
63.0484041576674	0.21\\
63.0984068574514	0.21\\
63.1484095572354	0.21\\
63.1984122570194	0.21\\
63.2484149568035	0.21\\
63.2984176565875	0.21\\
63.3484203563715	0.21\\
63.3984230561555	0.21\\
63.4484257559395	0.21\\
63.4984284557235	0.21\\
63.5484311555076	0.21\\
63.5984338552916	0.21\\
63.6484365550756	0.21\\
63.6984392548596	0.21\\
63.7484419546436	0.21\\
63.7984446544276	0.21\\
63.8484473542117	0.21\\
63.8984500539957	0.21\\
63.9484527537797	0.21\\
63.9984554535637	0.21\\
64.0484581533477	0.21\\
64.0984608531318	0.21\\
64.1484635529158	0.21\\
64.1984662526998	0.21\\
64.2484689524838	0.21\\
64.2984716522678	0.21\\
64.3484743520518	0.21\\
64.3984770518359	0.21\\
64.4484797516199	0.21\\
64.4984824514039	0.21\\
64.5484851511879	0.21\\
64.5984878509719	0.21\\
64.648490550756	0.21\\
64.69849325054	0.21\\
64.748495950324	0.21\\
64.798498650108	0.21\\
64.848501349892	0.21\\
64.898504049676	0.21\\
64.94850674946	0.21\\
64.9985094492441	0.21\\
65.0485121490281	0.21\\
65.0985148488121	0.21\\
65.1485175485961	0.21\\
65.1985202483801	0.21\\
65.2485229481642	0.21\\
65.2985256479482	0.21\\
65.3485283477322	0.21\\
65.3985310475162	0.21\\
65.4485337473002	0.21\\
65.4985364470842	0.21\\
65.5485391468683	0.21\\
65.5985418466523	0.21\\
65.6485445464363	0.21\\
65.6985472462203	0.21\\
65.7485499460043	0.21\\
65.7985526457883	0.21\\
65.8485553455724	0.21\\
65.8985580453564	0.21\\
65.9485607451404	0.21\\
65.9985634449244	0.21\\
66.0485661447084	0.21\\
66.0985688444924	0.21\\
66.1485715442765	0.21\\
66.1985742440605	0.21\\
66.2485769438445	0.21\\
66.2985796436285	0.21\\
66.3485823434125	0.21\\
66.3985850431965	0.21\\
66.4485877429806	0.21\\
66.4985904427646	0.21\\
66.5485931425486	0.21\\
66.5985958423326	0.21\\
66.6485985421166	0.21\\
66.6986012419006	0.21\\
66.7486039416847	0.21\\
66.7986066414687	0.21\\
66.8486093412527	0.21\\
66.8986120410367	0.21\\
66.9486147408207	0.21\\
66.9986174406047	0.21\\
67.0486201403888	0.21\\
67.0986228401728	0.21\\
67.1486255399568	0.21\\
67.1986282397408	0.21\\
67.2486309395248	0.21\\
67.2986336393089	0.21\\
67.3486363390929	0.21\\
67.3986390388769	0.21\\
67.4486417386609	0.21\\
67.4986444384449	0.21\\
67.5486471382289	0.21\\
67.598649838013	0.21\\
67.648652537797	0.21\\
67.698655237581	0.21\\
67.748657937365	0.21\\
67.798660637149	0.21\\
67.8486633369331	0.21\\
67.8986660367171	0.21\\
67.9486687365011	0.21\\
67.9986714362851	0.21\\
68.0486741360691	0.21\\
68.0986768358531	0.21\\
68.1486795356372	0.21\\
68.1986822354212	0.21\\
68.2486849352052	0.21\\
68.2986876349892	0.21\\
68.3486903347732	0.21\\
68.3986930345572	0.21\\
68.4486957343413	0.21\\
68.4986984341253	0.21\\
68.5487011339093	0.21\\
68.5987038336933	0.21\\
68.6487065334773	0.21\\
68.6987092332613	0.21\\
68.7487119330454	0.21\\
68.7987146328294	0.21\\
68.8487173326134	0.21\\
68.8987200323974	0.21\\
68.9487227321814	0.21\\
68.9987254319654	0.21\\
69.0487281317495	0.21\\
69.0987308315335	0.21\\
69.1487335313175	0.21\\
69.1987362311015	0.21\\
69.2487389308855	0.21\\
69.2987416306696	0.21\\
69.3487443304536	0.21\\
69.3987470302376	0.21\\
69.4487497300216	0.21\\
69.4987524298056	0.21\\
69.5487551295896	0.21\\
69.5987578293737	0.21\\
69.6487605291577	0.21\\
69.6987632289417	0.21\\
69.7487659287257	0.21\\
69.7987686285097	0.21\\
69.8487713282937	0.21\\
69.8987740280778	0.21\\
69.9487767278618	0.21\\
69.9987794276458	0.21\\
70.0487821274298	0.21\\
70.0987848272138	0.21\\
70.1487875269979	0.21\\
70.1987902267819	0.21\\
70.2487929265659	0.21\\
70.2987956263499	0.21\\
70.3487983261339	0.21\\
70.3988010259179	0.21\\
70.4488037257019	0.21\\
70.498806425486	0.21\\
70.54880912527	0.21\\
70.598811825054	0.21\\
70.648814524838	0.21\\
70.698817224622	0.21\\
70.7488199244061	0.21\\
70.7988226241901	0.21\\
70.8488253239741	0.21\\
70.8988280237581	0.21\\
70.9488307235421	0.21\\
70.9988334233261	0.21\\
71.0488361231101	0.21\\
71.0988388228942	0.21\\
71.1488415226782	0.21\\
71.1988442224622	0.21\\
71.2488469222462	0.21\\
71.2988496220302	0.21\\
71.3488523218143	0.21\\
71.3988550215983	0.21\\
71.4488577213823	0.21\\
71.4988604211663	0.21\\
71.5488631209503	0.21\\
71.5988658207343	0.21\\
71.6488685205184	0.21\\
71.6988712203024	0.21\\
71.7488739200864	0.21\\
71.7988766198704	0.21\\
71.8488793196544	0.21\\
71.8988820194385	0.21\\
71.9488847192225	0.21\\
71.9988874190065	0.21\\
72.0488901187905	0.21\\
72.0988928185745	0.21\\
72.1488955183585	0.21\\
72.1988982181425	0.21\\
72.2489009179266	0.21\\
72.2989036177106	0.21\\
72.3489063174946	0.21\\
72.3989090172786	0.21\\
72.4489117170626	0.21\\
72.4989144168467	0.21\\
72.5489171166307	0.21\\
72.5989198164147	0.21\\
72.6489225161987	0.21\\
72.6989252159827	0.21\\
72.7489279157667	0.21\\
72.7989306155508	0.21\\
72.8489333153348	0.21\\
72.8989360151188	0.21\\
72.9489387149028	0.21\\
72.9989414146868	0.21\\
73.0489441144708	0.21\\
73.0989468142549	0.21\\
73.1489495140389	0.21\\
73.1989522138229	0.21\\
73.2489549136069	0.21\\
73.2989576133909	0.21\\
73.348960313175	0.21\\
73.398963012959	0.21\\
73.448965712743	0.21\\
73.498968412527	0.21\\
73.548971112311	0.21\\
73.598973812095	0.21\\
73.6489765118791	0.21\\
73.6989792116631	0.21\\
73.7489819114471	0.21\\
73.7989846112311	0.21\\
73.8489873110151	0.21\\
73.8989900107991	0.21\\
73.9489927105832	0.21\\
73.9989954103672	0.21\\
74.0489981101512	0.21\\
74.0990008099352	0.21\\
74.1490035097192	0.21\\
74.1990062095032	0.21\\
74.2490089092873	0.21\\
74.2990116090713	0.21\\
74.3490143088553	0.21\\
74.3990170086393	0.21\\
74.4490197084233	0.21\\
74.4990224082073	0.21\\
74.5490251079914	0.21\\
74.5990278077754	0.21\\
74.6490305075594	0.21\\
74.6990332073434	0.21\\
74.7490359071274	0.21\\
74.7990386069115	0.21\\
74.8490413066955	0.21\\
74.8990440064795	0.21\\
74.9490467062635	0.21\\
74.9990494060475	0.21\\
75.0490521058315	0.21\\
75.0990548056156	0.21\\
75.1490575053996	0.21\\
75.1990602051836	0.21\\
75.2490629049676	0.21\\
75.2990656047516	0.21\\
75.3490683045356	0.21\\
75.3990710043197	0.21\\
75.4490737041037	0.21\\
75.4990764038877	0.21\\
75.5490791036717	0.21\\
75.5990818034557	0.21\\
75.6490845032397	0.21\\
75.6990872030238	0.21\\
75.7490899028078	0.21\\
75.7990926025918	0.21\\
75.8490953023758	0.21\\
75.8990980021598	0.21\\
75.9491007019438	0.21\\
75.9991034017279	0.21\\
76.0491061015119	0.21\\
76.0991088012959	0.21\\
76.1491115010799	0.21\\
76.1991142008639	0.21\\
76.2491169006479	0.21\\
76.299119600432	0.21\\
76.349122300216	0.21\\
76.399125	0.21\\
76.449127699784	0.21\\
76.499130399568	0.21\\
76.5491330993521	0.21\\
76.5991357991361	0.21\\
76.6491384989201	0.21\\
76.6991411987041	0.21\\
76.7491438984881	0.21\\
76.7991465982721	0.21\\
76.8491492980562	0.21\\
76.8991519978402	0.21\\
76.9491546976242	0.21\\
76.9991573974082	0.21\\
77.0491600971922	0.21\\
77.0991627969762	0.21\\
77.1491654967603	0.21\\
77.1991681965443	0.21\\
77.2491708963283	0.21\\
77.2991735961123	0.21\\
77.3491762958963	0.21\\
77.3991789956804	0.21\\
77.4491816954644	0.21\\
77.4991843952484	0.21\\
77.5491870950324	0.21\\
77.5991897948164	0.21\\
77.6491924946004	0.21\\
77.6991951943844	0.21\\
77.7491978941685	0.21\\
77.7992005939525	0.21\\
77.8492032937365	0.21\\
77.8992059935205	0.21\\
77.9492086933045	0.21\\
77.9992113930886	0.21\\
78.0492140928726	0.21\\
78.0992167926566	0.21\\
78.1492194924406	0.21\\
78.1992221922246	0.21\\
78.2492248920086	0.21\\
78.2992275917927	0.21\\
78.3492302915767	0.21\\
78.3992329913607	0.21\\
78.4492356911447	0.21\\
78.4992383909287	0.21\\
78.5492410907127	0.21\\
78.5992437904968	0.21\\
78.6492464902808	0.21\\
78.6992491900648	0.21\\
78.7492518898488	0.21\\
78.7992545896328	0.21\\
78.8492572894169	0.21\\
78.8992599892009	0.21\\
78.9492626889849	0.21\\
78.9992653887689	0.21\\
79.0492680885529	0.21\\
79.0992707883369	0.21\\
79.149273488121	0.21\\
79.199276187905	0.21\\
79.249278887689	0.21\\
79.299281587473	0.21\\
79.349284287257	0.21\\
79.3992869870411	0.21\\
79.4492896868251	0.21\\
79.4992923866091	0.21\\
79.5492950863931	0.21\\
79.5992977861771	0.21\\
79.6493004859611	0.21\\
79.6993031857451	0.21\\
79.7493058855292	0.21\\
79.7993085853132	0.21\\
79.8493112850972	0.21\\
79.8993139848812	0.21\\
79.9493166846652	0.21\\
79.9993193844493	0.21\\
80.0493220842333	0.21\\
80.0993247840173	0.21\\
80.1493274838013	0.21\\
80.1993301835853	0.21\\
80.2493328833693	0.21\\
80.2993355831534	0.21\\
80.3493382829374	0.21\\
80.3993409827214	0.21\\
80.4493436825054	0.21\\
80.4993463822894	0.21\\
80.5493490820734	0.21\\
80.5993517818575	0.21\\
80.6493544816415	0.21\\
80.6993571814255	0.21\\
80.7493598812095	0.21\\
80.7993625809935	0.21\\
80.8493652807775	0.21\\
80.8993679805616	0.21\\
80.9493706803456	0.21\\
80.9993733801296	0.21\\
81.0493760799136	0.21\\
81.0993787796976	0.21\\
81.1493814794816	0.21\\
81.1993841792657	0.21\\
81.2493868790497	0.21\\
81.2993895788337	0.21\\
81.3493922786177	0.21\\
81.3993949784017	0.21\\
81.4493976781857	0.21\\
81.4994003779698	0.21\\
81.5494030777538	0.21\\
81.5994057775378	0.21\\
81.6494084773218	0.21\\
81.6994111771058	0.21\\
81.7494138768898	0.21\\
81.7994165766739	0.21\\
81.8494192764579	0.21\\
81.8994219762419	0.21\\
81.9494246760259	0.21\\
81.9994273758099	0.21\\
82.049430075594	0.21\\
82.099432775378	0.21\\
82.149435475162	0.21\\
82.199438174946	0.21\\
82.24944087473	0.21\\
82.299443574514	0.21\\
82.3494462742981	0.21\\
82.3994489740821	0.21\\
82.4494516738661	0.21\\
82.4994543736501	0.21\\
82.5494570734341	0.21\\
82.5994597732181	0.21\\
82.6294613930885	0.01\\
82.6794640928726	0.01\\
82.7294667926566	0.01\\
82.7794694924406	0.01\\
82.8294721922246	0.01\\
82.8794748920086	0.01\\
82.9294775917927	0.01\\
82.9794802915767	0.01\\
83.0294829913607	0.01\\
83.0794856911447	0.01\\
83.1294883909287	0.01\\
83.1794910907127	0.01\\
83.2294937904968	0.01\\
83.2794964902808	0.01\\
83.3294991900648	0.01\\
83.3795018898488	0.01\\
83.4295045896328	0.01\\
83.4795072894169	0.01\\
83.5295099892009	0.01\\
83.5795126889849	0.01\\
83.6295153887689	0.01\\
83.6795180885529	0.01\\
83.7295207883369	0.01\\
83.779523488121	0.01\\
83.829526187905	0.01\\
83.879528887689	0.01\\
83.929531587473	0.01\\
83.979534287257	0.01\\
84.029536987041	0.01\\
84.0795396868251	0.01\\
84.1295423866091	0.01\\
84.1795450863931	0.01\\
84.2295477861771	0.01\\
84.2795504859611	0.01\\
84.3295531857451	0.01\\
84.3795558855292	0.01\\
84.4295585853132	0.01\\
84.4795612850972	0.01\\
84.5295639848812	0.01\\
84.5795666846652	0.01\\
84.6295693844492	0.01\\
84.6795720842333	0.01\\
84.7295747840173	0.01\\
84.7795774838013	0.01\\
84.8295801835853	0.01\\
84.8795828833693	0.01\\
84.9295855831534	0.01\\
84.9795882829374	0.01\\
85.0295909827214	0.01\\
85.0795936825054	0.01\\
85.1295963822894	0.01\\
85.1795990820734	0.01\\
85.2296017818575	0.01\\
85.2796044816415	0.01\\
85.3296071814255	0.01\\
85.3796098812095	0.01\\
85.4296125809935	0.01\\
85.4796152807775	0.01\\
85.5296179805616	0.01\\
85.5796206803456	0.01\\
85.6296233801296	0.01\\
85.6796260799136	0.01\\
85.7296287796976	0.01\\
85.7796314794817	0.01\\
85.8296341792657	0.01\\
85.8796368790497	0.01\\
85.9296395788337	0.01\\
85.9796422786177	0.01\\
86.0296449784017	0.01\\
86.0796476781857	0.01\\
86.1296503779698	0.01\\
86.1796530777538	0.01\\
86.2296557775378	0.01\\
86.2796584773218	0.01\\
86.3296611771058	0.01\\
86.3796638768899	0.01\\
86.4296665766739	0.01\\
86.4796692764579	0.01\\
86.5296719762419	0.01\\
86.5796746760259	0.01\\
86.6296773758099	0.01\\
86.679680075594	0.01\\
86.729682775378	0.01\\
86.779685475162	0.01\\
86.829688174946	0.01\\
86.87969087473	0.01\\
86.929693574514	0.01\\
86.9796962742981	0.01\\
87.0296989740821	0.01\\
87.0797016738661	0.01\\
87.1297043736501	0.01\\
87.1797070734341	0.01\\
87.2297097732181	0.01\\
87.2797124730022	0.01\\
87.3297151727862	0.01\\
87.3797178725702	0.01\\
87.4297205723542	0.01\\
87.4797232721382	0.01\\
87.5297259719222	0.01\\
87.5797286717063	0.01\\
87.6297313714903	0.01\\
87.6797340712743	0.01\\
87.7297367710583	0.01\\
87.7797394708423	0.01\\
87.8297421706264	0.01\\
87.8797448704104	0.01\\
87.9297475701944	0.01\\
87.9797502699784	0.01\\
88.0297529697624	0.01\\
88.0797556695464	0.01\\
88.1297583693305	0.01\\
88.1797610691145	0.01\\
88.2297637688985	0.01\\
88.2797664686825	0.01\\
88.3297691684665	0.01\\
88.3797718682505	0.01\\
88.4297745680346	0.01\\
88.4797772678186	0.01\\
88.5297799676026	0.01\\
88.5797826673866	0.01\\
88.6297853671706	0.01\\
88.6797880669546	0.01\\
88.7297907667387	0.01\\
88.7797934665227	0.01\\
88.8297961663067	0.01\\
88.8797988660907	0.01\\
88.9298015658747	0.01\\
88.9798042656588	0.01\\
89.0298069654428	0.01\\
89.0798096652268	0.01\\
89.1298123650108	0.01\\
89.1798150647948	0.01\\
89.2298177645788	0.01\\
89.2798204643629	0.01\\
89.3298231641469	0.01\\
89.3798258639309	0.01\\
89.4298285637149	0.01\\
89.4798312634989	0.01\\
89.5298339632829	0.01\\
89.579836663067	0.01\\
89.629839362851	0.01\\
89.679842062635	0.01\\
89.729844762419	0.01\\
89.779847462203	0.01\\
89.8298501619871	0.01\\
89.8798528617711	0.01\\
89.9298555615551	0.01\\
89.9798582613391	0.01\\
90.0298609611231	0.01\\
90.0798636609071	0.01\\
90.1298663606911	0.01\\
90.1798690604752	0.01\\
90.2298717602592	0.01\\
90.2798744600432	0.01\\
90.3298771598272	0.01\\
90.3798798596112	0.01\\
90.4298825593953	0.01\\
90.4798852591793	0.01\\
90.5298879589633	0.01\\
90.5798906587473	0.01\\
90.6298933585313	0.01\\
90.6798960583153	0.01\\
90.7298987580994	0.01\\
90.7799014578834	0.01\\
90.8299041576674	0.01\\
90.8799068574514	0.01\\
90.9299095572354	0.01\\
90.9799122570195	0.01\\
91.0299149568035	0.01\\
91.0799176565875	0.01\\
91.1299203563715	0.01\\
91.1799230561555	0.01\\
91.2299257559395	0.01\\
91.2799284557236	0.01\\
91.3299311555076	0.01\\
91.3799338552916	0.01\\
91.4299365550756	0.01\\
91.4799392548596	0.01\\
91.5299419546436	0.01\\
91.5799446544276	0.01\\
91.6299473542117	0.01\\
91.6799500539957	0.01\\
91.7299527537797	0.01\\
91.7799554535637	0.01\\
91.8299581533477	0.01\\
91.8799608531317	0.01\\
91.9299635529158	0.01\\
91.9799662526998	0.01\\
92.0299689524838	0.01\\
92.0799716522678	0.01\\
92.1299743520518	0.01\\
92.1799770518358	0.01\\
92.2299797516199	0.01\\
92.2799824514039	0.01\\
92.3299851511879	0.01\\
92.3799878509719	0.01\\
92.4299905507559	0.01\\
92.47999325054	0.01\\
92.529995950324	0.01\\
92.579998650108	0.01\\
92.605	0.01\\
};
\addlegendentry{+ 1cm};

\addplot [color=gray,solid,line width=0.2pt]
  table[row sep=crcr]{0	-0.01\\
0.0500026997840173	-0.01\\
0.100005399568035	-0.01\\
0.150008099352052	-0.01\\
0.200010799136069	-0.01\\
0.250013498920086	-0.01\\
0.300016198704104	-0.01\\
0.350018898488121	-0.01\\
0.400021598272138	-0.01\\
0.450024298056156	-0.01\\
0.500026997840173	-0.01\\
0.55002969762419	-0.01\\
0.600032397408207	-0.01\\
0.650035097192225	-0.01\\
0.700037796976242	-0.01\\
0.750040496760259	-0.01\\
0.800043196544276	-0.01\\
0.850045896328294	-0.01\\
0.900048596112311	-0.01\\
0.950051295896328	-0.01\\
1.00005399568035	-0.01\\
1.05005669546436	-0.01\\
1.10005939524838	-0.01\\
1.1500620950324	-0.01\\
1.20006479481641	-0.01\\
1.25006749460043	-0.01\\
1.30007019438445	-0.01\\
1.35007289416847	-0.01\\
1.40007559395248	-0.01\\
1.4500782937365	-0.01\\
1.50008099352052	-0.01\\
1.55008369330454	-0.01\\
1.60008639308855	-0.01\\
1.65008909287257	-0.01\\
1.70009179265659	-0.01\\
1.7500944924406	-0.01\\
1.80009719222462	-0.01\\
1.85009989200864	-0.01\\
1.90010259179266	-0.01\\
1.95010529157667	-0.01\\
2.00010799136069	-0.01\\
2.05011069114471	-0.01\\
2.10011339092873	-0.01\\
2.15011609071274	-0.01\\
2.20011879049676	-0.01\\
2.25012149028078	-0.01\\
2.30012419006479	-0.01\\
2.35012688984881	-0.01\\
2.40012958963283	-0.01\\
2.45013228941685	-0.01\\
2.50013498920086	-0.01\\
2.55013768898488	-0.01\\
2.6001403887689	-0.01\\
2.65014308855292	-0.01\\
2.70014578833693	-0.01\\
2.75014848812095	-0.01\\
2.80015118790497	-0.01\\
2.85015388768899	-0.01\\
2.900156587473	-0.01\\
2.95015928725702	-0.01\\
3.00016198704104	-0.01\\
3.05016468682505	-0.01\\
3.10016738660907	-0.01\\
3.15017008639309	-0.01\\
3.20017278617711	-0.01\\
3.25017548596112	-0.01\\
3.30017818574514	-0.01\\
3.35018088552916	-0.01\\
3.40018358531318	-0.01\\
3.45018628509719	-0.01\\
3.50018898488121	-0.01\\
3.55019168466523	-0.01\\
3.60019438444924	-0.01\\
3.65019708423326	-0.01\\
3.70019978401728	-0.01\\
3.7502024838013	-0.01\\
3.80020518358531	-0.01\\
3.85020788336933	-0.01\\
3.90021058315335	-0.01\\
3.95021328293736	-0.01\\
4.00021598272138	-0.01\\
4.0502186825054	-0.01\\
4.10022138228942	-0.01\\
4.15022408207343	-0.01\\
4.20022678185745	-0.01\\
4.25022948164147	-0.01\\
4.30023218142549	-0.01\\
4.3502348812095	-0.01\\
4.40023758099352	-0.01\\
4.45024028077754	-0.01\\
4.50024298056155	-0.01\\
4.55024568034557	-0.01\\
4.60024838012959	-0.01\\
4.65025107991361	-0.01\\
4.70025377969762	-0.01\\
4.75025647948164	-0.01\\
4.80025917926566	-0.01\\
4.85026187904968	-0.01\\
4.90026457883369	-0.01\\
4.95026727861771	-0.01\\
5.00026997840173	-0.01\\
5.05027267818575	-0.01\\
5.10027537796976	-0.01\\
5.15027807775378	-0.01\\
5.2002807775378	-0.01\\
5.25028347732181	-0.01\\
5.30028617710583	-0.01\\
5.35028887688985	-0.01\\
5.40029157667387	-0.01\\
5.45029427645788	-0.01\\
5.5002969762419	-0.01\\
5.55029967602592	-0.01\\
5.60030237580994	-0.01\\
5.65030507559395	-0.01\\
5.70030777537797	-0.01\\
5.75031047516199	-0.01\\
5.800313174946	-0.01\\
5.85031587473002	-0.01\\
5.90031857451404	-0.01\\
5.95032127429806	-0.01\\
6.00032397408207	-0.01\\
6.05032667386609	-0.01\\
6.10032937365011	-0.01\\
6.15033207343413	-0.01\\
6.20033477321814	-0.01\\
6.25033747300216	-0.01\\
6.30034017278618	-0.01\\
6.3503428725702	-0.01\\
6.40034557235421	-0.01\\
6.45034827213823	-0.01\\
6.50035097192225	-0.01\\
6.55035367170626	-0.01\\
6.60035637149028	-0.01\\
6.6503590712743	-0.01\\
6.70036177105832	-0.01\\
6.75036447084233	-0.01\\
6.80036717062635	-0.01\\
6.85036987041037	-0.01\\
6.90037257019439	-0.01\\
6.9503752699784	-0.01\\
7.00037796976242	-0.01\\
7.05038066954644	-0.01\\
7.10038336933045	-0.01\\
7.15038606911447	-0.01\\
7.20038876889849	-0.01\\
7.25039146868251	-0.01\\
7.30039416846652	-0.01\\
7.35039686825054	-0.01\\
7.40039956803456	-0.01\\
7.45040226781857	-0.01\\
7.50040496760259	-0.01\\
7.55040766738661	-0.01\\
7.60041036717063	-0.01\\
7.65041306695464	-0.01\\
7.70041576673866	-0.01\\
7.75041846652268	-0.01\\
7.8004211663067	-0.01\\
7.85042386609071	-0.01\\
7.90042656587473	-0.01\\
7.95042926565875	-0.01\\
8.00043196544276	-0.01\\
8.05043466522678	-0.01\\
8.1004373650108	-0.01\\
8.15044006479482	-0.01\\
8.20044276457883	-0.01\\
8.25044546436285	-0.01\\
8.30044816414687	-0.01\\
8.35045086393089	-0.01\\
8.4004535637149	-0.01\\
8.45045626349892	-0.01\\
8.50045896328294	-0.01\\
8.55046166306696	-0.01\\
8.60046436285097	-0.01\\
8.65046706263499	-0.01\\
8.70046976241901	-0.01\\
8.75047246220302	-0.01\\
8.80047516198704	-0.01\\
8.85047786177106	-0.01\\
8.90048056155508	-0.01\\
8.95048326133909	-0.01\\
9.00048596112311	-0.01\\
9.05048866090713	-0.01\\
9.10049136069114	-0.01\\
9.15049406047516	-0.01\\
9.20049676025918	-0.01\\
9.2504994600432	-0.01\\
9.30050215982721	-0.01\\
9.35050485961123	-0.01\\
9.40050755939525	-0.01\\
9.45051025917927	-0.01\\
9.50051295896328	-0.01\\
9.5505156587473	-0.01\\
9.60051835853132	-0.01\\
9.65052105831533	-0.01\\
9.70052375809935	-0.01\\
9.75052645788337	-0.01\\
9.80052915766739	-0.01\\
9.8505318574514	-0.01\\
9.90053455723542	-0.01\\
9.95053725701944	-0.01\\
10.0005399568035	-0.01\\
10.0505426565875	-0.01\\
10.1005453563715	-0.01\\
10.1505480561555	-0.01\\
10.2005507559395	-0.01\\
10.2505534557235	-0.01\\
10.3005561555076	-0.01\\
10.3505588552916	-0.01\\
10.4005615550756	-0.01\\
10.4505642548596	-0.01\\
10.5005669546436	-0.01\\
10.5505696544276	-0.01\\
10.6005723542117	-0.01\\
10.6505750539957	-0.01\\
10.7005777537797	-0.01\\
10.7505804535637	-0.01\\
10.8005831533477	-0.01\\
10.8505858531317	-0.01\\
10.9005885529158	-0.01\\
10.9505912526998	-0.01\\
11.0005939524838	-0.01\\
11.0505966522678	-0.01\\
11.1005993520518	-0.01\\
11.1506020518359	-0.01\\
11.2006047516199	-0.01\\
11.2506074514039	-0.01\\
11.3006101511879	-0.01\\
11.3506128509719	-0.01\\
11.4006155507559	-0.01\\
11.45061825054	-0.01\\
11.500620950324	-0.01\\
11.550623650108	-0.01\\
11.600626349892	-0.01\\
11.650629049676	-0.01\\
11.70063174946	-0.01\\
11.7506344492441	-0.01\\
11.8006371490281	-0.01\\
11.8506398488121	-0.01\\
11.9006425485961	-0.01\\
11.9506452483801	-0.01\\
12.0006479481641	-0.01\\
12.0506506479482	-0.01\\
12.1006533477322	-0.01\\
12.1506560475162	-0.01\\
12.2006587473002	-0.01\\
12.2506614470842	-0.01\\
12.3006641468683	-0.01\\
12.3506668466523	-0.01\\
12.4006695464363	-0.01\\
12.4506722462203	-0.01\\
12.5006749460043	-0.01\\
12.5506776457883	-0.01\\
12.6006803455724	-0.01\\
12.6506830453564	-0.01\\
12.7006857451404	-0.01\\
12.7506884449244	-0.01\\
12.8006911447084	-0.01\\
12.8506938444924	-0.01\\
12.9006965442765	-0.01\\
12.9506992440605	-0.01\\
13.0007019438445	-0.01\\
13.0507046436285	-0.01\\
13.1007073434125	-0.01\\
13.1507100431965	-0.01\\
13.2007127429806	-0.01\\
13.2507154427646	-0.01\\
13.3007181425486	-0.01\\
13.3507208423326	-0.01\\
13.4007235421166	-0.01\\
13.4507262419006	-0.01\\
13.5007289416847	-0.01\\
13.5507316414687	-0.01\\
13.6007343412527	-0.01\\
13.6507370410367	-0.01\\
13.7007397408207	-0.01\\
13.7507424406048	-0.01\\
13.8007451403888	-0.01\\
13.8507478401728	-0.01\\
13.9007505399568	-0.01\\
13.9507532397408	-0.01\\
14.0007559395248	-0.01\\
14.0507586393089	-0.01\\
14.1007613390929	-0.01\\
14.1507640388769	-0.01\\
14.2007667386609	-0.01\\
14.2507694384449	-0.01\\
14.3007721382289	-0.01\\
14.350774838013	-0.01\\
14.400777537797	-0.01\\
14.450780237581	-0.01\\
14.500782937365	-0.01\\
14.550785637149	-0.01\\
14.600788336933	-0.01\\
14.6507910367171	-0.01\\
14.7007937365011	-0.01\\
14.7507964362851	-0.01\\
14.8007991360691	-0.01\\
14.8508018358531	-0.01\\
14.9008045356372	-0.01\\
14.9508072354212	-0.01\\
15.0008099352052	-0.01\\
15.0508126349892	-0.01\\
15.1008153347732	-0.01\\
15.1508180345572	-0.01\\
15.2008207343413	-0.01\\
15.2508234341253	-0.01\\
15.3008261339093	-0.01\\
15.3508288336933	-0.01\\
15.4008315334773	-0.01\\
15.4508342332613	-0.01\\
15.5008369330454	-0.01\\
15.5508396328294	-0.01\\
15.6008423326134	-0.01\\
15.6508450323974	-0.01\\
15.7008477321814	-0.01\\
15.7508504319654	-0.01\\
15.8008531317495	-0.01\\
15.8508558315335	-0.01\\
15.9008585313175	-0.01\\
15.9508612311015	-0.01\\
16.0008639308855	-0.01\\
16.0508666306695	-0.01\\
16.1008693304536	-0.01\\
16.1508720302376	-0.01\\
16.2008747300216	-0.01\\
16.2508774298056	-0.01\\
16.3008801295896	-0.01\\
16.3508828293737	-0.01\\
16.4008855291577	-0.01\\
16.4508882289417	-0.01\\
16.5008909287257	-0.01\\
16.5508936285097	-0.01\\
16.6008963282937	-0.01\\
16.6508990280778	-0.01\\
16.7009017278618	-0.01\\
16.7509044276458	-0.01\\
16.8009071274298	-0.01\\
16.8509098272138	-0.01\\
16.9009125269978	-0.01\\
16.9509152267819	-0.01\\
17.0009179265659	-0.01\\
17.0509206263499	-0.01\\
17.1009233261339	-0.01\\
17.1509260259179	-0.01\\
17.2009287257019	-0.01\\
17.250931425486	-0.01\\
17.30093412527	-0.01\\
17.350936825054	-0.01\\
17.400939524838	-0.01\\
17.450942224622	-0.01\\
17.500944924406	-0.01\\
17.5509476241901	-0.01\\
17.6009503239741	-0.01\\
17.6509530237581	-0.01\\
17.7009557235421	-0.01\\
17.7509584233261	-0.01\\
17.8009611231102	-0.01\\
17.8509638228942	-0.01\\
17.9009665226782	-0.01\\
17.9509692224622	-0.01\\
18.0009719222462	-0.01\\
18.0509746220302	-0.01\\
18.1009773218143	-0.01\\
18.1509800215983	-0.01\\
18.2009827213823	-0.01\\
18.2509854211663	-0.01\\
18.3009881209503	-0.01\\
18.3509908207343	-0.01\\
18.4009935205184	-0.01\\
18.4509962203024	-0.01\\
18.5009989200864	-0.01\\
18.5510016198704	-0.01\\
18.6010043196544	-0.01\\
18.6510070194384	-0.01\\
18.7010097192225	-0.01\\
18.7510124190065	-0.01\\
18.8010151187905	-0.01\\
18.8510178185745	-0.01\\
18.9010205183585	-0.01\\
18.9510232181425	-0.01\\
19.0010259179266	-0.01\\
19.0510286177106	-0.01\\
19.1010313174946	-0.01\\
19.1510340172786	-0.01\\
19.2010367170626	-0.01\\
19.2510394168467	-0.01\\
19.3010421166307	-0.01\\
19.3510448164147	-0.01\\
19.4010475161987	-0.01\\
19.4510502159827	-0.01\\
19.5010529157667	-0.01\\
19.5510556155508	-0.01\\
19.6010583153348	-0.01\\
19.6510610151188	-0.01\\
19.7010637149028	-0.01\\
19.7510664146868	-0.01\\
19.8010691144708	-0.01\\
19.8510718142549	-0.01\\
19.9010745140389	-0.01\\
19.9510772138229	-0.01\\
20.0010799136069	-0.01\\
20.0510826133909	-0.01\\
20.1010853131749	-0.01\\
20.151088012959	-0.01\\
20.201090712743	-0.01\\
20.251093412527	-0.01\\
20.301096112311	-0.01\\
20.351098812095	-0.01\\
20.4011015118791	-0.01\\
20.4511042116631	-0.01\\
20.5011069114471	-0.01\\
20.5511096112311	-0.01\\
20.6011123110151	-0.01\\
20.6511150107991	-0.01\\
20.7011177105832	-0.01\\
20.7511204103672	-0.01\\
20.8011231101512	-0.01\\
20.8511258099352	-0.01\\
20.9011285097192	-0.01\\
20.9511312095032	-0.01\\
21.0011339092873	-0.01\\
21.0511366090713	-0.01\\
21.1011393088553	-0.01\\
21.1511420086393	-0.01\\
21.2011447084233	-0.01\\
21.2511474082073	-0.01\\
21.3011501079914	-0.01\\
21.3511528077754	-0.01\\
21.4011555075594	-0.01\\
21.4511582073434	-0.01\\
21.5011609071274	-0.01\\
21.5511636069114	-0.01\\
21.6011663066955	-0.01\\
21.6511690064795	-0.01\\
21.7011717062635	-0.01\\
21.7511744060475	-0.01\\
21.8011771058315	-0.01\\
21.8511798056156	-0.01\\
21.9011825053996	-0.01\\
21.9511852051836	-0.01\\
22.0011879049676	-0.01\\
22.0511906047516	-0.01\\
22.1011933045356	-0.01\\
22.1511960043197	-0.01\\
22.2011987041037	-0.01\\
22.2512014038877	-0.01\\
22.3012041036717	-0.01\\
22.3512068034557	-0.01\\
22.4012095032397	-0.01\\
22.4512122030238	-0.01\\
22.5012149028078	-0.01\\
22.5512176025918	-0.01\\
22.6012203023758	-0.01\\
22.6512230021598	-0.01\\
22.7012257019438	-0.01\\
22.7512284017279	-0.01\\
22.8012311015119	-0.01\\
22.8512338012959	-0.01\\
22.9012365010799	-0.01\\
22.9512392008639	-0.01\\
23.0012419006479	-0.01\\
23.051244600432	-0.01\\
23.101247300216	-0.01\\
23.15125	-0.01\\
23.201252699784	-0.01\\
23.251255399568	-0.01\\
23.3012580993521	-0.01\\
23.3512607991361	-0.01\\
23.4012634989201	-0.01\\
23.4512661987041	-0.01\\
23.5012688984881	-0.01\\
23.5512715982721	-0.01\\
23.6012742980562	-0.01\\
23.6512769978402	-0.01\\
23.7012796976242	-0.01\\
23.7512823974082	-0.01\\
23.8012850971922	-0.01\\
23.8512877969762	-0.01\\
23.9012904967603	-0.01\\
23.9512931965443	-0.01\\
24.0012958963283	-0.01\\
24.0512985961123	-0.01\\
24.1013012958963	-0.01\\
24.1513039956803	-0.01\\
24.2013066954644	-0.01\\
24.2513093952484	-0.01\\
24.3013120950324	-0.01\\
24.3513147948164	-0.01\\
24.4013174946004	-0.01\\
24.4513201943845	-0.01\\
24.5013228941685	-0.01\\
24.5513255939525	-0.01\\
24.6013282937365	-0.01\\
24.6513309935205	-0.01\\
24.7013336933045	-0.01\\
24.7513363930886	-0.01\\
24.8013390928726	-0.01\\
24.8513417926566	-0.01\\
24.9013444924406	-0.01\\
24.9513471922246	-0.01\\
25.0013498920086	-0.01\\
25.0513525917927	-0.01\\
25.1013552915767	-0.01\\
25.1513579913607	-0.01\\
25.2013606911447	-0.01\\
25.2513633909287	-0.01\\
25.3013660907127	-0.01\\
25.3513687904968	-0.01\\
25.4013714902808	-0.01\\
25.4513741900648	-0.01\\
25.5013768898488	-0.01\\
25.5513795896328	-0.01\\
25.6013822894168	-0.01\\
25.6513849892009	-0.01\\
25.7013876889849	-0.01\\
25.7513903887689	-0.01\\
25.8013930885529	-0.01\\
25.8513957883369	-0.01\\
25.901398488121	-0.01\\
25.951401187905	-0.01\\
26.001403887689	-0.01\\
26.051406587473	-0.01\\
26.101409287257	-0.01\\
26.151411987041	-0.01\\
26.2014146868251	-0.01\\
26.2514173866091	-0.01\\
26.3014200863931	-0.01\\
26.3514227861771	-0.01\\
26.4014254859611	-0.01\\
26.4514281857451	-0.01\\
26.5014308855292	-0.01\\
26.5514335853132	-0.01\\
26.6014362850972	-0.01\\
26.6514389848812	-0.01\\
26.7014416846652	-0.01\\
26.7514443844492	-0.01\\
26.8014470842333	-0.01\\
26.8514497840173	-0.01\\
26.9014524838013	-0.01\\
26.9514551835853	-0.01\\
27.0014578833693	-0.01\\
27.0514605831534	-0.01\\
27.1014632829374	-0.01\\
27.1514659827214	-0.01\\
27.2014686825054	-0.01\\
27.2514713822894	-0.01\\
27.3014740820734	-0.01\\
27.3514767818575	-0.01\\
27.4014794816415	-0.01\\
27.4514821814255	-0.01\\
27.5014848812095	-0.01\\
27.5514875809935	-0.01\\
27.6014902807775	-0.01\\
27.6514929805616	-0.01\\
27.7014956803456	-0.01\\
27.7514983801296	-0.01\\
27.8015010799136	-0.01\\
27.8515037796976	-0.01\\
27.9015064794816	-0.01\\
27.9515091792657	-0.01\\
28.0015118790497	-0.01\\
28.0515145788337	-0.01\\
28.1015172786177	-0.01\\
28.1515199784017	-0.01\\
28.2015226781857	-0.01\\
28.2515253779698	-0.01\\
28.3015280777538	-0.01\\
28.3515307775378	-0.01\\
28.4015334773218	-0.01\\
28.4515361771058	-0.01\\
28.5015388768899	-0.01\\
28.5515415766739	-0.01\\
28.6015442764579	-0.01\\
28.6515469762419	-0.01\\
28.7015496760259	-0.01\\
28.7515523758099	-0.01\\
28.801555075594	-0.01\\
28.851557775378	-0.01\\
28.901560475162	-0.01\\
28.951563174946	-0.01\\
29.00156587473	-0.01\\
29.051568574514	-0.01\\
29.1015712742981	-0.01\\
29.1515739740821	-0.01\\
29.2015766738661	-0.01\\
29.2515793736501	-0.01\\
29.3015820734341	-0.01\\
29.3515847732181	-0.01\\
29.4015874730022	-0.01\\
29.4515901727862	-0.01\\
29.5015928725702	-0.01\\
29.5515955723542	-0.01\\
29.6015982721382	-0.01\\
29.6516009719222	-0.01\\
29.7016036717063	-0.01\\
29.7516063714903	-0.01\\
29.8016090712743	-0.01\\
29.8516117710583	-0.01\\
29.9016144708423	-0.01\\
29.9516171706264	-0.01\\
30.0016198704104	-0.01\\
30.0516225701944	-0.01\\
30.1016252699784	-0.01\\
30.1516279697624	-0.01\\
30.2016306695464	-0.01\\
30.2516333693305	-0.01\\
30.3016360691145	-0.01\\
30.3516387688985	-0.01\\
30.4016414686825	-0.01\\
30.4516441684665	-0.01\\
30.5016468682505	-0.01\\
30.5516495680346	-0.01\\
30.6016522678186	-0.01\\
30.6516549676026	-0.01\\
30.7016576673866	-0.01\\
30.7516603671706	-0.01\\
30.8016630669546	-0.01\\
30.8516657667387	-0.01\\
30.9016684665227	-0.01\\
30.9516711663067	-0.01\\
31.0016738660907	-0.01\\
31.0516765658747	-0.01\\
31.1016792656587	-0.01\\
31.1516819654428	-0.01\\
31.2016846652268	-0.01\\
31.2516873650108	-0.01\\
31.3016900647948	-0.01\\
31.3516927645788	-0.01\\
31.4016954643629	-0.01\\
31.4516981641469	-0.01\\
31.5017008639309	-0.01\\
31.5517035637149	-0.01\\
31.6017062634989	-0.01\\
31.6517089632829	-0.01\\
31.701711663067	-0.01\\
31.751714362851	-0.01\\
31.801717062635	-0.01\\
31.851719762419	-0.01\\
31.901722462203	-0.01\\
31.951725161987	-0.01\\
32.0017278617711	-0.01\\
32.0517305615551	-0.01\\
32.1017332613391	-0.01\\
32.1517359611231	-0.01\\
32.2017386609071	-0.01\\
32.2517413606911	-0.01\\
32.3017440604752	-0.01\\
32.3517467602592	-0.01\\
32.4017494600432	-0.01\\
32.4517521598272	-0.01\\
32.5017548596112	-0.01\\
32.5517575593952	-0.01\\
32.6017602591793	-0.01\\
32.6517629589633	-0.01\\
32.7017656587473	-0.01\\
32.7517683585313	-0.01\\
32.8017710583153	-0.01\\
32.8517737580993	-0.01\\
32.9017764578834	-0.01\\
32.9517791576674	-0.01\\
33.0017818574514	-0.01\\
33.0517845572354	-0.01\\
33.1017872570194	-0.01\\
33.1517899568035	-0.01\\
33.2017926565875	-0.01\\
33.2517953563715	-0.01\\
33.3017980561555	-0.01\\
33.3518007559395	-0.01\\
33.4018034557235	-0.01\\
33.4518061555076	-0.01\\
33.5018088552916	-0.01\\
33.5518115550756	-0.01\\
33.6018142548596	-0.01\\
33.6518169546436	-0.01\\
33.7018196544277	-0.01\\
33.7518223542117	-0.01\\
33.8018250539957	-0.01\\
33.8518277537797	-0.01\\
33.9018304535637	-0.01\\
33.9518331533477	-0.01\\
34.0018358531318	-0.01\\
34.0518385529158	-0.01\\
34.1018412526998	-0.01\\
34.1518439524838	-0.01\\
34.2018466522678	-0.01\\
34.2518493520518	-0.01\\
34.3018520518359	-0.01\\
34.3518547516199	-0.01\\
34.4018574514039	-0.01\\
34.4518601511879	-0.01\\
34.5018628509719	-0.01\\
34.5518655507559	-0.01\\
34.60186825054	-0.01\\
34.651870950324	-0.01\\
34.701873650108	-0.01\\
34.751876349892	-0.01\\
34.801879049676	-0.01\\
34.85188174946	-0.01\\
34.9018844492441	-0.01\\
34.9518871490281	-0.01\\
35.0018898488121	-0.01\\
35.0518925485961	-0.01\\
35.1018952483801	-0.01\\
35.1518979481641	-0.01\\
35.2019006479482	-0.01\\
35.2519033477322	-0.01\\
35.3019060475162	-0.01\\
35.3519087473002	-0.01\\
35.4019114470842	-0.01\\
35.4519141468683	-0.01\\
35.5019168466523	-0.01\\
35.5519195464363	-0.01\\
35.6019222462203	-0.01\\
35.6519249460043	-0.01\\
35.7019276457883	-0.01\\
35.7519303455724	-0.01\\
35.8019330453564	-0.01\\
35.8519357451404	-0.01\\
35.9019384449244	-0.01\\
35.9519411447084	-0.01\\
36.0019438444924	-0.01\\
36.0519465442765	-0.01\\
36.1019492440605	-0.01\\
36.1519519438445	-0.01\\
36.2019546436285	-0.01\\
36.2519573434125	-0.01\\
36.3019600431965	-0.01\\
36.3519627429806	-0.01\\
36.4019654427646	-0.01\\
36.4519681425486	-0.01\\
36.5019708423326	-0.01\\
36.5519735421166	-0.01\\
36.6019762419006	-0.01\\
36.6519789416847	-0.01\\
36.7019816414687	-0.01\\
36.7519843412527	-0.01\\
36.8019870410367	-0.01\\
36.8519897408207	-0.01\\
36.9019924406048	-0.01\\
36.9519951403888	-0.01\\
37.0019978401728	-0.01\\
37.0520005399568	-0.01\\
37.1020032397408	-0.01\\
37.1520059395248	-0.01\\
37.2020086393089	-0.01\\
37.2520113390929	-0.01\\
37.3020140388769	-0.01\\
37.3520167386609	-0.01\\
37.4020194384449	-0.01\\
37.4520221382289	-0.01\\
37.502024838013	-0.01\\
37.552027537797	-0.01\\
37.602030237581	-0.01\\
37.652032937365	-0.01\\
37.702035637149	-0.01\\
37.752038336933	-0.01\\
37.8020410367171	-0.01\\
37.8520437365011	-0.01\\
37.9020464362851	-0.01\\
37.9520491360691	-0.01\\
38.0020518358531	-0.01\\
38.0520545356372	-0.01\\
38.1020572354212	-0.01\\
38.1520599352052	-0.01\\
38.2020626349892	-0.01\\
38.2520653347732	-0.01\\
38.3020680345572	-0.01\\
38.3520707343413	-0.01\\
38.4020734341253	-0.01\\
38.4520761339093	-0.01\\
38.5020788336933	-0.01\\
38.5520815334773	-0.01\\
38.6020842332613	-0.01\\
38.6520869330454	-0.01\\
38.7020896328294	-0.01\\
38.7520923326134	-0.01\\
38.8020950323974	-0.01\\
38.8520977321814	-0.01\\
38.9021004319654	-0.01\\
38.9521031317495	-0.01\\
39.0021058315335	-0.01\\
39.0521085313175	-0.01\\
39.1021112311015	-0.01\\
39.1521139308855	-0.01\\
39.2021166306696	-0.01\\
39.2521193304536	-0.01\\
39.3021220302376	-0.01\\
39.3521247300216	-0.01\\
39.4021274298056	-0.01\\
39.4521301295896	-0.01\\
39.5021328293737	-0.01\\
39.5521355291577	-0.01\\
39.6021382289417	-0.01\\
39.6521409287257	-0.01\\
39.7021436285097	-0.01\\
39.7521463282937	-0.01\\
39.8021490280778	-0.01\\
39.8521517278618	-0.01\\
39.9021544276458	-0.01\\
39.9521571274298	-0.01\\
40.0021598272138	-0.01\\
40.0521625269978	-0.01\\
40.1021652267819	-0.01\\
40.1521679265659	-0.01\\
40.2021706263499	-0.01\\
40.2521733261339	-0.01\\
40.3021760259179	-0.01\\
40.3521787257019	-0.01\\
40.402181425486	-0.01\\
40.45218412527	-0.01\\
40.502186825054	-0.01\\
40.552189524838	-0.01\\
40.602192224622	-0.01\\
40.652194924406	-0.01\\
40.7021976241901	-0.01\\
40.7522003239741	-0.01\\
40.8022030237581	-0.01\\
40.8522057235421	-0.01\\
40.9022084233261	-0.01\\
40.9522111231102	-0.01\\
41.0022138228942	-0.01\\
41.0522165226782	-0.01\\
41.1022192224622	-0.01\\
41.1522219222462	-0.01\\
41.2022246220302	-0.01\\
41.2522273218143	-0.01\\
41.3022300215983	-0.01\\
41.3522327213823	-0.01\\
41.4022354211663	-0.01\\
41.4522381209503	-0.01\\
41.5022408207343	-0.01\\
41.5522435205184	-0.01\\
41.6022462203024	-0.01\\
41.6522489200864	-0.01\\
41.7022516198704	-0.01\\
41.7522543196544	-0.01\\
41.8022570194385	-0.01\\
41.8522597192225	-0.01\\
41.9022624190065	-0.01\\
41.9522651187905	-0.01\\
42.0022678185745	-0.01\\
42.0522705183585	-0.01\\
42.1022732181426	-0.01\\
42.1522759179266	-0.01\\
42.2022786177106	-0.01\\
42.2522813174946	-0.01\\
42.3022840172786	-0.01\\
42.3522867170626	-0.01\\
42.4022894168467	-0.01\\
42.4522921166307	-0.01\\
42.5022948164147	-0.01\\
42.5522975161987	-0.01\\
42.6023002159827	-0.01\\
42.6523029157667	-0.01\\
42.7023056155508	-0.01\\
42.7523083153348	-0.01\\
42.8023110151188	-0.01\\
42.8523137149028	-0.01\\
42.9023164146868	-0.01\\
42.9523191144708	-0.01\\
43.0023218142549	-0.01\\
43.0523245140389	-0.01\\
43.1023272138229	-0.01\\
43.1523299136069	-0.01\\
43.2023326133909	-0.01\\
43.2523353131749	-0.01\\
43.302338012959	-0.01\\
43.352340712743	-0.01\\
43.402343412527	-0.01\\
43.452346112311	-0.01\\
43.502348812095	-0.01\\
43.5523515118791	-0.01\\
43.6023542116631	-0.01\\
43.6523569114471	-0.01\\
43.7023596112311	-0.01\\
43.7523623110151	-0.01\\
43.8023650107991	-0.01\\
43.8523677105832	-0.01\\
43.9023704103672	-0.01\\
43.9523731101512	-0.01\\
44.0023758099352	-0.01\\
44.0523785097192	-0.01\\
44.1023812095032	-0.01\\
44.1523839092873	-0.01\\
44.2023866090713	-0.01\\
44.2523893088553	-0.01\\
44.3023920086393	-0.01\\
44.3523947084233	-0.01\\
44.4023974082073	-0.01\\
44.4524001079914	-0.01\\
44.5024028077754	-0.01\\
44.5524055075594	-0.01\\
44.6024082073434	-0.01\\
44.6524109071274	-0.01\\
44.7024136069114	-0.01\\
44.7524163066955	-0.01\\
44.8024190064795	-0.01\\
44.8524217062635	-0.01\\
44.9024244060475	-0.01\\
44.9524271058315	-0.01\\
45.0024298056155	-0.01\\
45.0524325053996	-0.01\\
45.1024352051836	-0.01\\
45.1524379049676	-0.01\\
45.2024406047516	-0.01\\
45.2524433045356	-0.01\\
45.3024460043197	-0.01\\
45.3524487041037	-0.01\\
45.4024514038877	-0.01\\
45.4524541036717	-0.01\\
45.5024568034557	-0.01\\
45.5524595032397	-0.01\\
45.6024622030238	-0.01\\
45.6524649028078	-0.01\\
45.7024676025918	-0.01\\
45.7524703023758	-0.01\\
45.8024730021598	-0.01\\
45.8524757019439	-0.01\\
45.9024784017279	-0.01\\
45.9524811015119	-0.01\\
46.0024838012959	-0.01\\
46.0524865010799	-0.01\\
46.1024892008639	-0.01\\
46.152491900648	-0.01\\
46.202494600432	-0.01\\
46.252497300216	-0.01\\
46.3025	-0.01\\
46.352502699784	-0.01\\
46.402505399568	-0.01\\
46.4525080993521	-0.01\\
46.5025107991361	-0.01\\
46.5525134989201	-0.01\\
46.6025161987041	-0.01\\
46.6525188984881	-0.01\\
46.7025215982721	-0.01\\
46.7525242980562	-0.01\\
46.8025269978402	-0.01\\
46.8525296976242	-0.01\\
46.9025323974082	-0.01\\
46.9525350971922	-0.01\\
47.0025377969762	-0.01\\
47.0525404967603	-0.01\\
47.1025431965443	-0.01\\
47.1525458963283	-0.01\\
47.2025485961123	-0.01\\
47.2525512958963	-0.01\\
47.3025539956803	-0.01\\
47.3525566954644	-0.01\\
47.4025593952484	-0.01\\
47.4525620950324	-0.01\\
47.5025647948164	-0.01\\
47.5525674946004	-0.01\\
47.6025701943845	-0.01\\
47.6525728941685	-0.01\\
47.7025755939525	-0.01\\
47.7525782937365	-0.01\\
47.8025809935205	-0.01\\
47.8525836933045	-0.01\\
47.9025863930886	-0.01\\
47.9525890928726	-0.01\\
48.0025917926566	-0.01\\
48.0525944924406	-0.01\\
48.1025971922246	-0.01\\
48.1525998920086	-0.01\\
48.2026025917927	-0.01\\
48.2526052915767	-0.01\\
48.3026079913607	-0.01\\
48.3526106911447	-0.01\\
48.4026133909287	-0.01\\
48.4526160907127	-0.01\\
48.5026187904968	-0.01\\
48.5526214902808	-0.01\\
48.6026241900648	-0.01\\
48.6526268898488	-0.01\\
48.7026295896328	-0.01\\
48.7526322894168	-0.01\\
48.8026349892009	-0.01\\
48.8526376889849	-0.01\\
48.9026403887689	-0.01\\
48.9526430885529	-0.01\\
49.0026457883369	-0.01\\
49.052648488121	-0.01\\
49.102651187905	-0.01\\
49.152653887689	-0.01\\
49.202656587473	-0.01\\
49.252659287257	-0.01\\
49.302661987041	-0.01\\
49.3526646868251	-0.01\\
49.4026673866091	-0.01\\
49.4526700863931	-0.01\\
49.5026727861771	-0.01\\
49.5526754859611	-0.01\\
49.6026781857451	-0.01\\
49.6526808855292	-0.01\\
49.7026835853132	-0.01\\
49.7526862850972	-0.01\\
49.8026889848812	-0.01\\
49.8526916846652	-0.01\\
49.9026943844492	-0.01\\
49.9526970842333	-0.01\\
50.0026997840173	-0.01\\
50.0527024838013	-0.01\\
50.1027051835853	-0.01\\
50.1527078833693	-0.01\\
50.2027105831534	-0.01\\
50.2527132829374	-0.01\\
50.3027159827214	-0.01\\
50.3527186825054	-0.01\\
50.4027213822894	-0.01\\
50.4527240820734	-0.01\\
50.5027267818575	-0.01\\
50.5527294816415	-0.01\\
50.6027321814255	-0.01\\
50.6527348812095	-0.01\\
50.7027375809935	-0.01\\
50.7527402807775	-0.01\\
50.8027429805616	-0.01\\
50.8527456803456	-0.01\\
50.9027483801296	-0.01\\
50.9527510799136	-0.01\\
51.0027537796976	-0.01\\
51.0527564794816	-0.01\\
51.1027591792657	-0.01\\
51.1527618790497	-0.01\\
51.2027645788337	-0.01\\
51.2527672786177	-0.01\\
51.3027699784017	-0.01\\
51.3527726781858	-0.01\\
51.4027753779698	-0.01\\
51.4527780777538	-0.01\\
51.5027807775378	-0.01\\
51.5527834773218	-0.01\\
51.6027861771058	-0.01\\
51.6527888768899	-0.01\\
51.7027915766739	-0.01\\
51.7527942764579	-0.01\\
51.8027969762419	-0.01\\
51.8527996760259	-0.01\\
51.9028023758099	-0.01\\
51.952805075594	-0.01\\
52.002807775378	-0.01\\
52.052810475162	-0.01\\
52.102813174946	-0.01\\
52.15281587473	-0.01\\
52.202818574514	-0.01\\
52.2528212742981	-0.01\\
52.3028239740821	-0.01\\
52.3528266738661	-0.01\\
52.4028293736501	-0.01\\
52.4528320734341	-0.01\\
52.5028347732181	-0.01\\
52.5528374730022	-0.01\\
52.6028401727862	-0.01\\
52.6528428725702	-0.01\\
52.7028455723542	-0.01\\
52.7528482721382	-0.01\\
52.8028509719222	-0.01\\
52.8528536717063	-0.01\\
52.9028563714903	-0.01\\
52.9528590712743	-0.01\\
53.0028617710583	-0.01\\
53.0528644708423	-0.01\\
53.1028671706264	-0.01\\
53.1528698704104	-0.01\\
53.2028725701944	-0.01\\
53.2528752699784	-0.01\\
53.3028779697624	-0.01\\
53.3528806695464	-0.01\\
53.4028833693305	-0.01\\
53.4528860691145	-0.01\\
53.5028887688985	-0.01\\
53.5528914686825	-0.01\\
53.6028941684665	-0.01\\
53.6528968682505	-0.01\\
53.7028995680346	-0.01\\
53.7529022678186	-0.01\\
53.8029049676026	-0.01\\
53.8529076673866	-0.01\\
53.9029103671706	-0.01\\
53.9529130669547	-0.01\\
54.0029157667387	-0.01\\
54.0529184665227	-0.01\\
54.1029211663067	-0.01\\
54.1529238660907	-0.01\\
54.2029265658747	-0.01\\
54.2529292656588	-0.01\\
54.3029319654428	-0.01\\
54.3529346652268	-0.01\\
54.4029373650108	-0.01\\
54.4529400647948	-0.01\\
54.5029427645788	-0.01\\
54.5529454643629	-0.01\\
54.6029481641469	-0.01\\
54.6529508639309	-0.01\\
54.7029535637149	-0.01\\
54.7529562634989	-0.01\\
54.8029589632829	-0.01\\
54.852961663067	-0.01\\
54.902964362851	-0.01\\
54.952967062635	-0.01\\
55.002969762419	-0.01\\
55.052972462203	-0.01\\
55.102975161987	-0.01\\
55.1529778617711	-0.01\\
55.2029805615551	-0.01\\
55.2529832613391	-0.01\\
55.3029859611231	-0.01\\
55.3529886609071	-0.01\\
55.4029913606911	-0.01\\
55.4529940604752	-0.01\\
55.5029967602592	-0.01\\
55.5529994600432	-0.01\\
55.6030021598272	-0.01\\
55.6530048596112	-0.01\\
55.7030075593953	-0.01\\
55.7530102591793	-0.01\\
55.8030129589633	-0.01\\
55.8530156587473	-0.01\\
55.9030183585313	-0.01\\
55.9530210583153	-0.01\\
56.0030237580994	-0.01\\
56.0530264578834	-0.01\\
56.1030291576674	-0.01\\
56.1530318574514	-0.01\\
56.2030345572354	-0.01\\
56.2530372570194	-0.01\\
56.3030399568035	-0.01\\
56.3530426565875	-0.01\\
56.4030453563715	-0.01\\
56.4530480561555	-0.01\\
56.5030507559395	-0.01\\
56.5530534557235	-0.01\\
56.6030561555076	-0.01\\
56.6530588552916	-0.01\\
56.7030615550756	-0.01\\
56.7530642548596	-0.01\\
56.8030669546436	-0.01\\
56.8530696544276	-0.01\\
56.9030723542117	-0.01\\
56.9530750539957	-0.01\\
57.0030777537797	-0.01\\
57.0530804535637	-0.01\\
57.1030831533477	-0.01\\
57.1530858531318	-0.01\\
57.2030885529158	-0.01\\
57.2530912526998	-0.01\\
57.3030939524838	-0.01\\
57.3530966522678	-0.01\\
57.4030993520518	-0.01\\
57.4531020518358	-0.01\\
57.5031047516199	-0.01\\
57.5531074514039	-0.01\\
57.6031101511879	-0.01\\
57.6531128509719	-0.01\\
57.7031155507559	-0.01\\
57.75311825054	-0.01\\
57.803120950324	-0.01\\
57.853123650108	-0.01\\
57.903126349892	-0.01\\
57.953129049676	-0.01\\
58.00313174946	-0.01\\
58.0531344492441	-0.01\\
58.1031371490281	-0.01\\
58.1531398488121	-0.01\\
58.2031425485961	-0.01\\
58.2531452483801	-0.01\\
58.3031479481642	-0.01\\
58.3531506479482	-0.01\\
58.4031533477322	-0.01\\
58.4531560475162	-0.01\\
58.5031587473002	-0.01\\
58.5531614470842	-0.01\\
58.6031641468683	-0.01\\
58.6531668466523	-0.01\\
58.7031695464363	-0.01\\
58.7531722462203	-0.01\\
58.8031749460043	-0.01\\
58.8531776457883	-0.01\\
58.9031803455724	-0.01\\
58.9531830453564	-0.01\\
59.0031857451404	-0.01\\
59.0531884449244	-0.01\\
59.1031911447084	-0.01\\
59.1531938444924	-0.01\\
59.2031965442765	-0.01\\
59.2531992440605	-0.01\\
59.3032019438445	-0.01\\
59.3532046436285	-0.01\\
59.4032073434125	-0.01\\
59.4532100431965	-0.01\\
59.5032127429806	-0.01\\
59.5532154427646	-0.01\\
59.6032181425486	-0.01\\
59.6532208423326	-0.01\\
59.7032235421166	-0.01\\
59.7532262419006	-0.01\\
59.8032289416847	-0.01\\
59.8532316414687	-0.01\\
59.9032343412527	-0.01\\
59.9532370410367	-0.01\\
60.0032397408207	-0.01\\
60.0532424406048	-0.01\\
60.1032451403888	-0.01\\
60.1532478401728	-0.01\\
60.2032505399568	-0.01\\
60.2532532397408	-0.01\\
60.3032559395248	-0.01\\
60.3532586393089	-0.01\\
60.4032613390929	-0.01\\
60.4532640388769	-0.01\\
60.5032667386609	-0.01\\
60.5532694384449	-0.01\\
60.6032721382289	-0.01\\
60.653274838013	-0.01\\
60.703277537797	-0.01\\
60.753280237581	-0.01\\
60.803282937365	-0.01\\
60.853285637149	-0.01\\
60.9032883369331	-0.01\\
60.9532910367171	-0.01\\
61.0032937365011	-0.01\\
61.0532964362851	-0.01\\
61.1032991360691	-0.01\\
61.1533018358531	-0.01\\
61.2033045356371	-0.01\\
61.2533072354212	-0.01\\
61.3033099352052	-0.01\\
61.3533126349892	-0.01\\
61.4033153347732	-0.01\\
61.4533180345572	-0.01\\
61.5033207343413	-0.01\\
61.5533234341253	-0.01\\
61.6033261339093	-0.01\\
61.6533288336933	-0.01\\
61.7033315334773	-0.01\\
61.7533342332613	-0.01\\
61.8033369330454	-0.01\\
61.8533396328294	-0.01\\
61.9033423326134	-0.01\\
61.9533450323974	-0.01\\
62.0033477321814	-0.01\\
62.0533504319654	-0.01\\
62.1033531317495	-0.01\\
62.1533558315335	-0.01\\
62.2033585313175	-0.01\\
62.2533612311015	-0.01\\
62.3033639308855	-0.01\\
62.3533666306695	-0.01\\
62.4033693304536	-0.01\\
62.4533720302376	-0.01\\
62.5033747300216	-0.01\\
62.5483771598272	0.19\\
62.5983798596112	0.19\\
62.6483825593953	0.19\\
62.6983852591793	0.19\\
62.7483879589633	0.19\\
62.7983906587473	0.19\\
62.8483933585313	0.19\\
62.8983960583153	0.19\\
62.9483987580994	0.19\\
62.9984014578834	0.19\\
63.0484041576674	0.19\\
63.0984068574514	0.19\\
63.1484095572354	0.19\\
63.1984122570194	0.19\\
63.2484149568035	0.19\\
63.2984176565875	0.19\\
63.3484203563715	0.19\\
63.3984230561555	0.19\\
63.4484257559395	0.19\\
63.4984284557235	0.19\\
63.5484311555076	0.19\\
63.5984338552916	0.19\\
63.6484365550756	0.19\\
63.6984392548596	0.19\\
63.7484419546436	0.19\\
63.7984446544276	0.19\\
63.8484473542117	0.19\\
63.8984500539957	0.19\\
63.9484527537797	0.19\\
63.9984554535637	0.19\\
64.0484581533477	0.19\\
64.0984608531318	0.19\\
64.1484635529158	0.19\\
64.1984662526998	0.19\\
64.2484689524838	0.19\\
64.2984716522678	0.19\\
64.3484743520518	0.19\\
64.3984770518359	0.19\\
64.4484797516199	0.19\\
64.4984824514039	0.19\\
64.5484851511879	0.19\\
64.5984878509719	0.19\\
64.648490550756	0.19\\
64.69849325054	0.19\\
64.748495950324	0.19\\
64.798498650108	0.19\\
64.848501349892	0.19\\
64.898504049676	0.19\\
64.94850674946	0.19\\
64.9985094492441	0.19\\
65.0485121490281	0.19\\
65.0985148488121	0.19\\
65.1485175485961	0.19\\
65.1985202483801	0.19\\
65.2485229481642	0.19\\
65.2985256479482	0.19\\
65.3485283477322	0.19\\
65.3985310475162	0.19\\
65.4485337473002	0.19\\
65.4985364470842	0.19\\
65.5485391468683	0.19\\
65.5985418466523	0.19\\
65.6485445464363	0.19\\
65.6985472462203	0.19\\
65.7485499460043	0.19\\
65.7985526457883	0.19\\
65.8485553455724	0.19\\
65.8985580453564	0.19\\
65.9485607451404	0.19\\
65.9985634449244	0.19\\
66.0485661447084	0.19\\
66.0985688444924	0.19\\
66.1485715442765	0.19\\
66.1985742440605	0.19\\
66.2485769438445	0.19\\
66.2985796436285	0.19\\
66.3485823434125	0.19\\
66.3985850431965	0.19\\
66.4485877429806	0.19\\
66.4985904427646	0.19\\
66.5485931425486	0.19\\
66.5985958423326	0.19\\
66.6485985421166	0.19\\
66.6986012419006	0.19\\
66.7486039416847	0.19\\
66.7986066414687	0.19\\
66.8486093412527	0.19\\
66.8986120410367	0.19\\
66.9486147408207	0.19\\
66.9986174406047	0.19\\
67.0486201403888	0.19\\
67.0986228401728	0.19\\
67.1486255399568	0.19\\
67.1986282397408	0.19\\
67.2486309395248	0.19\\
67.2986336393089	0.19\\
67.3486363390929	0.19\\
67.3986390388769	0.19\\
67.4486417386609	0.19\\
67.4986444384449	0.19\\
67.5486471382289	0.19\\
67.598649838013	0.19\\
67.648652537797	0.19\\
67.698655237581	0.19\\
67.748657937365	0.19\\
67.798660637149	0.19\\
67.8486633369331	0.19\\
67.8986660367171	0.19\\
67.9486687365011	0.19\\
67.9986714362851	0.19\\
68.0486741360691	0.19\\
68.0986768358531	0.19\\
68.1486795356372	0.19\\
68.1986822354212	0.19\\
68.2486849352052	0.19\\
68.2986876349892	0.19\\
68.3486903347732	0.19\\
68.3986930345572	0.19\\
68.4486957343413	0.19\\
68.4986984341253	0.19\\
68.5487011339093	0.19\\
68.5987038336933	0.19\\
68.6487065334773	0.19\\
68.6987092332613	0.19\\
68.7487119330454	0.19\\
68.7987146328294	0.19\\
68.8487173326134	0.19\\
68.8987200323974	0.19\\
68.9487227321814	0.19\\
68.9987254319654	0.19\\
69.0487281317495	0.19\\
69.0987308315335	0.19\\
69.1487335313175	0.19\\
69.1987362311015	0.19\\
69.2487389308855	0.19\\
69.2987416306696	0.19\\
69.3487443304536	0.19\\
69.3987470302376	0.19\\
69.4487497300216	0.19\\
69.4987524298056	0.19\\
69.5487551295896	0.19\\
69.5987578293737	0.19\\
69.6487605291577	0.19\\
69.6987632289417	0.19\\
69.7487659287257	0.19\\
69.7987686285097	0.19\\
69.8487713282937	0.19\\
69.8987740280778	0.19\\
69.9487767278618	0.19\\
69.9987794276458	0.19\\
70.0487821274298	0.19\\
70.0987848272138	0.19\\
70.1487875269979	0.19\\
70.1987902267819	0.19\\
70.2487929265659	0.19\\
70.2987956263499	0.19\\
70.3487983261339	0.19\\
70.3988010259179	0.19\\
70.4488037257019	0.19\\
70.498806425486	0.19\\
70.54880912527	0.19\\
70.598811825054	0.19\\
70.648814524838	0.19\\
70.698817224622	0.19\\
70.7488199244061	0.19\\
70.7988226241901	0.19\\
70.8488253239741	0.19\\
70.8988280237581	0.19\\
70.9488307235421	0.19\\
70.9988334233261	0.19\\
71.0488361231101	0.19\\
71.0988388228942	0.19\\
71.1488415226782	0.19\\
71.1988442224622	0.19\\
71.2488469222462	0.19\\
71.2988496220302	0.19\\
71.3488523218143	0.19\\
71.3988550215983	0.19\\
71.4488577213823	0.19\\
71.4988604211663	0.19\\
71.5488631209503	0.19\\
71.5988658207343	0.19\\
71.6488685205184	0.19\\
71.6988712203024	0.19\\
71.7488739200864	0.19\\
71.7988766198704	0.19\\
71.8488793196544	0.19\\
71.8988820194385	0.19\\
71.9488847192225	0.19\\
71.9988874190065	0.19\\
72.0488901187905	0.19\\
72.0988928185745	0.19\\
72.1488955183585	0.19\\
72.1988982181425	0.19\\
72.2489009179266	0.19\\
72.2989036177106	0.19\\
72.3489063174946	0.19\\
72.3989090172786	0.19\\
72.4489117170626	0.19\\
72.4989144168467	0.19\\
72.5489171166307	0.19\\
72.5989198164147	0.19\\
72.6489225161987	0.19\\
72.6989252159827	0.19\\
72.7489279157667	0.19\\
72.7989306155508	0.19\\
72.8489333153348	0.19\\
72.8989360151188	0.19\\
72.9489387149028	0.19\\
72.9989414146868	0.19\\
73.0489441144708	0.19\\
73.0989468142549	0.19\\
73.1489495140389	0.19\\
73.1989522138229	0.19\\
73.2489549136069	0.19\\
73.2989576133909	0.19\\
73.348960313175	0.19\\
73.398963012959	0.19\\
73.448965712743	0.19\\
73.498968412527	0.19\\
73.548971112311	0.19\\
73.598973812095	0.19\\
73.6489765118791	0.19\\
73.6989792116631	0.19\\
73.7489819114471	0.19\\
73.7989846112311	0.19\\
73.8489873110151	0.19\\
73.8989900107991	0.19\\
73.9489927105832	0.19\\
73.9989954103672	0.19\\
74.0489981101512	0.19\\
74.0990008099352	0.19\\
74.1490035097192	0.19\\
74.1990062095032	0.19\\
74.2490089092873	0.19\\
74.2990116090713	0.19\\
74.3490143088553	0.19\\
74.3990170086393	0.19\\
74.4490197084233	0.19\\
74.4990224082073	0.19\\
74.5490251079914	0.19\\
74.5990278077754	0.19\\
74.6490305075594	0.19\\
74.6990332073434	0.19\\
74.7490359071274	0.19\\
74.7990386069115	0.19\\
74.8490413066955	0.19\\
74.8990440064795	0.19\\
74.9490467062635	0.19\\
74.9990494060475	0.19\\
75.0490521058315	0.19\\
75.0990548056156	0.19\\
75.1490575053996	0.19\\
75.1990602051836	0.19\\
75.2490629049676	0.19\\
75.2990656047516	0.19\\
75.3490683045356	0.19\\
75.3990710043197	0.19\\
75.4490737041037	0.19\\
75.4990764038877	0.19\\
75.5490791036717	0.19\\
75.5990818034557	0.19\\
75.6490845032397	0.19\\
75.6990872030238	0.19\\
75.7490899028078	0.19\\
75.7990926025918	0.19\\
75.8490953023758	0.19\\
75.8990980021598	0.19\\
75.9491007019438	0.19\\
75.9991034017279	0.19\\
76.0491061015119	0.19\\
76.0991088012959	0.19\\
76.1491115010799	0.19\\
76.1991142008639	0.19\\
76.2491169006479	0.19\\
76.299119600432	0.19\\
76.349122300216	0.19\\
76.399125	0.19\\
76.449127699784	0.19\\
76.499130399568	0.19\\
76.5491330993521	0.19\\
76.5991357991361	0.19\\
76.6491384989201	0.19\\
76.6991411987041	0.19\\
76.7491438984881	0.19\\
76.7991465982721	0.19\\
76.8491492980562	0.19\\
76.8991519978402	0.19\\
76.9491546976242	0.19\\
76.9991573974082	0.19\\
77.0491600971922	0.19\\
77.0991627969762	0.19\\
77.1491654967603	0.19\\
77.1991681965443	0.19\\
77.2491708963283	0.19\\
77.2991735961123	0.19\\
77.3491762958963	0.19\\
77.3991789956804	0.19\\
77.4491816954644	0.19\\
77.4991843952484	0.19\\
77.5491870950324	0.19\\
77.5991897948164	0.19\\
77.6491924946004	0.19\\
77.6991951943844	0.19\\
77.7491978941685	0.19\\
77.7992005939525	0.19\\
77.8492032937365	0.19\\
77.8992059935205	0.19\\
77.9492086933045	0.19\\
77.9992113930886	0.19\\
78.0492140928726	0.19\\
78.0992167926566	0.19\\
78.1492194924406	0.19\\
78.1992221922246	0.19\\
78.2492248920086	0.19\\
78.2992275917927	0.19\\
78.3492302915767	0.19\\
78.3992329913607	0.19\\
78.4492356911447	0.19\\
78.4992383909287	0.19\\
78.5492410907127	0.19\\
78.5992437904968	0.19\\
78.6492464902808	0.19\\
78.6992491900648	0.19\\
78.7492518898488	0.19\\
78.7992545896328	0.19\\
78.8492572894169	0.19\\
78.8992599892009	0.19\\
78.9492626889849	0.19\\
78.9992653887689	0.19\\
79.0492680885529	0.19\\
79.0992707883369	0.19\\
79.149273488121	0.19\\
79.199276187905	0.19\\
79.249278887689	0.19\\
79.299281587473	0.19\\
79.349284287257	0.19\\
79.3992869870411	0.19\\
79.4492896868251	0.19\\
79.4992923866091	0.19\\
79.5492950863931	0.19\\
79.5992977861771	0.19\\
79.6493004859611	0.19\\
79.6993031857451	0.19\\
79.7493058855292	0.19\\
79.7993085853132	0.19\\
79.8493112850972	0.19\\
79.8993139848812	0.19\\
79.9493166846652	0.19\\
79.9993193844493	0.19\\
80.0493220842333	0.19\\
80.0993247840173	0.19\\
80.1493274838013	0.19\\
80.1993301835853	0.19\\
80.2493328833693	0.19\\
80.2993355831534	0.19\\
80.3493382829374	0.19\\
80.3993409827214	0.19\\
80.4493436825054	0.19\\
80.4993463822894	0.19\\
80.5493490820734	0.19\\
80.5993517818575	0.19\\
80.6493544816415	0.19\\
80.6993571814255	0.19\\
80.7493598812095	0.19\\
80.7993625809935	0.19\\
80.8493652807775	0.19\\
80.8993679805616	0.19\\
80.9493706803456	0.19\\
80.9993733801296	0.19\\
81.0493760799136	0.19\\
81.0993787796976	0.19\\
81.1493814794816	0.19\\
81.1993841792657	0.19\\
81.2493868790497	0.19\\
81.2993895788337	0.19\\
81.3493922786177	0.19\\
81.3993949784017	0.19\\
81.4493976781857	0.19\\
81.4994003779698	0.19\\
81.5494030777538	0.19\\
81.5994057775378	0.19\\
81.6494084773218	0.19\\
81.6994111771058	0.19\\
81.7494138768898	0.19\\
81.7994165766739	0.19\\
81.8494192764579	0.19\\
81.8994219762419	0.19\\
81.9494246760259	0.19\\
81.9994273758099	0.19\\
82.049430075594	0.19\\
82.099432775378	0.19\\
82.149435475162	0.19\\
82.199438174946	0.19\\
82.24944087473	0.19\\
82.299443574514	0.19\\
82.3494462742981	0.19\\
82.3994489740821	0.19\\
82.4494516738661	0.19\\
82.4994543736501	0.19\\
82.5494570734341	0.19\\
82.5994597732181	0.19\\
82.6294613930885	-0.01\\
82.6794640928726	-0.01\\
82.7294667926566	-0.01\\
82.7794694924406	-0.01\\
82.8294721922246	-0.01\\
82.8794748920086	-0.01\\
82.9294775917927	-0.01\\
82.9794802915767	-0.01\\
83.0294829913607	-0.01\\
83.0794856911447	-0.01\\
83.1294883909287	-0.01\\
83.1794910907127	-0.01\\
83.2294937904968	-0.01\\
83.2794964902808	-0.01\\
83.3294991900648	-0.01\\
83.3795018898488	-0.01\\
83.4295045896328	-0.01\\
83.4795072894169	-0.01\\
83.5295099892009	-0.01\\
83.5795126889849	-0.01\\
83.6295153887689	-0.01\\
83.6795180885529	-0.01\\
83.7295207883369	-0.01\\
83.779523488121	-0.01\\
83.829526187905	-0.01\\
83.879528887689	-0.01\\
83.929531587473	-0.01\\
83.979534287257	-0.01\\
84.029536987041	-0.01\\
84.0795396868251	-0.01\\
84.1295423866091	-0.01\\
84.1795450863931	-0.01\\
84.2295477861771	-0.01\\
84.2795504859611	-0.01\\
84.3295531857451	-0.01\\
84.3795558855292	-0.01\\
84.4295585853132	-0.01\\
84.4795612850972	-0.01\\
84.5295639848812	-0.01\\
84.5795666846652	-0.01\\
84.6295693844492	-0.01\\
84.6795720842333	-0.01\\
84.7295747840173	-0.01\\
84.7795774838013	-0.01\\
84.8295801835853	-0.01\\
84.8795828833693	-0.01\\
84.9295855831534	-0.01\\
84.9795882829374	-0.01\\
85.0295909827214	-0.01\\
85.0795936825054	-0.01\\
85.1295963822894	-0.01\\
85.1795990820734	-0.01\\
85.2296017818575	-0.01\\
85.2796044816415	-0.01\\
85.3296071814255	-0.01\\
85.3796098812095	-0.01\\
85.4296125809935	-0.01\\
85.4796152807775	-0.01\\
85.5296179805616	-0.01\\
85.5796206803456	-0.01\\
85.6296233801296	-0.01\\
85.6796260799136	-0.01\\
85.7296287796976	-0.01\\
85.7796314794817	-0.01\\
85.8296341792657	-0.01\\
85.8796368790497	-0.01\\
85.9296395788337	-0.01\\
85.9796422786177	-0.01\\
86.0296449784017	-0.01\\
86.0796476781857	-0.01\\
86.1296503779698	-0.01\\
86.1796530777538	-0.01\\
86.2296557775378	-0.01\\
86.2796584773218	-0.01\\
86.3296611771058	-0.01\\
86.3796638768899	-0.01\\
86.4296665766739	-0.01\\
86.4796692764579	-0.01\\
86.5296719762419	-0.01\\
86.5796746760259	-0.01\\
86.6296773758099	-0.01\\
86.679680075594	-0.01\\
86.729682775378	-0.01\\
86.779685475162	-0.01\\
86.829688174946	-0.01\\
86.87969087473	-0.01\\
86.929693574514	-0.01\\
86.9796962742981	-0.01\\
87.0296989740821	-0.01\\
87.0797016738661	-0.01\\
87.1297043736501	-0.01\\
87.1797070734341	-0.01\\
87.2297097732181	-0.01\\
87.2797124730022	-0.01\\
87.3297151727862	-0.01\\
87.3797178725702	-0.01\\
87.4297205723542	-0.01\\
87.4797232721382	-0.01\\
87.5297259719222	-0.01\\
87.5797286717063	-0.01\\
87.6297313714903	-0.01\\
87.6797340712743	-0.01\\
87.7297367710583	-0.01\\
87.7797394708423	-0.01\\
87.8297421706264	-0.01\\
87.8797448704104	-0.01\\
87.9297475701944	-0.01\\
87.9797502699784	-0.01\\
88.0297529697624	-0.01\\
88.0797556695464	-0.01\\
88.1297583693305	-0.01\\
88.1797610691145	-0.01\\
88.2297637688985	-0.01\\
88.2797664686825	-0.01\\
88.3297691684665	-0.01\\
88.3797718682505	-0.01\\
88.4297745680346	-0.01\\
88.4797772678186	-0.01\\
88.5297799676026	-0.01\\
88.5797826673866	-0.01\\
88.6297853671706	-0.01\\
88.6797880669546	-0.01\\
88.7297907667387	-0.01\\
88.7797934665227	-0.01\\
88.8297961663067	-0.01\\
88.8797988660907	-0.01\\
88.9298015658747	-0.01\\
88.9798042656588	-0.01\\
89.0298069654428	-0.01\\
89.0798096652268	-0.01\\
89.1298123650108	-0.01\\
89.1798150647948	-0.01\\
89.2298177645788	-0.01\\
89.2798204643629	-0.01\\
89.3298231641469	-0.01\\
89.3798258639309	-0.01\\
89.4298285637149	-0.01\\
89.4798312634989	-0.01\\
89.5298339632829	-0.01\\
89.579836663067	-0.01\\
89.629839362851	-0.01\\
89.679842062635	-0.01\\
89.729844762419	-0.01\\
89.779847462203	-0.01\\
89.8298501619871	-0.01\\
89.8798528617711	-0.01\\
89.9298555615551	-0.01\\
89.9798582613391	-0.01\\
90.0298609611231	-0.01\\
90.0798636609071	-0.01\\
90.1298663606911	-0.01\\
90.1798690604752	-0.01\\
90.2298717602592	-0.01\\
90.2798744600432	-0.01\\
90.3298771598272	-0.01\\
90.3798798596112	-0.01\\
90.4298825593953	-0.01\\
90.4798852591793	-0.01\\
90.5298879589633	-0.01\\
90.5798906587473	-0.01\\
90.6298933585313	-0.01\\
90.6798960583153	-0.01\\
90.7298987580994	-0.01\\
90.7799014578834	-0.01\\
90.8299041576674	-0.01\\
90.8799068574514	-0.01\\
90.9299095572354	-0.01\\
90.9799122570195	-0.01\\
91.0299149568035	-0.01\\
91.0799176565875	-0.01\\
91.1299203563715	-0.01\\
91.1799230561555	-0.01\\
91.2299257559395	-0.01\\
91.2799284557236	-0.01\\
91.3299311555076	-0.01\\
91.3799338552916	-0.01\\
91.4299365550756	-0.01\\
91.4799392548596	-0.01\\
91.5299419546436	-0.01\\
91.5799446544276	-0.01\\
91.6299473542117	-0.01\\
91.6799500539957	-0.01\\
91.7299527537797	-0.01\\
91.7799554535637	-0.01\\
91.8299581533477	-0.01\\
91.8799608531317	-0.01\\
91.9299635529158	-0.01\\
91.9799662526998	-0.01\\
92.0299689524838	-0.01\\
92.0799716522678	-0.01\\
92.1299743520518	-0.01\\
92.1799770518358	-0.01\\
92.2299797516199	-0.01\\
92.2799824514039	-0.01\\
92.3299851511879	-0.01\\
92.3799878509719	-0.01\\
92.4299905507559	-0.01\\
92.47999325054	-0.01\\
92.529995950324	-0.01\\
92.579998650108	-0.01\\
92.605	-0.01\\
};
\addlegendentry{- 1cm};

\addplot [color=red,solid,line width=0.2pt]
  table[row sep=crcr]{0	-0.075003\\
0.0500026997840173	-0.074983\\
0.100005399568035	-0.074955\\
0.150008099352052	-0.073954\\
0.200010799136069	-0.075066\\
0.250013498920086	-0.074146\\
0.300016198704104	-0.073995\\
0.350018898488121	-0.073456\\
0.400021598272138	-0.073304\\
0.450024298056156	-0.073288\\
0.500026997840173	-0.073859\\
0.55002969762419	-0.073271\\
0.600032397408207	-0.075337\\
0.650035097192225	-0.073829\\
0.700037796976242	-0.073349\\
0.750040496760259	-0.073073\\
0.800043196544276	-0.073011\\
0.850045896328294	-0.074833\\
0.900048596112311	-0.073879\\
0.950051295896328	-0.074291\\
1.00005399568035	-0.073886\\
1.05005669546436	-0.073226\\
1.10005939524838	-0.07297\\
1.1500620950324	-0.07345\\
1.20006479481641	-0.073253\\
1.25006749460043	-0.073608\\
1.30007019438445	-0.073357\\
1.35007289416847	-0.072791\\
1.40007559395248	-0.072473\\
1.4500782937365	-0.072299\\
1.50008099352052	-0.071944\\
1.55008369330454	-0.071813\\
1.60008639308855	-0.072657\\
1.65008909287257	-0.072566\\
1.70009179265659	-0.072121\\
1.7500944924406	-0.071908\\
1.80009719222462	-0.071798\\
1.85009989200864	-0.072185\\
1.90010259179266	-0.072439\\
1.95010529157667	-0.072051\\
2.00010799136069	-0.072449\\
2.05011069114471	-0.070773\\
2.10011339092873	-0.066297\\
2.15011609071274	-0.060216\\
2.20011879049676	-0.049576\\
2.25012149028078	-0.038458\\
2.30012419006479	-0.030138\\
2.35012688984881	-0.02659\\
2.40012958963283	-0.024174\\
2.45013228941685	-0.021831\\
2.50013498920086	-0.019046\\
2.55013768898488	-0.016436\\
2.6001403887689	-0.013247\\
2.65014308855292	-0.012173\\
2.70014578833693	-0.011279\\
2.75014848812095	-0.010226\\
2.80015118790497	-0.009076\\
2.85015388768899	-0.007366\\
2.900156587473	-0.007382\\
2.95015928725702	-0.006207\\
3.00016198704104	-0.006207\\
3.05016468682505	-0.00545\\
3.10016738660907	-0.003491\\
3.15017008639309	-0.003284\\
3.20017278617711	-0.003722\\
3.25017548596112	-0.004076\\
3.30017818574514	-0.004393\\
3.35018088552916	-0.003226\\
3.40018358531318	-0.002523\\
3.45018628509719	-0.003838\\
3.50018898488121	-0.004748\\
3.55019168466523	-0.005132\\
3.60019438444924	-0.004096\\
3.65019708423326	-0.004085\\
3.70019978401728	-0.00422\\
3.7502024838013	-0.00456\\
3.80020518358531	-0.004908\\
3.85020788336933	-0.0044\\
3.90021058315335	-0.003887\\
3.95021328293736	-0.002921\\
4.00021598272138	-0.001625\\
4.0502186825054	-0.001229\\
4.10022138228942	-0.002481\\
4.15022408207343	-0.003454\\
4.20022678185745	-0.003244\\
4.25022948164147	-0.003109\\
4.30023218142549	-0.00209\\
4.3502348812095	-0.002396\\
4.40023758099352	-0.002751\\
4.45024028077754	-0.001355\\
4.50024298056155	-0.001316\\
4.55024568034557	-0.001413\\
4.60024838012959	-0.001619\\
4.65025107991361	-0.001125\\
4.70025377969762	-0.001186\\
4.75025647948164	-0.000338\\
4.80025917926566	-0.000139\\
4.85026187904968	-0.000115\\
4.90026457883369	-0.00022\\
4.95026727861771	7.6e-05\\
5.00026997840173	-0.000148\\
5.05027267818575	-0.000546\\
5.10027537796976	-0.000892\\
5.15027807775378	-0.001127\\
5.2002807775378	-0.000283\\
5.25028347732181	0.000874\\
5.30028617710583	0.000295\\
5.35028887688985	8.6e-05\\
5.40029157667387	0.000259\\
5.45029427645788	-0.000391\\
5.5002969762419	-0.001348\\
5.55029967602592	-0.000577\\
5.60030237580994	-0.000573\\
5.65030507559395	-0.000616\\
5.70030777537797	-0.000678\\
5.75031047516199	-0.000872\\
5.800313174946	-0.00113\\
5.85031587473002	-0.000484\\
5.90031857451404	-0.000644\\
5.95032127429806	-0.000821\\
6.00032397408207	-0.00062\\
6.05032667386609	-0.00029\\
6.10032937365011	0.000829\\
6.15033207343413	0.000662\\
6.20033477321814	0.000289\\
6.25033747300216	0.001395\\
6.30034017278618	0.00207\\
6.3503428725702	0.001229\\
6.40034557235421	0.000862\\
6.45034827213823	0.00131\\
6.50035097192225	0.000227\\
6.55035367170626	0.001733\\
6.60035637149028	0.001414\\
6.6503590712743	-0.000218\\
6.70036177105832	-0.001463\\
6.75036447084233	-0.001581\\
6.80036717062635	-0.000933\\
6.85036987041037	0.001031\\
6.90037257019439	-0.000786\\
6.9503752699784	-0.002603\\
7.00037796976242	-0.003766\\
7.05038066954644	-0.004201\\
7.10038336933045	-0.003364\\
7.15038606911447	-0.003124\\
7.20038876889849	-0.003674\\
7.25039146868251	-0.002977\\
7.30039416846652	-0.003123\\
7.35039686825054	-0.003868\\
7.40039956803456	-0.003522\\
7.45040226781857	-0.003455\\
7.50040496760259	-0.002184\\
7.55040766738661	-0.00125\\
7.60041036717063	0.000722\\
7.65041306695464	-0.000264\\
7.70041576673866	-0.000339\\
7.75041846652268	-0.001041\\
7.8004211663067	-0.000131\\
7.85042386609071	0.00348\\
7.90042656587473	0.002925\\
7.95042926565875	0.001898\\
8.00043196544276	0.001288\\
8.05043466522678	0.000726\\
8.1004373650108	0.001815\\
8.15044006479482	0.001858\\
8.20044276457883	0.001495\\
8.25044546436285	0.000999\\
8.30044816414687	-0.000346\\
8.35045086393089	-0.003011\\
8.4004535637149	-0.004537\\
8.45045626349892	-0.004557\\
8.50045896328294	-0.002099\\
8.55046166306696	-0.002409\\
8.60046436285097	-0.003178\\
8.65046706263499	-0.004498\\
8.70046976241901	-0.005089\\
8.75047246220302	-0.004953\\
8.80047516198704	-0.004918\\
8.85047786177106	-0.0049\\
8.90048056155508	-0.00396\\
8.95048326133909	-0.00345\\
9.00048596112311	-0.001789\\
9.05048866090713	-0.001724\\
9.10049136069114	-0.001457\\
9.15049406047516	-0.000356\\
9.20049676025918	-0.000451\\
9.2504994600432	-0.000413\\
9.30050215982721	-0.000287\\
9.35050485961123	0.000213\\
9.40050755939525	0.000973\\
9.45051025917927	0.000947\\
9.50051295896328	0.002084\\
9.5505156587473	0.001466\\
9.60051835853132	0.001353\\
9.65052105831533	0.000178\\
9.70052375809935	-0.000551\\
9.75052645788337	-0.001248\\
9.80052915766739	-0.00129\\
9.8505318574514	-0.001355\\
9.90053455723542	-0.00132\\
9.95053725701944	-0.000797\\
10.0005399568035	-0.000423\\
10.0505426565875	-0.000156\\
10.1005453563715	0.000418\\
10.1505480561555	0.00023\\
10.2005507559395	0.000394\\
10.2505534557235	0.001086\\
10.3005561555076	0.000637\\
10.3505588552916	0.000321\\
10.4005615550756	-4.1e-05\\
10.4505642548596	-0.000723\\
10.5005669546436	-0.000904\\
10.5505696544276	-0.000589\\
10.6005723542117	-0.00056\\
10.6505750539957	-0.000241\\
10.7005777537797	-0.000559\\
10.7505804535637	0.00028\\
10.8005831533477	0.001932\\
10.8505858531317	0.001772\\
10.9005885529158	0.000876\\
10.9505912526998	0.001343\\
11.0005939524838	0.001237\\
11.0505966522678	0.000811\\
11.1005993520518	0.001141\\
11.1506020518359	0.000993\\
11.2006047516199	0.001048\\
11.2506074514039	0.001857\\
11.3006101511879	0.001288\\
11.3506128509719	0.000592\\
11.4006155507559	0.00039\\
11.45061825054	0.000264\\
11.500620950324	-4e-05\\
11.550623650108	0.000601\\
11.600626349892	0.000572\\
11.650629049676	0.000258\\
11.70063174946	0.000704\\
11.7506344492441	0.000452\\
11.8006371490281	-0.000178\\
11.8506398488121	-0.000898\\
11.9006425485961	-0.001077\\
11.9506452483801	-0.000951\\
12.0006479481641	-0.000318\\
12.0506506479482	-5.8e-05\\
12.1006533477322	-9.6e-05\\
12.1506560475162	-3e-06\\
12.2006587473002	0.000441\\
12.2506614470842	0.001098\\
12.3006641468683	-2.6e-05\\
12.3506668466523	-0.000746\\
12.4006695464363	-0.000981\\
12.4506722462203	-0.000914\\
12.5006749460043	-0.000464\\
12.5506776457883	-0.000378\\
12.6006803455724	-0.000711\\
12.6506830453564	-0.000839\\
12.7006857451404	0.00066\\
12.7506884449244	0.000303\\
12.8006911447084	0.000347\\
12.8506938444924	0.000356\\
12.9006965442765	-0.001082\\
12.9506992440605	-0.001425\\
13.0007019438445	-0.001715\\
13.0507046436285	-0.001476\\
13.1007073434125	-0.001141\\
13.1507100431965	-0.001183\\
13.2007127429806	-0.001379\\
13.2507154427646	-0.000768\\
13.3007181425486	-0.001726\\
13.3507208423326	-0.002396\\
13.4007235421166	-0.00239\\
13.4507262419006	-0.001559\\
13.5007289416847	-0.002046\\
13.5507316414687	-0.001881\\
13.6007343412527	-0.001983\\
13.6507370410367	-0.001245\\
13.7007397408207	-0.001056\\
13.7507424406048	-0.001496\\
13.8007451403888	-0.001744\\
13.8507478401728	-0.002279\\
13.9007505399568	-0.00252\\
13.9507532397408	-0.002517\\
14.0007559395248	-0.001835\\
14.0507586393089	-0.000818\\
14.1007613390929	-0.000685\\
14.1507640388769	-0.000948\\
14.2007667386609	-0.001333\\
14.2507694384449	-0.001336\\
14.3007721382289	-0.000471\\
14.350774838013	0.000806\\
14.400777537797	0.001366\\
14.450780237581	0.001269\\
14.500782937365	0.000187\\
14.550785637149	0.000404\\
14.600788336933	-0.000773\\
14.6507910367171	-0.001171\\
14.7007937365011	0.00059\\
14.7507964362851	-2.8e-05\\
14.8007991360691	-0.000833\\
14.8508018358531	-0.001261\\
14.9008045356372	-0.002331\\
14.9508072354212	-0.003037\\
15.0008099352052	-0.003226\\
15.0508126349892	-0.002441\\
15.1008153347732	-0.001735\\
15.1508180345572	-0.002867\\
15.2008207343413	-0.00366\\
15.2508234341253	-0.003202\\
15.3008261339093	-0.002875\\
15.3508288336933	-0.002258\\
15.4008315334773	-0.000562\\
15.4508342332613	-0.000544\\
15.5008369330454	-0.000525\\
15.5508396328294	-0.0013\\
15.6008423326134	-0.002581\\
15.6508450323974	-0.002729\\
15.7008477321814	-0.002326\\
15.7508504319654	-0.002416\\
15.8008531317495	-0.000636\\
15.8508558315335	0.000348\\
15.9008585313175	-0.000927\\
15.9508612311015	-0.000598\\
16.0008639308855	-0.000968\\
16.0508666306695	-0.000954\\
16.1008693304536	-0.001695\\
16.1508720302376	-0.00107\\
16.2008747300216	-0.00211\\
16.2508774298056	-0.00295\\
16.3008801295896	-0.003853\\
16.3508828293737	-0.003772\\
16.4008855291577	-0.003106\\
16.4508882289417	-0.002199\\
16.5008909287257	-0.000958\\
16.5508936285097	-1.2e-05\\
16.6008963282937	0.000271\\
16.6508990280778	-0.000899\\
16.7009017278618	-0.001266\\
16.7509044276458	-7.7e-05\\
16.8009071274298	-0.000147\\
16.8509098272138	-0.001036\\
16.9009125269978	-0.000567\\
16.9509152267819	-0.001218\\
17.0009179265659	-0.001096\\
17.0509206263499	-0.000867\\
17.1009233261339	-0.001656\\
17.1509260259179	-0.002278\\
17.2009287257019	-0.002688\\
17.250931425486	-0.002569\\
17.30093412527	-0.002739\\
17.350936825054	-0.001736\\
17.400939524838	-0.001091\\
17.450942224622	-0.000953\\
17.500944924406	-0.000666\\
17.5509476241901	-0.000664\\
17.6009503239741	-0.0001\\
17.6509530237581	0.000654\\
17.7009557235421	0.00044\\
17.7509584233261	0.000312\\
17.8009611231102	0.000842\\
17.8509638228942	0.001009\\
17.9009665226782	0.000912\\
17.9509692224622	0.001315\\
18.0009719222462	0.001191\\
18.0509746220302	0.000482\\
18.1009773218143	-0.000669\\
18.1509800215983	-0.000821\\
18.2009827213823	0.00016\\
18.2509854211663	0.00052\\
18.3009881209503	0.000571\\
18.3509908207343	0.000527\\
18.4009935205184	0.000782\\
18.4509962203024	0.00124\\
18.5009989200864	0.001153\\
18.5510016198704	0.002918\\
18.6010043196544	0.003492\\
18.6510070194384	0.001586\\
18.7010097192225	0.001251\\
18.7510124190065	0.000436\\
18.8010151187905	-4.6e-05\\
18.8510178185745	-0.000634\\
18.9010205183585	-0.000461\\
18.9510232181425	-0.001073\\
19.0010259179266	-0.000593\\
19.0510286177106	-0.000474\\
19.1010313174946	-0.00037\\
19.1510340172786	-0.000341\\
19.2010367170626	-0.000302\\
19.2510394168467	-0.000211\\
19.3010421166307	-0.000801\\
19.3510448164147	-0.001197\\
19.4010475161987	-0.001381\\
19.4510502159827	-0.001101\\
19.5010529157667	-0.001062\\
19.5510556155508	-0.000857\\
19.6010583153348	-0.001064\\
19.6510610151188	-0.001314\\
19.7010637149028	-0.000624\\
19.7510664146868	-0.000732\\
19.8010691144708	-0.000762\\
19.8510718142549	-0.000849\\
19.9010745140389	-0.000421\\
19.9510772138229	-0.000378\\
20.0010799136069	-0.000334\\
20.0510826133909	-0.000315\\
20.1010853131749	6.7e-05\\
20.151088012959	-9e-05\\
20.201090712743	-0.000298\\
20.251093412527	-0.000804\\
20.301096112311	-0.00085\\
20.351098812095	-0.000548\\
20.4011015118791	-0.000628\\
20.4511042116631	-0.000562\\
20.5011069114471	-0.000286\\
20.5511096112311	-0.000442\\
20.6011123110151	-0.00102\\
20.6511150107991	-0.000222\\
20.7011177105832	0.000443\\
20.7511204103672	0.000409\\
20.8011231101512	0.001064\\
20.8511258099352	0.001397\\
20.9011285097192	0.003367\\
20.9511312095032	0.004354\\
21.0011339092873	0.002344\\
21.0511366090713	0.000658\\
21.1011393088553	0.000298\\
21.1511420086393	0.000475\\
21.2011447084233	0.000384\\
21.2511474082073	0.000978\\
21.3011501079914	0.00066\\
21.3511528077754	0.000426\\
21.4011555075594	0.000426\\
21.4511582073434	0.00043\\
21.5011609071274	0.000806\\
21.5511636069114	0.00172\\
21.6011663066955	0.001087\\
21.6511690064795	0.00035\\
21.7011717062635	0.003488\\
21.7511744060475	0.003507\\
21.8011771058315	0.00285\\
21.8511798056156	0.002122\\
21.9011825053996	0.001713\\
21.9511852051836	0.001712\\
22.0011879049676	0.001462\\
22.0511906047516	0.001087\\
22.1011933045356	0.000533\\
22.1511960043197	8.3e-05\\
22.2011987041037	-0.000132\\
22.2512014038877	-0.001331\\
22.3012041036717	-0.004184\\
22.3512068034557	-0.005679\\
22.4012095032397	-0.005819\\
22.4512122030238	-0.004488\\
22.5012149028078	-0.003657\\
22.5512176025918	-0.003425\\
22.6012203023758	-0.002426\\
22.6512230021598	-0.002608\\
22.7012257019438	-0.003989\\
22.7512284017279	-0.00293\\
22.8012311015119	-0.001825\\
22.8512338012959	-0.001131\\
22.9012365010799	-0.000975\\
22.9512392008639	-0.000849\\
23.0012419006479	-0.000703\\
23.051244600432	-0.001702\\
23.101247300216	-0.002215\\
23.15125	-0.001022\\
23.201252699784	0.001431\\
23.251255399568	-0.000626\\
23.3012580993521	-0.001549\\
23.3512607991361	-0.000209\\
23.4012634989201	-0.000746\\
23.4512661987041	0.000256\\
23.5012688984881	0.001305\\
23.5512715982721	0.000309\\
23.6012742980562	9.7e-05\\
23.6512769978402	0.001325\\
23.7012796976242	-0.000449\\
23.7512823974082	-0.000801\\
23.8012850971922	-0.001959\\
23.8512877969762	-0.002036\\
23.9012904967603	-0.002507\\
23.9512931965443	-0.00281\\
24.0012958963283	-0.0019\\
24.0512985961123	0.000461\\
24.1013012958963	0.000797\\
24.1513039956803	-9.7e-05\\
24.2013066954644	-0.000129\\
24.2513093952484	0.000708\\
24.3013120950324	0.000809\\
24.3513147948164	0.000495\\
24.4013174946004	0.000436\\
24.4513201943845	0.000767\\
24.5013228941685	0.000331\\
24.5513255939525	0.000577\\
24.6013282937365	-0.00065\\
24.6513309935205	-0.001352\\
24.7013336933045	-0.00186\\
24.7513363930886	-0.000869\\
24.8013390928726	-0.000372\\
24.8513417926566	-0.000186\\
24.9013444924406	0.00017\\
24.9513471922246	-0.000184\\
25.0013498920086	-0.000705\\
25.0513525917927	-0.000624\\
25.1013552915767	-0.000371\\
25.1513579913607	0.000798\\
25.2013606911447	0.000267\\
25.2513633909287	0.002109\\
25.3013660907127	0.001984\\
25.3513687904968	0.001681\\
25.4013714902808	0.001423\\
25.4513741900648	0.000817\\
25.5013768898488	0.000909\\
25.5513795896328	0.000249\\
25.6013822894168	-0.0002\\
25.6513849892009	0.001483\\
25.7013876889849	0.001361\\
25.7513903887689	0.000378\\
25.8013930885529	0.000234\\
25.8513957883369	0.001121\\
25.901398488121	0.002784\\
25.951401187905	0.00306\\
26.001403887689	0.00249\\
26.051406587473	0.00134\\
26.101409287257	0.00015\\
26.151411987041	-0.00106\\
26.2014146868251	0.00017\\
26.2514173866091	0.000122\\
26.3014200863931	0.000275\\
26.3514227861771	0.001988\\
26.4014254859611	0.002515\\
26.4514281857451	0.000666\\
26.5014308855292	-0.000814\\
26.5514335853132	-0.000797\\
26.6014362850972	-0.000399\\
26.6514389848812	0.000624\\
26.7014416846652	-0.000467\\
26.7514443844492	-0.000968\\
26.8014470842333	-0.001945\\
26.8514497840173	-0.001372\\
26.9014524838013	-0.00144\\
26.9514551835853	-0.00101\\
27.0014578833693	0.000302\\
27.0514605831534	-0.00084\\
27.1014632829374	-0.001684\\
27.1514659827214	-0.001495\\
27.2014686825054	-0.000821\\
27.2514713822894	2.1e-05\\
27.3014740820734	9.8e-05\\
27.3514767818575	-5.5e-05\\
27.4014794816415	-0.000255\\
27.4514821814255	0.000702\\
27.5014848812095	0.002969\\
27.5514875809935	0.001437\\
27.6014902807775	-0.000664\\
27.6514929805616	-0.000659\\
27.7014956803456	0.00022\\
27.7514983801296	0.000681\\
27.8015010799136	0.001145\\
27.8515037796976	0.001482\\
27.9015064794816	0.002174\\
27.9515091792657	0.002821\\
28.0015118790497	0.002352\\
28.0515145788337	0.003329\\
28.1015172786177	0.003065\\
28.1515199784017	0.001469\\
28.2015226781857	0.001404\\
28.2515253779698	0.002168\\
28.3015280777538	0.002557\\
28.3515307775378	0.002174\\
28.4015334773218	0.002307\\
28.4515361771058	0.001834\\
28.5015388768899	0.001462\\
28.5515415766739	0.001309\\
28.6015442764579	0.000767\\
28.6515469762419	0.000287\\
28.7015496760259	-7.1e-05\\
28.7515523758099	0.000532\\
28.801555075594	0.000476\\
28.851557775378	0.000184\\
28.901560475162	0.00087\\
28.951563174946	0.000492\\
29.00156587473	0.000195\\
29.051568574514	4.2e-05\\
29.1015712742981	0.000132\\
29.1515739740821	0.000536\\
29.2015766738661	-0.001627\\
29.2515793736501	-0.002352\\
29.3015820734341	6.8e-05\\
29.3515847732181	0.002701\\
29.4015874730022	0.001543\\
29.4515901727862	-0.000644\\
29.5015928725702	-0.001642\\
29.5515955723542	-0.002692\\
29.6015982721382	-0.003291\\
29.6516009719222	-0.004052\\
29.7016036717063	-0.003789\\
29.7516063714903	-0.004365\\
29.8016090712743	-0.003849\\
29.8516117710583	-0.002884\\
29.9016144708423	-0.000968\\
29.9516171706264	-0.000265\\
30.0016198704104	-0.00035\\
30.0516225701944	-7.9e-05\\
30.1016252699784	5.9e-05\\
30.1516279697624	8e-05\\
30.2016306695464	-0.000124\\
30.2516333693305	-0.001495\\
30.3016360691145	-0.001724\\
30.3516387688985	-0.001759\\
30.4016414686825	-0.002406\\
30.4516441684665	-0.000678\\
30.5016468682505	0.000659\\
30.5516495680346	0.000427\\
30.6016522678186	0.000386\\
30.6516549676026	-0.000313\\
30.7016576673866	-6.7e-05\\
30.7516603671706	6.2e-05\\
30.8016630669546	0.000122\\
30.8516657667387	8e-06\\
30.9016684665227	-0.002042\\
30.9516711663067	-0.000988\\
31.0016738660907	0.000239\\
31.0516765658747	0.001477\\
31.1016792656587	0.002072\\
31.1516819654428	0.002352\\
31.2016846652268	0.001227\\
31.2516873650108	0.000156\\
31.3016900647948	-0.001407\\
31.3516927645788	-0.002247\\
31.4016954643629	-0.001753\\
31.4516981641469	-0.001927\\
31.5017008639309	-0.000482\\
31.5517035637149	-0.001014\\
31.6017062634989	-0.002057\\
31.6517089632829	-0.002118\\
31.701711663067	-0.001594\\
31.751714362851	-0.001417\\
31.801717062635	-0.0008\\
31.851719762419	-7.4e-05\\
31.901722462203	0.00142\\
31.951725161987	0.001301\\
32.0017278617711	0.001576\\
32.0517305615551	0.003084\\
32.1017332613391	0.003937\\
32.1517359611231	0.003751\\
32.2017386609071	0.004106\\
32.2517413606911	0.003356\\
32.3017440604752	0.003239\\
32.3517467602592	0.003579\\
32.4017494600432	0.004336\\
32.4517521598272	0.00576\\
32.5017548596112	0.004938\\
32.5517575593952	0.003938\\
32.6017602591793	0.003459\\
32.6517629589633	0.004388\\
32.7017656587473	0.002456\\
32.7517683585313	0.001991\\
32.8017710583153	0.000901\\
32.8517737580993	-0.000228\\
32.9017764578834	-0.000426\\
32.9517791576674	-0.000698\\
33.0017818574514	-0.000773\\
33.0517845572354	-1e-06\\
33.1017872570194	0.001055\\
33.1517899568035	0.001259\\
33.2017926565875	0.001874\\
33.2517953563715	0.001415\\
33.3017980561555	0.000956\\
33.3518007559395	0.001491\\
33.4018034557235	0.000896\\
33.4518061555076	0.000446\\
33.5018088552916	-0.000154\\
33.5518115550756	-0.00028\\
33.6018142548596	0.000636\\
33.6518169546436	0.000594\\
33.7018196544277	0.001963\\
33.7518223542117	0.001605\\
33.8018250539957	-0.000556\\
33.8518277537797	-0.000975\\
33.9018304535637	-0.001765\\
33.9518331533477	-0.001546\\
34.0018358531318	-0.000955\\
34.0518385529158	-0.000721\\
34.1018412526998	-0.000503\\
34.1518439524838	0.000312\\
34.2018466522678	0.000513\\
34.2518493520518	-0.000383\\
34.3018520518359	-0.001435\\
34.3518547516199	-0.001256\\
34.4018574514039	-0.000319\\
34.4518601511879	0.000336\\
34.5018628509719	0.000396\\
34.5518655507559	0.00021\\
34.60186825054	0.000896\\
34.651870950324	0.000974\\
34.701873650108	0.000541\\
34.751876349892	0.000916\\
34.801879049676	0.001037\\
34.85188174946	0.000144\\
34.9018844492441	-0.000927\\
34.9518871490281	-0.000795\\
35.0018898488121	-0.000744\\
35.0518925485961	-0.000466\\
35.1018952483801	-0.000585\\
35.1518979481641	-0.000563\\
35.2019006479482	-0.000468\\
35.2519033477322	-7.3e-05\\
35.3019060475162	0.000127\\
35.3519087473002	0.000304\\
35.4019114470842	0.000302\\
35.4519141468683	0.000382\\
35.5019168466523	0.0006\\
35.5519195464363	0.000268\\
35.6019222462203	0.001058\\
35.6519249460043	-0.000153\\
35.7019276457883	-0.000187\\
35.7519303455724	-9.9e-05\\
35.8019330453564	-0.000302\\
35.8519357451404	-0.000503\\
35.9019384449244	-0.000419\\
35.9519411447084	-0.00054\\
36.0019438444924	-0.000704\\
36.0519465442765	-0.000754\\
36.1019492440605	-0.000678\\
36.1519519438445	-0.000524\\
36.2019546436285	-0.00027\\
36.2519573434125	0.00022\\
36.3019600431965	0.000374\\
36.3519627429806	0.00045\\
36.4019654427646	0.000601\\
36.4519681425486	0.001176\\
36.5019708423326	0.001753\\
36.5519735421166	0.001199\\
36.6019762419006	0.000332\\
36.6519789416847	-0.000237\\
36.7019816414687	-0.000368\\
36.7519843412527	-0.00011\\
36.8019870410367	-0.000386\\
36.8519897408207	-0.000655\\
36.9019924406048	4.7e-05\\
36.9519951403888	0.003273\\
37.0019978401728	0.002291\\
37.0520005399568	0.000155\\
37.1020032397408	-0.000589\\
37.1520059395248	-0.000857\\
37.2020086393089	-0.001545\\
37.2520113390929	-0.002093\\
37.3020140388769	-0.002181\\
37.3520167386609	-0.002056\\
37.4020194384449	-0.001086\\
37.4520221382289	0.000163\\
37.502024838013	0.000101\\
37.552027537797	-0.001578\\
37.602030237581	-0.002819\\
37.652032937365	-0.002939\\
37.702035637149	-0.002286\\
37.752038336933	-0.003363\\
37.8020410367171	-0.002636\\
37.8520437365011	-0.002885\\
37.9020464362851	-0.002836\\
37.9520491360691	-0.00196\\
38.0020518358531	-0.001381\\
38.0520545356372	-0.001641\\
38.1020572354212	-0.002199\\
38.1520599352052	-0.002522\\
38.2020626349892	-0.001957\\
38.2520653347732	-0.000627\\
38.3020680345572	-0.000366\\
38.3520707343413	-0.000876\\
38.4020734341253	-0.000837\\
38.4520761339093	-0.001529\\
38.5020788336933	-0.002187\\
38.5520815334773	-0.001354\\
38.6020842332613	-0.001315\\
38.6520869330454	-0.001128\\
38.7020896328294	-0.001426\\
38.7520923326134	-0.001604\\
38.8020950323974	-0.000835\\
38.8520977321814	-0.001468\\
38.9021004319654	-0.000856\\
38.9521031317495	-0.001313\\
39.0021058315335	-0.001383\\
39.0521085313175	-0.001014\\
39.1021112311015	-0.000163\\
39.1521139308855	-2.4e-05\\
39.2021166306696	-0.00022\\
39.2521193304536	0.001562\\
39.3021220302376	0.001112\\
39.3521247300216	0.000787\\
39.4021274298056	0.00132\\
39.4521301295896	0.003684\\
39.5021328293737	0.002306\\
39.5521355291577	0.001448\\
39.6021382289417	0.000143\\
39.6521409287257	-7e-05\\
39.7021436285097	-0.000516\\
39.7521463282937	-0.001232\\
39.8021490280778	-0.002899\\
39.8521517278618	-0.002467\\
39.9021544276458	-0.001196\\
39.9521571274298	-0.000397\\
40.0021598272138	0.000115\\
40.0521625269978	-0.000184\\
40.1021652267819	-0.000202\\
40.1521679265659	-0.000804\\
40.2021706263499	-0.000118\\
40.2521733261339	-0.001098\\
40.3021760259179	-0.001513\\
40.3521787257019	-0.001057\\
40.402181425486	-0.000858\\
40.45218412527	-0.000206\\
40.502186825054	0.000655\\
40.552189524838	0.001291\\
40.602192224622	0.000446\\
40.652194924406	0.000979\\
40.7021976241901	0.000357\\
40.7522003239741	0.001848\\
40.8022030237581	0.004324\\
40.8522057235421	0.00229\\
40.9022084233261	9.5e-05\\
40.9522111231102	-0.001619\\
41.0022138228942	-0.000867\\
41.0522165226782	3.4e-05\\
41.1022192224622	0.00111\\
41.1522219222462	0.000494\\
41.2022246220302	0.001196\\
41.2522273218143	0.000845\\
41.3022300215983	0.000868\\
41.3522327213823	0.000314\\
41.4022354211663	-0.000309\\
41.4522381209503	-0.000806\\
41.5022408207343	-0.001845\\
41.5522435205184	-0.002723\\
41.6022462203024	-0.00334\\
41.6522489200864	-0.002688\\
41.7022516198704	-0.002991\\
41.7522543196544	-0.003819\\
41.8022570194385	-0.002887\\
41.8522597192225	-0.001264\\
41.9022624190065	-0.000734\\
41.9522651187905	-0.001823\\
42.0022678185745	-0.001766\\
42.0522705183585	-0.002119\\
42.1022732181426	-0.002127\\
42.1522759179266	-0.001823\\
42.2022786177106	-0.000605\\
42.2522813174946	-0.001301\\
42.3022840172786	-0.002154\\
42.3522867170626	-0.001791\\
42.4022894168467	-0.001799\\
42.4522921166307	-0.00134\\
42.5022948164147	-0.000558\\
42.5522975161987	-0.001274\\
42.6023002159827	-0.000149\\
42.6523029157667	0.000437\\
42.7023056155508	9.9e-05\\
42.7523083153348	-0.000723\\
42.8023110151188	-0.001561\\
42.8523137149028	-0.001574\\
42.9023164146868	-0.00024\\
42.9523191144708	-0.000911\\
43.0023218142549	0.000252\\
43.0523245140389	0.001542\\
43.1023272138229	0.003832\\
43.1523299136069	0.002894\\
43.2023326133909	4.8e-05\\
43.2523353131749	-0.000508\\
43.302338012959	-0.000639\\
43.352340712743	-0.000619\\
43.402343412527	-8e-06\\
43.452346112311	-0.001281\\
43.502348812095	-0.00331\\
43.5523515118791	-0.003527\\
43.6023542116631	-0.003863\\
43.6523569114471	-0.003416\\
43.7023596112311	-0.003178\\
43.7523623110151	-0.003327\\
43.8023650107991	-0.002505\\
43.8523677105832	-0.002666\\
43.9023704103672	-0.003542\\
43.9523731101512	-0.004918\\
44.0023758099352	-0.002447\\
44.0523785097192	-0.002436\\
44.1023812095032	-0.001803\\
44.1523839092873	-0.0018\\
44.2023866090713	-0.001768\\
44.2523893088553	-0.001645\\
44.3023920086393	-0.000576\\
44.3523947084233	-0.001367\\
44.4023974082073	-0.002905\\
44.4524001079914	-0.004984\\
44.5024028077754	-0.003465\\
44.5524055075594	-0.004206\\
44.6024082073434	-0.002896\\
44.6524109071274	-0.003545\\
44.7024136069114	-0.003577\\
44.7524163066955	-0.003469\\
44.8024190064795	-0.002618\\
44.8524217062635	-0.002806\\
44.9024244060475	-0.003069\\
44.9524271058315	-0.002938\\
45.0024298056155	-0.002644\\
45.0524325053996	-0.002632\\
45.1024352051836	-0.00244\\
45.1524379049676	-0.002071\\
45.2024406047516	-0.002047\\
45.2524433045356	-0.002471\\
45.3024460043197	-0.00216\\
45.3524487041037	-0.00199\\
45.4024514038877	-0.001866\\
45.4524541036717	-0.001641\\
45.5024568034557	-0.002111\\
45.5524595032397	-0.002515\\
45.6024622030238	-0.001963\\
45.6524649028078	-0.002063\\
45.7024676025918	-0.002239\\
45.7524703023758	-0.002363\\
45.8024730021598	-0.001057\\
45.8524757019439	-0.000384\\
45.9024784017279	-0.000584\\
45.9524811015119	-0.000986\\
46.0024838012959	-0.001498\\
46.0524865010799	-0.002185\\
46.1024892008639	-0.002259\\
46.152491900648	-0.002252\\
46.202494600432	-0.001481\\
46.252497300216	-0.002611\\
46.3025	-0.003405\\
46.352502699784	-0.001891\\
46.402505399568	-0.001959\\
46.4525080993521	-0.002243\\
46.5025107991361	-0.00304\\
46.5525134989201	-0.003185\\
46.6025161987041	-0.003029\\
46.6525188984881	-0.002869\\
46.7025215982721	-0.002152\\
46.7525242980562	-0.000599\\
46.8025269978402	0.000168\\
46.8525296976242	0.000693\\
46.9025323974082	-0.000146\\
46.9525350971922	-0.00033\\
47.0025377969762	-0.000447\\
47.0525404967603	-0.000408\\
47.1025431965443	-0.000201\\
47.1525458963283	-0.00068\\
47.2025485961123	-0.000731\\
47.2525512958963	-0.000779\\
47.3025539956803	-7.9e-05\\
47.3525566954644	-0.000812\\
47.4025593952484	-0.000704\\
47.4525620950324	0.000393\\
47.5025647948164	-0.001894\\
47.5525674946004	-0.003063\\
47.6025701943845	-0.004501\\
47.6525728941685	-0.004738\\
47.7025755939525	-0.005063\\
47.7525782937365	-0.005582\\
47.8025809935205	-0.005517\\
47.8525836933045	-0.004928\\
47.9025863930886	-0.004895\\
47.9525890928726	-0.003785\\
48.0025917926566	-0.00371\\
48.0525944924406	-0.004048\\
48.1025971922246	-0.004175\\
48.1525998920086	-0.003263\\
48.2026025917927	-0.001863\\
48.2526052915767	-0.001425\\
48.3026079913607	-0.001444\\
48.3526106911447	-0.00173\\
48.4026133909287	-0.00149\\
48.4526160907127	-0.000102\\
48.5026187904968	0.000626\\
48.5526214902808	-0.000229\\
48.6026241900648	-0.000453\\
48.6526268898488	-0.00125\\
48.7026295896328	-0.001913\\
48.7526322894168	-0.001937\\
48.8026349892009	-0.002132\\
48.8526376889849	-0.001852\\
48.9026403887689	-0.000987\\
48.9526430885529	0.000231\\
49.0026457883369	0.000141\\
49.052648488121	-4.1e-05\\
49.102651187905	-0.000822\\
49.152653887689	0.000427\\
49.202656587473	0.000446\\
49.252659287257	-0.000201\\
49.302661987041	-0.000481\\
49.3526646868251	-0.001367\\
49.4026673866091	-0.001303\\
49.4526700863931	-0.000603\\
49.5026727861771	-0.000909\\
49.5526754859611	-0.001499\\
49.6026781857451	-0.00151\\
49.6526808855292	-0.001485\\
49.7026835853132	-0.000807\\
49.7526862850972	-0.001407\\
49.8026889848812	-0.001112\\
49.8526916846652	-0.001677\\
49.9026943844492	-0.001922\\
49.9526970842333	-0.001382\\
50.0026997840173	8.8e-05\\
50.0527024838013	0.001873\\
50.1027051835853	0.001002\\
50.1527078833693	0.001078\\
50.2027105831534	0.000939\\
50.2527132829374	0.000328\\
50.3027159827214	-0.000237\\
50.3527186825054	-0.000787\\
50.4027213822894	-0.00029\\
50.4527240820734	-0.000632\\
50.5027267818575	-0.000151\\
50.5527294816415	0.000624\\
50.6027321814255	0.000355\\
50.6527348812095	5.1e-05\\
50.7027375809935	-4.1e-05\\
50.7527402807775	-0.000506\\
50.8027429805616	-0.001314\\
50.8527456803456	0.000359\\
50.9027483801296	-0.00096\\
50.9527510799136	-0.001901\\
51.0027537796976	-0.002947\\
51.0527564794816	-0.002002\\
51.1027591792657	-0.00142\\
51.1527618790497	-0.001424\\
51.2027645788337	-0.001347\\
51.2527672786177	0.000382\\
51.3027699784017	-0.000836\\
51.3527726781858	-0.002007\\
51.4027753779698	-0.003193\\
51.4527780777538	-0.002022\\
51.5027807775378	-0.001807\\
51.5527834773218	-0.001439\\
51.6027861771058	-0.002094\\
51.6527888768899	-0.000734\\
51.7027915766739	-0.001545\\
51.7527942764579	-0.001464\\
51.8027969762419	-0.001235\\
51.8527996760259	-0.001228\\
51.9028023758099	-0.001339\\
51.952805075594	-0.001694\\
52.002807775378	-0.002643\\
52.052810475162	-0.002978\\
52.102813174946	-0.003149\\
52.15281587473	-0.003454\\
52.202818574514	-0.003025\\
52.2528212742981	-0.00172\\
52.3028239740821	-0.001405\\
52.3528266738661	-0.001853\\
52.4028293736501	-0.002155\\
52.4528320734341	-0.003213\\
52.5028347732181	-0.002219\\
52.5528374730022	-0.00234\\
52.6028401727862	-0.002304\\
52.6528428725702	-0.001948\\
52.7028455723542	-0.001393\\
52.7528482721382	-0.000995\\
52.8028509719222	-0.000534\\
52.8528536717063	0.000495\\
52.9028563714903	0.000636\\
52.9528590712743	-0.000811\\
53.0028617710583	-0.000674\\
53.0528644708423	-0.000553\\
53.1028671706264	-0.001695\\
53.1528698704104	-0.002949\\
53.2028725701944	-0.002658\\
53.2528752699784	-0.000867\\
53.3028779697624	-0.001431\\
53.3528806695464	-0.002689\\
53.4028833693305	-0.003605\\
53.4528860691145	-0.003788\\
53.5028887688985	-0.003629\\
53.5528914686825	-0.003284\\
53.6028941684665	-0.002468\\
53.6528968682505	-0.002477\\
53.7028995680346	-0.001863\\
53.7529022678186	0.00067\\
53.8029049676026	0.001978\\
53.8529076673866	0.001269\\
53.9029103671706	-0.000812\\
53.9529130669547	-0.001214\\
54.0029157667387	-5.1e-05\\
54.0529184665227	0.000514\\
54.1029211663067	0.001192\\
54.1529238660907	0.002882\\
54.2029265658747	0.003517\\
54.2529292656588	0.001256\\
54.3029319654428	-0.001867\\
54.3529346652268	-0.001594\\
54.4029373650108	0.000661\\
54.4529400647948	-8e-05\\
54.5029427645788	-0.002743\\
54.5529454643629	-0.003544\\
54.6029481641469	-0.004488\\
54.6529508639309	-0.003031\\
54.7029535637149	-0.000565\\
54.7529562634989	-0.000271\\
54.8029589632829	-0.002065\\
54.852961663067	-0.003111\\
54.902964362851	-0.002274\\
54.952967062635	-0.001285\\
55.002969762419	-0.000808\\
55.052972462203	-0.000772\\
55.102975161987	-0.001004\\
55.1529778617711	-0.003166\\
55.2029805615551	-0.004529\\
55.2529832613391	-0.003937\\
55.3029859611231	-0.002646\\
55.3529886609071	-0.003674\\
55.4029913606911	-0.002948\\
55.4529940604752	-0.001488\\
55.5029967602592	-0.000413\\
55.5529994600432	-0.000574\\
55.6030021598272	-0.001775\\
55.6530048596112	-0.002\\
55.7030075593953	-0.002148\\
55.7530102591793	-0.001529\\
55.8030129589633	-0.00099\\
55.8530156587473	-0.001782\\
55.9030183585313	-0.002284\\
55.9530210583153	-0.002648\\
56.0030237580994	-0.002479\\
56.0530264578834	-0.001724\\
56.1030291576674	0.000197\\
56.1530318574514	0.000194\\
56.2030345572354	0.000278\\
56.2530372570194	-0.000147\\
56.3030399568035	-0.000184\\
56.3530426565875	0.00076\\
56.4030453563715	0.000574\\
56.4530480561555	0.000281\\
56.5030507559395	-0.000198\\
56.5530534557235	-0.000185\\
56.6030561555076	0.001157\\
56.6530588552916	0.001644\\
56.7030615550756	0.000788\\
56.7530642548596	0.000483\\
56.8030669546436	0.000289\\
56.8530696544276	0.000377\\
56.9030723542117	0.000883\\
56.9530750539957	0.000512\\
57.0030777537797	-0.000376\\
57.0530804535637	-0.001051\\
57.1030831533477	-0.001333\\
57.1530858531318	-0.000678\\
57.2030885529158	0.000921\\
57.2530912526998	0.00054\\
57.3030939524838	-0.00012\\
57.3530966522678	0.000372\\
57.4030993520518	7.6e-05\\
57.4531020518358	-7.2e-05\\
57.5031047516199	0.000286\\
57.5531074514039	-0.000418\\
57.6031101511879	-0.000726\\
57.6531128509719	-0.000283\\
57.7031155507559	-0.001326\\
57.75311825054	-0.002458\\
57.803120950324	-0.002663\\
57.853123650108	-0.002176\\
57.903126349892	-0.001027\\
57.953129049676	-0.001172\\
58.00313174946	0.000723\\
58.0531344492441	-0.00106\\
58.1031371490281	-0.002301\\
58.1531398488121	-0.002903\\
58.2031425485961	-0.002291\\
58.2531452483801	-0.000757\\
58.3031479481642	-0.000762\\
58.3531506479482	0.000244\\
58.4031533477322	-0.000239\\
58.4531560475162	0.001046\\
58.5031587473002	0.000343\\
58.5531614470842	-0.0001\\
58.6031641468683	-0.00076\\
58.6531668466523	-0.001297\\
58.7031695464363	-2.3e-05\\
58.7531722462203	0.000405\\
58.8031749460043	0.000761\\
58.8531776457883	-0.000399\\
58.9031803455724	-0.000771\\
58.9531830453564	-0.002974\\
59.0031857451404	-0.004057\\
59.0531884449244	-0.004759\\
59.1031911447084	-0.005312\\
59.1531938444924	-0.005328\\
59.2031965442765	-0.005358\\
59.2531992440605	-0.004697\\
59.3032019438445	-0.004715\\
59.3532046436285	-0.0044\\
59.4032073434125	-0.002977\\
59.4532100431965	-0.00188\\
59.5032127429806	-0.001484\\
59.5532154427646	-0.001\\
59.6032181425486	-0.001507\\
59.6532208423326	-0.001765\\
59.7032235421166	-0.000578\\
59.7532262419006	-0.000879\\
59.8032289416847	-1.7e-05\\
59.8532316414687	0.000717\\
59.9032343412527	0.000141\\
59.9532370410367	-0.000806\\
60.0032397408207	-0.002428\\
60.0532424406048	-0.002233\\
60.1032451403888	-0.002036\\
60.1532478401728	-0.001703\\
60.2032505399568	-0.001437\\
60.2532532397408	-0.001439\\
60.3032559395248	-0.000312\\
60.3532586393089	0.000129\\
60.4032613390929	-0.000573\\
60.4532640388769	0.000139\\
60.5032667386609	-0.001185\\
60.5532694384449	-0.002569\\
60.6032721382289	-0.003136\\
60.653274838013	-0.002828\\
60.703277537797	-0.003\\
60.753280237581	-0.003516\\
60.803282937365	-0.002424\\
60.853285637149	-0.001302\\
60.9032883369331	-0.001825\\
60.9532910367171	-0.002181\\
61.0032937365011	-0.001629\\
61.0532964362851	-0.001031\\
61.1032991360691	0.000179\\
61.1533018358531	0.000829\\
61.2033045356371	0.000852\\
61.2533072354212	3.6e-05\\
61.3033099352052	-0.000191\\
61.3533126349892	-0.000515\\
61.4033153347732	-0.000245\\
61.4533180345572	0.000949\\
61.5033207343413	0.000657\\
61.5533234341253	0.000554\\
61.6033261339093	0.000466\\
61.6533288336933	0.000853\\
61.7033315334773	0.000969\\
61.7533342332613	0.001278\\
61.8033369330454	0.001363\\
61.8533396328294	0.000617\\
61.9033423326134	0.000165\\
61.9533450323974	-0.000305\\
62.0033477321814	-0.000307\\
62.0533504319654	0.000556\\
62.1033531317495	0.000563\\
62.1533558315335	0.000515\\
62.2033585313175	0.000779\\
62.2533612311015	-1.7e-05\\
62.3033639308855	0.002252\\
62.3533666306695	0.002984\\
62.4033693304536	0.001301\\
62.4533720302376	-0.000137\\
62.5033747300216	6.5e-05\\
62.5533774298056	0.00034\\
62.6033801295896	0.002642\\
62.6533828293737	0.014417\\
62.6983852591793	0.036296\\
62.7433876889849	0.066141\\
62.7883901187905	0.09605\\
62.8333925485961	0.121222\\
62.8833952483801	0.141309\\
62.9333979481641	0.151022\\
62.9834006479482	0.155142\\
63.0334033477322	0.159414\\
63.0834060475162	0.166637\\
63.1334087473002	0.173664\\
63.1834114470842	0.178592\\
63.2334141468683	0.18183\\
63.2834168466523	0.184875\\
63.3334195464363	0.18768\\
63.3834222462203	0.190456\\
63.4334249460043	0.193105\\
63.4834276457883	0.195156\\
63.5334303455724	0.196708\\
63.5834330453564	0.197873\\
63.6334357451404	0.199034\\
63.6834384449244	0.19983\\
63.7334411447084	0.200343\\
63.7834438444924	0.200503\\
63.8334465442765	0.201718\\
63.8834492440605	0.202574\\
63.9334519438445	0.202726\\
63.9834546436285	0.202862\\
64.0334573434125	0.20272\\
64.0834600431965	0.203007\\
64.1334627429806	0.203288\\
64.1834654427646	0.203454\\
64.2334681425486	0.203445\\
64.2834708423326	0.203459\\
64.3334735421166	0.203516\\
64.3834762419007	0.202874\\
64.4334789416847	0.202032\\
64.4834816414687	0.201554\\
64.5334843412527	0.201426\\
64.5834870410367	0.201694\\
64.6334897408207	0.201893\\
64.6834924406048	0.202098\\
64.7334951403888	0.202161\\
64.7834978401728	0.201787\\
64.8335005399568	0.201817\\
64.8835032397408	0.201642\\
64.9335059395248	0.201283\\
64.9835086393089	0.200854\\
65.0335113390929	0.200952\\
65.0835140388769	0.201085\\
65.1335167386609	0.201288\\
65.1835194384449	0.201408\\
65.2335221382289	0.201285\\
65.283524838013	0.201574\\
65.333527537797	0.201591\\
65.383530237581	0.201806\\
65.433532937365	0.201915\\
65.483535637149	0.202041\\
65.533538336933	0.202123\\
65.5835410367171	0.20231\\
65.6335437365011	0.202297\\
65.6835464362851	0.202037\\
65.7335491360691	0.201688\\
65.7835518358531	0.201264\\
65.8335545356372	0.200999\\
65.8835572354212	0.200845\\
65.9335599352052	0.200777\\
65.9835626349892	0.200565\\
66.0335653347732	0.200237\\
66.0835680345572	0.200148\\
66.1335707343412	0.200118\\
66.1835734341253	0.200189\\
66.2335761339093	0.200502\\
66.2835788336933	0.200789\\
66.3335815334773	0.200927\\
66.3835842332613	0.200981\\
66.4335869330454	0.201087\\
66.4835896328294	0.201103\\
66.5335923326134	0.201112\\
66.5835950323974	0.200848\\
66.6335977321814	0.200899\\
66.6836004319654	0.201008\\
66.7336031317495	0.201146\\
66.7836058315335	0.201239\\
66.8336085313175	0.201274\\
66.8836112311015	0.201318\\
66.9336139308855	0.201404\\
66.9836166306695	0.201522\\
67.0336193304536	0.201643\\
67.0836220302376	0.201592\\
67.1336247300216	0.20131\\
67.1836274298056	0.201433\\
67.2336301295896	0.201725\\
67.2836328293737	0.201901\\
67.3336355291577	0.202029\\
67.3836382289417	0.201844\\
67.4336409287257	0.201894\\
67.4836436285097	0.201917\\
67.5336463282937	0.201727\\
67.5836490280778	0.201856\\
67.6336517278618	0.201946\\
67.6836544276458	0.202175\\
67.7336571274298	0.202373\\
67.7836598272138	0.202331\\
67.8336625269979	0.202191\\
67.8836652267819	0.202154\\
67.9336679265659	0.202235\\
67.9836706263499	0.202324\\
68.0336733261339	0.20307\\
68.0836760259179	0.203558\\
68.1336787257019	0.2039\\
68.183681425486	0.204087\\
68.23368412527	0.204074\\
68.283686825054	0.203999\\
68.333689524838	0.203588\\
68.383692224622	0.202999\\
68.4336949244061	0.202441\\
68.4836976241901	0.202203\\
68.5337003239741	0.202151\\
68.5837030237581	0.201865\\
68.6337057235421	0.201796\\
68.6837084233261	0.201512\\
68.7337111231102	0.201313\\
68.7837138228942	0.201374\\
68.8337165226782	0.201687\\
68.8837192224622	0.201784\\
68.9337219222462	0.201976\\
68.9837246220302	0.202052\\
69.0337273218143	0.2021\\
69.0837300215983	0.202204\\
69.1337327213823	0.202518\\
69.1837354211663	0.202701\\
69.2337381209503	0.202252\\
69.2837408207344	0.202488\\
69.3337435205184	0.203084\\
69.3837462203024	0.203379\\
69.4337489200864	0.203494\\
69.4837516198704	0.203539\\
69.5337543196544	0.203714\\
69.5837570194385	0.2032\\
69.6337597192225	0.202855\\
69.6837624190065	0.202469\\
69.7337651187905	0.202526\\
69.7837678185745	0.202685\\
69.8337705183585	0.20253\\
69.8837732181426	0.202437\\
69.9337759179266	0.202234\\
69.9837786177106	0.202711\\
70.0337813174946	0.203225\\
70.0837840172786	0.203595\\
70.1337867170626	0.203766\\
70.1837894168467	0.20321\\
70.2337921166307	0.202877\\
70.2837948164147	0.202761\\
70.3337975161987	0.202732\\
70.3838002159827	0.202549\\
70.4338029157667	0.202445\\
70.4838056155508	0.202154\\
70.5338083153348	0.202153\\
70.5838110151188	0.202065\\
70.6338137149028	0.202323\\
70.6838164146868	0.2023\\
70.7338191144708	0.202156\\
70.7838218142549	0.202008\\
70.8338245140389	0.201922\\
70.8838272138229	0.201973\\
70.9338299136069	0.202038\\
70.9838326133909	0.20207\\
71.0338353131749	0.202123\\
71.083838012959	0.201951\\
71.133840712743	0.202631\\
71.183843412527	0.203185\\
71.233846112311	0.202786\\
71.283848812095	0.203137\\
71.3338515118791	0.203348\\
71.3838542116631	0.203153\\
71.4338569114471	0.202946\\
71.4838596112311	0.202923\\
71.5338623110151	0.203298\\
71.5838650107991	0.203595\\
71.6338677105831	0.203618\\
71.6838704103672	0.203693\\
71.7338731101512	0.203516\\
71.7838758099352	0.20359\\
71.8338785097192	0.20356\\
71.8838812095032	0.203032\\
71.9338839092873	0.202657\\
71.9838866090713	0.202482\\
72.0338893088553	0.203069\\
72.0838920086393	0.203427\\
72.1338947084233	0.203588\\
72.1838974082073	0.20369\\
72.2339001079914	0.203769\\
72.2839028077754	0.203861\\
72.3339055075594	0.203207\\
72.3839082073434	0.202745\\
72.4339109071274	0.202324\\
72.4839136069115	0.20239\\
72.5339163066955	0.203073\\
72.5839190064795	0.203404\\
72.6339217062635	0.20338\\
72.6839244060475	0.203528\\
72.7339271058315	0.203598\\
72.7839298056156	0.203513\\
72.8339325053996	0.203075\\
72.8839352051836	0.202633\\
72.9339379049676	0.20241\\
72.9839406047516	0.202839\\
73.0339433045356	0.203345\\
73.0839460043197	0.203567\\
73.1339487041037	0.203682\\
73.1839514038877	0.203158\\
73.2339541036717	0.202369\\
73.2839568034557	0.202201\\
73.3339595032398	0.202239\\
73.3839622030238	0.202887\\
73.4339649028078	0.203255\\
73.4839676025918	0.203435\\
73.5339703023758	0.203239\\
73.5839730021598	0.20263\\
73.6339757019438	0.202221\\
73.6839784017279	0.202298\\
73.7339811015119	0.201874\\
73.7839838012959	0.202039\\
73.8339865010799	0.201898\\
73.8839892008639	0.201797\\
73.933991900648	0.201809\\
73.983994600432	0.201514\\
74.033997300216	0.201534\\
74.084	0.201587\\
74.134002699784	0.201616\\
74.184005399568	0.201402\\
74.2340080993521	0.201831\\
74.2840107991361	0.202178\\
74.3340134989201	0.202252\\
74.3840161987041	0.202331\\
74.4340188984881	0.202158\\
74.4840215982721	0.20227\\
74.5340242980562	0.20234\\
74.5840269978402	0.20268\\
74.6340296976242	0.203319\\
74.6840323974082	0.203421\\
74.7340350971922	0.20373\\
74.7840377969763	0.203927\\
74.8340404967603	0.203381\\
74.8840431965443	0.202858\\
74.9340458963283	0.202607\\
74.9840485961123	0.202408\\
75.0340512958963	0.201879\\
75.0840539956804	0.202421\\
75.1340566954644	0.202154\\
75.1840593952484	0.202187\\
75.2340620950324	0.201902\\
75.2840647948164	0.201807\\
75.3340674946004	0.201822\\
75.3840701943844	0.201893\\
75.4340728941685	0.201953\\
75.4840755939525	0.201979\\
75.5340782937365	0.20215\\
75.5840809935205	0.201999\\
75.6340836933045	0.201869\\
75.6840863930885	0.201785\\
75.7340890928726	0.201688\\
75.7840917926566	0.20164\\
75.8340944924406	0.201842\\
75.8840971922246	0.202001\\
75.9340998920086	0.20186\\
75.9841025917927	0.201815\\
76.0341052915767	0.201663\\
76.0841079913607	0.202289\\
76.1341106911447	0.202301\\
76.1841133909287	0.20228\\
76.2341160907127	0.201849\\
76.2841187904968	0.20173\\
76.3341214902808	0.20174\\
76.3841241900648	0.201855\\
76.4341268898488	0.201885\\
76.4841295896328	0.202387\\
76.5341322894169	0.202987\\
76.5841349892009	0.203171\\
76.6341376889849	0.202951\\
76.6841403887689	0.202687\\
76.7341430885529	0.202423\\
76.7841457883369	0.201811\\
76.834148488121	0.201889\\
76.884151187905	0.202296\\
76.934153887689	0.203021\\
76.984156587473	0.203057\\
77.034159287257	0.202906\\
77.084161987041	0.202708\\
77.1341646868251	0.202439\\
77.1841673866091	0.202025\\
77.2341700863931	0.201916\\
77.2841727861771	0.20203\\
77.3341754859611	0.202516\\
77.3841781857451	0.203135\\
77.4341808855292	0.20343\\
77.4841835853132	0.203513\\
77.5341862850972	0.202987\\
77.5841889848812	0.202095\\
77.6341916846652	0.201628\\
77.6841943844492	0.201458\\
77.7341970842333	0.200586\\
77.7841997840173	0.201127\\
77.8342024838013	0.202021\\
77.8842051835853	0.202771\\
77.9342078833693	0.203168\\
77.9842105831534	0.203315\\
78.0342132829374	0.202659\\
78.0842159827214	0.202179\\
78.1342186825054	0.201603\\
78.1842213822894	0.201541\\
78.2342240820734	0.201471\\
78.2842267818575	0.201387\\
78.3342294816415	0.201095\\
78.3842321814255	0.201105\\
78.4342348812095	0.200949\\
78.4842375809935	0.201073\\
78.5342402807775	0.201024\\
78.5842429805616	0.200975\\
78.6342456803456	0.201445\\
78.6842483801296	0.201945\\
78.7342510799136	0.201516\\
78.7842537796976	0.201501\\
78.8342564794817	0.201463\\
78.8842591792657	0.201274\\
78.9342618790497	0.201107\\
78.9842645788337	0.20094\\
79.0342672786177	0.2008\\
79.0842699784017	0.200691\\
79.1342726781857	0.200386\\
79.1842753779698	0.200392\\
79.2342780777538	0.200434\\
79.2842807775378	0.200567\\
79.3342834773218	0.20061\\
79.3842861771058	0.200759\\
79.4342888768899	0.200873\\
79.4842915766739	0.200876\\
79.5342942764579	0.200423\\
79.5842969762419	0.200559\\
79.6342996760259	0.200762\\
79.6843023758099	0.200911\\
79.734305075594	0.200966\\
79.784307775378	0.200924\\
79.834310475162	0.200608\\
79.884313174946	0.20082\\
79.93431587473	0.200875\\
79.984318574514	0.200685\\
80.0343212742981	0.200811\\
80.0843239740821	0.200996\\
80.1343266738661	0.201148\\
80.1843293736501	0.200803\\
80.2343320734341	0.201082\\
80.2843347732181	0.20139\\
80.3343374730022	0.20161\\
80.3843401727862	0.201675\\
80.4343428725702	0.201667\\
80.4843455723542	0.201527\\
80.5343482721382	0.201224\\
80.5843509719223	0.201313\\
80.6343536717063	0.20135\\
80.6843563714903	0.201304\\
80.7343590712743	0.201125\\
80.7843617710583	0.20136\\
80.8343644708423	0.20147\\
80.8843671706263	0.201202\\
80.9343698704104	0.201256\\
80.9843725701944	0.201153\\
81.0343752699784	0.201345\\
81.0843779697624	0.201435\\
81.1343806695464	0.20121\\
81.1843833693305	0.201289\\
81.2343860691145	0.201418\\
81.2843887688985	0.201394\\
81.3343914686825	0.201603\\
81.3843941684665	0.20174\\
81.4343968682505	0.201848\\
81.4843995680346	0.201917\\
81.5344022678186	0.201957\\
81.5844049676026	0.201923\\
81.6344076673866	0.201776\\
81.6844103671706	0.201694\\
81.7344130669546	0.20143\\
81.7844157667387	0.201274\\
81.8344184665227	0.201379\\
81.8844211663067	0.201832\\
81.9344238660907	0.202128\\
81.9844265658747	0.201958\\
82.0344292656588	0.201554\\
82.0844319654428	0.201199\\
82.1344346652268	0.200967\\
82.1844373650108	0.201212\\
82.2344400647948	0.20136\\
82.2844427645788	0.201626\\
82.3344454643629	0.201692\\
82.3844481641469	0.20138\\
82.4344508639309	0.201146\\
82.4844535637149	0.201043\\
82.5344562634989	0.200973\\
82.5844589632829	0.20116\\
82.634461663067	0.200959\\
82.684464362851	0.199022\\
82.734467062635	0.189471\\
82.784469762419	0.168776\\
82.8294721922246	0.139521\\
82.8744746220302	0.108195\\
82.9194770518359	0.083092\\
82.9694797516199	0.06655\\
83.0194824514039	0.057887\\
83.0694851511879	0.050696\\
83.1194878509719	0.041472\\
83.1694905507559	0.033\\
83.21949325054	0.026332\\
83.269495950324	0.022057\\
83.319498650108	0.018593\\
83.369501349892	0.014812\\
83.419504049676	0.01088\\
83.4695067494601	0.008768\\
83.5195094492441	0.008386\\
83.5695121490281	0.007319\\
83.6195148488121	0.005662\\
83.6695175485961	0.003801\\
83.7195202483801	0.003023\\
83.7695229481642	0.002741\\
83.8195256479482	0.002743\\
83.8695283477322	0.001931\\
83.9195310475162	0.001214\\
83.9695337473002	0.000596\\
84.0195364470842	0.00024\\
84.0695391468683	0.001009\\
84.1195418466523	0.001001\\
84.1695445464363	-0.000389\\
84.2195472462203	-0.002576\\
84.2695499460043	-0.002941\\
84.3195526457883	-0.001291\\
84.3695553455724	-0.000597\\
84.4195580453564	-0.000677\\
84.4695607451404	-0.001232\\
84.5195634449244	-0.001129\\
84.5695661447084	-0.000848\\
84.6195688444924	-0.000569\\
84.6695715442765	-0.00054\\
84.7195742440605	0.000295\\
84.7695769438445	0.000536\\
84.8195796436285	0.000497\\
84.8695823434125	0.000329\\
84.9195850431965	0.000358\\
84.9695877429806	0.001734\\
85.0195904427646	0.001105\\
85.0695931425486	0.000672\\
85.1195958423326	-0.000331\\
85.1695985421166	0.000362\\
85.2196012419006	0.000912\\
85.2696039416847	0.002235\\
85.3196066414687	0.002055\\
85.3696093412527	0.001953\\
85.4196120410367	0.002127\\
85.4696147408207	0.001322\\
85.5196174406048	0.000747\\
85.5696201403888	0.001082\\
85.6196228401728	0.000829\\
85.6696255399568	0.001906\\
85.7196282397408	0.001978\\
85.7696309395248	0.001035\\
85.8196336393089	0.000934\\
85.8696363390929	0.00222\\
85.9196390388769	0.002693\\
85.9696417386609	0.001568\\
86.0196444384449	0.000773\\
86.0696471382289	-0.000723\\
86.119649838013	-0.001303\\
86.169652537797	-0.000326\\
86.219655237581	-0.000527\\
86.269657937365	-0.000521\\
86.319660637149	-0.000839\\
86.369663336933	-0.001169\\
86.4196660367171	-0.001302\\
86.4696687365011	-0.000596\\
86.5196714362851	-0.000136\\
86.5696741360691	0.001274\\
86.6196768358531	0.001077\\
86.6696795356372	0.000311\\
86.7196822354212	0.000236\\
86.7696849352052	0.000744\\
86.8196876349892	0.000964\\
86.8696903347732	0.001639\\
86.9196930345572	0.001317\\
86.9696957343413	9.5e-05\\
87.0196984341253	-0.000386\\
87.0697011339093	0.000913\\
87.1197038336933	0.002098\\
87.1697065334773	0.001729\\
87.2197092332613	0.000379\\
87.2697119330454	-0.000594\\
87.3197146328294	-0.000562\\
87.3697173326134	-0.000397\\
87.4197200323974	-0.000693\\
87.4697227321814	-6.9e-05\\
87.5197254319655	-0.000607\\
87.5697281317495	-0.000568\\
87.6197308315335	0.001747\\
87.6697335313175	0.001617\\
87.7197362311015	0.000508\\
87.7697389308855	0.000241\\
87.8197416306695	0.00023\\
87.8697443304536	0.0005\\
87.9197470302376	-0.000248\\
87.9697497300216	-0.000481\\
88.0197524298056	-0.000898\\
88.0697551295896	-0.00091\\
88.1197578293737	-0.001023\\
88.1697605291577	-0.001069\\
88.2197632289417	-0.000959\\
88.2697659287257	-0.000366\\
88.3197686285097	0.000267\\
88.3697713282937	0.001655\\
88.4197740280778	0.001731\\
88.4697767278618	0.003801\\
88.5197794276458	0.00382\\
88.5697821274298	0.002682\\
88.6197848272138	0.000945\\
88.6697875269979	0.000375\\
88.7197902267819	4.5e-05\\
88.7697929265659	0.001715\\
88.8197956263499	0.00197\\
88.8697983261339	0.001677\\
88.9198010259179	0.001338\\
88.969803725702	0.000488\\
89.019806425486	0.0009\\
89.06980912527	0.000766\\
89.119811825054	0.000675\\
89.169814524838	0.001351\\
89.219817224622	0.002006\\
89.2698199244061	0.002146\\
89.3198226241901	0.002362\\
89.3698253239741	0.002307\\
89.4198280237581	0.001465\\
89.4698307235421	0.00156\\
89.5198334233261	0.001772\\
89.5698361231102	0.001424\\
89.6198388228942	0.001852\\
89.6698415226782	0.001046\\
89.7198442224622	0.00183\\
89.7698469222462	0.001939\\
89.8198496220302	0.002001\\
89.8698523218143	0.00098\\
89.9198550215983	-5.9e-05\\
89.9698577213823	-0.000762\\
90.0198604211663	8.2e-05\\
90.0698631209503	0.00051\\
90.1198658207343	-0.000239\\
90.1698685205184	-0.00082\\
90.2198712203024	-0.00081\\
90.2698739200864	0.000115\\
90.3198766198704	0.003246\\
90.3698793196544	0.00488\\
90.4198820194384	0.005064\\
90.4698847192225	0.002435\\
90.5198874190065	0.00157\\
90.5698901187905	0.000199\\
90.6198928185745	-7e-06\\
90.6698955183585	-0.000211\\
90.7198982181426	0.000249\\
90.7699009179266	-0.000424\\
90.8199036177106	-0.000941\\
90.8699063174946	-0.001261\\
90.9199090172786	-0.000779\\
90.9699117170626	-0.002059\\
91.0199144168467	-0.002635\\
91.0699171166307	-0.001968\\
91.1199198164147	-0.000395\\
91.1699225161987	0.000681\\
91.2199252159827	0.000896\\
91.2699279157667	0.001861\\
91.3199306155508	0.001365\\
91.3699333153348	0.00051\\
91.4199360151188	0.000113\\
91.4699387149028	0.000488\\
91.5199414146868	0.000654\\
91.5699441144708	0.001186\\
91.6199468142549	0.000824\\
91.6699495140389	0.001013\\
91.7199522138229	0.00069\\
91.7699549136069	0.002648\\
91.8199576133909	0.002392\\
91.8699603131749	0.002036\\
91.919963012959	0.001627\\
91.969965712743	0.001747\\
92.019968412527	0.002095\\
92.069971112311	0.001551\\
92.119973812095	0.002309\\
92.1699765118791	0.002372\\
92.2199792116631	0.001649\\
92.2699819114471	0.002331\\
92.3199846112311	0.001723\\
92.3699873110151	0.001579\\
92.4199900107991	0.002541\\
92.4699927105832	0.002829\\
92.5199954103672	0.001885\\
92.5699981101512	0.003252\\
92.605	0.003252\\
};
\addlegendentry{Estimated position [m]};

\addplot [color=blue,solid,line width=0.2pt]
  table[row sep=crcr]{0	-0.0749067719101089\\
0.0500026997840173	-0.0756575527496417\\
0.100005399568035	-0.0755463917366384\\
0.150008099352052	-0.0754819939799291\\
0.200010799136069	-0.0754424823058127\\
0.250013498920086	-0.0755106348605658\\
0.300016198704104	-0.0755070659702472\\
0.350018898488121	-0.0754657842409125\\
0.400021598272138	-0.0753940467724746\\
0.450024298056156	-0.0751573862684106\\
0.500026997840173	-0.0753222664025992\\
0.55002969762419	-0.075684518712868\\
0.600032397408207	-0.0756881669400238\\
0.650035097192225	-0.0755985775994107\\
0.700037796976242	-0.07560929083947\\
0.750040496760259	-0.0754711685200185\\
0.800043196544276	-0.0755896288163063\\
0.850045896328294	-0.0755178348716048\\
0.900048596112311	-0.0754425418342413\\
0.950051295896328	-0.0753779265865785\\
1.00005399568035	-0.0755356821504329\\
1.05005669546436	-0.0752613055781661\\
1.10005939524838	-0.075397633537239\\
1.1500620950324	-0.0757400724448426\\
1.20006479481641	-0.0756577405276306\\
1.25006749460043	-0.0755986035211804\\
1.30007019438445	-0.0754801426142594\\
1.35007289416847	-0.075196859228696\\
1.40007559395248	-0.0751913834723345\\
1.4500782937365	-0.075324119572121\\
1.50008099352052	-0.07524167792677\\
1.55008369330454	-0.0756630542406475\\
1.60008639308855	-0.0758100507059259\\
1.65008909287257	-0.0757096105790356\\
1.70009179265659	-0.0755895975825141\\
1.7500944924406	-0.0755322125550733\\
1.80009719222462	-0.0755231490336972\\
1.85009989200864	-0.0753617467508887\\
1.90010259179266	-0.0752578490561568\\
1.95010529157667	-0.0757687928378589\\
2.00010799136069	-0.0755965601809048\\
2.05011069114471	-0.0762455577957876\\
2.10011339092873	-0.0726628665586936\\
2.15011609071274	-0.0653693271160434\\
2.20011879049676	-0.0544194462585877\\
2.25012149028078	-0.0428974159908262\\
2.30012419006479	-0.0345433641546035\\
2.35012688984881	-0.0307157467317714\\
2.40012958963283	-0.0281870673800009\\
2.45013228941685	-0.0249247063455462\\
2.50013498920086	-0.0209469889515255\\
2.55013768898488	-0.0177849845374986\\
2.6001403887689	-0.0150774642480426\\
2.65014308855292	-0.0137183512275043\\
2.70014578833693	-0.0131682526312357\\
2.75014848812095	-0.0117226205947694\\
2.80015118790497	-0.0103904380092325\\
2.85015388768899	-0.00922714362317916\\
2.900156587473	-0.00824569123136266\\
2.95015928725702	-0.00735734185180289\\
3.00016198704104	-0.00691141401176108\\
3.05016468682505	-0.0064314624708795\\
3.10016738660907	-0.00584883465341678\\
3.15017008639309	-0.00575186768904735\\
3.20017278617711	-0.00563135917754207\\
3.25017548596112	-0.00571034804003099\\
3.30017818574514	-0.00511706579560307\\
3.35018088552916	-0.00470346959643639\\
3.40018358531318	-0.00483495896974648\\
3.45018628509719	-0.00509385439619457\\
3.50018898488121	-0.00537774069671827\\
3.55019168466523	-0.0048240401651585\\
3.60019438444924	-0.00405979815827595\\
3.65019708423326	-0.00367144993652071\\
3.70019978401728	-0.00355107045091932\\
3.7502024838013	-0.00313041447951463\\
3.80020518358531	-0.00247219680868018\\
3.85020788336933	-0.00202998694297108\\
3.90021058315335	-0.00139888902555756\\
3.95021328293736	-0.000708429778284784\\
4.00021598272138	-0.000312869911554552\\
4.0502186825054	-0.0005790396553982\\
4.10022138228942	-0.000836049662513396\\
4.15022408207343	-0.000981789379811676\\
4.20022678185745	-0.000246379813992724\\
4.25022948164147	-6.30099922993897e-05\\
4.30023218142549	0.000445929894498322\\
4.3502348812095	0.000174450119761246\\
4.40023758099352	0.000629349867364495\\
4.45024028077754	0.00117220930130371\\
4.50024298056155	0.00143476853460119\\
4.55024568034557	0.00150141914709917\\
4.60024838012959	0.00134848852764166\\
4.65025107991361	0.00163623835515118\\
4.70025377969762	0.00206417786312853\\
4.75025647948164	0.00248130669648662\\
4.80025917926566	0.00256577538035016\\
4.85026187904968	0.00252798535382201\\
4.90026457883369	0.00221510736971197\\
4.95026727861771	0.00188237788011579\\
5.00026997840173	0.00183204847945183\\
5.05027267818575	0.00191484830919665\\
5.10027537796976	0.00229254728696066\\
5.15027807775378	0.00278156379283156\\
5.2002807775378	0.00344854110048071\\
5.25028347732181	0.00381360960967787\\
5.30028617710583	0.00357973081268452\\
5.35028887688985	0.00345752188828196\\
5.40029157667387	0.0034863097553107\\
5.45029427645788	0.00338022158927959\\
5.5002969762419	0.00356714089445493\\
5.55029967602592	0.00337307059732386\\
5.60030237580994	0.0037505896846448\\
5.65030507559395	0.00366425036547586\\
5.70030777537797	0.00363008083642352\\
5.75031047516199	0.00352586078444241\\
5.800313174946	0.00376495747464059\\
5.85031587473002	0.00409410721593653\\
5.90031857451404	0.00397720776327857\\
5.95032127429806	0.00414442508379436\\
6.00032397408207	0.00434752436859698\\
6.05032667386609	0.00469445103656204\\
6.10032937365011	0.00470349107706353\\
6.15033207343413	0.00436377464668437\\
6.20033477321814	0.00379735889525196\\
6.25033747300216	0.00334426208278469\\
6.30034017278618	0.00301890337604044\\
6.3503428725702	0.00233376605724472\\
6.40034557235421	0.00173148828123556\\
6.45034827213823	0.00121363937391702\\
6.50035097192225	0.00113083954294815\\
6.55035367170626	0.00067608976409079\\
6.60035637149028	0.000359649799037169\\
6.6503590712743	-0.000469319777677215\\
6.70036177105832	-0.000604169807233929\\
6.75036447084233	-0.00067971958597613\\
6.80036717062635	-0.000582659069966439\\
6.85036987041037	-0.000665349433361255\\
6.90037257019439	-0.00103382926105933\\
6.9503752699784	-0.00170098892628634\\
7.00037796976242	-0.00155522873620708\\
7.05038066954644	-0.00108236901358709\\
7.10038336933045	-0.000677849513434459\\
7.15038606911447	-0.000179829947778432\\
7.20038876889849	-0.000267909928984278\\
7.25039146868251	-0.000127650031092487\\
7.30039416846652	0.000348829778529024\\
7.35039686825054	0.000639989738128178\\
7.40039956803456	0.00116678927838821\\
7.45040226781857	0.00183380797875145\\
7.50040496760259	0.00240929586277716\\
7.55040766738661	0.0026700456071208\\
7.60041036717063	0.00252429669239999\\
7.65041306695464	0.00199229810402074\\
7.70041576673866	0.00181060854596582\\
7.75041846652268	0.00188062790981115\\
7.8004211663067	0.00212343738400876\\
7.85042386609071	0.00192380731266607\\
7.90042656587473	0.00128923884119827\\
7.95042926565875	7.54799206101253e-05\\
8.00043196544276	-0.000650799634913074\\
8.05043466522678	-0.00117408888928762\\
8.1004373650108	-0.00150304833092228\\
8.15044006479482	-0.0021522171344814\\
8.20044276457883	-0.00317885347681124\\
8.25044546436285	-0.00416582620039677\\
8.30044816414687	-0.00477706030430347\\
8.35045086393089	-0.00531836128852041\\
8.4004535637149	-0.00496236401241241\\
8.45045626349892	-0.00413174634732054\\
8.50045896328294	-0.00310860412400277\\
8.55046166306696	-0.00258543608967034\\
8.60046436285097	-0.00263760475627263\\
8.65046706263499	-0.00247044629153745\\
8.70046976241901	-0.00192370799300143\\
8.75047246220302	-0.000911629516475116\\
8.80047516198704	-0.000181829699903031\\
8.85047786177106	0.000900789042198345\\
8.90048056155508	0.0019095668440927\\
8.95048326133909	0.00297379445880644\\
9.00048596112311	0.00333358285258419\\
9.05048866090713	0.00390355873692718\\
9.10049136069114	0.00434209461348836\\
9.15049406047516	0.0044105032495615\\
9.20049676025918	0.00468561960531963\\
9.2504994600432	0.00498053741970767\\
9.30050215982721	0.0053994223718181\\
9.35050485961123	0.00584706400516743\\
9.40050755939525	0.00575896506734892\\
9.45051025917927	0.00546941015793189\\
9.50051295896328	0.00544436001910002\\
9.5505156587473	0.00486555954972894\\
9.60051835853132	0.00413175614455598\\
9.65052105831533	0.00342335070781257\\
9.70052375809935	0.00345754965190532\\
9.75052645788337	0.00369661016272354\\
9.80052915766739	0.00429183598888206\\
9.8505318574514	0.0045003438742042\\
9.90053455723542	0.00469107012490006\\
9.95053725701944	0.00522869424075406\\
10.0005399568035	0.00566909859974647\\
10.0505426565875	0.00589197378142339\\
10.1005453563715	0.00578413634413553\\
10.1505480561555	0.0058291750820762\\
10.2005507559395	0.00595859153565075\\
10.2505534557235	0.00591901257978793\\
10.3005561555076	0.0056367267987096\\
10.3505588552916	0.00568357658257157\\
10.4005615550756	0.00562056899694706\\
10.4505642548596	0.00578943496028602\\
10.5005669546436	0.00605552949430342\\
10.5505696544276	0.00661300648173535\\
10.6005723542117	0.0068268715316075\\
10.6505750539957	0.00700321051556172\\
10.7005777537797	0.00739513071690921\\
10.7505804535637	0.00798303342142096\\
10.8005831533477	0.00830837154789158\\
10.8505858531317	0.00813405526012492\\
10.9005885529158	0.00801889732467205\\
10.9505912526998	0.00797936297252162\\
11.0005939524838	0.00805489127720868\\
11.0505966522678	0.00793811355732946\\
11.1005993520518	0.00787696459743851\\
11.1506020518359	0.0078374366830522\\
11.2006047516199	0.0081628350039113\\
11.2506074514039	0.00772597972670861\\
11.3006101511879	0.00742744927726425\\
11.3506128509719	0.00732139194216939\\
11.4006155507559	0.00708591878622384\\
11.45061825054	0.00698338943130212\\
11.500620950324	0.00712355846792163\\
11.550623650108	0.00656806874034679\\
11.600626349892	0.00668498784198731\\
11.650629049676	0.00605758015371286\\
11.70063174946	0.0057444448252748\\
11.7506344492441	0.00553598803698523\\
11.8006371490281	0.005395769644015\\
11.8506398488121	0.00518899538249185\\
11.9006425485961	0.0053940317341587\\
11.9506452483801	0.00557559854024851\\
12.0006479481641	0.00576629435182912\\
12.0506506479482	0.00589573367379907\\
12.1006533477322	0.00606812103310179\\
12.1506560475162	0.00593505363028023\\
12.2006587473002	0.00600351233950317\\
12.2506614470842	0.00616705837377553\\
12.3006641468683	0.00579851341459548\\
12.3506668466523	0.00545147024646954\\
12.4006695464363	0.005625879084251\\
12.4506722462203	0.00584712573548209\\
12.5006749460043	0.00574634603964122\\
12.5506776457883	0.00552335895131969\\
12.6006803455724	0.00538859074619017\\
12.6506830453564	0.00503785539425297\\
12.7006857451404	0.00503243801140823\\
12.7506884449244	0.00468189101033363\\
12.8006911447084	0.00430082299806487\\
12.8506938444924	0.00350421032530727\\
12.9006965442765	0.0024614864107951\\
12.9506992440605	0.00234460741733558\\
13.0007019438445	0.00285901489311952\\
13.0507046436285	0.00261620546990234\\
13.1007073434125	0.00268442514492828\\
13.1507100431965	0.00264306516826795\\
13.2007127429806	0.00230872720458827\\
13.2507154427646	0.0021827977121206\\
13.3007181425486	0.00243816698895168\\
13.3507208423326	0.00209289777212395\\
13.4007235421166	0.00233740635218109\\
13.4507262419006	0.00280305366232335\\
13.5007289416847	0.00277067529286448\\
13.5507316414687	0.00275458573435396\\
13.6007343412527	0.00283182547837866\\
13.6507370410367	0.0030638337666442\\
13.7007397408207	0.00323283219434706\\
13.7507424406048	0.00343057140178159\\
13.8007451403888	0.0031969534056526\\
13.8507478401728	0.00319861301619651\\
13.9007505399568	0.00300266421901345\\
13.9507532397408	0.00349168123155314\\
14.0007559395248	0.00376333873079369\\
14.0507586393089	0.00423608383689138\\
14.1007613390929	0.00472153760741455\\
14.1507640388769	0.00462442174792232\\
14.2007667386609	0.00448792325658788\\
14.2507694384449	0.0045004142193252\\
14.3007721382289	0.00475033028788265\\
14.350774838013	0.00511162397542966\\
14.400777537797	0.00503249685014477\\
14.450780237581	0.00418400617367782\\
14.500782937365	0.00336227124020106\\
14.550785637149	0.00272585462142152\\
14.600788336933	0.00223324679498001\\
14.6507910367171	0.00165782884638781\\
14.7007937365011	0.00165960857889543\\
14.7507964362851	0.000891809273292958\\
14.8007991360691	0.000503449821803863\\
14.8508018358531	-7.72598263329883e-05\\
14.9008045356372	-0.000524989729380052\\
14.9508072354212	-0.000350539489990333\\
15.0008099352052	-0.000242639592597299\\
15.0508126349892	0.000131270023620464\\
15.1008153347732	9.00999778047551e-06\\
15.1508180345572	-0.000213869846729365\\
15.2008207343413	-0.000381029714717765\\
15.2508234341253	0.000187029856140875\\
15.3008261339093	0.000985239164585737\\
15.3508288336933	0.00140957898752667\\
15.4008315334773	0.00147436905167427\\
15.4508342332613	0.00127646918384729\\
15.5008369330454	0.000679649669343693\\
15.5508396328294	-0.000183399887470766\\
15.6008423326134	-0.000359479889720392\\
15.6508450323974	-0.00032535959877671\\
15.7008477321814	-0.000523129613549287\\
15.7508504319654	-0.000783859553213259\\
15.8008531317495	-0.000789299707794974\\
15.8508558315335	-0.0010770294225889\\
15.9008585313175	-0.00144031864304545\\
15.9508612311015	-0.00185015790468893\\
16.0008639308855	-0.0020892474356563\\
16.0508666306695	-0.00203706797581211\\
16.1008693304536	-0.00175489868893561\\
16.1508720302376	-0.00142232869665297\\
16.2008747300216	-0.00133941854281566\\
16.2508774298056	-0.00148692834626524\\
16.3008801295896	-0.000890019669062207\\
16.3508828293737	-6.8370009657087e-05\\
16.4008855291577	0.00105725910850375\\
16.4508882289417	0.00168650850110484\\
16.5008909287257	0.00248124652279106\\
16.5508936285097	0.00268616607250207\\
16.6008963282937	0.00232291672702059\\
16.6508990280778	0.00225649710095432\\
16.7009017278618	0.00197054781448886\\
16.7509044276458	0.00184827845682872\\
16.8009071274298	0.00221687740795353\\
16.8509098272138	0.00213964702713628\\
16.9009125269978	0.00194718748019252\\
16.9509152267819	0.00184115847517189\\
17.0009179265659	0.00220450786234546\\
17.0509206263499	0.00254062624565287\\
17.1009233261339	0.00237705640968658\\
17.1509260259179	0.00266475514939011\\
17.2009287257019	0.00317537289540647\\
17.250931425486	0.00359416046776427\\
17.30093412527	0.00425771425306453\\
17.350936825054	0.00480415935936595\\
17.400939524838	0.00509197552660602\\
17.450942224622	0.00579136319276883\\
17.500944924406	0.0060322496742723\\
17.5509476241901	0.00591002400454603\\
17.6009503239741	0.00608798036753664\\
17.6509530237581	0.00641518345200116\\
17.7009557235421	0.00635413528889422\\
17.7509584233261	0.00613656914244406\\
17.8009611231102	0.006064571266275\\
17.8509638228942	0.00641335298189671\\
17.9009665226782	0.00658421083202115\\
17.9509692224622	0.00639913420711237\\
18.0009719222462	0.00590279374376315\\
18.0509746220302	0.00550003134815767\\
18.1009773218143	0.00530760269749266\\
18.1509800215983	0.00540107205864101\\
18.2009827213823	0.00559540596778667\\
18.2509854211663	0.00585971528192784\\
18.3009881209503	0.00592805169032435\\
18.3509908207343	0.00612030957310163\\
18.4009935205184	0.00620305797447331\\
18.4509962203024	0.00616359983378497\\
18.5009989200864	0.00570686372800143\\
18.5510016198704	0.00566191800160119\\
18.6010043196544	0.0052393436129763\\
18.6510070194384	0.00389245802818534\\
18.7010097192225	0.00317890120858389\\
18.7510124190065	0.0031464731119914\\
18.8010151187905	0.00283908459484358\\
18.8510178185745	0.00238783695361686\\
18.9010205183585	0.0026700062130619\\
18.9510232181425	0.00294156337315796\\
19.0010259179266	0.00334059319065943\\
19.0510286177106	0.00361032996086967\\
19.1010313174946	0.00394298696075862\\
19.1510340172786	0.00443748170465587\\
19.2010367170626	0.00424872462478818\\
19.2510394168467	0.00418217586376731\\
19.3010421166307	0.00448230419781984\\
19.3510448164147	0.00432765350822959\\
19.4010475161987	0.00418394609104758\\
19.4510502159827	0.0045057941798455\\
19.5010529157667	0.00486891723413525\\
19.5510556155508	0.00500374545666346\\
19.6010583153348	0.00508652731719281\\
19.6510610151188	0.00515840628204952\\
19.7010637149028	0.00532025234140701\\
19.7510664146868	0.00539558075737698\\
19.8010691144708	0.00571216671417141\\
19.8510718142549	0.0057804651641632\\
19.9010745140389	0.00538855912060207\\
19.9510772138229	0.00591895272869695\\
20.0010799136069	0.00594415387329483\\
20.0510826133909	0.00593514434938282\\
20.1010853131749	0.00569951505860091\\
20.151088012959	0.00604298666088811\\
20.201090712743	0.00586682504438395\\
20.251093412527	0.00545688235042049\\
20.301096112311	0.00517989625426309\\
20.351098812095	0.00530579272019762\\
20.4011015118791	0.00556477176752429\\
20.4511042116631	0.00552887989637731\\
20.5011069114471	0.00564928846398874\\
20.5511096112311	0.00527721051564733\\
20.6011123110151	0.00492290827400756\\
20.6511150107991	0.00480236039841602\\
20.7011177105832	0.00484381911603742\\
20.7511204103672	0.00517285921232203\\
20.8011231101512	0.00485642666793257\\
20.8511258099352	0.00454349227841524\\
20.9011285097192	0.00469630905831151\\
20.9511312095032	0.00431341548690792\\
21.0011339092873	0.00313386332241982\\
21.0511366090713	0.00230503700012473\\
21.1011393088553	0.00199932797195455\\
21.1511420086393	0.00252246684826434\\
21.2011447084233	0.00280296510736208\\
21.2511474082073	0.0020533269712745\\
21.3011501079914	0.00190418829130572\\
21.3511528077754	0.00165949875233573\\
21.4011555075594	0.00147428921674325\\
21.4511582073434	0.00148868884403736\\
21.5011609071274	0.00106613957271884\\
21.5511636069114	0.000719289688580044\\
21.6011663066955	0.000605909772445469\\
21.6511690064795	0.000185270072122915\\
21.7011717062635	-4.49099890778777e-05\\
21.7511744060475	-0.000634779446769759\\
21.8011771058315	-0.00176023783081997\\
21.8511798056156	-0.00293267437337592\\
21.9011825053996	-0.00377221723904167\\
21.9511852051836	-0.00392322776549649\\
22.0011879049676	-0.00446439887302389\\
22.0511906047516	-0.00562047359753456\\
22.1011933045356	-0.00668668864518672\\
22.1511960043197	-0.00756234226152629\\
22.2011987041037	-0.00842347693920958\\
22.2512014038877	-0.00948788368421323\\
22.3012041036717	-0.00959768103822648\\
22.3512068034557	-0.00950410401824841\\
22.4012095032397	-0.00864279957339023\\
22.4512122030238	-0.00777082862433696\\
22.5012149028078	-0.00694188627691208\\
22.5512176025918	-0.00667058764974037\\
22.6012203023758	-0.00594237331347719\\
22.6512230021598	-0.00569424725881398\\
22.7012257019438	-0.0053166225840675\\
22.7512284017279	-0.00435273282540516\\
22.8012311015119	-0.00330650142214388\\
22.8512338012959	-0.00249736673089154\\
22.9012365010799	-0.00231208694931201\\
22.9512392008639	-0.00219708765909093\\
23.0012419006479	-0.0017601987147139\\
23.051244600432	-0.00180878676022762\\
23.101247300216	-0.00138446879408728\\
23.15125	-0.000683279811616574\\
23.201252699784	-0.000426209596171138\\
23.251255399568	-0.000922379206049305\\
23.3012580993521	-0.000904539643661433\\
23.3512607991361	-0.000703059733203253\\
23.4012634989201	0.000291279830132503\\
23.4512661987041	0.000293009994653618\\
23.5012688984881	0.000183499837065832\\
23.5512715982721	0.000199669869154975\\
23.6012742980562	0.000330829907350269\\
23.6512769978402	0.000877299291191873\\
23.7012796976242	0.000882659655705361\\
23.7512823974082	0.000339829715217881\\
23.8012850971922	0.000323579872628062\\
23.8512877969762	0.000560989592868841\\
23.9012904967603	0.00133424928494087\\
23.9512931965443	0.00199208768806045\\
24.0012958963283	0.00304746347917761\\
24.0512985961123	0.00343963177213336\\
24.1013012958963	0.00338388182503185\\
24.1513039956803	0.00302422433184002\\
24.2013066954644	0.00279405551608377\\
24.2513093952484	0.00304029454027128\\
24.3013120950324	0.00327766276802533\\
24.3513147948164	0.00339289216229332\\
24.4013174946004	0.00350793142998796\\
24.4513201943845	0.0029900336611875\\
24.5013228941685	0.00274912550788514\\
24.5513255939525	0.00234646628786219\\
24.6013282937365	0.00222762717366508\\
24.6513309935205	0.00245069656379573\\
24.7013336933045	0.00272392556471188\\
24.7513363930886	0.00287312434467878\\
24.8013390928726	0.00364467993278136\\
24.8513417926566	0.00426846482277529\\
24.9013444924406	0.00425949406098344\\
24.9513471922246	0.00431334501480859\\
25.0013498920086	0.00439605413842229\\
25.0513525917927	0.00491204734217163\\
25.1013552915767	0.00524463370145238\\
25.1513579913607	0.00582552436931003\\
25.2013606911447	0.00567985717112449\\
25.2513633909287	0.00601967972365235\\
25.3013660907127	0.00603233176534936\\
25.3513687904968	0.00574471711316518\\
25.4013714902808	0.00558992622889527\\
25.4513741900648	0.00538133065885374\\
25.5013768898488	0.00502899628228773\\
25.5513795896328	0.00533467153965397\\
25.6013822894168	0.00542623111119736\\
25.6513849892009	0.00538315287928594\\
25.7013876889849	0.0054929397081371\\
25.7513903887689	0.00533828104981426\\
25.8013930885529	0.00513690557329559\\
25.8513957883369	0.00527888260450313\\
25.901398488121	0.00542457057739379\\
25.951401187905	0.00446086147071748\\
26.001403887689	0.00294862249540413\\
26.051406587473	0.00242372585260081\\
26.101409287257	0.00212707633642342\\
26.151411987041	0.00194183728102899\\
26.2014146868251	0.00255850590450707\\
26.2514173866091	0.00309269313183119\\
26.3014200863931	0.00300089453768856\\
26.3514227861771	0.00286612542595932\\
26.4014254859611	0.00237698688890595\\
26.4514281857451	0.00149958874320624\\
26.5014308855292	0.000458579703979448\\
26.5514335853132	0.000214039696794486\\
26.6014362850972	0.000550079369976169\\
26.6514389848812	0.000562609641191814\\
26.7014416846652	0.00037754985172562\\
26.7514443844492	0.000235649862124914\\
26.8014470842333	0.000427969665200734\\
26.8514497840173	0.000845139576806593\\
26.9014524838013	0.00123508888908566\\
26.9514551835853	0.00133940848276814\\
27.0014578833693	0.00112195872085287\\
27.0514605831534	0.000854009321760929\\
27.1014632829374	0.00110230915982468\\
27.1514659827214	0.00161652876177592\\
27.2014686825054	0.00227620723155335\\
27.2514713822894	0.00249744600693547\\
27.3014740820734	0.00243101613316888\\
27.3514767818575	0.00252445588599725\\
27.4014794816415	0.00274185542076253\\
27.4514821814255	0.00320938252485714\\
27.5014848812095	0.00343243152494549\\
27.5514875809935	0.0033281820415451\\
27.6014902807775	0.0033676021510743\\
27.6514929805616	0.00383869920534139\\
27.7014956803456	0.00519795415252214\\
27.7514983801296	0.00604481093774999\\
27.8015010799136	0.00632357652892712\\
27.8515037796976	0.00637205482835713\\
27.9015064794816	0.00639906393372067\\
27.9515091792657	0.00618496788575335\\
28.0015118790497	0.00605011039369309\\
28.0515145788337	0.00590102358395891\\
28.1015172786177	0.00539577070385783\\
28.1515199784017	0.00494093748665625\\
28.2015226781857	0.00485271951563599\\
28.2515253779698	0.00494633824803935\\
28.3015280777538	0.00509017649197831\\
28.3515307775378	0.0048707888261675\\
28.4015334773218	0.00417305663539301\\
28.4515361771058	0.00346645126854507\\
28.5015388768899	0.00277430430457159\\
28.5515415766739	0.00219903654294797\\
28.6015442764579	0.0013628586944874\\
28.6515469762419	0.000609399738529948\\
28.7015496760259	0.000424409897677328\\
28.7515523758099	0.000257199926784578\\
28.801555075594	-0.000300269713257299\\
28.851557775378	-0.000807279357423429\\
28.901560475162	-0.00147437882891631\\
28.951563174946	-0.00206219756514822\\
29.00156587473	-0.00242730643745416\\
29.051568574514	-0.00304584352863491\\
29.1015712742981	-0.00336942223407156\\
29.1515739740821	-0.00356720042703173\\
29.2015766738661	-0.00369660917123532\\
29.2515793736501	-0.00409044549999944\\
29.3015820734341	-0.00389077887153352\\
29.3515847732181	-0.00391068833624874\\
29.4015874730022	-0.00457968112662776\\
29.4515901727862	-0.00555944763101826\\
29.5015928725702	-0.0055683483910907\\
29.5515955723542	-0.00528610144531621\\
29.6015982721382	-0.00494642733178722\\
29.6516009719222	-0.00434203406009637\\
29.7016036717063	-0.00401487669103709\\
29.7516063714903	-0.00369494954444185\\
29.8016090712743	-0.00344680167686239\\
29.8516117710583	-0.00285152434063921\\
29.9016144708423	-0.00192738709972887\\
29.9516171706264	-0.0015462190395103\\
30.0016198704104	-0.00208903667733709\\
30.0516225701944	-0.00188240829751046\\
30.1016252699784	-0.00164520795980134\\
30.1516279697624	-0.00166853848167276\\
30.2016306695464	-0.00180690836465789\\
30.2516333693305	-0.00209106811928193\\
30.3016360691145	-0.00164347866771455\\
30.3516387688985	-0.00107523931664185\\
30.4016414686825	-0.000661679621226868\\
30.4516441684665	0.000224689915381493\\
30.5016468682505	0.000598779899376188\\
30.5516495680346	0.000724669758866384\\
30.6016522678186	0.000738919666674803\\
30.6516549676026	0.000496279637794361\\
30.7016576673866	0.000773259692716715\\
30.7516603671706	0.00108968933440783\\
30.8016630669546	0.00146538886216005\\
30.8516657667387	0.00137005901465024\\
30.9016684665227	0.00127660939645029\\
30.9516711663067	0.00172248855634045\\
31.0016738660907	0.00229426705440814\\
31.0516765658747	0.00248131679196432\\
31.1016792656587	0.00233740666053548\\
31.1516819654428	0.00226543748952018\\
31.2016846652268	0.0015731583037256\\
31.2516873650108	0.00119392929768067\\
31.3016900647948	0.000787459672460606\\
31.3516927645788	0.00099778936737817\\
31.4016954643629	0.00200842757799059\\
31.4516981641469	0.00224397723492307\\
31.5017008639309	0.00273642559510133\\
31.5517035637149	0.00327953303227104\\
31.6017062634989	0.0033982419296411\\
31.6517089632829	0.00432236363166546\\
31.701711663067	0.00492645716808901\\
31.751714362851	0.00577515532821822\\
31.801717062635	0.00645297169425284\\
31.851719762419	0.0074238991021229\\
31.901722462203	0.00821492380230794\\
31.951725161987	0.00870037637713264\\
32.0017278617711	0.0089466457051069\\
32.0517305615551	0.00927034303882632\\
32.1017332613391	0.00942674730853918\\
32.1517359611231	0.00950397299057843\\
32.2017386609071	0.00901486351539632\\
32.2517413606911	0.0085601513426075\\
32.3017440604752	0.0082276013610388\\
32.3517467602592	0.00794348029564186\\
32.4017494600432	0.00788971330447455\\
32.4517521598272	0.00734668058175393\\
32.5017548596112	0.00667606701884247\\
32.5517575593952	0.00589210217084455\\
32.6017602591793	0.00547140024737316\\
32.6517629589633	0.00522669404171772\\
32.7017656587473	0.00492651539021049\\
32.7517683585313	0.0042180651981794\\
32.8017710583153	0.00388183884366414\\
32.8517737580993	0.00379196929803827\\
32.9017764578834	0.00387828916932585\\
32.9517791576674	0.00433307405089926\\
33.0017818574514	0.0048526288558031\\
33.0517845572354	0.00520522355454912\\
33.1017872570194	0.00517638545550738\\
33.1517899568035	0.00520337364677479\\
33.2017926565875	0.00446444246368493\\
33.2517953563715	0.00404539708232453\\
33.3017980561555	0.0032417523116352\\
33.3518007559395	0.00330108317583371\\
33.4018034557235	0.00292892451958916\\
33.4518061555076	0.00238063661839234\\
33.5018088552916	0.00222242752488762\\
33.5518115550756	0.00201735793342532\\
33.6018142548596	0.00242012549989068\\
33.6518169546436	0.00208747721211707\\
33.7018196544277	0.00153909863093976\\
33.7518223542117	0.0012351589310057\\
33.8018250539957	0.000526670060462438\\
33.8518277537797	0.000640149741576534\\
33.9018304535637	0.00090620950017878\\
33.9518331533477	0.00139886917636172\\
34.0018358531318	0.00180696827719675\\
34.0518385529158	0.00235884698452354\\
34.1018412526998	0.00238757680239016\\
34.1518439524838	0.00281214485910528\\
34.2018466522678	0.00271138573153159\\
34.2518493520518	0.00290553470997876\\
34.3018520518359	0.00302783421751983\\
34.3518547516199	0.00357968144832602\\
34.4018574514039	0.00433315272128077\\
34.4518601511879	0.00491932620099826\\
34.5018628509719	0.00490491936017063\\
34.5518655507559	0.00471797025052832\\
34.60186825054	0.00496784784030365\\
34.651870950324	0.00512426575760255\\
34.701873650108	0.00540302101685182\\
34.751876349892	0.00511719487777197\\
34.801879049676	0.00530969294837072\\
34.85188174946	0.00512424570014062\\
34.9018844492441	0.00481493047084053\\
34.9518871490281	0.00505949704861202\\
35.0018898488121	0.00543526173109744\\
35.0518925485961	0.00566374789012745\\
35.1018952483801	0.00591905410253261\\
35.1518979481641	0.00580752530239062\\
35.2019006479482	0.00595328317091551\\
35.2519033477322	0.0062642769004534\\
35.3019060475162	0.00648724067393887\\
35.3519087473002	0.00639002294669871\\
35.4019114470842	0.00619579773228755\\
35.4519141468683	0.00621746646880665\\
35.5019168466523	0.00631642569470169\\
35.5519195464363	0.00613473801544346\\
35.6019222462203	0.00598012124262992\\
35.6519249460043	0.00601790050461425\\
35.7019276457883	0.00604495165217911\\
35.7519303455724	0.00597649304155158\\
35.8019330453564	0.00583632559660245\\
35.8519357451404	0.00573919489269142\\
35.9019384449244	0.00558110801468417\\
35.9519411447084	0.00534008234113301\\
36.0019438444924	0.00549458078840487\\
36.0519465442765	0.0056888674134802\\
36.1019492440605	0.00590288198363911\\
36.1519519438445	0.00575708511573964\\
36.2019546436285	0.00587777304969314\\
36.2519573434125	0.00590830377927248\\
36.3019600431965	0.00595141198642355\\
36.3519627429806	0.00590285152312133\\
36.4019654427646	0.00555046935881876\\
36.4519681425486	0.00521422526520973\\
36.5019708423326	0.00496791807295861\\
36.5519735421166	0.0045326223762739\\
36.6019762419006	0.00424491525572717\\
36.6519789416847	0.00334608173831331\\
36.7019816414687	0.00309965385426933\\
36.7519843412527	0.00261598495283963\\
36.8019870410367	0.00237695678537874\\
36.8519897408207	0.00262503582656922\\
36.9019924406048	0.00253864615488404\\
36.9519951403888	0.00228880717450668\\
37.0019978401728	0.00142751835636407\\
37.0520005399568	0.0003757201082467\\
37.1020032397408	5.75202021288718e-05\\
37.1520059395248	0.000384709809858219\\
37.2020086393089	0.000693959637150849\\
37.2520113390929	0.00122252920905154\\
37.3020140388769	0.00169184790198186\\
37.3520167386609	0.00179806827677754\\
37.4020194384449	0.00192208834049123\\
37.4520221382289	0.0018986682249785\\
37.502024838013	0.00126213895347824\\
37.552027537797	0.000356029868759906\\
37.602030237581	0.000438529513249345\\
37.652032937365	0.000449649749579412\\
37.702035637149	0.000886429507537473\\
37.752038336933	0.00130895867259276\\
37.8020410367171	0.00198507780632461\\
37.8520437365011	0.00234274643857579\\
37.9020464362851	0.00278506437823052\\
37.9520491360691	0.00326860236335206\\
38.0020518358531	0.00379371949994972\\
38.0520545356372	0.00369478929009691\\
38.1020572354212	0.00352582990986555\\
38.1520599352052	0.00375593894846816\\
38.2020626349892	0.00419286568241482\\
38.2520653347732	0.00461532150735514\\
38.3020680345572	0.00433489392248629\\
38.3520707343413	0.00456151308378561\\
38.4020734341253	0.00469088053628568\\
38.4520761339093	0.00444277250122623\\
38.5020788336933	0.00443743371814253\\
38.5520815334773	0.00459398125886192\\
38.6020842332613	0.00514222587576821\\
38.6520869330454	0.00477718030834715\\
38.7020896328294	0.00442306361084213\\
38.7520923326134	0.00467309062272553\\
38.8020950323974	0.00490486770907312\\
38.8520977321814	0.00480419918584166\\
38.9021004319654	0.00456693245711129\\
38.9521031317495	0.00471970034402126\\
39.0021058315335	0.00443915103149122\\
39.0521085313175	0.00459197128609839\\
39.1021112311015	0.00471787996161244\\
39.1521139308855	0.0046747610728386\\
39.2021166306696	0.00471435000436613\\
39.2521193304536	0.00401149688417061\\
39.3021220302376	0.00425594534659075\\
39.3521247300216	0.00389089701142924\\
39.4021274298056	0.00322913327345712\\
39.4521301295896	0.00265206543182994\\
39.5021328293737	0.00147975836109475\\
39.5521355291577	0.000251629811654192\\
39.6021382289417	-0.000195959896192169\\
39.6521409287257	-0.000508939605322318\\
39.7021436285097	-0.000794769295509856\\
39.7521463282937	-0.00109690892743403\\
39.8021490280778	-0.000618519511049038\\
39.8521517278618	-0.000373959801902219\\
39.9021544276458	0.000341629886748533\\
39.9521571274298	0.0015084487992411\\
40.0021598272138	0.00201369789093214\\
40.0521625269978	0.00221148762521484\\
40.1021652267819	0.00236620695748794\\
40.1521679265659	0.00216486756244311\\
40.2021706263499	0.00175126853189745\\
40.2521733261339	0.00172249782426752\\
40.3021760259179	0.00251180601809696\\
40.3521787257019	0.003137403298325\\
40.402181425486	0.00351323094244012\\
40.45218412527	0.00310880381685368\\
40.502186825054	0.00300264396792672\\
40.552189524838	0.00280841529006896\\
40.602192224622	0.00265751564958432\\
40.652194924406	0.00280850529174357\\
40.7021976241901	0.00285356538104834\\
40.7522003239741	0.00305294340439655\\
40.8022030237581	0.00285875509664806\\
40.8522057235421	0.00214331697248389\\
40.9022084233261	0.00129272876301639\\
40.9522111231102	0.00143475891559517\\
41.0022138228942	0.00200474739997376\\
41.0522165226782	0.00270767428946571\\
41.1022192224622	0.00294687302766485\\
41.1522219222462	0.00246319656078214\\
41.2022246220302	0.00187699806489323\\
41.2522273218143	0.0017888983648984\\
41.3022300215983	0.00143126902524554\\
41.3522327213823	0.000769619582045679\\
41.4022354211663	0.000237249876907835\\
41.4522381209503	-0.000271599753943099\\
41.5022408207343	-0.000649229505605303\\
41.5522435205184	-0.000127719768687107\\
41.6022462203024	0.000683219626375215\\
41.6522489200864	0.00128566892514395\\
41.7022516198704	0.00148513855018477\\
41.7522543196544	0.00176395790543268\\
41.8022570194385	0.00302055329641319\\
41.8522597192225	0.0038064287574006\\
41.9022624190065	0.00413356671983891\\
41.9522651187905	0.00368573894300746\\
42.0022678185745	0.00357445035956827\\
42.0522705183585	0.00386024946583087\\
42.1022732181426	0.00422526609184498\\
42.1522759179266	0.00457776281849368\\
42.2022786177106	0.0043638137093383\\
42.2522813174946	0.00454349276497787\\
42.3022840172786	0.00451663390717768\\
42.3522867170626	0.00444112130882181\\
42.4022894168467	0.00416953536806037\\
42.4522921166307	0.00466582111658377\\
42.5022948164147	0.00450770192757692\\
42.5522975161987	0.00433500458469667\\
42.6023002159827	0.00428276462297503\\
42.6523029157667	0.00388724781381283\\
42.7023056155508	0.00337306249996738\\
42.7523083153348	0.00276715545650488\\
42.8023110151188	0.00262315652112114\\
42.8523137149028	0.00293435382815401\\
42.9023164146868	0.00254779535671025\\
42.9523191144708	0.00235898696223635\\
43.0023218142549	0.00186078830750413\\
43.0523245140389	0.00128192916081836\\
43.1023272138229	0.0006508696609716\\
43.1523299136069	-0.000582509590815581\\
43.2023326133909	-0.0018248774012112\\
43.2523353131749	-0.00226916667362878\\
43.302338012959	-0.00278865373355642\\
43.352340712743	-0.00279585459994314\\
43.402343412527	-0.0032885824522709\\
43.452346112311	-0.00396999868065576\\
43.502348812095	-0.00437817383077025\\
43.5523515118791	-0.00423973618199759\\
43.6023542116631	-0.00408136703572331\\
43.6523569114471	-0.00336220223904603\\
43.7023596112311	-0.00351497065544172\\
43.7523623110151	-0.00337489255934782\\
43.8023650107991	-0.00363746065500374\\
43.8523677105832	-0.00337667161880986\\
43.9023704103672	-0.00326692266930452\\
43.9523731101512	-0.00357073113031836\\
44.0023758099352	-0.00311583306423877\\
44.0523785097192	-0.00268971583403272\\
44.1023812095032	-0.00257650581368028\\
44.1523839092873	-0.00292723440316876\\
44.2023866090713	-0.00281032574431531\\
44.2523893088553	-0.0027112950860048\\
44.3023920086393	-0.00293085451976324\\
44.3523947084233	-0.00337111189129592\\
44.4023974082073	-0.00352588037371459\\
44.4524001079914	-0.00404735663034087\\
44.5024028077754	-0.00330119203658418\\
44.5524055075594	-0.00239498619890092\\
44.6024082073434	-0.00182318822435234\\
44.6524109071274	-0.0016327186619111\\
44.7024136069114	-0.00154450897065468\\
44.7524163066955	-0.00100499952045453\\
44.8024190064795	-0.000591479774582077\\
44.8524217062635	-0.000424429839627637\\
44.9024244060475	-0.000363209716596425\\
44.9524271058315	-0.000264309720565549\\
45.0024298056155	-3.60698305887373e-05\\
45.0524325053996	0.000188700094289764\\
45.1024352051836	0.000176160282847857\\
45.1524379049676	0.000507079858725142\\
45.2024406047516	0.00025539989767019\\
45.2524433045356	0.000361489825815174\\
45.3024460043197	0.000476629871251091\\
45.3524487041037	0.000620299735768219\\
45.4024514038877	0.000632889640619591\\
45.4524541036717	0.000433339968929961\\
45.5024568034557	0.000190569984372465\\
45.5524595032397	-0.000133069933037499\\
45.6024622030238	5.57802689794284e-05\\
45.6524649028078	0.000744339485824126\\
45.7024676025918	0.00100868948637548\\
45.7524703023758	0.00126219910546192\\
45.8024730021598	0.0016721476023496\\
45.8524757019439	0.00148343854339895\\
45.9024784017279	0.000924289520510235\\
45.9524811015119	0.000226419996922166\\
46.0024838012959	6.11400047985896e-05\\
46.0524865010799	-7.55499949421792e-05\\
46.1024892008639	0.00013316998789784\\
46.152491900648	-0.000352339658275781\\
46.202494600432	-0.000474599764417297\\
46.252497300216	-0.00074621931414939\\
46.3025	-0.000879259477395778\\
46.352502699784	-0.000658109858804642\\
46.402505399568	-0.000593219186858262\\
46.4525080993521	-0.000920599231488993\\
46.5025107991361	-0.00113992946063732\\
46.5525134989201	-0.0011236692959578\\
46.6025161987041	-0.00068867959171149\\
46.6525188984881	-0.000535759779240732\\
46.7025215982721	-6.30801608565412e-05\\
46.7525242980562	-7.91399981448291e-05\\
46.8025269978402	-0.00033810964329721\\
46.8525296976242	-0.000600489652576696\\
46.9025323974082	-0.00124602908456639\\
46.9525350971922	-0.00189149793775259\\
47.0025377969762	-0.00201906783278563\\
47.0525404967603	-0.00210016748770153\\
47.1025431965443	-0.00227435698662607\\
47.1525458963283	-0.00259089534395604\\
47.2025485961123	-0.00307990301636289\\
47.2525512958963	-0.00293450481358719\\
47.3025539956803	-0.00323996250365947\\
47.3525566954644	-0.00352579036346967\\
47.4025593952484	-0.00439960356811501\\
47.4525620950324	-0.0045058513960571\\
47.5025647948164	-0.00433492415988929\\
47.5525674946004	-0.00444105340959154\\
47.6025701943845	-0.00409048697860514\\
47.6525728941685	-0.00334429131692292\\
47.7025755939525	-0.00309435393979714\\
47.7525782937365	-0.00230316641359661\\
47.8025809935205	-0.00127289906527233\\
47.8525836933045	-0.000674209783795028\\
47.9025863930886	-1.43500385224513e-05\\
47.9525890928726	0.000548359842153551\\
48.0025917926566	0.0010967493762624\\
48.0525944924406	0.00171708765235766\\
48.1025971922246	0.00230311651108682\\
48.1525998920086	0.00292723418306\\
48.2026025917927	0.00389445851159452\\
48.2526052915767	0.00453987222823123\\
48.3026079913607	0.00459211215953575\\
48.3526106911447	0.00442126353425929\\
48.4026133909287	0.0048653394910996\\
48.4526160907127	0.00494633766804388\\
48.5026187904968	0.00502708705030539\\
48.5526214902808	0.0046747206174938\\
48.6026241900648	0.00442130486781359\\
48.6526268898488	0.00413008564644757\\
48.7026295896328	0.00421450427938129\\
48.7526322894168	0.00449678219488376\\
48.8026349892009	0.00471438962464723\\
48.8526376889849	0.00428818556102667\\
48.9026403887689	0.00416593589708221\\
48.9526430885529	0.00383509717116442\\
49.0026457883369	0.00307453374203581\\
49.052648488121	0.00218451707970831\\
49.102651187905	0.00214313691955187\\
49.152653887689	0.00211085727990739\\
49.202656587473	0.00201914748782639\\
49.252659287257	0.00187535706382901\\
49.302661987041	0.000960089537344542\\
49.3526646868251	0.00115785921459375\\
49.4026673866091	0.0017782683235262\\
49.4526700863931	0.00162898853573733\\
49.5026727861771	0.00169179854438567\\
49.5526754859611	0.00168833794459515\\
49.6026781857451	0.0017889571952132\\
49.6526808855292	0.00165040867697787\\
49.7026835853132	0.00194722806751218\\
49.7526862850972	0.00211628692888515\\
49.8026889848812	0.00171522838953027\\
49.8526916846652	0.00194180735960732\\
49.9026943844492	0.00174578832698388\\
49.9526970842333	0.00221501750009263\\
50.0026997840173	0.00219347745321166\\
50.0527024838013	0.00222415708389657\\
50.1027051835853	0.00175305834118241\\
50.1527078833693	0.00114002951328118\\
50.2027105831534	0.000843169408893043\\
50.2527132829374	0.000647179529573017\\
50.3027159827214	0.00134115851637777\\
50.3527186825054	0.00198855791893899\\
50.4027213822894	0.00216669643758935\\
50.4527240820734	0.00252440498362477\\
50.5027267818575	0.00259627604106286\\
50.5527294816415	0.00224392730385893\\
50.6027321814255	0.001884448080172\\
50.6527348812095	0.00110927936035129\\
50.7027375809935	0.0014690290298229\\
50.7527402807775	0.00125861900930967\\
50.8027429805616	0.000897399358710261\\
50.8527456803456	0.000490919775444955\\
50.9027483801296	7.00799181265671e-05\\
50.9527510799136	-0.000510569831979365\\
51.0027537796976	-0.000631129729216842\\
51.0527564794816	-0.000600579690294316\\
51.1027591792657	-0.000348719656193978\\
51.1527618790497	-0.000327179476639057\\
51.2027645788337	-0.000221189907494259\\
51.2527672786177	-0.000151029998258198\\
51.3027699784017	-0.000129409950710511\\
51.3527726781858	-0.000264249938517082\\
51.4027753779698	-0.000233829978992817\\
51.4527780777538	0.000154569961317225\\
51.5027807775378	0.000332540004261411\\
51.5527834773218	6.47201584153381e-05\\
51.6027861771058	0.000177979952722506\\
51.6527888768899	0.000231919916805988\\
51.7027915766739	0.000280469925258247\\
51.7527942764579	0.000239099971425318\\
51.8027969762419	0.000602249595044983\\
51.8527996760259	0.00070300952998984\\
51.9028023758099	0.000348879852118325\\
51.952805075594	3.41200151886253e-05\\
52.002807775378	-0.000170919989100204\\
52.052810475162	-5.2190005358168e-05\\
52.102813174946	0.000474719858712922\\
52.15281587473	0.00111285935688178\\
52.202818574514	0.00163623845600037\\
52.2528212742981	0.00236976668816891\\
52.3028239740821	0.00227454682930199\\
52.3528266738661	0.00218441689787237\\
52.4028293736501	0.00205505740265017\\
52.4528320734341	0.00232676685205987\\
52.5028347732181	0.00283374471294499\\
52.5528374730022	0.00338213090457357\\
52.6028401727862	0.00347010096198404\\
52.6528428725702	0.00308363367616927\\
52.7028455723542	0.00330110178975981\\
52.7528482721382	0.00345581129041286\\
52.8028509719222	0.00290540309004659\\
52.8528536717063	0.00266814499704332\\
52.9028563714903	0.00281565500927903\\
52.9528590712743	0.00208384690586102\\
53.0028617710583	0.00133410872375079\\
53.0528644708423	0.00116699886135287\\
53.1028671706264	0.000284220353132462\\
53.1528698704104	0.000181679927191552\\
53.2028725701944	-0.000203129701990933\\
53.2528752699784	0.000320109924441037\\
53.3028779697624	-8.99198129256683e-05\\
53.3528806695464	-0.00049986963067135\\
53.4028833693305	-0.000920459536657\\
53.4528860691145	-0.000640039823458378\\
53.5028887688985	-0.000237349937412331\\
53.5528914686825	0.000255409876868627\\
53.6028941684665	0.000706639600517943\\
53.6528968682505	0.00133583905157533\\
53.7028995680346	0.0014815977191148\\
53.7529022678186	0.00178550788763731\\
53.8029049676026	0.00153726826925826\\
53.8529076673866	0.000285819692207322\\
53.9029103671706	-0.000426129737918173\\
53.9529130669547	-0.000881039637203463\\
54.0029157667387	-0.000476469864959895\\
54.0529184665227	-0.000751669435273273\\
54.1029211663067	-0.000767869542033553\\
54.1529238660907	-0.00129278914574383\\
54.2029265658747	-0.00202628779789695\\
54.2529292656588	-0.00297752400697056\\
54.3029319654428	-0.00382801884888528\\
54.3529346652268	-0.00415162556875782\\
54.4029373650108	-0.00368052933594275\\
54.4529400647948	-0.00365698995329947\\
54.5029427645788	-0.0040563154700856\\
54.5529454643629	-0.00472141964214133\\
54.6029481641469	-0.00404721708616086\\
54.6529508639309	-0.00301516405810238\\
54.7029535637149	-0.00285332511483205\\
54.7529562634989	-0.00294329469427896\\
54.8029589632829	-0.00335507120949101\\
54.852961663067	-0.00390703721405686\\
54.902964362851	-0.00331005317330301\\
54.952967062635	-0.00235719604092604\\
55.002969762419	-0.00212173744220744\\
55.052972462203	-0.00242370617805109\\
55.102975161987	-0.00293063403253336\\
55.1529778617711	-0.00344495182112723\\
55.2029805615551	-0.00274186402208192\\
55.2529832613391	-0.00175833802991206\\
55.3029859611231	-0.000906179602444737\\
55.3529886609071	-0.000458479794210678\\
55.4029913606911	-0.000215609801766176\\
55.4529940604752	0.000165430070748296\\
55.5029967602592	0.000587839904501372\\
55.5529994600432	0.000790949731627559\\
55.6030021598272	0.000872039416211696\\
55.6530048596112	0.00105366936145565\\
55.7030075593953	0.00144913885979256\\
55.7530102591793	0.00197593800268725\\
55.8030129589633	0.00262327636080257\\
55.8530156587473	0.00260170585973311\\
55.9030183585313	0.00273293472677179\\
55.9530210583153	0.00307640399613353\\
56.0030237580994	0.00415328535223158\\
56.0530264578834	0.00505965552493703\\
56.1030291576674	0.00570499553413222\\
56.1530318574514	0.00589560277591765\\
56.2030345572354	0.00537074122625559\\
56.2530372570194	0.00525934398197636\\
56.3030399568035	0.00522145339648332\\
56.3530426565875	0.00563142687374216\\
56.4030453563715	0.00548936738991504\\
56.4530480561555	0.00533651283469107\\
56.5030507559395	0.00500932686318114\\
56.5530534557235	0.00546411086599676\\
56.6030561555076	0.00576801596854608\\
56.6530588552916	0.00547673115483838\\
56.7030615550756	0.0049319077870376\\
56.7530642548596	0.00437278341398136\\
56.8030669546436	0.00471426076751116\\
56.8530696544276	0.00467650120627834\\
56.9030723542117	0.00489054565877607\\
56.9530750539957	0.00463889162594013\\
57.0030777537797	0.00403651646423361\\
57.0530804535637	0.00375594958531539\\
57.1030831533477	0.00435108558318214\\
57.1530858531318	0.00438002487969307\\
57.2030885529158	0.0045040127206874\\
57.2530912526998	0.00404909804593291\\
57.3030939524838	0.00281913479748756\\
57.3530966522678	0.00220237636332837\\
57.4030993520518	0.00210723768347943\\
57.4531020518358	0.00185902755558906\\
57.5031047516199	0.00191483753913722\\
57.5531074514039	0.0013863389442798\\
57.6031101511879	0.00138267898508889\\
57.6531128509719	0.0013412389205864\\
57.7031155507559	0.000981709447922823\\
57.75311825054	0.00118835933671663\\
57.803120950324	0.00133763876707803\\
57.853123650108	0.00165602810473862\\
57.903126349892	0.00183218751423801\\
57.953129049676	0.00221506753286109\\
58.00313174946	0.00194900816273889\\
58.0531344492441	0.00147251872789674\\
58.1031371490281	0.00107687948746148\\
58.1531398488121	0.00164340888797416\\
58.2031425485961	0.00245072697592513\\
58.2531452483801	0.00297391371142186\\
58.3031479481642	0.00314095278610468\\
58.3531506479482	0.00270412551897773\\
58.4031533477322	0.00186990680233634\\
58.4531560475162	0.00137737923772818\\
58.5031587473002	0.00093324972465021\\
58.5531614470842	0.000584339862454044\\
58.6031641468683	0.000127670019611553\\
58.6531668466523	-0.00039733978316605\\
58.7031695464363	-0.000809009666513718\\
58.7531722462203	-0.00133241829414712\\
58.8031749460043	-0.00204433752864902\\
58.8531776457883	-0.00247040625326514\\
58.9031803455724	-0.00342701154484009\\
58.9531830453564	-0.00386749702959152\\
59.0031857451404	-0.00373987950651004\\
59.0531884449244	-0.00343410228485632\\
59.1031911447084	-0.00274552400491497\\
59.1531938444924	-0.0020983272769289\\
59.2031965442765	-0.00173871832367263\\
59.2531992440605	-0.0013862688991717\\
59.3032019438445	-0.000987209512049864\\
59.3532046436285	-0.00041356979463572\\
59.4032073434125	0.000384719935650402\\
59.4532100431965	0.000909779325083631\\
59.5032127429806	0.000733559680435983\\
59.5532154427646	0.00101956934750335\\
59.6032181425486	0.00108060929596085\\
59.6532208423326	0.00129271918405674\\
59.7032235421166	0.00158398868364583\\
59.7532262419006	0.00181954828606523\\
59.8032289416847	0.00159659872175605\\
59.8532316414687	0.00103551927073854\\
59.9032343412527	0.000773049442948285\\
59.9532370410367	0.000571769714363663\\
60.0032397408207	0.000521469847180079\\
60.0532424406048	0.000780319318524992\\
60.1032451403888	0.00151751876226227\\
60.1532478401728	0.00153545856836991\\
60.2032505399568	0.001341248983216\\
60.2532532397408	0.00123160900508831\\
60.3032559395248	0.00131412870617014\\
60.3532586393089	0.00125492900510075\\
60.4032613390929	0.00059331967283991\\
60.4532640388769	-0.000282289746829835\\
60.5032667386609	-0.000958349166287875\\
60.5532694384449	-0.00125128937347248\\
60.6032721382289	-0.00114527931822214\\
60.653274838013	-0.000695939463907662\\
60.703277537797	-5.56900749343967e-05\\
60.753280237581	0.000436939442653485\\
60.803282937365	0.000906199531019596\\
60.853285637149	0.00148501889647518\\
60.9032883369331	0.00202448753273262\\
60.9532910367171	0.00252433641711433\\
61.0032937365011	0.00288764507782959\\
61.0532964362851	0.00357087079043846\\
61.1032991360691	0.00439795442000325\\
61.1533018358531	0.00448411290875431\\
61.2033045356371	0.00456507284757473\\
61.2533072354212	0.00451116279979938\\
61.3033099352052	0.00498580631086683\\
61.3533126349892	0.00516917418032289\\
61.4033153347732	0.00522297393019774\\
61.4533180345572	0.00566893766234594\\
61.5033207343413	0.00568702732932883\\
61.5533234341253	0.00577339605712297\\
61.6033261339093	0.00581823398624873\\
61.6533288336933	0.00583261364872622\\
61.7033315334773	0.00642425375363245\\
61.7533342332613	0.00638108299943514\\
61.8033369330454	0.00587392427279314\\
61.8533396328294	0.00558984756623299\\
61.9033423326134	0.00532385241643266\\
61.9533450323974	0.00534015075260241\\
62.0033477321814	0.00525371340870952\\
62.0533504319654	0.00533467286513017\\
62.1033531317495	0.00570509489316111\\
62.1533558315335	0.00513508431931235\\
62.2033585313175	0.00458495184243939\\
62.2533612311015	0.00428279313849757\\
62.3033639308855	0.00410131657483443\\
62.3533666306695	0.0038693381282825\\
62.4033693304536	0.00290382430229006\\
62.4533720302376	0.00248838663485624\\
62.5033747300216	0.00256201674308039\\
62.5533774298056	0.00293969373360033\\
62.6033801295896	0.00730158037406515\\
62.6533828293737	0.0209804584179161\\
62.6983852591793	0.0443165373128708\\
62.7433876889849	0.0773065720009305\\
62.7883901187905	0.107975776811472\\
62.8333925485961	0.132401873201575\\
62.8833952483801	0.151120863574408\\
62.9333979481641	0.161097951541105\\
62.9834006479482	0.167220659514066\\
63.0334033477322	0.174147197072698\\
63.0834060475162	0.183101929436818\\
63.1334087473002	0.191018891849376\\
63.1834114470842	0.196952668310739\\
63.2334141468683	0.20137008400624\\
63.2834168466523	0.2055281673067\\
63.3334195464363	0.209942728578313\\
63.3834222462203	0.213768035539077\\
63.4334249460043	0.21765871193351\\
63.4834276457883	0.220498912423392\\
63.5334303455724	0.222248807200975\\
63.5834330453564	0.223994402523092\\
63.6334357451404	0.225933825208333\\
63.6834384449244	0.226659074170141\\
63.7334411447084	0.227287712377391\\
63.7834438444924	0.228044111394442\\
63.8334465442765	0.228850803410512\\
63.8834492440605	0.228639097044047\\
63.9334519438445	0.22836246570446\\
63.9834546436285	0.228726745914892\\
64.0334573434125	0.228702082984959\\
64.0834600431965	0.228714222501722\\
64.1334627429806	0.228775612763585\\
64.1834654427646	0.22849902076264\\
64.2334681425486	0.227783169831619\\
64.2834708423326	0.227508275400174\\
64.3334735421166	0.227084606585917\\
64.3834762419007	0.226946348299605\\
64.4334789416847	0.226387543666201\\
64.4834816414687	0.226709732458104\\
64.5334843412527	0.227109004501926\\
64.5834870410367	0.227324459452586\\
64.6334897408207	0.227436576353541\\
64.6834924406048	0.227529219394336\\
64.7334951403888	0.226932039512644\\
64.7834978401728	0.226926746519794\\
64.8335005399568	0.226450733094179\\
64.8835032397408	0.227110785782134\\
64.9335059395248	0.22693746069593\\
64.9835086393089	0.226872687185046\\
65.0335113390929	0.226907733881642\\
65.0835140388769	0.227067041086207\\
65.1335167386609	0.227273712781603\\
65.1835194384449	0.227434715332753\\
65.2335221382289	0.227522273562301\\
65.283524838013	0.227471527756065\\
65.333527537797	0.227376846635155\\
65.383530237581	0.226872572709928\\
65.433532937365	0.227201924381569\\
65.483535637149	0.227235125397334\\
65.533538336933	0.227123026696113\\
65.5835410367171	0.227219347096309\\
65.6335437365011	0.226926821155741\\
65.6835464362851	0.226165088597998\\
65.7335491360691	0.225879490673085\\
65.7835518358531	0.225332937617479\\
65.8335545356372	0.225282246640105\\
65.8835572354212	0.224537774169903\\
65.9335599352052	0.224637606495278\\
65.9835626349892	0.224721799161723\\
66.0335653347732	0.22449215129886\\
66.0835680345572	0.224555079092082\\
66.1335707343412	0.225028024457346\\
66.1835734341253	0.225566080452077\\
66.2335761339093	0.225658819646868\\
66.2835788336933	0.225650094448491\\
66.3335815334773	0.225729092744312\\
66.3835842332613	0.225558787218479\\
66.4335869330454	0.225280391134651\\
66.4835896328294	0.225583456119156\\
66.5335923326134	0.225373412038953\\
66.5835950323974	0.225396041968847\\
66.6335977321814	0.225467957918357\\
66.6836004319654	0.225781515901777\\
66.7336031317495	0.225864009053866\\
66.7836058315335	0.225877951538374\\
66.8336085313175	0.225930399908062\\
66.8836112311015	0.225872458582879\\
66.9336139308855	0.226042467536164\\
66.9836166306695	0.226017913809417\\
67.0336193304536	0.226476886042622\\
67.0836220302376	0.226872527843495\\
67.1336247300216	0.226732422861491\\
67.1836274298056	0.22703012353998\\
67.2336301295896	0.227443163443539\\
67.2836328293737	0.227122988095744\\
67.3336355291577	0.227364690212217\\
67.3836382289417	0.227038910309663\\
67.4336409287257	0.227480376230828\\
67.4836436285097	0.227207120462992\\
67.5336463282937	0.22724375202725\\
67.5836490280778	0.227257646508258\\
67.6336517278618	0.227499491978991\\
67.6836544276458	0.227250789714039\\
67.7336571274298	0.227385699959371\\
67.7836598272138	0.227154650139796\\
67.8336625269979	0.227406620744997\\
67.8836652267819	0.22714934087652\\
67.9336679265659	0.227618602463614\\
67.9836706263499	0.227690235351587\\
68.0336733261339	0.227837173379935\\
68.0836760259179	0.228408125072835\\
68.1336787257019	0.228448291047751\\
68.183681425486	0.228079093063157\\
68.23368412527	0.227327815626012\\
68.283686825054	0.227052969375949\\
68.333689524838	0.226746614721429\\
68.383692224622	0.226434829804418\\
68.4336949244061	0.226610061979202\\
68.4836976241901	0.226340232830462\\
68.5337003239741	0.226646742762452\\
68.5837030237581	0.226580297187269\\
68.6337057235421	0.226695794527674\\
68.6837084233261	0.226257919852844\\
68.7337111231102	0.226205363321338\\
68.7837138228942	0.226515440713694\\
68.8337165226782	0.227003758094325\\
68.8837192224622	0.22711941622174\\
68.9337219222462	0.226741358759102\\
68.9837246220302	0.226956661451517\\
69.0337273218143	0.227177204943962\\
69.0837300215983	0.227655213972991\\
69.1337327213823	0.227326048888493\\
69.1837354211663	0.227376923985825\\
69.2337381209503	0.227319130717815\\
69.2837408207344	0.227578138183414\\
69.3337435205184	0.227347142970338\\
69.3837462203024	0.227541600837216\\
69.4337489200864	0.227370026449022\\
69.4837516198704	0.227312096753621\\
69.5337543196544	0.227441571157644\\
69.5837570194385	0.226905672147285\\
69.6337597192225	0.226669433326853\\
69.6837624190065	0.227047662009919\\
69.7337651187905	0.227348916321184\\
69.7837678185745	0.227159729101177\\
69.8337705183585	0.227154504937698\\
69.8837732181426	0.227024971077747\\
69.9337759179266	0.227494328659743\\
69.9837786177106	0.228135054219434\\
70.0337813174946	0.228261066731367\\
70.0837840172786	0.228077234370484\\
70.1337867170626	0.227301634220992\\
70.1837894168467	0.227028491548167\\
70.2337921166307	0.227103847987583\\
70.2837948164147	0.226867610939524\\
70.3337975161987	0.226723826073888\\
70.3838002159827	0.226595946763771\\
70.4338029157667	0.226816707711521\\
70.4838056155508	0.226930546123243\\
70.5338083153348	0.226906095921723\\
70.5838110151188	0.226709811050348\\
70.6338137149028	0.226830568400871\\
70.6838164146868	0.226718556109747\\
70.7338191144708	0.226506590256718\\
70.7838218142549	0.22648717111665\\
70.8338245140389	0.226750052453913\\
70.8838272138229	0.22698999273731\\
70.9338299136069	0.226867371123008\\
70.9838326133909	0.227039039599266\\
71.0338353131749	0.227427616418656\\
71.083838012959	0.227355957791133\\
71.133840712743	0.227515182888692\\
71.183843412527	0.227588768293458\\
71.233846112311	0.22743666743276\\
71.283848812095	0.227453878608007\\
71.3338515118791	0.227035431167785\\
71.3838542116631	0.227163287213078\\
71.4338569114471	0.227007489486918\\
71.4838596112311	0.227107179704546\\
71.5338623110151	0.227576491447949\\
71.5838650107991	0.227963450897876\\
71.6338677105831	0.22733665864003\\
71.6838704103672	0.227156352042792\\
71.7338731101512	0.22706339700041\\
71.7838758099352	0.22723145512094\\
71.8338785097192	0.226988198986808\\
71.8838812095032	0.2270477962494\\
71.9338839092873	0.227231715816556\\
71.9838866090713	0.227184321815122\\
72.0338893088553	0.227410199135695\\
72.0838920086393	0.22780775648971\\
72.1338947084233	0.227720129519331\\
72.1838974082073	0.227544977516418\\
72.2339001079914	0.227557131988372\\
72.2839028077754	0.227385664799045\\
72.3339055075594	0.227179060240081\\
72.3839082073434	0.226925134230673\\
72.4339109071274	0.227417117025614\\
72.4839136069115	0.227709582878879\\
72.5339163066955	0.227914323622725\\
72.5839190064795	0.227886356675259\\
72.6339217062635	0.227930154822992\\
72.6839244060475	0.227623901936983\\
72.7339271058315	0.227476676062433\\
72.7839298056156	0.227126553568641\\
72.8339325053996	0.227140658575291\\
72.8839352051836	0.227156236381577\\
72.9339379049676	0.22750823894483\\
72.9839406047516	0.227685203287089\\
73.0339433045356	0.227965323619949\\
73.0839460043197	0.227651878075114\\
73.1339487041037	0.227261425086144\\
73.1839514038877	0.226797461698874\\
73.2339541036717	0.226683531397192\\
73.2839568034557	0.227082832842588\\
73.3339595032398	0.227469901691643\\
73.3839622030238	0.227735821254563\\
73.4339649028078	0.22799158539563\\
73.4839676025918	0.227291128335084\\
73.5339703023758	0.226709768563354\\
73.5839730021598	0.226578654098223\\
73.6339757019438	0.226060103194358\\
73.6839784017279	0.226079312896566\\
73.7339811015119	0.226103898691942\\
73.7839838012959	0.226361325420596\\
73.8339865010799	0.226210754210587\\
73.8839892008639	0.22654875189556\\
73.933991900648	0.226375154311748\\
73.983994600432	0.226068778024761\\
74.033997300216	0.226187827759521\\
74.084	0.226275517337194\\
74.134002699784	0.226534628114258\\
74.184005399568	0.226536490918794\\
74.2340080993521	0.226983050977593\\
74.2840107991361	0.227200203238592\\
74.3340134989201	0.227219376662465\\
74.3840161987041	0.227357807776448\\
74.4340188984881	0.226969115053647\\
74.4840215982721	0.226541760186137\\
74.5340242980562	0.226678328429215\\
74.5840269978402	0.227186166227889\\
74.6340296976242	0.227532753582388\\
74.6840323974082	0.22711949294522\\
74.7340350971922	0.227131880975317\\
74.7840377969763	0.226886778470079\\
74.8340404967603	0.226618809860777\\
74.8840431965443	0.226103943890954\\
74.9340458963283	0.225951650651653\\
74.9840485961123	0.22597612800623\\
75.0340512958963	0.226012857081478\\
75.0840539956804	0.226372025233828\\
75.1340566954644	0.226319454436018\\
75.1840593952484	0.22624228638923\\
75.2340620950324	0.226079456372338\\
75.2840647948164	0.226047846078021\\
75.3340674946004	0.226051481237267\\
75.3840701943844	0.226291184471901\\
75.4340728941685	0.226352576639098\\
75.4840755939525	0.226352466419687\\
75.5340782937365	0.22633845015395\\
75.5840809935205	0.225923416112247\\
75.6340836933045	0.225732483505411\\
75.6840863930885	0.225742954972336\\
75.7340890928726	0.225862195875967\\
75.7840917926566	0.225758868919044\\
75.8340944924406	0.225594184963269\\
75.8840971922246	0.226042576618594\\
75.9340998920086	0.226207176563544\\
75.9841025917927	0.226674869987872\\
76.0341052915767	0.226569705227704\\
76.0841079913607	0.227065315844649\\
76.1341106911447	0.227263319470776\\
76.1841133909287	0.226709958379363\\
76.2341160907127	0.226834205941133\\
76.2841187904968	0.226527585686737\\
76.3341214902808	0.226814751098542\\
76.3841241900648	0.227170267270074\\
76.4341268898488	0.227326137955795\\
76.4841295896328	0.227769222171422\\
76.5341322894169	0.227658930105112\\
76.5841349892009	0.226972365566557\\
76.6341376889849	0.227063461430736\\
76.6841403887689	0.227070619105458\\
76.7341430885529	0.226737788411929\\
76.7841457883369	0.226785103425983\\
76.834148488121	0.227254352535118\\
76.884151187905	0.227622100696805\\
76.934153887689	0.227966967802951\\
76.984156587473	0.227459260151877\\
77.034159287257	0.227060126490483\\
77.084161987041	0.227142345400423\\
77.1341646868251	0.227147709389291\\
77.1841673866091	0.227287597684309\\
77.2341700863931	0.227809399484808\\
77.2841727861771	0.228040580442081\\
77.3341754859611	0.228240115045913\\
77.3841781857451	0.228287217031012\\
77.4341808855292	0.227583506489992\\
77.4841835853132	0.226907613790637\\
77.5341862850972	0.226198471084379\\
77.5841889848812	0.225800863252245\\
77.6341916846652	0.226292992588555\\
77.6841943844492	0.22652577233064\\
77.7341970842333	0.227168435203466\\
77.7841997840173	0.227560737529728\\
77.8342024838013	0.228019517279264\\
77.8842051835853	0.228166585510011\\
77.9342078833693	0.227392672471546\\
77.9842105831534	0.226646741080907\\
78.0342132829374	0.226371912043483\\
78.0842159827214	0.226004089530256\\
78.1342186825054	0.225835870025411\\
78.1842213822894	0.225839347922292\\
78.2342240820734	0.226100361072936\\
78.2842267818575	0.225748386030993\\
78.3342294816415	0.22581482153241\\
78.3842321814255	0.225532996866684\\
78.4342348812095	0.225489135494228\\
78.4842375809935	0.225728958328691\\
78.5342402807775	0.225883030352626\\
78.5842429805616	0.225883007659046\\
78.6342456803456	0.225860322397814\\
78.6842483801296	0.225615203108192\\
78.7342510799136	0.225350777965655\\
78.7842537796976	0.224830278935497\\
78.8342564794817	0.224536129369535\\
78.8842591792657	0.224483340143306\\
78.9342618790497	0.224557023322254\\
78.9842645788337	0.224467732484615\\
79.0342672786177	0.224190881247234\\
79.0842699784017	0.223959488736878\\
79.1342726781857	0.224111956333133\\
79.1842753779698	0.224388786617838\\
79.2342780777538	0.224527287664327\\
79.2842807775378	0.225043970277815\\
79.3342834773218	0.225284168587219\\
79.3842861771058	0.225355935248789\\
79.4342888768899	0.225492677255886\\
79.4842915766739	0.225170297187574\\
79.5342942764579	0.22496006401053\\
79.5842969762419	0.225044109560655\\
79.6342996760259	0.225352387514199\\
79.6843023758099	0.225422355400427\\
79.734305075594	0.225089612561671\\
79.784307775378	0.225105532850016\\
79.834310475162	0.224742923307069\\
79.884313174946	0.224753243234981\\
79.93431587473	0.224961699124244\\
79.984318574514	0.225319050886594\\
80.0343212742981	0.225415481210633\\
80.0843239740821	0.225853359755869\\
80.1343266738661	0.225639547277676\\
80.1843293736501	0.225744644360478\\
80.2343320734341	0.225858624266574\\
80.2843347732181	0.226373645830824\\
80.3343374730022	0.22670633837153\\
80.3843401727862	0.226366543418611\\
80.4343428725702	0.22560990033447\\
80.4843455723542	0.225361187951978\\
80.5343482721382	0.225508190431281\\
80.5843509719223	0.225769249165602\\
80.6343536717063	0.225797450016728\\
80.6843563714903	0.225779892118661\\
80.7343590712743	0.226098417023271\\
80.7843617710583	0.226525720871782\\
80.8343644708423	0.22626657100547\\
80.8843671706263	0.22615298150602\\
80.9343698704104	0.226177463583559\\
80.9843725701944	0.226345501555486\\
81.0343752699784	0.22653997323741\\
81.0843779697624	0.226518858993858\\
81.1343806695464	0.226487330802165\\
81.1843833693305	0.227072265561315\\
81.2343860691145	0.227279066251018\\
81.2843887688985	0.227531120229253\\
81.3343914686825	0.227711366428718\\
81.3843941684665	0.227644809087872\\
81.4343968682505	0.227389222295694\\
81.4843995680346	0.226760493708642\\
81.5344022678186	0.226371815538025\\
81.5844049676026	0.226065362503017\\
81.6344076673866	0.226237042191789\\
81.6844103671706	0.226089931916124\\
81.7344130669546	0.225720230860688\\
81.7844157667387	0.22549262045075\\
81.8344184665227	0.225643221108665\\
81.8844211663067	0.226023165353749\\
81.9344238660907	0.225974189084588\\
81.9844265658747	0.22573951252602\\
82.0344292656588	0.225089765593038\\
82.0844319654428	0.225072055829493\\
82.1344346652268	0.225434593042995\\
82.1844373650108	0.225707920790958\\
82.2344400647948	0.226284179412601\\
82.2844427645788	0.226189574606463\\
82.3344454643629	0.225515362037069\\
82.3844481641469	0.224933759103168\\
82.4344508639309	0.224837505188081\\
82.4844535637149	0.224968693661684\\
82.5344562634989	0.224891617880986\\
82.5844589632829	0.224318602284448\\
82.634461663067	0.223935175873868\\
82.684464362851	0.22035270169412\\
82.734467062635	0.207916524503569\\
82.784469762419	0.183328983417589\\
82.8294721922246	0.153412554190566\\
82.8744746220302	0.12306992036011\\
82.9194770518359	0.0996001673622788\\
82.9694797516199	0.0835282550061769\\
83.0194824514039	0.0740236799496861\\
83.0694851511879	0.0640471313885887\\
83.1194878509719	0.0531646232911571\\
83.1694905507559	0.0428059383998715\\
83.21949325054	0.0352712113741453\\
83.269495950324	0.0305683245920346\\
83.319498650108	0.0265279956712241\\
83.369501349892	0.0224460666228536\\
83.419504049676	0.0182972973913731\\
83.4695067494601	0.0153471979930889\\
83.5195094492441	0.0136158592903449\\
83.5695121490281	0.0119780296127673\\
83.6195148488121	0.0104244701914272\\
83.6695175485961	0.00904560210827476\\
83.7195202483801	0.00828147419313903\\
83.7695229481642	0.00735016887896697\\
83.8195256479482	0.007080625498911\\
83.8695283477322	0.00631999370314961\\
83.9195310475162	0.00531497205592703\\
83.9695337473002	0.00489239725917009\\
84.0195364470842	0.00448783180672328\\
84.0695391468683	0.00415345575496839\\
84.1195418466523	0.00358529011788749\\
84.1695445464363	0.00279418464566026\\
84.2195472462203	0.00212527764728735\\
84.2695499460043	0.00263225599875939\\
84.3195526457883	0.00377037962636581\\
84.3695553455724	0.00448961307073124\\
84.4195580453564	0.00433671471979191\\
84.4695607451404	0.00421999477444365\\
84.5195634449244	0.00443564368657821\\
84.5695661447084	0.00436913291121546\\
84.6195688444924	0.00520353342184274\\
84.6695715442765	0.00544432980749576\\
84.7195742440605	0.0056420470379737\\
84.7695769438445	0.00569967692191078\\
84.8195796436285	0.00574993661657542\\
84.8695823434125	0.00574649726373412\\
84.9195850431965	0.00573732698007875\\
84.9695877429806	0.00569234759153559\\
85.0195904427646	0.00564924806138927\\
85.0695931425486	0.00554143978801434\\
85.1195958423326	0.00566186775627214\\
85.1695985421166	0.00587409438145352\\
85.2196012419006	0.00582723545126923\\
85.2696039416847	0.00558444789600639\\
85.3196066414687	0.00548398075420764\\
85.3696093412527	0.00536334303105489\\
85.4196120410367	0.00559358981062458\\
85.4696147408207	0.00530966112351205\\
85.5196174406048	0.00533113191316668\\
85.5696201403888	0.0051675845428977\\
85.6196228401728	0.00550733916916022\\
85.6696255399568	0.00558094940450933\\
85.7196282397408	0.00563136891314282\\
85.7696309395248	0.0055774186954233\\
85.8196336393089	0.00566194855139695\\
85.8696363390929	0.00567428720235431\\
85.9196390388769	0.0053757925867606\\
85.9696417386609	0.00508650600980657\\
86.0196444384449	0.00411563611157582\\
86.0696471382289	0.00420916470850511\\
86.119649838013	0.00469822052844111\\
86.169652537797	0.00515118574319424\\
86.219655237581	0.00558447830372627\\
86.269657937365	0.00565826705706104\\
86.319660637149	0.00565469723049222\\
86.369663336933	0.00594239166309741\\
86.4196660367171	0.00647463328721985\\
86.4696687365011	0.00692219203157614\\
86.5196714362851	0.00714340651498583\\
86.5696741360691	0.00701212803431326\\
86.6196768358531	0.00684671257649742\\
86.6696795356372	0.0063954935961842\\
86.7196822354212	0.00634147460325669\\
86.7696849352052	0.00645842249476133\\
86.8196876349892	0.00644944239785992\\
86.8696903347732	0.00632904538683023\\
86.9196930345572	0.00592630228214516\\
86.9696957343413	0.00569604666040618\\
87.0196984341253	0.00574103695938795\\
87.0697011339093	0.00593697255520646\\
87.1197038336933	0.00630010605433688\\
87.1697065334773	0.00576972556807831\\
87.2197092332613	0.0052591332004222\\
87.2697119330454	0.00507735581727484\\
87.3197146328294	0.00518545437518709\\
87.3697173326134	0.00569443521090498\\
87.4197200323974	0.00558627965336371\\
87.4697227321814	0.00554142821307241\\
87.5197254319655	0.00556125918506151\\
87.5697281317495	0.00574109659625791\\
87.6197308315335	0.00598550313166618\\
87.6697335313175	0.00605017032841072\\
87.7197362311015	0.00582539429039633\\
87.7697389308855	0.00547856050937998\\
87.8197416306695	0.00597107952395287\\
87.8697443304536	0.00575900556352167\\
87.9197470302376	0.00560434771982412\\
87.9697497300216	0.00531131212919511\\
88.0197524298056	0.00559360949780868\\
88.0697551295896	0.00571385787175204\\
88.1197578293737	0.00573193635581792\\
88.1697605291577	0.00591189395675378\\
88.2197632289417	0.00610056030840477\\
88.2697659287257	0.00631454623234289\\
88.3197686285097	0.00639546412215314\\
88.3697713282937	0.00615614753873913\\
88.4197740280778	0.00597483097353473\\
88.4697767278618	0.005645726602368\\
88.5197794276458	0.00506311604119508\\
88.5697821274298	0.00440143372962299\\
88.6197848272138	0.00386395829387582\\
88.6697875269979	0.00376686943519728\\
88.7197902267819	0.00443562269655846\\
88.7697929265659	0.00509179384860369\\
88.8197956263499	0.00489408742832628\\
88.8697983261339	0.0045704517534367\\
88.9198010259179	0.00470699967205324\\
88.969803725702	0.00452548325542936\\
89.019806425486	0.0046945306376808\\
89.06980912527	0.00500741712408223\\
89.119811825054	0.0052375347252177\\
89.169814524838	0.0053255827683078\\
89.219817224622	0.00536343111505793\\
89.2698199244061	0.00518547549966283\\
89.3198226241901	0.00481851024787414\\
89.3698253239741	0.00480767921814178\\
89.4198280237581	0.00476294109881334\\
89.4698307235421	0.0047289004136007\\
89.5198334233261	0.00484561571415186\\
89.5698361231102	0.00483838946767312\\
89.6198388228942	0.00469281112172522\\
89.6698415226782	0.00442127429495032\\
89.7198442224622	0.00458299225642242\\
89.7698469222462	0.00455964087958895\\
89.8198496220302	0.0043961344652495\\
89.8698523218143	0.0038891471054401\\
89.9198550215983	0.00355650045240237\\
89.9698577213823	0.00360313053698644\\
90.0198604211663	0.00388185816533716\\
90.0698631209503	0.00404547601205708\\
90.1198658207343	0.00398791813281693\\
90.1698685205184	0.00413541695858906\\
90.2198712203024	0.00427017546977268\\
90.2698739200864	0.00488698903273049\\
90.3198766198704	0.00526268239567088\\
90.3698793196544	0.00478797758899237\\
90.4198820194384	0.00294157309306403\\
90.4698847192225	0.00150843833897814\\
90.5198874190065	0.000742709661971891\\
90.5698901187905	0.00087552959325023\\
90.6198928185745	0.00152655883007551\\
90.6698955183585	0.00217370736264713\\
90.7198982181426	0.00288207520349131\\
90.7699009179266	0.00305294427411397\\
90.8199036177106	0.00334256250167673\\
90.8699063174946	0.00425227544946609\\
90.9199090172786	0.00477191882215444\\
90.9699117170626	0.00515125500376009\\
91.0199144168467	0.0056366180481516\\
91.0699171166307	0.00667959577381927\\
91.1199198164147	0.0073322011123369\\
91.1699225161987	0.00780138910052387\\
91.2199252159827	0.00789672627052344\\
91.2699279157667	0.00738978010240725\\
91.3199306155508	0.00739852858948746\\
91.3699333153348	0.00751375742176313\\
91.4199360151188	0.00749391645678491\\
91.4699387149028	0.00777444750897286\\
91.5199414146868	0.00818815406393797\\
91.5699441144708	0.00828143290751964\\
91.6199468142549	0.00830476184911301\\
91.6699495140389	0.00843438619274514\\
91.7199522138229	0.00866986886449887\\
91.7699549136069	0.00859065090246094\\
91.8199576133909	0.00868423634309281\\
91.8699603131749	0.00799196061403987\\
91.919963012959	0.00767013220219464\\
91.969965712743	0.0074958261159595\\
92.019968412527	0.00745080583998396\\
92.069971112311	0.00739513043255994\\
92.119973812095	0.00740409904751166\\
92.1699765118791	0.00702485917642045\\
92.2199792116631	0.00649975256353147\\
92.2699819114471	0.00651950206378776\\
92.3199846112311	0.00627858656317666\\
92.3699873110151	0.00617968769436338\\
92.4199900107991	0.00621938835006525\\
92.4699927105832	0.00579686477326045\\
92.5199954103672	0.00524848408128662\\
92.5699981101512	0.00525540557448442\\
92.605	0.00521397637557539\\
};
\addlegendentry{Measured angle [rad]};

\end{axis}
\end{tikzpicture}%
	}
	\subfloat[][]{\setlength\figureheight{4cm}
		\setlength\figurewidth{6cm}
		% This file was created by matlab2tikz v0.4.7 running on MATLAB 8.0.
% Copyright (c) 2008--2014, Nico Schlömer <nico.schloemer@gmail.com>
% All rights reserved.
% Minimal pgfplots version: 1.3
% 
% The latest updates can be retrieved from
%   http://www.mathworks.com/matlabcentral/fileexchange/22022-matlab2tikz
% where you can also make suggestions and rate matlab2tikz.
% 
\begin{tikzpicture}

\begin{axis}[%
width=\figurewidth,
height=\figureheight,
scale only axis,
xmin=0,
xmax=92.605,
xlabel={time [s]},
ymin=-0.1,
ymax=0.25,
ylabel={distance [m] / angle [rad]},
axis x line*=bottom,
axis y line*=left,
legend style={at={(0.03,0.97)},anchor=north west,draw=black,fill=white,legend cell align=left},
%legend columns = 2
]
\addplot [color=black!50!green,solid,line width=0.2pt]
  table[row sep=crcr]{0	0\\
0.0500026997840173	0\\
0.100005399568035	0\\
0.150008099352052	0\\
0.200010799136069	0\\
0.250013498920086	0\\
0.300016198704104	0\\
0.350018898488121	0\\
0.400021598272138	0\\
0.450024298056156	0\\
0.500026997840173	0\\
0.55002969762419	0\\
0.600032397408207	0\\
0.650035097192225	0\\
0.700037796976242	0\\
0.750040496760259	0\\
0.800043196544276	0\\
0.850045896328294	0\\
0.900048596112311	0\\
0.950051295896328	0\\
1.00005399568035	0\\
1.05005669546436	0\\
1.10005939524838	0\\
1.1500620950324	0\\
1.20006479481641	0\\
1.25006749460043	0\\
1.30007019438445	0\\
1.35007289416847	0\\
1.40007559395248	0\\
1.4500782937365	0\\
1.50008099352052	0\\
1.55008369330454	0\\
1.60008639308855	0\\
1.65008909287257	0\\
1.70009179265659	0\\
1.7500944924406	0\\
1.80009719222462	0\\
1.85009989200864	0\\
1.90010259179266	0\\
1.95010529157667	0\\
2.00010799136069	0\\
2.05011069114471	0\\
2.10011339092873	0\\
2.15011609071274	0\\
2.20011879049676	0\\
2.25012149028078	0\\
2.30012419006479	0\\
2.35012688984881	0\\
2.40012958963283	0\\
2.45013228941685	0\\
2.50013498920086	0\\
2.55013768898488	0\\
2.6001403887689	0\\
2.65014308855292	0\\
2.70014578833693	0\\
2.75014848812095	0\\
2.80015118790497	0\\
2.85015388768899	0\\
2.900156587473	0\\
2.95015928725702	0\\
3.00016198704104	0\\
3.05016468682505	0\\
3.10016738660907	0\\
3.15017008639309	0\\
3.20017278617711	0\\
3.25017548596112	0\\
3.30017818574514	0\\
3.35018088552916	0\\
3.40018358531318	0\\
3.45018628509719	0\\
3.50018898488121	0\\
3.55019168466523	0\\
3.60019438444924	0\\
3.65019708423326	0\\
3.70019978401728	0\\
3.7502024838013	0\\
3.80020518358531	0\\
3.85020788336933	0\\
3.90021058315335	0\\
3.95021328293736	0\\
4.00021598272138	0\\
4.0502186825054	0\\
4.10022138228942	0\\
4.15022408207343	0\\
4.20022678185745	0\\
4.25022948164147	0\\
4.30023218142549	0\\
4.3502348812095	0\\
4.40023758099352	0\\
4.45024028077754	0\\
4.50024298056155	0\\
4.55024568034557	0\\
4.60024838012959	0\\
4.65025107991361	0\\
4.70025377969762	0\\
4.75025647948164	0\\
4.80025917926566	0\\
4.85026187904968	0\\
4.90026457883369	0\\
4.95026727861771	0\\
5.00026997840173	0\\
5.05027267818575	0\\
5.10027537796976	0\\
5.15027807775378	0\\
5.2002807775378	0\\
5.25028347732181	0\\
5.30028617710583	0\\
5.35028887688985	0\\
5.40029157667387	0\\
5.45029427645788	0\\
5.5002969762419	0\\
5.55029967602592	0\\
5.60030237580994	0\\
5.65030507559395	0\\
5.70030777537797	0\\
5.75031047516199	0\\
5.800313174946	0\\
5.85031587473002	0\\
5.90031857451404	0\\
5.95032127429806	0\\
6.00032397408207	0\\
6.05032667386609	0\\
6.10032937365011	0\\
6.15033207343413	0\\
6.20033477321814	0\\
6.25033747300216	0\\
6.30034017278618	0\\
6.3503428725702	0\\
6.40034557235421	0\\
6.45034827213823	0\\
6.50035097192225	0\\
6.55035367170626	0\\
6.60035637149028	0\\
6.6503590712743	0\\
6.70036177105832	0\\
6.75036447084233	0\\
6.80036717062635	0\\
6.85036987041037	0\\
6.90037257019439	0\\
6.9503752699784	0\\
7.00037796976242	0\\
7.05038066954644	0\\
7.10038336933045	0\\
7.15038606911447	0\\
7.20038876889849	0\\
7.25039146868251	0\\
7.30039416846652	0\\
7.35039686825054	0\\
7.40039956803456	0\\
7.45040226781857	0\\
7.50040496760259	0\\
7.55040766738661	0\\
7.60041036717063	0\\
7.65041306695464	0\\
7.70041576673866	0\\
7.75041846652268	0\\
7.8004211663067	0\\
7.85042386609071	0\\
7.90042656587473	0\\
7.95042926565875	0\\
8.00043196544276	0\\
8.05043466522678	0\\
8.1004373650108	0\\
8.15044006479482	0\\
8.20044276457883	0\\
8.25044546436285	0\\
8.30044816414687	0\\
8.35045086393089	0\\
8.4004535637149	0\\
8.45045626349892	0\\
8.50045896328294	0\\
8.55046166306696	0\\
8.60046436285097	0\\
8.65046706263499	0\\
8.70046976241901	0\\
8.75047246220302	0\\
8.80047516198704	0\\
8.85047786177106	0\\
8.90048056155508	0\\
8.95048326133909	0\\
9.00048596112311	0\\
9.05048866090713	0\\
9.10049136069114	0\\
9.15049406047516	0\\
9.20049676025918	0\\
9.2504994600432	0\\
9.30050215982721	0\\
9.35050485961123	0\\
9.40050755939525	0\\
9.45051025917927	0\\
9.50051295896328	0\\
9.5505156587473	0\\
9.60051835853132	0\\
9.65052105831533	0\\
9.70052375809935	0\\
9.75052645788337	0\\
9.80052915766739	0\\
9.8505318574514	0\\
9.90053455723542	0\\
9.95053725701944	0\\
10.0005399568035	0\\
10.0505426565875	0\\
10.1005453563715	0\\
10.1505480561555	0\\
10.2005507559395	0\\
10.2505534557235	0\\
10.3005561555076	0\\
10.3505588552916	0\\
10.4005615550756	0\\
10.4505642548596	0\\
10.5005669546436	0\\
10.5505696544276	0\\
10.6005723542117	0\\
10.6505750539957	0\\
10.7005777537797	0\\
10.7505804535637	0\\
10.8005831533477	0\\
10.8505858531317	0\\
10.9005885529158	0\\
10.9505912526998	0\\
11.0005939524838	0\\
11.0505966522678	0\\
11.1005993520518	0\\
11.1506020518359	0\\
11.2006047516199	0\\
11.2506074514039	0\\
11.3006101511879	0\\
11.3506128509719	0\\
11.4006155507559	0\\
11.45061825054	0\\
11.500620950324	0\\
11.550623650108	0\\
11.600626349892	0\\
11.650629049676	0\\
11.70063174946	0\\
11.7506344492441	0\\
11.8006371490281	0\\
11.8506398488121	0\\
11.9006425485961	0\\
11.9506452483801	0\\
12.0006479481641	0\\
12.0506506479482	0\\
12.1006533477322	0\\
12.1506560475162	0\\
12.2006587473002	0\\
12.2506614470842	0\\
12.3006641468683	0\\
12.3506668466523	0\\
12.4006695464363	0\\
12.4506722462203	0\\
12.5006749460043	0\\
12.5506776457883	0\\
12.6006803455724	0\\
12.6506830453564	0\\
12.7006857451404	0\\
12.7506884449244	0\\
12.8006911447084	0\\
12.8506938444924	0\\
12.9006965442765	0\\
12.9506992440605	0\\
13.0007019438445	0\\
13.0507046436285	0\\
13.1007073434125	0\\
13.1507100431965	0\\
13.2007127429806	0\\
13.2507154427646	0\\
13.3007181425486	0\\
13.3507208423326	0\\
13.4007235421166	0\\
13.4507262419006	0\\
13.5007289416847	0\\
13.5507316414687	0\\
13.6007343412527	0\\
13.6507370410367	0\\
13.7007397408207	0\\
13.7507424406048	0\\
13.8007451403888	0\\
13.8507478401728	0\\
13.9007505399568	0\\
13.9507532397408	0\\
14.0007559395248	0\\
14.0507586393089	0\\
14.1007613390929	0\\
14.1507640388769	0\\
14.2007667386609	0\\
14.2507694384449	0\\
14.3007721382289	0\\
14.350774838013	0\\
14.400777537797	0\\
14.450780237581	0\\
14.500782937365	0\\
14.550785637149	0\\
14.600788336933	0\\
14.6507910367171	0\\
14.7007937365011	0\\
14.7507964362851	0\\
14.8007991360691	0\\
14.8508018358531	0\\
14.9008045356372	0\\
14.9508072354212	0\\
15.0008099352052	0\\
15.0508126349892	0\\
15.1008153347732	0\\
15.1508180345572	0\\
15.2008207343413	0\\
15.2508234341253	0\\
15.3008261339093	0\\
15.3508288336933	0\\
15.4008315334773	0\\
15.4508342332613	0\\
15.5008369330454	0\\
15.5508396328294	0\\
15.6008423326134	0\\
15.6508450323974	0\\
15.7008477321814	0\\
15.7508504319654	0\\
15.8008531317495	0\\
15.8508558315335	0\\
15.9008585313175	0\\
15.9508612311015	0\\
16.0008639308855	0\\
16.0508666306695	0\\
16.1008693304536	0\\
16.1508720302376	0\\
16.2008747300216	0\\
16.2508774298056	0\\
16.3008801295896	0\\
16.3508828293737	0\\
16.4008855291577	0\\
16.4508882289417	0\\
16.5008909287257	0\\
16.5508936285097	0\\
16.6008963282937	0\\
16.6508990280778	0\\
16.7009017278618	0\\
16.7509044276458	0\\
16.8009071274298	0\\
16.8509098272138	0\\
16.9009125269978	0\\
16.9509152267819	0\\
17.0009179265659	0\\
17.0509206263499	0\\
17.1009233261339	0\\
17.1509260259179	0\\
17.2009287257019	0\\
17.250931425486	0\\
17.30093412527	0\\
17.350936825054	0\\
17.400939524838	0\\
17.450942224622	0\\
17.500944924406	0\\
17.5509476241901	0\\
17.6009503239741	0\\
17.6509530237581	0\\
17.7009557235421	0\\
17.7509584233261	0\\
17.8009611231102	0\\
17.8509638228942	0\\
17.9009665226782	0\\
17.9509692224622	0\\
18.0009719222462	0\\
18.0509746220302	0\\
18.1009773218143	0\\
18.1509800215983	0\\
18.2009827213823	0\\
18.2509854211663	0\\
18.3009881209503	0\\
18.3509908207343	0\\
18.4009935205184	0\\
18.4509962203024	0\\
18.5009989200864	0\\
18.5510016198704	0\\
18.6010043196544	0\\
18.6510070194384	0\\
18.7010097192225	0\\
18.7510124190065	0\\
18.8010151187905	0\\
18.8510178185745	0\\
18.9010205183585	0\\
18.9510232181425	0\\
19.0010259179266	0\\
19.0510286177106	0\\
19.1010313174946	0\\
19.1510340172786	0\\
19.2010367170626	0\\
19.2510394168467	0\\
19.3010421166307	0\\
19.3510448164147	0\\
19.4010475161987	0\\
19.4510502159827	0\\
19.5010529157667	0\\
19.5510556155508	0\\
19.6010583153348	0\\
19.6510610151188	0\\
19.7010637149028	0\\
19.7510664146868	0\\
19.8010691144708	0\\
19.8510718142549	0\\
19.9010745140389	0\\
19.9510772138229	0\\
20.0010799136069	0\\
20.0510826133909	0\\
20.1010853131749	0\\
20.151088012959	0\\
20.201090712743	0\\
20.251093412527	0\\
20.301096112311	0\\
20.351098812095	0\\
20.4011015118791	0\\
20.4511042116631	0\\
20.5011069114471	0\\
20.5511096112311	0\\
20.6011123110151	0\\
20.6511150107991	0\\
20.7011177105832	0\\
20.7511204103672	0\\
20.8011231101512	0\\
20.8511258099352	0\\
20.9011285097192	0\\
20.9511312095032	0\\
21.0011339092873	0\\
21.0511366090713	0\\
21.1011393088553	0\\
21.1511420086393	0\\
21.2011447084233	0\\
21.2511474082073	0\\
21.3011501079914	0\\
21.3511528077754	0\\
21.4011555075594	0\\
21.4511582073434	0\\
21.5011609071274	0\\
21.5511636069114	0\\
21.6011663066955	0\\
21.6511690064795	0\\
21.7011717062635	0\\
21.7511744060475	0\\
21.8011771058315	0\\
21.8511798056156	0\\
21.9011825053996	0\\
21.9511852051836	0\\
22.0011879049676	0\\
22.0511906047516	0\\
22.1011933045356	0\\
22.1511960043197	0\\
22.2011987041037	0\\
22.2512014038877	0\\
22.3012041036717	0\\
22.3512068034557	0\\
22.4012095032397	0\\
22.4512122030238	0\\
22.5012149028078	0\\
22.5512176025918	0\\
22.6012203023758	0\\
22.6512230021598	0\\
22.7012257019438	0\\
22.7512284017279	0\\
22.8012311015119	0\\
22.8512338012959	0\\
22.9012365010799	0\\
22.9512392008639	0\\
23.0012419006479	0\\
23.051244600432	0\\
23.101247300216	0\\
23.15125	0\\
23.201252699784	0\\
23.251255399568	0\\
23.3012580993521	0\\
23.3512607991361	0\\
23.4012634989201	0\\
23.4512661987041	0\\
23.5012688984881	0\\
23.5512715982721	0\\
23.6012742980562	0\\
23.6512769978402	0\\
23.7012796976242	0\\
23.7512823974082	0\\
23.8012850971922	0\\
23.8512877969762	0\\
23.9012904967603	0\\
23.9512931965443	0\\
24.0012958963283	0\\
24.0512985961123	0\\
24.1013012958963	0\\
24.1513039956803	0\\
24.2013066954644	0\\
24.2513093952484	0\\
24.3013120950324	0\\
24.3513147948164	0\\
24.4013174946004	0\\
24.4513201943845	0\\
24.5013228941685	0\\
24.5513255939525	0\\
24.6013282937365	0\\
24.6513309935205	0\\
24.7013336933045	0\\
24.7513363930886	0\\
24.8013390928726	0\\
24.8513417926566	0\\
24.9013444924406	0\\
24.9513471922246	0\\
25.0013498920086	0\\
25.0513525917927	0\\
25.1013552915767	0\\
25.1513579913607	0\\
25.2013606911447	0\\
25.2513633909287	0\\
25.3013660907127	0\\
25.3513687904968	0\\
25.4013714902808	0\\
25.4513741900648	0\\
25.5013768898488	0\\
25.5513795896328	0\\
25.6013822894168	0\\
25.6513849892009	0\\
25.7013876889849	0\\
25.7513903887689	0\\
25.8013930885529	0\\
25.8513957883369	0\\
25.901398488121	0\\
25.951401187905	0\\
26.001403887689	0\\
26.051406587473	0\\
26.101409287257	0\\
26.151411987041	0\\
26.2014146868251	0\\
26.2514173866091	0\\
26.3014200863931	0\\
26.3514227861771	0\\
26.4014254859611	0\\
26.4514281857451	0\\
26.5014308855292	0\\
26.5514335853132	0\\
26.6014362850972	0\\
26.6514389848812	0\\
26.7014416846652	0\\
26.7514443844492	0\\
26.8014470842333	0\\
26.8514497840173	0\\
26.9014524838013	0\\
26.9514551835853	0\\
27.0014578833693	0\\
27.0514605831534	0\\
27.1014632829374	0\\
27.1514659827214	0\\
27.2014686825054	0\\
27.2514713822894	0\\
27.3014740820734	0\\
27.3514767818575	0\\
27.4014794816415	0\\
27.4514821814255	0\\
27.5014848812095	0\\
27.5514875809935	0\\
27.6014902807775	0\\
27.6514929805616	0\\
27.7014956803456	0\\
27.7514983801296	0\\
27.8015010799136	0\\
27.8515037796976	0\\
27.9015064794816	0\\
27.9515091792657	0\\
28.0015118790497	0\\
28.0515145788337	0\\
28.1015172786177	0\\
28.1515199784017	0\\
28.2015226781857	0\\
28.2515253779698	0\\
28.3015280777538	0\\
28.3515307775378	0\\
28.4015334773218	0\\
28.4515361771058	0\\
28.5015388768899	0\\
28.5515415766739	0\\
28.6015442764579	0\\
28.6515469762419	0\\
28.7015496760259	0\\
28.7515523758099	0\\
28.801555075594	0\\
28.851557775378	0\\
28.901560475162	0\\
28.951563174946	0\\
29.00156587473	0\\
29.051568574514	0\\
29.1015712742981	0\\
29.1515739740821	0\\
29.2015766738661	0\\
29.2515793736501	0\\
29.3015820734341	0\\
29.3515847732181	0\\
29.4015874730022	0\\
29.4515901727862	0\\
29.5015928725702	0\\
29.5515955723542	0\\
29.6015982721382	0\\
29.6516009719222	0\\
29.7016036717063	0\\
29.7516063714903	0\\
29.8016090712743	0\\
29.8516117710583	0\\
29.9016144708423	0\\
29.9516171706264	0\\
30.0016198704104	0\\
30.0516225701944	0\\
30.1016252699784	0\\
30.1516279697624	0\\
30.2016306695464	0\\
30.2516333693305	0\\
30.3016360691145	0\\
30.3516387688985	0\\
30.4016414686825	0\\
30.4516441684665	0\\
30.5016468682505	0\\
30.5516495680346	0\\
30.6016522678186	0\\
30.6516549676026	0\\
30.7016576673866	0\\
30.7516603671706	0\\
30.8016630669546	0\\
30.8516657667387	0\\
30.9016684665227	0\\
30.9516711663067	0\\
31.0016738660907	0\\
31.0516765658747	0\\
31.1016792656587	0\\
31.1516819654428	0\\
31.2016846652268	0\\
31.2516873650108	0\\
31.3016900647948	0\\
31.3516927645788	0\\
31.4016954643629	0\\
31.4516981641469	0\\
31.5017008639309	0\\
31.5517035637149	0\\
31.6017062634989	0\\
31.6517089632829	0\\
31.701711663067	0\\
31.751714362851	0\\
31.801717062635	0\\
31.851719762419	0\\
31.901722462203	0\\
31.951725161987	0\\
32.0017278617711	0\\
32.0517305615551	0\\
32.1017332613391	0\\
32.1517359611231	0\\
32.2017386609071	0\\
32.2517413606911	0\\
32.3017440604752	0\\
32.3517467602592	0\\
32.4017494600432	0\\
32.4517521598272	0\\
32.5017548596112	0\\
32.5517575593952	0\\
32.6017602591793	0\\
32.6517629589633	0\\
32.7017656587473	0\\
32.7517683585313	0\\
32.8017710583153	0\\
32.8517737580993	0\\
32.9017764578834	0\\
32.9517791576674	0\\
33.0017818574514	0\\
33.0517845572354	0\\
33.1017872570194	0\\
33.1517899568035	0\\
33.2017926565875	0\\
33.2517953563715	0\\
33.3017980561555	0\\
33.3518007559395	0\\
33.4018034557235	0\\
33.4518061555076	0\\
33.5018088552916	0\\
33.5518115550756	0\\
33.6018142548596	0\\
33.6518169546436	0\\
33.7018196544277	0\\
33.7518223542117	0\\
33.8018250539957	0\\
33.8518277537797	0\\
33.9018304535637	0\\
33.9518331533477	0\\
34.0018358531318	0\\
34.0518385529158	0\\
34.1018412526998	0\\
34.1518439524838	0\\
34.2018466522678	0\\
34.2518493520518	0\\
34.3018520518359	0\\
34.3518547516199	0\\
34.4018574514039	0\\
34.4518601511879	0\\
34.5018628509719	0\\
34.5518655507559	0\\
34.60186825054	0\\
34.651870950324	0\\
34.701873650108	0\\
34.751876349892	0\\
34.801879049676	0\\
34.85188174946	0\\
34.9018844492441	0\\
34.9518871490281	0\\
35.0018898488121	0\\
35.0518925485961	0\\
35.1018952483801	0\\
35.1518979481641	0\\
35.2019006479482	0\\
35.2519033477322	0\\
35.3019060475162	0\\
35.3519087473002	0\\
35.4019114470842	0\\
35.4519141468683	0\\
35.5019168466523	0\\
35.5519195464363	0\\
35.6019222462203	0\\
35.6519249460043	0\\
35.7019276457883	0\\
35.7519303455724	0\\
35.8019330453564	0\\
35.8519357451404	0\\
35.9019384449244	0\\
35.9519411447084	0\\
36.0019438444924	0\\
36.0519465442765	0\\
36.1019492440605	0\\
36.1519519438445	0\\
36.2019546436285	0\\
36.2519573434125	0\\
36.3019600431965	0\\
36.3519627429806	0\\
36.4019654427646	0\\
36.4519681425486	0\\
36.5019708423326	0\\
36.5519735421166	0\\
36.6019762419006	0\\
36.6519789416847	0\\
36.7019816414687	0\\
36.7519843412527	0\\
36.8019870410367	0\\
36.8519897408207	0\\
36.9019924406048	0\\
36.9519951403888	0\\
37.0019978401728	0\\
37.0520005399568	0\\
37.1020032397408	0\\
37.1520059395248	0\\
37.2020086393089	0\\
37.2520113390929	0\\
37.3020140388769	0\\
37.3520167386609	0\\
37.4020194384449	0\\
37.4520221382289	0\\
37.502024838013	0\\
37.552027537797	0\\
37.602030237581	0\\
37.652032937365	0\\
37.702035637149	0\\
37.752038336933	0\\
37.8020410367171	0\\
37.8520437365011	0\\
37.9020464362851	0\\
37.9520491360691	0\\
38.0020518358531	0\\
38.0520545356372	0\\
38.1020572354212	0\\
38.1520599352052	0\\
38.2020626349892	0\\
38.2520653347732	0\\
38.3020680345572	0\\
38.3520707343413	0\\
38.4020734341253	0\\
38.4520761339093	0\\
38.5020788336933	0\\
38.5520815334773	0\\
38.6020842332613	0\\
38.6520869330454	0\\
38.7020896328294	0\\
38.7520923326134	0\\
38.8020950323974	0\\
38.8520977321814	0\\
38.9021004319654	0\\
38.9521031317495	0\\
39.0021058315335	0\\
39.0521085313175	0\\
39.1021112311015	0\\
39.1521139308855	0\\
39.2021166306696	0\\
39.2521193304536	0\\
39.3021220302376	0\\
39.3521247300216	0\\
39.4021274298056	0\\
39.4521301295896	0\\
39.5021328293737	0\\
39.5521355291577	0\\
39.6021382289417	0\\
39.6521409287257	0\\
39.7021436285097	0\\
39.7521463282937	0\\
39.8021490280778	0\\
39.8521517278618	0\\
39.9021544276458	0\\
39.9521571274298	0\\
40.0021598272138	0\\
40.0521625269978	0\\
40.1021652267819	0\\
40.1521679265659	0\\
40.2021706263499	0\\
40.2521733261339	0\\
40.3021760259179	0\\
40.3521787257019	0\\
40.402181425486	0\\
40.45218412527	0\\
40.502186825054	0\\
40.552189524838	0\\
40.602192224622	0\\
40.652194924406	0\\
40.7021976241901	0\\
40.7522003239741	0\\
40.8022030237581	0\\
40.8522057235421	0\\
40.9022084233261	0\\
40.9522111231102	0\\
41.0022138228942	0\\
41.0522165226782	0\\
41.1022192224622	0\\
41.1522219222462	0\\
41.2022246220302	0\\
41.2522273218143	0\\
41.3022300215983	0\\
41.3522327213823	0\\
41.4022354211663	0\\
41.4522381209503	0\\
41.5022408207343	0\\
41.5522435205184	0\\
41.6022462203024	0\\
41.6522489200864	0\\
41.7022516198704	0\\
41.7522543196544	0\\
41.8022570194385	0\\
41.8522597192225	0\\
41.9022624190065	0\\
41.9522651187905	0\\
42.0022678185745	0\\
42.0522705183585	0\\
42.1022732181426	0\\
42.1522759179266	0\\
42.2022786177106	0\\
42.2522813174946	0\\
42.3022840172786	0\\
42.3522867170626	0\\
42.4022894168467	0\\
42.4522921166307	0\\
42.5022948164147	0\\
42.5522975161987	0\\
42.6023002159827	0\\
42.6523029157667	0\\
42.7023056155508	0\\
42.7523083153348	0\\
42.8023110151188	0\\
42.8523137149028	0\\
42.9023164146868	0\\
42.9523191144708	0\\
43.0023218142549	0\\
43.0523245140389	0\\
43.1023272138229	0\\
43.1523299136069	0\\
43.2023326133909	0\\
43.2523353131749	0\\
43.302338012959	0\\
43.352340712743	0\\
43.402343412527	0\\
43.452346112311	0\\
43.502348812095	0\\
43.5523515118791	0\\
43.6023542116631	0\\
43.6523569114471	0\\
43.7023596112311	0\\
43.7523623110151	0\\
43.8023650107991	0\\
43.8523677105832	0\\
43.9023704103672	0\\
43.9523731101512	0\\
44.0023758099352	0\\
44.0523785097192	0\\
44.1023812095032	0\\
44.1523839092873	0\\
44.2023866090713	0\\
44.2523893088553	0\\
44.3023920086393	0\\
44.3523947084233	0\\
44.4023974082073	0\\
44.4524001079914	0\\
44.5024028077754	0\\
44.5524055075594	0\\
44.6024082073434	0\\
44.6524109071274	0\\
44.7024136069114	0\\
44.7524163066955	0\\
44.8024190064795	0\\
44.8524217062635	0\\
44.9024244060475	0\\
44.9524271058315	0\\
45.0024298056155	0\\
45.0524325053996	0\\
45.1024352051836	0\\
45.1524379049676	0\\
45.2024406047516	0\\
45.2524433045356	0\\
45.3024460043197	0\\
45.3524487041037	0\\
45.4024514038877	0\\
45.4524541036717	0\\
45.5024568034557	0\\
45.5524595032397	0\\
45.6024622030238	0\\
45.6524649028078	0\\
45.7024676025918	0\\
45.7524703023758	0\\
45.8024730021598	0\\
45.8524757019439	0\\
45.9024784017279	0\\
45.9524811015119	0\\
46.0024838012959	0\\
46.0524865010799	0\\
46.1024892008639	0\\
46.152491900648	0\\
46.202494600432	0\\
46.252497300216	0\\
46.3025	0\\
46.352502699784	0\\
46.402505399568	0\\
46.4525080993521	0\\
46.5025107991361	0\\
46.5525134989201	0\\
46.6025161987041	0\\
46.6525188984881	0\\
46.7025215982721	0\\
46.7525242980562	0\\
46.8025269978402	0\\
46.8525296976242	0\\
46.9025323974082	0\\
46.9525350971922	0\\
47.0025377969762	0\\
47.0525404967603	0\\
47.1025431965443	0\\
47.1525458963283	0\\
47.2025485961123	0\\
47.2525512958963	0\\
47.3025539956803	0\\
47.3525566954644	0\\
47.4025593952484	0\\
47.4525620950324	0\\
47.5025647948164	0\\
47.5525674946004	0\\
47.6025701943845	0\\
47.6525728941685	0\\
47.7025755939525	0\\
47.7525782937365	0\\
47.8025809935205	0\\
47.8525836933045	0\\
47.9025863930886	0\\
47.9525890928726	0\\
48.0025917926566	0\\
48.0525944924406	0\\
48.1025971922246	0\\
48.1525998920086	0\\
48.2026025917927	0\\
48.2526052915767	0\\
48.3026079913607	0\\
48.3526106911447	0\\
48.4026133909287	0\\
48.4526160907127	0\\
48.5026187904968	0\\
48.5526214902808	0\\
48.6026241900648	0\\
48.6526268898488	0\\
48.7026295896328	0\\
48.7526322894168	0\\
48.8026349892009	0\\
48.8526376889849	0\\
48.9026403887689	0\\
48.9526430885529	0\\
49.0026457883369	0\\
49.052648488121	0\\
49.102651187905	0\\
49.152653887689	0\\
49.202656587473	0\\
49.252659287257	0\\
49.302661987041	0\\
49.3526646868251	0\\
49.4026673866091	0\\
49.4526700863931	0\\
49.5026727861771	0\\
49.5526754859611	0\\
49.6026781857451	0\\
49.6526808855292	0\\
49.7026835853132	0\\
49.7526862850972	0\\
49.8026889848812	0\\
49.8526916846652	0\\
49.9026943844492	0\\
49.9526970842333	0\\
50.0026997840173	0\\
50.0527024838013	0\\
50.1027051835853	0\\
50.1527078833693	0\\
50.2027105831534	0\\
50.2527132829374	0\\
50.3027159827214	0\\
50.3527186825054	0\\
50.4027213822894	0\\
50.4527240820734	0\\
50.5027267818575	0\\
50.5527294816415	0\\
50.6027321814255	0\\
50.6527348812095	0\\
50.7027375809935	0\\
50.7527402807775	0\\
50.8027429805616	0\\
50.8527456803456	0\\
50.9027483801296	0\\
50.9527510799136	0\\
51.0027537796976	0\\
51.0527564794816	0\\
51.1027591792657	0\\
51.1527618790497	0\\
51.2027645788337	0\\
51.2527672786177	0\\
51.3027699784017	0\\
51.3527726781858	0\\
51.4027753779698	0\\
51.4527780777538	0\\
51.5027807775378	0\\
51.5527834773218	0\\
51.6027861771058	0\\
51.6527888768899	0\\
51.7027915766739	0\\
51.7527942764579	0\\
51.8027969762419	0\\
51.8527996760259	0\\
51.9028023758099	0\\
51.952805075594	0\\
52.002807775378	0\\
52.052810475162	0\\
52.102813174946	0\\
52.15281587473	0\\
52.202818574514	0\\
52.2528212742981	0\\
52.3028239740821	0\\
52.3528266738661	0\\
52.4028293736501	0\\
52.4528320734341	0\\
52.5028347732181	0\\
52.5528374730022	0\\
52.6028401727862	0\\
52.6528428725702	0\\
52.7028455723542	0\\
52.7528482721382	0\\
52.8028509719222	0\\
52.8528536717063	0\\
52.9028563714903	0\\
52.9528590712743	0\\
53.0028617710583	0\\
53.0528644708423	0\\
53.1028671706264	0\\
53.1528698704104	0\\
53.2028725701944	0\\
53.2528752699784	0\\
53.3028779697624	0\\
53.3528806695464	0\\
53.4028833693305	0\\
53.4528860691145	0\\
53.5028887688985	0\\
53.5528914686825	0\\
53.6028941684665	0\\
53.6528968682505	0\\
53.7028995680346	0\\
53.7529022678186	0\\
53.8029049676026	0\\
53.8529076673866	0\\
53.9029103671706	0\\
53.9529130669547	0\\
54.0029157667387	0\\
54.0529184665227	0\\
54.1029211663067	0\\
54.1529238660907	0\\
54.2029265658747	0\\
54.2529292656588	0\\
54.3029319654428	0\\
54.3529346652268	0\\
54.4029373650108	0\\
54.4529400647948	0\\
54.5029427645788	0\\
54.5529454643629	0\\
54.6029481641469	0\\
54.6529508639309	0\\
54.7029535637149	0\\
54.7529562634989	0\\
54.8029589632829	0\\
54.852961663067	0\\
54.902964362851	0\\
54.952967062635	0\\
55.002969762419	0\\
55.052972462203	0\\
55.102975161987	0\\
55.1529778617711	0\\
55.2029805615551	0\\
55.2529832613391	0\\
55.3029859611231	0\\
55.3529886609071	0\\
55.4029913606911	0\\
55.4529940604752	0\\
55.5029967602592	0\\
55.5529994600432	0\\
55.6030021598272	0\\
55.6530048596112	0\\
55.7030075593953	0\\
55.7530102591793	0\\
55.8030129589633	0\\
55.8530156587473	0\\
55.9030183585313	0\\
55.9530210583153	0\\
56.0030237580994	0\\
56.0530264578834	0\\
56.1030291576674	0\\
56.1530318574514	0\\
56.2030345572354	0\\
56.2530372570194	0\\
56.3030399568035	0\\
56.3530426565875	0\\
56.4030453563715	0\\
56.4530480561555	0\\
56.5030507559395	0\\
56.5530534557235	0\\
56.6030561555076	0\\
56.6530588552916	0\\
56.7030615550756	0\\
56.7530642548596	0\\
56.8030669546436	0\\
56.8530696544276	0\\
56.9030723542117	0\\
56.9530750539957	0\\
57.0030777537797	0\\
57.0530804535637	0\\
57.1030831533477	0\\
57.1530858531318	0\\
57.2030885529158	0\\
57.2530912526998	0\\
57.3030939524838	0\\
57.3530966522678	0\\
57.4030993520518	0\\
57.4531020518358	0\\
57.5031047516199	0\\
57.5531074514039	0\\
57.6031101511879	0\\
57.6531128509719	0\\
57.7031155507559	0\\
57.75311825054	0\\
57.803120950324	0\\
57.853123650108	0\\
57.903126349892	0\\
57.953129049676	0\\
58.00313174946	0\\
58.0531344492441	0\\
58.1031371490281	0\\
58.1531398488121	0\\
58.2031425485961	0\\
58.2531452483801	0\\
58.3031479481642	0\\
58.3531506479482	0\\
58.4031533477322	0\\
58.4531560475162	0\\
58.5031587473002	0\\
58.5531614470842	0\\
58.6031641468683	0\\
58.6531668466523	0\\
58.7031695464363	0\\
58.7531722462203	0\\
58.8031749460043	0\\
58.8531776457883	0\\
58.9031803455724	0\\
58.9531830453564	0\\
59.0031857451404	0\\
59.0531884449244	0\\
59.1031911447084	0\\
59.1531938444924	0\\
59.2031965442765	0\\
59.2531992440605	0\\
59.3032019438445	0\\
59.3532046436285	0\\
59.4032073434125	0\\
59.4532100431965	0\\
59.5032127429806	0\\
59.5532154427646	0\\
59.6032181425486	0\\
59.6532208423326	0\\
59.7032235421166	0\\
59.7532262419006	0\\
59.8032289416847	0\\
59.8532316414687	0\\
59.9032343412527	0\\
59.9532370410367	0\\
60.0032397408207	0\\
60.0532424406048	0\\
60.1032451403888	0\\
60.1532478401728	0\\
60.2032505399568	0\\
60.2532532397408	0\\
60.3032559395248	0\\
60.3532586393089	0\\
60.4032613390929	0\\
60.4532640388769	0\\
60.5032667386609	0\\
60.5532694384449	0\\
60.6032721382289	0\\
60.653274838013	0\\
60.703277537797	0\\
60.753280237581	0\\
60.803282937365	0\\
60.853285637149	0\\
60.9032883369331	0\\
60.9532910367171	0\\
61.0032937365011	0\\
61.0532964362851	0\\
61.1032991360691	0\\
61.1533018358531	0\\
61.2033045356371	0\\
61.2533072354212	0\\
61.3033099352052	0\\
61.3533126349892	0\\
61.4033153347732	0\\
61.4533180345572	0\\
61.5033207343413	0\\
61.5533234341253	0\\
61.6033261339093	0\\
61.6533288336933	0\\
61.7033315334773	0\\
61.7533342332613	0\\
61.8033369330454	0\\
61.8533396328294	0\\
61.9033423326134	0\\
61.9533450323974	0\\
62.0033477321814	0\\
62.0533504319654	0\\
62.1033531317495	0\\
62.1533558315335	0\\
62.2033585313175	0\\
62.2533612311015	0\\
62.3033639308855	0\\
62.3533666306695	0\\
62.4033693304536	0\\
62.4533720302376	0\\
62.5033747300216	0\\
62.5483771598272	0.2\\
62.5983798596112	0.2\\
62.6483825593953	0.2\\
62.6983852591793	0.2\\
62.7483879589633	0.2\\
62.7983906587473	0.2\\
62.8483933585313	0.2\\
62.8983960583153	0.2\\
62.9483987580994	0.2\\
62.9984014578834	0.2\\
63.0484041576674	0.2\\
63.0984068574514	0.2\\
63.1484095572354	0.2\\
63.1984122570194	0.2\\
63.2484149568035	0.2\\
63.2984176565875	0.2\\
63.3484203563715	0.2\\
63.3984230561555	0.2\\
63.4484257559395	0.2\\
63.4984284557235	0.2\\
63.5484311555076	0.2\\
63.5984338552916	0.2\\
63.6484365550756	0.2\\
63.6984392548596	0.2\\
63.7484419546436	0.2\\
63.7984446544276	0.2\\
63.8484473542117	0.2\\
63.8984500539957	0.2\\
63.9484527537797	0.2\\
63.9984554535637	0.2\\
64.0484581533477	0.2\\
64.0984608531318	0.2\\
64.1484635529158	0.2\\
64.1984662526998	0.2\\
64.2484689524838	0.2\\
64.2984716522678	0.2\\
64.3484743520518	0.2\\
64.3984770518359	0.2\\
64.4484797516199	0.2\\
64.4984824514039	0.2\\
64.5484851511879	0.2\\
64.5984878509719	0.2\\
64.648490550756	0.2\\
64.69849325054	0.2\\
64.748495950324	0.2\\
64.798498650108	0.2\\
64.848501349892	0.2\\
64.898504049676	0.2\\
64.94850674946	0.2\\
64.9985094492441	0.2\\
65.0485121490281	0.2\\
65.0985148488121	0.2\\
65.1485175485961	0.2\\
65.1985202483801	0.2\\
65.2485229481642	0.2\\
65.2985256479482	0.2\\
65.3485283477322	0.2\\
65.3985310475162	0.2\\
65.4485337473002	0.2\\
65.4985364470842	0.2\\
65.5485391468683	0.2\\
65.5985418466523	0.2\\
65.6485445464363	0.2\\
65.6985472462203	0.2\\
65.7485499460043	0.2\\
65.7985526457883	0.2\\
65.8485553455724	0.2\\
65.8985580453564	0.2\\
65.9485607451404	0.2\\
65.9985634449244	0.2\\
66.0485661447084	0.2\\
66.0985688444924	0.2\\
66.1485715442765	0.2\\
66.1985742440605	0.2\\
66.2485769438445	0.2\\
66.2985796436285	0.2\\
66.3485823434125	0.2\\
66.3985850431965	0.2\\
66.4485877429806	0.2\\
66.4985904427646	0.2\\
66.5485931425486	0.2\\
66.5985958423326	0.2\\
66.6485985421166	0.2\\
66.6986012419006	0.2\\
66.7486039416847	0.2\\
66.7986066414687	0.2\\
66.8486093412527	0.2\\
66.8986120410367	0.2\\
66.9486147408207	0.2\\
66.9986174406047	0.2\\
67.0486201403888	0.2\\
67.0986228401728	0.2\\
67.1486255399568	0.2\\
67.1986282397408	0.2\\
67.2486309395248	0.2\\
67.2986336393089	0.2\\
67.3486363390929	0.2\\
67.3986390388769	0.2\\
67.4486417386609	0.2\\
67.4986444384449	0.2\\
67.5486471382289	0.2\\
67.598649838013	0.2\\
67.648652537797	0.2\\
67.698655237581	0.2\\
67.748657937365	0.2\\
67.798660637149	0.2\\
67.8486633369331	0.2\\
67.8986660367171	0.2\\
67.9486687365011	0.2\\
67.9986714362851	0.2\\
68.0486741360691	0.2\\
68.0986768358531	0.2\\
68.1486795356372	0.2\\
68.1986822354212	0.2\\
68.2486849352052	0.2\\
68.2986876349892	0.2\\
68.3486903347732	0.2\\
68.3986930345572	0.2\\
68.4486957343413	0.2\\
68.4986984341253	0.2\\
68.5487011339093	0.2\\
68.5987038336933	0.2\\
68.6487065334773	0.2\\
68.6987092332613	0.2\\
68.7487119330454	0.2\\
68.7987146328294	0.2\\
68.8487173326134	0.2\\
68.8987200323974	0.2\\
68.9487227321814	0.2\\
68.9987254319654	0.2\\
69.0487281317495	0.2\\
69.0987308315335	0.2\\
69.1487335313175	0.2\\
69.1987362311015	0.2\\
69.2487389308855	0.2\\
69.2987416306696	0.2\\
69.3487443304536	0.2\\
69.3987470302376	0.2\\
69.4487497300216	0.2\\
69.4987524298056	0.2\\
69.5487551295896	0.2\\
69.5987578293737	0.2\\
69.6487605291577	0.2\\
69.6987632289417	0.2\\
69.7487659287257	0.2\\
69.7987686285097	0.2\\
69.8487713282937	0.2\\
69.8987740280778	0.2\\
69.9487767278618	0.2\\
69.9987794276458	0.2\\
70.0487821274298	0.2\\
70.0987848272138	0.2\\
70.1487875269979	0.2\\
70.1987902267819	0.2\\
70.2487929265659	0.2\\
70.2987956263499	0.2\\
70.3487983261339	0.2\\
70.3988010259179	0.2\\
70.4488037257019	0.2\\
70.498806425486	0.2\\
70.54880912527	0.2\\
70.598811825054	0.2\\
70.648814524838	0.2\\
70.698817224622	0.2\\
70.7488199244061	0.2\\
70.7988226241901	0.2\\
70.8488253239741	0.2\\
70.8988280237581	0.2\\
70.9488307235421	0.2\\
70.9988334233261	0.2\\
71.0488361231101	0.2\\
71.0988388228942	0.2\\
71.1488415226782	0.2\\
71.1988442224622	0.2\\
71.2488469222462	0.2\\
71.2988496220302	0.2\\
71.3488523218143	0.2\\
71.3988550215983	0.2\\
71.4488577213823	0.2\\
71.4988604211663	0.2\\
71.5488631209503	0.2\\
71.5988658207343	0.2\\
71.6488685205184	0.2\\
71.6988712203024	0.2\\
71.7488739200864	0.2\\
71.7988766198704	0.2\\
71.8488793196544	0.2\\
71.8988820194385	0.2\\
71.9488847192225	0.2\\
71.9988874190065	0.2\\
72.0488901187905	0.2\\
72.0988928185745	0.2\\
72.1488955183585	0.2\\
72.1988982181425	0.2\\
72.2489009179266	0.2\\
72.2989036177106	0.2\\
72.3489063174946	0.2\\
72.3989090172786	0.2\\
72.4489117170626	0.2\\
72.4989144168467	0.2\\
72.5489171166307	0.2\\
72.5989198164147	0.2\\
72.6489225161987	0.2\\
72.6989252159827	0.2\\
72.7489279157667	0.2\\
72.7989306155508	0.2\\
72.8489333153348	0.2\\
72.8989360151188	0.2\\
72.9489387149028	0.2\\
72.9989414146868	0.2\\
73.0489441144708	0.2\\
73.0989468142549	0.2\\
73.1489495140389	0.2\\
73.1989522138229	0.2\\
73.2489549136069	0.2\\
73.2989576133909	0.2\\
73.348960313175	0.2\\
73.398963012959	0.2\\
73.448965712743	0.2\\
73.498968412527	0.2\\
73.548971112311	0.2\\
73.598973812095	0.2\\
73.6489765118791	0.2\\
73.6989792116631	0.2\\
73.7489819114471	0.2\\
73.7989846112311	0.2\\
73.8489873110151	0.2\\
73.8989900107991	0.2\\
73.9489927105832	0.2\\
73.9989954103672	0.2\\
74.0489981101512	0.2\\
74.0990008099352	0.2\\
74.1490035097192	0.2\\
74.1990062095032	0.2\\
74.2490089092873	0.2\\
74.2990116090713	0.2\\
74.3490143088553	0.2\\
74.3990170086393	0.2\\
74.4490197084233	0.2\\
74.4990224082073	0.2\\
74.5490251079914	0.2\\
74.5990278077754	0.2\\
74.6490305075594	0.2\\
74.6990332073434	0.2\\
74.7490359071274	0.2\\
74.7990386069115	0.2\\
74.8490413066955	0.2\\
74.8990440064795	0.2\\
74.9490467062635	0.2\\
74.9990494060475	0.2\\
75.0490521058315	0.2\\
75.0990548056156	0.2\\
75.1490575053996	0.2\\
75.1990602051836	0.2\\
75.2490629049676	0.2\\
75.2990656047516	0.2\\
75.3490683045356	0.2\\
75.3990710043197	0.2\\
75.4490737041037	0.2\\
75.4990764038877	0.2\\
75.5490791036717	0.2\\
75.5990818034557	0.2\\
75.6490845032397	0.2\\
75.6990872030238	0.2\\
75.7490899028078	0.2\\
75.7990926025918	0.2\\
75.8490953023758	0.2\\
75.8990980021598	0.2\\
75.9491007019438	0.2\\
75.9991034017279	0.2\\
76.0491061015119	0.2\\
76.0991088012959	0.2\\
76.1491115010799	0.2\\
76.1991142008639	0.2\\
76.2491169006479	0.2\\
76.299119600432	0.2\\
76.349122300216	0.2\\
76.399125	0.2\\
76.449127699784	0.2\\
76.499130399568	0.2\\
76.5491330993521	0.2\\
76.5991357991361	0.2\\
76.6491384989201	0.2\\
76.6991411987041	0.2\\
76.7491438984881	0.2\\
76.7991465982721	0.2\\
76.8491492980562	0.2\\
76.8991519978402	0.2\\
76.9491546976242	0.2\\
76.9991573974082	0.2\\
77.0491600971922	0.2\\
77.0991627969762	0.2\\
77.1491654967603	0.2\\
77.1991681965443	0.2\\
77.2491708963283	0.2\\
77.2991735961123	0.2\\
77.3491762958963	0.2\\
77.3991789956804	0.2\\
77.4491816954644	0.2\\
77.4991843952484	0.2\\
77.5491870950324	0.2\\
77.5991897948164	0.2\\
77.6491924946004	0.2\\
77.6991951943844	0.2\\
77.7491978941685	0.2\\
77.7992005939525	0.2\\
77.8492032937365	0.2\\
77.8992059935205	0.2\\
77.9492086933045	0.2\\
77.9992113930886	0.2\\
78.0492140928726	0.2\\
78.0992167926566	0.2\\
78.1492194924406	0.2\\
78.1992221922246	0.2\\
78.2492248920086	0.2\\
78.2992275917927	0.2\\
78.3492302915767	0.2\\
78.3992329913607	0.2\\
78.4492356911447	0.2\\
78.4992383909287	0.2\\
78.5492410907127	0.2\\
78.5992437904968	0.2\\
78.6492464902808	0.2\\
78.6992491900648	0.2\\
78.7492518898488	0.2\\
78.7992545896328	0.2\\
78.8492572894169	0.2\\
78.8992599892009	0.2\\
78.9492626889849	0.2\\
78.9992653887689	0.2\\
79.0492680885529	0.2\\
79.0992707883369	0.2\\
79.149273488121	0.2\\
79.199276187905	0.2\\
79.249278887689	0.2\\
79.299281587473	0.2\\
79.349284287257	0.2\\
79.3992869870411	0.2\\
79.4492896868251	0.2\\
79.4992923866091	0.2\\
79.5492950863931	0.2\\
79.5992977861771	0.2\\
79.6493004859611	0.2\\
79.6993031857451	0.2\\
79.7493058855292	0.2\\
79.7993085853132	0.2\\
79.8493112850972	0.2\\
79.8993139848812	0.2\\
79.9493166846652	0.2\\
79.9993193844493	0.2\\
80.0493220842333	0.2\\
80.0993247840173	0.2\\
80.1493274838013	0.2\\
80.1993301835853	0.2\\
80.2493328833693	0.2\\
80.2993355831534	0.2\\
80.3493382829374	0.2\\
80.3993409827214	0.2\\
80.4493436825054	0.2\\
80.4993463822894	0.2\\
80.5493490820734	0.2\\
80.5993517818575	0.2\\
80.6493544816415	0.2\\
80.6993571814255	0.2\\
80.7493598812095	0.2\\
80.7993625809935	0.2\\
80.8493652807775	0.2\\
80.8993679805616	0.2\\
80.9493706803456	0.2\\
80.9993733801296	0.2\\
81.0493760799136	0.2\\
81.0993787796976	0.2\\
81.1493814794816	0.2\\
81.1993841792657	0.2\\
81.2493868790497	0.2\\
81.2993895788337	0.2\\
81.3493922786177	0.2\\
81.3993949784017	0.2\\
81.4493976781857	0.2\\
81.4994003779698	0.2\\
81.5494030777538	0.2\\
81.5994057775378	0.2\\
81.6494084773218	0.2\\
81.6994111771058	0.2\\
81.7494138768898	0.2\\
81.7994165766739	0.2\\
81.8494192764579	0.2\\
81.8994219762419	0.2\\
81.9494246760259	0.2\\
81.9994273758099	0.2\\
82.049430075594	0.2\\
82.099432775378	0.2\\
82.149435475162	0.2\\
82.199438174946	0.2\\
82.24944087473	0.2\\
82.299443574514	0.2\\
82.3494462742981	0.2\\
82.3994489740821	0.2\\
82.4494516738661	0.2\\
82.4994543736501	0.2\\
82.5494570734341	0.2\\
82.5994597732181	0.2\\
82.6294613930885	0\\
82.6794640928726	0\\
82.7294667926566	0\\
82.7794694924406	0\\
82.8294721922246	0\\
82.8794748920086	0\\
82.9294775917927	0\\
82.9794802915767	0\\
83.0294829913607	0\\
83.0794856911447	0\\
83.1294883909287	0\\
83.1794910907127	0\\
83.2294937904968	0\\
83.2794964902808	0\\
83.3294991900648	0\\
83.3795018898488	0\\
83.4295045896328	0\\
83.4795072894169	0\\
83.5295099892009	0\\
83.5795126889849	0\\
83.6295153887689	0\\
83.6795180885529	0\\
83.7295207883369	0\\
83.779523488121	0\\
83.829526187905	0\\
83.879528887689	0\\
83.929531587473	0\\
83.979534287257	0\\
84.029536987041	0\\
84.0795396868251	0\\
84.1295423866091	0\\
84.1795450863931	0\\
84.2295477861771	0\\
84.2795504859611	0\\
84.3295531857451	0\\
84.3795558855292	0\\
84.4295585853132	0\\
84.4795612850972	0\\
84.5295639848812	0\\
84.5795666846652	0\\
84.6295693844492	0\\
84.6795720842333	0\\
84.7295747840173	0\\
84.7795774838013	0\\
84.8295801835853	0\\
84.8795828833693	0\\
84.9295855831534	0\\
84.9795882829374	0\\
85.0295909827214	0\\
85.0795936825054	0\\
85.1295963822894	0\\
85.1795990820734	0\\
85.2296017818575	0\\
85.2796044816415	0\\
85.3296071814255	0\\
85.3796098812095	0\\
85.4296125809935	0\\
85.4796152807775	0\\
85.5296179805616	0\\
85.5796206803456	0\\
85.6296233801296	0\\
85.6796260799136	0\\
85.7296287796976	0\\
85.7796314794817	0\\
85.8296341792657	0\\
85.8796368790497	0\\
85.9296395788337	0\\
85.9796422786177	0\\
86.0296449784017	0\\
86.0796476781857	0\\
86.1296503779698	0\\
86.1796530777538	0\\
86.2296557775378	0\\
86.2796584773218	0\\
86.3296611771058	0\\
86.3796638768899	0\\
86.4296665766739	0\\
86.4796692764579	0\\
86.5296719762419	0\\
86.5796746760259	0\\
86.6296773758099	0\\
86.679680075594	0\\
86.729682775378	0\\
86.779685475162	0\\
86.829688174946	0\\
86.87969087473	0\\
86.929693574514	0\\
86.9796962742981	0\\
87.0296989740821	0\\
87.0797016738661	0\\
87.1297043736501	0\\
87.1797070734341	0\\
87.2297097732181	0\\
87.2797124730022	0\\
87.3297151727862	0\\
87.3797178725702	0\\
87.4297205723542	0\\
87.4797232721382	0\\
87.5297259719222	0\\
87.5797286717063	0\\
87.6297313714903	0\\
87.6797340712743	0\\
87.7297367710583	0\\
87.7797394708423	0\\
87.8297421706264	0\\
87.8797448704104	0\\
87.9297475701944	0\\
87.9797502699784	0\\
88.0297529697624	0\\
88.0797556695464	0\\
88.1297583693305	0\\
88.1797610691145	0\\
88.2297637688985	0\\
88.2797664686825	0\\
88.3297691684665	0\\
88.3797718682505	0\\
88.4297745680346	0\\
88.4797772678186	0\\
88.5297799676026	0\\
88.5797826673866	0\\
88.6297853671706	0\\
88.6797880669546	0\\
88.7297907667387	0\\
88.7797934665227	0\\
88.8297961663067	0\\
88.8797988660907	0\\
88.9298015658747	0\\
88.9798042656588	0\\
89.0298069654428	0\\
89.0798096652268	0\\
89.1298123650108	0\\
89.1798150647948	0\\
89.2298177645788	0\\
89.2798204643629	0\\
89.3298231641469	0\\
89.3798258639309	0\\
89.4298285637149	0\\
89.4798312634989	0\\
89.5298339632829	0\\
89.579836663067	0\\
89.629839362851	0\\
89.679842062635	0\\
89.729844762419	0\\
89.779847462203	0\\
89.8298501619871	0\\
89.8798528617711	0\\
89.9298555615551	0\\
89.9798582613391	0\\
90.0298609611231	0\\
90.0798636609071	0\\
90.1298663606911	0\\
90.1798690604752	0\\
90.2298717602592	0\\
90.2798744600432	0\\
90.3298771598272	0\\
90.3798798596112	0\\
90.4298825593953	0\\
90.4798852591793	0\\
90.5298879589633	0\\
90.5798906587473	0\\
90.6298933585313	0\\
90.6798960583153	0\\
90.7298987580994	0\\
90.7799014578834	0\\
90.8299041576674	0\\
90.8799068574514	0\\
90.9299095572354	0\\
90.9799122570195	0\\
91.0299149568035	0\\
91.0799176565875	0\\
91.1299203563715	0\\
91.1799230561555	0\\
91.2299257559395	0\\
91.2799284557236	0\\
91.3299311555076	0\\
91.3799338552916	0\\
91.4299365550756	0\\
91.4799392548596	0\\
91.5299419546436	0\\
91.5799446544276	0\\
91.6299473542117	0\\
91.6799500539957	0\\
91.7299527537797	0\\
91.7799554535637	0\\
91.8299581533477	0\\
91.8799608531317	0\\
91.9299635529158	0\\
91.9799662526998	0\\
92.0299689524838	0\\
92.0799716522678	0\\
92.1299743520518	0\\
92.1799770518358	0\\
92.2299797516199	0\\
92.2799824514039	0\\
92.3299851511879	0\\
92.3799878509719	0\\
92.4299905507559	0\\
92.47999325054	0\\
92.529995950324	0\\
92.579998650108	0\\
92.605	0\\
};
\addlegendentry{set point};

\addplot [color=gray,solid,line width=0.2pt]
  table[row sep=crcr]{0	0.01\\
0.0500026997840173	0.01\\
0.100005399568035	0.01\\
0.150008099352052	0.01\\
0.200010799136069	0.01\\
0.250013498920086	0.01\\
0.300016198704104	0.01\\
0.350018898488121	0.01\\
0.400021598272138	0.01\\
0.450024298056156	0.01\\
0.500026997840173	0.01\\
0.55002969762419	0.01\\
0.600032397408207	0.01\\
0.650035097192225	0.01\\
0.700037796976242	0.01\\
0.750040496760259	0.01\\
0.800043196544276	0.01\\
0.850045896328294	0.01\\
0.900048596112311	0.01\\
0.950051295896328	0.01\\
1.00005399568035	0.01\\
1.05005669546436	0.01\\
1.10005939524838	0.01\\
1.1500620950324	0.01\\
1.20006479481641	0.01\\
1.25006749460043	0.01\\
1.30007019438445	0.01\\
1.35007289416847	0.01\\
1.40007559395248	0.01\\
1.4500782937365	0.01\\
1.50008099352052	0.01\\
1.55008369330454	0.01\\
1.60008639308855	0.01\\
1.65008909287257	0.01\\
1.70009179265659	0.01\\
1.7500944924406	0.01\\
1.80009719222462	0.01\\
1.85009989200864	0.01\\
1.90010259179266	0.01\\
1.95010529157667	0.01\\
2.00010799136069	0.01\\
2.05011069114471	0.01\\
2.10011339092873	0.01\\
2.15011609071274	0.01\\
2.20011879049676	0.01\\
2.25012149028078	0.01\\
2.30012419006479	0.01\\
2.35012688984881	0.01\\
2.40012958963283	0.01\\
2.45013228941685	0.01\\
2.50013498920086	0.01\\
2.55013768898488	0.01\\
2.6001403887689	0.01\\
2.65014308855292	0.01\\
2.70014578833693	0.01\\
2.75014848812095	0.01\\
2.80015118790497	0.01\\
2.85015388768899	0.01\\
2.900156587473	0.01\\
2.95015928725702	0.01\\
3.00016198704104	0.01\\
3.05016468682505	0.01\\
3.10016738660907	0.01\\
3.15017008639309	0.01\\
3.20017278617711	0.01\\
3.25017548596112	0.01\\
3.30017818574514	0.01\\
3.35018088552916	0.01\\
3.40018358531318	0.01\\
3.45018628509719	0.01\\
3.50018898488121	0.01\\
3.55019168466523	0.01\\
3.60019438444924	0.01\\
3.65019708423326	0.01\\
3.70019978401728	0.01\\
3.7502024838013	0.01\\
3.80020518358531	0.01\\
3.85020788336933	0.01\\
3.90021058315335	0.01\\
3.95021328293736	0.01\\
4.00021598272138	0.01\\
4.0502186825054	0.01\\
4.10022138228942	0.01\\
4.15022408207343	0.01\\
4.20022678185745	0.01\\
4.25022948164147	0.01\\
4.30023218142549	0.01\\
4.3502348812095	0.01\\
4.40023758099352	0.01\\
4.45024028077754	0.01\\
4.50024298056155	0.01\\
4.55024568034557	0.01\\
4.60024838012959	0.01\\
4.65025107991361	0.01\\
4.70025377969762	0.01\\
4.75025647948164	0.01\\
4.80025917926566	0.01\\
4.85026187904968	0.01\\
4.90026457883369	0.01\\
4.95026727861771	0.01\\
5.00026997840173	0.01\\
5.05027267818575	0.01\\
5.10027537796976	0.01\\
5.15027807775378	0.01\\
5.2002807775378	0.01\\
5.25028347732181	0.01\\
5.30028617710583	0.01\\
5.35028887688985	0.01\\
5.40029157667387	0.01\\
5.45029427645788	0.01\\
5.5002969762419	0.01\\
5.55029967602592	0.01\\
5.60030237580994	0.01\\
5.65030507559395	0.01\\
5.70030777537797	0.01\\
5.75031047516199	0.01\\
5.800313174946	0.01\\
5.85031587473002	0.01\\
5.90031857451404	0.01\\
5.95032127429806	0.01\\
6.00032397408207	0.01\\
6.05032667386609	0.01\\
6.10032937365011	0.01\\
6.15033207343413	0.01\\
6.20033477321814	0.01\\
6.25033747300216	0.01\\
6.30034017278618	0.01\\
6.3503428725702	0.01\\
6.40034557235421	0.01\\
6.45034827213823	0.01\\
6.50035097192225	0.01\\
6.55035367170626	0.01\\
6.60035637149028	0.01\\
6.6503590712743	0.01\\
6.70036177105832	0.01\\
6.75036447084233	0.01\\
6.80036717062635	0.01\\
6.85036987041037	0.01\\
6.90037257019439	0.01\\
6.9503752699784	0.01\\
7.00037796976242	0.01\\
7.05038066954644	0.01\\
7.10038336933045	0.01\\
7.15038606911447	0.01\\
7.20038876889849	0.01\\
7.25039146868251	0.01\\
7.30039416846652	0.01\\
7.35039686825054	0.01\\
7.40039956803456	0.01\\
7.45040226781857	0.01\\
7.50040496760259	0.01\\
7.55040766738661	0.01\\
7.60041036717063	0.01\\
7.65041306695464	0.01\\
7.70041576673866	0.01\\
7.75041846652268	0.01\\
7.8004211663067	0.01\\
7.85042386609071	0.01\\
7.90042656587473	0.01\\
7.95042926565875	0.01\\
8.00043196544276	0.01\\
8.05043466522678	0.01\\
8.1004373650108	0.01\\
8.15044006479482	0.01\\
8.20044276457883	0.01\\
8.25044546436285	0.01\\
8.30044816414687	0.01\\
8.35045086393089	0.01\\
8.4004535637149	0.01\\
8.45045626349892	0.01\\
8.50045896328294	0.01\\
8.55046166306696	0.01\\
8.60046436285097	0.01\\
8.65046706263499	0.01\\
8.70046976241901	0.01\\
8.75047246220302	0.01\\
8.80047516198704	0.01\\
8.85047786177106	0.01\\
8.90048056155508	0.01\\
8.95048326133909	0.01\\
9.00048596112311	0.01\\
9.05048866090713	0.01\\
9.10049136069114	0.01\\
9.15049406047516	0.01\\
9.20049676025918	0.01\\
9.2504994600432	0.01\\
9.30050215982721	0.01\\
9.35050485961123	0.01\\
9.40050755939525	0.01\\
9.45051025917927	0.01\\
9.50051295896328	0.01\\
9.5505156587473	0.01\\
9.60051835853132	0.01\\
9.65052105831533	0.01\\
9.70052375809935	0.01\\
9.75052645788337	0.01\\
9.80052915766739	0.01\\
9.8505318574514	0.01\\
9.90053455723542	0.01\\
9.95053725701944	0.01\\
10.0005399568035	0.01\\
10.0505426565875	0.01\\
10.1005453563715	0.01\\
10.1505480561555	0.01\\
10.2005507559395	0.01\\
10.2505534557235	0.01\\
10.3005561555076	0.01\\
10.3505588552916	0.01\\
10.4005615550756	0.01\\
10.4505642548596	0.01\\
10.5005669546436	0.01\\
10.5505696544276	0.01\\
10.6005723542117	0.01\\
10.6505750539957	0.01\\
10.7005777537797	0.01\\
10.7505804535637	0.01\\
10.8005831533477	0.01\\
10.8505858531317	0.01\\
10.9005885529158	0.01\\
10.9505912526998	0.01\\
11.0005939524838	0.01\\
11.0505966522678	0.01\\
11.1005993520518	0.01\\
11.1506020518359	0.01\\
11.2006047516199	0.01\\
11.2506074514039	0.01\\
11.3006101511879	0.01\\
11.3506128509719	0.01\\
11.4006155507559	0.01\\
11.45061825054	0.01\\
11.500620950324	0.01\\
11.550623650108	0.01\\
11.600626349892	0.01\\
11.650629049676	0.01\\
11.70063174946	0.01\\
11.7506344492441	0.01\\
11.8006371490281	0.01\\
11.8506398488121	0.01\\
11.9006425485961	0.01\\
11.9506452483801	0.01\\
12.0006479481641	0.01\\
12.0506506479482	0.01\\
12.1006533477322	0.01\\
12.1506560475162	0.01\\
12.2006587473002	0.01\\
12.2506614470842	0.01\\
12.3006641468683	0.01\\
12.3506668466523	0.01\\
12.4006695464363	0.01\\
12.4506722462203	0.01\\
12.5006749460043	0.01\\
12.5506776457883	0.01\\
12.6006803455724	0.01\\
12.6506830453564	0.01\\
12.7006857451404	0.01\\
12.7506884449244	0.01\\
12.8006911447084	0.01\\
12.8506938444924	0.01\\
12.9006965442765	0.01\\
12.9506992440605	0.01\\
13.0007019438445	0.01\\
13.0507046436285	0.01\\
13.1007073434125	0.01\\
13.1507100431965	0.01\\
13.2007127429806	0.01\\
13.2507154427646	0.01\\
13.3007181425486	0.01\\
13.3507208423326	0.01\\
13.4007235421166	0.01\\
13.4507262419006	0.01\\
13.5007289416847	0.01\\
13.5507316414687	0.01\\
13.6007343412527	0.01\\
13.6507370410367	0.01\\
13.7007397408207	0.01\\
13.7507424406048	0.01\\
13.8007451403888	0.01\\
13.8507478401728	0.01\\
13.9007505399568	0.01\\
13.9507532397408	0.01\\
14.0007559395248	0.01\\
14.0507586393089	0.01\\
14.1007613390929	0.01\\
14.1507640388769	0.01\\
14.2007667386609	0.01\\
14.2507694384449	0.01\\
14.3007721382289	0.01\\
14.350774838013	0.01\\
14.400777537797	0.01\\
14.450780237581	0.01\\
14.500782937365	0.01\\
14.550785637149	0.01\\
14.600788336933	0.01\\
14.6507910367171	0.01\\
14.7007937365011	0.01\\
14.7507964362851	0.01\\
14.8007991360691	0.01\\
14.8508018358531	0.01\\
14.9008045356372	0.01\\
14.9508072354212	0.01\\
15.0008099352052	0.01\\
15.0508126349892	0.01\\
15.1008153347732	0.01\\
15.1508180345572	0.01\\
15.2008207343413	0.01\\
15.2508234341253	0.01\\
15.3008261339093	0.01\\
15.3508288336933	0.01\\
15.4008315334773	0.01\\
15.4508342332613	0.01\\
15.5008369330454	0.01\\
15.5508396328294	0.01\\
15.6008423326134	0.01\\
15.6508450323974	0.01\\
15.7008477321814	0.01\\
15.7508504319654	0.01\\
15.8008531317495	0.01\\
15.8508558315335	0.01\\
15.9008585313175	0.01\\
15.9508612311015	0.01\\
16.0008639308855	0.01\\
16.0508666306695	0.01\\
16.1008693304536	0.01\\
16.1508720302376	0.01\\
16.2008747300216	0.01\\
16.2508774298056	0.01\\
16.3008801295896	0.01\\
16.3508828293737	0.01\\
16.4008855291577	0.01\\
16.4508882289417	0.01\\
16.5008909287257	0.01\\
16.5508936285097	0.01\\
16.6008963282937	0.01\\
16.6508990280778	0.01\\
16.7009017278618	0.01\\
16.7509044276458	0.01\\
16.8009071274298	0.01\\
16.8509098272138	0.01\\
16.9009125269978	0.01\\
16.9509152267819	0.01\\
17.0009179265659	0.01\\
17.0509206263499	0.01\\
17.1009233261339	0.01\\
17.1509260259179	0.01\\
17.2009287257019	0.01\\
17.250931425486	0.01\\
17.30093412527	0.01\\
17.350936825054	0.01\\
17.400939524838	0.01\\
17.450942224622	0.01\\
17.500944924406	0.01\\
17.5509476241901	0.01\\
17.6009503239741	0.01\\
17.6509530237581	0.01\\
17.7009557235421	0.01\\
17.7509584233261	0.01\\
17.8009611231102	0.01\\
17.8509638228942	0.01\\
17.9009665226782	0.01\\
17.9509692224622	0.01\\
18.0009719222462	0.01\\
18.0509746220302	0.01\\
18.1009773218143	0.01\\
18.1509800215983	0.01\\
18.2009827213823	0.01\\
18.2509854211663	0.01\\
18.3009881209503	0.01\\
18.3509908207343	0.01\\
18.4009935205184	0.01\\
18.4509962203024	0.01\\
18.5009989200864	0.01\\
18.5510016198704	0.01\\
18.6010043196544	0.01\\
18.6510070194384	0.01\\
18.7010097192225	0.01\\
18.7510124190065	0.01\\
18.8010151187905	0.01\\
18.8510178185745	0.01\\
18.9010205183585	0.01\\
18.9510232181425	0.01\\
19.0010259179266	0.01\\
19.0510286177106	0.01\\
19.1010313174946	0.01\\
19.1510340172786	0.01\\
19.2010367170626	0.01\\
19.2510394168467	0.01\\
19.3010421166307	0.01\\
19.3510448164147	0.01\\
19.4010475161987	0.01\\
19.4510502159827	0.01\\
19.5010529157667	0.01\\
19.5510556155508	0.01\\
19.6010583153348	0.01\\
19.6510610151188	0.01\\
19.7010637149028	0.01\\
19.7510664146868	0.01\\
19.8010691144708	0.01\\
19.8510718142549	0.01\\
19.9010745140389	0.01\\
19.9510772138229	0.01\\
20.0010799136069	0.01\\
20.0510826133909	0.01\\
20.1010853131749	0.01\\
20.151088012959	0.01\\
20.201090712743	0.01\\
20.251093412527	0.01\\
20.301096112311	0.01\\
20.351098812095	0.01\\
20.4011015118791	0.01\\
20.4511042116631	0.01\\
20.5011069114471	0.01\\
20.5511096112311	0.01\\
20.6011123110151	0.01\\
20.6511150107991	0.01\\
20.7011177105832	0.01\\
20.7511204103672	0.01\\
20.8011231101512	0.01\\
20.8511258099352	0.01\\
20.9011285097192	0.01\\
20.9511312095032	0.01\\
21.0011339092873	0.01\\
21.0511366090713	0.01\\
21.1011393088553	0.01\\
21.1511420086393	0.01\\
21.2011447084233	0.01\\
21.2511474082073	0.01\\
21.3011501079914	0.01\\
21.3511528077754	0.01\\
21.4011555075594	0.01\\
21.4511582073434	0.01\\
21.5011609071274	0.01\\
21.5511636069114	0.01\\
21.6011663066955	0.01\\
21.6511690064795	0.01\\
21.7011717062635	0.01\\
21.7511744060475	0.01\\
21.8011771058315	0.01\\
21.8511798056156	0.01\\
21.9011825053996	0.01\\
21.9511852051836	0.01\\
22.0011879049676	0.01\\
22.0511906047516	0.01\\
22.1011933045356	0.01\\
22.1511960043197	0.01\\
22.2011987041037	0.01\\
22.2512014038877	0.01\\
22.3012041036717	0.01\\
22.3512068034557	0.01\\
22.4012095032397	0.01\\
22.4512122030238	0.01\\
22.5012149028078	0.01\\
22.5512176025918	0.01\\
22.6012203023758	0.01\\
22.6512230021598	0.01\\
22.7012257019438	0.01\\
22.7512284017279	0.01\\
22.8012311015119	0.01\\
22.8512338012959	0.01\\
22.9012365010799	0.01\\
22.9512392008639	0.01\\
23.0012419006479	0.01\\
23.051244600432	0.01\\
23.101247300216	0.01\\
23.15125	0.01\\
23.201252699784	0.01\\
23.251255399568	0.01\\
23.3012580993521	0.01\\
23.3512607991361	0.01\\
23.4012634989201	0.01\\
23.4512661987041	0.01\\
23.5012688984881	0.01\\
23.5512715982721	0.01\\
23.6012742980562	0.01\\
23.6512769978402	0.01\\
23.7012796976242	0.01\\
23.7512823974082	0.01\\
23.8012850971922	0.01\\
23.8512877969762	0.01\\
23.9012904967603	0.01\\
23.9512931965443	0.01\\
24.0012958963283	0.01\\
24.0512985961123	0.01\\
24.1013012958963	0.01\\
24.1513039956803	0.01\\
24.2013066954644	0.01\\
24.2513093952484	0.01\\
24.3013120950324	0.01\\
24.3513147948164	0.01\\
24.4013174946004	0.01\\
24.4513201943845	0.01\\
24.5013228941685	0.01\\
24.5513255939525	0.01\\
24.6013282937365	0.01\\
24.6513309935205	0.01\\
24.7013336933045	0.01\\
24.7513363930886	0.01\\
24.8013390928726	0.01\\
24.8513417926566	0.01\\
24.9013444924406	0.01\\
24.9513471922246	0.01\\
25.0013498920086	0.01\\
25.0513525917927	0.01\\
25.1013552915767	0.01\\
25.1513579913607	0.01\\
25.2013606911447	0.01\\
25.2513633909287	0.01\\
25.3013660907127	0.01\\
25.3513687904968	0.01\\
25.4013714902808	0.01\\
25.4513741900648	0.01\\
25.5013768898488	0.01\\
25.5513795896328	0.01\\
25.6013822894168	0.01\\
25.6513849892009	0.01\\
25.7013876889849	0.01\\
25.7513903887689	0.01\\
25.8013930885529	0.01\\
25.8513957883369	0.01\\
25.901398488121	0.01\\
25.951401187905	0.01\\
26.001403887689	0.01\\
26.051406587473	0.01\\
26.101409287257	0.01\\
26.151411987041	0.01\\
26.2014146868251	0.01\\
26.2514173866091	0.01\\
26.3014200863931	0.01\\
26.3514227861771	0.01\\
26.4014254859611	0.01\\
26.4514281857451	0.01\\
26.5014308855292	0.01\\
26.5514335853132	0.01\\
26.6014362850972	0.01\\
26.6514389848812	0.01\\
26.7014416846652	0.01\\
26.7514443844492	0.01\\
26.8014470842333	0.01\\
26.8514497840173	0.01\\
26.9014524838013	0.01\\
26.9514551835853	0.01\\
27.0014578833693	0.01\\
27.0514605831534	0.01\\
27.1014632829374	0.01\\
27.1514659827214	0.01\\
27.2014686825054	0.01\\
27.2514713822894	0.01\\
27.3014740820734	0.01\\
27.3514767818575	0.01\\
27.4014794816415	0.01\\
27.4514821814255	0.01\\
27.5014848812095	0.01\\
27.5514875809935	0.01\\
27.6014902807775	0.01\\
27.6514929805616	0.01\\
27.7014956803456	0.01\\
27.7514983801296	0.01\\
27.8015010799136	0.01\\
27.8515037796976	0.01\\
27.9015064794816	0.01\\
27.9515091792657	0.01\\
28.0015118790497	0.01\\
28.0515145788337	0.01\\
28.1015172786177	0.01\\
28.1515199784017	0.01\\
28.2015226781857	0.01\\
28.2515253779698	0.01\\
28.3015280777538	0.01\\
28.3515307775378	0.01\\
28.4015334773218	0.01\\
28.4515361771058	0.01\\
28.5015388768899	0.01\\
28.5515415766739	0.01\\
28.6015442764579	0.01\\
28.6515469762419	0.01\\
28.7015496760259	0.01\\
28.7515523758099	0.01\\
28.801555075594	0.01\\
28.851557775378	0.01\\
28.901560475162	0.01\\
28.951563174946	0.01\\
29.00156587473	0.01\\
29.051568574514	0.01\\
29.1015712742981	0.01\\
29.1515739740821	0.01\\
29.2015766738661	0.01\\
29.2515793736501	0.01\\
29.3015820734341	0.01\\
29.3515847732181	0.01\\
29.4015874730022	0.01\\
29.4515901727862	0.01\\
29.5015928725702	0.01\\
29.5515955723542	0.01\\
29.6015982721382	0.01\\
29.6516009719222	0.01\\
29.7016036717063	0.01\\
29.7516063714903	0.01\\
29.8016090712743	0.01\\
29.8516117710583	0.01\\
29.9016144708423	0.01\\
29.9516171706264	0.01\\
30.0016198704104	0.01\\
30.0516225701944	0.01\\
30.1016252699784	0.01\\
30.1516279697624	0.01\\
30.2016306695464	0.01\\
30.2516333693305	0.01\\
30.3016360691145	0.01\\
30.3516387688985	0.01\\
30.4016414686825	0.01\\
30.4516441684665	0.01\\
30.5016468682505	0.01\\
30.5516495680346	0.01\\
30.6016522678186	0.01\\
30.6516549676026	0.01\\
30.7016576673866	0.01\\
30.7516603671706	0.01\\
30.8016630669546	0.01\\
30.8516657667387	0.01\\
30.9016684665227	0.01\\
30.9516711663067	0.01\\
31.0016738660907	0.01\\
31.0516765658747	0.01\\
31.1016792656587	0.01\\
31.1516819654428	0.01\\
31.2016846652268	0.01\\
31.2516873650108	0.01\\
31.3016900647948	0.01\\
31.3516927645788	0.01\\
31.4016954643629	0.01\\
31.4516981641469	0.01\\
31.5017008639309	0.01\\
31.5517035637149	0.01\\
31.6017062634989	0.01\\
31.6517089632829	0.01\\
31.701711663067	0.01\\
31.751714362851	0.01\\
31.801717062635	0.01\\
31.851719762419	0.01\\
31.901722462203	0.01\\
31.951725161987	0.01\\
32.0017278617711	0.01\\
32.0517305615551	0.01\\
32.1017332613391	0.01\\
32.1517359611231	0.01\\
32.2017386609071	0.01\\
32.2517413606911	0.01\\
32.3017440604752	0.01\\
32.3517467602592	0.01\\
32.4017494600432	0.01\\
32.4517521598272	0.01\\
32.5017548596112	0.01\\
32.5517575593952	0.01\\
32.6017602591793	0.01\\
32.6517629589633	0.01\\
32.7017656587473	0.01\\
32.7517683585313	0.01\\
32.8017710583153	0.01\\
32.8517737580993	0.01\\
32.9017764578834	0.01\\
32.9517791576674	0.01\\
33.0017818574514	0.01\\
33.0517845572354	0.01\\
33.1017872570194	0.01\\
33.1517899568035	0.01\\
33.2017926565875	0.01\\
33.2517953563715	0.01\\
33.3017980561555	0.01\\
33.3518007559395	0.01\\
33.4018034557235	0.01\\
33.4518061555076	0.01\\
33.5018088552916	0.01\\
33.5518115550756	0.01\\
33.6018142548596	0.01\\
33.6518169546436	0.01\\
33.7018196544277	0.01\\
33.7518223542117	0.01\\
33.8018250539957	0.01\\
33.8518277537797	0.01\\
33.9018304535637	0.01\\
33.9518331533477	0.01\\
34.0018358531318	0.01\\
34.0518385529158	0.01\\
34.1018412526998	0.01\\
34.1518439524838	0.01\\
34.2018466522678	0.01\\
34.2518493520518	0.01\\
34.3018520518359	0.01\\
34.3518547516199	0.01\\
34.4018574514039	0.01\\
34.4518601511879	0.01\\
34.5018628509719	0.01\\
34.5518655507559	0.01\\
34.60186825054	0.01\\
34.651870950324	0.01\\
34.701873650108	0.01\\
34.751876349892	0.01\\
34.801879049676	0.01\\
34.85188174946	0.01\\
34.9018844492441	0.01\\
34.9518871490281	0.01\\
35.0018898488121	0.01\\
35.0518925485961	0.01\\
35.1018952483801	0.01\\
35.1518979481641	0.01\\
35.2019006479482	0.01\\
35.2519033477322	0.01\\
35.3019060475162	0.01\\
35.3519087473002	0.01\\
35.4019114470842	0.01\\
35.4519141468683	0.01\\
35.5019168466523	0.01\\
35.5519195464363	0.01\\
35.6019222462203	0.01\\
35.6519249460043	0.01\\
35.7019276457883	0.01\\
35.7519303455724	0.01\\
35.8019330453564	0.01\\
35.8519357451404	0.01\\
35.9019384449244	0.01\\
35.9519411447084	0.01\\
36.0019438444924	0.01\\
36.0519465442765	0.01\\
36.1019492440605	0.01\\
36.1519519438445	0.01\\
36.2019546436285	0.01\\
36.2519573434125	0.01\\
36.3019600431965	0.01\\
36.3519627429806	0.01\\
36.4019654427646	0.01\\
36.4519681425486	0.01\\
36.5019708423326	0.01\\
36.5519735421166	0.01\\
36.6019762419006	0.01\\
36.6519789416847	0.01\\
36.7019816414687	0.01\\
36.7519843412527	0.01\\
36.8019870410367	0.01\\
36.8519897408207	0.01\\
36.9019924406048	0.01\\
36.9519951403888	0.01\\
37.0019978401728	0.01\\
37.0520005399568	0.01\\
37.1020032397408	0.01\\
37.1520059395248	0.01\\
37.2020086393089	0.01\\
37.2520113390929	0.01\\
37.3020140388769	0.01\\
37.3520167386609	0.01\\
37.4020194384449	0.01\\
37.4520221382289	0.01\\
37.502024838013	0.01\\
37.552027537797	0.01\\
37.602030237581	0.01\\
37.652032937365	0.01\\
37.702035637149	0.01\\
37.752038336933	0.01\\
37.8020410367171	0.01\\
37.8520437365011	0.01\\
37.9020464362851	0.01\\
37.9520491360691	0.01\\
38.0020518358531	0.01\\
38.0520545356372	0.01\\
38.1020572354212	0.01\\
38.1520599352052	0.01\\
38.2020626349892	0.01\\
38.2520653347732	0.01\\
38.3020680345572	0.01\\
38.3520707343413	0.01\\
38.4020734341253	0.01\\
38.4520761339093	0.01\\
38.5020788336933	0.01\\
38.5520815334773	0.01\\
38.6020842332613	0.01\\
38.6520869330454	0.01\\
38.7020896328294	0.01\\
38.7520923326134	0.01\\
38.8020950323974	0.01\\
38.8520977321814	0.01\\
38.9021004319654	0.01\\
38.9521031317495	0.01\\
39.0021058315335	0.01\\
39.0521085313175	0.01\\
39.1021112311015	0.01\\
39.1521139308855	0.01\\
39.2021166306696	0.01\\
39.2521193304536	0.01\\
39.3021220302376	0.01\\
39.3521247300216	0.01\\
39.4021274298056	0.01\\
39.4521301295896	0.01\\
39.5021328293737	0.01\\
39.5521355291577	0.01\\
39.6021382289417	0.01\\
39.6521409287257	0.01\\
39.7021436285097	0.01\\
39.7521463282937	0.01\\
39.8021490280778	0.01\\
39.8521517278618	0.01\\
39.9021544276458	0.01\\
39.9521571274298	0.01\\
40.0021598272138	0.01\\
40.0521625269978	0.01\\
40.1021652267819	0.01\\
40.1521679265659	0.01\\
40.2021706263499	0.01\\
40.2521733261339	0.01\\
40.3021760259179	0.01\\
40.3521787257019	0.01\\
40.402181425486	0.01\\
40.45218412527	0.01\\
40.502186825054	0.01\\
40.552189524838	0.01\\
40.602192224622	0.01\\
40.652194924406	0.01\\
40.7021976241901	0.01\\
40.7522003239741	0.01\\
40.8022030237581	0.01\\
40.8522057235421	0.01\\
40.9022084233261	0.01\\
40.9522111231102	0.01\\
41.0022138228942	0.01\\
41.0522165226782	0.01\\
41.1022192224622	0.01\\
41.1522219222462	0.01\\
41.2022246220302	0.01\\
41.2522273218143	0.01\\
41.3022300215983	0.01\\
41.3522327213823	0.01\\
41.4022354211663	0.01\\
41.4522381209503	0.01\\
41.5022408207343	0.01\\
41.5522435205184	0.01\\
41.6022462203024	0.01\\
41.6522489200864	0.01\\
41.7022516198704	0.01\\
41.7522543196544	0.01\\
41.8022570194385	0.01\\
41.8522597192225	0.01\\
41.9022624190065	0.01\\
41.9522651187905	0.01\\
42.0022678185745	0.01\\
42.0522705183585	0.01\\
42.1022732181426	0.01\\
42.1522759179266	0.01\\
42.2022786177106	0.01\\
42.2522813174946	0.01\\
42.3022840172786	0.01\\
42.3522867170626	0.01\\
42.4022894168467	0.01\\
42.4522921166307	0.01\\
42.5022948164147	0.01\\
42.5522975161987	0.01\\
42.6023002159827	0.01\\
42.6523029157667	0.01\\
42.7023056155508	0.01\\
42.7523083153348	0.01\\
42.8023110151188	0.01\\
42.8523137149028	0.01\\
42.9023164146868	0.01\\
42.9523191144708	0.01\\
43.0023218142549	0.01\\
43.0523245140389	0.01\\
43.1023272138229	0.01\\
43.1523299136069	0.01\\
43.2023326133909	0.01\\
43.2523353131749	0.01\\
43.302338012959	0.01\\
43.352340712743	0.01\\
43.402343412527	0.01\\
43.452346112311	0.01\\
43.502348812095	0.01\\
43.5523515118791	0.01\\
43.6023542116631	0.01\\
43.6523569114471	0.01\\
43.7023596112311	0.01\\
43.7523623110151	0.01\\
43.8023650107991	0.01\\
43.8523677105832	0.01\\
43.9023704103672	0.01\\
43.9523731101512	0.01\\
44.0023758099352	0.01\\
44.0523785097192	0.01\\
44.1023812095032	0.01\\
44.1523839092873	0.01\\
44.2023866090713	0.01\\
44.2523893088553	0.01\\
44.3023920086393	0.01\\
44.3523947084233	0.01\\
44.4023974082073	0.01\\
44.4524001079914	0.01\\
44.5024028077754	0.01\\
44.5524055075594	0.01\\
44.6024082073434	0.01\\
44.6524109071274	0.01\\
44.7024136069114	0.01\\
44.7524163066955	0.01\\
44.8024190064795	0.01\\
44.8524217062635	0.01\\
44.9024244060475	0.01\\
44.9524271058315	0.01\\
45.0024298056155	0.01\\
45.0524325053996	0.01\\
45.1024352051836	0.01\\
45.1524379049676	0.01\\
45.2024406047516	0.01\\
45.2524433045356	0.01\\
45.3024460043197	0.01\\
45.3524487041037	0.01\\
45.4024514038877	0.01\\
45.4524541036717	0.01\\
45.5024568034557	0.01\\
45.5524595032397	0.01\\
45.6024622030238	0.01\\
45.6524649028078	0.01\\
45.7024676025918	0.01\\
45.7524703023758	0.01\\
45.8024730021598	0.01\\
45.8524757019439	0.01\\
45.9024784017279	0.01\\
45.9524811015119	0.01\\
46.0024838012959	0.01\\
46.0524865010799	0.01\\
46.1024892008639	0.01\\
46.152491900648	0.01\\
46.202494600432	0.01\\
46.252497300216	0.01\\
46.3025	0.01\\
46.352502699784	0.01\\
46.402505399568	0.01\\
46.4525080993521	0.01\\
46.5025107991361	0.01\\
46.5525134989201	0.01\\
46.6025161987041	0.01\\
46.6525188984881	0.01\\
46.7025215982721	0.01\\
46.7525242980562	0.01\\
46.8025269978402	0.01\\
46.8525296976242	0.01\\
46.9025323974082	0.01\\
46.9525350971922	0.01\\
47.0025377969762	0.01\\
47.0525404967603	0.01\\
47.1025431965443	0.01\\
47.1525458963283	0.01\\
47.2025485961123	0.01\\
47.2525512958963	0.01\\
47.3025539956803	0.01\\
47.3525566954644	0.01\\
47.4025593952484	0.01\\
47.4525620950324	0.01\\
47.5025647948164	0.01\\
47.5525674946004	0.01\\
47.6025701943845	0.01\\
47.6525728941685	0.01\\
47.7025755939525	0.01\\
47.7525782937365	0.01\\
47.8025809935205	0.01\\
47.8525836933045	0.01\\
47.9025863930886	0.01\\
47.9525890928726	0.01\\
48.0025917926566	0.01\\
48.0525944924406	0.01\\
48.1025971922246	0.01\\
48.1525998920086	0.01\\
48.2026025917927	0.01\\
48.2526052915767	0.01\\
48.3026079913607	0.01\\
48.3526106911447	0.01\\
48.4026133909287	0.01\\
48.4526160907127	0.01\\
48.5026187904968	0.01\\
48.5526214902808	0.01\\
48.6026241900648	0.01\\
48.6526268898488	0.01\\
48.7026295896328	0.01\\
48.7526322894168	0.01\\
48.8026349892009	0.01\\
48.8526376889849	0.01\\
48.9026403887689	0.01\\
48.9526430885529	0.01\\
49.0026457883369	0.01\\
49.052648488121	0.01\\
49.102651187905	0.01\\
49.152653887689	0.01\\
49.202656587473	0.01\\
49.252659287257	0.01\\
49.302661987041	0.01\\
49.3526646868251	0.01\\
49.4026673866091	0.01\\
49.4526700863931	0.01\\
49.5026727861771	0.01\\
49.5526754859611	0.01\\
49.6026781857451	0.01\\
49.6526808855292	0.01\\
49.7026835853132	0.01\\
49.7526862850972	0.01\\
49.8026889848812	0.01\\
49.8526916846652	0.01\\
49.9026943844492	0.01\\
49.9526970842333	0.01\\
50.0026997840173	0.01\\
50.0527024838013	0.01\\
50.1027051835853	0.01\\
50.1527078833693	0.01\\
50.2027105831534	0.01\\
50.2527132829374	0.01\\
50.3027159827214	0.01\\
50.3527186825054	0.01\\
50.4027213822894	0.01\\
50.4527240820734	0.01\\
50.5027267818575	0.01\\
50.5527294816415	0.01\\
50.6027321814255	0.01\\
50.6527348812095	0.01\\
50.7027375809935	0.01\\
50.7527402807775	0.01\\
50.8027429805616	0.01\\
50.8527456803456	0.01\\
50.9027483801296	0.01\\
50.9527510799136	0.01\\
51.0027537796976	0.01\\
51.0527564794816	0.01\\
51.1027591792657	0.01\\
51.1527618790497	0.01\\
51.2027645788337	0.01\\
51.2527672786177	0.01\\
51.3027699784017	0.01\\
51.3527726781858	0.01\\
51.4027753779698	0.01\\
51.4527780777538	0.01\\
51.5027807775378	0.01\\
51.5527834773218	0.01\\
51.6027861771058	0.01\\
51.6527888768899	0.01\\
51.7027915766739	0.01\\
51.7527942764579	0.01\\
51.8027969762419	0.01\\
51.8527996760259	0.01\\
51.9028023758099	0.01\\
51.952805075594	0.01\\
52.002807775378	0.01\\
52.052810475162	0.01\\
52.102813174946	0.01\\
52.15281587473	0.01\\
52.202818574514	0.01\\
52.2528212742981	0.01\\
52.3028239740821	0.01\\
52.3528266738661	0.01\\
52.4028293736501	0.01\\
52.4528320734341	0.01\\
52.5028347732181	0.01\\
52.5528374730022	0.01\\
52.6028401727862	0.01\\
52.6528428725702	0.01\\
52.7028455723542	0.01\\
52.7528482721382	0.01\\
52.8028509719222	0.01\\
52.8528536717063	0.01\\
52.9028563714903	0.01\\
52.9528590712743	0.01\\
53.0028617710583	0.01\\
53.0528644708423	0.01\\
53.1028671706264	0.01\\
53.1528698704104	0.01\\
53.2028725701944	0.01\\
53.2528752699784	0.01\\
53.3028779697624	0.01\\
53.3528806695464	0.01\\
53.4028833693305	0.01\\
53.4528860691145	0.01\\
53.5028887688985	0.01\\
53.5528914686825	0.01\\
53.6028941684665	0.01\\
53.6528968682505	0.01\\
53.7028995680346	0.01\\
53.7529022678186	0.01\\
53.8029049676026	0.01\\
53.8529076673866	0.01\\
53.9029103671706	0.01\\
53.9529130669547	0.01\\
54.0029157667387	0.01\\
54.0529184665227	0.01\\
54.1029211663067	0.01\\
54.1529238660907	0.01\\
54.2029265658747	0.01\\
54.2529292656588	0.01\\
54.3029319654428	0.01\\
54.3529346652268	0.01\\
54.4029373650108	0.01\\
54.4529400647948	0.01\\
54.5029427645788	0.01\\
54.5529454643629	0.01\\
54.6029481641469	0.01\\
54.6529508639309	0.01\\
54.7029535637149	0.01\\
54.7529562634989	0.01\\
54.8029589632829	0.01\\
54.852961663067	0.01\\
54.902964362851	0.01\\
54.952967062635	0.01\\
55.002969762419	0.01\\
55.052972462203	0.01\\
55.102975161987	0.01\\
55.1529778617711	0.01\\
55.2029805615551	0.01\\
55.2529832613391	0.01\\
55.3029859611231	0.01\\
55.3529886609071	0.01\\
55.4029913606911	0.01\\
55.4529940604752	0.01\\
55.5029967602592	0.01\\
55.5529994600432	0.01\\
55.6030021598272	0.01\\
55.6530048596112	0.01\\
55.7030075593953	0.01\\
55.7530102591793	0.01\\
55.8030129589633	0.01\\
55.8530156587473	0.01\\
55.9030183585313	0.01\\
55.9530210583153	0.01\\
56.0030237580994	0.01\\
56.0530264578834	0.01\\
56.1030291576674	0.01\\
56.1530318574514	0.01\\
56.2030345572354	0.01\\
56.2530372570194	0.01\\
56.3030399568035	0.01\\
56.3530426565875	0.01\\
56.4030453563715	0.01\\
56.4530480561555	0.01\\
56.5030507559395	0.01\\
56.5530534557235	0.01\\
56.6030561555076	0.01\\
56.6530588552916	0.01\\
56.7030615550756	0.01\\
56.7530642548596	0.01\\
56.8030669546436	0.01\\
56.8530696544276	0.01\\
56.9030723542117	0.01\\
56.9530750539957	0.01\\
57.0030777537797	0.01\\
57.0530804535637	0.01\\
57.1030831533477	0.01\\
57.1530858531318	0.01\\
57.2030885529158	0.01\\
57.2530912526998	0.01\\
57.3030939524838	0.01\\
57.3530966522678	0.01\\
57.4030993520518	0.01\\
57.4531020518358	0.01\\
57.5031047516199	0.01\\
57.5531074514039	0.01\\
57.6031101511879	0.01\\
57.6531128509719	0.01\\
57.7031155507559	0.01\\
57.75311825054	0.01\\
57.803120950324	0.01\\
57.853123650108	0.01\\
57.903126349892	0.01\\
57.953129049676	0.01\\
58.00313174946	0.01\\
58.0531344492441	0.01\\
58.1031371490281	0.01\\
58.1531398488121	0.01\\
58.2031425485961	0.01\\
58.2531452483801	0.01\\
58.3031479481642	0.01\\
58.3531506479482	0.01\\
58.4031533477322	0.01\\
58.4531560475162	0.01\\
58.5031587473002	0.01\\
58.5531614470842	0.01\\
58.6031641468683	0.01\\
58.6531668466523	0.01\\
58.7031695464363	0.01\\
58.7531722462203	0.01\\
58.8031749460043	0.01\\
58.8531776457883	0.01\\
58.9031803455724	0.01\\
58.9531830453564	0.01\\
59.0031857451404	0.01\\
59.0531884449244	0.01\\
59.1031911447084	0.01\\
59.1531938444924	0.01\\
59.2031965442765	0.01\\
59.2531992440605	0.01\\
59.3032019438445	0.01\\
59.3532046436285	0.01\\
59.4032073434125	0.01\\
59.4532100431965	0.01\\
59.5032127429806	0.01\\
59.5532154427646	0.01\\
59.6032181425486	0.01\\
59.6532208423326	0.01\\
59.7032235421166	0.01\\
59.7532262419006	0.01\\
59.8032289416847	0.01\\
59.8532316414687	0.01\\
59.9032343412527	0.01\\
59.9532370410367	0.01\\
60.0032397408207	0.01\\
60.0532424406048	0.01\\
60.1032451403888	0.01\\
60.1532478401728	0.01\\
60.2032505399568	0.01\\
60.2532532397408	0.01\\
60.3032559395248	0.01\\
60.3532586393089	0.01\\
60.4032613390929	0.01\\
60.4532640388769	0.01\\
60.5032667386609	0.01\\
60.5532694384449	0.01\\
60.6032721382289	0.01\\
60.653274838013	0.01\\
60.703277537797	0.01\\
60.753280237581	0.01\\
60.803282937365	0.01\\
60.853285637149	0.01\\
60.9032883369331	0.01\\
60.9532910367171	0.01\\
61.0032937365011	0.01\\
61.0532964362851	0.01\\
61.1032991360691	0.01\\
61.1533018358531	0.01\\
61.2033045356371	0.01\\
61.2533072354212	0.01\\
61.3033099352052	0.01\\
61.3533126349892	0.01\\
61.4033153347732	0.01\\
61.4533180345572	0.01\\
61.5033207343413	0.01\\
61.5533234341253	0.01\\
61.6033261339093	0.01\\
61.6533288336933	0.01\\
61.7033315334773	0.01\\
61.7533342332613	0.01\\
61.8033369330454	0.01\\
61.8533396328294	0.01\\
61.9033423326134	0.01\\
61.9533450323974	0.01\\
62.0033477321814	0.01\\
62.0533504319654	0.01\\
62.1033531317495	0.01\\
62.1533558315335	0.01\\
62.2033585313175	0.01\\
62.2533612311015	0.01\\
62.3033639308855	0.01\\
62.3533666306695	0.01\\
62.4033693304536	0.01\\
62.4533720302376	0.01\\
62.5033747300216	0.01\\
62.5483771598272	0.21\\
62.5983798596112	0.21\\
62.6483825593953	0.21\\
62.6983852591793	0.21\\
62.7483879589633	0.21\\
62.7983906587473	0.21\\
62.8483933585313	0.21\\
62.8983960583153	0.21\\
62.9483987580994	0.21\\
62.9984014578834	0.21\\
63.0484041576674	0.21\\
63.0984068574514	0.21\\
63.1484095572354	0.21\\
63.1984122570194	0.21\\
63.2484149568035	0.21\\
63.2984176565875	0.21\\
63.3484203563715	0.21\\
63.3984230561555	0.21\\
63.4484257559395	0.21\\
63.4984284557235	0.21\\
63.5484311555076	0.21\\
63.5984338552916	0.21\\
63.6484365550756	0.21\\
63.6984392548596	0.21\\
63.7484419546436	0.21\\
63.7984446544276	0.21\\
63.8484473542117	0.21\\
63.8984500539957	0.21\\
63.9484527537797	0.21\\
63.9984554535637	0.21\\
64.0484581533477	0.21\\
64.0984608531318	0.21\\
64.1484635529158	0.21\\
64.1984662526998	0.21\\
64.2484689524838	0.21\\
64.2984716522678	0.21\\
64.3484743520518	0.21\\
64.3984770518359	0.21\\
64.4484797516199	0.21\\
64.4984824514039	0.21\\
64.5484851511879	0.21\\
64.5984878509719	0.21\\
64.648490550756	0.21\\
64.69849325054	0.21\\
64.748495950324	0.21\\
64.798498650108	0.21\\
64.848501349892	0.21\\
64.898504049676	0.21\\
64.94850674946	0.21\\
64.9985094492441	0.21\\
65.0485121490281	0.21\\
65.0985148488121	0.21\\
65.1485175485961	0.21\\
65.1985202483801	0.21\\
65.2485229481642	0.21\\
65.2985256479482	0.21\\
65.3485283477322	0.21\\
65.3985310475162	0.21\\
65.4485337473002	0.21\\
65.4985364470842	0.21\\
65.5485391468683	0.21\\
65.5985418466523	0.21\\
65.6485445464363	0.21\\
65.6985472462203	0.21\\
65.7485499460043	0.21\\
65.7985526457883	0.21\\
65.8485553455724	0.21\\
65.8985580453564	0.21\\
65.9485607451404	0.21\\
65.9985634449244	0.21\\
66.0485661447084	0.21\\
66.0985688444924	0.21\\
66.1485715442765	0.21\\
66.1985742440605	0.21\\
66.2485769438445	0.21\\
66.2985796436285	0.21\\
66.3485823434125	0.21\\
66.3985850431965	0.21\\
66.4485877429806	0.21\\
66.4985904427646	0.21\\
66.5485931425486	0.21\\
66.5985958423326	0.21\\
66.6485985421166	0.21\\
66.6986012419006	0.21\\
66.7486039416847	0.21\\
66.7986066414687	0.21\\
66.8486093412527	0.21\\
66.8986120410367	0.21\\
66.9486147408207	0.21\\
66.9986174406047	0.21\\
67.0486201403888	0.21\\
67.0986228401728	0.21\\
67.1486255399568	0.21\\
67.1986282397408	0.21\\
67.2486309395248	0.21\\
67.2986336393089	0.21\\
67.3486363390929	0.21\\
67.3986390388769	0.21\\
67.4486417386609	0.21\\
67.4986444384449	0.21\\
67.5486471382289	0.21\\
67.598649838013	0.21\\
67.648652537797	0.21\\
67.698655237581	0.21\\
67.748657937365	0.21\\
67.798660637149	0.21\\
67.8486633369331	0.21\\
67.8986660367171	0.21\\
67.9486687365011	0.21\\
67.9986714362851	0.21\\
68.0486741360691	0.21\\
68.0986768358531	0.21\\
68.1486795356372	0.21\\
68.1986822354212	0.21\\
68.2486849352052	0.21\\
68.2986876349892	0.21\\
68.3486903347732	0.21\\
68.3986930345572	0.21\\
68.4486957343413	0.21\\
68.4986984341253	0.21\\
68.5487011339093	0.21\\
68.5987038336933	0.21\\
68.6487065334773	0.21\\
68.6987092332613	0.21\\
68.7487119330454	0.21\\
68.7987146328294	0.21\\
68.8487173326134	0.21\\
68.8987200323974	0.21\\
68.9487227321814	0.21\\
68.9987254319654	0.21\\
69.0487281317495	0.21\\
69.0987308315335	0.21\\
69.1487335313175	0.21\\
69.1987362311015	0.21\\
69.2487389308855	0.21\\
69.2987416306696	0.21\\
69.3487443304536	0.21\\
69.3987470302376	0.21\\
69.4487497300216	0.21\\
69.4987524298056	0.21\\
69.5487551295896	0.21\\
69.5987578293737	0.21\\
69.6487605291577	0.21\\
69.6987632289417	0.21\\
69.7487659287257	0.21\\
69.7987686285097	0.21\\
69.8487713282937	0.21\\
69.8987740280778	0.21\\
69.9487767278618	0.21\\
69.9987794276458	0.21\\
70.0487821274298	0.21\\
70.0987848272138	0.21\\
70.1487875269979	0.21\\
70.1987902267819	0.21\\
70.2487929265659	0.21\\
70.2987956263499	0.21\\
70.3487983261339	0.21\\
70.3988010259179	0.21\\
70.4488037257019	0.21\\
70.498806425486	0.21\\
70.54880912527	0.21\\
70.598811825054	0.21\\
70.648814524838	0.21\\
70.698817224622	0.21\\
70.7488199244061	0.21\\
70.7988226241901	0.21\\
70.8488253239741	0.21\\
70.8988280237581	0.21\\
70.9488307235421	0.21\\
70.9988334233261	0.21\\
71.0488361231101	0.21\\
71.0988388228942	0.21\\
71.1488415226782	0.21\\
71.1988442224622	0.21\\
71.2488469222462	0.21\\
71.2988496220302	0.21\\
71.3488523218143	0.21\\
71.3988550215983	0.21\\
71.4488577213823	0.21\\
71.4988604211663	0.21\\
71.5488631209503	0.21\\
71.5988658207343	0.21\\
71.6488685205184	0.21\\
71.6988712203024	0.21\\
71.7488739200864	0.21\\
71.7988766198704	0.21\\
71.8488793196544	0.21\\
71.8988820194385	0.21\\
71.9488847192225	0.21\\
71.9988874190065	0.21\\
72.0488901187905	0.21\\
72.0988928185745	0.21\\
72.1488955183585	0.21\\
72.1988982181425	0.21\\
72.2489009179266	0.21\\
72.2989036177106	0.21\\
72.3489063174946	0.21\\
72.3989090172786	0.21\\
72.4489117170626	0.21\\
72.4989144168467	0.21\\
72.5489171166307	0.21\\
72.5989198164147	0.21\\
72.6489225161987	0.21\\
72.6989252159827	0.21\\
72.7489279157667	0.21\\
72.7989306155508	0.21\\
72.8489333153348	0.21\\
72.8989360151188	0.21\\
72.9489387149028	0.21\\
72.9989414146868	0.21\\
73.0489441144708	0.21\\
73.0989468142549	0.21\\
73.1489495140389	0.21\\
73.1989522138229	0.21\\
73.2489549136069	0.21\\
73.2989576133909	0.21\\
73.348960313175	0.21\\
73.398963012959	0.21\\
73.448965712743	0.21\\
73.498968412527	0.21\\
73.548971112311	0.21\\
73.598973812095	0.21\\
73.6489765118791	0.21\\
73.6989792116631	0.21\\
73.7489819114471	0.21\\
73.7989846112311	0.21\\
73.8489873110151	0.21\\
73.8989900107991	0.21\\
73.9489927105832	0.21\\
73.9989954103672	0.21\\
74.0489981101512	0.21\\
74.0990008099352	0.21\\
74.1490035097192	0.21\\
74.1990062095032	0.21\\
74.2490089092873	0.21\\
74.2990116090713	0.21\\
74.3490143088553	0.21\\
74.3990170086393	0.21\\
74.4490197084233	0.21\\
74.4990224082073	0.21\\
74.5490251079914	0.21\\
74.5990278077754	0.21\\
74.6490305075594	0.21\\
74.6990332073434	0.21\\
74.7490359071274	0.21\\
74.7990386069115	0.21\\
74.8490413066955	0.21\\
74.8990440064795	0.21\\
74.9490467062635	0.21\\
74.9990494060475	0.21\\
75.0490521058315	0.21\\
75.0990548056156	0.21\\
75.1490575053996	0.21\\
75.1990602051836	0.21\\
75.2490629049676	0.21\\
75.2990656047516	0.21\\
75.3490683045356	0.21\\
75.3990710043197	0.21\\
75.4490737041037	0.21\\
75.4990764038877	0.21\\
75.5490791036717	0.21\\
75.5990818034557	0.21\\
75.6490845032397	0.21\\
75.6990872030238	0.21\\
75.7490899028078	0.21\\
75.7990926025918	0.21\\
75.8490953023758	0.21\\
75.8990980021598	0.21\\
75.9491007019438	0.21\\
75.9991034017279	0.21\\
76.0491061015119	0.21\\
76.0991088012959	0.21\\
76.1491115010799	0.21\\
76.1991142008639	0.21\\
76.2491169006479	0.21\\
76.299119600432	0.21\\
76.349122300216	0.21\\
76.399125	0.21\\
76.449127699784	0.21\\
76.499130399568	0.21\\
76.5491330993521	0.21\\
76.5991357991361	0.21\\
76.6491384989201	0.21\\
76.6991411987041	0.21\\
76.7491438984881	0.21\\
76.7991465982721	0.21\\
76.8491492980562	0.21\\
76.8991519978402	0.21\\
76.9491546976242	0.21\\
76.9991573974082	0.21\\
77.0491600971922	0.21\\
77.0991627969762	0.21\\
77.1491654967603	0.21\\
77.1991681965443	0.21\\
77.2491708963283	0.21\\
77.2991735961123	0.21\\
77.3491762958963	0.21\\
77.3991789956804	0.21\\
77.4491816954644	0.21\\
77.4991843952484	0.21\\
77.5491870950324	0.21\\
77.5991897948164	0.21\\
77.6491924946004	0.21\\
77.6991951943844	0.21\\
77.7491978941685	0.21\\
77.7992005939525	0.21\\
77.8492032937365	0.21\\
77.8992059935205	0.21\\
77.9492086933045	0.21\\
77.9992113930886	0.21\\
78.0492140928726	0.21\\
78.0992167926566	0.21\\
78.1492194924406	0.21\\
78.1992221922246	0.21\\
78.2492248920086	0.21\\
78.2992275917927	0.21\\
78.3492302915767	0.21\\
78.3992329913607	0.21\\
78.4492356911447	0.21\\
78.4992383909287	0.21\\
78.5492410907127	0.21\\
78.5992437904968	0.21\\
78.6492464902808	0.21\\
78.6992491900648	0.21\\
78.7492518898488	0.21\\
78.7992545896328	0.21\\
78.8492572894169	0.21\\
78.8992599892009	0.21\\
78.9492626889849	0.21\\
78.9992653887689	0.21\\
79.0492680885529	0.21\\
79.0992707883369	0.21\\
79.149273488121	0.21\\
79.199276187905	0.21\\
79.249278887689	0.21\\
79.299281587473	0.21\\
79.349284287257	0.21\\
79.3992869870411	0.21\\
79.4492896868251	0.21\\
79.4992923866091	0.21\\
79.5492950863931	0.21\\
79.5992977861771	0.21\\
79.6493004859611	0.21\\
79.6993031857451	0.21\\
79.7493058855292	0.21\\
79.7993085853132	0.21\\
79.8493112850972	0.21\\
79.8993139848812	0.21\\
79.9493166846652	0.21\\
79.9993193844493	0.21\\
80.0493220842333	0.21\\
80.0993247840173	0.21\\
80.1493274838013	0.21\\
80.1993301835853	0.21\\
80.2493328833693	0.21\\
80.2993355831534	0.21\\
80.3493382829374	0.21\\
80.3993409827214	0.21\\
80.4493436825054	0.21\\
80.4993463822894	0.21\\
80.5493490820734	0.21\\
80.5993517818575	0.21\\
80.6493544816415	0.21\\
80.6993571814255	0.21\\
80.7493598812095	0.21\\
80.7993625809935	0.21\\
80.8493652807775	0.21\\
80.8993679805616	0.21\\
80.9493706803456	0.21\\
80.9993733801296	0.21\\
81.0493760799136	0.21\\
81.0993787796976	0.21\\
81.1493814794816	0.21\\
81.1993841792657	0.21\\
81.2493868790497	0.21\\
81.2993895788337	0.21\\
81.3493922786177	0.21\\
81.3993949784017	0.21\\
81.4493976781857	0.21\\
81.4994003779698	0.21\\
81.5494030777538	0.21\\
81.5994057775378	0.21\\
81.6494084773218	0.21\\
81.6994111771058	0.21\\
81.7494138768898	0.21\\
81.7994165766739	0.21\\
81.8494192764579	0.21\\
81.8994219762419	0.21\\
81.9494246760259	0.21\\
81.9994273758099	0.21\\
82.049430075594	0.21\\
82.099432775378	0.21\\
82.149435475162	0.21\\
82.199438174946	0.21\\
82.24944087473	0.21\\
82.299443574514	0.21\\
82.3494462742981	0.21\\
82.3994489740821	0.21\\
82.4494516738661	0.21\\
82.4994543736501	0.21\\
82.5494570734341	0.21\\
82.5994597732181	0.21\\
82.6294613930885	0.01\\
82.6794640928726	0.01\\
82.7294667926566	0.01\\
82.7794694924406	0.01\\
82.8294721922246	0.01\\
82.8794748920086	0.01\\
82.9294775917927	0.01\\
82.9794802915767	0.01\\
83.0294829913607	0.01\\
83.0794856911447	0.01\\
83.1294883909287	0.01\\
83.1794910907127	0.01\\
83.2294937904968	0.01\\
83.2794964902808	0.01\\
83.3294991900648	0.01\\
83.3795018898488	0.01\\
83.4295045896328	0.01\\
83.4795072894169	0.01\\
83.5295099892009	0.01\\
83.5795126889849	0.01\\
83.6295153887689	0.01\\
83.6795180885529	0.01\\
83.7295207883369	0.01\\
83.779523488121	0.01\\
83.829526187905	0.01\\
83.879528887689	0.01\\
83.929531587473	0.01\\
83.979534287257	0.01\\
84.029536987041	0.01\\
84.0795396868251	0.01\\
84.1295423866091	0.01\\
84.1795450863931	0.01\\
84.2295477861771	0.01\\
84.2795504859611	0.01\\
84.3295531857451	0.01\\
84.3795558855292	0.01\\
84.4295585853132	0.01\\
84.4795612850972	0.01\\
84.5295639848812	0.01\\
84.5795666846652	0.01\\
84.6295693844492	0.01\\
84.6795720842333	0.01\\
84.7295747840173	0.01\\
84.7795774838013	0.01\\
84.8295801835853	0.01\\
84.8795828833693	0.01\\
84.9295855831534	0.01\\
84.9795882829374	0.01\\
85.0295909827214	0.01\\
85.0795936825054	0.01\\
85.1295963822894	0.01\\
85.1795990820734	0.01\\
85.2296017818575	0.01\\
85.2796044816415	0.01\\
85.3296071814255	0.01\\
85.3796098812095	0.01\\
85.4296125809935	0.01\\
85.4796152807775	0.01\\
85.5296179805616	0.01\\
85.5796206803456	0.01\\
85.6296233801296	0.01\\
85.6796260799136	0.01\\
85.7296287796976	0.01\\
85.7796314794817	0.01\\
85.8296341792657	0.01\\
85.8796368790497	0.01\\
85.9296395788337	0.01\\
85.9796422786177	0.01\\
86.0296449784017	0.01\\
86.0796476781857	0.01\\
86.1296503779698	0.01\\
86.1796530777538	0.01\\
86.2296557775378	0.01\\
86.2796584773218	0.01\\
86.3296611771058	0.01\\
86.3796638768899	0.01\\
86.4296665766739	0.01\\
86.4796692764579	0.01\\
86.5296719762419	0.01\\
86.5796746760259	0.01\\
86.6296773758099	0.01\\
86.679680075594	0.01\\
86.729682775378	0.01\\
86.779685475162	0.01\\
86.829688174946	0.01\\
86.87969087473	0.01\\
86.929693574514	0.01\\
86.9796962742981	0.01\\
87.0296989740821	0.01\\
87.0797016738661	0.01\\
87.1297043736501	0.01\\
87.1797070734341	0.01\\
87.2297097732181	0.01\\
87.2797124730022	0.01\\
87.3297151727862	0.01\\
87.3797178725702	0.01\\
87.4297205723542	0.01\\
87.4797232721382	0.01\\
87.5297259719222	0.01\\
87.5797286717063	0.01\\
87.6297313714903	0.01\\
87.6797340712743	0.01\\
87.7297367710583	0.01\\
87.7797394708423	0.01\\
87.8297421706264	0.01\\
87.8797448704104	0.01\\
87.9297475701944	0.01\\
87.9797502699784	0.01\\
88.0297529697624	0.01\\
88.0797556695464	0.01\\
88.1297583693305	0.01\\
88.1797610691145	0.01\\
88.2297637688985	0.01\\
88.2797664686825	0.01\\
88.3297691684665	0.01\\
88.3797718682505	0.01\\
88.4297745680346	0.01\\
88.4797772678186	0.01\\
88.5297799676026	0.01\\
88.5797826673866	0.01\\
88.6297853671706	0.01\\
88.6797880669546	0.01\\
88.7297907667387	0.01\\
88.7797934665227	0.01\\
88.8297961663067	0.01\\
88.8797988660907	0.01\\
88.9298015658747	0.01\\
88.9798042656588	0.01\\
89.0298069654428	0.01\\
89.0798096652268	0.01\\
89.1298123650108	0.01\\
89.1798150647948	0.01\\
89.2298177645788	0.01\\
89.2798204643629	0.01\\
89.3298231641469	0.01\\
89.3798258639309	0.01\\
89.4298285637149	0.01\\
89.4798312634989	0.01\\
89.5298339632829	0.01\\
89.579836663067	0.01\\
89.629839362851	0.01\\
89.679842062635	0.01\\
89.729844762419	0.01\\
89.779847462203	0.01\\
89.8298501619871	0.01\\
89.8798528617711	0.01\\
89.9298555615551	0.01\\
89.9798582613391	0.01\\
90.0298609611231	0.01\\
90.0798636609071	0.01\\
90.1298663606911	0.01\\
90.1798690604752	0.01\\
90.2298717602592	0.01\\
90.2798744600432	0.01\\
90.3298771598272	0.01\\
90.3798798596112	0.01\\
90.4298825593953	0.01\\
90.4798852591793	0.01\\
90.5298879589633	0.01\\
90.5798906587473	0.01\\
90.6298933585313	0.01\\
90.6798960583153	0.01\\
90.7298987580994	0.01\\
90.7799014578834	0.01\\
90.8299041576674	0.01\\
90.8799068574514	0.01\\
90.9299095572354	0.01\\
90.9799122570195	0.01\\
91.0299149568035	0.01\\
91.0799176565875	0.01\\
91.1299203563715	0.01\\
91.1799230561555	0.01\\
91.2299257559395	0.01\\
91.2799284557236	0.01\\
91.3299311555076	0.01\\
91.3799338552916	0.01\\
91.4299365550756	0.01\\
91.4799392548596	0.01\\
91.5299419546436	0.01\\
91.5799446544276	0.01\\
91.6299473542117	0.01\\
91.6799500539957	0.01\\
91.7299527537797	0.01\\
91.7799554535637	0.01\\
91.8299581533477	0.01\\
91.8799608531317	0.01\\
91.9299635529158	0.01\\
91.9799662526998	0.01\\
92.0299689524838	0.01\\
92.0799716522678	0.01\\
92.1299743520518	0.01\\
92.1799770518358	0.01\\
92.2299797516199	0.01\\
92.2799824514039	0.01\\
92.3299851511879	0.01\\
92.3799878509719	0.01\\
92.4299905507559	0.01\\
92.47999325054	0.01\\
92.529995950324	0.01\\
92.579998650108	0.01\\
92.605	0.01\\
};
\addlegendentry{+ 1cm};

\addplot [color=gray,solid,line width=0.2pt]
  table[row sep=crcr]{0	-0.01\\
0.0500026997840173	-0.01\\
0.100005399568035	-0.01\\
0.150008099352052	-0.01\\
0.200010799136069	-0.01\\
0.250013498920086	-0.01\\
0.300016198704104	-0.01\\
0.350018898488121	-0.01\\
0.400021598272138	-0.01\\
0.450024298056156	-0.01\\
0.500026997840173	-0.01\\
0.55002969762419	-0.01\\
0.600032397408207	-0.01\\
0.650035097192225	-0.01\\
0.700037796976242	-0.01\\
0.750040496760259	-0.01\\
0.800043196544276	-0.01\\
0.850045896328294	-0.01\\
0.900048596112311	-0.01\\
0.950051295896328	-0.01\\
1.00005399568035	-0.01\\
1.05005669546436	-0.01\\
1.10005939524838	-0.01\\
1.1500620950324	-0.01\\
1.20006479481641	-0.01\\
1.25006749460043	-0.01\\
1.30007019438445	-0.01\\
1.35007289416847	-0.01\\
1.40007559395248	-0.01\\
1.4500782937365	-0.01\\
1.50008099352052	-0.01\\
1.55008369330454	-0.01\\
1.60008639308855	-0.01\\
1.65008909287257	-0.01\\
1.70009179265659	-0.01\\
1.7500944924406	-0.01\\
1.80009719222462	-0.01\\
1.85009989200864	-0.01\\
1.90010259179266	-0.01\\
1.95010529157667	-0.01\\
2.00010799136069	-0.01\\
2.05011069114471	-0.01\\
2.10011339092873	-0.01\\
2.15011609071274	-0.01\\
2.20011879049676	-0.01\\
2.25012149028078	-0.01\\
2.30012419006479	-0.01\\
2.35012688984881	-0.01\\
2.40012958963283	-0.01\\
2.45013228941685	-0.01\\
2.50013498920086	-0.01\\
2.55013768898488	-0.01\\
2.6001403887689	-0.01\\
2.65014308855292	-0.01\\
2.70014578833693	-0.01\\
2.75014848812095	-0.01\\
2.80015118790497	-0.01\\
2.85015388768899	-0.01\\
2.900156587473	-0.01\\
2.95015928725702	-0.01\\
3.00016198704104	-0.01\\
3.05016468682505	-0.01\\
3.10016738660907	-0.01\\
3.15017008639309	-0.01\\
3.20017278617711	-0.01\\
3.25017548596112	-0.01\\
3.30017818574514	-0.01\\
3.35018088552916	-0.01\\
3.40018358531318	-0.01\\
3.45018628509719	-0.01\\
3.50018898488121	-0.01\\
3.55019168466523	-0.01\\
3.60019438444924	-0.01\\
3.65019708423326	-0.01\\
3.70019978401728	-0.01\\
3.7502024838013	-0.01\\
3.80020518358531	-0.01\\
3.85020788336933	-0.01\\
3.90021058315335	-0.01\\
3.95021328293736	-0.01\\
4.00021598272138	-0.01\\
4.0502186825054	-0.01\\
4.10022138228942	-0.01\\
4.15022408207343	-0.01\\
4.20022678185745	-0.01\\
4.25022948164147	-0.01\\
4.30023218142549	-0.01\\
4.3502348812095	-0.01\\
4.40023758099352	-0.01\\
4.45024028077754	-0.01\\
4.50024298056155	-0.01\\
4.55024568034557	-0.01\\
4.60024838012959	-0.01\\
4.65025107991361	-0.01\\
4.70025377969762	-0.01\\
4.75025647948164	-0.01\\
4.80025917926566	-0.01\\
4.85026187904968	-0.01\\
4.90026457883369	-0.01\\
4.95026727861771	-0.01\\
5.00026997840173	-0.01\\
5.05027267818575	-0.01\\
5.10027537796976	-0.01\\
5.15027807775378	-0.01\\
5.2002807775378	-0.01\\
5.25028347732181	-0.01\\
5.30028617710583	-0.01\\
5.35028887688985	-0.01\\
5.40029157667387	-0.01\\
5.45029427645788	-0.01\\
5.5002969762419	-0.01\\
5.55029967602592	-0.01\\
5.60030237580994	-0.01\\
5.65030507559395	-0.01\\
5.70030777537797	-0.01\\
5.75031047516199	-0.01\\
5.800313174946	-0.01\\
5.85031587473002	-0.01\\
5.90031857451404	-0.01\\
5.95032127429806	-0.01\\
6.00032397408207	-0.01\\
6.05032667386609	-0.01\\
6.10032937365011	-0.01\\
6.15033207343413	-0.01\\
6.20033477321814	-0.01\\
6.25033747300216	-0.01\\
6.30034017278618	-0.01\\
6.3503428725702	-0.01\\
6.40034557235421	-0.01\\
6.45034827213823	-0.01\\
6.50035097192225	-0.01\\
6.55035367170626	-0.01\\
6.60035637149028	-0.01\\
6.6503590712743	-0.01\\
6.70036177105832	-0.01\\
6.75036447084233	-0.01\\
6.80036717062635	-0.01\\
6.85036987041037	-0.01\\
6.90037257019439	-0.01\\
6.9503752699784	-0.01\\
7.00037796976242	-0.01\\
7.05038066954644	-0.01\\
7.10038336933045	-0.01\\
7.15038606911447	-0.01\\
7.20038876889849	-0.01\\
7.25039146868251	-0.01\\
7.30039416846652	-0.01\\
7.35039686825054	-0.01\\
7.40039956803456	-0.01\\
7.45040226781857	-0.01\\
7.50040496760259	-0.01\\
7.55040766738661	-0.01\\
7.60041036717063	-0.01\\
7.65041306695464	-0.01\\
7.70041576673866	-0.01\\
7.75041846652268	-0.01\\
7.8004211663067	-0.01\\
7.85042386609071	-0.01\\
7.90042656587473	-0.01\\
7.95042926565875	-0.01\\
8.00043196544276	-0.01\\
8.05043466522678	-0.01\\
8.1004373650108	-0.01\\
8.15044006479482	-0.01\\
8.20044276457883	-0.01\\
8.25044546436285	-0.01\\
8.30044816414687	-0.01\\
8.35045086393089	-0.01\\
8.4004535637149	-0.01\\
8.45045626349892	-0.01\\
8.50045896328294	-0.01\\
8.55046166306696	-0.01\\
8.60046436285097	-0.01\\
8.65046706263499	-0.01\\
8.70046976241901	-0.01\\
8.75047246220302	-0.01\\
8.80047516198704	-0.01\\
8.85047786177106	-0.01\\
8.90048056155508	-0.01\\
8.95048326133909	-0.01\\
9.00048596112311	-0.01\\
9.05048866090713	-0.01\\
9.10049136069114	-0.01\\
9.15049406047516	-0.01\\
9.20049676025918	-0.01\\
9.2504994600432	-0.01\\
9.30050215982721	-0.01\\
9.35050485961123	-0.01\\
9.40050755939525	-0.01\\
9.45051025917927	-0.01\\
9.50051295896328	-0.01\\
9.5505156587473	-0.01\\
9.60051835853132	-0.01\\
9.65052105831533	-0.01\\
9.70052375809935	-0.01\\
9.75052645788337	-0.01\\
9.80052915766739	-0.01\\
9.8505318574514	-0.01\\
9.90053455723542	-0.01\\
9.95053725701944	-0.01\\
10.0005399568035	-0.01\\
10.0505426565875	-0.01\\
10.1005453563715	-0.01\\
10.1505480561555	-0.01\\
10.2005507559395	-0.01\\
10.2505534557235	-0.01\\
10.3005561555076	-0.01\\
10.3505588552916	-0.01\\
10.4005615550756	-0.01\\
10.4505642548596	-0.01\\
10.5005669546436	-0.01\\
10.5505696544276	-0.01\\
10.6005723542117	-0.01\\
10.6505750539957	-0.01\\
10.7005777537797	-0.01\\
10.7505804535637	-0.01\\
10.8005831533477	-0.01\\
10.8505858531317	-0.01\\
10.9005885529158	-0.01\\
10.9505912526998	-0.01\\
11.0005939524838	-0.01\\
11.0505966522678	-0.01\\
11.1005993520518	-0.01\\
11.1506020518359	-0.01\\
11.2006047516199	-0.01\\
11.2506074514039	-0.01\\
11.3006101511879	-0.01\\
11.3506128509719	-0.01\\
11.4006155507559	-0.01\\
11.45061825054	-0.01\\
11.500620950324	-0.01\\
11.550623650108	-0.01\\
11.600626349892	-0.01\\
11.650629049676	-0.01\\
11.70063174946	-0.01\\
11.7506344492441	-0.01\\
11.8006371490281	-0.01\\
11.8506398488121	-0.01\\
11.9006425485961	-0.01\\
11.9506452483801	-0.01\\
12.0006479481641	-0.01\\
12.0506506479482	-0.01\\
12.1006533477322	-0.01\\
12.1506560475162	-0.01\\
12.2006587473002	-0.01\\
12.2506614470842	-0.01\\
12.3006641468683	-0.01\\
12.3506668466523	-0.01\\
12.4006695464363	-0.01\\
12.4506722462203	-0.01\\
12.5006749460043	-0.01\\
12.5506776457883	-0.01\\
12.6006803455724	-0.01\\
12.6506830453564	-0.01\\
12.7006857451404	-0.01\\
12.7506884449244	-0.01\\
12.8006911447084	-0.01\\
12.8506938444924	-0.01\\
12.9006965442765	-0.01\\
12.9506992440605	-0.01\\
13.0007019438445	-0.01\\
13.0507046436285	-0.01\\
13.1007073434125	-0.01\\
13.1507100431965	-0.01\\
13.2007127429806	-0.01\\
13.2507154427646	-0.01\\
13.3007181425486	-0.01\\
13.3507208423326	-0.01\\
13.4007235421166	-0.01\\
13.4507262419006	-0.01\\
13.5007289416847	-0.01\\
13.5507316414687	-0.01\\
13.6007343412527	-0.01\\
13.6507370410367	-0.01\\
13.7007397408207	-0.01\\
13.7507424406048	-0.01\\
13.8007451403888	-0.01\\
13.8507478401728	-0.01\\
13.9007505399568	-0.01\\
13.9507532397408	-0.01\\
14.0007559395248	-0.01\\
14.0507586393089	-0.01\\
14.1007613390929	-0.01\\
14.1507640388769	-0.01\\
14.2007667386609	-0.01\\
14.2507694384449	-0.01\\
14.3007721382289	-0.01\\
14.350774838013	-0.01\\
14.400777537797	-0.01\\
14.450780237581	-0.01\\
14.500782937365	-0.01\\
14.550785637149	-0.01\\
14.600788336933	-0.01\\
14.6507910367171	-0.01\\
14.7007937365011	-0.01\\
14.7507964362851	-0.01\\
14.8007991360691	-0.01\\
14.8508018358531	-0.01\\
14.9008045356372	-0.01\\
14.9508072354212	-0.01\\
15.0008099352052	-0.01\\
15.0508126349892	-0.01\\
15.1008153347732	-0.01\\
15.1508180345572	-0.01\\
15.2008207343413	-0.01\\
15.2508234341253	-0.01\\
15.3008261339093	-0.01\\
15.3508288336933	-0.01\\
15.4008315334773	-0.01\\
15.4508342332613	-0.01\\
15.5008369330454	-0.01\\
15.5508396328294	-0.01\\
15.6008423326134	-0.01\\
15.6508450323974	-0.01\\
15.7008477321814	-0.01\\
15.7508504319654	-0.01\\
15.8008531317495	-0.01\\
15.8508558315335	-0.01\\
15.9008585313175	-0.01\\
15.9508612311015	-0.01\\
16.0008639308855	-0.01\\
16.0508666306695	-0.01\\
16.1008693304536	-0.01\\
16.1508720302376	-0.01\\
16.2008747300216	-0.01\\
16.2508774298056	-0.01\\
16.3008801295896	-0.01\\
16.3508828293737	-0.01\\
16.4008855291577	-0.01\\
16.4508882289417	-0.01\\
16.5008909287257	-0.01\\
16.5508936285097	-0.01\\
16.6008963282937	-0.01\\
16.6508990280778	-0.01\\
16.7009017278618	-0.01\\
16.7509044276458	-0.01\\
16.8009071274298	-0.01\\
16.8509098272138	-0.01\\
16.9009125269978	-0.01\\
16.9509152267819	-0.01\\
17.0009179265659	-0.01\\
17.0509206263499	-0.01\\
17.1009233261339	-0.01\\
17.1509260259179	-0.01\\
17.2009287257019	-0.01\\
17.250931425486	-0.01\\
17.30093412527	-0.01\\
17.350936825054	-0.01\\
17.400939524838	-0.01\\
17.450942224622	-0.01\\
17.500944924406	-0.01\\
17.5509476241901	-0.01\\
17.6009503239741	-0.01\\
17.6509530237581	-0.01\\
17.7009557235421	-0.01\\
17.7509584233261	-0.01\\
17.8009611231102	-0.01\\
17.8509638228942	-0.01\\
17.9009665226782	-0.01\\
17.9509692224622	-0.01\\
18.0009719222462	-0.01\\
18.0509746220302	-0.01\\
18.1009773218143	-0.01\\
18.1509800215983	-0.01\\
18.2009827213823	-0.01\\
18.2509854211663	-0.01\\
18.3009881209503	-0.01\\
18.3509908207343	-0.01\\
18.4009935205184	-0.01\\
18.4509962203024	-0.01\\
18.5009989200864	-0.01\\
18.5510016198704	-0.01\\
18.6010043196544	-0.01\\
18.6510070194384	-0.01\\
18.7010097192225	-0.01\\
18.7510124190065	-0.01\\
18.8010151187905	-0.01\\
18.8510178185745	-0.01\\
18.9010205183585	-0.01\\
18.9510232181425	-0.01\\
19.0010259179266	-0.01\\
19.0510286177106	-0.01\\
19.1010313174946	-0.01\\
19.1510340172786	-0.01\\
19.2010367170626	-0.01\\
19.2510394168467	-0.01\\
19.3010421166307	-0.01\\
19.3510448164147	-0.01\\
19.4010475161987	-0.01\\
19.4510502159827	-0.01\\
19.5010529157667	-0.01\\
19.5510556155508	-0.01\\
19.6010583153348	-0.01\\
19.6510610151188	-0.01\\
19.7010637149028	-0.01\\
19.7510664146868	-0.01\\
19.8010691144708	-0.01\\
19.8510718142549	-0.01\\
19.9010745140389	-0.01\\
19.9510772138229	-0.01\\
20.0010799136069	-0.01\\
20.0510826133909	-0.01\\
20.1010853131749	-0.01\\
20.151088012959	-0.01\\
20.201090712743	-0.01\\
20.251093412527	-0.01\\
20.301096112311	-0.01\\
20.351098812095	-0.01\\
20.4011015118791	-0.01\\
20.4511042116631	-0.01\\
20.5011069114471	-0.01\\
20.5511096112311	-0.01\\
20.6011123110151	-0.01\\
20.6511150107991	-0.01\\
20.7011177105832	-0.01\\
20.7511204103672	-0.01\\
20.8011231101512	-0.01\\
20.8511258099352	-0.01\\
20.9011285097192	-0.01\\
20.9511312095032	-0.01\\
21.0011339092873	-0.01\\
21.0511366090713	-0.01\\
21.1011393088553	-0.01\\
21.1511420086393	-0.01\\
21.2011447084233	-0.01\\
21.2511474082073	-0.01\\
21.3011501079914	-0.01\\
21.3511528077754	-0.01\\
21.4011555075594	-0.01\\
21.4511582073434	-0.01\\
21.5011609071274	-0.01\\
21.5511636069114	-0.01\\
21.6011663066955	-0.01\\
21.6511690064795	-0.01\\
21.7011717062635	-0.01\\
21.7511744060475	-0.01\\
21.8011771058315	-0.01\\
21.8511798056156	-0.01\\
21.9011825053996	-0.01\\
21.9511852051836	-0.01\\
22.0011879049676	-0.01\\
22.0511906047516	-0.01\\
22.1011933045356	-0.01\\
22.1511960043197	-0.01\\
22.2011987041037	-0.01\\
22.2512014038877	-0.01\\
22.3012041036717	-0.01\\
22.3512068034557	-0.01\\
22.4012095032397	-0.01\\
22.4512122030238	-0.01\\
22.5012149028078	-0.01\\
22.5512176025918	-0.01\\
22.6012203023758	-0.01\\
22.6512230021598	-0.01\\
22.7012257019438	-0.01\\
22.7512284017279	-0.01\\
22.8012311015119	-0.01\\
22.8512338012959	-0.01\\
22.9012365010799	-0.01\\
22.9512392008639	-0.01\\
23.0012419006479	-0.01\\
23.051244600432	-0.01\\
23.101247300216	-0.01\\
23.15125	-0.01\\
23.201252699784	-0.01\\
23.251255399568	-0.01\\
23.3012580993521	-0.01\\
23.3512607991361	-0.01\\
23.4012634989201	-0.01\\
23.4512661987041	-0.01\\
23.5012688984881	-0.01\\
23.5512715982721	-0.01\\
23.6012742980562	-0.01\\
23.6512769978402	-0.01\\
23.7012796976242	-0.01\\
23.7512823974082	-0.01\\
23.8012850971922	-0.01\\
23.8512877969762	-0.01\\
23.9012904967603	-0.01\\
23.9512931965443	-0.01\\
24.0012958963283	-0.01\\
24.0512985961123	-0.01\\
24.1013012958963	-0.01\\
24.1513039956803	-0.01\\
24.2013066954644	-0.01\\
24.2513093952484	-0.01\\
24.3013120950324	-0.01\\
24.3513147948164	-0.01\\
24.4013174946004	-0.01\\
24.4513201943845	-0.01\\
24.5013228941685	-0.01\\
24.5513255939525	-0.01\\
24.6013282937365	-0.01\\
24.6513309935205	-0.01\\
24.7013336933045	-0.01\\
24.7513363930886	-0.01\\
24.8013390928726	-0.01\\
24.8513417926566	-0.01\\
24.9013444924406	-0.01\\
24.9513471922246	-0.01\\
25.0013498920086	-0.01\\
25.0513525917927	-0.01\\
25.1013552915767	-0.01\\
25.1513579913607	-0.01\\
25.2013606911447	-0.01\\
25.2513633909287	-0.01\\
25.3013660907127	-0.01\\
25.3513687904968	-0.01\\
25.4013714902808	-0.01\\
25.4513741900648	-0.01\\
25.5013768898488	-0.01\\
25.5513795896328	-0.01\\
25.6013822894168	-0.01\\
25.6513849892009	-0.01\\
25.7013876889849	-0.01\\
25.7513903887689	-0.01\\
25.8013930885529	-0.01\\
25.8513957883369	-0.01\\
25.901398488121	-0.01\\
25.951401187905	-0.01\\
26.001403887689	-0.01\\
26.051406587473	-0.01\\
26.101409287257	-0.01\\
26.151411987041	-0.01\\
26.2014146868251	-0.01\\
26.2514173866091	-0.01\\
26.3014200863931	-0.01\\
26.3514227861771	-0.01\\
26.4014254859611	-0.01\\
26.4514281857451	-0.01\\
26.5014308855292	-0.01\\
26.5514335853132	-0.01\\
26.6014362850972	-0.01\\
26.6514389848812	-0.01\\
26.7014416846652	-0.01\\
26.7514443844492	-0.01\\
26.8014470842333	-0.01\\
26.8514497840173	-0.01\\
26.9014524838013	-0.01\\
26.9514551835853	-0.01\\
27.0014578833693	-0.01\\
27.0514605831534	-0.01\\
27.1014632829374	-0.01\\
27.1514659827214	-0.01\\
27.2014686825054	-0.01\\
27.2514713822894	-0.01\\
27.3014740820734	-0.01\\
27.3514767818575	-0.01\\
27.4014794816415	-0.01\\
27.4514821814255	-0.01\\
27.5014848812095	-0.01\\
27.5514875809935	-0.01\\
27.6014902807775	-0.01\\
27.6514929805616	-0.01\\
27.7014956803456	-0.01\\
27.7514983801296	-0.01\\
27.8015010799136	-0.01\\
27.8515037796976	-0.01\\
27.9015064794816	-0.01\\
27.9515091792657	-0.01\\
28.0015118790497	-0.01\\
28.0515145788337	-0.01\\
28.1015172786177	-0.01\\
28.1515199784017	-0.01\\
28.2015226781857	-0.01\\
28.2515253779698	-0.01\\
28.3015280777538	-0.01\\
28.3515307775378	-0.01\\
28.4015334773218	-0.01\\
28.4515361771058	-0.01\\
28.5015388768899	-0.01\\
28.5515415766739	-0.01\\
28.6015442764579	-0.01\\
28.6515469762419	-0.01\\
28.7015496760259	-0.01\\
28.7515523758099	-0.01\\
28.801555075594	-0.01\\
28.851557775378	-0.01\\
28.901560475162	-0.01\\
28.951563174946	-0.01\\
29.00156587473	-0.01\\
29.051568574514	-0.01\\
29.1015712742981	-0.01\\
29.1515739740821	-0.01\\
29.2015766738661	-0.01\\
29.2515793736501	-0.01\\
29.3015820734341	-0.01\\
29.3515847732181	-0.01\\
29.4015874730022	-0.01\\
29.4515901727862	-0.01\\
29.5015928725702	-0.01\\
29.5515955723542	-0.01\\
29.6015982721382	-0.01\\
29.6516009719222	-0.01\\
29.7016036717063	-0.01\\
29.7516063714903	-0.01\\
29.8016090712743	-0.01\\
29.8516117710583	-0.01\\
29.9016144708423	-0.01\\
29.9516171706264	-0.01\\
30.0016198704104	-0.01\\
30.0516225701944	-0.01\\
30.1016252699784	-0.01\\
30.1516279697624	-0.01\\
30.2016306695464	-0.01\\
30.2516333693305	-0.01\\
30.3016360691145	-0.01\\
30.3516387688985	-0.01\\
30.4016414686825	-0.01\\
30.4516441684665	-0.01\\
30.5016468682505	-0.01\\
30.5516495680346	-0.01\\
30.6016522678186	-0.01\\
30.6516549676026	-0.01\\
30.7016576673866	-0.01\\
30.7516603671706	-0.01\\
30.8016630669546	-0.01\\
30.8516657667387	-0.01\\
30.9016684665227	-0.01\\
30.9516711663067	-0.01\\
31.0016738660907	-0.01\\
31.0516765658747	-0.01\\
31.1016792656587	-0.01\\
31.1516819654428	-0.01\\
31.2016846652268	-0.01\\
31.2516873650108	-0.01\\
31.3016900647948	-0.01\\
31.3516927645788	-0.01\\
31.4016954643629	-0.01\\
31.4516981641469	-0.01\\
31.5017008639309	-0.01\\
31.5517035637149	-0.01\\
31.6017062634989	-0.01\\
31.6517089632829	-0.01\\
31.701711663067	-0.01\\
31.751714362851	-0.01\\
31.801717062635	-0.01\\
31.851719762419	-0.01\\
31.901722462203	-0.01\\
31.951725161987	-0.01\\
32.0017278617711	-0.01\\
32.0517305615551	-0.01\\
32.1017332613391	-0.01\\
32.1517359611231	-0.01\\
32.2017386609071	-0.01\\
32.2517413606911	-0.01\\
32.3017440604752	-0.01\\
32.3517467602592	-0.01\\
32.4017494600432	-0.01\\
32.4517521598272	-0.01\\
32.5017548596112	-0.01\\
32.5517575593952	-0.01\\
32.6017602591793	-0.01\\
32.6517629589633	-0.01\\
32.7017656587473	-0.01\\
32.7517683585313	-0.01\\
32.8017710583153	-0.01\\
32.8517737580993	-0.01\\
32.9017764578834	-0.01\\
32.9517791576674	-0.01\\
33.0017818574514	-0.01\\
33.0517845572354	-0.01\\
33.1017872570194	-0.01\\
33.1517899568035	-0.01\\
33.2017926565875	-0.01\\
33.2517953563715	-0.01\\
33.3017980561555	-0.01\\
33.3518007559395	-0.01\\
33.4018034557235	-0.01\\
33.4518061555076	-0.01\\
33.5018088552916	-0.01\\
33.5518115550756	-0.01\\
33.6018142548596	-0.01\\
33.6518169546436	-0.01\\
33.7018196544277	-0.01\\
33.7518223542117	-0.01\\
33.8018250539957	-0.01\\
33.8518277537797	-0.01\\
33.9018304535637	-0.01\\
33.9518331533477	-0.01\\
34.0018358531318	-0.01\\
34.0518385529158	-0.01\\
34.1018412526998	-0.01\\
34.1518439524838	-0.01\\
34.2018466522678	-0.01\\
34.2518493520518	-0.01\\
34.3018520518359	-0.01\\
34.3518547516199	-0.01\\
34.4018574514039	-0.01\\
34.4518601511879	-0.01\\
34.5018628509719	-0.01\\
34.5518655507559	-0.01\\
34.60186825054	-0.01\\
34.651870950324	-0.01\\
34.701873650108	-0.01\\
34.751876349892	-0.01\\
34.801879049676	-0.01\\
34.85188174946	-0.01\\
34.9018844492441	-0.01\\
34.9518871490281	-0.01\\
35.0018898488121	-0.01\\
35.0518925485961	-0.01\\
35.1018952483801	-0.01\\
35.1518979481641	-0.01\\
35.2019006479482	-0.01\\
35.2519033477322	-0.01\\
35.3019060475162	-0.01\\
35.3519087473002	-0.01\\
35.4019114470842	-0.01\\
35.4519141468683	-0.01\\
35.5019168466523	-0.01\\
35.5519195464363	-0.01\\
35.6019222462203	-0.01\\
35.6519249460043	-0.01\\
35.7019276457883	-0.01\\
35.7519303455724	-0.01\\
35.8019330453564	-0.01\\
35.8519357451404	-0.01\\
35.9019384449244	-0.01\\
35.9519411447084	-0.01\\
36.0019438444924	-0.01\\
36.0519465442765	-0.01\\
36.1019492440605	-0.01\\
36.1519519438445	-0.01\\
36.2019546436285	-0.01\\
36.2519573434125	-0.01\\
36.3019600431965	-0.01\\
36.3519627429806	-0.01\\
36.4019654427646	-0.01\\
36.4519681425486	-0.01\\
36.5019708423326	-0.01\\
36.5519735421166	-0.01\\
36.6019762419006	-0.01\\
36.6519789416847	-0.01\\
36.7019816414687	-0.01\\
36.7519843412527	-0.01\\
36.8019870410367	-0.01\\
36.8519897408207	-0.01\\
36.9019924406048	-0.01\\
36.9519951403888	-0.01\\
37.0019978401728	-0.01\\
37.0520005399568	-0.01\\
37.1020032397408	-0.01\\
37.1520059395248	-0.01\\
37.2020086393089	-0.01\\
37.2520113390929	-0.01\\
37.3020140388769	-0.01\\
37.3520167386609	-0.01\\
37.4020194384449	-0.01\\
37.4520221382289	-0.01\\
37.502024838013	-0.01\\
37.552027537797	-0.01\\
37.602030237581	-0.01\\
37.652032937365	-0.01\\
37.702035637149	-0.01\\
37.752038336933	-0.01\\
37.8020410367171	-0.01\\
37.8520437365011	-0.01\\
37.9020464362851	-0.01\\
37.9520491360691	-0.01\\
38.0020518358531	-0.01\\
38.0520545356372	-0.01\\
38.1020572354212	-0.01\\
38.1520599352052	-0.01\\
38.2020626349892	-0.01\\
38.2520653347732	-0.01\\
38.3020680345572	-0.01\\
38.3520707343413	-0.01\\
38.4020734341253	-0.01\\
38.4520761339093	-0.01\\
38.5020788336933	-0.01\\
38.5520815334773	-0.01\\
38.6020842332613	-0.01\\
38.6520869330454	-0.01\\
38.7020896328294	-0.01\\
38.7520923326134	-0.01\\
38.8020950323974	-0.01\\
38.8520977321814	-0.01\\
38.9021004319654	-0.01\\
38.9521031317495	-0.01\\
39.0021058315335	-0.01\\
39.0521085313175	-0.01\\
39.1021112311015	-0.01\\
39.1521139308855	-0.01\\
39.2021166306696	-0.01\\
39.2521193304536	-0.01\\
39.3021220302376	-0.01\\
39.3521247300216	-0.01\\
39.4021274298056	-0.01\\
39.4521301295896	-0.01\\
39.5021328293737	-0.01\\
39.5521355291577	-0.01\\
39.6021382289417	-0.01\\
39.6521409287257	-0.01\\
39.7021436285097	-0.01\\
39.7521463282937	-0.01\\
39.8021490280778	-0.01\\
39.8521517278618	-0.01\\
39.9021544276458	-0.01\\
39.9521571274298	-0.01\\
40.0021598272138	-0.01\\
40.0521625269978	-0.01\\
40.1021652267819	-0.01\\
40.1521679265659	-0.01\\
40.2021706263499	-0.01\\
40.2521733261339	-0.01\\
40.3021760259179	-0.01\\
40.3521787257019	-0.01\\
40.402181425486	-0.01\\
40.45218412527	-0.01\\
40.502186825054	-0.01\\
40.552189524838	-0.01\\
40.602192224622	-0.01\\
40.652194924406	-0.01\\
40.7021976241901	-0.01\\
40.7522003239741	-0.01\\
40.8022030237581	-0.01\\
40.8522057235421	-0.01\\
40.9022084233261	-0.01\\
40.9522111231102	-0.01\\
41.0022138228942	-0.01\\
41.0522165226782	-0.01\\
41.1022192224622	-0.01\\
41.1522219222462	-0.01\\
41.2022246220302	-0.01\\
41.2522273218143	-0.01\\
41.3022300215983	-0.01\\
41.3522327213823	-0.01\\
41.4022354211663	-0.01\\
41.4522381209503	-0.01\\
41.5022408207343	-0.01\\
41.5522435205184	-0.01\\
41.6022462203024	-0.01\\
41.6522489200864	-0.01\\
41.7022516198704	-0.01\\
41.7522543196544	-0.01\\
41.8022570194385	-0.01\\
41.8522597192225	-0.01\\
41.9022624190065	-0.01\\
41.9522651187905	-0.01\\
42.0022678185745	-0.01\\
42.0522705183585	-0.01\\
42.1022732181426	-0.01\\
42.1522759179266	-0.01\\
42.2022786177106	-0.01\\
42.2522813174946	-0.01\\
42.3022840172786	-0.01\\
42.3522867170626	-0.01\\
42.4022894168467	-0.01\\
42.4522921166307	-0.01\\
42.5022948164147	-0.01\\
42.5522975161987	-0.01\\
42.6023002159827	-0.01\\
42.6523029157667	-0.01\\
42.7023056155508	-0.01\\
42.7523083153348	-0.01\\
42.8023110151188	-0.01\\
42.8523137149028	-0.01\\
42.9023164146868	-0.01\\
42.9523191144708	-0.01\\
43.0023218142549	-0.01\\
43.0523245140389	-0.01\\
43.1023272138229	-0.01\\
43.1523299136069	-0.01\\
43.2023326133909	-0.01\\
43.2523353131749	-0.01\\
43.302338012959	-0.01\\
43.352340712743	-0.01\\
43.402343412527	-0.01\\
43.452346112311	-0.01\\
43.502348812095	-0.01\\
43.5523515118791	-0.01\\
43.6023542116631	-0.01\\
43.6523569114471	-0.01\\
43.7023596112311	-0.01\\
43.7523623110151	-0.01\\
43.8023650107991	-0.01\\
43.8523677105832	-0.01\\
43.9023704103672	-0.01\\
43.9523731101512	-0.01\\
44.0023758099352	-0.01\\
44.0523785097192	-0.01\\
44.1023812095032	-0.01\\
44.1523839092873	-0.01\\
44.2023866090713	-0.01\\
44.2523893088553	-0.01\\
44.3023920086393	-0.01\\
44.3523947084233	-0.01\\
44.4023974082073	-0.01\\
44.4524001079914	-0.01\\
44.5024028077754	-0.01\\
44.5524055075594	-0.01\\
44.6024082073434	-0.01\\
44.6524109071274	-0.01\\
44.7024136069114	-0.01\\
44.7524163066955	-0.01\\
44.8024190064795	-0.01\\
44.8524217062635	-0.01\\
44.9024244060475	-0.01\\
44.9524271058315	-0.01\\
45.0024298056155	-0.01\\
45.0524325053996	-0.01\\
45.1024352051836	-0.01\\
45.1524379049676	-0.01\\
45.2024406047516	-0.01\\
45.2524433045356	-0.01\\
45.3024460043197	-0.01\\
45.3524487041037	-0.01\\
45.4024514038877	-0.01\\
45.4524541036717	-0.01\\
45.5024568034557	-0.01\\
45.5524595032397	-0.01\\
45.6024622030238	-0.01\\
45.6524649028078	-0.01\\
45.7024676025918	-0.01\\
45.7524703023758	-0.01\\
45.8024730021598	-0.01\\
45.8524757019439	-0.01\\
45.9024784017279	-0.01\\
45.9524811015119	-0.01\\
46.0024838012959	-0.01\\
46.0524865010799	-0.01\\
46.1024892008639	-0.01\\
46.152491900648	-0.01\\
46.202494600432	-0.01\\
46.252497300216	-0.01\\
46.3025	-0.01\\
46.352502699784	-0.01\\
46.402505399568	-0.01\\
46.4525080993521	-0.01\\
46.5025107991361	-0.01\\
46.5525134989201	-0.01\\
46.6025161987041	-0.01\\
46.6525188984881	-0.01\\
46.7025215982721	-0.01\\
46.7525242980562	-0.01\\
46.8025269978402	-0.01\\
46.8525296976242	-0.01\\
46.9025323974082	-0.01\\
46.9525350971922	-0.01\\
47.0025377969762	-0.01\\
47.0525404967603	-0.01\\
47.1025431965443	-0.01\\
47.1525458963283	-0.01\\
47.2025485961123	-0.01\\
47.2525512958963	-0.01\\
47.3025539956803	-0.01\\
47.3525566954644	-0.01\\
47.4025593952484	-0.01\\
47.4525620950324	-0.01\\
47.5025647948164	-0.01\\
47.5525674946004	-0.01\\
47.6025701943845	-0.01\\
47.6525728941685	-0.01\\
47.7025755939525	-0.01\\
47.7525782937365	-0.01\\
47.8025809935205	-0.01\\
47.8525836933045	-0.01\\
47.9025863930886	-0.01\\
47.9525890928726	-0.01\\
48.0025917926566	-0.01\\
48.0525944924406	-0.01\\
48.1025971922246	-0.01\\
48.1525998920086	-0.01\\
48.2026025917927	-0.01\\
48.2526052915767	-0.01\\
48.3026079913607	-0.01\\
48.3526106911447	-0.01\\
48.4026133909287	-0.01\\
48.4526160907127	-0.01\\
48.5026187904968	-0.01\\
48.5526214902808	-0.01\\
48.6026241900648	-0.01\\
48.6526268898488	-0.01\\
48.7026295896328	-0.01\\
48.7526322894168	-0.01\\
48.8026349892009	-0.01\\
48.8526376889849	-0.01\\
48.9026403887689	-0.01\\
48.9526430885529	-0.01\\
49.0026457883369	-0.01\\
49.052648488121	-0.01\\
49.102651187905	-0.01\\
49.152653887689	-0.01\\
49.202656587473	-0.01\\
49.252659287257	-0.01\\
49.302661987041	-0.01\\
49.3526646868251	-0.01\\
49.4026673866091	-0.01\\
49.4526700863931	-0.01\\
49.5026727861771	-0.01\\
49.5526754859611	-0.01\\
49.6026781857451	-0.01\\
49.6526808855292	-0.01\\
49.7026835853132	-0.01\\
49.7526862850972	-0.01\\
49.8026889848812	-0.01\\
49.8526916846652	-0.01\\
49.9026943844492	-0.01\\
49.9526970842333	-0.01\\
50.0026997840173	-0.01\\
50.0527024838013	-0.01\\
50.1027051835853	-0.01\\
50.1527078833693	-0.01\\
50.2027105831534	-0.01\\
50.2527132829374	-0.01\\
50.3027159827214	-0.01\\
50.3527186825054	-0.01\\
50.4027213822894	-0.01\\
50.4527240820734	-0.01\\
50.5027267818575	-0.01\\
50.5527294816415	-0.01\\
50.6027321814255	-0.01\\
50.6527348812095	-0.01\\
50.7027375809935	-0.01\\
50.7527402807775	-0.01\\
50.8027429805616	-0.01\\
50.8527456803456	-0.01\\
50.9027483801296	-0.01\\
50.9527510799136	-0.01\\
51.0027537796976	-0.01\\
51.0527564794816	-0.01\\
51.1027591792657	-0.01\\
51.1527618790497	-0.01\\
51.2027645788337	-0.01\\
51.2527672786177	-0.01\\
51.3027699784017	-0.01\\
51.3527726781858	-0.01\\
51.4027753779698	-0.01\\
51.4527780777538	-0.01\\
51.5027807775378	-0.01\\
51.5527834773218	-0.01\\
51.6027861771058	-0.01\\
51.6527888768899	-0.01\\
51.7027915766739	-0.01\\
51.7527942764579	-0.01\\
51.8027969762419	-0.01\\
51.8527996760259	-0.01\\
51.9028023758099	-0.01\\
51.952805075594	-0.01\\
52.002807775378	-0.01\\
52.052810475162	-0.01\\
52.102813174946	-0.01\\
52.15281587473	-0.01\\
52.202818574514	-0.01\\
52.2528212742981	-0.01\\
52.3028239740821	-0.01\\
52.3528266738661	-0.01\\
52.4028293736501	-0.01\\
52.4528320734341	-0.01\\
52.5028347732181	-0.01\\
52.5528374730022	-0.01\\
52.6028401727862	-0.01\\
52.6528428725702	-0.01\\
52.7028455723542	-0.01\\
52.7528482721382	-0.01\\
52.8028509719222	-0.01\\
52.8528536717063	-0.01\\
52.9028563714903	-0.01\\
52.9528590712743	-0.01\\
53.0028617710583	-0.01\\
53.0528644708423	-0.01\\
53.1028671706264	-0.01\\
53.1528698704104	-0.01\\
53.2028725701944	-0.01\\
53.2528752699784	-0.01\\
53.3028779697624	-0.01\\
53.3528806695464	-0.01\\
53.4028833693305	-0.01\\
53.4528860691145	-0.01\\
53.5028887688985	-0.01\\
53.5528914686825	-0.01\\
53.6028941684665	-0.01\\
53.6528968682505	-0.01\\
53.7028995680346	-0.01\\
53.7529022678186	-0.01\\
53.8029049676026	-0.01\\
53.8529076673866	-0.01\\
53.9029103671706	-0.01\\
53.9529130669547	-0.01\\
54.0029157667387	-0.01\\
54.0529184665227	-0.01\\
54.1029211663067	-0.01\\
54.1529238660907	-0.01\\
54.2029265658747	-0.01\\
54.2529292656588	-0.01\\
54.3029319654428	-0.01\\
54.3529346652268	-0.01\\
54.4029373650108	-0.01\\
54.4529400647948	-0.01\\
54.5029427645788	-0.01\\
54.5529454643629	-0.01\\
54.6029481641469	-0.01\\
54.6529508639309	-0.01\\
54.7029535637149	-0.01\\
54.7529562634989	-0.01\\
54.8029589632829	-0.01\\
54.852961663067	-0.01\\
54.902964362851	-0.01\\
54.952967062635	-0.01\\
55.002969762419	-0.01\\
55.052972462203	-0.01\\
55.102975161987	-0.01\\
55.1529778617711	-0.01\\
55.2029805615551	-0.01\\
55.2529832613391	-0.01\\
55.3029859611231	-0.01\\
55.3529886609071	-0.01\\
55.4029913606911	-0.01\\
55.4529940604752	-0.01\\
55.5029967602592	-0.01\\
55.5529994600432	-0.01\\
55.6030021598272	-0.01\\
55.6530048596112	-0.01\\
55.7030075593953	-0.01\\
55.7530102591793	-0.01\\
55.8030129589633	-0.01\\
55.8530156587473	-0.01\\
55.9030183585313	-0.01\\
55.9530210583153	-0.01\\
56.0030237580994	-0.01\\
56.0530264578834	-0.01\\
56.1030291576674	-0.01\\
56.1530318574514	-0.01\\
56.2030345572354	-0.01\\
56.2530372570194	-0.01\\
56.3030399568035	-0.01\\
56.3530426565875	-0.01\\
56.4030453563715	-0.01\\
56.4530480561555	-0.01\\
56.5030507559395	-0.01\\
56.5530534557235	-0.01\\
56.6030561555076	-0.01\\
56.6530588552916	-0.01\\
56.7030615550756	-0.01\\
56.7530642548596	-0.01\\
56.8030669546436	-0.01\\
56.8530696544276	-0.01\\
56.9030723542117	-0.01\\
56.9530750539957	-0.01\\
57.0030777537797	-0.01\\
57.0530804535637	-0.01\\
57.1030831533477	-0.01\\
57.1530858531318	-0.01\\
57.2030885529158	-0.01\\
57.2530912526998	-0.01\\
57.3030939524838	-0.01\\
57.3530966522678	-0.01\\
57.4030993520518	-0.01\\
57.4531020518358	-0.01\\
57.5031047516199	-0.01\\
57.5531074514039	-0.01\\
57.6031101511879	-0.01\\
57.6531128509719	-0.01\\
57.7031155507559	-0.01\\
57.75311825054	-0.01\\
57.803120950324	-0.01\\
57.853123650108	-0.01\\
57.903126349892	-0.01\\
57.953129049676	-0.01\\
58.00313174946	-0.01\\
58.0531344492441	-0.01\\
58.1031371490281	-0.01\\
58.1531398488121	-0.01\\
58.2031425485961	-0.01\\
58.2531452483801	-0.01\\
58.3031479481642	-0.01\\
58.3531506479482	-0.01\\
58.4031533477322	-0.01\\
58.4531560475162	-0.01\\
58.5031587473002	-0.01\\
58.5531614470842	-0.01\\
58.6031641468683	-0.01\\
58.6531668466523	-0.01\\
58.7031695464363	-0.01\\
58.7531722462203	-0.01\\
58.8031749460043	-0.01\\
58.8531776457883	-0.01\\
58.9031803455724	-0.01\\
58.9531830453564	-0.01\\
59.0031857451404	-0.01\\
59.0531884449244	-0.01\\
59.1031911447084	-0.01\\
59.1531938444924	-0.01\\
59.2031965442765	-0.01\\
59.2531992440605	-0.01\\
59.3032019438445	-0.01\\
59.3532046436285	-0.01\\
59.4032073434125	-0.01\\
59.4532100431965	-0.01\\
59.5032127429806	-0.01\\
59.5532154427646	-0.01\\
59.6032181425486	-0.01\\
59.6532208423326	-0.01\\
59.7032235421166	-0.01\\
59.7532262419006	-0.01\\
59.8032289416847	-0.01\\
59.8532316414687	-0.01\\
59.9032343412527	-0.01\\
59.9532370410367	-0.01\\
60.0032397408207	-0.01\\
60.0532424406048	-0.01\\
60.1032451403888	-0.01\\
60.1532478401728	-0.01\\
60.2032505399568	-0.01\\
60.2532532397408	-0.01\\
60.3032559395248	-0.01\\
60.3532586393089	-0.01\\
60.4032613390929	-0.01\\
60.4532640388769	-0.01\\
60.5032667386609	-0.01\\
60.5532694384449	-0.01\\
60.6032721382289	-0.01\\
60.653274838013	-0.01\\
60.703277537797	-0.01\\
60.753280237581	-0.01\\
60.803282937365	-0.01\\
60.853285637149	-0.01\\
60.9032883369331	-0.01\\
60.9532910367171	-0.01\\
61.0032937365011	-0.01\\
61.0532964362851	-0.01\\
61.1032991360691	-0.01\\
61.1533018358531	-0.01\\
61.2033045356371	-0.01\\
61.2533072354212	-0.01\\
61.3033099352052	-0.01\\
61.3533126349892	-0.01\\
61.4033153347732	-0.01\\
61.4533180345572	-0.01\\
61.5033207343413	-0.01\\
61.5533234341253	-0.01\\
61.6033261339093	-0.01\\
61.6533288336933	-0.01\\
61.7033315334773	-0.01\\
61.7533342332613	-0.01\\
61.8033369330454	-0.01\\
61.8533396328294	-0.01\\
61.9033423326134	-0.01\\
61.9533450323974	-0.01\\
62.0033477321814	-0.01\\
62.0533504319654	-0.01\\
62.1033531317495	-0.01\\
62.1533558315335	-0.01\\
62.2033585313175	-0.01\\
62.2533612311015	-0.01\\
62.3033639308855	-0.01\\
62.3533666306695	-0.01\\
62.4033693304536	-0.01\\
62.4533720302376	-0.01\\
62.5033747300216	-0.01\\
62.5483771598272	0.19\\
62.5983798596112	0.19\\
62.6483825593953	0.19\\
62.6983852591793	0.19\\
62.7483879589633	0.19\\
62.7983906587473	0.19\\
62.8483933585313	0.19\\
62.8983960583153	0.19\\
62.9483987580994	0.19\\
62.9984014578834	0.19\\
63.0484041576674	0.19\\
63.0984068574514	0.19\\
63.1484095572354	0.19\\
63.1984122570194	0.19\\
63.2484149568035	0.19\\
63.2984176565875	0.19\\
63.3484203563715	0.19\\
63.3984230561555	0.19\\
63.4484257559395	0.19\\
63.4984284557235	0.19\\
63.5484311555076	0.19\\
63.5984338552916	0.19\\
63.6484365550756	0.19\\
63.6984392548596	0.19\\
63.7484419546436	0.19\\
63.7984446544276	0.19\\
63.8484473542117	0.19\\
63.8984500539957	0.19\\
63.9484527537797	0.19\\
63.9984554535637	0.19\\
64.0484581533477	0.19\\
64.0984608531318	0.19\\
64.1484635529158	0.19\\
64.1984662526998	0.19\\
64.2484689524838	0.19\\
64.2984716522678	0.19\\
64.3484743520518	0.19\\
64.3984770518359	0.19\\
64.4484797516199	0.19\\
64.4984824514039	0.19\\
64.5484851511879	0.19\\
64.5984878509719	0.19\\
64.648490550756	0.19\\
64.69849325054	0.19\\
64.748495950324	0.19\\
64.798498650108	0.19\\
64.848501349892	0.19\\
64.898504049676	0.19\\
64.94850674946	0.19\\
64.9985094492441	0.19\\
65.0485121490281	0.19\\
65.0985148488121	0.19\\
65.1485175485961	0.19\\
65.1985202483801	0.19\\
65.2485229481642	0.19\\
65.2985256479482	0.19\\
65.3485283477322	0.19\\
65.3985310475162	0.19\\
65.4485337473002	0.19\\
65.4985364470842	0.19\\
65.5485391468683	0.19\\
65.5985418466523	0.19\\
65.6485445464363	0.19\\
65.6985472462203	0.19\\
65.7485499460043	0.19\\
65.7985526457883	0.19\\
65.8485553455724	0.19\\
65.8985580453564	0.19\\
65.9485607451404	0.19\\
65.9985634449244	0.19\\
66.0485661447084	0.19\\
66.0985688444924	0.19\\
66.1485715442765	0.19\\
66.1985742440605	0.19\\
66.2485769438445	0.19\\
66.2985796436285	0.19\\
66.3485823434125	0.19\\
66.3985850431965	0.19\\
66.4485877429806	0.19\\
66.4985904427646	0.19\\
66.5485931425486	0.19\\
66.5985958423326	0.19\\
66.6485985421166	0.19\\
66.6986012419006	0.19\\
66.7486039416847	0.19\\
66.7986066414687	0.19\\
66.8486093412527	0.19\\
66.8986120410367	0.19\\
66.9486147408207	0.19\\
66.9986174406047	0.19\\
67.0486201403888	0.19\\
67.0986228401728	0.19\\
67.1486255399568	0.19\\
67.1986282397408	0.19\\
67.2486309395248	0.19\\
67.2986336393089	0.19\\
67.3486363390929	0.19\\
67.3986390388769	0.19\\
67.4486417386609	0.19\\
67.4986444384449	0.19\\
67.5486471382289	0.19\\
67.598649838013	0.19\\
67.648652537797	0.19\\
67.698655237581	0.19\\
67.748657937365	0.19\\
67.798660637149	0.19\\
67.8486633369331	0.19\\
67.8986660367171	0.19\\
67.9486687365011	0.19\\
67.9986714362851	0.19\\
68.0486741360691	0.19\\
68.0986768358531	0.19\\
68.1486795356372	0.19\\
68.1986822354212	0.19\\
68.2486849352052	0.19\\
68.2986876349892	0.19\\
68.3486903347732	0.19\\
68.3986930345572	0.19\\
68.4486957343413	0.19\\
68.4986984341253	0.19\\
68.5487011339093	0.19\\
68.5987038336933	0.19\\
68.6487065334773	0.19\\
68.6987092332613	0.19\\
68.7487119330454	0.19\\
68.7987146328294	0.19\\
68.8487173326134	0.19\\
68.8987200323974	0.19\\
68.9487227321814	0.19\\
68.9987254319654	0.19\\
69.0487281317495	0.19\\
69.0987308315335	0.19\\
69.1487335313175	0.19\\
69.1987362311015	0.19\\
69.2487389308855	0.19\\
69.2987416306696	0.19\\
69.3487443304536	0.19\\
69.3987470302376	0.19\\
69.4487497300216	0.19\\
69.4987524298056	0.19\\
69.5487551295896	0.19\\
69.5987578293737	0.19\\
69.6487605291577	0.19\\
69.6987632289417	0.19\\
69.7487659287257	0.19\\
69.7987686285097	0.19\\
69.8487713282937	0.19\\
69.8987740280778	0.19\\
69.9487767278618	0.19\\
69.9987794276458	0.19\\
70.0487821274298	0.19\\
70.0987848272138	0.19\\
70.1487875269979	0.19\\
70.1987902267819	0.19\\
70.2487929265659	0.19\\
70.2987956263499	0.19\\
70.3487983261339	0.19\\
70.3988010259179	0.19\\
70.4488037257019	0.19\\
70.498806425486	0.19\\
70.54880912527	0.19\\
70.598811825054	0.19\\
70.648814524838	0.19\\
70.698817224622	0.19\\
70.7488199244061	0.19\\
70.7988226241901	0.19\\
70.8488253239741	0.19\\
70.8988280237581	0.19\\
70.9488307235421	0.19\\
70.9988334233261	0.19\\
71.0488361231101	0.19\\
71.0988388228942	0.19\\
71.1488415226782	0.19\\
71.1988442224622	0.19\\
71.2488469222462	0.19\\
71.2988496220302	0.19\\
71.3488523218143	0.19\\
71.3988550215983	0.19\\
71.4488577213823	0.19\\
71.4988604211663	0.19\\
71.5488631209503	0.19\\
71.5988658207343	0.19\\
71.6488685205184	0.19\\
71.6988712203024	0.19\\
71.7488739200864	0.19\\
71.7988766198704	0.19\\
71.8488793196544	0.19\\
71.8988820194385	0.19\\
71.9488847192225	0.19\\
71.9988874190065	0.19\\
72.0488901187905	0.19\\
72.0988928185745	0.19\\
72.1488955183585	0.19\\
72.1988982181425	0.19\\
72.2489009179266	0.19\\
72.2989036177106	0.19\\
72.3489063174946	0.19\\
72.3989090172786	0.19\\
72.4489117170626	0.19\\
72.4989144168467	0.19\\
72.5489171166307	0.19\\
72.5989198164147	0.19\\
72.6489225161987	0.19\\
72.6989252159827	0.19\\
72.7489279157667	0.19\\
72.7989306155508	0.19\\
72.8489333153348	0.19\\
72.8989360151188	0.19\\
72.9489387149028	0.19\\
72.9989414146868	0.19\\
73.0489441144708	0.19\\
73.0989468142549	0.19\\
73.1489495140389	0.19\\
73.1989522138229	0.19\\
73.2489549136069	0.19\\
73.2989576133909	0.19\\
73.348960313175	0.19\\
73.398963012959	0.19\\
73.448965712743	0.19\\
73.498968412527	0.19\\
73.548971112311	0.19\\
73.598973812095	0.19\\
73.6489765118791	0.19\\
73.6989792116631	0.19\\
73.7489819114471	0.19\\
73.7989846112311	0.19\\
73.8489873110151	0.19\\
73.8989900107991	0.19\\
73.9489927105832	0.19\\
73.9989954103672	0.19\\
74.0489981101512	0.19\\
74.0990008099352	0.19\\
74.1490035097192	0.19\\
74.1990062095032	0.19\\
74.2490089092873	0.19\\
74.2990116090713	0.19\\
74.3490143088553	0.19\\
74.3990170086393	0.19\\
74.4490197084233	0.19\\
74.4990224082073	0.19\\
74.5490251079914	0.19\\
74.5990278077754	0.19\\
74.6490305075594	0.19\\
74.6990332073434	0.19\\
74.7490359071274	0.19\\
74.7990386069115	0.19\\
74.8490413066955	0.19\\
74.8990440064795	0.19\\
74.9490467062635	0.19\\
74.9990494060475	0.19\\
75.0490521058315	0.19\\
75.0990548056156	0.19\\
75.1490575053996	0.19\\
75.1990602051836	0.19\\
75.2490629049676	0.19\\
75.2990656047516	0.19\\
75.3490683045356	0.19\\
75.3990710043197	0.19\\
75.4490737041037	0.19\\
75.4990764038877	0.19\\
75.5490791036717	0.19\\
75.5990818034557	0.19\\
75.6490845032397	0.19\\
75.6990872030238	0.19\\
75.7490899028078	0.19\\
75.7990926025918	0.19\\
75.8490953023758	0.19\\
75.8990980021598	0.19\\
75.9491007019438	0.19\\
75.9991034017279	0.19\\
76.0491061015119	0.19\\
76.0991088012959	0.19\\
76.1491115010799	0.19\\
76.1991142008639	0.19\\
76.2491169006479	0.19\\
76.299119600432	0.19\\
76.349122300216	0.19\\
76.399125	0.19\\
76.449127699784	0.19\\
76.499130399568	0.19\\
76.5491330993521	0.19\\
76.5991357991361	0.19\\
76.6491384989201	0.19\\
76.6991411987041	0.19\\
76.7491438984881	0.19\\
76.7991465982721	0.19\\
76.8491492980562	0.19\\
76.8991519978402	0.19\\
76.9491546976242	0.19\\
76.9991573974082	0.19\\
77.0491600971922	0.19\\
77.0991627969762	0.19\\
77.1491654967603	0.19\\
77.1991681965443	0.19\\
77.2491708963283	0.19\\
77.2991735961123	0.19\\
77.3491762958963	0.19\\
77.3991789956804	0.19\\
77.4491816954644	0.19\\
77.4991843952484	0.19\\
77.5491870950324	0.19\\
77.5991897948164	0.19\\
77.6491924946004	0.19\\
77.6991951943844	0.19\\
77.7491978941685	0.19\\
77.7992005939525	0.19\\
77.8492032937365	0.19\\
77.8992059935205	0.19\\
77.9492086933045	0.19\\
77.9992113930886	0.19\\
78.0492140928726	0.19\\
78.0992167926566	0.19\\
78.1492194924406	0.19\\
78.1992221922246	0.19\\
78.2492248920086	0.19\\
78.2992275917927	0.19\\
78.3492302915767	0.19\\
78.3992329913607	0.19\\
78.4492356911447	0.19\\
78.4992383909287	0.19\\
78.5492410907127	0.19\\
78.5992437904968	0.19\\
78.6492464902808	0.19\\
78.6992491900648	0.19\\
78.7492518898488	0.19\\
78.7992545896328	0.19\\
78.8492572894169	0.19\\
78.8992599892009	0.19\\
78.9492626889849	0.19\\
78.9992653887689	0.19\\
79.0492680885529	0.19\\
79.0992707883369	0.19\\
79.149273488121	0.19\\
79.199276187905	0.19\\
79.249278887689	0.19\\
79.299281587473	0.19\\
79.349284287257	0.19\\
79.3992869870411	0.19\\
79.4492896868251	0.19\\
79.4992923866091	0.19\\
79.5492950863931	0.19\\
79.5992977861771	0.19\\
79.6493004859611	0.19\\
79.6993031857451	0.19\\
79.7493058855292	0.19\\
79.7993085853132	0.19\\
79.8493112850972	0.19\\
79.8993139848812	0.19\\
79.9493166846652	0.19\\
79.9993193844493	0.19\\
80.0493220842333	0.19\\
80.0993247840173	0.19\\
80.1493274838013	0.19\\
80.1993301835853	0.19\\
80.2493328833693	0.19\\
80.2993355831534	0.19\\
80.3493382829374	0.19\\
80.3993409827214	0.19\\
80.4493436825054	0.19\\
80.4993463822894	0.19\\
80.5493490820734	0.19\\
80.5993517818575	0.19\\
80.6493544816415	0.19\\
80.6993571814255	0.19\\
80.7493598812095	0.19\\
80.7993625809935	0.19\\
80.8493652807775	0.19\\
80.8993679805616	0.19\\
80.9493706803456	0.19\\
80.9993733801296	0.19\\
81.0493760799136	0.19\\
81.0993787796976	0.19\\
81.1493814794816	0.19\\
81.1993841792657	0.19\\
81.2493868790497	0.19\\
81.2993895788337	0.19\\
81.3493922786177	0.19\\
81.3993949784017	0.19\\
81.4493976781857	0.19\\
81.4994003779698	0.19\\
81.5494030777538	0.19\\
81.5994057775378	0.19\\
81.6494084773218	0.19\\
81.6994111771058	0.19\\
81.7494138768898	0.19\\
81.7994165766739	0.19\\
81.8494192764579	0.19\\
81.8994219762419	0.19\\
81.9494246760259	0.19\\
81.9994273758099	0.19\\
82.049430075594	0.19\\
82.099432775378	0.19\\
82.149435475162	0.19\\
82.199438174946	0.19\\
82.24944087473	0.19\\
82.299443574514	0.19\\
82.3494462742981	0.19\\
82.3994489740821	0.19\\
82.4494516738661	0.19\\
82.4994543736501	0.19\\
82.5494570734341	0.19\\
82.5994597732181	0.19\\
82.6294613930885	-0.01\\
82.6794640928726	-0.01\\
82.7294667926566	-0.01\\
82.7794694924406	-0.01\\
82.8294721922246	-0.01\\
82.8794748920086	-0.01\\
82.9294775917927	-0.01\\
82.9794802915767	-0.01\\
83.0294829913607	-0.01\\
83.0794856911447	-0.01\\
83.1294883909287	-0.01\\
83.1794910907127	-0.01\\
83.2294937904968	-0.01\\
83.2794964902808	-0.01\\
83.3294991900648	-0.01\\
83.3795018898488	-0.01\\
83.4295045896328	-0.01\\
83.4795072894169	-0.01\\
83.5295099892009	-0.01\\
83.5795126889849	-0.01\\
83.6295153887689	-0.01\\
83.6795180885529	-0.01\\
83.7295207883369	-0.01\\
83.779523488121	-0.01\\
83.829526187905	-0.01\\
83.879528887689	-0.01\\
83.929531587473	-0.01\\
83.979534287257	-0.01\\
84.029536987041	-0.01\\
84.0795396868251	-0.01\\
84.1295423866091	-0.01\\
84.1795450863931	-0.01\\
84.2295477861771	-0.01\\
84.2795504859611	-0.01\\
84.3295531857451	-0.01\\
84.3795558855292	-0.01\\
84.4295585853132	-0.01\\
84.4795612850972	-0.01\\
84.5295639848812	-0.01\\
84.5795666846652	-0.01\\
84.6295693844492	-0.01\\
84.6795720842333	-0.01\\
84.7295747840173	-0.01\\
84.7795774838013	-0.01\\
84.8295801835853	-0.01\\
84.8795828833693	-0.01\\
84.9295855831534	-0.01\\
84.9795882829374	-0.01\\
85.0295909827214	-0.01\\
85.0795936825054	-0.01\\
85.1295963822894	-0.01\\
85.1795990820734	-0.01\\
85.2296017818575	-0.01\\
85.2796044816415	-0.01\\
85.3296071814255	-0.01\\
85.3796098812095	-0.01\\
85.4296125809935	-0.01\\
85.4796152807775	-0.01\\
85.5296179805616	-0.01\\
85.5796206803456	-0.01\\
85.6296233801296	-0.01\\
85.6796260799136	-0.01\\
85.7296287796976	-0.01\\
85.7796314794817	-0.01\\
85.8296341792657	-0.01\\
85.8796368790497	-0.01\\
85.9296395788337	-0.01\\
85.9796422786177	-0.01\\
86.0296449784017	-0.01\\
86.0796476781857	-0.01\\
86.1296503779698	-0.01\\
86.1796530777538	-0.01\\
86.2296557775378	-0.01\\
86.2796584773218	-0.01\\
86.3296611771058	-0.01\\
86.3796638768899	-0.01\\
86.4296665766739	-0.01\\
86.4796692764579	-0.01\\
86.5296719762419	-0.01\\
86.5796746760259	-0.01\\
86.6296773758099	-0.01\\
86.679680075594	-0.01\\
86.729682775378	-0.01\\
86.779685475162	-0.01\\
86.829688174946	-0.01\\
86.87969087473	-0.01\\
86.929693574514	-0.01\\
86.9796962742981	-0.01\\
87.0296989740821	-0.01\\
87.0797016738661	-0.01\\
87.1297043736501	-0.01\\
87.1797070734341	-0.01\\
87.2297097732181	-0.01\\
87.2797124730022	-0.01\\
87.3297151727862	-0.01\\
87.3797178725702	-0.01\\
87.4297205723542	-0.01\\
87.4797232721382	-0.01\\
87.5297259719222	-0.01\\
87.5797286717063	-0.01\\
87.6297313714903	-0.01\\
87.6797340712743	-0.01\\
87.7297367710583	-0.01\\
87.7797394708423	-0.01\\
87.8297421706264	-0.01\\
87.8797448704104	-0.01\\
87.9297475701944	-0.01\\
87.9797502699784	-0.01\\
88.0297529697624	-0.01\\
88.0797556695464	-0.01\\
88.1297583693305	-0.01\\
88.1797610691145	-0.01\\
88.2297637688985	-0.01\\
88.2797664686825	-0.01\\
88.3297691684665	-0.01\\
88.3797718682505	-0.01\\
88.4297745680346	-0.01\\
88.4797772678186	-0.01\\
88.5297799676026	-0.01\\
88.5797826673866	-0.01\\
88.6297853671706	-0.01\\
88.6797880669546	-0.01\\
88.7297907667387	-0.01\\
88.7797934665227	-0.01\\
88.8297961663067	-0.01\\
88.8797988660907	-0.01\\
88.9298015658747	-0.01\\
88.9798042656588	-0.01\\
89.0298069654428	-0.01\\
89.0798096652268	-0.01\\
89.1298123650108	-0.01\\
89.1798150647948	-0.01\\
89.2298177645788	-0.01\\
89.2798204643629	-0.01\\
89.3298231641469	-0.01\\
89.3798258639309	-0.01\\
89.4298285637149	-0.01\\
89.4798312634989	-0.01\\
89.5298339632829	-0.01\\
89.579836663067	-0.01\\
89.629839362851	-0.01\\
89.679842062635	-0.01\\
89.729844762419	-0.01\\
89.779847462203	-0.01\\
89.8298501619871	-0.01\\
89.8798528617711	-0.01\\
89.9298555615551	-0.01\\
89.9798582613391	-0.01\\
90.0298609611231	-0.01\\
90.0798636609071	-0.01\\
90.1298663606911	-0.01\\
90.1798690604752	-0.01\\
90.2298717602592	-0.01\\
90.2798744600432	-0.01\\
90.3298771598272	-0.01\\
90.3798798596112	-0.01\\
90.4298825593953	-0.01\\
90.4798852591793	-0.01\\
90.5298879589633	-0.01\\
90.5798906587473	-0.01\\
90.6298933585313	-0.01\\
90.6798960583153	-0.01\\
90.7298987580994	-0.01\\
90.7799014578834	-0.01\\
90.8299041576674	-0.01\\
90.8799068574514	-0.01\\
90.9299095572354	-0.01\\
90.9799122570195	-0.01\\
91.0299149568035	-0.01\\
91.0799176565875	-0.01\\
91.1299203563715	-0.01\\
91.1799230561555	-0.01\\
91.2299257559395	-0.01\\
91.2799284557236	-0.01\\
91.3299311555076	-0.01\\
91.3799338552916	-0.01\\
91.4299365550756	-0.01\\
91.4799392548596	-0.01\\
91.5299419546436	-0.01\\
91.5799446544276	-0.01\\
91.6299473542117	-0.01\\
91.6799500539957	-0.01\\
91.7299527537797	-0.01\\
91.7799554535637	-0.01\\
91.8299581533477	-0.01\\
91.8799608531317	-0.01\\
91.9299635529158	-0.01\\
91.9799662526998	-0.01\\
92.0299689524838	-0.01\\
92.0799716522678	-0.01\\
92.1299743520518	-0.01\\
92.1799770518358	-0.01\\
92.2299797516199	-0.01\\
92.2799824514039	-0.01\\
92.3299851511879	-0.01\\
92.3799878509719	-0.01\\
92.4299905507559	-0.01\\
92.47999325054	-0.01\\
92.529995950324	-0.01\\
92.579998650108	-0.01\\
92.605	-0.01\\
};
\addlegendentry{- 1cm};

\addplot [color=red,solid,line width=0.2pt]
  table[row sep=crcr]{0	-0.075003\\
0.0500026997840173	-0.074983\\
0.100005399568035	-0.074955\\
0.150008099352052	-0.073954\\
0.200010799136069	-0.075066\\
0.250013498920086	-0.074146\\
0.300016198704104	-0.073995\\
0.350018898488121	-0.073456\\
0.400021598272138	-0.073304\\
0.450024298056156	-0.073288\\
0.500026997840173	-0.073859\\
0.55002969762419	-0.073271\\
0.600032397408207	-0.075337\\
0.650035097192225	-0.073829\\
0.700037796976242	-0.073349\\
0.750040496760259	-0.073073\\
0.800043196544276	-0.073011\\
0.850045896328294	-0.074833\\
0.900048596112311	-0.073879\\
0.950051295896328	-0.074291\\
1.00005399568035	-0.073886\\
1.05005669546436	-0.073226\\
1.10005939524838	-0.07297\\
1.1500620950324	-0.07345\\
1.20006479481641	-0.073253\\
1.25006749460043	-0.073608\\
1.30007019438445	-0.073357\\
1.35007289416847	-0.072791\\
1.40007559395248	-0.072473\\
1.4500782937365	-0.072299\\
1.50008099352052	-0.071944\\
1.55008369330454	-0.071813\\
1.60008639308855	-0.072657\\
1.65008909287257	-0.072566\\
1.70009179265659	-0.072121\\
1.7500944924406	-0.071908\\
1.80009719222462	-0.071798\\
1.85009989200864	-0.072185\\
1.90010259179266	-0.072439\\
1.95010529157667	-0.072051\\
2.00010799136069	-0.072449\\
2.05011069114471	-0.070773\\
2.10011339092873	-0.066297\\
2.15011609071274	-0.060216\\
2.20011879049676	-0.049576\\
2.25012149028078	-0.038458\\
2.30012419006479	-0.030138\\
2.35012688984881	-0.02659\\
2.40012958963283	-0.024174\\
2.45013228941685	-0.021831\\
2.50013498920086	-0.019046\\
2.55013768898488	-0.016436\\
2.6001403887689	-0.013247\\
2.65014308855292	-0.012173\\
2.70014578833693	-0.011279\\
2.75014848812095	-0.010226\\
2.80015118790497	-0.009076\\
2.85015388768899	-0.007366\\
2.900156587473	-0.007382\\
2.95015928725702	-0.006207\\
3.00016198704104	-0.006207\\
3.05016468682505	-0.00545\\
3.10016738660907	-0.003491\\
3.15017008639309	-0.003284\\
3.20017278617711	-0.003722\\
3.25017548596112	-0.004076\\
3.30017818574514	-0.004393\\
3.35018088552916	-0.003226\\
3.40018358531318	-0.002523\\
3.45018628509719	-0.003838\\
3.50018898488121	-0.004748\\
3.55019168466523	-0.005132\\
3.60019438444924	-0.004096\\
3.65019708423326	-0.004085\\
3.70019978401728	-0.00422\\
3.7502024838013	-0.00456\\
3.80020518358531	-0.004908\\
3.85020788336933	-0.0044\\
3.90021058315335	-0.003887\\
3.95021328293736	-0.002921\\
4.00021598272138	-0.001625\\
4.0502186825054	-0.001229\\
4.10022138228942	-0.002481\\
4.15022408207343	-0.003454\\
4.20022678185745	-0.003244\\
4.25022948164147	-0.003109\\
4.30023218142549	-0.00209\\
4.3502348812095	-0.002396\\
4.40023758099352	-0.002751\\
4.45024028077754	-0.001355\\
4.50024298056155	-0.001316\\
4.55024568034557	-0.001413\\
4.60024838012959	-0.001619\\
4.65025107991361	-0.001125\\
4.70025377969762	-0.001186\\
4.75025647948164	-0.000338\\
4.80025917926566	-0.000139\\
4.85026187904968	-0.000115\\
4.90026457883369	-0.00022\\
4.95026727861771	7.6e-05\\
5.00026997840173	-0.000148\\
5.05027267818575	-0.000546\\
5.10027537796976	-0.000892\\
5.15027807775378	-0.001127\\
5.2002807775378	-0.000283\\
5.25028347732181	0.000874\\
5.30028617710583	0.000295\\
5.35028887688985	8.6e-05\\
5.40029157667387	0.000259\\
5.45029427645788	-0.000391\\
5.5002969762419	-0.001348\\
5.55029967602592	-0.000577\\
5.60030237580994	-0.000573\\
5.65030507559395	-0.000616\\
5.70030777537797	-0.000678\\
5.75031047516199	-0.000872\\
5.800313174946	-0.00113\\
5.85031587473002	-0.000484\\
5.90031857451404	-0.000644\\
5.95032127429806	-0.000821\\
6.00032397408207	-0.00062\\
6.05032667386609	-0.00029\\
6.10032937365011	0.000829\\
6.15033207343413	0.000662\\
6.20033477321814	0.000289\\
6.25033747300216	0.001395\\
6.30034017278618	0.00207\\
6.3503428725702	0.001229\\
6.40034557235421	0.000862\\
6.45034827213823	0.00131\\
6.50035097192225	0.000227\\
6.55035367170626	0.001733\\
6.60035637149028	0.001414\\
6.6503590712743	-0.000218\\
6.70036177105832	-0.001463\\
6.75036447084233	-0.001581\\
6.80036717062635	-0.000933\\
6.85036987041037	0.001031\\
6.90037257019439	-0.000786\\
6.9503752699784	-0.002603\\
7.00037796976242	-0.003766\\
7.05038066954644	-0.004201\\
7.10038336933045	-0.003364\\
7.15038606911447	-0.003124\\
7.20038876889849	-0.003674\\
7.25039146868251	-0.002977\\
7.30039416846652	-0.003123\\
7.35039686825054	-0.003868\\
7.40039956803456	-0.003522\\
7.45040226781857	-0.003455\\
7.50040496760259	-0.002184\\
7.55040766738661	-0.00125\\
7.60041036717063	0.000722\\
7.65041306695464	-0.000264\\
7.70041576673866	-0.000339\\
7.75041846652268	-0.001041\\
7.8004211663067	-0.000131\\
7.85042386609071	0.00348\\
7.90042656587473	0.002925\\
7.95042926565875	0.001898\\
8.00043196544276	0.001288\\
8.05043466522678	0.000726\\
8.1004373650108	0.001815\\
8.15044006479482	0.001858\\
8.20044276457883	0.001495\\
8.25044546436285	0.000999\\
8.30044816414687	-0.000346\\
8.35045086393089	-0.003011\\
8.4004535637149	-0.004537\\
8.45045626349892	-0.004557\\
8.50045896328294	-0.002099\\
8.55046166306696	-0.002409\\
8.60046436285097	-0.003178\\
8.65046706263499	-0.004498\\
8.70046976241901	-0.005089\\
8.75047246220302	-0.004953\\
8.80047516198704	-0.004918\\
8.85047786177106	-0.0049\\
8.90048056155508	-0.00396\\
8.95048326133909	-0.00345\\
9.00048596112311	-0.001789\\
9.05048866090713	-0.001724\\
9.10049136069114	-0.001457\\
9.15049406047516	-0.000356\\
9.20049676025918	-0.000451\\
9.2504994600432	-0.000413\\
9.30050215982721	-0.000287\\
9.35050485961123	0.000213\\
9.40050755939525	0.000973\\
9.45051025917927	0.000947\\
9.50051295896328	0.002084\\
9.5505156587473	0.001466\\
9.60051835853132	0.001353\\
9.65052105831533	0.000178\\
9.70052375809935	-0.000551\\
9.75052645788337	-0.001248\\
9.80052915766739	-0.00129\\
9.8505318574514	-0.001355\\
9.90053455723542	-0.00132\\
9.95053725701944	-0.000797\\
10.0005399568035	-0.000423\\
10.0505426565875	-0.000156\\
10.1005453563715	0.000418\\
10.1505480561555	0.00023\\
10.2005507559395	0.000394\\
10.2505534557235	0.001086\\
10.3005561555076	0.000637\\
10.3505588552916	0.000321\\
10.4005615550756	-4.1e-05\\
10.4505642548596	-0.000723\\
10.5005669546436	-0.000904\\
10.5505696544276	-0.000589\\
10.6005723542117	-0.00056\\
10.6505750539957	-0.000241\\
10.7005777537797	-0.000559\\
10.7505804535637	0.00028\\
10.8005831533477	0.001932\\
10.8505858531317	0.001772\\
10.9005885529158	0.000876\\
10.9505912526998	0.001343\\
11.0005939524838	0.001237\\
11.0505966522678	0.000811\\
11.1005993520518	0.001141\\
11.1506020518359	0.000993\\
11.2006047516199	0.001048\\
11.2506074514039	0.001857\\
11.3006101511879	0.001288\\
11.3506128509719	0.000592\\
11.4006155507559	0.00039\\
11.45061825054	0.000264\\
11.500620950324	-4e-05\\
11.550623650108	0.000601\\
11.600626349892	0.000572\\
11.650629049676	0.000258\\
11.70063174946	0.000704\\
11.7506344492441	0.000452\\
11.8006371490281	-0.000178\\
11.8506398488121	-0.000898\\
11.9006425485961	-0.001077\\
11.9506452483801	-0.000951\\
12.0006479481641	-0.000318\\
12.0506506479482	-5.8e-05\\
12.1006533477322	-9.6e-05\\
12.1506560475162	-3e-06\\
12.2006587473002	0.000441\\
12.2506614470842	0.001098\\
12.3006641468683	-2.6e-05\\
12.3506668466523	-0.000746\\
12.4006695464363	-0.000981\\
12.4506722462203	-0.000914\\
12.5006749460043	-0.000464\\
12.5506776457883	-0.000378\\
12.6006803455724	-0.000711\\
12.6506830453564	-0.000839\\
12.7006857451404	0.00066\\
12.7506884449244	0.000303\\
12.8006911447084	0.000347\\
12.8506938444924	0.000356\\
12.9006965442765	-0.001082\\
12.9506992440605	-0.001425\\
13.0007019438445	-0.001715\\
13.0507046436285	-0.001476\\
13.1007073434125	-0.001141\\
13.1507100431965	-0.001183\\
13.2007127429806	-0.001379\\
13.2507154427646	-0.000768\\
13.3007181425486	-0.001726\\
13.3507208423326	-0.002396\\
13.4007235421166	-0.00239\\
13.4507262419006	-0.001559\\
13.5007289416847	-0.002046\\
13.5507316414687	-0.001881\\
13.6007343412527	-0.001983\\
13.6507370410367	-0.001245\\
13.7007397408207	-0.001056\\
13.7507424406048	-0.001496\\
13.8007451403888	-0.001744\\
13.8507478401728	-0.002279\\
13.9007505399568	-0.00252\\
13.9507532397408	-0.002517\\
14.0007559395248	-0.001835\\
14.0507586393089	-0.000818\\
14.1007613390929	-0.000685\\
14.1507640388769	-0.000948\\
14.2007667386609	-0.001333\\
14.2507694384449	-0.001336\\
14.3007721382289	-0.000471\\
14.350774838013	0.000806\\
14.400777537797	0.001366\\
14.450780237581	0.001269\\
14.500782937365	0.000187\\
14.550785637149	0.000404\\
14.600788336933	-0.000773\\
14.6507910367171	-0.001171\\
14.7007937365011	0.00059\\
14.7507964362851	-2.8e-05\\
14.8007991360691	-0.000833\\
14.8508018358531	-0.001261\\
14.9008045356372	-0.002331\\
14.9508072354212	-0.003037\\
15.0008099352052	-0.003226\\
15.0508126349892	-0.002441\\
15.1008153347732	-0.001735\\
15.1508180345572	-0.002867\\
15.2008207343413	-0.00366\\
15.2508234341253	-0.003202\\
15.3008261339093	-0.002875\\
15.3508288336933	-0.002258\\
15.4008315334773	-0.000562\\
15.4508342332613	-0.000544\\
15.5008369330454	-0.000525\\
15.5508396328294	-0.0013\\
15.6008423326134	-0.002581\\
15.6508450323974	-0.002729\\
15.7008477321814	-0.002326\\
15.7508504319654	-0.002416\\
15.8008531317495	-0.000636\\
15.8508558315335	0.000348\\
15.9008585313175	-0.000927\\
15.9508612311015	-0.000598\\
16.0008639308855	-0.000968\\
16.0508666306695	-0.000954\\
16.1008693304536	-0.001695\\
16.1508720302376	-0.00107\\
16.2008747300216	-0.00211\\
16.2508774298056	-0.00295\\
16.3008801295896	-0.003853\\
16.3508828293737	-0.003772\\
16.4008855291577	-0.003106\\
16.4508882289417	-0.002199\\
16.5008909287257	-0.000958\\
16.5508936285097	-1.2e-05\\
16.6008963282937	0.000271\\
16.6508990280778	-0.000899\\
16.7009017278618	-0.001266\\
16.7509044276458	-7.7e-05\\
16.8009071274298	-0.000147\\
16.8509098272138	-0.001036\\
16.9009125269978	-0.000567\\
16.9509152267819	-0.001218\\
17.0009179265659	-0.001096\\
17.0509206263499	-0.000867\\
17.1009233261339	-0.001656\\
17.1509260259179	-0.002278\\
17.2009287257019	-0.002688\\
17.250931425486	-0.002569\\
17.30093412527	-0.002739\\
17.350936825054	-0.001736\\
17.400939524838	-0.001091\\
17.450942224622	-0.000953\\
17.500944924406	-0.000666\\
17.5509476241901	-0.000664\\
17.6009503239741	-0.0001\\
17.6509530237581	0.000654\\
17.7009557235421	0.00044\\
17.7509584233261	0.000312\\
17.8009611231102	0.000842\\
17.8509638228942	0.001009\\
17.9009665226782	0.000912\\
17.9509692224622	0.001315\\
18.0009719222462	0.001191\\
18.0509746220302	0.000482\\
18.1009773218143	-0.000669\\
18.1509800215983	-0.000821\\
18.2009827213823	0.00016\\
18.2509854211663	0.00052\\
18.3009881209503	0.000571\\
18.3509908207343	0.000527\\
18.4009935205184	0.000782\\
18.4509962203024	0.00124\\
18.5009989200864	0.001153\\
18.5510016198704	0.002918\\
18.6010043196544	0.003492\\
18.6510070194384	0.001586\\
18.7010097192225	0.001251\\
18.7510124190065	0.000436\\
18.8010151187905	-4.6e-05\\
18.8510178185745	-0.000634\\
18.9010205183585	-0.000461\\
18.9510232181425	-0.001073\\
19.0010259179266	-0.000593\\
19.0510286177106	-0.000474\\
19.1010313174946	-0.00037\\
19.1510340172786	-0.000341\\
19.2010367170626	-0.000302\\
19.2510394168467	-0.000211\\
19.3010421166307	-0.000801\\
19.3510448164147	-0.001197\\
19.4010475161987	-0.001381\\
19.4510502159827	-0.001101\\
19.5010529157667	-0.001062\\
19.5510556155508	-0.000857\\
19.6010583153348	-0.001064\\
19.6510610151188	-0.001314\\
19.7010637149028	-0.000624\\
19.7510664146868	-0.000732\\
19.8010691144708	-0.000762\\
19.8510718142549	-0.000849\\
19.9010745140389	-0.000421\\
19.9510772138229	-0.000378\\
20.0010799136069	-0.000334\\
20.0510826133909	-0.000315\\
20.1010853131749	6.7e-05\\
20.151088012959	-9e-05\\
20.201090712743	-0.000298\\
20.251093412527	-0.000804\\
20.301096112311	-0.00085\\
20.351098812095	-0.000548\\
20.4011015118791	-0.000628\\
20.4511042116631	-0.000562\\
20.5011069114471	-0.000286\\
20.5511096112311	-0.000442\\
20.6011123110151	-0.00102\\
20.6511150107991	-0.000222\\
20.7011177105832	0.000443\\
20.7511204103672	0.000409\\
20.8011231101512	0.001064\\
20.8511258099352	0.001397\\
20.9011285097192	0.003367\\
20.9511312095032	0.004354\\
21.0011339092873	0.002344\\
21.0511366090713	0.000658\\
21.1011393088553	0.000298\\
21.1511420086393	0.000475\\
21.2011447084233	0.000384\\
21.2511474082073	0.000978\\
21.3011501079914	0.00066\\
21.3511528077754	0.000426\\
21.4011555075594	0.000426\\
21.4511582073434	0.00043\\
21.5011609071274	0.000806\\
21.5511636069114	0.00172\\
21.6011663066955	0.001087\\
21.6511690064795	0.00035\\
21.7011717062635	0.003488\\
21.7511744060475	0.003507\\
21.8011771058315	0.00285\\
21.8511798056156	0.002122\\
21.9011825053996	0.001713\\
21.9511852051836	0.001712\\
22.0011879049676	0.001462\\
22.0511906047516	0.001087\\
22.1011933045356	0.000533\\
22.1511960043197	8.3e-05\\
22.2011987041037	-0.000132\\
22.2512014038877	-0.001331\\
22.3012041036717	-0.004184\\
22.3512068034557	-0.005679\\
22.4012095032397	-0.005819\\
22.4512122030238	-0.004488\\
22.5012149028078	-0.003657\\
22.5512176025918	-0.003425\\
22.6012203023758	-0.002426\\
22.6512230021598	-0.002608\\
22.7012257019438	-0.003989\\
22.7512284017279	-0.00293\\
22.8012311015119	-0.001825\\
22.8512338012959	-0.001131\\
22.9012365010799	-0.000975\\
22.9512392008639	-0.000849\\
23.0012419006479	-0.000703\\
23.051244600432	-0.001702\\
23.101247300216	-0.002215\\
23.15125	-0.001022\\
23.201252699784	0.001431\\
23.251255399568	-0.000626\\
23.3012580993521	-0.001549\\
23.3512607991361	-0.000209\\
23.4012634989201	-0.000746\\
23.4512661987041	0.000256\\
23.5012688984881	0.001305\\
23.5512715982721	0.000309\\
23.6012742980562	9.7e-05\\
23.6512769978402	0.001325\\
23.7012796976242	-0.000449\\
23.7512823974082	-0.000801\\
23.8012850971922	-0.001959\\
23.8512877969762	-0.002036\\
23.9012904967603	-0.002507\\
23.9512931965443	-0.00281\\
24.0012958963283	-0.0019\\
24.0512985961123	0.000461\\
24.1013012958963	0.000797\\
24.1513039956803	-9.7e-05\\
24.2013066954644	-0.000129\\
24.2513093952484	0.000708\\
24.3013120950324	0.000809\\
24.3513147948164	0.000495\\
24.4013174946004	0.000436\\
24.4513201943845	0.000767\\
24.5013228941685	0.000331\\
24.5513255939525	0.000577\\
24.6013282937365	-0.00065\\
24.6513309935205	-0.001352\\
24.7013336933045	-0.00186\\
24.7513363930886	-0.000869\\
24.8013390928726	-0.000372\\
24.8513417926566	-0.000186\\
24.9013444924406	0.00017\\
24.9513471922246	-0.000184\\
25.0013498920086	-0.000705\\
25.0513525917927	-0.000624\\
25.1013552915767	-0.000371\\
25.1513579913607	0.000798\\
25.2013606911447	0.000267\\
25.2513633909287	0.002109\\
25.3013660907127	0.001984\\
25.3513687904968	0.001681\\
25.4013714902808	0.001423\\
25.4513741900648	0.000817\\
25.5013768898488	0.000909\\
25.5513795896328	0.000249\\
25.6013822894168	-0.0002\\
25.6513849892009	0.001483\\
25.7013876889849	0.001361\\
25.7513903887689	0.000378\\
25.8013930885529	0.000234\\
25.8513957883369	0.001121\\
25.901398488121	0.002784\\
25.951401187905	0.00306\\
26.001403887689	0.00249\\
26.051406587473	0.00134\\
26.101409287257	0.00015\\
26.151411987041	-0.00106\\
26.2014146868251	0.00017\\
26.2514173866091	0.000122\\
26.3014200863931	0.000275\\
26.3514227861771	0.001988\\
26.4014254859611	0.002515\\
26.4514281857451	0.000666\\
26.5014308855292	-0.000814\\
26.5514335853132	-0.000797\\
26.6014362850972	-0.000399\\
26.6514389848812	0.000624\\
26.7014416846652	-0.000467\\
26.7514443844492	-0.000968\\
26.8014470842333	-0.001945\\
26.8514497840173	-0.001372\\
26.9014524838013	-0.00144\\
26.9514551835853	-0.00101\\
27.0014578833693	0.000302\\
27.0514605831534	-0.00084\\
27.1014632829374	-0.001684\\
27.1514659827214	-0.001495\\
27.2014686825054	-0.000821\\
27.2514713822894	2.1e-05\\
27.3014740820734	9.8e-05\\
27.3514767818575	-5.5e-05\\
27.4014794816415	-0.000255\\
27.4514821814255	0.000702\\
27.5014848812095	0.002969\\
27.5514875809935	0.001437\\
27.6014902807775	-0.000664\\
27.6514929805616	-0.000659\\
27.7014956803456	0.00022\\
27.7514983801296	0.000681\\
27.8015010799136	0.001145\\
27.8515037796976	0.001482\\
27.9015064794816	0.002174\\
27.9515091792657	0.002821\\
28.0015118790497	0.002352\\
28.0515145788337	0.003329\\
28.1015172786177	0.003065\\
28.1515199784017	0.001469\\
28.2015226781857	0.001404\\
28.2515253779698	0.002168\\
28.3015280777538	0.002557\\
28.3515307775378	0.002174\\
28.4015334773218	0.002307\\
28.4515361771058	0.001834\\
28.5015388768899	0.001462\\
28.5515415766739	0.001309\\
28.6015442764579	0.000767\\
28.6515469762419	0.000287\\
28.7015496760259	-7.1e-05\\
28.7515523758099	0.000532\\
28.801555075594	0.000476\\
28.851557775378	0.000184\\
28.901560475162	0.00087\\
28.951563174946	0.000492\\
29.00156587473	0.000195\\
29.051568574514	4.2e-05\\
29.1015712742981	0.000132\\
29.1515739740821	0.000536\\
29.2015766738661	-0.001627\\
29.2515793736501	-0.002352\\
29.3015820734341	6.8e-05\\
29.3515847732181	0.002701\\
29.4015874730022	0.001543\\
29.4515901727862	-0.000644\\
29.5015928725702	-0.001642\\
29.5515955723542	-0.002692\\
29.6015982721382	-0.003291\\
29.6516009719222	-0.004052\\
29.7016036717063	-0.003789\\
29.7516063714903	-0.004365\\
29.8016090712743	-0.003849\\
29.8516117710583	-0.002884\\
29.9016144708423	-0.000968\\
29.9516171706264	-0.000265\\
30.0016198704104	-0.00035\\
30.0516225701944	-7.9e-05\\
30.1016252699784	5.9e-05\\
30.1516279697624	8e-05\\
30.2016306695464	-0.000124\\
30.2516333693305	-0.001495\\
30.3016360691145	-0.001724\\
30.3516387688985	-0.001759\\
30.4016414686825	-0.002406\\
30.4516441684665	-0.000678\\
30.5016468682505	0.000659\\
30.5516495680346	0.000427\\
30.6016522678186	0.000386\\
30.6516549676026	-0.000313\\
30.7016576673866	-6.7e-05\\
30.7516603671706	6.2e-05\\
30.8016630669546	0.000122\\
30.8516657667387	8e-06\\
30.9016684665227	-0.002042\\
30.9516711663067	-0.000988\\
31.0016738660907	0.000239\\
31.0516765658747	0.001477\\
31.1016792656587	0.002072\\
31.1516819654428	0.002352\\
31.2016846652268	0.001227\\
31.2516873650108	0.000156\\
31.3016900647948	-0.001407\\
31.3516927645788	-0.002247\\
31.4016954643629	-0.001753\\
31.4516981641469	-0.001927\\
31.5017008639309	-0.000482\\
31.5517035637149	-0.001014\\
31.6017062634989	-0.002057\\
31.6517089632829	-0.002118\\
31.701711663067	-0.001594\\
31.751714362851	-0.001417\\
31.801717062635	-0.0008\\
31.851719762419	-7.4e-05\\
31.901722462203	0.00142\\
31.951725161987	0.001301\\
32.0017278617711	0.001576\\
32.0517305615551	0.003084\\
32.1017332613391	0.003937\\
32.1517359611231	0.003751\\
32.2017386609071	0.004106\\
32.2517413606911	0.003356\\
32.3017440604752	0.003239\\
32.3517467602592	0.003579\\
32.4017494600432	0.004336\\
32.4517521598272	0.00576\\
32.5017548596112	0.004938\\
32.5517575593952	0.003938\\
32.6017602591793	0.003459\\
32.6517629589633	0.004388\\
32.7017656587473	0.002456\\
32.7517683585313	0.001991\\
32.8017710583153	0.000901\\
32.8517737580993	-0.000228\\
32.9017764578834	-0.000426\\
32.9517791576674	-0.000698\\
33.0017818574514	-0.000773\\
33.0517845572354	-1e-06\\
33.1017872570194	0.001055\\
33.1517899568035	0.001259\\
33.2017926565875	0.001874\\
33.2517953563715	0.001415\\
33.3017980561555	0.000956\\
33.3518007559395	0.001491\\
33.4018034557235	0.000896\\
33.4518061555076	0.000446\\
33.5018088552916	-0.000154\\
33.5518115550756	-0.00028\\
33.6018142548596	0.000636\\
33.6518169546436	0.000594\\
33.7018196544277	0.001963\\
33.7518223542117	0.001605\\
33.8018250539957	-0.000556\\
33.8518277537797	-0.000975\\
33.9018304535637	-0.001765\\
33.9518331533477	-0.001546\\
34.0018358531318	-0.000955\\
34.0518385529158	-0.000721\\
34.1018412526998	-0.000503\\
34.1518439524838	0.000312\\
34.2018466522678	0.000513\\
34.2518493520518	-0.000383\\
34.3018520518359	-0.001435\\
34.3518547516199	-0.001256\\
34.4018574514039	-0.000319\\
34.4518601511879	0.000336\\
34.5018628509719	0.000396\\
34.5518655507559	0.00021\\
34.60186825054	0.000896\\
34.651870950324	0.000974\\
34.701873650108	0.000541\\
34.751876349892	0.000916\\
34.801879049676	0.001037\\
34.85188174946	0.000144\\
34.9018844492441	-0.000927\\
34.9518871490281	-0.000795\\
35.0018898488121	-0.000744\\
35.0518925485961	-0.000466\\
35.1018952483801	-0.000585\\
35.1518979481641	-0.000563\\
35.2019006479482	-0.000468\\
35.2519033477322	-7.3e-05\\
35.3019060475162	0.000127\\
35.3519087473002	0.000304\\
35.4019114470842	0.000302\\
35.4519141468683	0.000382\\
35.5019168466523	0.0006\\
35.5519195464363	0.000268\\
35.6019222462203	0.001058\\
35.6519249460043	-0.000153\\
35.7019276457883	-0.000187\\
35.7519303455724	-9.9e-05\\
35.8019330453564	-0.000302\\
35.8519357451404	-0.000503\\
35.9019384449244	-0.000419\\
35.9519411447084	-0.00054\\
36.0019438444924	-0.000704\\
36.0519465442765	-0.000754\\
36.1019492440605	-0.000678\\
36.1519519438445	-0.000524\\
36.2019546436285	-0.00027\\
36.2519573434125	0.00022\\
36.3019600431965	0.000374\\
36.3519627429806	0.00045\\
36.4019654427646	0.000601\\
36.4519681425486	0.001176\\
36.5019708423326	0.001753\\
36.5519735421166	0.001199\\
36.6019762419006	0.000332\\
36.6519789416847	-0.000237\\
36.7019816414687	-0.000368\\
36.7519843412527	-0.00011\\
36.8019870410367	-0.000386\\
36.8519897408207	-0.000655\\
36.9019924406048	4.7e-05\\
36.9519951403888	0.003273\\
37.0019978401728	0.002291\\
37.0520005399568	0.000155\\
37.1020032397408	-0.000589\\
37.1520059395248	-0.000857\\
37.2020086393089	-0.001545\\
37.2520113390929	-0.002093\\
37.3020140388769	-0.002181\\
37.3520167386609	-0.002056\\
37.4020194384449	-0.001086\\
37.4520221382289	0.000163\\
37.502024838013	0.000101\\
37.552027537797	-0.001578\\
37.602030237581	-0.002819\\
37.652032937365	-0.002939\\
37.702035637149	-0.002286\\
37.752038336933	-0.003363\\
37.8020410367171	-0.002636\\
37.8520437365011	-0.002885\\
37.9020464362851	-0.002836\\
37.9520491360691	-0.00196\\
38.0020518358531	-0.001381\\
38.0520545356372	-0.001641\\
38.1020572354212	-0.002199\\
38.1520599352052	-0.002522\\
38.2020626349892	-0.001957\\
38.2520653347732	-0.000627\\
38.3020680345572	-0.000366\\
38.3520707343413	-0.000876\\
38.4020734341253	-0.000837\\
38.4520761339093	-0.001529\\
38.5020788336933	-0.002187\\
38.5520815334773	-0.001354\\
38.6020842332613	-0.001315\\
38.6520869330454	-0.001128\\
38.7020896328294	-0.001426\\
38.7520923326134	-0.001604\\
38.8020950323974	-0.000835\\
38.8520977321814	-0.001468\\
38.9021004319654	-0.000856\\
38.9521031317495	-0.001313\\
39.0021058315335	-0.001383\\
39.0521085313175	-0.001014\\
39.1021112311015	-0.000163\\
39.1521139308855	-2.4e-05\\
39.2021166306696	-0.00022\\
39.2521193304536	0.001562\\
39.3021220302376	0.001112\\
39.3521247300216	0.000787\\
39.4021274298056	0.00132\\
39.4521301295896	0.003684\\
39.5021328293737	0.002306\\
39.5521355291577	0.001448\\
39.6021382289417	0.000143\\
39.6521409287257	-7e-05\\
39.7021436285097	-0.000516\\
39.7521463282937	-0.001232\\
39.8021490280778	-0.002899\\
39.8521517278618	-0.002467\\
39.9021544276458	-0.001196\\
39.9521571274298	-0.000397\\
40.0021598272138	0.000115\\
40.0521625269978	-0.000184\\
40.1021652267819	-0.000202\\
40.1521679265659	-0.000804\\
40.2021706263499	-0.000118\\
40.2521733261339	-0.001098\\
40.3021760259179	-0.001513\\
40.3521787257019	-0.001057\\
40.402181425486	-0.000858\\
40.45218412527	-0.000206\\
40.502186825054	0.000655\\
40.552189524838	0.001291\\
40.602192224622	0.000446\\
40.652194924406	0.000979\\
40.7021976241901	0.000357\\
40.7522003239741	0.001848\\
40.8022030237581	0.004324\\
40.8522057235421	0.00229\\
40.9022084233261	9.5e-05\\
40.9522111231102	-0.001619\\
41.0022138228942	-0.000867\\
41.0522165226782	3.4e-05\\
41.1022192224622	0.00111\\
41.1522219222462	0.000494\\
41.2022246220302	0.001196\\
41.2522273218143	0.000845\\
41.3022300215983	0.000868\\
41.3522327213823	0.000314\\
41.4022354211663	-0.000309\\
41.4522381209503	-0.000806\\
41.5022408207343	-0.001845\\
41.5522435205184	-0.002723\\
41.6022462203024	-0.00334\\
41.6522489200864	-0.002688\\
41.7022516198704	-0.002991\\
41.7522543196544	-0.003819\\
41.8022570194385	-0.002887\\
41.8522597192225	-0.001264\\
41.9022624190065	-0.000734\\
41.9522651187905	-0.001823\\
42.0022678185745	-0.001766\\
42.0522705183585	-0.002119\\
42.1022732181426	-0.002127\\
42.1522759179266	-0.001823\\
42.2022786177106	-0.000605\\
42.2522813174946	-0.001301\\
42.3022840172786	-0.002154\\
42.3522867170626	-0.001791\\
42.4022894168467	-0.001799\\
42.4522921166307	-0.00134\\
42.5022948164147	-0.000558\\
42.5522975161987	-0.001274\\
42.6023002159827	-0.000149\\
42.6523029157667	0.000437\\
42.7023056155508	9.9e-05\\
42.7523083153348	-0.000723\\
42.8023110151188	-0.001561\\
42.8523137149028	-0.001574\\
42.9023164146868	-0.00024\\
42.9523191144708	-0.000911\\
43.0023218142549	0.000252\\
43.0523245140389	0.001542\\
43.1023272138229	0.003832\\
43.1523299136069	0.002894\\
43.2023326133909	4.8e-05\\
43.2523353131749	-0.000508\\
43.302338012959	-0.000639\\
43.352340712743	-0.000619\\
43.402343412527	-8e-06\\
43.452346112311	-0.001281\\
43.502348812095	-0.00331\\
43.5523515118791	-0.003527\\
43.6023542116631	-0.003863\\
43.6523569114471	-0.003416\\
43.7023596112311	-0.003178\\
43.7523623110151	-0.003327\\
43.8023650107991	-0.002505\\
43.8523677105832	-0.002666\\
43.9023704103672	-0.003542\\
43.9523731101512	-0.004918\\
44.0023758099352	-0.002447\\
44.0523785097192	-0.002436\\
44.1023812095032	-0.001803\\
44.1523839092873	-0.0018\\
44.2023866090713	-0.001768\\
44.2523893088553	-0.001645\\
44.3023920086393	-0.000576\\
44.3523947084233	-0.001367\\
44.4023974082073	-0.002905\\
44.4524001079914	-0.004984\\
44.5024028077754	-0.003465\\
44.5524055075594	-0.004206\\
44.6024082073434	-0.002896\\
44.6524109071274	-0.003545\\
44.7024136069114	-0.003577\\
44.7524163066955	-0.003469\\
44.8024190064795	-0.002618\\
44.8524217062635	-0.002806\\
44.9024244060475	-0.003069\\
44.9524271058315	-0.002938\\
45.0024298056155	-0.002644\\
45.0524325053996	-0.002632\\
45.1024352051836	-0.00244\\
45.1524379049676	-0.002071\\
45.2024406047516	-0.002047\\
45.2524433045356	-0.002471\\
45.3024460043197	-0.00216\\
45.3524487041037	-0.00199\\
45.4024514038877	-0.001866\\
45.4524541036717	-0.001641\\
45.5024568034557	-0.002111\\
45.5524595032397	-0.002515\\
45.6024622030238	-0.001963\\
45.6524649028078	-0.002063\\
45.7024676025918	-0.002239\\
45.7524703023758	-0.002363\\
45.8024730021598	-0.001057\\
45.8524757019439	-0.000384\\
45.9024784017279	-0.000584\\
45.9524811015119	-0.000986\\
46.0024838012959	-0.001498\\
46.0524865010799	-0.002185\\
46.1024892008639	-0.002259\\
46.152491900648	-0.002252\\
46.202494600432	-0.001481\\
46.252497300216	-0.002611\\
46.3025	-0.003405\\
46.352502699784	-0.001891\\
46.402505399568	-0.001959\\
46.4525080993521	-0.002243\\
46.5025107991361	-0.00304\\
46.5525134989201	-0.003185\\
46.6025161987041	-0.003029\\
46.6525188984881	-0.002869\\
46.7025215982721	-0.002152\\
46.7525242980562	-0.000599\\
46.8025269978402	0.000168\\
46.8525296976242	0.000693\\
46.9025323974082	-0.000146\\
46.9525350971922	-0.00033\\
47.0025377969762	-0.000447\\
47.0525404967603	-0.000408\\
47.1025431965443	-0.000201\\
47.1525458963283	-0.00068\\
47.2025485961123	-0.000731\\
47.2525512958963	-0.000779\\
47.3025539956803	-7.9e-05\\
47.3525566954644	-0.000812\\
47.4025593952484	-0.000704\\
47.4525620950324	0.000393\\
47.5025647948164	-0.001894\\
47.5525674946004	-0.003063\\
47.6025701943845	-0.004501\\
47.6525728941685	-0.004738\\
47.7025755939525	-0.005063\\
47.7525782937365	-0.005582\\
47.8025809935205	-0.005517\\
47.8525836933045	-0.004928\\
47.9025863930886	-0.004895\\
47.9525890928726	-0.003785\\
48.0025917926566	-0.00371\\
48.0525944924406	-0.004048\\
48.1025971922246	-0.004175\\
48.1525998920086	-0.003263\\
48.2026025917927	-0.001863\\
48.2526052915767	-0.001425\\
48.3026079913607	-0.001444\\
48.3526106911447	-0.00173\\
48.4026133909287	-0.00149\\
48.4526160907127	-0.000102\\
48.5026187904968	0.000626\\
48.5526214902808	-0.000229\\
48.6026241900648	-0.000453\\
48.6526268898488	-0.00125\\
48.7026295896328	-0.001913\\
48.7526322894168	-0.001937\\
48.8026349892009	-0.002132\\
48.8526376889849	-0.001852\\
48.9026403887689	-0.000987\\
48.9526430885529	0.000231\\
49.0026457883369	0.000141\\
49.052648488121	-4.1e-05\\
49.102651187905	-0.000822\\
49.152653887689	0.000427\\
49.202656587473	0.000446\\
49.252659287257	-0.000201\\
49.302661987041	-0.000481\\
49.3526646868251	-0.001367\\
49.4026673866091	-0.001303\\
49.4526700863931	-0.000603\\
49.5026727861771	-0.000909\\
49.5526754859611	-0.001499\\
49.6026781857451	-0.00151\\
49.6526808855292	-0.001485\\
49.7026835853132	-0.000807\\
49.7526862850972	-0.001407\\
49.8026889848812	-0.001112\\
49.8526916846652	-0.001677\\
49.9026943844492	-0.001922\\
49.9526970842333	-0.001382\\
50.0026997840173	8.8e-05\\
50.0527024838013	0.001873\\
50.1027051835853	0.001002\\
50.1527078833693	0.001078\\
50.2027105831534	0.000939\\
50.2527132829374	0.000328\\
50.3027159827214	-0.000237\\
50.3527186825054	-0.000787\\
50.4027213822894	-0.00029\\
50.4527240820734	-0.000632\\
50.5027267818575	-0.000151\\
50.5527294816415	0.000624\\
50.6027321814255	0.000355\\
50.6527348812095	5.1e-05\\
50.7027375809935	-4.1e-05\\
50.7527402807775	-0.000506\\
50.8027429805616	-0.001314\\
50.8527456803456	0.000359\\
50.9027483801296	-0.00096\\
50.9527510799136	-0.001901\\
51.0027537796976	-0.002947\\
51.0527564794816	-0.002002\\
51.1027591792657	-0.00142\\
51.1527618790497	-0.001424\\
51.2027645788337	-0.001347\\
51.2527672786177	0.000382\\
51.3027699784017	-0.000836\\
51.3527726781858	-0.002007\\
51.4027753779698	-0.003193\\
51.4527780777538	-0.002022\\
51.5027807775378	-0.001807\\
51.5527834773218	-0.001439\\
51.6027861771058	-0.002094\\
51.6527888768899	-0.000734\\
51.7027915766739	-0.001545\\
51.7527942764579	-0.001464\\
51.8027969762419	-0.001235\\
51.8527996760259	-0.001228\\
51.9028023758099	-0.001339\\
51.952805075594	-0.001694\\
52.002807775378	-0.002643\\
52.052810475162	-0.002978\\
52.102813174946	-0.003149\\
52.15281587473	-0.003454\\
52.202818574514	-0.003025\\
52.2528212742981	-0.00172\\
52.3028239740821	-0.001405\\
52.3528266738661	-0.001853\\
52.4028293736501	-0.002155\\
52.4528320734341	-0.003213\\
52.5028347732181	-0.002219\\
52.5528374730022	-0.00234\\
52.6028401727862	-0.002304\\
52.6528428725702	-0.001948\\
52.7028455723542	-0.001393\\
52.7528482721382	-0.000995\\
52.8028509719222	-0.000534\\
52.8528536717063	0.000495\\
52.9028563714903	0.000636\\
52.9528590712743	-0.000811\\
53.0028617710583	-0.000674\\
53.0528644708423	-0.000553\\
53.1028671706264	-0.001695\\
53.1528698704104	-0.002949\\
53.2028725701944	-0.002658\\
53.2528752699784	-0.000867\\
53.3028779697624	-0.001431\\
53.3528806695464	-0.002689\\
53.4028833693305	-0.003605\\
53.4528860691145	-0.003788\\
53.5028887688985	-0.003629\\
53.5528914686825	-0.003284\\
53.6028941684665	-0.002468\\
53.6528968682505	-0.002477\\
53.7028995680346	-0.001863\\
53.7529022678186	0.00067\\
53.8029049676026	0.001978\\
53.8529076673866	0.001269\\
53.9029103671706	-0.000812\\
53.9529130669547	-0.001214\\
54.0029157667387	-5.1e-05\\
54.0529184665227	0.000514\\
54.1029211663067	0.001192\\
54.1529238660907	0.002882\\
54.2029265658747	0.003517\\
54.2529292656588	0.001256\\
54.3029319654428	-0.001867\\
54.3529346652268	-0.001594\\
54.4029373650108	0.000661\\
54.4529400647948	-8e-05\\
54.5029427645788	-0.002743\\
54.5529454643629	-0.003544\\
54.6029481641469	-0.004488\\
54.6529508639309	-0.003031\\
54.7029535637149	-0.000565\\
54.7529562634989	-0.000271\\
54.8029589632829	-0.002065\\
54.852961663067	-0.003111\\
54.902964362851	-0.002274\\
54.952967062635	-0.001285\\
55.002969762419	-0.000808\\
55.052972462203	-0.000772\\
55.102975161987	-0.001004\\
55.1529778617711	-0.003166\\
55.2029805615551	-0.004529\\
55.2529832613391	-0.003937\\
55.3029859611231	-0.002646\\
55.3529886609071	-0.003674\\
55.4029913606911	-0.002948\\
55.4529940604752	-0.001488\\
55.5029967602592	-0.000413\\
55.5529994600432	-0.000574\\
55.6030021598272	-0.001775\\
55.6530048596112	-0.002\\
55.7030075593953	-0.002148\\
55.7530102591793	-0.001529\\
55.8030129589633	-0.00099\\
55.8530156587473	-0.001782\\
55.9030183585313	-0.002284\\
55.9530210583153	-0.002648\\
56.0030237580994	-0.002479\\
56.0530264578834	-0.001724\\
56.1030291576674	0.000197\\
56.1530318574514	0.000194\\
56.2030345572354	0.000278\\
56.2530372570194	-0.000147\\
56.3030399568035	-0.000184\\
56.3530426565875	0.00076\\
56.4030453563715	0.000574\\
56.4530480561555	0.000281\\
56.5030507559395	-0.000198\\
56.5530534557235	-0.000185\\
56.6030561555076	0.001157\\
56.6530588552916	0.001644\\
56.7030615550756	0.000788\\
56.7530642548596	0.000483\\
56.8030669546436	0.000289\\
56.8530696544276	0.000377\\
56.9030723542117	0.000883\\
56.9530750539957	0.000512\\
57.0030777537797	-0.000376\\
57.0530804535637	-0.001051\\
57.1030831533477	-0.001333\\
57.1530858531318	-0.000678\\
57.2030885529158	0.000921\\
57.2530912526998	0.00054\\
57.3030939524838	-0.00012\\
57.3530966522678	0.000372\\
57.4030993520518	7.6e-05\\
57.4531020518358	-7.2e-05\\
57.5031047516199	0.000286\\
57.5531074514039	-0.000418\\
57.6031101511879	-0.000726\\
57.6531128509719	-0.000283\\
57.7031155507559	-0.001326\\
57.75311825054	-0.002458\\
57.803120950324	-0.002663\\
57.853123650108	-0.002176\\
57.903126349892	-0.001027\\
57.953129049676	-0.001172\\
58.00313174946	0.000723\\
58.0531344492441	-0.00106\\
58.1031371490281	-0.002301\\
58.1531398488121	-0.002903\\
58.2031425485961	-0.002291\\
58.2531452483801	-0.000757\\
58.3031479481642	-0.000762\\
58.3531506479482	0.000244\\
58.4031533477322	-0.000239\\
58.4531560475162	0.001046\\
58.5031587473002	0.000343\\
58.5531614470842	-0.0001\\
58.6031641468683	-0.00076\\
58.6531668466523	-0.001297\\
58.7031695464363	-2.3e-05\\
58.7531722462203	0.000405\\
58.8031749460043	0.000761\\
58.8531776457883	-0.000399\\
58.9031803455724	-0.000771\\
58.9531830453564	-0.002974\\
59.0031857451404	-0.004057\\
59.0531884449244	-0.004759\\
59.1031911447084	-0.005312\\
59.1531938444924	-0.005328\\
59.2031965442765	-0.005358\\
59.2531992440605	-0.004697\\
59.3032019438445	-0.004715\\
59.3532046436285	-0.0044\\
59.4032073434125	-0.002977\\
59.4532100431965	-0.00188\\
59.5032127429806	-0.001484\\
59.5532154427646	-0.001\\
59.6032181425486	-0.001507\\
59.6532208423326	-0.001765\\
59.7032235421166	-0.000578\\
59.7532262419006	-0.000879\\
59.8032289416847	-1.7e-05\\
59.8532316414687	0.000717\\
59.9032343412527	0.000141\\
59.9532370410367	-0.000806\\
60.0032397408207	-0.002428\\
60.0532424406048	-0.002233\\
60.1032451403888	-0.002036\\
60.1532478401728	-0.001703\\
60.2032505399568	-0.001437\\
60.2532532397408	-0.001439\\
60.3032559395248	-0.000312\\
60.3532586393089	0.000129\\
60.4032613390929	-0.000573\\
60.4532640388769	0.000139\\
60.5032667386609	-0.001185\\
60.5532694384449	-0.002569\\
60.6032721382289	-0.003136\\
60.653274838013	-0.002828\\
60.703277537797	-0.003\\
60.753280237581	-0.003516\\
60.803282937365	-0.002424\\
60.853285637149	-0.001302\\
60.9032883369331	-0.001825\\
60.9532910367171	-0.002181\\
61.0032937365011	-0.001629\\
61.0532964362851	-0.001031\\
61.1032991360691	0.000179\\
61.1533018358531	0.000829\\
61.2033045356371	0.000852\\
61.2533072354212	3.6e-05\\
61.3033099352052	-0.000191\\
61.3533126349892	-0.000515\\
61.4033153347732	-0.000245\\
61.4533180345572	0.000949\\
61.5033207343413	0.000657\\
61.5533234341253	0.000554\\
61.6033261339093	0.000466\\
61.6533288336933	0.000853\\
61.7033315334773	0.000969\\
61.7533342332613	0.001278\\
61.8033369330454	0.001363\\
61.8533396328294	0.000617\\
61.9033423326134	0.000165\\
61.9533450323974	-0.000305\\
62.0033477321814	-0.000307\\
62.0533504319654	0.000556\\
62.1033531317495	0.000563\\
62.1533558315335	0.000515\\
62.2033585313175	0.000779\\
62.2533612311015	-1.7e-05\\
62.3033639308855	0.002252\\
62.3533666306695	0.002984\\
62.4033693304536	0.001301\\
62.4533720302376	-0.000137\\
62.5033747300216	6.5e-05\\
62.5533774298056	0.00034\\
62.6033801295896	0.002642\\
62.6533828293737	0.014417\\
62.6983852591793	0.036296\\
62.7433876889849	0.066141\\
62.7883901187905	0.09605\\
62.8333925485961	0.121222\\
62.8833952483801	0.141309\\
62.9333979481641	0.151022\\
62.9834006479482	0.155142\\
63.0334033477322	0.159414\\
63.0834060475162	0.166637\\
63.1334087473002	0.173664\\
63.1834114470842	0.178592\\
63.2334141468683	0.18183\\
63.2834168466523	0.184875\\
63.3334195464363	0.18768\\
63.3834222462203	0.190456\\
63.4334249460043	0.193105\\
63.4834276457883	0.195156\\
63.5334303455724	0.196708\\
63.5834330453564	0.197873\\
63.6334357451404	0.199034\\
63.6834384449244	0.19983\\
63.7334411447084	0.200343\\
63.7834438444924	0.200503\\
63.8334465442765	0.201718\\
63.8834492440605	0.202574\\
63.9334519438445	0.202726\\
63.9834546436285	0.202862\\
64.0334573434125	0.20272\\
64.0834600431965	0.203007\\
64.1334627429806	0.203288\\
64.1834654427646	0.203454\\
64.2334681425486	0.203445\\
64.2834708423326	0.203459\\
64.3334735421166	0.203516\\
64.3834762419007	0.202874\\
64.4334789416847	0.202032\\
64.4834816414687	0.201554\\
64.5334843412527	0.201426\\
64.5834870410367	0.201694\\
64.6334897408207	0.201893\\
64.6834924406048	0.202098\\
64.7334951403888	0.202161\\
64.7834978401728	0.201787\\
64.8335005399568	0.201817\\
64.8835032397408	0.201642\\
64.9335059395248	0.201283\\
64.9835086393089	0.200854\\
65.0335113390929	0.200952\\
65.0835140388769	0.201085\\
65.1335167386609	0.201288\\
65.1835194384449	0.201408\\
65.2335221382289	0.201285\\
65.283524838013	0.201574\\
65.333527537797	0.201591\\
65.383530237581	0.201806\\
65.433532937365	0.201915\\
65.483535637149	0.202041\\
65.533538336933	0.202123\\
65.5835410367171	0.20231\\
65.6335437365011	0.202297\\
65.6835464362851	0.202037\\
65.7335491360691	0.201688\\
65.7835518358531	0.201264\\
65.8335545356372	0.200999\\
65.8835572354212	0.200845\\
65.9335599352052	0.200777\\
65.9835626349892	0.200565\\
66.0335653347732	0.200237\\
66.0835680345572	0.200148\\
66.1335707343412	0.200118\\
66.1835734341253	0.200189\\
66.2335761339093	0.200502\\
66.2835788336933	0.200789\\
66.3335815334773	0.200927\\
66.3835842332613	0.200981\\
66.4335869330454	0.201087\\
66.4835896328294	0.201103\\
66.5335923326134	0.201112\\
66.5835950323974	0.200848\\
66.6335977321814	0.200899\\
66.6836004319654	0.201008\\
66.7336031317495	0.201146\\
66.7836058315335	0.201239\\
66.8336085313175	0.201274\\
66.8836112311015	0.201318\\
66.9336139308855	0.201404\\
66.9836166306695	0.201522\\
67.0336193304536	0.201643\\
67.0836220302376	0.201592\\
67.1336247300216	0.20131\\
67.1836274298056	0.201433\\
67.2336301295896	0.201725\\
67.2836328293737	0.201901\\
67.3336355291577	0.202029\\
67.3836382289417	0.201844\\
67.4336409287257	0.201894\\
67.4836436285097	0.201917\\
67.5336463282937	0.201727\\
67.5836490280778	0.201856\\
67.6336517278618	0.201946\\
67.6836544276458	0.202175\\
67.7336571274298	0.202373\\
67.7836598272138	0.202331\\
67.8336625269979	0.202191\\
67.8836652267819	0.202154\\
67.9336679265659	0.202235\\
67.9836706263499	0.202324\\
68.0336733261339	0.20307\\
68.0836760259179	0.203558\\
68.1336787257019	0.2039\\
68.183681425486	0.204087\\
68.23368412527	0.204074\\
68.283686825054	0.203999\\
68.333689524838	0.203588\\
68.383692224622	0.202999\\
68.4336949244061	0.202441\\
68.4836976241901	0.202203\\
68.5337003239741	0.202151\\
68.5837030237581	0.201865\\
68.6337057235421	0.201796\\
68.6837084233261	0.201512\\
68.7337111231102	0.201313\\
68.7837138228942	0.201374\\
68.8337165226782	0.201687\\
68.8837192224622	0.201784\\
68.9337219222462	0.201976\\
68.9837246220302	0.202052\\
69.0337273218143	0.2021\\
69.0837300215983	0.202204\\
69.1337327213823	0.202518\\
69.1837354211663	0.202701\\
69.2337381209503	0.202252\\
69.2837408207344	0.202488\\
69.3337435205184	0.203084\\
69.3837462203024	0.203379\\
69.4337489200864	0.203494\\
69.4837516198704	0.203539\\
69.5337543196544	0.203714\\
69.5837570194385	0.2032\\
69.6337597192225	0.202855\\
69.6837624190065	0.202469\\
69.7337651187905	0.202526\\
69.7837678185745	0.202685\\
69.8337705183585	0.20253\\
69.8837732181426	0.202437\\
69.9337759179266	0.202234\\
69.9837786177106	0.202711\\
70.0337813174946	0.203225\\
70.0837840172786	0.203595\\
70.1337867170626	0.203766\\
70.1837894168467	0.20321\\
70.2337921166307	0.202877\\
70.2837948164147	0.202761\\
70.3337975161987	0.202732\\
70.3838002159827	0.202549\\
70.4338029157667	0.202445\\
70.4838056155508	0.202154\\
70.5338083153348	0.202153\\
70.5838110151188	0.202065\\
70.6338137149028	0.202323\\
70.6838164146868	0.2023\\
70.7338191144708	0.202156\\
70.7838218142549	0.202008\\
70.8338245140389	0.201922\\
70.8838272138229	0.201973\\
70.9338299136069	0.202038\\
70.9838326133909	0.20207\\
71.0338353131749	0.202123\\
71.083838012959	0.201951\\
71.133840712743	0.202631\\
71.183843412527	0.203185\\
71.233846112311	0.202786\\
71.283848812095	0.203137\\
71.3338515118791	0.203348\\
71.3838542116631	0.203153\\
71.4338569114471	0.202946\\
71.4838596112311	0.202923\\
71.5338623110151	0.203298\\
71.5838650107991	0.203595\\
71.6338677105831	0.203618\\
71.6838704103672	0.203693\\
71.7338731101512	0.203516\\
71.7838758099352	0.20359\\
71.8338785097192	0.20356\\
71.8838812095032	0.203032\\
71.9338839092873	0.202657\\
71.9838866090713	0.202482\\
72.0338893088553	0.203069\\
72.0838920086393	0.203427\\
72.1338947084233	0.203588\\
72.1838974082073	0.20369\\
72.2339001079914	0.203769\\
72.2839028077754	0.203861\\
72.3339055075594	0.203207\\
72.3839082073434	0.202745\\
72.4339109071274	0.202324\\
72.4839136069115	0.20239\\
72.5339163066955	0.203073\\
72.5839190064795	0.203404\\
72.6339217062635	0.20338\\
72.6839244060475	0.203528\\
72.7339271058315	0.203598\\
72.7839298056156	0.203513\\
72.8339325053996	0.203075\\
72.8839352051836	0.202633\\
72.9339379049676	0.20241\\
72.9839406047516	0.202839\\
73.0339433045356	0.203345\\
73.0839460043197	0.203567\\
73.1339487041037	0.203682\\
73.1839514038877	0.203158\\
73.2339541036717	0.202369\\
73.2839568034557	0.202201\\
73.3339595032398	0.202239\\
73.3839622030238	0.202887\\
73.4339649028078	0.203255\\
73.4839676025918	0.203435\\
73.5339703023758	0.203239\\
73.5839730021598	0.20263\\
73.6339757019438	0.202221\\
73.6839784017279	0.202298\\
73.7339811015119	0.201874\\
73.7839838012959	0.202039\\
73.8339865010799	0.201898\\
73.8839892008639	0.201797\\
73.933991900648	0.201809\\
73.983994600432	0.201514\\
74.033997300216	0.201534\\
74.084	0.201587\\
74.134002699784	0.201616\\
74.184005399568	0.201402\\
74.2340080993521	0.201831\\
74.2840107991361	0.202178\\
74.3340134989201	0.202252\\
74.3840161987041	0.202331\\
74.4340188984881	0.202158\\
74.4840215982721	0.20227\\
74.5340242980562	0.20234\\
74.5840269978402	0.20268\\
74.6340296976242	0.203319\\
74.6840323974082	0.203421\\
74.7340350971922	0.20373\\
74.7840377969763	0.203927\\
74.8340404967603	0.203381\\
74.8840431965443	0.202858\\
74.9340458963283	0.202607\\
74.9840485961123	0.202408\\
75.0340512958963	0.201879\\
75.0840539956804	0.202421\\
75.1340566954644	0.202154\\
75.1840593952484	0.202187\\
75.2340620950324	0.201902\\
75.2840647948164	0.201807\\
75.3340674946004	0.201822\\
75.3840701943844	0.201893\\
75.4340728941685	0.201953\\
75.4840755939525	0.201979\\
75.5340782937365	0.20215\\
75.5840809935205	0.201999\\
75.6340836933045	0.201869\\
75.6840863930885	0.201785\\
75.7340890928726	0.201688\\
75.7840917926566	0.20164\\
75.8340944924406	0.201842\\
75.8840971922246	0.202001\\
75.9340998920086	0.20186\\
75.9841025917927	0.201815\\
76.0341052915767	0.201663\\
76.0841079913607	0.202289\\
76.1341106911447	0.202301\\
76.1841133909287	0.20228\\
76.2341160907127	0.201849\\
76.2841187904968	0.20173\\
76.3341214902808	0.20174\\
76.3841241900648	0.201855\\
76.4341268898488	0.201885\\
76.4841295896328	0.202387\\
76.5341322894169	0.202987\\
76.5841349892009	0.203171\\
76.6341376889849	0.202951\\
76.6841403887689	0.202687\\
76.7341430885529	0.202423\\
76.7841457883369	0.201811\\
76.834148488121	0.201889\\
76.884151187905	0.202296\\
76.934153887689	0.203021\\
76.984156587473	0.203057\\
77.034159287257	0.202906\\
77.084161987041	0.202708\\
77.1341646868251	0.202439\\
77.1841673866091	0.202025\\
77.2341700863931	0.201916\\
77.2841727861771	0.20203\\
77.3341754859611	0.202516\\
77.3841781857451	0.203135\\
77.4341808855292	0.20343\\
77.4841835853132	0.203513\\
77.5341862850972	0.202987\\
77.5841889848812	0.202095\\
77.6341916846652	0.201628\\
77.6841943844492	0.201458\\
77.7341970842333	0.200586\\
77.7841997840173	0.201127\\
77.8342024838013	0.202021\\
77.8842051835853	0.202771\\
77.9342078833693	0.203168\\
77.9842105831534	0.203315\\
78.0342132829374	0.202659\\
78.0842159827214	0.202179\\
78.1342186825054	0.201603\\
78.1842213822894	0.201541\\
78.2342240820734	0.201471\\
78.2842267818575	0.201387\\
78.3342294816415	0.201095\\
78.3842321814255	0.201105\\
78.4342348812095	0.200949\\
78.4842375809935	0.201073\\
78.5342402807775	0.201024\\
78.5842429805616	0.200975\\
78.6342456803456	0.201445\\
78.6842483801296	0.201945\\
78.7342510799136	0.201516\\
78.7842537796976	0.201501\\
78.8342564794817	0.201463\\
78.8842591792657	0.201274\\
78.9342618790497	0.201107\\
78.9842645788337	0.20094\\
79.0342672786177	0.2008\\
79.0842699784017	0.200691\\
79.1342726781857	0.200386\\
79.1842753779698	0.200392\\
79.2342780777538	0.200434\\
79.2842807775378	0.200567\\
79.3342834773218	0.20061\\
79.3842861771058	0.200759\\
79.4342888768899	0.200873\\
79.4842915766739	0.200876\\
79.5342942764579	0.200423\\
79.5842969762419	0.200559\\
79.6342996760259	0.200762\\
79.6843023758099	0.200911\\
79.734305075594	0.200966\\
79.784307775378	0.200924\\
79.834310475162	0.200608\\
79.884313174946	0.20082\\
79.93431587473	0.200875\\
79.984318574514	0.200685\\
80.0343212742981	0.200811\\
80.0843239740821	0.200996\\
80.1343266738661	0.201148\\
80.1843293736501	0.200803\\
80.2343320734341	0.201082\\
80.2843347732181	0.20139\\
80.3343374730022	0.20161\\
80.3843401727862	0.201675\\
80.4343428725702	0.201667\\
80.4843455723542	0.201527\\
80.5343482721382	0.201224\\
80.5843509719223	0.201313\\
80.6343536717063	0.20135\\
80.6843563714903	0.201304\\
80.7343590712743	0.201125\\
80.7843617710583	0.20136\\
80.8343644708423	0.20147\\
80.8843671706263	0.201202\\
80.9343698704104	0.201256\\
80.9843725701944	0.201153\\
81.0343752699784	0.201345\\
81.0843779697624	0.201435\\
81.1343806695464	0.20121\\
81.1843833693305	0.201289\\
81.2343860691145	0.201418\\
81.2843887688985	0.201394\\
81.3343914686825	0.201603\\
81.3843941684665	0.20174\\
81.4343968682505	0.201848\\
81.4843995680346	0.201917\\
81.5344022678186	0.201957\\
81.5844049676026	0.201923\\
81.6344076673866	0.201776\\
81.6844103671706	0.201694\\
81.7344130669546	0.20143\\
81.7844157667387	0.201274\\
81.8344184665227	0.201379\\
81.8844211663067	0.201832\\
81.9344238660907	0.202128\\
81.9844265658747	0.201958\\
82.0344292656588	0.201554\\
82.0844319654428	0.201199\\
82.1344346652268	0.200967\\
82.1844373650108	0.201212\\
82.2344400647948	0.20136\\
82.2844427645788	0.201626\\
82.3344454643629	0.201692\\
82.3844481641469	0.20138\\
82.4344508639309	0.201146\\
82.4844535637149	0.201043\\
82.5344562634989	0.200973\\
82.5844589632829	0.20116\\
82.634461663067	0.200959\\
82.684464362851	0.199022\\
82.734467062635	0.189471\\
82.784469762419	0.168776\\
82.8294721922246	0.139521\\
82.8744746220302	0.108195\\
82.9194770518359	0.083092\\
82.9694797516199	0.06655\\
83.0194824514039	0.057887\\
83.0694851511879	0.050696\\
83.1194878509719	0.041472\\
83.1694905507559	0.033\\
83.21949325054	0.026332\\
83.269495950324	0.022057\\
83.319498650108	0.018593\\
83.369501349892	0.014812\\
83.419504049676	0.01088\\
83.4695067494601	0.008768\\
83.5195094492441	0.008386\\
83.5695121490281	0.007319\\
83.6195148488121	0.005662\\
83.6695175485961	0.003801\\
83.7195202483801	0.003023\\
83.7695229481642	0.002741\\
83.8195256479482	0.002743\\
83.8695283477322	0.001931\\
83.9195310475162	0.001214\\
83.9695337473002	0.000596\\
84.0195364470842	0.00024\\
84.0695391468683	0.001009\\
84.1195418466523	0.001001\\
84.1695445464363	-0.000389\\
84.2195472462203	-0.002576\\
84.2695499460043	-0.002941\\
84.3195526457883	-0.001291\\
84.3695553455724	-0.000597\\
84.4195580453564	-0.000677\\
84.4695607451404	-0.001232\\
84.5195634449244	-0.001129\\
84.5695661447084	-0.000848\\
84.6195688444924	-0.000569\\
84.6695715442765	-0.00054\\
84.7195742440605	0.000295\\
84.7695769438445	0.000536\\
84.8195796436285	0.000497\\
84.8695823434125	0.000329\\
84.9195850431965	0.000358\\
84.9695877429806	0.001734\\
85.0195904427646	0.001105\\
85.0695931425486	0.000672\\
85.1195958423326	-0.000331\\
85.1695985421166	0.000362\\
85.2196012419006	0.000912\\
85.2696039416847	0.002235\\
85.3196066414687	0.002055\\
85.3696093412527	0.001953\\
85.4196120410367	0.002127\\
85.4696147408207	0.001322\\
85.5196174406048	0.000747\\
85.5696201403888	0.001082\\
85.6196228401728	0.000829\\
85.6696255399568	0.001906\\
85.7196282397408	0.001978\\
85.7696309395248	0.001035\\
85.8196336393089	0.000934\\
85.8696363390929	0.00222\\
85.9196390388769	0.002693\\
85.9696417386609	0.001568\\
86.0196444384449	0.000773\\
86.0696471382289	-0.000723\\
86.119649838013	-0.001303\\
86.169652537797	-0.000326\\
86.219655237581	-0.000527\\
86.269657937365	-0.000521\\
86.319660637149	-0.000839\\
86.369663336933	-0.001169\\
86.4196660367171	-0.001302\\
86.4696687365011	-0.000596\\
86.5196714362851	-0.000136\\
86.5696741360691	0.001274\\
86.6196768358531	0.001077\\
86.6696795356372	0.000311\\
86.7196822354212	0.000236\\
86.7696849352052	0.000744\\
86.8196876349892	0.000964\\
86.8696903347732	0.001639\\
86.9196930345572	0.001317\\
86.9696957343413	9.5e-05\\
87.0196984341253	-0.000386\\
87.0697011339093	0.000913\\
87.1197038336933	0.002098\\
87.1697065334773	0.001729\\
87.2197092332613	0.000379\\
87.2697119330454	-0.000594\\
87.3197146328294	-0.000562\\
87.3697173326134	-0.000397\\
87.4197200323974	-0.000693\\
87.4697227321814	-6.9e-05\\
87.5197254319655	-0.000607\\
87.5697281317495	-0.000568\\
87.6197308315335	0.001747\\
87.6697335313175	0.001617\\
87.7197362311015	0.000508\\
87.7697389308855	0.000241\\
87.8197416306695	0.00023\\
87.8697443304536	0.0005\\
87.9197470302376	-0.000248\\
87.9697497300216	-0.000481\\
88.0197524298056	-0.000898\\
88.0697551295896	-0.00091\\
88.1197578293737	-0.001023\\
88.1697605291577	-0.001069\\
88.2197632289417	-0.000959\\
88.2697659287257	-0.000366\\
88.3197686285097	0.000267\\
88.3697713282937	0.001655\\
88.4197740280778	0.001731\\
88.4697767278618	0.003801\\
88.5197794276458	0.00382\\
88.5697821274298	0.002682\\
88.6197848272138	0.000945\\
88.6697875269979	0.000375\\
88.7197902267819	4.5e-05\\
88.7697929265659	0.001715\\
88.8197956263499	0.00197\\
88.8697983261339	0.001677\\
88.9198010259179	0.001338\\
88.969803725702	0.000488\\
89.019806425486	0.0009\\
89.06980912527	0.000766\\
89.119811825054	0.000675\\
89.169814524838	0.001351\\
89.219817224622	0.002006\\
89.2698199244061	0.002146\\
89.3198226241901	0.002362\\
89.3698253239741	0.002307\\
89.4198280237581	0.001465\\
89.4698307235421	0.00156\\
89.5198334233261	0.001772\\
89.5698361231102	0.001424\\
89.6198388228942	0.001852\\
89.6698415226782	0.001046\\
89.7198442224622	0.00183\\
89.7698469222462	0.001939\\
89.8198496220302	0.002001\\
89.8698523218143	0.00098\\
89.9198550215983	-5.9e-05\\
89.9698577213823	-0.000762\\
90.0198604211663	8.2e-05\\
90.0698631209503	0.00051\\
90.1198658207343	-0.000239\\
90.1698685205184	-0.00082\\
90.2198712203024	-0.00081\\
90.2698739200864	0.000115\\
90.3198766198704	0.003246\\
90.3698793196544	0.00488\\
90.4198820194384	0.005064\\
90.4698847192225	0.002435\\
90.5198874190065	0.00157\\
90.5698901187905	0.000199\\
90.6198928185745	-7e-06\\
90.6698955183585	-0.000211\\
90.7198982181426	0.000249\\
90.7699009179266	-0.000424\\
90.8199036177106	-0.000941\\
90.8699063174946	-0.001261\\
90.9199090172786	-0.000779\\
90.9699117170626	-0.002059\\
91.0199144168467	-0.002635\\
91.0699171166307	-0.001968\\
91.1199198164147	-0.000395\\
91.1699225161987	0.000681\\
91.2199252159827	0.000896\\
91.2699279157667	0.001861\\
91.3199306155508	0.001365\\
91.3699333153348	0.00051\\
91.4199360151188	0.000113\\
91.4699387149028	0.000488\\
91.5199414146868	0.000654\\
91.5699441144708	0.001186\\
91.6199468142549	0.000824\\
91.6699495140389	0.001013\\
91.7199522138229	0.00069\\
91.7699549136069	0.002648\\
91.8199576133909	0.002392\\
91.8699603131749	0.002036\\
91.919963012959	0.001627\\
91.969965712743	0.001747\\
92.019968412527	0.002095\\
92.069971112311	0.001551\\
92.119973812095	0.002309\\
92.1699765118791	0.002372\\
92.2199792116631	0.001649\\
92.2699819114471	0.002331\\
92.3199846112311	0.001723\\
92.3699873110151	0.001579\\
92.4199900107991	0.002541\\
92.4699927105832	0.002829\\
92.5199954103672	0.001885\\
92.5699981101512	0.003252\\
92.605	0.003252\\
};
\addlegendentry{Estimated position [m]};

\addplot [color=blue,solid,line width=0.2pt]
  table[row sep=crcr]{0	-0.0749067719101089\\
0.0500026997840173	-0.0756575527496417\\
0.100005399568035	-0.0755463917366384\\
0.150008099352052	-0.0754819939799291\\
0.200010799136069	-0.0754424823058127\\
0.250013498920086	-0.0755106348605658\\
0.300016198704104	-0.0755070659702472\\
0.350018898488121	-0.0754657842409125\\
0.400021598272138	-0.0753940467724746\\
0.450024298056156	-0.0751573862684106\\
0.500026997840173	-0.0753222664025992\\
0.55002969762419	-0.075684518712868\\
0.600032397408207	-0.0756881669400238\\
0.650035097192225	-0.0755985775994107\\
0.700037796976242	-0.07560929083947\\
0.750040496760259	-0.0754711685200185\\
0.800043196544276	-0.0755896288163063\\
0.850045896328294	-0.0755178348716048\\
0.900048596112311	-0.0754425418342413\\
0.950051295896328	-0.0753779265865785\\
1.00005399568035	-0.0755356821504329\\
1.05005669546436	-0.0752613055781661\\
1.10005939524838	-0.075397633537239\\
1.1500620950324	-0.0757400724448426\\
1.20006479481641	-0.0756577405276306\\
1.25006749460043	-0.0755986035211804\\
1.30007019438445	-0.0754801426142594\\
1.35007289416847	-0.075196859228696\\
1.40007559395248	-0.0751913834723345\\
1.4500782937365	-0.075324119572121\\
1.50008099352052	-0.07524167792677\\
1.55008369330454	-0.0756630542406475\\
1.60008639308855	-0.0758100507059259\\
1.65008909287257	-0.0757096105790356\\
1.70009179265659	-0.0755895975825141\\
1.7500944924406	-0.0755322125550733\\
1.80009719222462	-0.0755231490336972\\
1.85009989200864	-0.0753617467508887\\
1.90010259179266	-0.0752578490561568\\
1.95010529157667	-0.0757687928378589\\
2.00010799136069	-0.0755965601809048\\
2.05011069114471	-0.0762455577957876\\
2.10011339092873	-0.0726628665586936\\
2.15011609071274	-0.0653693271160434\\
2.20011879049676	-0.0544194462585877\\
2.25012149028078	-0.0428974159908262\\
2.30012419006479	-0.0345433641546035\\
2.35012688984881	-0.0307157467317714\\
2.40012958963283	-0.0281870673800009\\
2.45013228941685	-0.0249247063455462\\
2.50013498920086	-0.0209469889515255\\
2.55013768898488	-0.0177849845374986\\
2.6001403887689	-0.0150774642480426\\
2.65014308855292	-0.0137183512275043\\
2.70014578833693	-0.0131682526312357\\
2.75014848812095	-0.0117226205947694\\
2.80015118790497	-0.0103904380092325\\
2.85015388768899	-0.00922714362317916\\
2.900156587473	-0.00824569123136266\\
2.95015928725702	-0.00735734185180289\\
3.00016198704104	-0.00691141401176108\\
3.05016468682505	-0.0064314624708795\\
3.10016738660907	-0.00584883465341678\\
3.15017008639309	-0.00575186768904735\\
3.20017278617711	-0.00563135917754207\\
3.25017548596112	-0.00571034804003099\\
3.30017818574514	-0.00511706579560307\\
3.35018088552916	-0.00470346959643639\\
3.40018358531318	-0.00483495896974648\\
3.45018628509719	-0.00509385439619457\\
3.50018898488121	-0.00537774069671827\\
3.55019168466523	-0.0048240401651585\\
3.60019438444924	-0.00405979815827595\\
3.65019708423326	-0.00367144993652071\\
3.70019978401728	-0.00355107045091932\\
3.7502024838013	-0.00313041447951463\\
3.80020518358531	-0.00247219680868018\\
3.85020788336933	-0.00202998694297108\\
3.90021058315335	-0.00139888902555756\\
3.95021328293736	-0.000708429778284784\\
4.00021598272138	-0.000312869911554552\\
4.0502186825054	-0.0005790396553982\\
4.10022138228942	-0.000836049662513396\\
4.15022408207343	-0.000981789379811676\\
4.20022678185745	-0.000246379813992724\\
4.25022948164147	-6.30099922993897e-05\\
4.30023218142549	0.000445929894498322\\
4.3502348812095	0.000174450119761246\\
4.40023758099352	0.000629349867364495\\
4.45024028077754	0.00117220930130371\\
4.50024298056155	0.00143476853460119\\
4.55024568034557	0.00150141914709917\\
4.60024838012959	0.00134848852764166\\
4.65025107991361	0.00163623835515118\\
4.70025377969762	0.00206417786312853\\
4.75025647948164	0.00248130669648662\\
4.80025917926566	0.00256577538035016\\
4.85026187904968	0.00252798535382201\\
4.90026457883369	0.00221510736971197\\
4.95026727861771	0.00188237788011579\\
5.00026997840173	0.00183204847945183\\
5.05027267818575	0.00191484830919665\\
5.10027537796976	0.00229254728696066\\
5.15027807775378	0.00278156379283156\\
5.2002807775378	0.00344854110048071\\
5.25028347732181	0.00381360960967787\\
5.30028617710583	0.00357973081268452\\
5.35028887688985	0.00345752188828196\\
5.40029157667387	0.0034863097553107\\
5.45029427645788	0.00338022158927959\\
5.5002969762419	0.00356714089445493\\
5.55029967602592	0.00337307059732386\\
5.60030237580994	0.0037505896846448\\
5.65030507559395	0.00366425036547586\\
5.70030777537797	0.00363008083642352\\
5.75031047516199	0.00352586078444241\\
5.800313174946	0.00376495747464059\\
5.85031587473002	0.00409410721593653\\
5.90031857451404	0.00397720776327857\\
5.95032127429806	0.00414442508379436\\
6.00032397408207	0.00434752436859698\\
6.05032667386609	0.00469445103656204\\
6.10032937365011	0.00470349107706353\\
6.15033207343413	0.00436377464668437\\
6.20033477321814	0.00379735889525196\\
6.25033747300216	0.00334426208278469\\
6.30034017278618	0.00301890337604044\\
6.3503428725702	0.00233376605724472\\
6.40034557235421	0.00173148828123556\\
6.45034827213823	0.00121363937391702\\
6.50035097192225	0.00113083954294815\\
6.55035367170626	0.00067608976409079\\
6.60035637149028	0.000359649799037169\\
6.6503590712743	-0.000469319777677215\\
6.70036177105832	-0.000604169807233929\\
6.75036447084233	-0.00067971958597613\\
6.80036717062635	-0.000582659069966439\\
6.85036987041037	-0.000665349433361255\\
6.90037257019439	-0.00103382926105933\\
6.9503752699784	-0.00170098892628634\\
7.00037796976242	-0.00155522873620708\\
7.05038066954644	-0.00108236901358709\\
7.10038336933045	-0.000677849513434459\\
7.15038606911447	-0.000179829947778432\\
7.20038876889849	-0.000267909928984278\\
7.25039146868251	-0.000127650031092487\\
7.30039416846652	0.000348829778529024\\
7.35039686825054	0.000639989738128178\\
7.40039956803456	0.00116678927838821\\
7.45040226781857	0.00183380797875145\\
7.50040496760259	0.00240929586277716\\
7.55040766738661	0.0026700456071208\\
7.60041036717063	0.00252429669239999\\
7.65041306695464	0.00199229810402074\\
7.70041576673866	0.00181060854596582\\
7.75041846652268	0.00188062790981115\\
7.8004211663067	0.00212343738400876\\
7.85042386609071	0.00192380731266607\\
7.90042656587473	0.00128923884119827\\
7.95042926565875	7.54799206101253e-05\\
8.00043196544276	-0.000650799634913074\\
8.05043466522678	-0.00117408888928762\\
8.1004373650108	-0.00150304833092228\\
8.15044006479482	-0.0021522171344814\\
8.20044276457883	-0.00317885347681124\\
8.25044546436285	-0.00416582620039677\\
8.30044816414687	-0.00477706030430347\\
8.35045086393089	-0.00531836128852041\\
8.4004535637149	-0.00496236401241241\\
8.45045626349892	-0.00413174634732054\\
8.50045896328294	-0.00310860412400277\\
8.55046166306696	-0.00258543608967034\\
8.60046436285097	-0.00263760475627263\\
8.65046706263499	-0.00247044629153745\\
8.70046976241901	-0.00192370799300143\\
8.75047246220302	-0.000911629516475116\\
8.80047516198704	-0.000181829699903031\\
8.85047786177106	0.000900789042198345\\
8.90048056155508	0.0019095668440927\\
8.95048326133909	0.00297379445880644\\
9.00048596112311	0.00333358285258419\\
9.05048866090713	0.00390355873692718\\
9.10049136069114	0.00434209461348836\\
9.15049406047516	0.0044105032495615\\
9.20049676025918	0.00468561960531963\\
9.2504994600432	0.00498053741970767\\
9.30050215982721	0.0053994223718181\\
9.35050485961123	0.00584706400516743\\
9.40050755939525	0.00575896506734892\\
9.45051025917927	0.00546941015793189\\
9.50051295896328	0.00544436001910002\\
9.5505156587473	0.00486555954972894\\
9.60051835853132	0.00413175614455598\\
9.65052105831533	0.00342335070781257\\
9.70052375809935	0.00345754965190532\\
9.75052645788337	0.00369661016272354\\
9.80052915766739	0.00429183598888206\\
9.8505318574514	0.0045003438742042\\
9.90053455723542	0.00469107012490006\\
9.95053725701944	0.00522869424075406\\
10.0005399568035	0.00566909859974647\\
10.0505426565875	0.00589197378142339\\
10.1005453563715	0.00578413634413553\\
10.1505480561555	0.0058291750820762\\
10.2005507559395	0.00595859153565075\\
10.2505534557235	0.00591901257978793\\
10.3005561555076	0.0056367267987096\\
10.3505588552916	0.00568357658257157\\
10.4005615550756	0.00562056899694706\\
10.4505642548596	0.00578943496028602\\
10.5005669546436	0.00605552949430342\\
10.5505696544276	0.00661300648173535\\
10.6005723542117	0.0068268715316075\\
10.6505750539957	0.00700321051556172\\
10.7005777537797	0.00739513071690921\\
10.7505804535637	0.00798303342142096\\
10.8005831533477	0.00830837154789158\\
10.8505858531317	0.00813405526012492\\
10.9005885529158	0.00801889732467205\\
10.9505912526998	0.00797936297252162\\
11.0005939524838	0.00805489127720868\\
11.0505966522678	0.00793811355732946\\
11.1005993520518	0.00787696459743851\\
11.1506020518359	0.0078374366830522\\
11.2006047516199	0.0081628350039113\\
11.2506074514039	0.00772597972670861\\
11.3006101511879	0.00742744927726425\\
11.3506128509719	0.00732139194216939\\
11.4006155507559	0.00708591878622384\\
11.45061825054	0.00698338943130212\\
11.500620950324	0.00712355846792163\\
11.550623650108	0.00656806874034679\\
11.600626349892	0.00668498784198731\\
11.650629049676	0.00605758015371286\\
11.70063174946	0.0057444448252748\\
11.7506344492441	0.00553598803698523\\
11.8006371490281	0.005395769644015\\
11.8506398488121	0.00518899538249185\\
11.9006425485961	0.0053940317341587\\
11.9506452483801	0.00557559854024851\\
12.0006479481641	0.00576629435182912\\
12.0506506479482	0.00589573367379907\\
12.1006533477322	0.00606812103310179\\
12.1506560475162	0.00593505363028023\\
12.2006587473002	0.00600351233950317\\
12.2506614470842	0.00616705837377553\\
12.3006641468683	0.00579851341459548\\
12.3506668466523	0.00545147024646954\\
12.4006695464363	0.005625879084251\\
12.4506722462203	0.00584712573548209\\
12.5006749460043	0.00574634603964122\\
12.5506776457883	0.00552335895131969\\
12.6006803455724	0.00538859074619017\\
12.6506830453564	0.00503785539425297\\
12.7006857451404	0.00503243801140823\\
12.7506884449244	0.00468189101033363\\
12.8006911447084	0.00430082299806487\\
12.8506938444924	0.00350421032530727\\
12.9006965442765	0.0024614864107951\\
12.9506992440605	0.00234460741733558\\
13.0007019438445	0.00285901489311952\\
13.0507046436285	0.00261620546990234\\
13.1007073434125	0.00268442514492828\\
13.1507100431965	0.00264306516826795\\
13.2007127429806	0.00230872720458827\\
13.2507154427646	0.0021827977121206\\
13.3007181425486	0.00243816698895168\\
13.3507208423326	0.00209289777212395\\
13.4007235421166	0.00233740635218109\\
13.4507262419006	0.00280305366232335\\
13.5007289416847	0.00277067529286448\\
13.5507316414687	0.00275458573435396\\
13.6007343412527	0.00283182547837866\\
13.6507370410367	0.0030638337666442\\
13.7007397408207	0.00323283219434706\\
13.7507424406048	0.00343057140178159\\
13.8007451403888	0.0031969534056526\\
13.8507478401728	0.00319861301619651\\
13.9007505399568	0.00300266421901345\\
13.9507532397408	0.00349168123155314\\
14.0007559395248	0.00376333873079369\\
14.0507586393089	0.00423608383689138\\
14.1007613390929	0.00472153760741455\\
14.1507640388769	0.00462442174792232\\
14.2007667386609	0.00448792325658788\\
14.2507694384449	0.0045004142193252\\
14.3007721382289	0.00475033028788265\\
14.350774838013	0.00511162397542966\\
14.400777537797	0.00503249685014477\\
14.450780237581	0.00418400617367782\\
14.500782937365	0.00336227124020106\\
14.550785637149	0.00272585462142152\\
14.600788336933	0.00223324679498001\\
14.6507910367171	0.00165782884638781\\
14.7007937365011	0.00165960857889543\\
14.7507964362851	0.000891809273292958\\
14.8007991360691	0.000503449821803863\\
14.8508018358531	-7.72598263329883e-05\\
14.9008045356372	-0.000524989729380052\\
14.9508072354212	-0.000350539489990333\\
15.0008099352052	-0.000242639592597299\\
15.0508126349892	0.000131270023620464\\
15.1008153347732	9.00999778047551e-06\\
15.1508180345572	-0.000213869846729365\\
15.2008207343413	-0.000381029714717765\\
15.2508234341253	0.000187029856140875\\
15.3008261339093	0.000985239164585737\\
15.3508288336933	0.00140957898752667\\
15.4008315334773	0.00147436905167427\\
15.4508342332613	0.00127646918384729\\
15.5008369330454	0.000679649669343693\\
15.5508396328294	-0.000183399887470766\\
15.6008423326134	-0.000359479889720392\\
15.6508450323974	-0.00032535959877671\\
15.7008477321814	-0.000523129613549287\\
15.7508504319654	-0.000783859553213259\\
15.8008531317495	-0.000789299707794974\\
15.8508558315335	-0.0010770294225889\\
15.9008585313175	-0.00144031864304545\\
15.9508612311015	-0.00185015790468893\\
16.0008639308855	-0.0020892474356563\\
16.0508666306695	-0.00203706797581211\\
16.1008693304536	-0.00175489868893561\\
16.1508720302376	-0.00142232869665297\\
16.2008747300216	-0.00133941854281566\\
16.2508774298056	-0.00148692834626524\\
16.3008801295896	-0.000890019669062207\\
16.3508828293737	-6.8370009657087e-05\\
16.4008855291577	0.00105725910850375\\
16.4508882289417	0.00168650850110484\\
16.5008909287257	0.00248124652279106\\
16.5508936285097	0.00268616607250207\\
16.6008963282937	0.00232291672702059\\
16.6508990280778	0.00225649710095432\\
16.7009017278618	0.00197054781448886\\
16.7509044276458	0.00184827845682872\\
16.8009071274298	0.00221687740795353\\
16.8509098272138	0.00213964702713628\\
16.9009125269978	0.00194718748019252\\
16.9509152267819	0.00184115847517189\\
17.0009179265659	0.00220450786234546\\
17.0509206263499	0.00254062624565287\\
17.1009233261339	0.00237705640968658\\
17.1509260259179	0.00266475514939011\\
17.2009287257019	0.00317537289540647\\
17.250931425486	0.00359416046776427\\
17.30093412527	0.00425771425306453\\
17.350936825054	0.00480415935936595\\
17.400939524838	0.00509197552660602\\
17.450942224622	0.00579136319276883\\
17.500944924406	0.0060322496742723\\
17.5509476241901	0.00591002400454603\\
17.6009503239741	0.00608798036753664\\
17.6509530237581	0.00641518345200116\\
17.7009557235421	0.00635413528889422\\
17.7509584233261	0.00613656914244406\\
17.8009611231102	0.006064571266275\\
17.8509638228942	0.00641335298189671\\
17.9009665226782	0.00658421083202115\\
17.9509692224622	0.00639913420711237\\
18.0009719222462	0.00590279374376315\\
18.0509746220302	0.00550003134815767\\
18.1009773218143	0.00530760269749266\\
18.1509800215983	0.00540107205864101\\
18.2009827213823	0.00559540596778667\\
18.2509854211663	0.00585971528192784\\
18.3009881209503	0.00592805169032435\\
18.3509908207343	0.00612030957310163\\
18.4009935205184	0.00620305797447331\\
18.4509962203024	0.00616359983378497\\
18.5009989200864	0.00570686372800143\\
18.5510016198704	0.00566191800160119\\
18.6010043196544	0.0052393436129763\\
18.6510070194384	0.00389245802818534\\
18.7010097192225	0.00317890120858389\\
18.7510124190065	0.0031464731119914\\
18.8010151187905	0.00283908459484358\\
18.8510178185745	0.00238783695361686\\
18.9010205183585	0.0026700062130619\\
18.9510232181425	0.00294156337315796\\
19.0010259179266	0.00334059319065943\\
19.0510286177106	0.00361032996086967\\
19.1010313174946	0.00394298696075862\\
19.1510340172786	0.00443748170465587\\
19.2010367170626	0.00424872462478818\\
19.2510394168467	0.00418217586376731\\
19.3010421166307	0.00448230419781984\\
19.3510448164147	0.00432765350822959\\
19.4010475161987	0.00418394609104758\\
19.4510502159827	0.0045057941798455\\
19.5010529157667	0.00486891723413525\\
19.5510556155508	0.00500374545666346\\
19.6010583153348	0.00508652731719281\\
19.6510610151188	0.00515840628204952\\
19.7010637149028	0.00532025234140701\\
19.7510664146868	0.00539558075737698\\
19.8010691144708	0.00571216671417141\\
19.8510718142549	0.0057804651641632\\
19.9010745140389	0.00538855912060207\\
19.9510772138229	0.00591895272869695\\
20.0010799136069	0.00594415387329483\\
20.0510826133909	0.00593514434938282\\
20.1010853131749	0.00569951505860091\\
20.151088012959	0.00604298666088811\\
20.201090712743	0.00586682504438395\\
20.251093412527	0.00545688235042049\\
20.301096112311	0.00517989625426309\\
20.351098812095	0.00530579272019762\\
20.4011015118791	0.00556477176752429\\
20.4511042116631	0.00552887989637731\\
20.5011069114471	0.00564928846398874\\
20.5511096112311	0.00527721051564733\\
20.6011123110151	0.00492290827400756\\
20.6511150107991	0.00480236039841602\\
20.7011177105832	0.00484381911603742\\
20.7511204103672	0.00517285921232203\\
20.8011231101512	0.00485642666793257\\
20.8511258099352	0.00454349227841524\\
20.9011285097192	0.00469630905831151\\
20.9511312095032	0.00431341548690792\\
21.0011339092873	0.00313386332241982\\
21.0511366090713	0.00230503700012473\\
21.1011393088553	0.00199932797195455\\
21.1511420086393	0.00252246684826434\\
21.2011447084233	0.00280296510736208\\
21.2511474082073	0.0020533269712745\\
21.3011501079914	0.00190418829130572\\
21.3511528077754	0.00165949875233573\\
21.4011555075594	0.00147428921674325\\
21.4511582073434	0.00148868884403736\\
21.5011609071274	0.00106613957271884\\
21.5511636069114	0.000719289688580044\\
21.6011663066955	0.000605909772445469\\
21.6511690064795	0.000185270072122915\\
21.7011717062635	-4.49099890778777e-05\\
21.7511744060475	-0.000634779446769759\\
21.8011771058315	-0.00176023783081997\\
21.8511798056156	-0.00293267437337592\\
21.9011825053996	-0.00377221723904167\\
21.9511852051836	-0.00392322776549649\\
22.0011879049676	-0.00446439887302389\\
22.0511906047516	-0.00562047359753456\\
22.1011933045356	-0.00668668864518672\\
22.1511960043197	-0.00756234226152629\\
22.2011987041037	-0.00842347693920958\\
22.2512014038877	-0.00948788368421323\\
22.3012041036717	-0.00959768103822648\\
22.3512068034557	-0.00950410401824841\\
22.4012095032397	-0.00864279957339023\\
22.4512122030238	-0.00777082862433696\\
22.5012149028078	-0.00694188627691208\\
22.5512176025918	-0.00667058764974037\\
22.6012203023758	-0.00594237331347719\\
22.6512230021598	-0.00569424725881398\\
22.7012257019438	-0.0053166225840675\\
22.7512284017279	-0.00435273282540516\\
22.8012311015119	-0.00330650142214388\\
22.8512338012959	-0.00249736673089154\\
22.9012365010799	-0.00231208694931201\\
22.9512392008639	-0.00219708765909093\\
23.0012419006479	-0.0017601987147139\\
23.051244600432	-0.00180878676022762\\
23.101247300216	-0.00138446879408728\\
23.15125	-0.000683279811616574\\
23.201252699784	-0.000426209596171138\\
23.251255399568	-0.000922379206049305\\
23.3012580993521	-0.000904539643661433\\
23.3512607991361	-0.000703059733203253\\
23.4012634989201	0.000291279830132503\\
23.4512661987041	0.000293009994653618\\
23.5012688984881	0.000183499837065832\\
23.5512715982721	0.000199669869154975\\
23.6012742980562	0.000330829907350269\\
23.6512769978402	0.000877299291191873\\
23.7012796976242	0.000882659655705361\\
23.7512823974082	0.000339829715217881\\
23.8012850971922	0.000323579872628062\\
23.8512877969762	0.000560989592868841\\
23.9012904967603	0.00133424928494087\\
23.9512931965443	0.00199208768806045\\
24.0012958963283	0.00304746347917761\\
24.0512985961123	0.00343963177213336\\
24.1013012958963	0.00338388182503185\\
24.1513039956803	0.00302422433184002\\
24.2013066954644	0.00279405551608377\\
24.2513093952484	0.00304029454027128\\
24.3013120950324	0.00327766276802533\\
24.3513147948164	0.00339289216229332\\
24.4013174946004	0.00350793142998796\\
24.4513201943845	0.0029900336611875\\
24.5013228941685	0.00274912550788514\\
24.5513255939525	0.00234646628786219\\
24.6013282937365	0.00222762717366508\\
24.6513309935205	0.00245069656379573\\
24.7013336933045	0.00272392556471188\\
24.7513363930886	0.00287312434467878\\
24.8013390928726	0.00364467993278136\\
24.8513417926566	0.00426846482277529\\
24.9013444924406	0.00425949406098344\\
24.9513471922246	0.00431334501480859\\
25.0013498920086	0.00439605413842229\\
25.0513525917927	0.00491204734217163\\
25.1013552915767	0.00524463370145238\\
25.1513579913607	0.00582552436931003\\
25.2013606911447	0.00567985717112449\\
25.2513633909287	0.00601967972365235\\
25.3013660907127	0.00603233176534936\\
25.3513687904968	0.00574471711316518\\
25.4013714902808	0.00558992622889527\\
25.4513741900648	0.00538133065885374\\
25.5013768898488	0.00502899628228773\\
25.5513795896328	0.00533467153965397\\
25.6013822894168	0.00542623111119736\\
25.6513849892009	0.00538315287928594\\
25.7013876889849	0.0054929397081371\\
25.7513903887689	0.00533828104981426\\
25.8013930885529	0.00513690557329559\\
25.8513957883369	0.00527888260450313\\
25.901398488121	0.00542457057739379\\
25.951401187905	0.00446086147071748\\
26.001403887689	0.00294862249540413\\
26.051406587473	0.00242372585260081\\
26.101409287257	0.00212707633642342\\
26.151411987041	0.00194183728102899\\
26.2014146868251	0.00255850590450707\\
26.2514173866091	0.00309269313183119\\
26.3014200863931	0.00300089453768856\\
26.3514227861771	0.00286612542595932\\
26.4014254859611	0.00237698688890595\\
26.4514281857451	0.00149958874320624\\
26.5014308855292	0.000458579703979448\\
26.5514335853132	0.000214039696794486\\
26.6014362850972	0.000550079369976169\\
26.6514389848812	0.000562609641191814\\
26.7014416846652	0.00037754985172562\\
26.7514443844492	0.000235649862124914\\
26.8014470842333	0.000427969665200734\\
26.8514497840173	0.000845139576806593\\
26.9014524838013	0.00123508888908566\\
26.9514551835853	0.00133940848276814\\
27.0014578833693	0.00112195872085287\\
27.0514605831534	0.000854009321760929\\
27.1014632829374	0.00110230915982468\\
27.1514659827214	0.00161652876177592\\
27.2014686825054	0.00227620723155335\\
27.2514713822894	0.00249744600693547\\
27.3014740820734	0.00243101613316888\\
27.3514767818575	0.00252445588599725\\
27.4014794816415	0.00274185542076253\\
27.4514821814255	0.00320938252485714\\
27.5014848812095	0.00343243152494549\\
27.5514875809935	0.0033281820415451\\
27.6014902807775	0.0033676021510743\\
27.6514929805616	0.00383869920534139\\
27.7014956803456	0.00519795415252214\\
27.7514983801296	0.00604481093774999\\
27.8015010799136	0.00632357652892712\\
27.8515037796976	0.00637205482835713\\
27.9015064794816	0.00639906393372067\\
27.9515091792657	0.00618496788575335\\
28.0015118790497	0.00605011039369309\\
28.0515145788337	0.00590102358395891\\
28.1015172786177	0.00539577070385783\\
28.1515199784017	0.00494093748665625\\
28.2015226781857	0.00485271951563599\\
28.2515253779698	0.00494633824803935\\
28.3015280777538	0.00509017649197831\\
28.3515307775378	0.0048707888261675\\
28.4015334773218	0.00417305663539301\\
28.4515361771058	0.00346645126854507\\
28.5015388768899	0.00277430430457159\\
28.5515415766739	0.00219903654294797\\
28.6015442764579	0.0013628586944874\\
28.6515469762419	0.000609399738529948\\
28.7015496760259	0.000424409897677328\\
28.7515523758099	0.000257199926784578\\
28.801555075594	-0.000300269713257299\\
28.851557775378	-0.000807279357423429\\
28.901560475162	-0.00147437882891631\\
28.951563174946	-0.00206219756514822\\
29.00156587473	-0.00242730643745416\\
29.051568574514	-0.00304584352863491\\
29.1015712742981	-0.00336942223407156\\
29.1515739740821	-0.00356720042703173\\
29.2015766738661	-0.00369660917123532\\
29.2515793736501	-0.00409044549999944\\
29.3015820734341	-0.00389077887153352\\
29.3515847732181	-0.00391068833624874\\
29.4015874730022	-0.00457968112662776\\
29.4515901727862	-0.00555944763101826\\
29.5015928725702	-0.0055683483910907\\
29.5515955723542	-0.00528610144531621\\
29.6015982721382	-0.00494642733178722\\
29.6516009719222	-0.00434203406009637\\
29.7016036717063	-0.00401487669103709\\
29.7516063714903	-0.00369494954444185\\
29.8016090712743	-0.00344680167686239\\
29.8516117710583	-0.00285152434063921\\
29.9016144708423	-0.00192738709972887\\
29.9516171706264	-0.0015462190395103\\
30.0016198704104	-0.00208903667733709\\
30.0516225701944	-0.00188240829751046\\
30.1016252699784	-0.00164520795980134\\
30.1516279697624	-0.00166853848167276\\
30.2016306695464	-0.00180690836465789\\
30.2516333693305	-0.00209106811928193\\
30.3016360691145	-0.00164347866771455\\
30.3516387688985	-0.00107523931664185\\
30.4016414686825	-0.000661679621226868\\
30.4516441684665	0.000224689915381493\\
30.5016468682505	0.000598779899376188\\
30.5516495680346	0.000724669758866384\\
30.6016522678186	0.000738919666674803\\
30.6516549676026	0.000496279637794361\\
30.7016576673866	0.000773259692716715\\
30.7516603671706	0.00108968933440783\\
30.8016630669546	0.00146538886216005\\
30.8516657667387	0.00137005901465024\\
30.9016684665227	0.00127660939645029\\
30.9516711663067	0.00172248855634045\\
31.0016738660907	0.00229426705440814\\
31.0516765658747	0.00248131679196432\\
31.1016792656587	0.00233740666053548\\
31.1516819654428	0.00226543748952018\\
31.2016846652268	0.0015731583037256\\
31.2516873650108	0.00119392929768067\\
31.3016900647948	0.000787459672460606\\
31.3516927645788	0.00099778936737817\\
31.4016954643629	0.00200842757799059\\
31.4516981641469	0.00224397723492307\\
31.5017008639309	0.00273642559510133\\
31.5517035637149	0.00327953303227104\\
31.6017062634989	0.0033982419296411\\
31.6517089632829	0.00432236363166546\\
31.701711663067	0.00492645716808901\\
31.751714362851	0.00577515532821822\\
31.801717062635	0.00645297169425284\\
31.851719762419	0.0074238991021229\\
31.901722462203	0.00821492380230794\\
31.951725161987	0.00870037637713264\\
32.0017278617711	0.0089466457051069\\
32.0517305615551	0.00927034303882632\\
32.1017332613391	0.00942674730853918\\
32.1517359611231	0.00950397299057843\\
32.2017386609071	0.00901486351539632\\
32.2517413606911	0.0085601513426075\\
32.3017440604752	0.0082276013610388\\
32.3517467602592	0.00794348029564186\\
32.4017494600432	0.00788971330447455\\
32.4517521598272	0.00734668058175393\\
32.5017548596112	0.00667606701884247\\
32.5517575593952	0.00589210217084455\\
32.6017602591793	0.00547140024737316\\
32.6517629589633	0.00522669404171772\\
32.7017656587473	0.00492651539021049\\
32.7517683585313	0.0042180651981794\\
32.8017710583153	0.00388183884366414\\
32.8517737580993	0.00379196929803827\\
32.9017764578834	0.00387828916932585\\
32.9517791576674	0.00433307405089926\\
33.0017818574514	0.0048526288558031\\
33.0517845572354	0.00520522355454912\\
33.1017872570194	0.00517638545550738\\
33.1517899568035	0.00520337364677479\\
33.2017926565875	0.00446444246368493\\
33.2517953563715	0.00404539708232453\\
33.3017980561555	0.0032417523116352\\
33.3518007559395	0.00330108317583371\\
33.4018034557235	0.00292892451958916\\
33.4518061555076	0.00238063661839234\\
33.5018088552916	0.00222242752488762\\
33.5518115550756	0.00201735793342532\\
33.6018142548596	0.00242012549989068\\
33.6518169546436	0.00208747721211707\\
33.7018196544277	0.00153909863093976\\
33.7518223542117	0.0012351589310057\\
33.8018250539957	0.000526670060462438\\
33.8518277537797	0.000640149741576534\\
33.9018304535637	0.00090620950017878\\
33.9518331533477	0.00139886917636172\\
34.0018358531318	0.00180696827719675\\
34.0518385529158	0.00235884698452354\\
34.1018412526998	0.00238757680239016\\
34.1518439524838	0.00281214485910528\\
34.2018466522678	0.00271138573153159\\
34.2518493520518	0.00290553470997876\\
34.3018520518359	0.00302783421751983\\
34.3518547516199	0.00357968144832602\\
34.4018574514039	0.00433315272128077\\
34.4518601511879	0.00491932620099826\\
34.5018628509719	0.00490491936017063\\
34.5518655507559	0.00471797025052832\\
34.60186825054	0.00496784784030365\\
34.651870950324	0.00512426575760255\\
34.701873650108	0.00540302101685182\\
34.751876349892	0.00511719487777197\\
34.801879049676	0.00530969294837072\\
34.85188174946	0.00512424570014062\\
34.9018844492441	0.00481493047084053\\
34.9518871490281	0.00505949704861202\\
35.0018898488121	0.00543526173109744\\
35.0518925485961	0.00566374789012745\\
35.1018952483801	0.00591905410253261\\
35.1518979481641	0.00580752530239062\\
35.2019006479482	0.00595328317091551\\
35.2519033477322	0.0062642769004534\\
35.3019060475162	0.00648724067393887\\
35.3519087473002	0.00639002294669871\\
35.4019114470842	0.00619579773228755\\
35.4519141468683	0.00621746646880665\\
35.5019168466523	0.00631642569470169\\
35.5519195464363	0.00613473801544346\\
35.6019222462203	0.00598012124262992\\
35.6519249460043	0.00601790050461425\\
35.7019276457883	0.00604495165217911\\
35.7519303455724	0.00597649304155158\\
35.8019330453564	0.00583632559660245\\
35.8519357451404	0.00573919489269142\\
35.9019384449244	0.00558110801468417\\
35.9519411447084	0.00534008234113301\\
36.0019438444924	0.00549458078840487\\
36.0519465442765	0.0056888674134802\\
36.1019492440605	0.00590288198363911\\
36.1519519438445	0.00575708511573964\\
36.2019546436285	0.00587777304969314\\
36.2519573434125	0.00590830377927248\\
36.3019600431965	0.00595141198642355\\
36.3519627429806	0.00590285152312133\\
36.4019654427646	0.00555046935881876\\
36.4519681425486	0.00521422526520973\\
36.5019708423326	0.00496791807295861\\
36.5519735421166	0.0045326223762739\\
36.6019762419006	0.00424491525572717\\
36.6519789416847	0.00334608173831331\\
36.7019816414687	0.00309965385426933\\
36.7519843412527	0.00261598495283963\\
36.8019870410367	0.00237695678537874\\
36.8519897408207	0.00262503582656922\\
36.9019924406048	0.00253864615488404\\
36.9519951403888	0.00228880717450668\\
37.0019978401728	0.00142751835636407\\
37.0520005399568	0.0003757201082467\\
37.1020032397408	5.75202021288718e-05\\
37.1520059395248	0.000384709809858219\\
37.2020086393089	0.000693959637150849\\
37.2520113390929	0.00122252920905154\\
37.3020140388769	0.00169184790198186\\
37.3520167386609	0.00179806827677754\\
37.4020194384449	0.00192208834049123\\
37.4520221382289	0.0018986682249785\\
37.502024838013	0.00126213895347824\\
37.552027537797	0.000356029868759906\\
37.602030237581	0.000438529513249345\\
37.652032937365	0.000449649749579412\\
37.702035637149	0.000886429507537473\\
37.752038336933	0.00130895867259276\\
37.8020410367171	0.00198507780632461\\
37.8520437365011	0.00234274643857579\\
37.9020464362851	0.00278506437823052\\
37.9520491360691	0.00326860236335206\\
38.0020518358531	0.00379371949994972\\
38.0520545356372	0.00369478929009691\\
38.1020572354212	0.00352582990986555\\
38.1520599352052	0.00375593894846816\\
38.2020626349892	0.00419286568241482\\
38.2520653347732	0.00461532150735514\\
38.3020680345572	0.00433489392248629\\
38.3520707343413	0.00456151308378561\\
38.4020734341253	0.00469088053628568\\
38.4520761339093	0.00444277250122623\\
38.5020788336933	0.00443743371814253\\
38.5520815334773	0.00459398125886192\\
38.6020842332613	0.00514222587576821\\
38.6520869330454	0.00477718030834715\\
38.7020896328294	0.00442306361084213\\
38.7520923326134	0.00467309062272553\\
38.8020950323974	0.00490486770907312\\
38.8520977321814	0.00480419918584166\\
38.9021004319654	0.00456693245711129\\
38.9521031317495	0.00471970034402126\\
39.0021058315335	0.00443915103149122\\
39.0521085313175	0.00459197128609839\\
39.1021112311015	0.00471787996161244\\
39.1521139308855	0.0046747610728386\\
39.2021166306696	0.00471435000436613\\
39.2521193304536	0.00401149688417061\\
39.3021220302376	0.00425594534659075\\
39.3521247300216	0.00389089701142924\\
39.4021274298056	0.00322913327345712\\
39.4521301295896	0.00265206543182994\\
39.5021328293737	0.00147975836109475\\
39.5521355291577	0.000251629811654192\\
39.6021382289417	-0.000195959896192169\\
39.6521409287257	-0.000508939605322318\\
39.7021436285097	-0.000794769295509856\\
39.7521463282937	-0.00109690892743403\\
39.8021490280778	-0.000618519511049038\\
39.8521517278618	-0.000373959801902219\\
39.9021544276458	0.000341629886748533\\
39.9521571274298	0.0015084487992411\\
40.0021598272138	0.00201369789093214\\
40.0521625269978	0.00221148762521484\\
40.1021652267819	0.00236620695748794\\
40.1521679265659	0.00216486756244311\\
40.2021706263499	0.00175126853189745\\
40.2521733261339	0.00172249782426752\\
40.3021760259179	0.00251180601809696\\
40.3521787257019	0.003137403298325\\
40.402181425486	0.00351323094244012\\
40.45218412527	0.00310880381685368\\
40.502186825054	0.00300264396792672\\
40.552189524838	0.00280841529006896\\
40.602192224622	0.00265751564958432\\
40.652194924406	0.00280850529174357\\
40.7021976241901	0.00285356538104834\\
40.7522003239741	0.00305294340439655\\
40.8022030237581	0.00285875509664806\\
40.8522057235421	0.00214331697248389\\
40.9022084233261	0.00129272876301639\\
40.9522111231102	0.00143475891559517\\
41.0022138228942	0.00200474739997376\\
41.0522165226782	0.00270767428946571\\
41.1022192224622	0.00294687302766485\\
41.1522219222462	0.00246319656078214\\
41.2022246220302	0.00187699806489323\\
41.2522273218143	0.0017888983648984\\
41.3022300215983	0.00143126902524554\\
41.3522327213823	0.000769619582045679\\
41.4022354211663	0.000237249876907835\\
41.4522381209503	-0.000271599753943099\\
41.5022408207343	-0.000649229505605303\\
41.5522435205184	-0.000127719768687107\\
41.6022462203024	0.000683219626375215\\
41.6522489200864	0.00128566892514395\\
41.7022516198704	0.00148513855018477\\
41.7522543196544	0.00176395790543268\\
41.8022570194385	0.00302055329641319\\
41.8522597192225	0.0038064287574006\\
41.9022624190065	0.00413356671983891\\
41.9522651187905	0.00368573894300746\\
42.0022678185745	0.00357445035956827\\
42.0522705183585	0.00386024946583087\\
42.1022732181426	0.00422526609184498\\
42.1522759179266	0.00457776281849368\\
42.2022786177106	0.0043638137093383\\
42.2522813174946	0.00454349276497787\\
42.3022840172786	0.00451663390717768\\
42.3522867170626	0.00444112130882181\\
42.4022894168467	0.00416953536806037\\
42.4522921166307	0.00466582111658377\\
42.5022948164147	0.00450770192757692\\
42.5522975161987	0.00433500458469667\\
42.6023002159827	0.00428276462297503\\
42.6523029157667	0.00388724781381283\\
42.7023056155508	0.00337306249996738\\
42.7523083153348	0.00276715545650488\\
42.8023110151188	0.00262315652112114\\
42.8523137149028	0.00293435382815401\\
42.9023164146868	0.00254779535671025\\
42.9523191144708	0.00235898696223635\\
43.0023218142549	0.00186078830750413\\
43.0523245140389	0.00128192916081836\\
43.1023272138229	0.0006508696609716\\
43.1523299136069	-0.000582509590815581\\
43.2023326133909	-0.0018248774012112\\
43.2523353131749	-0.00226916667362878\\
43.302338012959	-0.00278865373355642\\
43.352340712743	-0.00279585459994314\\
43.402343412527	-0.0032885824522709\\
43.452346112311	-0.00396999868065576\\
43.502348812095	-0.00437817383077025\\
43.5523515118791	-0.00423973618199759\\
43.6023542116631	-0.00408136703572331\\
43.6523569114471	-0.00336220223904603\\
43.7023596112311	-0.00351497065544172\\
43.7523623110151	-0.00337489255934782\\
43.8023650107991	-0.00363746065500374\\
43.8523677105832	-0.00337667161880986\\
43.9023704103672	-0.00326692266930452\\
43.9523731101512	-0.00357073113031836\\
44.0023758099352	-0.00311583306423877\\
44.0523785097192	-0.00268971583403272\\
44.1023812095032	-0.00257650581368028\\
44.1523839092873	-0.00292723440316876\\
44.2023866090713	-0.00281032574431531\\
44.2523893088553	-0.0027112950860048\\
44.3023920086393	-0.00293085451976324\\
44.3523947084233	-0.00337111189129592\\
44.4023974082073	-0.00352588037371459\\
44.4524001079914	-0.00404735663034087\\
44.5024028077754	-0.00330119203658418\\
44.5524055075594	-0.00239498619890092\\
44.6024082073434	-0.00182318822435234\\
44.6524109071274	-0.0016327186619111\\
44.7024136069114	-0.00154450897065468\\
44.7524163066955	-0.00100499952045453\\
44.8024190064795	-0.000591479774582077\\
44.8524217062635	-0.000424429839627637\\
44.9024244060475	-0.000363209716596425\\
44.9524271058315	-0.000264309720565549\\
45.0024298056155	-3.60698305887373e-05\\
45.0524325053996	0.000188700094289764\\
45.1024352051836	0.000176160282847857\\
45.1524379049676	0.000507079858725142\\
45.2024406047516	0.00025539989767019\\
45.2524433045356	0.000361489825815174\\
45.3024460043197	0.000476629871251091\\
45.3524487041037	0.000620299735768219\\
45.4024514038877	0.000632889640619591\\
45.4524541036717	0.000433339968929961\\
45.5024568034557	0.000190569984372465\\
45.5524595032397	-0.000133069933037499\\
45.6024622030238	5.57802689794284e-05\\
45.6524649028078	0.000744339485824126\\
45.7024676025918	0.00100868948637548\\
45.7524703023758	0.00126219910546192\\
45.8024730021598	0.0016721476023496\\
45.8524757019439	0.00148343854339895\\
45.9024784017279	0.000924289520510235\\
45.9524811015119	0.000226419996922166\\
46.0024838012959	6.11400047985896e-05\\
46.0524865010799	-7.55499949421792e-05\\
46.1024892008639	0.00013316998789784\\
46.152491900648	-0.000352339658275781\\
46.202494600432	-0.000474599764417297\\
46.252497300216	-0.00074621931414939\\
46.3025	-0.000879259477395778\\
46.352502699784	-0.000658109858804642\\
46.402505399568	-0.000593219186858262\\
46.4525080993521	-0.000920599231488993\\
46.5025107991361	-0.00113992946063732\\
46.5525134989201	-0.0011236692959578\\
46.6025161987041	-0.00068867959171149\\
46.6525188984881	-0.000535759779240732\\
46.7025215982721	-6.30801608565412e-05\\
46.7525242980562	-7.91399981448291e-05\\
46.8025269978402	-0.00033810964329721\\
46.8525296976242	-0.000600489652576696\\
46.9025323974082	-0.00124602908456639\\
46.9525350971922	-0.00189149793775259\\
47.0025377969762	-0.00201906783278563\\
47.0525404967603	-0.00210016748770153\\
47.1025431965443	-0.00227435698662607\\
47.1525458963283	-0.00259089534395604\\
47.2025485961123	-0.00307990301636289\\
47.2525512958963	-0.00293450481358719\\
47.3025539956803	-0.00323996250365947\\
47.3525566954644	-0.00352579036346967\\
47.4025593952484	-0.00439960356811501\\
47.4525620950324	-0.0045058513960571\\
47.5025647948164	-0.00433492415988929\\
47.5525674946004	-0.00444105340959154\\
47.6025701943845	-0.00409048697860514\\
47.6525728941685	-0.00334429131692292\\
47.7025755939525	-0.00309435393979714\\
47.7525782937365	-0.00230316641359661\\
47.8025809935205	-0.00127289906527233\\
47.8525836933045	-0.000674209783795028\\
47.9025863930886	-1.43500385224513e-05\\
47.9525890928726	0.000548359842153551\\
48.0025917926566	0.0010967493762624\\
48.0525944924406	0.00171708765235766\\
48.1025971922246	0.00230311651108682\\
48.1525998920086	0.00292723418306\\
48.2026025917927	0.00389445851159452\\
48.2526052915767	0.00453987222823123\\
48.3026079913607	0.00459211215953575\\
48.3526106911447	0.00442126353425929\\
48.4026133909287	0.0048653394910996\\
48.4526160907127	0.00494633766804388\\
48.5026187904968	0.00502708705030539\\
48.5526214902808	0.0046747206174938\\
48.6026241900648	0.00442130486781359\\
48.6526268898488	0.00413008564644757\\
48.7026295896328	0.00421450427938129\\
48.7526322894168	0.00449678219488376\\
48.8026349892009	0.00471438962464723\\
48.8526376889849	0.00428818556102667\\
48.9026403887689	0.00416593589708221\\
48.9526430885529	0.00383509717116442\\
49.0026457883369	0.00307453374203581\\
49.052648488121	0.00218451707970831\\
49.102651187905	0.00214313691955187\\
49.152653887689	0.00211085727990739\\
49.202656587473	0.00201914748782639\\
49.252659287257	0.00187535706382901\\
49.302661987041	0.000960089537344542\\
49.3526646868251	0.00115785921459375\\
49.4026673866091	0.0017782683235262\\
49.4526700863931	0.00162898853573733\\
49.5026727861771	0.00169179854438567\\
49.5526754859611	0.00168833794459515\\
49.6026781857451	0.0017889571952132\\
49.6526808855292	0.00165040867697787\\
49.7026835853132	0.00194722806751218\\
49.7526862850972	0.00211628692888515\\
49.8026889848812	0.00171522838953027\\
49.8526916846652	0.00194180735960732\\
49.9026943844492	0.00174578832698388\\
49.9526970842333	0.00221501750009263\\
50.0026997840173	0.00219347745321166\\
50.0527024838013	0.00222415708389657\\
50.1027051835853	0.00175305834118241\\
50.1527078833693	0.00114002951328118\\
50.2027105831534	0.000843169408893043\\
50.2527132829374	0.000647179529573017\\
50.3027159827214	0.00134115851637777\\
50.3527186825054	0.00198855791893899\\
50.4027213822894	0.00216669643758935\\
50.4527240820734	0.00252440498362477\\
50.5027267818575	0.00259627604106286\\
50.5527294816415	0.00224392730385893\\
50.6027321814255	0.001884448080172\\
50.6527348812095	0.00110927936035129\\
50.7027375809935	0.0014690290298229\\
50.7527402807775	0.00125861900930967\\
50.8027429805616	0.000897399358710261\\
50.8527456803456	0.000490919775444955\\
50.9027483801296	7.00799181265671e-05\\
50.9527510799136	-0.000510569831979365\\
51.0027537796976	-0.000631129729216842\\
51.0527564794816	-0.000600579690294316\\
51.1027591792657	-0.000348719656193978\\
51.1527618790497	-0.000327179476639057\\
51.2027645788337	-0.000221189907494259\\
51.2527672786177	-0.000151029998258198\\
51.3027699784017	-0.000129409950710511\\
51.3527726781858	-0.000264249938517082\\
51.4027753779698	-0.000233829978992817\\
51.4527780777538	0.000154569961317225\\
51.5027807775378	0.000332540004261411\\
51.5527834773218	6.47201584153381e-05\\
51.6027861771058	0.000177979952722506\\
51.6527888768899	0.000231919916805988\\
51.7027915766739	0.000280469925258247\\
51.7527942764579	0.000239099971425318\\
51.8027969762419	0.000602249595044983\\
51.8527996760259	0.00070300952998984\\
51.9028023758099	0.000348879852118325\\
51.952805075594	3.41200151886253e-05\\
52.002807775378	-0.000170919989100204\\
52.052810475162	-5.2190005358168e-05\\
52.102813174946	0.000474719858712922\\
52.15281587473	0.00111285935688178\\
52.202818574514	0.00163623845600037\\
52.2528212742981	0.00236976668816891\\
52.3028239740821	0.00227454682930199\\
52.3528266738661	0.00218441689787237\\
52.4028293736501	0.00205505740265017\\
52.4528320734341	0.00232676685205987\\
52.5028347732181	0.00283374471294499\\
52.5528374730022	0.00338213090457357\\
52.6028401727862	0.00347010096198404\\
52.6528428725702	0.00308363367616927\\
52.7028455723542	0.00330110178975981\\
52.7528482721382	0.00345581129041286\\
52.8028509719222	0.00290540309004659\\
52.8528536717063	0.00266814499704332\\
52.9028563714903	0.00281565500927903\\
52.9528590712743	0.00208384690586102\\
53.0028617710583	0.00133410872375079\\
53.0528644708423	0.00116699886135287\\
53.1028671706264	0.000284220353132462\\
53.1528698704104	0.000181679927191552\\
53.2028725701944	-0.000203129701990933\\
53.2528752699784	0.000320109924441037\\
53.3028779697624	-8.99198129256683e-05\\
53.3528806695464	-0.00049986963067135\\
53.4028833693305	-0.000920459536657\\
53.4528860691145	-0.000640039823458378\\
53.5028887688985	-0.000237349937412331\\
53.5528914686825	0.000255409876868627\\
53.6028941684665	0.000706639600517943\\
53.6528968682505	0.00133583905157533\\
53.7028995680346	0.0014815977191148\\
53.7529022678186	0.00178550788763731\\
53.8029049676026	0.00153726826925826\\
53.8529076673866	0.000285819692207322\\
53.9029103671706	-0.000426129737918173\\
53.9529130669547	-0.000881039637203463\\
54.0029157667387	-0.000476469864959895\\
54.0529184665227	-0.000751669435273273\\
54.1029211663067	-0.000767869542033553\\
54.1529238660907	-0.00129278914574383\\
54.2029265658747	-0.00202628779789695\\
54.2529292656588	-0.00297752400697056\\
54.3029319654428	-0.00382801884888528\\
54.3529346652268	-0.00415162556875782\\
54.4029373650108	-0.00368052933594275\\
54.4529400647948	-0.00365698995329947\\
54.5029427645788	-0.0040563154700856\\
54.5529454643629	-0.00472141964214133\\
54.6029481641469	-0.00404721708616086\\
54.6529508639309	-0.00301516405810238\\
54.7029535637149	-0.00285332511483205\\
54.7529562634989	-0.00294329469427896\\
54.8029589632829	-0.00335507120949101\\
54.852961663067	-0.00390703721405686\\
54.902964362851	-0.00331005317330301\\
54.952967062635	-0.00235719604092604\\
55.002969762419	-0.00212173744220744\\
55.052972462203	-0.00242370617805109\\
55.102975161987	-0.00293063403253336\\
55.1529778617711	-0.00344495182112723\\
55.2029805615551	-0.00274186402208192\\
55.2529832613391	-0.00175833802991206\\
55.3029859611231	-0.000906179602444737\\
55.3529886609071	-0.000458479794210678\\
55.4029913606911	-0.000215609801766176\\
55.4529940604752	0.000165430070748296\\
55.5029967602592	0.000587839904501372\\
55.5529994600432	0.000790949731627559\\
55.6030021598272	0.000872039416211696\\
55.6530048596112	0.00105366936145565\\
55.7030075593953	0.00144913885979256\\
55.7530102591793	0.00197593800268725\\
55.8030129589633	0.00262327636080257\\
55.8530156587473	0.00260170585973311\\
55.9030183585313	0.00273293472677179\\
55.9530210583153	0.00307640399613353\\
56.0030237580994	0.00415328535223158\\
56.0530264578834	0.00505965552493703\\
56.1030291576674	0.00570499553413222\\
56.1530318574514	0.00589560277591765\\
56.2030345572354	0.00537074122625559\\
56.2530372570194	0.00525934398197636\\
56.3030399568035	0.00522145339648332\\
56.3530426565875	0.00563142687374216\\
56.4030453563715	0.00548936738991504\\
56.4530480561555	0.00533651283469107\\
56.5030507559395	0.00500932686318114\\
56.5530534557235	0.00546411086599676\\
56.6030561555076	0.00576801596854608\\
56.6530588552916	0.00547673115483838\\
56.7030615550756	0.0049319077870376\\
56.7530642548596	0.00437278341398136\\
56.8030669546436	0.00471426076751116\\
56.8530696544276	0.00467650120627834\\
56.9030723542117	0.00489054565877607\\
56.9530750539957	0.00463889162594013\\
57.0030777537797	0.00403651646423361\\
57.0530804535637	0.00375594958531539\\
57.1030831533477	0.00435108558318214\\
57.1530858531318	0.00438002487969307\\
57.2030885529158	0.0045040127206874\\
57.2530912526998	0.00404909804593291\\
57.3030939524838	0.00281913479748756\\
57.3530966522678	0.00220237636332837\\
57.4030993520518	0.00210723768347943\\
57.4531020518358	0.00185902755558906\\
57.5031047516199	0.00191483753913722\\
57.5531074514039	0.0013863389442798\\
57.6031101511879	0.00138267898508889\\
57.6531128509719	0.0013412389205864\\
57.7031155507559	0.000981709447922823\\
57.75311825054	0.00118835933671663\\
57.803120950324	0.00133763876707803\\
57.853123650108	0.00165602810473862\\
57.903126349892	0.00183218751423801\\
57.953129049676	0.00221506753286109\\
58.00313174946	0.00194900816273889\\
58.0531344492441	0.00147251872789674\\
58.1031371490281	0.00107687948746148\\
58.1531398488121	0.00164340888797416\\
58.2031425485961	0.00245072697592513\\
58.2531452483801	0.00297391371142186\\
58.3031479481642	0.00314095278610468\\
58.3531506479482	0.00270412551897773\\
58.4031533477322	0.00186990680233634\\
58.4531560475162	0.00137737923772818\\
58.5031587473002	0.00093324972465021\\
58.5531614470842	0.000584339862454044\\
58.6031641468683	0.000127670019611553\\
58.6531668466523	-0.00039733978316605\\
58.7031695464363	-0.000809009666513718\\
58.7531722462203	-0.00133241829414712\\
58.8031749460043	-0.00204433752864902\\
58.8531776457883	-0.00247040625326514\\
58.9031803455724	-0.00342701154484009\\
58.9531830453564	-0.00386749702959152\\
59.0031857451404	-0.00373987950651004\\
59.0531884449244	-0.00343410228485632\\
59.1031911447084	-0.00274552400491497\\
59.1531938444924	-0.0020983272769289\\
59.2031965442765	-0.00173871832367263\\
59.2531992440605	-0.0013862688991717\\
59.3032019438445	-0.000987209512049864\\
59.3532046436285	-0.00041356979463572\\
59.4032073434125	0.000384719935650402\\
59.4532100431965	0.000909779325083631\\
59.5032127429806	0.000733559680435983\\
59.5532154427646	0.00101956934750335\\
59.6032181425486	0.00108060929596085\\
59.6532208423326	0.00129271918405674\\
59.7032235421166	0.00158398868364583\\
59.7532262419006	0.00181954828606523\\
59.8032289416847	0.00159659872175605\\
59.8532316414687	0.00103551927073854\\
59.9032343412527	0.000773049442948285\\
59.9532370410367	0.000571769714363663\\
60.0032397408207	0.000521469847180079\\
60.0532424406048	0.000780319318524992\\
60.1032451403888	0.00151751876226227\\
60.1532478401728	0.00153545856836991\\
60.2032505399568	0.001341248983216\\
60.2532532397408	0.00123160900508831\\
60.3032559395248	0.00131412870617014\\
60.3532586393089	0.00125492900510075\\
60.4032613390929	0.00059331967283991\\
60.4532640388769	-0.000282289746829835\\
60.5032667386609	-0.000958349166287875\\
60.5532694384449	-0.00125128937347248\\
60.6032721382289	-0.00114527931822214\\
60.653274838013	-0.000695939463907662\\
60.703277537797	-5.56900749343967e-05\\
60.753280237581	0.000436939442653485\\
60.803282937365	0.000906199531019596\\
60.853285637149	0.00148501889647518\\
60.9032883369331	0.00202448753273262\\
60.9532910367171	0.00252433641711433\\
61.0032937365011	0.00288764507782959\\
61.0532964362851	0.00357087079043846\\
61.1032991360691	0.00439795442000325\\
61.1533018358531	0.00448411290875431\\
61.2033045356371	0.00456507284757473\\
61.2533072354212	0.00451116279979938\\
61.3033099352052	0.00498580631086683\\
61.3533126349892	0.00516917418032289\\
61.4033153347732	0.00522297393019774\\
61.4533180345572	0.00566893766234594\\
61.5033207343413	0.00568702732932883\\
61.5533234341253	0.00577339605712297\\
61.6033261339093	0.00581823398624873\\
61.6533288336933	0.00583261364872622\\
61.7033315334773	0.00642425375363245\\
61.7533342332613	0.00638108299943514\\
61.8033369330454	0.00587392427279314\\
61.8533396328294	0.00558984756623299\\
61.9033423326134	0.00532385241643266\\
61.9533450323974	0.00534015075260241\\
62.0033477321814	0.00525371340870952\\
62.0533504319654	0.00533467286513017\\
62.1033531317495	0.00570509489316111\\
62.1533558315335	0.00513508431931235\\
62.2033585313175	0.00458495184243939\\
62.2533612311015	0.00428279313849757\\
62.3033639308855	0.00410131657483443\\
62.3533666306695	0.0038693381282825\\
62.4033693304536	0.00290382430229006\\
62.4533720302376	0.00248838663485624\\
62.5033747300216	0.00256201674308039\\
62.5533774298056	0.00293969373360033\\
62.6033801295896	0.00730158037406515\\
62.6533828293737	0.0209804584179161\\
62.6983852591793	0.0443165373128708\\
62.7433876889849	0.0773065720009305\\
62.7883901187905	0.107975776811472\\
62.8333925485961	0.132401873201575\\
62.8833952483801	0.151120863574408\\
62.9333979481641	0.161097951541105\\
62.9834006479482	0.167220659514066\\
63.0334033477322	0.174147197072698\\
63.0834060475162	0.183101929436818\\
63.1334087473002	0.191018891849376\\
63.1834114470842	0.196952668310739\\
63.2334141468683	0.20137008400624\\
63.2834168466523	0.2055281673067\\
63.3334195464363	0.209942728578313\\
63.3834222462203	0.213768035539077\\
63.4334249460043	0.21765871193351\\
63.4834276457883	0.220498912423392\\
63.5334303455724	0.222248807200975\\
63.5834330453564	0.223994402523092\\
63.6334357451404	0.225933825208333\\
63.6834384449244	0.226659074170141\\
63.7334411447084	0.227287712377391\\
63.7834438444924	0.228044111394442\\
63.8334465442765	0.228850803410512\\
63.8834492440605	0.228639097044047\\
63.9334519438445	0.22836246570446\\
63.9834546436285	0.228726745914892\\
64.0334573434125	0.228702082984959\\
64.0834600431965	0.228714222501722\\
64.1334627429806	0.228775612763585\\
64.1834654427646	0.22849902076264\\
64.2334681425486	0.227783169831619\\
64.2834708423326	0.227508275400174\\
64.3334735421166	0.227084606585917\\
64.3834762419007	0.226946348299605\\
64.4334789416847	0.226387543666201\\
64.4834816414687	0.226709732458104\\
64.5334843412527	0.227109004501926\\
64.5834870410367	0.227324459452586\\
64.6334897408207	0.227436576353541\\
64.6834924406048	0.227529219394336\\
64.7334951403888	0.226932039512644\\
64.7834978401728	0.226926746519794\\
64.8335005399568	0.226450733094179\\
64.8835032397408	0.227110785782134\\
64.9335059395248	0.22693746069593\\
64.9835086393089	0.226872687185046\\
65.0335113390929	0.226907733881642\\
65.0835140388769	0.227067041086207\\
65.1335167386609	0.227273712781603\\
65.1835194384449	0.227434715332753\\
65.2335221382289	0.227522273562301\\
65.283524838013	0.227471527756065\\
65.333527537797	0.227376846635155\\
65.383530237581	0.226872572709928\\
65.433532937365	0.227201924381569\\
65.483535637149	0.227235125397334\\
65.533538336933	0.227123026696113\\
65.5835410367171	0.227219347096309\\
65.6335437365011	0.226926821155741\\
65.6835464362851	0.226165088597998\\
65.7335491360691	0.225879490673085\\
65.7835518358531	0.225332937617479\\
65.8335545356372	0.225282246640105\\
65.8835572354212	0.224537774169903\\
65.9335599352052	0.224637606495278\\
65.9835626349892	0.224721799161723\\
66.0335653347732	0.22449215129886\\
66.0835680345572	0.224555079092082\\
66.1335707343412	0.225028024457346\\
66.1835734341253	0.225566080452077\\
66.2335761339093	0.225658819646868\\
66.2835788336933	0.225650094448491\\
66.3335815334773	0.225729092744312\\
66.3835842332613	0.225558787218479\\
66.4335869330454	0.225280391134651\\
66.4835896328294	0.225583456119156\\
66.5335923326134	0.225373412038953\\
66.5835950323974	0.225396041968847\\
66.6335977321814	0.225467957918357\\
66.6836004319654	0.225781515901777\\
66.7336031317495	0.225864009053866\\
66.7836058315335	0.225877951538374\\
66.8336085313175	0.225930399908062\\
66.8836112311015	0.225872458582879\\
66.9336139308855	0.226042467536164\\
66.9836166306695	0.226017913809417\\
67.0336193304536	0.226476886042622\\
67.0836220302376	0.226872527843495\\
67.1336247300216	0.226732422861491\\
67.1836274298056	0.22703012353998\\
67.2336301295896	0.227443163443539\\
67.2836328293737	0.227122988095744\\
67.3336355291577	0.227364690212217\\
67.3836382289417	0.227038910309663\\
67.4336409287257	0.227480376230828\\
67.4836436285097	0.227207120462992\\
67.5336463282937	0.22724375202725\\
67.5836490280778	0.227257646508258\\
67.6336517278618	0.227499491978991\\
67.6836544276458	0.227250789714039\\
67.7336571274298	0.227385699959371\\
67.7836598272138	0.227154650139796\\
67.8336625269979	0.227406620744997\\
67.8836652267819	0.22714934087652\\
67.9336679265659	0.227618602463614\\
67.9836706263499	0.227690235351587\\
68.0336733261339	0.227837173379935\\
68.0836760259179	0.228408125072835\\
68.1336787257019	0.228448291047751\\
68.183681425486	0.228079093063157\\
68.23368412527	0.227327815626012\\
68.283686825054	0.227052969375949\\
68.333689524838	0.226746614721429\\
68.383692224622	0.226434829804418\\
68.4336949244061	0.226610061979202\\
68.4836976241901	0.226340232830462\\
68.5337003239741	0.226646742762452\\
68.5837030237581	0.226580297187269\\
68.6337057235421	0.226695794527674\\
68.6837084233261	0.226257919852844\\
68.7337111231102	0.226205363321338\\
68.7837138228942	0.226515440713694\\
68.8337165226782	0.227003758094325\\
68.8837192224622	0.22711941622174\\
68.9337219222462	0.226741358759102\\
68.9837246220302	0.226956661451517\\
69.0337273218143	0.227177204943962\\
69.0837300215983	0.227655213972991\\
69.1337327213823	0.227326048888493\\
69.1837354211663	0.227376923985825\\
69.2337381209503	0.227319130717815\\
69.2837408207344	0.227578138183414\\
69.3337435205184	0.227347142970338\\
69.3837462203024	0.227541600837216\\
69.4337489200864	0.227370026449022\\
69.4837516198704	0.227312096753621\\
69.5337543196544	0.227441571157644\\
69.5837570194385	0.226905672147285\\
69.6337597192225	0.226669433326853\\
69.6837624190065	0.227047662009919\\
69.7337651187905	0.227348916321184\\
69.7837678185745	0.227159729101177\\
69.8337705183585	0.227154504937698\\
69.8837732181426	0.227024971077747\\
69.9337759179266	0.227494328659743\\
69.9837786177106	0.228135054219434\\
70.0337813174946	0.228261066731367\\
70.0837840172786	0.228077234370484\\
70.1337867170626	0.227301634220992\\
70.1837894168467	0.227028491548167\\
70.2337921166307	0.227103847987583\\
70.2837948164147	0.226867610939524\\
70.3337975161987	0.226723826073888\\
70.3838002159827	0.226595946763771\\
70.4338029157667	0.226816707711521\\
70.4838056155508	0.226930546123243\\
70.5338083153348	0.226906095921723\\
70.5838110151188	0.226709811050348\\
70.6338137149028	0.226830568400871\\
70.6838164146868	0.226718556109747\\
70.7338191144708	0.226506590256718\\
70.7838218142549	0.22648717111665\\
70.8338245140389	0.226750052453913\\
70.8838272138229	0.22698999273731\\
70.9338299136069	0.226867371123008\\
70.9838326133909	0.227039039599266\\
71.0338353131749	0.227427616418656\\
71.083838012959	0.227355957791133\\
71.133840712743	0.227515182888692\\
71.183843412527	0.227588768293458\\
71.233846112311	0.22743666743276\\
71.283848812095	0.227453878608007\\
71.3338515118791	0.227035431167785\\
71.3838542116631	0.227163287213078\\
71.4338569114471	0.227007489486918\\
71.4838596112311	0.227107179704546\\
71.5338623110151	0.227576491447949\\
71.5838650107991	0.227963450897876\\
71.6338677105831	0.22733665864003\\
71.6838704103672	0.227156352042792\\
71.7338731101512	0.22706339700041\\
71.7838758099352	0.22723145512094\\
71.8338785097192	0.226988198986808\\
71.8838812095032	0.2270477962494\\
71.9338839092873	0.227231715816556\\
71.9838866090713	0.227184321815122\\
72.0338893088553	0.227410199135695\\
72.0838920086393	0.22780775648971\\
72.1338947084233	0.227720129519331\\
72.1838974082073	0.227544977516418\\
72.2339001079914	0.227557131988372\\
72.2839028077754	0.227385664799045\\
72.3339055075594	0.227179060240081\\
72.3839082073434	0.226925134230673\\
72.4339109071274	0.227417117025614\\
72.4839136069115	0.227709582878879\\
72.5339163066955	0.227914323622725\\
72.5839190064795	0.227886356675259\\
72.6339217062635	0.227930154822992\\
72.6839244060475	0.227623901936983\\
72.7339271058315	0.227476676062433\\
72.7839298056156	0.227126553568641\\
72.8339325053996	0.227140658575291\\
72.8839352051836	0.227156236381577\\
72.9339379049676	0.22750823894483\\
72.9839406047516	0.227685203287089\\
73.0339433045356	0.227965323619949\\
73.0839460043197	0.227651878075114\\
73.1339487041037	0.227261425086144\\
73.1839514038877	0.226797461698874\\
73.2339541036717	0.226683531397192\\
73.2839568034557	0.227082832842588\\
73.3339595032398	0.227469901691643\\
73.3839622030238	0.227735821254563\\
73.4339649028078	0.22799158539563\\
73.4839676025918	0.227291128335084\\
73.5339703023758	0.226709768563354\\
73.5839730021598	0.226578654098223\\
73.6339757019438	0.226060103194358\\
73.6839784017279	0.226079312896566\\
73.7339811015119	0.226103898691942\\
73.7839838012959	0.226361325420596\\
73.8339865010799	0.226210754210587\\
73.8839892008639	0.22654875189556\\
73.933991900648	0.226375154311748\\
73.983994600432	0.226068778024761\\
74.033997300216	0.226187827759521\\
74.084	0.226275517337194\\
74.134002699784	0.226534628114258\\
74.184005399568	0.226536490918794\\
74.2340080993521	0.226983050977593\\
74.2840107991361	0.227200203238592\\
74.3340134989201	0.227219376662465\\
74.3840161987041	0.227357807776448\\
74.4340188984881	0.226969115053647\\
74.4840215982721	0.226541760186137\\
74.5340242980562	0.226678328429215\\
74.5840269978402	0.227186166227889\\
74.6340296976242	0.227532753582388\\
74.6840323974082	0.22711949294522\\
74.7340350971922	0.227131880975317\\
74.7840377969763	0.226886778470079\\
74.8340404967603	0.226618809860777\\
74.8840431965443	0.226103943890954\\
74.9340458963283	0.225951650651653\\
74.9840485961123	0.22597612800623\\
75.0340512958963	0.226012857081478\\
75.0840539956804	0.226372025233828\\
75.1340566954644	0.226319454436018\\
75.1840593952484	0.22624228638923\\
75.2340620950324	0.226079456372338\\
75.2840647948164	0.226047846078021\\
75.3340674946004	0.226051481237267\\
75.3840701943844	0.226291184471901\\
75.4340728941685	0.226352576639098\\
75.4840755939525	0.226352466419687\\
75.5340782937365	0.22633845015395\\
75.5840809935205	0.225923416112247\\
75.6340836933045	0.225732483505411\\
75.6840863930885	0.225742954972336\\
75.7340890928726	0.225862195875967\\
75.7840917926566	0.225758868919044\\
75.8340944924406	0.225594184963269\\
75.8840971922246	0.226042576618594\\
75.9340998920086	0.226207176563544\\
75.9841025917927	0.226674869987872\\
76.0341052915767	0.226569705227704\\
76.0841079913607	0.227065315844649\\
76.1341106911447	0.227263319470776\\
76.1841133909287	0.226709958379363\\
76.2341160907127	0.226834205941133\\
76.2841187904968	0.226527585686737\\
76.3341214902808	0.226814751098542\\
76.3841241900648	0.227170267270074\\
76.4341268898488	0.227326137955795\\
76.4841295896328	0.227769222171422\\
76.5341322894169	0.227658930105112\\
76.5841349892009	0.226972365566557\\
76.6341376889849	0.227063461430736\\
76.6841403887689	0.227070619105458\\
76.7341430885529	0.226737788411929\\
76.7841457883369	0.226785103425983\\
76.834148488121	0.227254352535118\\
76.884151187905	0.227622100696805\\
76.934153887689	0.227966967802951\\
76.984156587473	0.227459260151877\\
77.034159287257	0.227060126490483\\
77.084161987041	0.227142345400423\\
77.1341646868251	0.227147709389291\\
77.1841673866091	0.227287597684309\\
77.2341700863931	0.227809399484808\\
77.2841727861771	0.228040580442081\\
77.3341754859611	0.228240115045913\\
77.3841781857451	0.228287217031012\\
77.4341808855292	0.227583506489992\\
77.4841835853132	0.226907613790637\\
77.5341862850972	0.226198471084379\\
77.5841889848812	0.225800863252245\\
77.6341916846652	0.226292992588555\\
77.6841943844492	0.22652577233064\\
77.7341970842333	0.227168435203466\\
77.7841997840173	0.227560737529728\\
77.8342024838013	0.228019517279264\\
77.8842051835853	0.228166585510011\\
77.9342078833693	0.227392672471546\\
77.9842105831534	0.226646741080907\\
78.0342132829374	0.226371912043483\\
78.0842159827214	0.226004089530256\\
78.1342186825054	0.225835870025411\\
78.1842213822894	0.225839347922292\\
78.2342240820734	0.226100361072936\\
78.2842267818575	0.225748386030993\\
78.3342294816415	0.22581482153241\\
78.3842321814255	0.225532996866684\\
78.4342348812095	0.225489135494228\\
78.4842375809935	0.225728958328691\\
78.5342402807775	0.225883030352626\\
78.5842429805616	0.225883007659046\\
78.6342456803456	0.225860322397814\\
78.6842483801296	0.225615203108192\\
78.7342510799136	0.225350777965655\\
78.7842537796976	0.224830278935497\\
78.8342564794817	0.224536129369535\\
78.8842591792657	0.224483340143306\\
78.9342618790497	0.224557023322254\\
78.9842645788337	0.224467732484615\\
79.0342672786177	0.224190881247234\\
79.0842699784017	0.223959488736878\\
79.1342726781857	0.224111956333133\\
79.1842753779698	0.224388786617838\\
79.2342780777538	0.224527287664327\\
79.2842807775378	0.225043970277815\\
79.3342834773218	0.225284168587219\\
79.3842861771058	0.225355935248789\\
79.4342888768899	0.225492677255886\\
79.4842915766739	0.225170297187574\\
79.5342942764579	0.22496006401053\\
79.5842969762419	0.225044109560655\\
79.6342996760259	0.225352387514199\\
79.6843023758099	0.225422355400427\\
79.734305075594	0.225089612561671\\
79.784307775378	0.225105532850016\\
79.834310475162	0.224742923307069\\
79.884313174946	0.224753243234981\\
79.93431587473	0.224961699124244\\
79.984318574514	0.225319050886594\\
80.0343212742981	0.225415481210633\\
80.0843239740821	0.225853359755869\\
80.1343266738661	0.225639547277676\\
80.1843293736501	0.225744644360478\\
80.2343320734341	0.225858624266574\\
80.2843347732181	0.226373645830824\\
80.3343374730022	0.22670633837153\\
80.3843401727862	0.226366543418611\\
80.4343428725702	0.22560990033447\\
80.4843455723542	0.225361187951978\\
80.5343482721382	0.225508190431281\\
80.5843509719223	0.225769249165602\\
80.6343536717063	0.225797450016728\\
80.6843563714903	0.225779892118661\\
80.7343590712743	0.226098417023271\\
80.7843617710583	0.226525720871782\\
80.8343644708423	0.22626657100547\\
80.8843671706263	0.22615298150602\\
80.9343698704104	0.226177463583559\\
80.9843725701944	0.226345501555486\\
81.0343752699784	0.22653997323741\\
81.0843779697624	0.226518858993858\\
81.1343806695464	0.226487330802165\\
81.1843833693305	0.227072265561315\\
81.2343860691145	0.227279066251018\\
81.2843887688985	0.227531120229253\\
81.3343914686825	0.227711366428718\\
81.3843941684665	0.227644809087872\\
81.4343968682505	0.227389222295694\\
81.4843995680346	0.226760493708642\\
81.5344022678186	0.226371815538025\\
81.5844049676026	0.226065362503017\\
81.6344076673866	0.226237042191789\\
81.6844103671706	0.226089931916124\\
81.7344130669546	0.225720230860688\\
81.7844157667387	0.22549262045075\\
81.8344184665227	0.225643221108665\\
81.8844211663067	0.226023165353749\\
81.9344238660907	0.225974189084588\\
81.9844265658747	0.22573951252602\\
82.0344292656588	0.225089765593038\\
82.0844319654428	0.225072055829493\\
82.1344346652268	0.225434593042995\\
82.1844373650108	0.225707920790958\\
82.2344400647948	0.226284179412601\\
82.2844427645788	0.226189574606463\\
82.3344454643629	0.225515362037069\\
82.3844481641469	0.224933759103168\\
82.4344508639309	0.224837505188081\\
82.4844535637149	0.224968693661684\\
82.5344562634989	0.224891617880986\\
82.5844589632829	0.224318602284448\\
82.634461663067	0.223935175873868\\
82.684464362851	0.22035270169412\\
82.734467062635	0.207916524503569\\
82.784469762419	0.183328983417589\\
82.8294721922246	0.153412554190566\\
82.8744746220302	0.12306992036011\\
82.9194770518359	0.0996001673622788\\
82.9694797516199	0.0835282550061769\\
83.0194824514039	0.0740236799496861\\
83.0694851511879	0.0640471313885887\\
83.1194878509719	0.0531646232911571\\
83.1694905507559	0.0428059383998715\\
83.21949325054	0.0352712113741453\\
83.269495950324	0.0305683245920346\\
83.319498650108	0.0265279956712241\\
83.369501349892	0.0224460666228536\\
83.419504049676	0.0182972973913731\\
83.4695067494601	0.0153471979930889\\
83.5195094492441	0.0136158592903449\\
83.5695121490281	0.0119780296127673\\
83.6195148488121	0.0104244701914272\\
83.6695175485961	0.00904560210827476\\
83.7195202483801	0.00828147419313903\\
83.7695229481642	0.00735016887896697\\
83.8195256479482	0.007080625498911\\
83.8695283477322	0.00631999370314961\\
83.9195310475162	0.00531497205592703\\
83.9695337473002	0.00489239725917009\\
84.0195364470842	0.00448783180672328\\
84.0695391468683	0.00415345575496839\\
84.1195418466523	0.00358529011788749\\
84.1695445464363	0.00279418464566026\\
84.2195472462203	0.00212527764728735\\
84.2695499460043	0.00263225599875939\\
84.3195526457883	0.00377037962636581\\
84.3695553455724	0.00448961307073124\\
84.4195580453564	0.00433671471979191\\
84.4695607451404	0.00421999477444365\\
84.5195634449244	0.00443564368657821\\
84.5695661447084	0.00436913291121546\\
84.6195688444924	0.00520353342184274\\
84.6695715442765	0.00544432980749576\\
84.7195742440605	0.0056420470379737\\
84.7695769438445	0.00569967692191078\\
84.8195796436285	0.00574993661657542\\
84.8695823434125	0.00574649726373412\\
84.9195850431965	0.00573732698007875\\
84.9695877429806	0.00569234759153559\\
85.0195904427646	0.00564924806138927\\
85.0695931425486	0.00554143978801434\\
85.1195958423326	0.00566186775627214\\
85.1695985421166	0.00587409438145352\\
85.2196012419006	0.00582723545126923\\
85.2696039416847	0.00558444789600639\\
85.3196066414687	0.00548398075420764\\
85.3696093412527	0.00536334303105489\\
85.4196120410367	0.00559358981062458\\
85.4696147408207	0.00530966112351205\\
85.5196174406048	0.00533113191316668\\
85.5696201403888	0.0051675845428977\\
85.6196228401728	0.00550733916916022\\
85.6696255399568	0.00558094940450933\\
85.7196282397408	0.00563136891314282\\
85.7696309395248	0.0055774186954233\\
85.8196336393089	0.00566194855139695\\
85.8696363390929	0.00567428720235431\\
85.9196390388769	0.0053757925867606\\
85.9696417386609	0.00508650600980657\\
86.0196444384449	0.00411563611157582\\
86.0696471382289	0.00420916470850511\\
86.119649838013	0.00469822052844111\\
86.169652537797	0.00515118574319424\\
86.219655237581	0.00558447830372627\\
86.269657937365	0.00565826705706104\\
86.319660637149	0.00565469723049222\\
86.369663336933	0.00594239166309741\\
86.4196660367171	0.00647463328721985\\
86.4696687365011	0.00692219203157614\\
86.5196714362851	0.00714340651498583\\
86.5696741360691	0.00701212803431326\\
86.6196768358531	0.00684671257649742\\
86.6696795356372	0.0063954935961842\\
86.7196822354212	0.00634147460325669\\
86.7696849352052	0.00645842249476133\\
86.8196876349892	0.00644944239785992\\
86.8696903347732	0.00632904538683023\\
86.9196930345572	0.00592630228214516\\
86.9696957343413	0.00569604666040618\\
87.0196984341253	0.00574103695938795\\
87.0697011339093	0.00593697255520646\\
87.1197038336933	0.00630010605433688\\
87.1697065334773	0.00576972556807831\\
87.2197092332613	0.0052591332004222\\
87.2697119330454	0.00507735581727484\\
87.3197146328294	0.00518545437518709\\
87.3697173326134	0.00569443521090498\\
87.4197200323974	0.00558627965336371\\
87.4697227321814	0.00554142821307241\\
87.5197254319655	0.00556125918506151\\
87.5697281317495	0.00574109659625791\\
87.6197308315335	0.00598550313166618\\
87.6697335313175	0.00605017032841072\\
87.7197362311015	0.00582539429039633\\
87.7697389308855	0.00547856050937998\\
87.8197416306695	0.00597107952395287\\
87.8697443304536	0.00575900556352167\\
87.9197470302376	0.00560434771982412\\
87.9697497300216	0.00531131212919511\\
88.0197524298056	0.00559360949780868\\
88.0697551295896	0.00571385787175204\\
88.1197578293737	0.00573193635581792\\
88.1697605291577	0.00591189395675378\\
88.2197632289417	0.00610056030840477\\
88.2697659287257	0.00631454623234289\\
88.3197686285097	0.00639546412215314\\
88.3697713282937	0.00615614753873913\\
88.4197740280778	0.00597483097353473\\
88.4697767278618	0.005645726602368\\
88.5197794276458	0.00506311604119508\\
88.5697821274298	0.00440143372962299\\
88.6197848272138	0.00386395829387582\\
88.6697875269979	0.00376686943519728\\
88.7197902267819	0.00443562269655846\\
88.7697929265659	0.00509179384860369\\
88.8197956263499	0.00489408742832628\\
88.8697983261339	0.0045704517534367\\
88.9198010259179	0.00470699967205324\\
88.969803725702	0.00452548325542936\\
89.019806425486	0.0046945306376808\\
89.06980912527	0.00500741712408223\\
89.119811825054	0.0052375347252177\\
89.169814524838	0.0053255827683078\\
89.219817224622	0.00536343111505793\\
89.2698199244061	0.00518547549966283\\
89.3198226241901	0.00481851024787414\\
89.3698253239741	0.00480767921814178\\
89.4198280237581	0.00476294109881334\\
89.4698307235421	0.0047289004136007\\
89.5198334233261	0.00484561571415186\\
89.5698361231102	0.00483838946767312\\
89.6198388228942	0.00469281112172522\\
89.6698415226782	0.00442127429495032\\
89.7198442224622	0.00458299225642242\\
89.7698469222462	0.00455964087958895\\
89.8198496220302	0.0043961344652495\\
89.8698523218143	0.0038891471054401\\
89.9198550215983	0.00355650045240237\\
89.9698577213823	0.00360313053698644\\
90.0198604211663	0.00388185816533716\\
90.0698631209503	0.00404547601205708\\
90.1198658207343	0.00398791813281693\\
90.1698685205184	0.00413541695858906\\
90.2198712203024	0.00427017546977268\\
90.2698739200864	0.00488698903273049\\
90.3198766198704	0.00526268239567088\\
90.3698793196544	0.00478797758899237\\
90.4198820194384	0.00294157309306403\\
90.4698847192225	0.00150843833897814\\
90.5198874190065	0.000742709661971891\\
90.5698901187905	0.00087552959325023\\
90.6198928185745	0.00152655883007551\\
90.6698955183585	0.00217370736264713\\
90.7198982181426	0.00288207520349131\\
90.7699009179266	0.00305294427411397\\
90.8199036177106	0.00334256250167673\\
90.8699063174946	0.00425227544946609\\
90.9199090172786	0.00477191882215444\\
90.9699117170626	0.00515125500376009\\
91.0199144168467	0.0056366180481516\\
91.0699171166307	0.00667959577381927\\
91.1199198164147	0.0073322011123369\\
91.1699225161987	0.00780138910052387\\
91.2199252159827	0.00789672627052344\\
91.2699279157667	0.00738978010240725\\
91.3199306155508	0.00739852858948746\\
91.3699333153348	0.00751375742176313\\
91.4199360151188	0.00749391645678491\\
91.4699387149028	0.00777444750897286\\
91.5199414146868	0.00818815406393797\\
91.5699441144708	0.00828143290751964\\
91.6199468142549	0.00830476184911301\\
91.6699495140389	0.00843438619274514\\
91.7199522138229	0.00866986886449887\\
91.7699549136069	0.00859065090246094\\
91.8199576133909	0.00868423634309281\\
91.8699603131749	0.00799196061403987\\
91.919963012959	0.00767013220219464\\
91.969965712743	0.0074958261159595\\
92.019968412527	0.00745080583998396\\
92.069971112311	0.00739513043255994\\
92.119973812095	0.00740409904751166\\
92.1699765118791	0.00702485917642045\\
92.2199792116631	0.00649975256353147\\
92.2699819114471	0.00651950206378776\\
92.3199846112311	0.00627858656317666\\
92.3699873110151	0.00617968769436338\\
92.4199900107991	0.00621938835006525\\
92.4699927105832	0.00579686477326045\\
92.5199954103672	0.00524848408128662\\
92.5699981101512	0.00525540557448442\\
92.605	0.00521397637557539\\
};
\addlegendentry{Measured angle [rad]};

\end{axis}
\end{tikzpicture}%
	}
	\caption{(a): Our implementation using the IR proximity sensor as reference, sampling at \(\sim\)20Hz versus (b): 
	Hermansens's implementation using the IR proximity sensor as reference, sampling at 200Hz. For (a) the slow variations in 
altitude are a result of the measurements from the IR proximity sensor being quantized in 2cm steps giving the arm
free reign to drift within that range. The noise in (b) is not measurement noise, but rapid oscillation.}
\label{fig:irtest}
\vspace{3pt}
\hrulefill
\end{figure*}

Let \(x_k\) be the current measurement sample, \(w\) the number of measurements taken into the moving average and \(Z\) the threshold value.
A measurement is then accepted if 
\begin{equation*}
	Z >\frac{1}{w} \left\lvert \sum_{i=k-w}^{k}x_i \right\rvert
\end{equation*}
This improves stability greatly as we rely somewhat heavily on the low frequency distance measurements in the kalman filter.
For the kalman filter we use the state space vector where the acceleration, velocity and position are perpendicular to the ground.
\begin{equation*}
	\hat{\mathbf{x}} = \begin{bmatrix}
		a_{off} \\
		v \\
		x
	\end{bmatrix}
\end{equation*}
We include the term \(a_{off}\) because the accelerometer is not only affected by random noise with zero mean, but also a 
steady bias that depends on temperature as well as other factors. The \(a_{off}\) term is an offset that is modified along
with the velocity and position to yield a more accurate state estimation.
The state transition model where \(T\) is the sample period for the accelerometer.
\begin{equation*}
	\mathbf{F} = \begin{bmatrix}
		1 & 0 & 0 \\
		T & 1 & 0 \\
		0 & T & 1 
	\end{bmatrix}
\end{equation*}
Our control input model, where the control input is actually the acceleration, is.
\begin{equation*}
	\mathbf{B} = \begin{bmatrix}
		0 \\
		T \\
		0 
	\end{bmatrix}
\end{equation*}
And since our measurements give us the absolute height of the arm the observation matrix becomes:
\begin{equation*}
	\mathbf{H} = \begin{bmatrix}
		0 \\
		0 \\
		1
	\end{bmatrix}
\end{equation*}
Because the kalman filter is static, we need different kalman gains when using the potentiometer (P), ultrasound sensor (ult) and IR sensor (IR).
We have determined the following kalman gains by manual tuning.
\begin{equation*}
	\mathbf{K}_{ult} = \begin{bmatrix}
		0.025 \\
		0.075 \\
		0.0625
	\end{bmatrix}, \quad
	\mathbf{K}_{IR} = \begin{bmatrix}
		0.025 \\
		0.075 \\
		0.0625
	\end{bmatrix}
\end{equation*}
The potentiometer requires more aggressive gain constants when sampled at lower frequencies. We use the following kalman gains for 1Hz
and 20Hz sampling frequencies. 
\begin{equation*}
	\mathbf{K}_{P,1} = \begin{bmatrix}
		0.25 \\
		0.75 \\
		0.625 
	\end{bmatrix},\quad
	\mathbf{K}_{P,20} = \begin{bmatrix}
		0.025 \\
		0.075 \\
		0.0625
	\end{bmatrix}
\end{equation*}

The output from the kalman filter is used to contruct a full one dimensional state estimate in the form
\begin{equation*}
	{\mathbf{x}}_{est} = \begin{bmatrix}
		x \\
		v \\
		a
	\end{bmatrix}
\end{equation*}
Where \(x\) is the estimated height, \(v\) the velocity along the vertical axis and \(a\) the acceleration along the vertical axis. 
This state estimate is then used with an LQR controller to compute the control output for the motor.
\subsection{Controller}\label{sec:controller}
We employ a static LQR controller with the following gain constants:
\begin{equation*}
	\mathbf{K}_{LQR} = \begin{bmatrix}
		1000 \\
		275 \\
		0 
	\end{bmatrix}
\end{equation*}
such that the output $u$ is given by:
\begin{equation*}
u = \mathbf x _{est}^{\mbox T}\mathbf K_{LQR} + u_{off}
\end{equation*}
where $u_{off}$ is an offset given to counter the effect of gravity on the lever
arm.

The controller is essentially a PD controller since it uses only the current position and the derivative thereof, velocity, to calculate
the next control signal.
