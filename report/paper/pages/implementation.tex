\section{Implementation}
Our implementation consists of a realtime linux desktop machine, handling the localisation and control logic, and a microcontroller
positioned on the arm. The microcontroller handles measurements from the MEMs accelerometer and rate gyroscope as well as the ultrasound
distance sensor. The IR distance sensor is connected directly to the desktop machine through a National Instruments ADC/DAC, as is the 
potentiometer at the base of the arm which is used for reference measurements.

We have made several changes to the original setup by Hermansen. The most important of these are

\begin{itemize}
		\item A complete overhaul of the communications subsystem in order to support sending accelerometer, rate gyroscope and 
			distance measurements at 1kHz and avoid intermittent transmission errors. 
		\item The original complementary filter has been replaced by a static kalman filter for state estimation and a linear 
			quadratic regulator for motor command control.
			\item An ultrasound sensor has been added to the arm for distance measurements.
\end{itemize}

The microcontroller on the arm is an ARM® Cortex M4F mounted on a Tiva Launchpad development board.
It is connected to an MPU9150 on which the MEMs accelerometer and rate gyroscope resides.

\subsection{Communications subsystem}
The Tiva Launcpad recieves measurements from the accelerometer, rate gyroscope and ultrasound sensor. The accelerometer and 
rate gyroscope is sampled at 1kHz, while the ultrasound sensor has a varying sample rate that depends on the distance to the nearest
object. The measurement data is transformed into a packet where the leading bit of every byte contains a synchronization bit. When
the bit is set it signifies the start of a packet. These packets are then sent to the desktop machine at 1kHz. When the ultrasound 
sensor is not ready with a new measurement, zeroes are transmitted. The packets are received on the desktop machine, which is 
running a control loop at 200Hz. Here the packets are unpacked to their original form and corrupted packets are thrown away.
\subsection{Localisation}
Localisation is performed by a three-step process. First the 
\subsection{Controller}
