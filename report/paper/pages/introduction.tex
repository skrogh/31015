\section{Introduction}
Autonomous or semi-autonomous airborne robots must be able to locate and control their position in three dimensions instead of
the two required for ground-based robots. Various sensors can be used to provide a position estimate and thereby perform localisation.
There are only a few sensors that work well for lightweight aerial vehicles, e.g micro electromechanical (MEMs) accelerometers, 
gyroscopes and magnetometers, IR distance sensors, ultrasound distance sensors, gps and small cameras. Of these options, only the MEMs
sensors can deliver very fast measurements at a rate larger than 200Hz except in rare cases. It would therefore seems that the MEMs
sensors are the easy choice for any application, but since the accelerometers have noise and acceleration measurements need to be 
integrated twice to yield a position estimate, they are plagued by an unbouded error. Furthermore the available MEMs gyros are rate 
gyros, that is to say they measure the rate of rotation instead of absolute orientation. Again these have noise and their measurements
must be integrated to yield an estimate of orientation, leading to unbounded error. This leaves the MEMs magnetometer as a source of
absolute orientation measurements, but using the magnetometer in this way requires precise knowledge of how the magnetic field is 
shaped whereever the robot goes.
