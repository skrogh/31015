\input{pre}
\begin{document}

\section{Procesevaluering}
\subsection{Projektets forudsætninger}
\subsubsection{Bygge videre på tidligere projekt}
Planen var at vi skulle arbejde videre på et bachelor projekt om højde
stablilisering af en simplificeret drone (en proppel/motor på en arm).
Systemet burde virke stabilt med styring fra en vinkelmåler
(potmeter) og vi skulle arbejde med at fusionere data fra et accelerometer og en
langsom vinkel/højdemåler.

I det tidligere projekt var dette også forsøgt men blev droppet da det var for
vanskeligt. Promblemet belv afskrevet til mekanisk støj, i form af vibrationer.
Vi skulle således løse dette problem.

Det viste sig dog at forsøgsopstillingen var væsentlig mindre stabil en antaget.
Programmet, hvori rgulatore kørte, lukkede til tider ned (segmentation fault) og
den enhed, der var ansvarlig for kommunikation med motordriveren kunne der ikke
altid oprettes forbindelse til.

\subsubsection{Deltagernes forudsætninger}
På projektet er vi to deltager, \emph{Carsen Nielsen} og undertegnede.
Da vi begge ved projektets start skulle til at begyde kurserne \emph{31300
Reguleringstek} og \emph{31605 Signaler og lineære systemer i kontinuert tid}
og projektet i stor grad også var en reguleringsopgave, forsøgte vi med overlæg at
skuppe de regulerigskritiske ellementer langt nok ud i tidsplanen, til at vi
have gennemgået tilstrækkeligt af disse kurser.

Desuden viste det sig hurtigt at det var en fordel at vi ikke havde helt de
samme kompetancer. Eg. har \emph{Carsen Nielsen} mere erfaring med
højniveausprogrammering og \emph{Linux}, mens undertegnede har mere erfaring med
lavniveausprogrammering af µprocessore og lodning.

\subsection{Arbejdsprocessen}
Da vi startede projektet regnede vi, som sagt med at systemt var stabilt, men
dette viste sig ret hurtigt ikke at være tilfældet.
Det blev også klart at støj/vibration ikke var problemet, men fejl i
kommunikation med IMU.
\subsubsection{Arbejds-/tidsplan}
Vi planlage derfor efter at 13-uger primært skulle gå med at rydde/repparere op
på den opstilling vi havde overtaget - hvilket nemt lod sig parallelisere: En der
lodder og fiksker mekanik, samt en der refakturere og fejlfinder kode. Samt lave
undersøgelser med støj/vibrations målinger, således at 3-ugers kunne bruges på
alt det ``spændende'' (dessign af state-space opserver og regulator).
I dokumenteringsfasen kunne arbjdet også let paralleliseres, da een skrev og een
lavede grafer.
\subsubsection{Korrektioner}
Undervejs (lige før Påske) stoppede den enhed der stod for
PC$\leftrightarrow$motor-driver/IMU kommunikation med at virke helt - da vi ikke
selv havde kodet denne og kunne danne os overblik over koden på denne gav vi os
i kast med at skrive vores egen kommunikations enhed.
Dette bevirkede naturligvis at vi blev sat en smule bagud i tidsplanen, så vi
lavede en ny - så overblikket blev bevaret.

Ved start på 3-ugers fulgte vi således den nye plan og halede i løbet af den
første uge ind på den oprindelige plan.

\subsection{Samarbejdet}
\subsubsection{Internt i gruppen} Vi har før lavet projekter sammen og selvom vi
frsøgte at holde logbog over alle arbejdsdage her det også været en stor hjælp
at mange samtaler er foregået over Skype, så en log findes over mange af de
tanker der er udveklset undervejs.
\subsubsection{Vejleder} Vi skulle nok havde været bedere til aktivt at søge
vores vejleder i starten af projektet, men i 3-ugers var der et godt sammenspil.
Det fungerede godt at have en til at pege i den rigtige retning og give de
rigtige søgeord (eg. mulitrate kalman filter)
\subsubsection{Eget bidrag til produktet}
Primært fokus: Lodde og rydde op i testopstilling. Lave mikroprocessor kode,
både til håndtering af I2C og ultralyds sensor. Grafisk arbejde (opsætning af
poster skabelon i Inkscape), \emph{Matlab} plots, figure.

\end{document}